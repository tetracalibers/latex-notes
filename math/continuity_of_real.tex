\documentclass[../math-imaging]{subfiles}

\begin{document}

\chapter{実数の連続性}

ε-δ論法によって微分積分の理論を再定義しても、その議論は実数の連続性に依存している。

この章では、「実数は連続である」、平たく言えば「数直線には穴がない」という表現を観察する。

\section{区間の限界を表す}

区間の最大値や最小値は、その区間の中で最大もしくは最小となる数を指す。

閉区間の場合は、区間の端点が最大値・最小値となるが、開区間では端点を含まないため、「区間の中で」最大(もしくは最小)といえる数は存在しないことになる。

\vskip\baselineskip

しかし、「最大値(最小値)がない=区間は限りなく続く」というわけではない。

もしそうだとしたら、次の3つの開区間が区別できないことになる。

\begin{center}
  \scalebox{1.2}{
    \begin{tikzpicture}
      \nlAxisX{-4}{4}
      \nlnumnum{-1.5}[a]{o}{1.5}[b]{o}

      % 上中央にタイトルを表示
      \node[above=2em, midway, lightslategray] at (current bounding box.north) {$(a,b) =  \{ x \in \mathbb{R}  \mid a < x < b \}$};
    \end{tikzpicture}
  }
\end{center}

\begin{center}
  \scalebox{1.2}{
    \begin{tikzpicture}
      \nlAxisX{-4}{4}
      \nlinfnum{1.5}[a]{o}

      \node[above=2em, midway, lightslategray] at (current bounding box.north) {$(a,+\infty) =  \{ x \in \mathbb{R}  \mid a < x \}$};
    \end{tikzpicture}
  }
\end{center}

\begin{center}
  \scalebox{1.2}{
    \begin{tikzpicture}
      \nlAxisX{-4}{4}
      \nlnuminf{-1.5}[a]{o}

      \node[above=2em, midway, lightslategray] at (current bounding box.north) {$(-\infty, a) =  \{ x \in \mathbb{R}  \mid a > x \}$};
    \end{tikzpicture}
  }
\end{center}

そこで、最大値・最小値とは別に、区間に限界があることを表す概念を導入する。

\subsection{上界と下界}

区間内の数がとりうる値に「限\keyword{界}が\keyword{有}る」ことを、\keyword{有界}という概念で表す。

\subsubsection{上界、上に有界}

ある区間に属するどの数も、ある数$M$以下であるとき、この区間は\keyword{上に有界}であるといい、この$M$を\keyword{上界}という。

\begin{definition}{上界}
  $M$が区間$X$の\hl{上界}であるとは、
  \begin{center}
    任意の$x \in X$に対して$x \leq M$が成り立つ
  \end{center}
  ことをいう。このような上界$M$が存在するとき、$X$は\hl{上に有界}であるという。
\end{definition}

\begin{center}
  \scalebox{1.2}{
    \begin{tikzpicture}
      \def\x{-1}
      \def\minR{-4}
      \def\maxR{4}

      \nlAxisX{\minR}{\maxR}
      \nlinfnum{\x}[]{}

      \draw[lightslategray]
      (\minR,0) -- (\x,0) node[lightslategray, midway,above=2em]{\bfseries\small 上に有界な区間$X$};

      \draw [Rhodamine,thick, decorate,decoration={brace,amplitude=5pt,mirror,raise=3ex}]
      (\x,0) -- (\maxR+0.5,0) node[midway,yshift=-3em]{\bfseries\small $X$の上界\scriptsize($M$はここのどこか)};
    \end{tikzpicture}
  }
\end{center}

\subsubsection{下界、下に有界}

ある区間に属するどの数も、ある数$N$以上であるとき、この区間は\keyword{下に有界}であるといい、この$N$を\keyword{下界}という。

\begin{definition}{下界}
  $N$が区間$X$の\hl{下界}であるとは、
  \begin{center}
    任意の$x \in X$に対して$x \geq N$が成り立つ
  \end{center}
  ことをいう。このような下界$N$が存在するとき、$X$は\hl{下に有界}であるという。
\end{definition}

\begin{center}
  \scalebox{1.2}{
    \begin{tikzpicture}
      \def\x{1}
      \def\minR{-4}
      \def\maxR{4}

      \nlAxisX{\minR}{\maxR}
      \nlnuminf{\x}[]{}

      \draw[lightslategray]
      (\x,0) -- (\maxR,0) node[lightslategray, midway,above=2em]{\bfseries\small 下に有界な区間$X$};

      \draw [Cerulean,thick, decorate,decoration={brace,amplitude=5pt,mirror,raise=3ex}]
      (\minR,0) -- (\x,0) node[midway,yshift=-3em]{\bfseries\small $X$の下界\scriptsize($N$はここのどこか)};
    \end{tikzpicture}
  }
\end{center}

\subsubsection{有界}

ある区間が上にも下にも有界であるとき、この区間は\keyword{有界}であるという。

\begin{definition}{有界}
  区間$X$が\hl{有界}であるとは、
  \begin{center}
    $X$が上に有界かつ下に有界である
  \end{center}
  ことをいう。
\end{definition}

\begin{center}
  \scalebox{1.2}{
    \begin{tikzpicture}
      \def\minR{-4}
      \def\maxR{4}
      \def\minX{-1.5}
      \def\maxX{1.5}

      \nlAxisX{\minR}{\maxR}
      \nlnumnum{\minX}[]{}{\maxX}[]{}

      \node[above=2em, midway, lightslategray] at (current bounding box.north) {\bfseries\small 有界な区間$X$};

      \draw [Rhodamine,thick, decorate,decoration={brace,amplitude=5pt,mirror,raise=3ex}]
      (\maxX,0) -- (\maxR+0.5,0) node[midway,yshift=-3em]{\bfseries\small $X$の上界};

      \draw [Cerulean,thick, decorate,decoration={brace,amplitude=5pt,mirror,raise=3ex}]
      (\minR,0) -- (\minX,0) node[midway,yshift=-3em]{\bfseries\small $X$の下界};
    \end{tikzpicture}
  }
\end{center}

\subsection{上限と下限}

\subsection{上限定理}

\todo{公理3.1}

\section{数列の極限再訪}

\subsection{アルキメデスの公理}

\todo{命題3.2}

\subsection{収束列の有界性}

\todo{定理2.11}

\subsection{単調数列}

\todo{定義5.1}

\subsection{有界な単調数列の収束性}

\todo{定理5.4}

\section{区間縮小法}

\todo{定理5.11}

\section{収束する部分列}

\subsection{部分列}

\todo{定義6.5}

\subsection{収束する数列の部分列の極限}

\todo{定理6.7}

\subsection{ボルツァーノ・ワイエルシュトラスの定理}

\todo{定理6.8}

\section{コーシー列と実数の完備性}

\subsection{コーシー列}

\todo{定義6.9}

\subsection{実数の完備性}

\todo{定理6.11}

\section{上限定理再訪}

\todo{定理6.12}

\end{document}
