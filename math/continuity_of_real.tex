\documentclass[../math-imaging]{subfiles}

\begin{document}

\chapter{実数の連続性}

ε-δ論法によって微分積分の理論を再定義しても、その議論は実数の連続性に依存している。

さらに厳密な議論を追究したいのなら、「実数は連続である」、平たく言えば数直線は穴のない線である、ということを数学の言葉で表現する必要がある。

\section{上限と下限}

\subsection{上限定理}

\todo{公理3.1}

\section{数列の極限再訪}

\subsection{アルキメデスの公理}

\todo{命題3.2}

\subsection{収束列の有界性}

\todo{定理2.11}

\subsection{単調数列}

\todo{定義5.1}

\subsection{有界な単調数列の収束性}

\todo{定理5.4}

\section{区間縮小法}

\todo{定理5.11}

\section{収束する部分列}

\subsection{部分列}

\todo{定義6.5}

\subsection{収束する数列の部分列の極限}

\todo{定理6.7}

\subsection{ボルツァーノ・ワイエルシュトラスの定理}

\todo{定理6.8}

\section{コーシー列と実数の完備性}

\subsection{コーシー列}

\todo{定義6.9}

\subsection{実数の完備性}

\todo{定理6.11}

\section{上限定理再訪}

\todo{定理6.12}

\end{document}
