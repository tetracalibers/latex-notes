\documentclass[../math-imaging]{subfiles}

\begin{document}

\chapter{ε-δ論法と極限}

ここまでのこの本では、極限というものを厳密に定義していなかった。
また、微分と積分において、イメージで導出できることを最重視し、厳密な議論を避けた箇所が多くある。

\vskip\baselineskip

厳密には、極限はε-δ論法によって定義され、微分積分の基礎理論は極限の議論に基づいている。
ε-δ論法に踏み込んでいない私たちは、極限というものを語る言葉をまだ持ち合わせていない。

\section{実数の集合}

厳密な理論を展開する上で、知っておくべき言葉の定義を行う。

\subsection{区間}

2つの実数の間の範囲は、区間と呼ばれる。

\begin{definition}{区間}
  \newline
  実数全体の集合$\mathbb{R} $の部分集合のうち、$a<b$である実数$a$と$b$の間にあるすべての実数の集合を\hl{区間}という。
\end{definition}

区間は、端点を含むかどうかによって、開区間、閉区間、半開区間に分類される。

\subsubsection{開区間}

端点を含まない区間を開区間という。

\begin{definition}{開区間}
  $a \leq x \leq b$となる実数$x$の集合を\hl{開区間}といい、$(a,b)$と表す。
\end{definition}

\begin{center}
  \scalebox{1.2}{
    \begin{tikzpicture}
      \nlAxisX{-4}{4}
      \nlnumnum[rosepink]{-1.5}[a]{o}{1.5}[b]{o}

      % 上中央にタイトルを表示
      \node[above=2em, midway, deeppink!70!gray] at (current bounding box.north) {$(a,b) =  \{ x \in \mathbb{R}  \mid a < x < b \}$};
    \end{tikzpicture}
  }
\end{center}

\begin{center}
  \scalebox{1.2}{
    \begin{tikzpicture}
      \nlAxisX{-4}{4}
      \nlinfnum[rosepink]{1.5}[a]{o}

      \node[above=2em, midway, deeppink!70!gray] at (current bounding box.north) {$(a,+\infty) =  \{ x \in \mathbb{R}  \mid a < x \}$};
    \end{tikzpicture}
  }
\end{center}

\begin{center}
  \scalebox{1.2}{
    \begin{tikzpicture}
      \nlAxisX{-4}{4}
      \nlnuminf[rosepink]{-1.5}[a]{o}

      \node[above=2em, midway, deeppink!70!gray] at (current bounding box.north) {$(-\infty, a) =  \{ x \in \mathbb{R}  \mid a > x \}$};
    \end{tikzpicture}
  }
\end{center}

\subsubsection{閉区間}

端点を含まない区間を閉区間という。

\begin{definition}{閉区間}
  $a < x < b$となる実数$x$の集合を\hl{閉区間}といい、$[a,b]$と表す。
\end{definition}

\begin{center}
  \scalebox{1.2}{
    \begin{tikzpicture}
      \nlAxisX{-4}{4}
      \nlnumnum[capri]{-1.5}[a]{c}{1.5}[b]{c}

      % 上中央にタイトルを表示
      \node[above=2em, midway, capri!70!gray] at (current bounding box.north) {$[a,b] =  \{ x \in \mathbb{R}  \mid a \leq x \leq b \}$};
    \end{tikzpicture}
  }
\end{center}

\begin{center}
  \scalebox{1.2}{
    \begin{tikzpicture}
      \nlAxisX{-4}{4}
      \nlinfnum[capri]{1.5}[a]{c}

      \node[above=2em, midway, capri!70!gray] at (current bounding box.north) {$[a,+\infty] =  \{ x \in \mathbb{R}  \mid a \leq x \}$};
    \end{tikzpicture}
  }
\end{center}

\begin{center}
  \scalebox{1.2}{
    \begin{tikzpicture}
      \nlAxisX{-4}{4}
      \nlnuminf[capri]{-1.5}[a]{c}

      \node[above=2em, midway, capri!70!gray] at (current bounding box.north) {$[-\infty, a] =  \{ x \in \mathbb{R}  \mid a \geq  x \}$};
    \end{tikzpicture}
  }
\end{center}

\subsubsection{半開区間}

一方の端点を含み、他方の端点を含まない区間を半開区間という。

\begin{definition}{半開区間}
  次のような集合を\hl{半開区間}という。
  \begin{itemize}
    \item $a \leq x < b$となる実数$x$の集合を、$[a,b)$と表す。
    \item $a < x \leq b$となる実数$x$の集合を、$(a,b]$と表す。
  \end{itemize}
\end{definition}

\begin{center}
  \scalebox{1.2}{
    \begin{tikzpicture}
      \nlAxisX{-4}{4}
      \nlnumnum[princetonorange]{-1.5}[a]{o}{1.5}[b]{c}

      % 上中央にタイトルを表示
      \node[above=2em, midway, princetonorange!80!gray] at (current bounding box.north) {$[a,b) =  \{ x \in \mathbb{R}  \mid a \leq x < b \}$};
    \end{tikzpicture}
  }
\end{center}

\begin{center}
  \scalebox{1.2}{
    \begin{tikzpicture}
      \nlAxisX{-4}{4}
      \nlnumnum[princetonorange]{-1.5}[a]{c}{1.5}[b]{o}

      % 上中央にタイトルを表示
      \node[above=2em, midway, princetonorange!80!gray] at (current bounding box.north) {$(a,b] =  \{ x \in \mathbb{R}  \mid a < x \leq b \}$};
    \end{tikzpicture}
  }
\end{center}

\section{数列の極限}

微分を定義するには関数の極限を考えるが、関数の極限の諸性質は、数列の極限から導かれる。

まずは、$\varepsilon - \delta $論法(数列の場合は$\varepsilon - N$論法とも呼ばれる)によって数列の極限を定義し、その性質をひとつひとつ確かめていこう。

\subsection{εで「一致」をどう表現するか}

「限りなく近づく」という表現では、「限りなく」の部分に無限という概念が含まれてしまう。

有限の値$\varepsilon$を使って、無限を表現しようとするのがε-δ論法である。

\froufrou

ε-δ論法で極限を定義する前に、有限値$\varepsilon$を使った議論の例を見てみよう。

\begin{theorem}{有限値$\varepsilon $の不等式による一致の表現}
  \newline
  $a$、$b$を実数とするとき、任意の$\varepsilon >0$に対して、次のことがいえる。
  \LARGE
  \begin{equation}
    |a-b|<\varepsilon \quad \Longrightarrow  \quad a=b
  \end{equation}
\end{theorem}

実数は連続である(数直線には穴がない)ため、$a$と$b$が異なる実数であれば、$a$と$b$の間には無数の実数が存在する。

つまり、$a$と$b$が異なる限り、その間の距離$|a-b|$は絶対に$0$にはならない。

\begin{center}
  \begin{tikzpicture}[
      _txtnode/.style={text height=1em},
      _term/.style={RoyalBlue},
      _midterm/.style={lavendermagenta},
      _distance/.style={RoyalBlue, thick}
    ]
    \def\xmin{-3}
    \def\xmax{5}
    \def\a{0}
    \def\b{3}
    \def\pointr{2.5pt}
    \def\h{3}
    \def\dh{-2}
    \def\center#1#2{#1 + (#2-#1)/2}

    \draw[->] (\xmin,\h) -- (\xmax,\h) node[right] {$\mathbb{R} $};
    \fill[_term] (\a,\h) circle (\pointr) node[below, _txtnode] {$a$};
    \fill[_term] (\b,\h) circle (\pointr) node[below, _txtnode] {$b$};
    \fill[_midterm] ({\center{\a}{\b}}, \h) circle (\pointr) node[below, _txtnode] {$x$};
    \draw[<->, yshift=1em, _distance] (\a,\h) -- (\b,\h) node[midway, above, _txtnode] {$|a-b|$};

    \tikzmath{
      real \b;
      \b = \center{\a}{\b};
    }

    \draw[->] (\xmin,\h+\dh) -- (\xmax,\h+\dh) node[right] {$\mathbb{R} $};
    \fill[_term] (\a,\h+\dh) circle (\pointr) node[below, _txtnode] {$a$};
    \fill[_term] (\b,\h+\dh) circle (\pointr) node[below, _txtnode] {$b$};
    \fill[_midterm] ({\center{\a}{\b}}, \h+\dh) circle (\pointr) node[below, _txtnode] {$x$};
    \draw[<->, yshift=1em, _distance] (\a,\h+\dh) -- (\b,\h+\dh) node[midway, above, _txtnode] {$|a-b|$};

    \tikzmath{
      real \b;
      \b = \center{\a}{\b};
    }

    \draw[->] (\xmin,\h+\dh*2) -- (\xmax,\h+\dh*2) node[right] {$\mathbb{R} $};
    \fill[_term] (\a,\h+\dh*2) circle (\pointr) node[below, _txtnode] {$a$};
    \fill[_term] (\b,\h+\dh*2) circle (\pointr) node[below, _txtnode] {$b$};
    \fill[_midterm] ({\center{\a}{\b}}, \h+\dh*2) circle (\pointr) node[below, _txtnode] {$x$};
    \draw[<->, yshift=1em, _distance] (\a,\h+\dh*2) -- (\b,\h+\dh*2) node[midway, above, _txtnode] {$|a-b|$};
  \end{tikzpicture}
\end{center}

\vskip\baselineskip

$|a-b|$が$0$にならないということは、ここでも実数の連続性によって、$|a-b|$より小さい実数が存在してしまう。

たとえば、$a$と$b$の間の中点$x= \dfrac{|a-b|}{2}$は、$|a-b|$よりも小さい。

\begin{supplnote}
  $a$と$b$の間の中点というと$\dfrac{a-b}{2}$だが、正の数$\varepsilon$と比較するため、絶対値をつけて$\dfrac{|a-b|}{2}$としている。
\end{supplnote}

\vskip\baselineskip

$|a-b|$より小さい実数が存在してしまうと、「任意の」$\varepsilon >0$に対して、$|a-b|<\varepsilon$を成り立たせることができない。

$\varepsilon$はなんでもよいのだから、$|a-b|$より小さい実数を$\varepsilon$として選ぶこともできてしまう。

しかし、$|a-b|$より小さい実数を$\varepsilon$としたら、$|a-b|<\varepsilon$は満たされない。

\vskip\baselineskip

$|a-b|$が$0$でないという状況下では、あらゆる実数$\varepsilon$より$|a-b|$を小さくすることは不可能である。

したがって、$|a-b| < \varepsilon$を常に成り立たせるなら、$|a-b|=0$、すなわち$a=b$となる。

\froufrou

ここまでの考察から直観を取り除いて、この定理の数学的な証明をまとめておこう。

\vskip\baselineskip

\begin{proof}{有限値$\varepsilon $の不等式による一致の表現}
  $a\ne b$と仮定する。

  \vskip\baselineskip

  $\varepsilon_0 = \dfrac{|a-b|}{2}$とおくと、絶対値$|a-b|$が正の数であることから、$\varepsilon_0$も正の数となる。

  よって、$|a-b|<\varepsilon_0$が成り立つので、

  \begin{equation}
    \begin{WithArrows}
      |a-b| &< \dfrac{|a-b|}{2} \Arrow{両辺$\times 2$} \\
      2|a-b| &< |a-b| \\
      2|a-b| - |a-b| &< 0 \\
      |a-b| &< 0
    \end{WithArrows}
  \end{equation}

  絶対値が負になることはありえないので、$a \ne b$の仮定のもとでは矛盾が生じる。

  したがって、$a=b$でなければならない。$\qed$
\end{proof}

\end{document}
