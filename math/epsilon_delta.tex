\documentclass[../math-imaging]{subfiles}

\begin{document}

\chapter{ε-δ論法と極限}

ここまでのこの本では、極限というものを厳密に定義していなかった。
また、微分と積分において、イメージで導出できることを最重視し、厳密な議論を避けた箇所が多くある。

\vskip\baselineskip

厳密には、極限はε-δ論法によって定義され、微分積分の基礎理論は極限の議論に基づいている。
ε-δ論法に踏み込んでいない私たちは、極限というものを語る言葉をまだ持ち合わせていない。

\section{実数の集合}

厳密な理論を展開する上で、知っておくべき言葉の定義を行う。

\subsection{区間}

2つの実数の間の範囲は、区間と呼ばれる。

\begin{definition}{区間}
  \newline
  実数全体の集合$\mathbb{R} $の部分集合のうち、$a<b$である実数$a$と$b$の間にあるすべての実数の集合を\hl{区間}という。
\end{definition}

区間は、端点を含むかどうかによって、開区間、閉区間、半開区間に分類される。

\subsubsection{開区間}

端点を含まない区間を開区間という。

\begin{definition}{開区間}
  $a \leq x \leq b$となる実数$x$の集合を\hl{開区間}といい、$(a,b)$と表す。
\end{definition}

\begin{center}
  \scalebox{1.2}{
    \begin{tikzpicture}
      \nlAxisX{-4}{4}
      \nlnumnum[rosepink]{-1.5}[a]{o}{1.5}[b]{o}

      % 上中央にタイトルを表示
      \node[above=2em, midway, deeppink!70!gray] at (current bounding box.north) {$(a,b) =  \{ x \in \mathbb{R}  \mid a < x < b \}$};
    \end{tikzpicture}
  }
\end{center}

\begin{center}
  \scalebox{1.2}{
    \begin{tikzpicture}
      \nlAxisX{-4}{4}
      \nlinfnum[rosepink]{1.5}[a]{o}

      \node[above=2em, midway, deeppink!70!gray] at (current bounding box.north) {$(a,+\infty) =  \{ x \in \mathbb{R}  \mid a < x \}$};
    \end{tikzpicture}
  }
\end{center}

\begin{center}
  \scalebox{1.2}{
    \begin{tikzpicture}
      \nlAxisX{-4}{4}
      \nlnuminf[rosepink]{-1.5}[a]{o}

      \node[above=2em, midway, deeppink!70!gray] at (current bounding box.north) {$(-\infty, a) =  \{ x \in \mathbb{R}  \mid a > x \}$};
    \end{tikzpicture}
  }
\end{center}

\subsubsection{閉区間}

端点を含まない区間を閉区間という。

\begin{definition}{閉区間}
  $a < x < b$となる実数$x$の集合を\hl{閉区間}といい、$[a,b]$と表す。
\end{definition}

\begin{center}
  \scalebox{1.2}{
    \begin{tikzpicture}
      \nlAxisX{-4}{4}
      \nlnumnum[capri]{-1.5}[a]{c}{1.5}[b]{c}

      % 上中央にタイトルを表示
      \node[above=2em, midway, capri!70!gray] at (current bounding box.north) {$[a,b] =  \{ x \in \mathbb{R}  \mid a \leq x \leq b \}$};
    \end{tikzpicture}
  }
\end{center}

\begin{center}
  \scalebox{1.2}{
    \begin{tikzpicture}
      \nlAxisX{-4}{4}
      \nlinfnum[capri]{1.5}[a]{c}

      \node[above=2em, midway, capri!70!gray] at (current bounding box.north) {$[a,+\infty] =  \{ x \in \mathbb{R}  \mid a \leq x \}$};
    \end{tikzpicture}
  }
\end{center}

\begin{center}
  \scalebox{1.2}{
    \begin{tikzpicture}
      \nlAxisX{-4}{4}
      \nlnuminf[capri]{-1.5}[a]{c}

      \node[above=2em, midway, capri!70!gray] at (current bounding box.north) {$[-\infty, a] =  \{ x \in \mathbb{R}  \mid a \geq  x \}$};
    \end{tikzpicture}
  }
\end{center}

\subsubsection{半開区間}

一方の端点を含み、他方の端点を含まない区間を半開区間という。

\begin{definition}{半開区間}
  次のような集合を\hl{半開区間}という。
  \begin{itemize}
    \item $a \leq x < b$となる実数$x$の集合を、$[a,b)$と表す。
    \item $a < x \leq b$となる実数$x$の集合を、$(a,b]$と表す。
  \end{itemize}
\end{definition}

\begin{center}
  \scalebox{1.2}{
    \begin{tikzpicture}
      \nlAxisX{-4}{4}
      \nlnumnum[princetonorange]{-1.5}[a]{o}{1.5}[b]{c}

      % 上中央にタイトルを表示
      \node[above=2em, midway, princetonorange!80!gray] at (current bounding box.north) {$[a,b) =  \{ x \in \mathbb{R}  \mid a \leq x < b \}$};
    \end{tikzpicture}
  }
\end{center}

\begin{center}
  \scalebox{1.2}{
    \begin{tikzpicture}
      \nlAxisX{-4}{4}
      \nlnumnum[princetonorange]{-1.5}[a]{c}{1.5}[b]{o}

      % 上中央にタイトルを表示
      \node[above=2em, midway, princetonorange!80!gray] at (current bounding box.north) {$(a,b] =  \{ x \in \mathbb{R}  \mid a < x \leq b \}$};
    \end{tikzpicture}
  }
\end{center}

\section{数列の極限}

微分を定義するには関数の極限を考えるが、関数の極限の諸性質は、数列の極限から導かれる。

まずは、$\varepsilon - \delta $論法(数列の場合は$\varepsilon - N$論法とも呼ばれる)によって数列の極限を定義し、その性質をひとつひとつ確かめていこう。

\subsection{εで「一致」をどう表現するか}

「限りなく近づく」という表現では、「限りなく」の部分に無限という概念が含まれてしまう。

有限の値$\varepsilon$を使って、無限を表現しようとするのがε-δ論法である。

\froufrou

ε-δ論法で極限を定義する前に、有限値$\varepsilon$を使った議論の例を見てみよう。

\begin{theorem}{有限値$\varepsilon $の不等式による一致の表現}
  \newline
  $a$、$b$を実数とするとき、任意の$\varepsilon >0$に対して、次のことがいえる。
  \LARGE
  \begin{equation}
    |a-b|<\varepsilon \quad \Longrightarrow  \quad a=b
  \end{equation}
\end{theorem}

実数は連続である(数直線には穴がない)ため、$a$と$b$が異なる実数であれば、$a$と$b$の間には無数の実数が存在する。

つまり、$a$と$b$が異なる限り、その間の距離$|a-b|$は絶対に$0$にはならない。

\begin{center}
  \begin{tikzpicture}[
      _txtnode/.style={text height=1em},
      _term/.style={RoyalBlue},
      _midterm/.style={lavendermagenta},
      _distance/.style={RoyalBlue, thick}
    ]
    \def\xmin{-3}
    \def\xmax{5}
    \def\a{0}
    \def\b{3}
    \def\pointr{2.5pt}
    \def\h{3}
    \def\dh{-2}
    \def\center#1#2{#1 + (#2-#1)/2}

    \draw[->] (\xmin,\h) -- (\xmax,\h) node[right] {$\mathbb{R} $};
    \fill[_term] (\a,\h) circle (\pointr) node[below, _txtnode] {$a$};
    \fill[_term] (\b,\h) circle (\pointr) node[below, _txtnode] {$b$};
    \fill[_midterm] ({\center{\a}{\b}}, \h) circle (\pointr) node[below, _txtnode] {$x$};
    \draw[<->, yshift=1em, _distance] (\a,\h) -- (\b,\h) node[midway, above, _txtnode] {$|a-b|$};

    \tikzmath{
      real \b;
      \b = \center{\a}{\b};
    }

    \draw[->] (\xmin,\h+\dh) -- (\xmax,\h+\dh) node[right] {$\mathbb{R} $};
    \fill[_term] (\a,\h+\dh) circle (\pointr) node[below, _txtnode] {$a$};
    \fill[_term] (\b,\h+\dh) circle (\pointr) node[below, _txtnode] {$b$};
    \fill[_midterm] ({\center{\a}{\b}}, \h+\dh) circle (\pointr) node[below, _txtnode] {$x$};
    \draw[<->, yshift=1em, _distance] (\a,\h+\dh) -- (\b,\h+\dh) node[midway, above, _txtnode] {$|a-b|$};

    \tikzmath{
      real \b;
      \b = \center{\a}{\b};
    }

    \draw[->] (\xmin,\h+\dh*2) -- (\xmax,\h+\dh*2) node[right] {$\mathbb{R} $};
    \fill[_term] (\a,\h+\dh*2) circle (\pointr) node[below, _txtnode] {$a$};
    \fill[_term] (\b,\h+\dh*2) circle (\pointr) node[below, _txtnode] {$b$};
    \fill[_midterm] ({\center{\a}{\b}}, \h+\dh*2) circle (\pointr) node[below, _txtnode] {$x$};
    \draw[<->, yshift=1em, _distance] (\a,\h+\dh*2) -- (\b,\h+\dh*2) node[midway, above, _txtnode] {$|a-b|$};
  \end{tikzpicture}
\end{center}

\vskip\baselineskip

$|a-b|$が$0$にならないということは、ここでも実数の連続性によって、$|a-b|$より小さい実数が存在してしまう。

たとえば、$a$と$b$の間の中点$x= \dfrac{|a-b|}{2}$は、$|a-b|$よりも小さい。

\begin{supplnote}
  $a$と$b$の間の中点というと$\dfrac{a-b}{2}$だが、正の数$\varepsilon$と比較するため、絶対値をつけて$\dfrac{|a-b|}{2}$としている。
\end{supplnote}

\vskip\baselineskip

$|a-b|$より小さい実数が存在してしまうと、「任意の」$\varepsilon >0$に対して、$|a-b|<\varepsilon$を成り立たせることができない。

$\varepsilon$はなんでもよいのだから、$|a-b|$より小さい実数を$\varepsilon$として選ぶこともできてしまう。

しかし、$|a-b|$より小さい実数を$\varepsilon$としたら、$|a-b|<\varepsilon$は満たされない。

\vskip\baselineskip

$|a-b|$が$0$でないという状況下では、あらゆる実数$\varepsilon$より$|a-b|$を小さくすることは不可能である。

したがって、$|a-b| < \varepsilon$を常に成り立たせるなら、$|a-b|=0$、すなわち$a=b$となる。

\froufrou

ここまでの考察から直観を取り除いて、この定理の数学的な証明をまとめておこう。

\vskip\baselineskip

\begin{proof}{有限値$\varepsilon $の不等式による一致の表現}
  $a\ne b$と仮定する。

  \vskip\baselineskip

  $\varepsilon_0 = \dfrac{|a-b|}{2}$とおくと、絶対値$|a-b|$が正の数であることから、$\varepsilon_0$も正の数となる。

  よって、$|a-b|<\varepsilon_0$が成り立つので、

  \begin{equation}
    \begin{WithArrows}
      |a-b| &< \dfrac{|a-b|}{2} \Arrow{両辺$\times 2$} \\
      2|a-b| &< |a-b| \\
      2|a-b| - |a-b| &< 0 \\
      |a-b| &< 0
    \end{WithArrows}
  \end{equation}

  絶対値が負になることはありえないので、$a \ne b$の仮定のもとでは矛盾が生じる。

  したがって、$a=b$でなければならない。$\qed$
\end{proof}

\subsection{ε- N論法による数列の収束}

$\varepsilon - \delta$論法は、数列の極限に適用する場合、$\varepsilon - N$論法と呼ばれることが多い。

「数列が$\{a_n\}$が$\alpha$に収束する」ことの$\varepsilon - N$論法による表現を、まずはイメージで掴んでみよう。

\froufrou

まず、$\alpha$の周りに、両側それぞれ$\varepsilon$だけ広げた区間を考える。

$\varepsilon$は正の数ならなんでもよいとすれば、$\varepsilon$を小さな数に設定し、いくらでも区間を狭めることができる。

そして、「ここから先の項はすべて区間内に収まる」といえる位置に、$N$という印をつけておく。

\begin{center}
  \begin{tikzpicture}
    \def\ep{0.5} % ε
    \def\N{3.75} % N

    \def\xmax{9} % max x axis
    \def\ymax{3.5}
    \def\A{2.75} % 振幅を大きく
    \def\om{(7.5*360/(0.94*\xmax))} % 周期を短く(振動数を上げる)
    \def\t{1800/(0.94*\xmax)}
    \def\T{2.0} % 減衰をやや速める
    \def\samp{100} % number of samples
    %\def\tick#1#2{\draw[thick] (#1)++(#2:0.12) --++ (#2-180:0.24)}

    % AXIS
    \draw[axis] (0,-\ymax) -- (0,\ymax) node[above] {\large$a_n$};
    \draw[axis] (-0.2*\ymax,-\ymax + 1) -- (\xmax,-\ymax +1) node[right] {\large$n$};
    %\tick{0,\A}{0} node[left=-1,scale=0.9] {$A_0$};
    %\tick{0,-\A}{0} node[left=-1,scale=0.9] {$-A_0$};

    % ε区間
    \fill[carnationpink, fill opacity=0.5] ($($(0,\A)!.5!(0,-\A)$) + (0, \ep)$) rectangle ($($(\xmax,\A)!.5!(\xmax,-\A)$) - (0, \ep)$);
    \draw[magenta, thick] ($(0,\A)!.5!(0,-\A)$) --++ (\xmax,0);
    \draw[carnationpink] ($(0,\A)!.5!(0,-\A) - (0,\ep)$) --++ (\xmax,0);
    \draw[carnationpink] ($(0,\A)!.5!(0,-\A) + (0,\ep)$) --++ (\xmax,0);

    % ε
    \draw[<->, hotpink, xshift=0.5em] ($(\xmax,\A)!.5!(\xmax,-\A)$) -- ($(\xmax,\A)!.5!(\xmax,-\A) - (0,\ep)$) node[midway, right] {$\varepsilon$};
    \draw[<->, hotpink, xshift=0.5em] ($(\xmax,\A)!.5!(\xmax,-\A)$) -- ($(\xmax,\A)!.5!(\xmax,-\A) + (0,\ep)$) node[midway, right] {$\varepsilon$};

    % 縦軸上のラベル
    \node[left, magenta] at ($(0,\A)!.5!(0,-\A)$) {$\alpha$};
    \node[left, carnationpink] at ($(0,\A)!.5!(0,-\A) - (0,\ep)$) {$\alpha - \varepsilon$};
    \node[left, carnationpink] at ($(0,\A)!.5!(0,-\A) + (0,\ep)$) {$\alpha + \varepsilon$};

    % PLOT
    %\draw[dashed,samples=\samp,smooth,variable=\t,domain=0:0.96*\xmax]
    %plot(\t,{\A*exp(-\t/\T)}) plot(\t,{-\A*exp(-\t/\T)});
    \draw[dotted, cyan, thick, samples=100+\samp, smooth, variable=\t, domain=0:\xmax]
    plot(\t,{\A*exp(-\t/\T)*cos(\om*\t)});

    %%% 以下、重いのでデバッグ中はコメントアウト推奨 %%%

    \path[name path=ExpAbove,samples=375+\samp, variable=\t, domain=0:\xmax] plot(\t,{\A*exp(-\t/\T)});
    \path[name path=ExpBelow,samples=375+\samp, variable=\t, domain=0:\xmax] plot(\t,{-\A*exp(-\t/\T)});
    \path[name path=Wave,samples=375+\samp, variable=\t, domain=0:\xmax] plot(\t,{\A*exp(-\t/\T)*cos(\om*\t)});

    \fill[ProcessBlue,name intersections={of=ExpAbove and Wave, name=i, total=\t}]
    \foreach \s in {1,...,\t}{(i-\s) circle (2pt)};

    \fill[ProcessBlue,name intersections={of=ExpBelow and Wave, name=i, total=\t}]
    \foreach \s in {1,...,\t}{(i-\s) circle (2pt)};

    %%% 以上、重いのでデバッグ中はコメントアウト推奨 %%%

    % N以降を表す矢印
    \draw[lawngreen, opacity=0.8, -{Triangle[width = 18pt, length = 8pt]}, line width = 9pt] (\N,-2) -- (\xmax,-2);

    % N以降を表す空間
    \fill[lawngreen, opacity=0.2] (\N,-\ymax) rectangle (\xmax,\ymax);

    % N直線
    \draw[malachite, thick] (\N,-\ymax) -- (\N,\ymax);
    \node[malachite] at (\N,-\ymax) [below] {\Large$N$};
  \end{tikzpicture}
\end{center}

$\varepsilon$を小さくしていくと、$\varepsilon$による$\alpha$周辺の区間に入る項は少なくなる。

それでも、$N$をずらしていけば、$N$以降はこの区間に収まる項だけになる。

これこそが「収束」という現象だと定義するのが、$\varepsilon - N$論法の考え方である。

\begin{center}
  \begin{tikzpicture}
    \def\ep{0.15} % ε
    \def\N{6.5} % N

    \def\xmax{9} % max x axis
    \def\ymax{3.5}
    \def\A{2.75} % 振幅を大きく
    \def\om{(7.5*360/(0.94*\xmax))} % 周期を短く(振動数を上げる)
    \def\t{1800/(0.94*\xmax)}
    \def\T{2.0} % 減衰をやや速める
    \def\samp{100} % number of samples
    %\def\tick#1#2{\draw[thick] (#1)++(#2:0.12) --++ (#2-180:0.24)}

    % AXIS
    \draw[axis] (0,-\ymax) -- (0,\ymax) node[above] {\large$a_n$};
    \draw[axis] (-0.2*\ymax,-\ymax + 1) -- (\xmax,-\ymax +1) node[right] {\large$n$};
    %\tick{0,\A}{0} node[left=-1,scale=0.9] {$A_0$};
    %\tick{0,-\A}{0} node[left=-1,scale=0.9] {$-A_0$};

    % ε区間
    \fill[carnationpink, fill opacity=0.5] ($($(0,\A)!.5!(0,-\A)$) + (0, \ep)$) rectangle ($($(\xmax,\A)!.5!(\xmax,-\A)$) - (0, \ep)$);
    \draw[magenta, thick] ($(0,\A)!.5!(0,-\A)$) --++ (\xmax,0);
    \draw[carnationpink] ($(0,\A)!.5!(0,-\A) - (0,\ep)$) --++ (\xmax,0);
    \draw[carnationpink] ($(0,\A)!.5!(0,-\A) + (0,\ep)$) --++ (\xmax,0);

    % ε
    %\draw[<->, magenta, xshift=-0.5em] ($(0,\A)!.5!(0,-\A)$) -- ($(0,\A)!.5!(0,-\A) - (0,\ep)$) node[midway, left] {$\varepsilon$};
    %\draw[<->, magenta, xshift=-0.5em] ($(0,\A)!.5!(0,-\A)$) -- ($(0,\A)!.5!(0,-\A) + (0,\ep)$) node[midway, left] {$\varepsilon$};

    % PLOT
    %\draw[dashed,samples=\samp,smooth,variable=\t,domain=0:0.96*\xmax]
    %plot(\t,{\A*exp(-\t/\T)}) plot(\t,{-\A*exp(-\t/\T)});
    \draw[dotted, cyan, thick, samples=100+\samp, smooth, variable=\t, domain=0:\xmax]
    plot(\t,{\A*exp(-\t/\T)*cos(\om*\t)});

    %%% 以下、重いのでデバッグ中はコメントアウト推奨 %%%

    \path[name path=ExpAbove,samples=375+\samp, variable=\t, domain=0:\xmax] plot(\t,{\A*exp(-\t/\T)});
    \path[name path=ExpBelow,samples=375+\samp, variable=\t, domain=0:\xmax] plot(\t,{-\A*exp(-\t/\T)});
    \path[name path=Wave,samples=375+\samp, variable=\t, domain=0:\xmax] plot(\t,{\A*exp(-\t/\T)*cos(\om*\t)});

    \fill[ProcessBlue,name intersections={of=ExpAbove and Wave, name=i, total=\t}]
    \foreach \s in {1,...,\t}{(i-\s) circle (1.75pt)};

    \fill[ProcessBlue,name intersections={of=ExpBelow and Wave, name=i, total=\t}]
    \foreach \s in {1,...,\t}{(i-\s) circle (1.75pt)};

    %%% 以上、重いのでデバッグ中はコメントアウト推奨 %%%

    % N以降を表す矢印
    \draw[lawngreen, opacity=0.8, -{Triangle[width = 18pt, length = 8pt]}, line width = 9pt] (\N,-2) -- (\xmax,-2);

    % N以降を表す空間
    \fill[lawngreen, opacity=0.2] (\N,-\ymax) rectangle (\xmax,\ymax);

    % N直線
    \draw[malachite, thick] (\N,-\ymax) -- (\N,\ymax);
    \node[malachite] at (\N,-\ymax) [below] {\Large$N$};
  \end{tikzpicture}
\end{center}

区間幅(の半分)となる$\varepsilon$をどんなに小さくしても、「$N$番目以降は区間内に収まる項だけになる」といえるような$N$を設定できるか?が肝心で、そのような$N$が存在するなら、数列は収束するといえる。

このことを、数学の言葉でまとめておこう。

\begin{definition}{数列の収束と極限値}
  \newline
  数列$\{a_n\}_{n=1}^{\infty}$と実数$\alpha$について、次の条件を考える。
  \begin{spacebox}
    任意の正の数$\varepsilon$に対して
    \Large
    \begin{equation}
      n \geq N \quad \Longrightarrow \quad |a_n - \alpha| < \varepsilon
    \end{equation}
    \normalsize
    が成り立つような自然数$N$が存在する
  \end{spacebox}
  この条件が成り立つとき、数列$\{a_n\}$は$\alpha$に\hl{収束}するといい、次のように表す。
  \LARGE
  \begin{equation}
    \lim_{n \to \infty} a_n = \alpha \quad \text{\normalsize または} \quad a_n \to \alpha \quad (n \to \infty)
  \end{equation}
  \normalsize
  このとき、$\alpha$を数列$\{a_n\}$の\hl{極限値}という。
\end{definition}

$\varepsilon-\delta$論法によるこの定義を用いることで、数列の収束に関する諸性質を証明できるようになる。

\subsection{数列の極限の一意性}

数列が最終的に複数の極限値に散らばるとしたら、それは収束と呼べるだろうか?

$\varepsilon-\delta$論法による収束の定義は、そのような状況をきちんと除外するようになっている。

\vskip\baselineskip

数列が複数の値に収束することはない。このことを示すのが、次の定理である。

\begin{theorem}{数列の極限の一意性}
  \newline
  数列$\{a_n\}$が収束するならば、その極限値はただ1つに定まる。
\end{theorem}

\begin{proof}{数列の極限の一意性}
  数列$\{a_n\}$が$\alpha$と$\beta$の2つの極限値を持つと仮定する。

  \vskip\baselineskip

  このとき、任意の正の数$\varepsilon$に対して、
  \begin{align}
    n \geq N_1 \quad & \Longrightarrow \quad |a_n - \alpha| < \varepsilon \\
    n \geq N_2 \quad & \Longrightarrow \quad |a_n - \beta| < \varepsilon
  \end{align}
  が成り立つような自然数$N_1$と$N_2$が存在する。

  \vskip\baselineskip

  ここで、$N = \max\{N_1, N_2\}$とおくと、$n \geq N$のとき、$N_1$と$N_2$の大きい方が$n$以下に収まることから、$n \geq N_1$と$n \geq N_2$がともに成り立つ。

  \vskip\baselineskip

  よって、$n \geq N$のとき、$|\alpha - \beta|$を考えると、
  \begin{equation}
    \begin{WithArrows}
      |\alpha - \beta| & = |\alpha - \beta + \wavelabelmath{a_n - a_n}{$0$}| \\
      & = |(\alpha - a_n) + (a_n - \beta)| \Arrow{三角不等式} \\
      & \leq |\alpha - a_n| + |a_n - \beta| \\
      &= |-(a_n - \alpha)| + |a_n - \beta| \Arrow{$|-A| = |A|$} \\
      &= |a_n - \alpha| + |a_n - \beta| \Arrow{$n_1 \geq N$と$n_2 \geq N$より} \\
      & < \varepsilon + \varepsilon \\
      & = 2\varepsilon \\
      \therefore \quad |\alpha - \beta| & < 2\varepsilon
    \end{WithArrows}
  \end{equation}

  ここで、$\varepsilon$は任意の正の数であるから、$2\varepsilon$も任意の正の数をとりうる。

  よって、「有限値$\varepsilon$の不等式による一致の表現」より、$|\alpha - \beta| < 2\varepsilon$から、
  \begin{equation}
    \alpha = \beta
  \end{equation}
  がいえる。

  これで、数列$\{a_n\}$の極限値はただ1つに定まることが示された。$\qed$
\end{proof}

\end{document}
