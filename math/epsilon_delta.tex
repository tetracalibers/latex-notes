\documentclass[../math-imaging]{subfiles}

\begin{document}

\chapter{ε-δ論法による無限の記述}

ここまでのこの本では、極限というものを厳密に定義していなかった。
また、微分と積分において、イメージで導出できることを最重視し、厳密な議論を避けた箇所が多くある。

\vskip\baselineskip

厳密には、極限はε-δ論法によって定義され、微分積分の基礎理論は極限の議論に基づいている。
ε-δ論法に踏み込んでいない私たちは、極限というものを語る言葉をまだ持ち合わせていない。

\subfile{epsilon_delta/real_interval}
\subfile{epsilon_delta/sequence_limit}
\subfile{epsilon_delta/series}
\subfile{epsilon_delta/function_limit}
\subfile{epsilon_delta/function_continuity}

\end{document}
