\documentclass[../../imaging-math]{subfiles}

\begin{document}

\section{論理}

数学では「曖昧さ」を取り除いた議論が重要である。

ここでは、文章(表現)の曖昧さを取り除くために、文章の意味を記号化して扱おう、というアプローチを考えていく。

\subsection{命題:客観的視点に絞る}

数学では、多くの問いを立てて、それが正しいかどうかをひとつひとつ確かめていくことになる。

このとき、数学の議論の対象にできるのは、主観を含まない、「真偽を問うことができる」文章である。

\br

たとえば、「数学は面白い」という文章は、私たちにとっては事実かもしれないが、主観を含む(もはや個人的感想に近い)主張である。

このような、人によって回答が分かれるような文章について議論するのは、また別の学問領域に任せてしまおう。

\br

数学の議論の対象とできる文章は、「正しい」か「正しくない」かをはっきりと定められるもので、そのような文章は\keyword{命題}と呼ばれる。

\begin{definition}{命題}\quad\\
  客観的に正しいか正しくないかが判断できる主張を\hl{命題}という。
\end{definition}

\subsection{記号化:解釈を一通りに絞る}

数学で扱う命題は、文章ではなく記号で表現されることも多い。
どのような記号をどんな意味で使うかを見ていく前に、そもそもなぜ記号化するのか、ということを考えておきたい。

\br

私たちが日常で使う言葉は、表現のパターンがあまりにも多い。英語に比べると、日本語は特にその傾向がある。

たとえば、次の3つの文章は、まったく同じことを主張しているものである。

\begin{enumerate}
  \item パンケーキは太る
  \item パンケーキを食べると太る
  \item パンケーキを食べたならば、体重が増える
\end{enumerate}

1つめの文章は、ネイティブの日本人の間では問題なく伝わるが、日本語に慣れていない人には伝わらない可能性がある。

実際にこのような日本語を聞いて、「パンケーキ=太る」ってどういうこと…?となる外国人は多いらしい。
「パンケーキ」と「太る」が「=」で結ばれているかのように見える時点で、問題がある。

\br

2つめの文章では、より意味が明確になった。この文章であれば、英語が苦手な私が英訳を試みたとしても、"Pancakes is fat."みたいな奇妙な英文を生み出さずに済むだろう。

\br

3つめの文章は、少し堅苦しく感じるかもしれない。しかし、これが最も数学らしい表現である。
「パンケーキを食べた」と「体重が増える」という2つの事象が「ならば」で結ばれていて、論理の構造がはっきりしている。

\froufrou

このように、何通りもの表現がある文章は、人によって解釈が異なる可能性がある。

客観的な主張であるはずの命題も、その表し方によって解釈の齟齬が生まれてしまうと、結局正しいか正しくないかが人によって異なる事態に陥ってしまうかもしれない。

\br

解釈を一通りにするには、表し方を統一してしまうのが手っ取り早い。
そのために、共通認識として定義された記号を使って文章を表現するのが、数学のアプローチなのである。

\br

ここからは、命題を記号化する具体的なルールを見ていくことにする。

\subsection{真偽値:正しさを値とみなす}

命題は、主張の内容によって「正しい」か「正しくない」かが決まる。

このとき、「正しい」ことを\keyword{真}と呼び、「正しくない」ことを\keyword{偽}と呼ぶ。

\begin{definition}{命題の真偽}
  \begin{itemize}
    \item 命題が正しいとき、その命題は\hl{真である}という。
    \item 命題が正しくないとき、その命題は\hl{偽である}という。
  \end{itemize}
\end{definition}

命題は必ず「真」か「偽」のどちらかに決定されるため、この「真」や「偽」を命題が持つ値として考えて、\keyword{真偽値}と呼ぶことがある。

さらに、コンピュータでの応用を見据えて、真偽値を0と1で表すことがある。

\begin{definition}{真偽値}
  \begin{itemize}
    \item 命題が真であるとき、その命題の真偽値は\hl{1}であるとする。
    \item 命題が偽であるとき、その命題の真偽値は\hl{0}であるとする。
  \end{itemize}
\end{definition}

\subsection{論理式:命題を記号化する}

命題が真偽値という値を持つという視点に立つと、その値を反転させたり、組み合わせたりといった操作を考えることができる。

この考え方は、小学校からお馴染みの数の演算に近いものだ。たとえば、足し算という演算は、1と2という値を組み合わせて、3という新たな値を作り出す操作といえる。

\br

ある命題から「新たな命題を作り出す操作」を表す記号を\keyword{論理演算子}といい、論理演算子のような記号の組み合わせによって命題を表現したものを、\keyword{論理式}という。

\subsection{命題の否定}

さて、のび太くんの今回の算数テストの点数は0点だったとする。

このとき、のび太くんが「今回の算数テストは0点じゃなかった!」と主張しても、それは嘘である。

\br

0点だったという事実を$p$、0点じゃなかったという主張を$q$とおこう。

\begin{itemize}
  \item $p$ : のび太くんの算数テストの点数は0点である
  \item $q$ : のび太くんの算数テストの点数は0点ではない
\end{itemize}

このとき、$q$を$p$の\keyword{否定命題}と呼ぶ。

\br

$p$が真実で、$q$は嘘であるのだから、

\begin{emphabox}
  \begin{spacebox}
    \begin{center}
      否定命題になると真偽が入れ替わる
    \end{center}
  \end{spacebox}
\end{emphabox}

ということになる。

\begin{definition}{命題の否定}\quad\\
  命題$p$に対して、「$p$ではない」という命題を$p$の\hl{否定命題}といい、
  \LARGE
  \begin{equation*}
    \neg p
  \end{equation*}
  \normalsize
  と表す。
\end{definition}

$\neg$は「not」を意味する記号で、命題の否定を表す論理演算子である。

\subsubsection{真理値表}

ここで、初めての論理演算子が登場した。

論理演算子によって新たな命題を作り出すとき、元の命題からの真偽値の変化を一覧化しておくとわかりやすい。
そのように真偽値を一覧化した表を、\keyword{真理値表}という。

\br

\begin{wrapstuff}[r,type=table,width=0.3\linewidth]
  \centering
  \begin{NiceTabular}[hvlines]{W{c}{2em}W{c}{2em}}
    \CodeBefore
    \cellcolor{Rhodamine!40}{2-1,3-2}
    \cellcolor{periwinkle!60}{2-2,3-1}
    \Body
    $p$ & $\neg p$ \\
    1   & 0        \\
    0   & 1        \\
  \end{NiceTabular}
  \caption*{\bfseries 否定の真理値表}
\end{wrapstuff}

この「否定の真理値表」では、次の対応関係が示されている。

\begin{itemize}
  \item $p$の真偽値が1のとき、$\neg p$の真偽値は0である
  \item $p$の真偽値が0のとき、$\neg p$の真偽値は1である
\end{itemize}

\subsection{命題関数}

\end{document}
