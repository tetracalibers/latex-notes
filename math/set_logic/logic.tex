\documentclass[../../imaging-math]{subfiles}

\begin{document}

\section{論理}

数学では「曖昧さ」を取り除いた議論が重要である。

ここでは、文章(表現)の曖昧さを取り除くために、文章の意味を記号化して扱おう、というアプローチを考えていく。

\subsection{命題:客観的視点に絞る}

数学では、多くの問いを立てて、それが正しいかどうかをひとつひとつ確かめていくことになる。

このとき、数学の議論の対象にできるのは、主観を含まない、「真偽を問うことができる」文章である。

\br

たとえば、「数学は面白い」という文章は、私たちにとっては事実かもしれないが、主観を含む(もはや個人的感想に近い)主張である。

このような、人によって回答が分かれるような文章について議論するのは、また別の学問領域に任せてしまおう。

\br

数学の議論の対象とできる文章は、「正しい」か「正しくない」かをはっきりと定められるもので、そのような文章は\keyword{命題}と呼ばれる。

\begin{definition}{命題}\quad\\
  客観的に正しいか正しくないかが判断できる主張を\hl{命題}という。
\end{definition}

\subsection{記号化:解釈を一通りに絞る}

数学で扱う命題は、文章ではなく記号で表現されることも多い。
どのような記号をどんな意味で使うかを見ていく前に、そもそもなぜ記号化するのか、ということを考えておきたい。

\br

私たちが日常で使う言葉は、表現のパターンがあまりにも多い。英語に比べると、日本語は特にその傾向がある。

たとえば、次の3つの文章は、まったく同じことを主張しているものである。

\begin{enumerate}
  \item パンケーキは太る
  \item パンケーキを食べると太る
  \item パンケーキを食べたならば、体重が増える
\end{enumerate}

1つめの文章は、ネイティブの日本人の間では問題なく伝わるが、日本語に慣れていない人には伝わらない可能性がある。

実際にこのような日本語を聞いて、「パンケーキ=太る」ってどういうこと…?となる外国人は多いらしい。
「パンケーキ」と「太る」が「=」で結ばれているかのように見える時点で、問題がある。

\br

2つめの文章では、より意味が明確になった。この文章であれば、英語が苦手な私が英訳を試みたとしても、"Pancakes is fat."みたいな奇妙な英文を生み出さずに済むだろう。

\br

3つめの文章は、少し堅苦しく感じるかもしれない。しかし、これが最も数学らしい表現である。
「パンケーキを食べた」と「体重が増える」という2つの事象が「ならば」で結ばれていて、論理の構造がはっきりしている。

\froufrou

このように、何通りもの表現がある文章は、人によって解釈が異なる可能性がある。

客観的な主張であるはずの命題も、その表し方によって解釈の齟齬が生まれてしまうと、結局正しいか正しくないかが人によって異なる事態に陥ってしまうかもしれない。

\br

解釈を一通りにするには、表し方を統一してしまうのが手っ取り早い。
そのために、共通認識として定義された記号を使って文章を表現するのが、数学のアプローチなのである。

\end{document}
