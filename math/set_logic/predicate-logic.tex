\documentclass[../../imaging-math]{subfiles}

\begin{document}

\section{述語論理}

姉が弟に「冷蔵庫の中のプリン、食べていいよ」と言い、弟は嬉々としてプリンを食べた。

しかし、その30分後、冷蔵庫を見た姉は驚き、呆れたようにこう言った。

「まさか全部食べちゃうなんて…プリン5個もあったのに…!」

\br

どうしてこのようなことになったのだろうか?

弟があまりにも食いしん坊だから、に尽きる気もするが…最初の姉の言葉では、次のどちらを意味していたのかが明確ではなかったのだ。

\begin{itemize}
  \item 冷蔵庫の中の「すべての」プリンを食べていい
  \item 冷蔵庫の中に弟用のプリンが「存在して」、それは食べていい
\end{itemize}

\subsection{命題関数}

\end{document}