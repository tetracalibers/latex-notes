\documentclass[../../imaging-math]{subfiles}

\begin{document}

\ifSubfilesClassLoaded{
  \let\subsubsection\subsection
  \let\subsection\section
  \chapter{数列}
}{
  \section{数列}
}

数列は、「数の並びの規則性に着目しよう」という話題として語られる。

数の並びの規則性から、賢く足し算をする知恵も導かれる。

\subsection{数列を語る言葉}

ある規則によって数を並べた列を\keyword{数列}という。
\Large
\begin{equation*}
  a_1,\, a_2,\, a_3,\, \ldots,\, a_n,\, \ldots
\end{equation*}
\normalsize
無限個の数が並んでいるものは\keyword{無限数列}、有限個の数が並んでいるものは\keyword{有限数列}という。

\vskip\baselineskip

並んでいる数を\keyword{項}といい、最初の数は\keyword{初項}、有限数列の場合は最後の数を\keyword{末項}という。

先頭から数えて$n$番目の数を第$n$項、これは\keyword{一般項}ともよばれる。

\vskip\baselineskip

具体的な数をそのまま1つずつ並べて書いてもよいが、長くなってしまう上、規則性が見えにくい。

一般項だけを$\{\}$で囲むことで、一般項で表される数の集合として
\Large
\begin{equation*}
  \{a_n\}
\end{equation*}
\normalsize
と書くことも多い。

\froufrou

さて、重要な規則性を持つ数列には、名前がつけられている。

まずは簡単なものから見ていこう。

\subsection{等差数列}

次のような規則で定まる数列は\keyword{等差数列}とよばれる。

\begin{emphabox}
  \begin{spacebox}
    \begin{center}
      初項$a_1$に次々と一定の数$d$を足してつくられている数列
    \end{center}
  \end{spacebox}
\end{emphabox}

等差数列では、隣り合った2つの項の差がいつも一定の値$d$になり、この$d$を等差数列の\keyword{公差}という。
\subsection{漸化式}

初項と、項と項の間の関係を表す式があれば、その式によって初項の次の数、またその次の数、…というように、数列を復元することができる。

これはいわばドミノ倒しのようなもので、具体的に触れるのは初項だけでよい。あとは規則に従って数が決まっていく。

\subsection{「離散的な関数」としての視点}

数を「並べる」ということは、位置を表す自然数と、その位置にある数とのマッピング(対応づけ)としても捉えられる。

すなわち、位置を与えたらその位置にある数を返す関数
\Large
\begin{equation*}
  f(n) = a_n
\end{equation*}
\normalsize
という形で数列を捉えることもできる。

このとき、数列という名前の関数$f$は、$n$から$a_n$を定める規則のことになる。

\begin{emphabox}
  自然数$1, \, 2, \, 3, \, \ldots$のそれぞれに対して、数$a_1, \, a_2, \, a_3, \, \ldots$が定まっているとき、この対応づける規則を\keyword{数列}という。
\end{emphabox}

\todo{図:点で数列をプロットしたグラフ}

$f(n)$のグラフを書こうとしても、「線」にはならない。

$n$は自然数($1, \, 2, \, 3, \, \ldots$)という飛び飛びに並んだ数であるため、各自然数の間の数に対しては$f(n)$の値が決まらない。
そのため、$f(n)$はぽつりぽつりと点を並べて表すしかない。

\vskip\baselineskip

このように、飛び飛びに並んでいるものは\keyword{離散的}と言われる。

離散的なデータを扱う場面では、データを数列(離散的な関数)として見て調べる視点も重要になる。

\end{document}
