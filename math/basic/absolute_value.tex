\documentclass[../../math-imaging]{subfiles}

\begin{document}

\section{絶対値}

\subsection{数直線上の原点からの距離}

実数$a$の絶対値は、数直線上の原点$0$から$a$までの距離として定義される。

$3$と$-3$を例に考えると、どちらも絶対値は$3$となる。

\begin{center}
  \begin{tikzpicture}[
      _txtnode/.style={text height=1.5em},
    ]
    \def\xmin{-5}
    \def\xmax{5}
    \def\a{-3}
    \def\b{3}
    \def\pointr{2.5pt}
    \def\h{3}

    \draw[->] (\xmin,\h) -- (\xmax,\h) node[right] {$\mathbb{R} $};

    \fill[RoyalBlue] (\a,\h) circle (\pointr) node[below, _txtnode] {\large$-3$};
    \fill[RoyalBlue] (\b,\h) circle (\pointr) node[below, _txtnode] {\large$3$};
    \fill (0, \h) circle (\pointr) node[below, _txtnode] {$0$};

    \draw[<->, Straight Barb-Straight Barb, magenta, yshift=1.5em, shorten >=0.1em, very thick] (\a,\h) -- (0,\h) node[midway, above, _txtnode] {\large$3$};
    \draw[<->, Straight Barb-Straight Barb, magenta, yshift=1.5em, shorten <=0.1em, very thick] (0,\h) -- (\b,\h) node[midway, above, _txtnode] {\large$3$};
  \end{tikzpicture}
\end{center}

$-3$の絶対値が$3$であるように、負の数の絶対値は元の数から符号を取ったもの(元の数を$-1$倍したもの)となる。

まとめると、

\begin{itemize}
  \item 正の数の絶対値は元の数そのまま($0$の絶対値もそのまま$0$)
  \item 負の数の絶対値は元の数の$-1$倍
\end{itemize}

というように、絶対値は場合分けして定義される。

\begin{definition}{絶対値}
  \newline
  実数$a$について、$a$の\hl{絶対値}を次のように定義する。
  \LARGE
  \begin{align}
    |a| = \begin{cases}
            a  & (a \geq  0) \\
            -a & (a < 0)
          \end{cases}
  \end{align}
\end{definition}

\subsection{絶対値の性質}

\subsubsection{絶対値は$0$以上の数}

負の数の場合は、符号を取って正の数にしたものを絶対値とすることから、絶対値が負の数になることはない。

\begin{theorem}{絶対値は常に非負}
  \newline
  実数$a$の絶対値$|a|$は、常に$0$以上の数となる。
  \LARGE
  \begin{align}
    |a| \geq 0
  \end{align}
  \normalsize
  等号が成立するのは、$a=0$の場合である。
\end{theorem}

\subsubsection{中身の符号によらず絶対値は同じ}

$3$も$-3$も、絶対値はともに$3$だった。つまり、

\begin{equation}
  |3| = |-3| = 3
\end{equation}

このことを一般化したのが、次の性質である。

\begin{theorem}{中身の符号を変えても絶対値は不変}
  \newline
  実数$a$の絶対値について、次が成り立つ。
  \LARGE
  \begin{align}
    |-a| = |a|
  \end{align}
\end{theorem}

\subsubsection{積の絶対値は絶対値の積}

絶対値の計算と、積の計算は、どちらを先に行っても結果が同じになる。

\begin{theorem}{絶対値の積の性質}
  \newline
  実数$a$と$b$について、次の式が成り立つ。
  \LARGE
  \begin{equation}
    |a b| = |a||b|
  \end{equation}
\end{theorem}

$a$と$b$がともに正の数なら、

\begin{itemize}
  \item $a$と$b$は正の数なので、$|a| = a$、$|b| = b$
  \item $ab$も正の数なので、$|ab| = ab$
\end{itemize}

となり、$|ab| = |a||b|$が成り立つことがわかる。

\vskip\baselineskip

では、片方が負の数の場合はどうだろうか。

$a$か$b$のどちらかにマイナスの符号をつけてみると、
\begin{align}
  |-ab| & = |-a||b| \\
  |-ab| & = |a||-b|
\end{align}

のどちらかとなるが、前の節で解説した$|-X| = |X|$の関係から、これらはどちらも$|ab| = |a||b|$に帰着する。

\vskip\baselineskip

$a$と$b$の両方が負の数の場合は、
\begin{equation}
  |ab| = |-a||-b|
\end{equation}

となるが、これも$|-X| = |X|$の関係を使えば、やはり$|ab| = |a||b|$に帰着する。

\subsection{数直線上の2点間の距離}

\wip

\subsection{max関数による表現}

実数$a$の絶対値は、「$a$と$-a$のうち大きい方を選ぶ」という考え方でも表現できる。

たとえば、$3$と$-3$の絶対値はともに$3$だが、これは$3$と$-3$のうち大きい方(正の数の方)を絶対値として採用した、という見方もできる。

\begin{definition}{$\max$関数による絶対値の表現}
  \newline
  実数$a$について、$a$の\hl{絶対値}を次のように定義することもできる。
  \LARGE
  \begin{align}
    |a| = \max\{a, -a\}
  \end{align}
\end{definition}

ここで登場した$\max$は、「複数の数の中から最大のものを選ぶ」という操作を表している。

\subsection{三角不等式}

2つの実数$a$と$b$の「絶対値の和」と「和の絶対値」の間には、次のような大小関係がある。

\begin{theorem}{絶対値に関する三角不等式}
  \newline
  任意の実数$a$と$b$について、次の不等式が成り立つ。
  \LARGE
  \begin{align}
    |a + b| \leq |a| + |b|
  \end{align}
\end{theorem}

\begin{supplnote}
  この形の不等式は、実は今後登場するベクトルの長さ(ノルム)や、複素数の絶対値に対しても成り立つ。
  三角不等式と呼ばれる所以は、ベクトルに関する三角不等式で明らかになる。
\end{supplnote}

絶対値の定義から、この不等式の証明を考えてみよう。

\vskip\baselineskip

$a$の絶対値$|a|$は、$a$から符号を取り払ったものであるから、逆に絶対値$|a|$に$+$か$-$の符号をつけることで、元の数$a$に戻すことができる。

$a$が負の数だったなら、$-|a|$とすれば$a$に戻る。正の数だったなら、$|a|$がそのまま$a$に一致する。

\begin{center}
  \begin{tikzpicture}[
      _txtnode/.style={text height=1.5em},
    ]
    \def\xmin{-5}
    \def\xmax{5}
    \def\a{-3}
    \def\b{3}
    \def\pointr{2.5pt}
    \def\h{3}

    \draw[->] (\xmin,\h) -- (\xmax,\h) node[right] {$\mathbb{R} $};

    \fill[RoyalBlue] (\a,\h) circle (\pointr) node[below, _txtnode] {\large$-|a|$};
    \fill[RoyalBlue] (\b,\h) circle (\pointr) node[below, _txtnode] {\large$|a|$};
    \fill (0, \h) circle (\pointr) node[below, _txtnode] {$0$};

    \draw[Straight Barb-, cyan, yshift=1.5em, shorten >=0.1em, very thick] (\a,\h) -- (0,\h) node[midway, above, _txtnode] {\large$|a|$};
    \draw[-Straight Barb, magenta, yshift=1.5em, shorten <=0.1em, very thick] (0,\h) -- (\b,\h) node[midway, above, _txtnode] {\large$|a|$};
  \end{tikzpicture}
\end{center}

$a$は原点からの距離が$|a|$の場所にあり、$a$は$-|a|$か$|a|$のどちらかに一致する。

どちらに一致するかはわからないので、次のような不等式で表しておく。

\begin{equation}
  -|a| \leq a \leq |a|
\end{equation}

$b$についても、同じように考えることができる。

\begin{equation}
  -|b| \leq b \leq |b|
\end{equation}

これらの不等式を使って、さらに式変形を行うことで、三角不等式を導くことができる。

\begin{proof}{絶対値に関する三角不等式}
  絶対値の定義から、次の不等式が成り立つ。
  \begin{align}
    -|a| \leq a \leq |a| \\
    -|b| \leq b \leq |b|
  \end{align}

  両辺を足し合わせて、次の不等式を得る。
  \begin{align}
    -(|a| + |b|) \leq a + b \leq |a| + |b|
  \end{align}

  $-(|a| + |b|) \leq a + b$の両辺を$-1$倍することで、次の関係も得られる。
  (不等式の両辺を$-1$倍すると、不等号の向きが逆転することに注意)
  \begin{align}
    |a| + |b| \geq -(a + b)
  \end{align}

  ここまでで得られた、$a+b$についての不等式をまとめると、次のようになる。
  \begin{align}
    |a| + |b| & \geq a+b      \\
    |a| + |b| & \geq -(a + b)
  \end{align}

  一方、$a+b$の絶対値は、定義より次のように表せる。
  \begin{align}
    |a + b| = \max\{a + b, -(a + b)\}
  \end{align}

  $a+b$と$-(a+b)$のうち大きい方が$|a+b|$となるが、$a + b$と$-(a+b)$はどちらも$|a| + |b|$以下となることがすでに示されているので、

  \begin{equation}
    |a + b| \leq |a| + |b|
  \end{equation}

  となり、定理は示された。$\qed$
\end{proof}

\end{document}
