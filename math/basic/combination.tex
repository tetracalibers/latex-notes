\documentclass[../../math-imaging]{subfiles}

\begin{document}

\ifSubfilesClassLoaded{
  \let\subsubsection\subsection
  \let\subsection\section
  \chapter{場合の数}
}{
  \section{場合の数}
}

何通りの「場合」が起こり得るかを数え上げたものを\keyword{場合の数}という。

\subsection{和の法則}

\begin{figure}[h]
  \centering
  \begin{minipage}{0.49\columnwidth}
    たとえば、A市からB市まで行ける路線が、
    \begin{itemize}
      \item 電車で4路線
      \item バスで3路線
    \end{itemize}
    あるとする。
  \end{minipage}
  \begin{minipage}{0.49\columnwidth}
    \centering
    \begin{tikzpicture}
      \def\gap{1.5em}
      \def\arrowlen{2}

      % 4本のピンクの矢印
      \foreach \i in {1, 2, 3, 4} {
          \draw [->, semithick, -Straight Barb, Rhodamine] (0, -\gap*\i) -- ++(\arrowlen, 0);

          % 1本目の矢印の上中央に「電車」
          \ifnum\i=1
            \node [above=0.25em, Rhodamine] (first) at (\arrowlen/2, -\gap*\i) {\bfseries 電車};
          \fi
        }
      % 3本の青の矢印
      \foreach \i in {5, 6, 7} {
          \draw [->, semithick, -Straight Barb, Cerulean] (0, -\gap*\i) -- ++(\arrowlen, 0);

          % 最後の矢印の上中央に「バス」
          \ifnum\i=7
            \node [below=0.25em, Cerulean] (last) at (\arrowlen/2, -\gap*\i) {\bfseries バス};
          \fi
        }

      % 7本の矢印を囲む箱を、矢印の左にずらして描く
      \node [draw, lightslategray, rounded corners, fit={(first) (last)}, minimum width=2em, inner sep=0.2em, xshift=-6em] (cityA) {\bfseries A市};
      \node [draw, lightslategray, rounded corners, fit={(first) (last)}, minimum width=2em, inner sep=0.2em, xshift=6em] (cityB) {\bfseries B市};
    \end{tikzpicture}
  \end{minipage}
\end{figure}

このとき、電車かバスの「どちらか」でA市からB市まで行くときには、$4 + 3 = 7$パターンの路線から選ぶことになる。

\begin{theorem}{和の法則}\quad\\
  AとBは同時に起こらないとする。

  Aの起こり方が$m$通り、Bの起こり方が$n$通りあるとき、
  \begin{spacebox}
    \large
    \begin{center}
      AとBのどちらかが起こる場合は$m + n$通り
    \end{center}
  \end{spacebox}
\end{theorem}

\begin{center}
  \scalebox{1.2}{
    \begin{tikzpicture}
      \def\gap{1.5em}

      % 1, 2, ..., m を縦に並べる
      \node (a1) at (0, 0) {$1$};
      \node (a2) at ($(0, -\gap)+(a1)$) {$2$};
      \node (a3) at ($(0, -\gap)+(a2)$) {$\vdots$};
      \node (a4) at ($(0, -\gap)+(a3)$) {$m$};

      % a1からa4を包む箱
      \node [draw, rounded corners, Rhodamine, fill=carnationpink, fill opacity=0.4, fit={(a1) (a4)}, minimum width=2em, inner sep=0.2em] (boxA) {};
      % 箱の左に「A」
      \node [left=0.5em of boxA] {A};
      % 箱の右にブレースを描いて「m通り」
      \draw [decorate, decoration={brace, amplitude=5pt}] ($(boxA.north east)+(0.5em, 0)$) -- ($(boxA.south east)+(0.5em, 0)$) node [midway, right=0.7em] {$m$通り};

      % 1, 2, ..., n を縦に並べる
      \node (b1) at ($(0, -\gap*2)+(boxA.south)$) {$1$};
      \node (b2) at ($(0, -\gap)+(b1)$) {$2$};
      \node (b3) at ($(0, -\gap)+(b2)$) {$\vdots$};
      \node (b4) at ($(0, -\gap)+(b3)$) {$n$};

      % b1からb4を包む箱
      \node [draw, rounded corners, Cerulean, fill=SkyBlue, fill opacity=0.4, fit={(b1) (b4)}, minimum width=2em, inner sep=0.2em] (boxB) {};
      % 箱の左に「B」
      \node [left=0.5em of boxB] {B};
      % 箱の右にブレースを描いて「n通り」
      \draw [decorate, decoration={brace, amplitude=5pt}] ($(boxB.north east)+(0.5em, 0)$) -- ($(boxB.south east)+(0.5em, 0)$) node [midway, right=0.7em] {$n$通り};

      % 「m通り」と「n通り」の右にブレースを描いて「m+n通り」
      \draw [decorate, decoration={brace, amplitude=5pt}] ($(boxA.north east)+(5em, 0)$) -- ($(boxB.south east)+(5em, 0)$) node [midway, right=1em] {$m+n$通り};
    \end{tikzpicture}
  }
\end{center}

\subsection{積の法則}

今度は、A市からB市へ、駅を経由して行く場合を考えてみる。

\vskip\baselineskip

A市から駅までは電車で、駅からB市まではバスで行くとする。

つまり、電車とバスを「両方使って」移動することになる。

\begin{itemize}
  \item A市から駅までの電車は4路線
  \item 駅からB市までのバスは3路線
\end{itemize}

どの路線の電車で行くかを決めたら、今度はどの路線のバスに乗るかを選ぶことになる。

\begin{center}
  \begin{tikzpicture}
    \def\gap{1.5em}
    \def\arrowlen{2}

    % 4本のピンクの矢印
    \foreach \i in {1, 2, 3, 4} {
        \draw [->, semithick, -Straight Barb, Rhodamine] (0, -\gap*\i) -- ++(\arrowlen, 0);

        % 1本目の矢印の上中央に「電車」
        \ifnum\i=1
          \node [above=0.25em, Rhodamine] (first) at (\arrowlen/2, -\gap*\i) {\bfseries 電車};
        \fi

        % 最後の矢印をlastで参照できるように
        \ifnum\i=4
          \node [below=0.25em, Rhodamine] (last) at (\arrowlen/2, -\gap*\i) {};
        \fi
      }

    % 4本の矢印を囲む箱を、矢印の左にずらして描く
    \node [draw, lightslategray, rounded corners, fit={(first) (last)}, minimum width=2em, inner sep=0.2em, xshift=-6em] (cityA) {\bfseries A市};
    \node [draw, lightslategray, rounded corners, fit={(first) (last)}, minimum width=2em, inner sep=0.2em, xshift=6em] (station) {\bfseries 駅};

    % 駅の右側に3本の矢印を追加(中央揃え、上で計算した間隔を空ける)
    \foreach \i in {-1,0,1} {
        % 最初の矢印の上中央に「バス」
        \ifnum\i=-1
          \draw [->, semithick, -Straight Barb, Cerulean] ([xshift=(3em-\arrowlen)/2]station.east) ++(0, -\gap*\i) -- ++(\arrowlen, 0)
          node [above=0.25em, Cerulean, midway] {\bfseries バス};
        \else
          \draw [->, semithick, -Straight Barb, Cerulean] ([xshift=(3em-\arrowlen)/2]station.east) ++(0, -\gap*\i) -- ++(\arrowlen, 0);
        \fi
      }

    % 駅の右側に3本の矢印を囲む箱を、矢印の右にずらして描く
    \node [draw, lightslategray, rounded corners, fit={(first) (last)}, minimum width=2em, inner sep=0.2em, xshift=18em] (cityB) {\bfseries B市};
  \end{tikzpicture}
\end{center}

4通りの中からどの路線の電車を選んでも、次に乗るバスは3通りの中から選ぶ必要があるので、電車の路線1つにつき、次に乗るバスの路線は3パターン考えられる。

\vskip\baselineskip

「電車1路線につきバス3路線」というパターンの数は、かけ算で表すことができそうだ。

電車とバスを乗り継ぐ場合の路線の選び方は、$3 \times 4 = 12$通りになる。

\begin{theorem}{積の法則}\quad\\
  Aの起こり方が$m$通りあり、その各々についてBの起こり方が$n$通り考えられるとき、
  \begin{spacebox}
    \large
    \begin{center}
      AとBがともに起こる場合は$mn$通り
    \end{center}
  \end{spacebox}
\end{theorem}

「AとBがともに起こる」とは、Aが起こった後にBが起こる場合を指す。

\begin{center}
  \begin{tikzpicture}
    \def\gapA{3em}
    \def\gapB{1.5em}

    % 「1, 2, ..., n を縦に並べる」を4回繰り返す
    \foreach \i in {1, 2, 3, 4} {
        \ifnum\i=1
          \node (b_first) at (10em, -\gapB*5*\i) {};
        \fi

        \ifnum\i=3
          \node (b1{\i}) at (10em, -\gapB*5*\i) {};
          \node (b2{\i}) at ($(0, -\gapB)+(b1{\i})$) {};
          \node (b3{\i}) at ($(0, -\gapB)+(b2{\i})$) {};
          \node (b4{\i}) at ($(0, -\gapB)+(b3{\i})$) {};

          % b1からb4を包む箱
          \node [fit={(b1{\i}) (b4{\i})}, minimum width=2em, inner sep=0.2em] (boxB{\i}) {$\vdots$};
        \else

          \node (b1{\i}) at (10em, -\gapB*5*\i) {$1$};
          \node (b2{\i}) at ($(0, -\gapB)+(b1{\i})$) {$2$};
          \node (b3{\i}) at ($(0, -\gapB)+(b2{\i})$) {$\vdots$};
          \node (b4{\i}) at ($(0, -\gapB)+(b3{\i})$) {$n$};

          % b1からb4を包む箱
          \node [draw, rounded corners, Cerulean, fill=SkyBlue, fill opacity=0.4, fit={(b1{\i}) (b4{\i})}, minimum width=2em, inner sep=0.2em] (boxB{\i}) {};
          % ブレースを描いて「n通り」
          \draw [decorate, decoration={brace, amplitude=5pt}] ($(boxB{\i}.north east)+(1em, 0)$) -- ($(boxB{\i}.south east)+(1em, 0)$) node [midway, right=0.7em] {$n$通り};

        \fi
      }

    % 1, 2, ..., m を縦に並べる(boxB\iの左にそれぞれ並ぶように)
    \node[xshift=-10em] (a1) at (boxB{1}.base) {$1$};
    \node[xshift=-10em] (a2) at (boxB{2}.base) {$2$};
    \node[xshift=-10em] (a3) at (boxB{3}.base) {$\vdots$};
    \node[xshift=-10em] (a4) at (boxB{4}.base) {$m$};

    % a1からa4を包む箱
    \node [draw, rounded corners, Rhodamine, fill=carnationpink, fill opacity=0.4, fit={(a1) (a4)}, minimum width=2em, inner sep=0.2em] (boxA) {};
    % ブレースを描いて「m通り」
    \draw [decorate, decoration={brace, amplitude=5pt}] ($(boxA.south west)-(1em, 0)$) -- ($(boxA.north west)-(1em, 0)$) node [midway, left=0.7em] {$m$通り};

    % ブレースを描いて「mn通り」
    \draw [decorate, decoration={brace, amplitude=5pt}] ($(boxB{1}.north east)+(6em, 0)$) -- ($(boxB{4}.south east)+(6em, 0)$) node [midway, right=1em] {$m\times n$通り};

    % 箱の上に「A」「B」
    \node [above=0.5em of boxB{1}, xshift=-10em] {A};
    \node [above=0.5em of boxB{1}] {B};

    % a1からb1{1},b2{1},b3{1},b4{1}への矢印
    \foreach \i in {1, 2, 3, 4} {
        \draw [->, semithick, -Straight Barb, lightgray, shorten >=1em] (a1.east) -- (b\i{1}.west);
      }
    % a2からb1{2},b2{2},b3{2},b4{2}への矢印
    \foreach \i in {1, 2, 3, 4} {
        \draw [->, semithick, -Straight Barb, lightgray, shorten >=1em] (a2.east) -- (b\i{2}.west);
      }
    % a4からb1{4},b2{4},b3{4},b4{4}への矢印
    \foreach \i in {1, 2, 3, 4} {
        \draw [->, semithick, -Straight Barb, lightgray, shorten >=1em] (a4.east) -- (b\i{4}.west);
      }
  \end{tikzpicture}
\end{center}

\subsection{順列}

\subsection{階乗}

\subsection{組合せ}

\subsection{二項展開とパスカルの三角形}

\end{document}
