\documentclass[../../math-imaging]{subfiles}

\begin{document}

\section{場合の数}

何通りの「場合」が起こり得るかを数え上げたものを\keyword{場合の数}という。

\subsection{和の法則}

\begin{figure}[h]
  \centering
  \begin{minipage}{0.49\columnwidth}
    たとえば、A市からB市まで行ける路線が、
    \begin{itemize}
      \item 電車で4路線
      \item バスで3路線
    \end{itemize}
    あるとする。
  \end{minipage}
  \begin{minipage}{0.49\columnwidth}
    \centering
    \begin{tikzpicture}
      \def\gap{1.5em}
      \def\arrowlen{2}
      
      % 4本のピンクの矢印
      \foreach \i in {1, 2, 3, 4} {
        \draw [->, semithick, -Straight Barb, Rhodamine] (0, -\gap*\i) -- ++(\arrowlen, 0);
        
        % 1本目の矢印の上中央に「電車」
        \ifnum\i=1
          \node [above=0.25em, Rhodamine] (first) at (\arrowlen/2, -\gap*\i) {\bfseries 電車};
        \fi
      }
      % 3本の青の矢印
      \foreach \i in {5, 6, 7} {
        \draw [->, semithick, -Straight Barb, Cerulean] (0, -\gap*\i) -- ++(\arrowlen, 0);
        
        % 最後の矢印の上中央に「バス」
        \ifnum\i=7
          \node [below=0.25em, Cerulean] (last) at (\arrowlen/2, -\gap*\i) {\bfseries バス};
        \fi
      }
      
      % 7本の矢印を囲む箱を、矢印の左にずらして描く
      \node [draw, lightslategray, rounded corners, fit={(first) (last)}, minimum width=2em, inner sep=0.2em, xshift=-6em] (cityA) {A市};
      \node [draw, lightslategray, rounded corners, fit={(first) (last)}, minimum width=2em, inner sep=0.2em, xshift=6em] (cityB) {B市};
    \end{tikzpicture}
  \end{minipage}
\end{figure}

このとき、電車かバスの「どちらか」でA市からB市まで行くときには、$4 + 3 = 7$パターンの路線から選ぶことになる。

\begin{theorem}{和の法則}\quad\\
  AとBは同時に起こらないとする。
  
  Aの起こり方が$m$通り、Bの起こり方が$n$通りあるとき、
  \begin{spacebox}
    \large
    \begin{center}
      AとBのどちらかが起こる場合は$m + n$通り
    \end{center}
  \end{spacebox}
\end{theorem}

\begin{center}
  \scalebox{1.2}{
    \begin{tikzpicture}
      \def\gap{1.5em}
      
      % 1, 2, ..., m を縦に並べる
      \node (a1) at (0, 0) {$1$};
      \node (a2) at ($(0, -\gap)+(a1)$) {$2$};
      \node (a3) at ($(0, -\gap)+(a2)$) {$\vdots$};
      \node (a4) at ($(0, -\gap)+(a3)$) {$m$};
      
      % a1からa4を包む箱
      \node [draw, rounded corners, Rhodamine, fill=carnationpink, fill opacity=0.4, fit={(a1) (a4)}, minimum width=2em, inner sep=0.2em] (boxA) {};
      % 箱の左に「A」
      \node [left=0.5em of boxA] {A};
      % 箱の右にブレースを描いて「m通り」
      \draw [decorate, decoration={brace, amplitude=5pt}] ($(boxA.north east)+(0.5em, 0)$) -- ($(boxA.south east)+(0.5em, 0)$) node [midway, right=0.7em] {$m$通り};
      
      % 1, 2, ..., n を縦に並べる
      \node (b1) at ($(0, -\gap*2)+(boxA.south)$) {$1$};
      \node (b2) at ($(0, -\gap)+(b1)$) {$2$};
      \node (b3) at ($(0, -\gap)+(b2)$) {$\vdots$};
      \node (b4) at ($(0, -\gap)+(b3)$) {$n$};
      
      % b1からb4を包む箱
      \node [draw, rounded corners, Cerulean, fill=SkyBlue, fill opacity=0.4, fit={(b1) (b4)}, minimum width=2em, inner sep=0.2em] (boxB) {};
      % 箱の左に「B」
      \node [left=0.5em of boxB] {B};
      % 箱の右にブレースを描いて「n通り」
      \draw [decorate, decoration={brace, amplitude=5pt}] ($(boxB.north east)+(0.5em, 0)$) -- ($(boxB.south east)+(0.5em, 0)$) node [midway, right=0.7em] {$n$通り};
      
      % 「m通り」と「n通り」の右にブレースを描いて「m+n通り」
      \draw [decorate, decoration={brace, amplitude=5pt}] ($(boxA.north east)+(5em, 0)$) -- ($(boxB.south east)+(5em, 0)$) node [midway, right=1em] {$m+n$通り};
    \end{tikzpicture}
  }
\end{center}

\subsection{積の法則}

\subsection{順列}

\subsection{階乗}

\subsection{組合せ}

\subsection{二項展開とパスカルの三角形}

\end{document}
