\documentclass[../../math-imaging]{subfiles}

\begin{document}

\section{三角関数}

\subsection{円周率}

すべての円は、お互いを拡大もしくは縮小した関係にある。

\begin{center}
  \begin{tikzpicture}
    \def\d1{2cm}
    \def\k{2}
    \def\dst{5}

    % 円周
    \node      [draw,circle, inner sep=0pt, minimum size=\d1, draw=magenta!80, very thick, below, pattern=crosshatch dots, pattern color=pink] (C1) {};
    \node also [label={[above, text=magenta!80]:$l_1$}] (C1);
    % 直径
    \draw [auxline, cyan, very thick] (C1.west) -- (C1.east) node [midway, above, pos=0.75] {$d_1$};
    % 中心
    \fill (C1) circle [radius=0.05] node [below] {$C_1$};

    % 拡大した円
    \node      [draw,circle, inner sep=0pt, minimum size=\d1*\k, draw=magenta!80, very thick, below, pattern=crosshatch dots, pattern color=pink] (C2) at ($(C1)+(\dst, \d1)$) {};
    \node also [label={[above, text=magenta!80]:$l_2$}] (C2);
    % 直径
    \draw [auxline, cyan, very thick] (C2.west) -- (C2.east) node [midway, above, pos=0.75] {$d_2$};
    % 中心
    \fill (C2) circle [radius=0.05] node [below] {$C_2$};

    % k倍を表す矢印
    \draw [->] ($(C1.east)+(1em, 0)$) -- node [midway, above] {$k$倍} ($(C2.west)-(1em,0)$);
  \end{tikzpicture}
\end{center}

円$C_2$が、円$C_1$を$k$倍に拡大したものだとすると、その直径や円周も$C_1$の$k$倍となる。

\begin{align}
  d_{2} & = k\cdot d_{1} \\
  l_{2} & = k\cdot l_{1}
\end{align}

この2つの式を各辺どうし割ることで、$k$が約分されて消え、直径と円周の比が等しくなることがわかる。

\begin{equation}
  \dfrac{d_{2}}{l_{2}} = \dfrac{d_{1}}{l_{1}}
\end{equation}

\begin{theorem}{円の直径と円周の比}
  すべての円において、直径と円周の長さの比は一定である。
\end{theorem}

そして、この一定の比率は、円周率$\pi$として知られている。

\begin{definition}{円周率}
  円の円周の長さ$l$と直径の長さ$d$の比を、円周率といい、$\pi$で表す。
  \LARGE
  \begin{equation}
    \pi = \dfrac{l}{d} = 3.14\ldots
  \end{equation}
\end{definition}

$\pi$の定義式を変形すると、円周の長さを求める式が得られる。

半径を$r$とすると、直径$d = 2r$であるから、

\begin{equation}
  l = \pi \cdot d = 2\pi r
\end{equation}

\begin{theorem}{円周の長さ}
  円の円周の長さ$l$は、半径$r$を使って次のように表される。
  \LARGE
  \begin{equation}
    l = 2\pi r
  \end{equation}
\end{theorem}

\subsection{直角三角形の相似}

ある図形のすべての辺を$r$倍したとき、元の図形と$r$倍後の図形は\keyword{相似}であるという。

\begin{center}
  \begin{tikzpicture}[
    myangles/.style = {draw, angle radius = #1 mm, lightslategray},
    myangles/.default = 4,
    txtnode/.style = {font=\large},
  ]
    \begin{scope}[local bounding box=trg1]
      \coordinate (a) at (0,0);
      \coordinate (b) at (3,0);
      \coordinate (c) at (1,2);
      
      \draw (a) -- (b) -- (c) -- cycle;
    
      % 各辺に異なる色をつける
      \draw [very thick, Rhodamine] (a) -- (b) node[txtnode, midway, below] {$a$};
      \draw [very thick, Cerulean] (b) -- (c) node[txtnode, midway, above right=-0.2em] {$b$};
      \draw [very thick, Dandelion] (c) -- (a) node[txtnode, midway, left] {$c$};
      
      \pic [myangles]     {angle = b--a--c}; 
      
      \pic [myangles]     {angle = a--c--b};
      \pic [myangles=5]   {angle = a--c--b};
      
      \pic [myangles]     {angle = c--b--a};
      \pic [myangles=5]   {angle = c--b--a};
      \pic [myangles=6]   {angle = c--b--a};
    \end{scope}
    
    \begin{scope}[xshift=44mm, scale=1.5, local bounding box=trg2]
      \coordinate (a) at (0,0);
      \coordinate (b) at (3,0);
      \coordinate (c) at (1,2);
    
      \draw (a) -- (b) -- (c) -- cycle;
      
      \draw [very thick, Rhodamine] (a) -- (b) node[txtnode, midway, below] {$ra$};
      \draw [very thick, Cerulean] (b) -- (c) node[txtnode, midway, above right=-0.2em] {$rb$};
      \draw [very thick, Dandelion] (c) -- (a) node[txtnode, midway, left] {$rc$};
            
      \pic [myangles]     {angle = b--a--c};
    
      \pic [myangles]     {angle = a--c--b};
      \pic [myangles=5]   {angle = a--c--b};
    
      \pic [myangles]     {angle = c--b--a};
      \pic [myangles=5]   {angle = c--b--a};
      \pic [myangles=6]   {angle = c--b--a};
    \end{scope}
    
    % 相似を表す記号
    \node[gray] at ($(trg1.east)!0.5!(trg2.west)$) {\Huge$\sim$};
  \end{tikzpicture}
\end{center}

元の図形の辺の比を$a:b:c$とすると、$r$倍後の図形の辺の比は$ra:rb:rc = a:b:c$となる。

このように、相似な図形同士の辺の比は等しい。

\subsubsection{2つの角が一致する三角形同士は相似}

三角形の内角の和は$180^\circ$であるから、2つの角の大きさが等しければ、もう1つの角の大きさも等しくなる。

つまり、2つの角の大きさが一致する2つの三角形は、辺の間の角度は変わらず、各辺の長さを一定倍したものなので、相似といえる。

\vskip\baselineskip

「辺の間の角度がすべて同じ」ことと「各辺の長さが一定倍されている」ことがうまく結びつかない人は、次の図を見てみよう。

もしも辺の長さの拡大率が辺によって異なるとしたら、辺の間の角度を変えない限り、頂点として辺同士を結ぶことができない。

\begin{center}
  \begin{tikzpicture}[
    myangles/.style = {draw, angle radius = #1 mm, lightslategray},
    myangles/.default = 4,
    txtnode/.style = {font=\large},
  ]
    \begin{scope}[local bounding box=trg1]
      \coordinate (a) at (0,0);
      \coordinate (b) at (3,0);
      \coordinate (c) at (1,2);
    
      \draw (a) -- (b) -- (c) -- cycle;
    
      % 各辺に異なる色をつける
      \draw [very thick, Rhodamine] (a) -- (b) node[txtnode, midway, below] {$ra$};
      \draw [very thick, Cerulean] (b) -- ($(b)!15em!(c)$) node[txtnode, midway, above right=-0.2em] {$Rb$};
      \draw [very thick, Dandelion] (c) -- (a) node[txtnode, midway, left] {$rc$};
      
      \pic [myangles]     {angle = b--a--c}; 
      
      \pic [myangles]     {angle = a--c--b};
      \pic [myangles=5]   {angle = a--c--b};
      
      \pic [myangles]     {angle = c--b--a};
      \pic [myangles=5]   {angle = c--b--a};
      \pic [myangles=6]   {angle = c--b--a};
    \end{scope}
    
    \begin{scope}[xshift=80mm, local bounding box=trg2]
      \coordinate (a) at (0,0);
      \coordinate (b) at (3,0);
      \coordinate (c) at (1,2);
      
      \coordinate (c') at ($(b)!15em!(c)$);
    
      \draw (a) -- (b) -- (c') -- cycle;
    
      % 各辺に異なる色をつける
      \draw [very thick, Rhodamine] (a) -- (b) node[txtnode, midway, below] {$ra$};
      \draw [very thick, Cerulean] (b) -- (c') node[txtnode, midway, above right=-0.2em] {$Rb$};
      \draw [very thick, Dandelion] (c') -- (a) node[txtnode, midway, left] {$rc$};
      
      \pic [myangles]     {angle = b--a--c'}; 
      
      \pic [myangles]     {angle = a--c'--b};
      \pic [myangles=5]   {angle = a--c'--b};
      
      \pic [myangles]     {angle = c'--b--a};
      \pic [myangles=5]   {angle = c'--b--a};
      \pic [myangles=6]   {angle = c'--b--a};
    \end{scope}
    
    \draw[->, thick, -Straight Barb, gray] (trg1.east) -- node[above] {\small 無理やり三角形に…?} (trg2.west);
  \end{tikzpicture}
\end{center}

\subsubsection{1つの鋭角が一致する直角三角形同士は相似}

ある1つの角が直角である三角形を、\keyword{直角三角形}という。

2つの角が一致する三角形同士が相似であるなら、直角三角形の場合は、1つの鋭角が一致するだけで相似であることがわかる。

\begin{center}
  \begin{tikzpicture}
    \def\s{1.75}
    
    \coordinate (a1) at (0,0);
    \coordinate (b1) at (2,0);
    \coordinate (c1) at (2,1.5);
    
    \coordinate (a2) at (a1);
    \coordinate (c2) at ($(a1)!2!(c1)$);
    \coordinate (b2') at ($(a1)!2!(b1)$);
    \coordinate (b2) at ($(a1)!(c2)!(b2')$);
  
    \coordinate (a3) at (a1);
    \coordinate (c3) at ($(a1)!3!(c1)$);
    \coordinate (b3') at ($(a1)!3!(b1)$);
    \coordinate (b3) at ($(a1)!(c3)!(b3')$);
    
    \draw[Goldenrod, fill=Goldenrod!60, fill opacity=0.8] (a3) -- (b3) -- (c3) -- cycle;
    \draw[SkyBlue, fill=SkyBlue!60, fill opacity=0.8] (a2) -- (b2) -- (c2) -- cycle;
    \draw[carnationpink, fill=carnationpink!60, fill opacity=0.8] (a1) -- (b1) -- (c1) -- cycle;
    
    \pic [draw, Magenta, fill=Magenta, fill opacity=0.6, angle radius=7mm, angle eccentricity=1.5] {angle = b3--a3--c3};
    
    % 直角
    \pic [draw, lightslategray, angle radius=2.5mm] {right angle = a3--b3--c3};
    \pic [draw, lightslategray, angle radius=2.5mm] {right angle = a2--b2--c2};
    \pic [draw, lightslategray, angle radius=2.5mm] {right angle = a1--b1--c1};
    
    \draw[Dandelion]
      (b3) to[bend right=30] node[right,midway]{$nb$} (c3)
      (c3) to[bend right=30] node[near start,above=0.25em]{$nc$} (a3)
      (a3) to[bend right=30] node[near end,below=0.25em]{$na$} (b3);
    
    \draw[Cerulean]
      (b2) to[bend right=20] node[right,midway]{$mb$} (c2)
      (c2) to[bend right=20] node[near start,above=0.25em]{$mc$} (a2)
      (a2) to[bend right=20] node[near end,below=0.25em]{$ma$} (b2);
    
    \draw[Rhodamine]
      (b1) to[bend right=10] node[right,midway]{$b$} (c1)
      (c1) to[bend right=10] node[near start,above]{$c$} (a1)
      (a1) to[bend right=10] node[near end,below=-0.15em]{$a$} (b1);
  \end{tikzpicture}
\end{center}

つまり、鋭角が等しいすべての直角三角形は、お互いを拡大もしくは縮小した関係(互いに相似の関係)にあり、3辺の比も等しくなる。

\vskip\baselineskip

言い換えれば、
\begin{emphabox}
  \begin{spacebox}
    \begin{center}
      直角三角形の3辺の比は、1つの鋭角の大きさで決まる
    \end{center}
  \end{spacebox}
\end{emphabox}
ということになる。

\subsection{三角比}

\subsection{扇形の弧長と角}

\subsubsection{扇形の弧の長さ}

\subsubsection{扇形の弧長と半径による中心角の表現}

\end{document}
