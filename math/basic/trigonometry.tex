\documentclass[../../math-imaging]{subfiles}

\begin{document}

\section{三角関数}

\subsection{円周率}

すべての円は、お互いを拡大もしくは縮小した関係にある。

\begin{center}
  \begin{tikzpicture}
    \def\d1{2cm}
    \def\k{2}
    \def\dst{5}

    % 円周
    \node      [draw,circle, inner sep=0pt, minimum size=\d1, draw=magenta!80, very thick, below, pattern=crosshatch dots, pattern color=pink] (C1) {};
    \node also [label={[above, text=magenta!80]:$l_1$}] (C1);
    % 直径
    \draw [auxline, cyan, very thick] (C1.west) -- (C1.east) node [midway, above, pos=0.75] {$d_1$};
    % 中心
    \fill (C1) circle [radius=0.05] node [below] {$C_1$};

    % 拡大した円
    \node      [draw,circle, inner sep=0pt, minimum size=\d1*\k, draw=magenta!80, very thick, below, pattern=crosshatch dots, pattern color=pink] (C2) at ($(C1)+(\dst, \d1)$) {};
    \node also [label={[above, text=magenta!80]:$l_2$}] (C2);
    % 直径
    \draw [auxline, cyan, very thick] (C2.west) -- (C2.east) node [midway, above, pos=0.75] {$d_2$};
    % 中心
    \fill (C2) circle [radius=0.05] node [below] {$C_2$};

    % k倍を表す矢印
    \draw [->] ($(C1.east)+(1em, 0)$) -- node [midway, above] {$k$倍} ($(C2.west)-(1em,0)$);
  \end{tikzpicture}
\end{center}

円$C_2$が、円$C_1$を$k$倍に拡大したものだとすると、その直径や円周も$C_1$の$k$倍となる。

\begin{align}
  d_{2} & = k\cdot d_{1} \\
  l_{2} & = k\cdot l_{1}
\end{align}

この2つの式を各辺どうし割ることで、$k$が約分されて消え、直径と円周の比が等しくなることがわかる。

\begin{equation}
  \dfrac{d_{2}}{l_{2}} = \dfrac{d_{1}}{l_{1}}
\end{equation}

\begin{theorem}{円の直径と円周の比}
  すべての円において、直径と円周の長さの比は一定である。
\end{theorem}

そして、この一定の比率は、円周率$\pi$として知られている。

\begin{definition}{円周率}
  円の円周の長さ$l$と直径の長さ$d$の比を、円周率といい、$\pi$で表す。
  \LARGE
  \begin{equation}
    \pi = \dfrac{l}{d} = 3.14\ldots
  \end{equation}
\end{definition}

$\pi$の定義式を変形すると、円周の長さを求める式が得られる。

半径を$r$とすると、直径$d = 2r$であるから、

\begin{equation}
  l = \pi \cdot d = 2\pi r
\end{equation}

\begin{theorem}{円周の長さ}
  円の円周の長さ$l$は、半径$r$を使って次のように表される。
  \LARGE
  \begin{equation}
    l = 2\pi r
  \end{equation}
\end{theorem}

\end{document}
