\documentclass[../../math-imaging]{subfiles}

\begin{document}

\section{対数関数}

\subsection{対数:指数部分を関数で表す}

指数関数は、「$a$を$x$乗したら$y$になる」という関係を表現するものだった。

ここで、逆に「$y$は$a$の何乗か?」という関係を表現するものとして、対数関数を定義する。

これは、$y$から$x$を導き出す関数であるから、指数関数$y=a^x$の逆関数といえる。

\begin{definition}{対数}
  \newline
  $a^y = x$を満たす$y$を、$a$を底とする$x$の対数といい、次のように表す。
  \LARGE
  \begin{equation}
    y = \log_a x
  \end{equation}
  \normalsize
  ここで、$x$は真数、$a$は底と呼ばれる。
\end{definition}

\begin{definition}{対数関数は指数関数の逆関数}
  \newline
  対数関数$y=\log_a x$は、指数関数$x = a^y$の逆関数である。
  \LARGE
  \begin{equation}
    \log_a x = y \quad \Longleftrightarrow  \quad a^y = x
  \end{equation}
\end{definition}

対数は、指数関数の指数部分を表す。

$a^y = x$の$y$に、$y=\log_a x$を代入することで、次のような式にまとめることもできる。

\begin{theorem}{指数部分は対数で書き換えられる}
  \LARGE
  \begin{equation}
    a^{\log_a x} = x
  \end{equation}
\end{theorem}

\subsection{対数の性質}

指数法則を対数に翻訳することで、対数の性質を導くことができる。

\subsubsection{真数のかけ算は$\log$の足し算}

$x_1 = a^m, x_2 = a^n$として、指数法則$a^m \times a^n = a^{m+n}$を考える。

\begin{align}
  x_1  x_2 & = a^m \times a^n \\
           & = a^{m+n}
\end{align}

対数は指数部分を表すので、$m+n = \log_a (x_1x_2)$がいえる。

また、$x_1 = a^m$より$m = \log_a x_1$、$x_2 = a^n$より$n = \log_a x_2$と表せるから、

\begin{equation}
  m + n = \log_a x_1 + \log_a x_2 = \log_a (x_1x_2)
\end{equation}

\begin{theorem}{積の対数は対数の和}
  \LARGE
  \begin{equation}
    \log_a (x_1x_2) = \log_a x_1 + \log_a x_2
  \end{equation}
\end{theorem}

\subsubsection{真数の割り算は$\log$の引き算}

$x_1 = a^m, x_2 = a^n$として、指数法則$\dfrac{a^m}{a^n} = a^{m-n}$を考える。

\begin{align}
  \dfrac{x_1}{x_2} & = \dfrac{a^m}{a^n} \\
                   & = a^{m-n}
\end{align}

対数は指数部分を表すので、$m-n = \log_a \left( \dfrac{x_1}{x_2} \right)$がいえる。

また、$x_1 = a^m$より$m = \log_a x_1$、$x_2 = a^n$より$n = \log_a x_2$と表せるから、

\begin{equation}
  m - n = \log_a x_1 - \log_a x_2 = \log_a \left( \dfrac{x_1}{x_2} \right)
\end{equation}

\begin{theorem}{商の対数は対数の差}
  \LARGE
  \begin{equation}
    \log_a \left( \dfrac{x_1}{x_2} \right) = \log_a x_1 - \log_a x_2
  \end{equation}
\end{theorem}

\subsubsection{真数の冪乗は$\log$の指数倍}

$x = a^m$として、指数法則$(a^m)^n = a^{mn}$を考える。

\begin{align}
  x^n & = (a^m)^n \\
      & = a^{mn}
\end{align}

対数は指数部分を表すので、$mn = \log_a x^n$がいえる。

また、$x = a^m$より$m = \log_a x$と表せるから、

\begin{equation}
  mn = n \log_a x \log_a x^n
\end{equation}

\begin{theorem}{冪の対数は対数の指数倍}
  \LARGE
  \begin{equation}
    \log_a x^n = n \log_a x
  \end{equation}
\end{theorem}

\subsection{常用対数と桁数}

\wip

\begin{definition}{常用対数}
  底を$10$にした対数関数を、常用対数と呼ぶ。
  \LARGE
  \begin{equation}
    \log_{10} x
  \end{equation}
\end{definition}

\subsection{指数関数の底の変換:対数を用いた表現}

指数関数の底$a$から$b$に変換するには、「$a$は$b$の何乗か?」がわかっている必要があった。

\begin{review}
  $a=b^c$という関係があるなら、
  \begin{equation}
    a^x = b^{cx}
  \end{equation}
\end{review}

今では、$a= b^c$となるような$c$を、対数で表すことができる。

\begin{equation}
  b^c = a \Longleftrightarrow  c = \log_b a
\end{equation}

\begin{theorem}{指数関数の底の変換公式}
  \LARGE
  \begin{equation}
    a^x = b^{(\log_b a)x}
  \end{equation}
\end{theorem}

\end{document}
