\documentclass[../imaging-math]{subfiles}

\begin{document}

\chapter{集合と写像}

\begin{mindflow}
  \begin{itemize}
    \item 集合
    \item 写像
  \end{itemize}
\end{mindflow}

\subsection{実数の集合:区間}

2つの実数の間の範囲は、\keyword{区間}と呼ばれる。

\begin{definition}{区間}
  \newline
  実数全体の集合$\mathbb{R} $の部分集合のうち、$a<b$である実数$a$と$b$の間にあるすべての実数の集合を\hl{区間}という。
\end{definition}

区間は、端点を含むかどうかによって、開区間、閉区間、半開区間に分類される。

\subsubsection{開区間}

端点を含まない区間を開区間という。

\begin{definition}{開区間}
  $a \leq x \leq b$となる実数$x$の集合を\hl{開区間}といい、$(a,b)$と表す。
\end{definition}

\begin{center}
  \scalebox{1.2}{
    \begin{tikzpicture}
      \nlAxisX{-4}{4}
      \nlnumnum[rosepink]{-1.5}[a]{o}{1.5}[b]{o}

      % 上中央にタイトルを表示
      \node[above=2em, midway, deeppink!70!gray] at (current bounding box.north) {$(a,b) =  \{ x \in \mathbb{R}  \mid a < x < b \}$};
    \end{tikzpicture}
  }
\end{center}

\begin{center}
  \scalebox{1.2}{
    \begin{tikzpicture}
      \nlAxisX{-4}{4}
      \nlinfnum[rosepink]{1.5}[a]{o}

      \node[above=2em, midway, deeppink!70!gray] at (current bounding box.north) {$(a,+\infty) =  \{ x \in \mathbb{R}  \mid a < x \}$};
    \end{tikzpicture}
  }
\end{center}

\begin{center}
  \scalebox{1.2}{
    \begin{tikzpicture}
      \nlAxisX{-4}{4}
      \nlnuminf[rosepink]{-1.5}[a]{o}

      \node[above=2em, midway, deeppink!70!gray] at (current bounding box.north) {$(-\infty, a) =  \{ x \in \mathbb{R}  \mid a > x \}$};
    \end{tikzpicture}
  }
\end{center}

\subsubsection{閉区間}

端点を含まない区間を閉区間という。

\begin{definition}{閉区間}
  $a < x < b$となる実数$x$の集合を\hl{閉区間}といい、$[a,b]$と表す。
\end{definition}

\begin{center}
  \scalebox{1.2}{
    \begin{tikzpicture}
      \nlAxisX{-4}{4}
      \nlnumnum[capri]{-1.5}[a]{c}{1.5}[b]{c}

      % 上中央にタイトルを表示
      \node[above=2em, midway, capri!70!gray] at (current bounding box.north) {$[a,b] =  \{ x \in \mathbb{R}  \mid a \leq x \leq b \}$};
    \end{tikzpicture}
  }
\end{center}

\begin{center}
  \scalebox{1.2}{
    \begin{tikzpicture}
      \nlAxisX{-4}{4}
      \nlinfnum[capri]{1.5}[a]{c}

      \node[above=2em, midway, capri!70!gray] at (current bounding box.north) {$[a,+\infty] =  \{ x \in \mathbb{R}  \mid a \leq x \}$};
    \end{tikzpicture}
  }
\end{center}

\begin{center}
  \scalebox{1.2}{
    \begin{tikzpicture}
      \nlAxisX{-4}{4}
      \nlnuminf[capri]{-1.5}[a]{c}

      \node[above=2em, midway, capri!70!gray] at (current bounding box.north) {$[-\infty, a] =  \{ x \in \mathbb{R}  \mid a \geq  x \}$};
    \end{tikzpicture}
  }
\end{center}

\subsubsection{半開区間}

一方の端点を含み、他方の端点を含まない区間を半開区間という。

\begin{definition}{半開区間}
  次のような集合を\hl{半開区間}という。
  \begin{itemize}
    \item $a \leq x < b$となる実数$x$の集合を、$[a,b)$と表す。
    \item $a < x \leq b$となる実数$x$の集合を、$(a,b]$と表す。
  \end{itemize}
\end{definition}

\begin{center}
  \scalebox{1.2}{
    \begin{tikzpicture}
      \nlAxisX{-4}{4}
      \nlnumnum[princetonorange]{-1.5}[a]{o}{1.5}[b]{c}

      % 上中央にタイトルを表示
      \node[above=2em, midway, princetonorange!80!gray] at (current bounding box.north) {$[a,b) =  \{ x \in \mathbb{R}  \mid a \leq x < b \}$};
    \end{tikzpicture}
  }
\end{center}

\begin{center}
  \scalebox{1.2}{
    \begin{tikzpicture}
      \nlAxisX{-4}{4}
      \nlnumnum[princetonorange]{-1.5}[a]{c}{1.5}[b]{o}

      % 上中央にタイトルを表示
      \node[above=2em, midway, princetonorange!80!gray] at (current bounding box.north) {$(a,b] =  \{ x \in \mathbb{R}  \mid a < x \leq b \}$};
    \end{tikzpicture}
  }
\end{center}

\end{document}
