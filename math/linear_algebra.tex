\documentclass[../math-imaging]{subfiles}

\begin{document}

\chapter{線形代数}

線形代数は、高次元に立ち向かうための強力な道具となる。

\vskip\baselineskip

どれだけ高次元に話を広げたとしても、「関係」を語る言葉の複雑さが増すことはない。

この章では、そんな状況を実現するための理論を追いかけていく。

\section{ベクトルの作り方}

\subsection{移動の表現としてのベクトル}

平面上のある点の位置を表すのに、よく使われるのが\keyword{直交座標}である。

直交座標では、$x$軸と$y$軸を垂直に張り、
\begin{itemize}
  \item 原点$O$からの$x$軸方向の移動量($x$座標)
  \item 原点$O$からの$y$軸方向の移動量($y$座標)
\end{itemize}
という2つの数の組で点の位置を表す。

\begin{figure}[h]
  \centering
  \begin{minipage}[b]{0.49\columnwidth}
    \centering
    \scalebox{1.2}{
      \begin{tikzpicture}
        \def\xmin{-1}
        \def\xmax{4}
        \def\ymin{-1}
        \def\ymax{4}
        \def\vx{2}
        \def\vy{1.5}

        % 0.5刻みのグリッド
        \draw[dotted, lightslategray] (\xmin, \ymin) grid[step=0.5] (\xmax, \ymax);

        \draw[axis] (\xmin, 0) -- (\xmax, 0) node[right] {$x$};
        \draw[axis] (0, \ymin) -- (0, \ymax) node[above] {$y$};

        % 原点
        \node at (0, 0) [below left] {$O$};

        % x軸方向の移動量
        \draw[dashed, very thick, Rhodamine] (0, 0) -- (\vx, 0) node[below, midway] {$a$};
        % y軸方向の移動量
        \draw[dashed, very thick, Cerulean] (\vx,0) -- (\vx, \vy) node[right, midway] {$b$};

        % 点
        \draw (\vx, \vy) node[circle, fill, inner sep=1.5pt] {};
        \node at (\vx, \vy) [above right] {$(a,b)$};
      \end{tikzpicture}
    }
    \caption*{\bfseries「位置の特定」という視点}
  \end{minipage}
  \begin{minipage}[b]{0.49\columnwidth}
    \centering
    \scalebox{1.2}{
      \begin{tikzpicture}
        \def\xmin{-1}
        \def\xmax{4}
        \def\ymin{-1}
        \def\ymax{4}
        \def\vx{2}
        \def\vy{1.5}

        % 平面
        \draw[dotted, lightslategray] (\xmin, \ymin) -- (\xmax, \ymin) -- (\xmax, \ymax) -- (\xmin, \ymax) -- cycle;

        \coordinate (A) at (0,0);
        \coordinate (B) at (\vx, \vy);

        % ベクトル
        \draw[vector, very thick, BurntOrange, shorten >=0.25em, shorten <=0.25em] (A) -- (B);

        % 点A
        \draw (A) node[circle, fill, inner sep=1.5pt] {};
        \node at (A) [below left] {$A$};

        % 点B
        \draw (B) node[circle, fill, inner sep=1.5pt] {};
        \node at (B) [above right] {$B$};
      \end{tikzpicture}
    }
    \caption*{\bfseries「移動」という視点}
  \end{minipage}
\end{figure}

\br

座標とは、「$x$軸方向の移動」と「$y$軸方向の移動」という2回の移動を行った結果である。

右にどれくらい、上にどれくらい、という考え方で平面上の「位置」を特定しているわけだが、単に「移動」を表したいだけなら、点から点へ向かう矢印で一気に表すこともできる。

\br

ある地点から別のある地点への「移動」を表す矢印を\keyword{ベクトル}という。

\br

ベクトルが示す、ある地点からこのように移動すれば、この地点にたどり着く…といった「移動」の情報は、相対的な「位置関係」を表す上で役に立つ。

\subsubsection{座標とベクトルの違い}

座標は「位置」を表すものだが、ベクトルは「移動」を表すものにすぎない。

座標は「原点からの」移動量によって位置を表すが、ベクトルは始点の位置にはこだわらない。

\br

たとえば、次の2つのベクトルは始点の位置は異なるが、同じ向きに同じだけ移動している矢印なので、同じベクトルとみなせる。

\begin{center}
  \scalebox{1.2}{
    \begin{tikzpicture}
      \def\xmin{-1}
      \def\xmax{4}
      \def\ymin{-1}
      \def\ymax{4}
      \def\vx{1.5}
      \def\vy{2}
      \def\sx{2.5}
      \def\sy{1}

      \draw[axis] (\xmin, 0) -- (\xmax, 0) node[right] {$x$};
      \draw[axis] (0, \ymin) -- (0, \ymax) node[above] {$y$};

      % 原点
      \node at (0, 0) [below left] {$O$};

      % ベクトル
      \draw[vector, very thick, magenta] (0, 0) -- (\vx, \vy);

      % 平行移動したベクトル
      \draw[vector, very thick, magenta] (\sx, \sy) -- ($(\sx,\sy)+(\vx,\vy)$);

      % 平行移動を表す破線
      \draw[dashed, lightslategray] (0, 0) -- (\sx, \sy);
      \draw[dashed, lightslategray] (\vx, \vy) -- ($(\vx,\vy)+(\sx,\sy)$);
    \end{tikzpicture}
  }
\end{center}

このような「同じ向きに同じだけ移動している矢印」は、平面内では平行な関係にある。

つまり、平行移動して重なる矢印は、同じベクトルとみなすことができる。

\subsubsection{移動の合成とベクトルの分解}

ベクトルは、各方向への移動の合成として考えることもできる。

純粋に「縦」と「横」に分解した場合は直交座標の考え方によく似ているが、必ずしも直交する方向のベクトルに分解する必要はない。

\begin{figure}[h]
  \centering
  \begin{minipage}{0.49\columnwidth}
    \centering
    \scalebox{1.5}{
      \begin{tikzpicture}
        \def\xmin{-1}
        \def\xmax{4}
        \def\ymin{-1}
        \def\ymax{4}
        \def\vx{1.5}
        \def\vy{2}

        \coordinate (A) at (0,0);
        \coordinate (B) at (\vx, \vy);

        % ベクトル
        \draw[vector, very thick, BurntOrange, shorten >=0.25em, shorten <=0.25em] (A) -- (B);

        % x軸方向のベクトル
        \draw[vector, dashed, very thick, Rhodamine] (0,0) -- (\vx, 0);
        % y軸方向のベクトル
        \draw[vector, dashed, very thick, Cerulean, shorten >=0.25em] (\vx, 0) -- (\vx, \vy);

        % 点A
        \draw (A) node[circle, fill, inner sep=1.5pt] {};
        \node at (A) [below left] {$A$};

        % 点B
        \draw (B) node[circle, fill, inner sep=1.5pt] {};
        \node at (B) [above right] {$B$};
      \end{tikzpicture}
    }
    \caption*{\bfseries 「縦」と「横」に分解}
  \end{minipage}
  \begin{minipage}{0.49\columnwidth}
    \centering
    \scalebox{1.5}{
      \begin{tikzpicture}
        \def\xmin{-1}
        \def\xmax{4}
        \def\ymin{-1}
        \def\ymax{4}
        \def\vx{1.5}
        \def\vy{2}

        \coordinate (A) at (0,0);
        \coordinate (B) at (\vx, \vy);

        % ベクトル
        \draw[vector, very thick, BurntOrange, shorten >=0.25em, shorten <=0.25em] (A) -- (B);

        % x軸方向のベクトル
        \draw[vector, dashed, very thick, Rhodamine] (0,0) -- (\vx, 0.75);
        % y軸方向のベクトル
        \draw[vector, dashed, very thick, Cerulean, shorten >=0.25em] (\vx, 0.75) -- (\vx, \vy);

        % 点A
        \draw (A) node[circle, fill, inner sep=1.5pt] {};
        \node at (A) [below left] {$A$};

        % 点B
        \draw (B) node[circle, fill, inner sep=1.5pt] {};
        \node at (B) [above right] {$B$};
      \end{tikzpicture}
    }
    \caption*{\bfseries 他の分解も考えられる}
  \end{minipage}
\end{figure}

\subsection{多次元への対応:数ベクトル}

多次元空間内の「移動」を表すには、「縦」と「横」などといった2方向だけでなく、もっと多くの方向への移動量を組み合わせて考える必要がある。

また、4次元を超えてしまうと、矢印の描き方すら想像がつかなくなってしまう。
それは、方向となる軸が多すぎて、どの方向に進むかを表すのが難しくなるためだ。

\vskip\baselineskip

そこで、一旦「向き」の情報を取り除くことで、高次元に立ち向かえないかと考える。

移動を表す矢印は「どの方向に進むか」と「どれくらい進むか」という向きと大きさの情報を持っているが、その「どれくらい進むか」だけを取り出して並べよう。

\todo{平面ベクトルの場合の図}

\vskip\baselineskip

こうして単に「数を並べたもの」もベクトルと呼ぶことにし、このように定義したベクトルを\keyword{数ベクトル}という。

\vskip\baselineskip

数を並べるとき、縦と横の2通りがある。それぞれ\keyword{列ベクトル}、\keyword{行ベクトル}として定義する。

\begin{definition}{列ベクトル}
  数を縦に並べたものを\hl{列ベクトル}という。
  \Large
  \begin{equation*}
    \vb*{a} = [a_i] = \begin{bmatrix} a_1 \\ a_2 \\ \vdots \\ a_n \end{bmatrix}
  \end{equation*}
\end{definition}

\begin{definition}{行ベクトル}
  数を横に並べたものを\hl{行ベクトル}という。
  \Large
  \begin{equation*}
    \vb*{a} = [a_i] = \begin{bmatrix} a_1 & a_2 & \cdots & a_n \end{bmatrix}
  \end{equation*}
\end{definition}

単に「ベクトル」と言った場合は、列ベクトルを指すことが多い。

\vskip\baselineskip

行ベクトルは、列ベクトルを横倒しにしたもの(列ベクトルの\keyword{転置})と捉えることもできる。

\begin{theorem}{転置による行ベクトルの表現}\quad\\
  行ベクトルは、列ベクトル$\vb*{a}$を\hl{転置}したものとして表現できる。
  \Large
  \begin{equation*}
    \vb*{a}^\top = \begin{bmatrix} a_1 & a_2 & \cdots & a_n \end{bmatrix}
  \end{equation*}
\end{theorem}

\subsection{ベクトルの演算}

ベクトルによって数をまとめて扱えるようにするために、ベクトルどうしの演算を定義したい。

\subsubsection{ベクトルの和}

ベクトルどうしの足し算は、同じ位置にある数どうしの足し算として定義する。

\begin{definition}{ベクトルの和}
  2つの$n$次元ベクトル$\vb*{a}$と$\vb*{b}$の和を次のように定義する。
  \Large
  \begin{equation*}
    \vb*{a} + \vb*{b} = [a_i] + [b_i] = \begin{bmatrix} a_1 + b_1 \\ a_2 + b_2 \\ \vdots \\ a_n + b_n \end{bmatrix}
  \end{equation*}
\end{definition}

$i$番目の数が$\vb*{a}$と$\vb*{b}$の両方に存在していなければ、その位置の数どうしの足し算を考えることはできない。

そのため、ベクトルの和が定義できるのは、同じ次元を持つ(並べた数の個数が同じ)ベクトルどうしに限られる。

\vskip\baselineskip

数ベクトルを「どれくらい進むか」を並べたものと捉えると、同じ位置にある数どうしを足し合わせるということは、同じ向きに進む量を足し合わせるということになる。

たとえば、$x$軸方向に$a_1$、$y$軸方向に$a_2$進んだ場所から、さらに$x$軸方向に$b_1$、$y$軸方向に$b_2$進む…というような「移動の合成」を表すのが、ベクトルの和である。

\todo{図}

\subsubsection{ベクトルのスカラー倍}

「どれくらい進むか」を表す数たち全員に同じ数をかけることで、向きを変えずにベクトルを「引き伸ばす」ことができる。

\todo{図}

ここで位置ごとにかける数を変えてしまうと、いずれかの方向に多く進むことになり、ベクトルの向きが変わってしまう。そのため、「同じ」数をかけることに意味がある。

\todo{図}

\vskip\baselineskip

そこで、ベクトルの定数倍(スカラー倍)を次のように定義する。

\begin{definition}{ベクトルのスカラー倍}
  $n$次元ベクトル$\vb*{a}$の$k$倍を次のように定義する。
  \Large
  \begin{equation*}
    k\vb*{a} = k[a_i] = \begin{bmatrix} ka_1 \\ ka_2 \\ \vdots \\ ka_n \end{bmatrix}
  \end{equation*}
\end{definition}

\subsection{一次結合}

ベクトルを「引き伸ばす」スカラー倍と、「つなぎ合わせる」足し算を組み合わせることで、あるベクトルを他のベクトルを使って表すことができる。

\todo{図}

このように、スカラー倍と和のみを使った形を\keyword{一次結合}もしくは\keyword{線形結合}という。

\subsection{基底}

3次元までのベクトルは、矢印によって「ある点を指し示すもの」として定義できる。

しかし、4次元以上の世界に話を広げるため、ベクトルを単に「数を並べたもの」として再定義した。

\vskip\baselineskip

点を指し示すためのもう一つの概念として、座標がある。

座標も結局は矢印と同様に、$x$軸方向にこのくらい進み、$y$軸方向にこのくらい進む、というように、「進む方向」と「進む長さ」を持つ。

\vskip\baselineskip

単なる数の並びを、向きと大きさを持つ量として復元するための道具が、基底である。

\section{ベクトルの測り方}

\wip

\end{document}
