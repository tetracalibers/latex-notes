\documentclass[../math-imaging]{subfiles}

\begin{document}

\chapter{線形代数}

線形代数は、高次元に立ち向かうための強力な道具となる。

\vskip\baselineskip

どれだけ高次元に話を広げたとしても、「関係」を語る言葉の複雑さが増すことはない。

この章では、そんな状況を実現するための理論を追いかけていく。

\section{ベクトルの作り方}

\subsection{移動の表現としてのベクトル}

\wip

\subsection{ベクトルの多次元化:数ベクトル}

多次元空間内の「移動」を表すには、「縦」と「横」の2方向だけでなく、もっと多くの数が必要になる。

また、4次元を超えてしまうと、矢印の描き方すら想像がつかなくなってしまう。
それは、方向となる軸が多すぎて、どの方向に進むかを表すのが難しくなるためだ。

\vskip\baselineskip

そこで、一旦「向き」の情報を取り除くことで、高次元に立ち向かえないかと考える。

移動を表す矢印は「どの方向に進むか」と「どれくらい進むか」という向きと大きさの情報を持っているが、その「どれくらい進むか」だけを取り出して並べよう。

\vskip\baselineskip

こうして単に「数を並べたもの」もベクトルと呼ぶことにし、このように定義したベクトルを\keyword{数ベクトル}という。

\vskip\baselineskip

数を並べるとき、縦と横の2通りがある。それぞれ\keyword{列ベクトル}、\keyword{行ベクトル}として定義する。

\begin{definition}{列ベクトル}
  数を縦に並べたものを\hl{列ベクトル}という。
  \Large
  \begin{equation*}
    \vb*{a} = [a_i] = \begin{bmatrix} a_1 \\ a_2 \\ \vdots \\ a_n \end{bmatrix}
  \end{equation*}
\end{definition}

\begin{definition}{行ベクトル}
  数を横に並べたものを\hl{行ベクトル}という。
  \Large
  \begin{equation*}
    \vb*{a} = [a_i] = \begin{bmatrix} a_1 & a_2 & \cdots & a_n \end{bmatrix}
  \end{equation*}
\end{definition}

単に「ベクトル」と言った場合は、列ベクトルを指すことが多い。

\vskip\baselineskip

行ベクトルは、列ベクトルを横倒しにしたもの(列ベクトルの\keyword{転置})と捉えることもできる。

\begin{theorem}{転置による行ベクトルの表現}\quad\\
  行ベクトルは、列ベクトル$\vb*{a}$を\hl{転置}したものとして表現できる。
  \Large
  \begin{equation*}
    \vb*{a}^\top = \begin{bmatrix} a_1 & a_2 & \cdots & a_n \end{bmatrix}
  \end{equation*}
\end{theorem}

\subsection{ベクトルの演算}

ベクトルによって数をまとめて扱えるようにするために、ベクトルどうしの演算を定義したい。

\subsubsection{ベクトルの和}

ベクトルどうしの足し算は、同じ位置にある数どうしの足し算として定義する。

\begin{definition}{ベクトルの和}
  2つの$n$次元ベクトル$\vb*{a}$と$\vb*{b}$の和を次のように定義する。
  \Large
  \begin{equation*}
    \vb*{a} + \vb*{b} = [a_i] + [b_i] = \begin{bmatrix} a_1 + b_1 \\ a_2 + b_2 \\ \vdots \\ a_n + b_n \end{bmatrix}
  \end{equation*}
\end{definition}

$i$番目の数が$\vb*{a}$と$\vb*{b}$の両方に存在していなければ、その位置の数どうしの足し算を考えることはできない。

そのため、ベクトルの和が定義できるのは、同じ次元を持つ(並べた数の個数が同じ)ベクトルどうしに限られる。

\vskip\baselineskip

数ベクトルを「どれくらい進むか」を並べたものと捉えると、同じ位置にある数どうしを足し合わせるということは、同じ向きに進む量を足し合わせるということになる。

たとえば、$x$軸方向に$a_1$、$y$軸方向に$a_2$進んだ場所から、さらに$x$軸方向に$b_1$、$y$軸方向に$b_2$進む…というような「移動の合成」を表すのが、ベクトルの和である。

\todo{図}

\subsubsection{ベクトルのスカラー倍}

「どれくらい進むか」を表す数たち全員に同じ数をかけることで、向きを変えずにベクトルを「引き伸ばす」ことができる。

\todo{図}

ここで位置ごとにかける数を変えてしまうと、いずれかの方向に多く進むことになり、ベクトルの向きが変わってしまう。そのため、「同じ」数をかけることに意味がある。

\todo{図}

\vskip\baselineskip

そこで、ベクトルの定数倍(スカラー倍)を次のように定義する。

\begin{definition}{ベクトルのスカラー倍}
  $n$次元ベクトル$\vb*{a}$の$k$倍を次のように定義する。
  \Large
  \begin{equation*}
    k\vb*{a} = k[a_i] = \begin{bmatrix} ka_1 \\ ka_2 \\ \vdots \\ ka_n \end{bmatrix}
  \end{equation*}
\end{definition}

\subsection{一次結合}

ベクトルを「引き伸ばす」スカラー倍と、「つなぎ合わせる」足し算を組み合わせることで、あるベクトルを他のベクトルを使って表すことができる。

\todo{図}

このように、スカラー倍と和のみを使った形を\keyword{一次結合}もしくは\keyword{線形結合}という。

\subsection{基底}

3次元までのベクトルは、矢印によって「ある点を指し示すもの」として定義できる。

しかし、4次元以上の世界に話を広げるため、ベクトルを単に「数を並べたもの」として再定義した。

\vskip\baselineskip

点を指し示すためのもう一つの概念として、座標がある。

座標も結局は矢印と同様に、$x$軸方向にこのくらい進み、$y$軸方向にこのくらい進む、というように、「進む方向」と「進む長さ」を持つ。

\vskip\baselineskip

単なる数の並びを、向きと大きさを持つ量として復元するための道具が、基底である。

\section{ベクトルの測り方}

\wip

\end{document}
