\documentclass[../imaging-math]{subfiles}

\begin{document}

\chapter{線形代数}

線形代数は、高次元に立ち向かうための強力な道具となる。

\vskip\baselineskip

どれだけ高次元に話を広げたとしても、「関係」を語る言葉の複雑さが増すことはない。

この章では、そんな状況を実現するための理論を追いかけていく。

\subfile{linear_algebra/vector_def}
\subfile{linear_algebra/linear_dependence}
\subfile{linear_algebra/vector_space}
\subfile{linear_algebra/inner_product}

\section{ベクトルの直交性}

\begin{mindflow}
  % irobutsu: 2.3~2.4, 5.1~5.2
  \begin{enumerate}
    \item ベクトルの直交の定義
    \item 直交基底の一次結合の係数
    \item シュミットの直交化法
    \item 直交化で余分なベクトルを削る
  \end{enumerate}
\end{mindflow}

\section{ベクトルの外積}

\begin{mindflow}
  % irobutsu: 2.5~2.7
  \begin{enumerate}
    \item 外積
    \item レヴィ・チビタ記号
    \item 内積と外積の公式
  \end{enumerate}
\end{mindflow}

\section{行列と線形写像}

\begin{mindflow}
  % irobutsu: 1.1~1.4, 3.1, 3.2.4, 3.3~3.4, A.3〜A.5, B.4~B.6
  \begin{enumerate}
    \item 行列の定義と意味
    \item 行列の積
    \item 転置による内積の表記
    \item 線形写像
    \item 線形変換
    \item 逆行列
  \end{enumerate}
\end{mindflow}

\section{行列と連立方程式}

\begin{mindflow}
  % irobutsu: 4.1~4.8
  \begin{enumerate}
    \item 行列の基本変形
    \item ...
  \end{enumerate}
\end{mindflow}

\section{行列式}

\begin{mindflow}
  % irobutsu: 3.2, 3.5〜3.8
  \begin{enumerate}
    \item 行列式
    \item ヤコビアン
    \item 余因子
  \end{enumerate}
\end{mindflow}

\end{document}
