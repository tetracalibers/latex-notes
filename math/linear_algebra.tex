\documentclass[../imaging-math]{subfiles}

\begin{document}

\chapter{線形代数}

線形代数は、高次元に立ち向かうための強力な道具となる。

\vskip\baselineskip

どれだけ高次元に話を広げたとしても、「関係」を語る言葉の複雑さが増すことはない。

この章では、そんな状況を実現するための理論を追いかけていく。

\subfile{linear_algebra/vector_def}
\subfile{linear_algebra/linear_dependence}
\subfile{linear_algebra/vector_space}
\subfile{linear_algebra/inner_product}

\section{ベクトルの直交性}

\begin{mindflow}
  % irobutsu: 2.3~2.4, 5.1~5.2
  \begin{enumerate}
    \item ベクトルの直交の定義
    \item 直交基底の一次結合の係数
    \item シュミットの直交化法
    \item 直交化で余分なベクトルを削る
  \end{enumerate}
\end{mindflow}

\end{document}
