\documentclass[../../imaging-math]{subfiles}

\begin{document}

\section{基底にできるベクトルを探す}

2次元座標系では、平面上のあらゆる点を表すことができ、それらの点はベクトルで指し示す形でも表現できる。

基底が「座標系を設置するための土台」となるなら、基底とは、あらゆるベクトルを表すための材料とみなすことができる。

\br

では、基底として使えるベクトルとは、どのようなベクトルだろうか?

\subsection{基底とは過不足ない組み合わせ}

\subsubsection{不十分を考える}

2次元座標系を表現するにあたって、必ずしも基底ベクトルが直交している必要はない。

しかし、平行なベクトルは明らかに基底(座標軸の土台)として使うことはできない。

\begin{center}
  \begin{tikzpicture}
    \def\a{1.5}
    \def\b{2.5}

    % ベクトルa_1を引き伸ばしたx軸
    \draw[axis] (0, 0) -- (3*\a, 0) node[right] {$x$};
    % ベクトルa_1
    \draw[vector, Rhodamine] (0, 0) -- (\a, 0) node[midway, above] {$\vb*{a}_1$};

    \begin{scope}[yshift=-1em]
      % ベクトルa_2を引き伸ばしたy軸
      \draw[axis] (0, 0) -- (2.5*\b, 0) node[right] {$y$};
      % ベクトルa_2
      \draw[vector, Cerulean] (0, 0) -- (\b, 0) node[midway, below] {$\vb*{a}_2$};
    \end{scope}
  \end{tikzpicture}
\end{center}

$x$軸と$y$軸が平行だと、$(x,y)$の組で平面上の点を表すことはできない。

2次元平面$\mathbb{R}^2$上の点やベクトルは、2つの方向を用意しないと表せないのだから、基底となるベクトルは互いに平行でない必要がある。

\subsubsection{無駄を考える}

平行な2つのベクトルは、互いに互いをスカラー倍で表現できてしまう。このようなベクトルの組は基底にはできない。

\begin{equation*}
  \vb*{a}_2 = k \vb*{a}_1
\end{equation*}

この平行な2つのベクトル$\{\vb*{a}_1,\vb*{a}_2\}$に加えて、これらに平行でないもう1つのベクトル$\vb*{a}_3$を用意すれば、$\vb*{a}_1$と$\vb*{a}_3$の一次結合か、$\vb*{a}_2$と$\vb*{a}_3$の一次結合かのどちらかで、平面上の他のベクトルを表現できるようになる。

しかし、$\vb*{a}_2$は結局$\vb*{a}_1$のスカラー倍($\vb*{a}_1$と$\vb*{a}_3$の一次結合の特別な場合)で表現できてしまうのだから、「他のベクトルを表す材料」となるベクトルの組を考える上で、$\vb*{a}_2$は無駄なベクトルだといえる。

\br

2次元平面を表現するには2本の座標軸があれば十分なように、基底とは、「これさえあれば他のベクトルを表現できる」という、必要最低限のベクトルの組み合わせにしたい。

基底の候補の中に、互いに互いを表現できる複数のベクトルが含まれているなら、その中の1つを残せば十分である。

\froufrou

ここまでの考察から、あるベクトルの組を基底として使えるかどうかを考える上で、「互いに互いを表現できるか」という視点が重要になることがわかる。

\begin{itemize}
  \item 互いにスカラー倍で表現できるベクトルだけでは不十分
  \item 互いに一次結合で表現できるベクトルが含まれていると無駄がある
\end{itemize}

ベクトルの組の「互いに互いを表現できるか」に着目した性質を表現する概念として、\keyword{一次従属}と\keyword{一次独立}がある。

\begin{itemize}
  \item \keyword{一次従属}:互いに互いを表現できるベクトルが含まれていること
  \item \keyword{一次独立}:互いに互いを表現できない、独立したベクトルだけで構成されていること
\end{itemize}

\subsection{一次従属}

ベクトルの組を考え、どれか1つのベクトルがほかのベクトルの一次結合で表せるとき、それらのベクトルの組は\keyword{一次従属}であるという。

\begin{definition}{一次従属}\quad\\
  $k$個のベクトル$\vb*{a}_i = \{ \vb*{a}_1, \vb*{a}_2, \ldots, \vb*{a}_k \}$が\hl{一次従属}であるとは、少なくとも1つは$0$でない$k$個の係数$c_i = \{c_1, c_2, \ldots, c_k\}$を用意すれば、それらを使った一次結合を零ベクトル$\vb{0}$にできることをいう。
  \large
  \begin{equation*}
    \sum_{i=1}^{k-1} c_i \vb*{a}_i = c_1 \vb*{a}_1 + c_2 \vb*{a}_2 + \cdots + c_k\vb*{a}_{k} = \vb{0}
  \end{equation*}
\end{definition}

たとえば、$c_1$が$0$でないとき、一次結合を零ベクトルにできるということは、次のような式変形ができることになる。
\begin{equation*}
  \vb*{a}_1 = -\frac{c_2}{c_1} \vb*{a}_2 - \frac{c_3}{c_1} \vb*{a}_3 - \cdots - \frac{c_k}{c_1} \vb*{a}_{k}
\end{equation*}
つまり、ベクトル$\vb*{a}_1$をほかのベクトルの一次結合で表せている。

\subsubsection{「従属」という言葉を味わう}

自分自身をほかのベクトルを使って表現できるということは、ほかのベクトルに依存している(従っている)ということになる。

\br

たとえば、$\vb*{a}_1$と$\vb*{a}_2$の一次結合で表せるベクトル$\vb*{a}_3$は、 この2つのベクトル$\vb*{a}_1 ,\, \vb*{a}_2$に従っているといえる。
\begin{equation*}
  \vb*{a}_3 = 2 \vb*{a}_1 + \vb*{a}_2
\end{equation*}

しかし、「$\vb*{a}_3$が$\vb*{a}_1 ,\, \vb*{a}_2$に従っている」という一方的な主従関係になっているわけではない。その逆もまた然りである。

なぜなら、次のような式変形もできるからだ。
\begin{equation*}
  \vb*{a}_2 = \vb*{a}_3 - 2 \vb*{a}_1
\end{equation*}
この式で見れば、今度は$\vb*{a}_2$が$\vb*{a}_1 ,\, \vb*{a}_3$に従っていることになる。

\br

このように、一次従属とは、「どちらがどちらに従う」という主従関係ではなく、ベクトルの組の中での相互の依存関係を表すものである。

\subsection{一次独立}

一次従属は、いずれかのベクトルをほかのベクトルで表現できること、つまり基底の候補としては無駄が含まれている。
そこで、その逆を考える。

\br

互いに互いを表現できるような無駄なベクトルが含まれておらず、各々が独立している(無関係である)ベクトルの組は\keyword{一次独立}であるという。

\begin{definition}{一次独立}\quad\\
  $k$個のベクトル$\vb*{a}_i = \{ \vb*{a}_1, \vb*{a}_2, \ldots, \vb*{a}_k \}$が\hl{一次独立}であるとは、$k$個の係数$c_i = \{c_1, c_2, \ldots, c_k\}$がすべて$0$であるときしか、それらを使った一次結合を零ベクトル$\vb{0}$にできないことをいう。
  \large
  \begin{gather*}
    \sum_{i=1}^{k-1} c_i \vb*{a}_i = c_1 \vb*{a}_1 + c_2 \vb*{a}_2 + \cdots + c_k\vb*{a}_{k} = \vb{0} \\ \quad\Longrightarrow \quad c_1 = c_2 = \cdots = c_k = 0
  \end{gather*}
\end{definition}

たとえば、係数$c_1$が$0$でないとすると、
\begin{equation*}
  \vb*{a}_1 = -\frac{c_2}{c_1} \vb*{a}_2 - \frac{c_3}{c_1} \vb*{a}_3 - \cdots - \frac{c_k}{c_1} \vb*{a}_{k}
\end{equation*}
のように、$\vb*{a}_1$をほかのベクトルで表現できてしまう。これでは一次従属である。

ほかの係数についても同様で、どれか1つでも係数が$0$でなければ、いずれかのベクトルをほかのベクトルで表現できてしまうのである。

このような式変形ができないようにするには、係数はすべて$0$でなければならない。

\br

一次独立には、互いに互いを表現できないようにする条件が課されているため、一次独立なベクトルの組は無駄なベクトルを含まず、基底の候補となり得る。

\end{document}
