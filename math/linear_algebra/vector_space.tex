\documentclass[../../imaging-math]{subfiles}

\begin{document}

\section{ベクトルが作る空間}

ここまで、ベクトルとその演算(和とスカラー倍)を定義し、一次結合によってベクトルを作ったり、基底という特別なベクトルによって座標系を構成する例などを見てきた。

一次結合は和とスカラー倍という演算の組み合わせであるから、結局は\keyword{演算}によってベクトルを自由に表現できるようになったといえる。

\br

ここでは、演算の性質に着目してベクトルの集合を捉え、その中で基底がどのような役割を果たすのかを整理する。

\subsection{集合と演算で空間を作る}

演算によってベクトルを別なベクトルに変換したり、演算の組み合わせ(一次結合)によって新たなベクトルを作ったりすることができる。

\br

ベクトルと呼ばれる対象を集めて、さらに演算(和とスカラー倍)を導入することでベクトルからベクトルへの行き来を可能にした集合を\keyword{線形空間}(\keyword{ベクトル空間})という。

\br

線形空間は、ベクトルたちが自由に動き回れるルール付きの“空間”である。

この空間の中では、次のような操作(演算)がきちんと意味を持ってできるようになっている。

\begin{itemize}
  \item ベクトルどうしを合成する(和)
  \item ベクトルのスケールを変える(スカラー倍)
\end{itemize}

そして、これらの演算で作られた新しいベクトルも、この空間の中に入っていることが保証されている。
演算の結果が想定外にならない、安全な場所である。

\wip

\begin{mindflow}
  % irobutsu: B.1~B.3
  \begin{enumerate}
    \item 線形性と線形空間
    \item 一次結合で線形空間を作る
    \item 基底が作るもの(基底の厳密な定義)
  \end{enumerate}
\end{mindflow}

\end{document}
