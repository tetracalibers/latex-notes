\documentclass[../../imaging-math]{subfiles}

\begin{document}

\section{関数の極限}

\subsection{$\varepsilon - \delta$論法による関数の極限}

$\varepsilon$がどんなに小さい正の数であっても、$x$と$a$の誤差を$\delta$以内に収めることで$f(x)$と$b$の誤差が$\varepsilon$以内に収まるとき、関数$f(x)$は点$a$で$b$に収束するという。

\froufrou

まず、$y=b$の周りに、両側それぞれ$\varepsilon$だけ広げた区間を考える。(この区間を青い帯と呼ぶことにする。)

\begin{center}
  \scalebox{0.9}{
    \begin{tikzpicture}
      \def\a{3} % a
      \def\diff{-0.4} % aとxの誤差
      \def\ep{0.6} % ε
      \def\dl{0.5} % δ

      \def\xmin{-1};
      \def\xmax{6};
      \def\ymin{-1};
      \def\ymax{5};
      \def\fn#1{(#1-2)^3/6 - 1.5*(#1-2)/1.5 + 3};

      % 座標軸
      \draw[axis] (\xmin,0) -- (\xmax,0) node [right]{\large$x$};
      \draw[axis] (0,\ymin) -- (0,\ymax) node [above]{\large$y$};

      % 原点
      \node at (0,0) [below left]{$O$};

      % δ区間
      %\fill[carnationpink, fill opacity=0.5] (\a-\dl,0) rectangle (\a+\dl,\ymax);
      % ε区間
      \fill[SkyBlue, fill opacity=0.5] (0,{\fn{\a}-\ep}) rectangle (\xmax,{\fn{\a}+\ep});

      % δ区間を表す矢印
      %\draw[<->, hotpink, yshift=0.5em] (\a-\dl,\ymax) -- (\a,\ymax) node[midway, above]{$\delta$};
      %\draw[<->, hotpink, yshift=0.5em] (\a,\ymax) -- (\a+\dl,\ymax) node[midway, above]{$\delta$};

      % ε区間を表す矢印
      \draw[<->, SkyBlue, xshift=0.5em] (\xmax,{\fn{\a}-\ep}) -- (\xmax,{\fn{\a}}) node[midway, right]{$\varepsilon$};
      \draw[<->, SkyBlue, xshift=0.5em] (\xmax,{\fn{\a}}) -- (\xmax,{\fn{\a}+\ep}) node[midway, right]{$\varepsilon$};

      % 関数のグラフ
      \begin{scope}
        \clip (\xmin/3,\ymin) rectangle (\xmax,\ymax);
        \draw[BurntOrange,thick,samples=100] plot[domain=\xmin:\xmax] (\x,{\fn{\x}});
      \end{scope}
      % 文字に被る部分は点線にする
      \begin{scope}
        \clip (\xmin,0) rectangle (\xmin/3,\ymax);
        \draw[BurntOrange, dotted, thick,samples=100] plot[domain=\xmin:\xmax] (\x,{\fn{\x}});
      \end{scope}

      % x=aの直線
      %\draw[magenta, dashed] (\a,0) node[below]{$a$} -- (\a,\ymax);
      % y=bの直線
      \draw[cyan, dashed] (0,{\fn{\a}}) node[left]{$b$} -- (\xmax,{\fn{\a}});

      % 点(x, f(x))を表す補助線
      %\draw[gray, semithick] (\a+\diff,0) node[below]{$x$} -- (\a+\diff,{\fn{\a+\diff}});
      % y軸にも写す
      %\draw[gray, semithick] (0,{\fn{\a+\diff}}) node[left]{$f(x)$} -- (\a+\diff,{\fn{\a+\diff}});
    \end{tikzpicture}
  }
\end{center}

$x=a$の周りには、両側それぞれ$\delta$だけ広げた区間を考える。(この区間をピンクの帯と呼ぶことにする。)

このとき、「この$x$であれば、$f(x)$が青い帯に収まる」という$x$を探して、その$x$をピンクの帯で包むように$\delta$を設定する。

\begin{center}
  \scalebox{1}{
    \begin{tikzpicture}
      \def\a{3} % a
      \def\diff{-0.5} % aとxの誤差
      \def\ep{0.6} % ε
      \def\dl{0.7} % δ

      \def\xmin{-1};
      \def\xmax{6};
      \def\ymin{-1};
      \def\ymax{5};
      \def\fn#1{(#1-2)^3/6 - 1.5*(#1-2)/1.5 + 3};

      % 座標軸
      \draw[axis] (\xmin,0) -- (\xmax,0) node [right]{\large$x$};
      \draw[axis] (0,\ymin) -- (0,\ymax) node [above]{\large$y$};

      % 原点
      \node at (0,0) [below left]{$O$};

      % δ区間
      \fill[carnationpink, fill opacity=0.5] (\a-\dl,0) rectangle (\a+\dl,\ymax);
      % ε区間
      \fill[SkyBlue, fill opacity=0.5] (0,{\fn{\a}-\ep}) rectangle (\xmax,{\fn{\a}+\ep});

      % δ区間を表す矢印
      \draw[<->, hotpink, yshift=0.5em] (\a-\dl,\ymax) -- (\a,\ymax) node[midway, above]{$\delta$};
      \draw[<->, hotpink, yshift=0.5em] (\a,\ymax) -- (\a+\dl,\ymax) node[midway, above]{$\delta$};

      % ε区間を表す矢印
      \draw[<->, SkyBlue, xshift=0.5em] (\xmax,{\fn{\a}-\ep}) -- (\xmax,{\fn{\a}}) node[midway, right]{$\varepsilon$};
      \draw[<->, SkyBlue, xshift=0.5em] (\xmax,{\fn{\a}}) -- (\xmax,{\fn{\a}+\ep}) node[midway, right]{$\varepsilon$};

      % 関数のグラフ
      \begin{scope}
        \clip (\xmin/3,\ymin) rectangle (\xmax,\ymax);
        \draw[BurntOrange,thick,samples=100] plot[domain=\xmin:\xmax] (\x,{\fn{\x}});
      \end{scope}
      % 文字に被る部分は点線にする
      \begin{scope}
        \clip (\xmin,0) rectangle (\xmin/3,\ymax);
        \draw[BurntOrange, dotted, thick,samples=100] plot[domain=\xmin:\xmax] (\x,{\fn{\x}});
      \end{scope}

      % x=aの直線
      \draw[magenta, dashed] (\a,0) node[below]{$a$} -- (\a,\ymax);
      % y=bの直線
      \draw[cyan, dashed] (0,{\fn{\a}}) node[left]{$b$} -- (\xmax,{\fn{\a}});

      % 点(x, f(x))を表す補助線
      \draw[gray, semithick] (\a+\diff,0) node[below]{$x$} -- (\a+\diff,{\fn{\a+\diff}});
      % y軸にも写す
      \draw[gray, semithick] (0,{\fn{\a+\diff}}) node[left]{$f(x)$} -- (\a+\diff,{\fn{\a+\diff}});
    \end{tikzpicture}
  }
\end{center}

$\varepsilon$は正の数ならなんでもよいとすれば、$\varepsilon$を小さな数に設定し、いくらでも青い帯を狭めることができる。

しかしこのとき、$x$をピンクの帯に収まるようにしなければならない。

ピンクの帯の中心は$a$なので、$x$をピンクの帯に収めようとすると、$x$は$a$に近づいていくことになる。

\begin{center}
  \scalebox{1.2}{
    \begin{tikzpicture}
      \def\a{3} % a
      \def\diff{-0.15} % aとxの誤差
      \def\ep{0.175} % ε
      \def\dl{0.225} % δ

      \def\xmin{-1};
      \def\xmax{6};
      \def\ymin{-1};
      \def\ymax{5};
      \def\fn#1{(#1-2)^3/6 - 1.5*(#1-2)/1.5 + 3};

      % 座標軸
      \draw[axis] (\xmin,0) -- (\xmax,0) node [right]{\large$x$};
      \draw[axis] (0,\ymin) -- (0,\ymax) node [above]{\large$y$};

      % 原点
      \node at (0,0) [below left]{$O$};

      % δ区間
      \fill[carnationpink, fill opacity=0.5] (\a-\dl,0) rectangle (\a+\dl,\ymax);
      % ε区間
      \fill[SkyBlue, fill opacity=0.5] (0,{\fn{\a}-\ep}) rectangle (\xmax,{\fn{\a}+\ep});

      % δ区間を表す矢印
      %\draw[<->, hotpink, yshift=0.5em] (\a-\dl,\ymax) -- (\a,\ymax) node[midway, above]{$\delta$};
      %\draw[<->, hotpink, yshift=0.5em] (\a,\ymax) -- (\a+\dl,\ymax) node[midway, above]{$\delta$};

      % ε区間を表す矢印
      %\draw[<->, SkyBlue, xshift=0.5em] (\xmax,{\fn{\a}-\ep}) -- (\xmax,{\fn{\a}}) node[midway, right]{$\varepsilon$};
      %\draw[<->, SkyBlue, xshift=0.5em] (\xmax,{\fn{\a}}) -- (\xmax,{\fn{\a}+\ep}) node[midway, right]{$\varepsilon$};

      % 関数のグラフ
      \begin{scope}
        \clip (\xmin/3,\ymin) rectangle (\xmax,\ymax);
        \draw[BurntOrange,thick,samples=100] plot[domain=\xmin:\xmax] (\x,{\fn{\x}});
      \end{scope}
      % 文字に被る部分は点線にする
      \begin{scope}
        \clip (\xmin,0) rectangle (\xmin/3,\ymax);
        \draw[BurntOrange, dotted, thick,samples=100] plot[domain=\xmin:\xmax] (\x,{\fn{\x}});
      \end{scope}

      % x=aの直線
      \draw[magenta, dashed] (\a,0) node[below]{$a$} -- (\a,\ymax);
      % y=bの直線
      \draw[cyan, dashed] (0,{\fn{\a}}) node[left]{$b$} -- (\xmax,{\fn{\a}});

      % 点(x, f(x))を表す補助線
      \draw[gray, semithick] (\a+\diff,0) node[below left]{$x$} -- (\a+\diff,{\fn{\a+\diff}});
      % y軸にも写す
      \draw[gray, semithick] (0,{\fn{\a+\diff}}) node[above left]{$f(x)$} -- (\a+\diff,{\fn{\a+\diff}});
    \end{tikzpicture}
  }
\end{center}


青い帯の幅$\varepsilon$がどんなに小さくても、ピンクの帯の幅$\delta$を小さくしていけば、$x$と$f(x)$をそれぞれ帯の中に収めることができる。

このように、$x$を$a$に近い範囲に閉じ込めれば、$f(x)$も$b$に近い範囲に閉じ込められるという状況を、点$a$での関数の収束と定義する。

\vskip\baselineskip

青い帯の幅$\varepsilon$がどんなに小さくても、「この$x$であれば、$f(x)$が青い帯に収まる」という$x$がピンクの帯からはみ出ないように$\delta$を小さくしていけるなら、自動的に$x$も$f(x)$もそれぞれ帯の中に収まる。

つまり、$\delta$に課された制約が肝心で、「この$x$であれば、$f(x)$が青い帯に収まる」という$x$を包めるような$\delta$の存在が、収束を保証することになる。

\vskip\baselineskip

\begin{definition}{関数の収束と極限値($x\to a$の場合)}
  \titlegap
  関数$f(x)$と実数$a,b$について、次の条件を考える。
  \begin{spacebox}
    任意の正の数$\varepsilon$に対して
    \Large
    \begin{equation}
      |x - a| < \delta \quad \Longrightarrow \quad |f(x) - b| < \varepsilon
    \end{equation}
    \normalsize
    が成り立つような正の数$\delta$が存在する
  \end{spacebox}
  この条件が成り立つとき、関数$f(x)$は点$a$で$b$に\hl{収束}するといい、次のように表す。
  \LARGE
  \begin{equation}
    \lim_{x \to a} f(x) = b \quad \text{\normalsize または} \quad f(x) \to b \quad (x \to a)
  \end{equation}
  \normalsize
  このとき、$b$を数列$f(x)$の\hl{極限値}という。
\end{definition}

\todo{定義1.1}

\subsection{関数の極限と数列の極限の関係}

\todo{定理1.7}

\subsection{関数の極限の性質}

\todo{定理1.8}

\todo{定理1.9}

\subsection{はさみうち法}

\todo{定理1.10}

\subsection{合成関数の極限}

\todo{定理1.11}

\subsection{右極限と左極限}

\todo{定義1.15}

\todo{定義1.16}

\todo{定理1.19}

\end{document}
