\documentclass[../book_need-set-theory]{subfiles}

\begin{document}

\section{述語論理}

命題関数は、\keyword{述語}とも呼ばれる

述語は、変数がとりうる値の範囲とセットで考える

\begin{oframed}
  \paragraph{定義(自由変数、変域)}
  命題関数$P(x)$について、代入する$x$のことを\keyword{自由変数}といい、自由変数がとりうる範囲を\keyword{変域}という
\end{oframed}

\section{全称記号、存在記号}



\end{document}
