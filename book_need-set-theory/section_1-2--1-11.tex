\documentclass[../book_need-set-theory]{subfiles}

\begin{document}

\section{命題関数}

たとえば「$\alpha$は整数である」という命題は、$\alpha$が決まらなければ正しいか正しくないかは判断できない

\begin{itemize}
  \item $\alpha=1$のときは「1は整数である」という正しい命題
  \item $\alpha=2.5$のときは「2.5は整数である」という正しくない命題
\end{itemize}

文字が一つ確定すると命題が一つ定まるというシステムを関数と捉えて\keyword{命題関数}という

\begin{oframed}
  \paragraph{定義(命題関数)}
  文章中に変数を含み、その変数を定めるごとに命題になる文章を\keyword{命題関数}という
\end{oframed}

命題関数が正しいかどうかは、あらゆる変数をすべて入れていき、その都度作られる命題がすべて正しい場合と定義される

\begin{oframed}
  \paragraph{定義(命題関数が正しいとは)}
  命題関数が正しいとは、含まれる変数を定めるごとに決まる命題が全て正しいということ
\end{oframed}

\end{document}
