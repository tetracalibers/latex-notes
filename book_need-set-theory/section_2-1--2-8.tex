\documentclass[../book_need-set-theory]{subfiles}

\begin{document}

\section{集合とは}

「何かしらの対象」と「何かしらの集まり」としておけば、汎用性が高いまま抽象的な議論ができる点が集合を勉強する意義

\begin{oframed}
  \paragraph{定義}
  何かしらの対象の集まりを\keyword{集合}といい、その集合に入る何かしらの対象を\keyword{元}という
\end{oframed}

\begin{oframed}
  \paragraph{定義(空集合)}
  何も含まれていない集まりのことを\keyword{空集合}といい、$\phi $で表す
\end{oframed}

任意の対象は、「ある集合$A$の元」か「ある集合$A$の元でない」かどちらかが考えられる

\begin{oframed}
  \paragraph{定義}
  集合$A$があるとする。
  このとき、ある対象$a$が集合$A$に入ることを$a \in A$と表し、$a$が集合$A$に入らないことを$a \notin A$と表す
\end{oframed}

集合は、どんなものが集まっているかを表すために、$\{\text{元}\mid\text{条件}\}$という書き方をする

\begin{equation*}
  \text{(偶数の集合)} = \{y \mid y = 2m, m \in \mathbb{Z} \}
\end{equation*}

\section{集合同士の関係}

\subsection{補集合}

たとえば「スマホを持っている人」の集合を考えると、「スマホを持っていない人」の集合も自然と考えることができる

\begin{oframed}
  \paragraph{定義(補集合)}
  集合$A$に対して$A$の元でないものの集合$\{x \mid x \notin A\}$を集合$A$の\keyword{補集合}といい、$A^c$とかく
\end{oframed}

\subsection{積集合}

\end{document}
