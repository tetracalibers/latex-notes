\documentclass[b5paper,12pt]{jsarticle}

\title{Topic Note: 命題論理}
\author{tomixy}

% font
\usepackage[T1]{fontenc}
\usepackage{lxfonts}
% monospace font
\renewcommand*\ttdefault{cmvtt}

% === color ===

% ref: https://latexcolor.com/
\definecolor{hotpink}{rgb}{1.0, 0.41, 0.71}
\definecolor{carnationpink}{rgb}{1.0, 0.65, 0.79}
\definecolor{deeppink}{rgb}{1.0, 0.08, 0.58}
\definecolor{capri}{rgb}{0.0, 0.75, 1.0}
\definecolor{rosepink}{rgb}{1.0, 0.4, 0.8}
\definecolor{princetonorange}{rgb}{1.0, 0.56, 0.0}
\definecolor{lavendermagenta}{rgb}{0.93, 0.51, 0.93}
\definecolor{malachite}{rgb}{0.04, 0.85, 0.32}
\definecolor{lawngreen}{rgb}{0.49, 0.99, 0.0}
\definecolor{periwinkle}{rgb}{0.8, 0.8, 1.0}
\definecolor{lightslategray}{rgb}{0.47, 0.53, 0.6}
\definecolor{robineggblue}{rgb}{0.0, 0.8, 0.8}
\definecolor{rosebonbon}{rgb}{0.98, 0.26, 0.62}
\definecolor{airforceblue}{rgb}{0.36, 0.54, 0.66}
\definecolor{columbiablue}{rgb}{0.61, 0.87, 1.0}
\definecolor{magnolia}{rgb}{0.97, 0.96, 1.0}
\definecolor{coolgrey}{rgb}{0.55, 0.57, 0.67}

% === box ===

\usepackage{awesomebox}

% === math ===

\usepackage{physics}
\usepackage{braket}

\usepackage{amssymb} % use \blacksquare

\usepackage{amsthm} % 定理環境とQEDコマンド
\renewcommand{\qedsymbol}{\textcolor{coolgrey}{$\blacksquare$}}

\usepackage{mathtools}
% 別の場所で参照する数式以外は番号が付かないように
\mathtoolsset{showonlyrefs=true}

\usepackage{systeme} % 連立方程式を簡単に書く
\usepackage{empheq}

\newcommand{\id}{\operatorname{id}}
\newcommand{\Id}{\operatorname{Id}}
\newcommand{\diag}{\operatorname{diag}}
\newcommand{\Ker}{\operatorname{Ker}}
\newcommand{\sgn}{\operatorname{sgn}}

\newcommand{\suchthat}{\,\, s.t. \,\,}
\newcommand{\transpose}[1]{{}^t\! #1}

% === font ===

\usepackage{amsfonts} % use \mathbb

\usepackage[T1]{fontenc}
\usepackage{lxfonts}

% monospace font
\renewcommand*\ttdefault{cmvtt}

% === layout ===

\usepackage[top=20truemm,bottom=20truemm,left=20truemm,right=60truemm,marginparwidth=40truemm,marginparsep=10truemm]{geometry} % 余白
\renewcommand{\baselinestretch}{1.25} % 行間

\usepackage{leading}

\setlength{\parindent}{0pt} % 段落始めでの字下げをしない

\usepackage{enumitem}
\newcommand{\romanlabel}{\textsf{\roman*.}}
\newcommand{\romannum}[1]{\textsf{#1}}

\usepackage[noparboxrestore]{marginnote}

\usepackage{tocloft}
% chapterのnumwidthを広くする
\setlength{\cftchapnumwidth}{5em}

\usepackage{titling}
\renewcommand{\maketitlehooka}{\textsf}

% === tikz ===

\usepackage[dvipdfmx]{graphicx}

\usepackage{tikz}
\usetikzlibrary{
  fit,
  patterns,
  decorations.pathreplacing,
  cd,
  petri,
  positioning
}

\usepackage{ifthen}
\usepackage{listofitems} % for \readlist to create arrays

\usepackage{witharrows}
\usepackage{nicematrix}

% === tcolorbox ===

\usepackage{tcolorbox}
\tcbuselibrary{listings,breakable,xparse,skins,hooks,theorems}

\newcommand{\titlegap}{\quad\\[0.1cm]}

\DeclareTColorBox{definition}{m O{} }%
{
  enhanced,
  colframe=magnolia,
  colback=magnolia!20!white,
  coltitle=black,
  fonttitle=\bfseries,
  breakable,
  sharp corners,
  title={\textcolor{Cerulean!60!black}{\faGraduationCap}\hspace{0.1em} #1},
  detach title,
  before upper={\tcbtitle\quad},
  bottom=0.5cm,
  top=0.5cm,
  right=0.5cm,
  left=0.5cm,
  #2
}

\DeclareTColorBox{theorem}{m O{} }%
{
  enhanced,
  colframe=magnolia,
  colback=magnolia!20!white,
  coltitle=black,
  fonttitle=\bfseries,
  breakable,
  sharp corners,
  title={\textcolor{magenta!70!black}{\faAnchor}\hspace{0.1em} #1},
  detach title,
  before upper={\tcbtitle\quad},
  bottom=0.5cm,
  top=0.5cm,
  right=0.5cm,
  left=0.5cm,
  #2
}

% 背景がグレー
\DeclareTColorBox{shaded}{O{} }%
{
  enhanced,
  colframe=white,
  colback=gray!10,
  breakable=true,
  sharp corners,
  detach title,
  bottom=0.25cm,
  top=0.25cm,
  right=0.25cm,
  left=0.25cm,
  #1
}

\newcommand{\ProofColor}{coolgrey}
\DeclareTColorBox{proof}{O{証明}}{%
  empty,
  title={\faBroom #1},
  attach boxed title to top left,
  sharp corners,
  boxed title style={
      empty,
      size=minimal,
      toprule=2pt,
      top=4pt,
      left=1em,
      right=1em,
      top=0.25cm,
      overlay={
          \draw[\ProofColor, double,line width=1pt] ([yshift=-1pt]frame.north west)--([yshift=-1pt]frame.north east);
        }
    },
  coltitle=\ProofColor,
  fonttitle=\bfseries,
  before=\par\medskip\noindent,
  parbox=false,
  boxsep=0pt,
  left=1em,
  right=1em,
  top=0.5cm,
  bottom=0.5cm,
  breakable,
  pad at break*=0mm,
  vfill before first,
  overlay unbroken={
      \draw[\ProofColor,line width=0.5pt]
      ([yshift=-1pt]title.north east)
      --([xshift=-0.5pt,yshift=-1pt]title.north-|frame.east)
      --([xshift=-0.5pt]frame.south east)
      --(frame.south west);
    },
  overlay first={
      \draw[\ProofColor,line width=1pt]([yshift=-1pt]title.north east)--([xshift=-0.5pt,yshift=-1pt]title.north-|frame.east)--([xshift=-0.5pt]frame.south east);
    },
  overlay middle={
      \draw[\ProofColor,line width=1pt] ([xshift=-0.5pt]frame.north east)--([xshift=-0.5pt]frame.south east);
    },
  overlay last={
      \draw[\ProofColor,line width=1pt] ([xshift=-0.5pt]frame.north east)--([xshift=-0.5pt]frame.south east)--(frame.south west);
    },%
}
\NewDocumentCommand{\patterntitle}{m}{
  \tcbox[
    enhanced,
    empty,
    boxsep=0pt,
    left=0pt,right=0pt,
    bottom=2pt,
    fonttitle=\bfseries,
    borderline south={0.5pt}{0pt}{\ProofColor},
  ]{\textcolor{\ProofColor}{#1}}
}
\renewenvironment{quote}{%
  \list{}{%
    \leftmargin0.5cm   % this is the adjusting screw
    \rightmargin\leftmargin
  }
  \item\relax
}{\endlist}
\newenvironment{subpattern}[1]{
  \patterntitle{#1}
  \begin{quote}
    }{
  \end{quote}
}

% === memo ===

\usepackage{zebra-goodies} % TODOなどの注釈

% === original ===

\newcommand{\keyword}[1]{\textcolor{RubineRed}{\textbf{#1}}}
\newcommand{\en}[1]{\textcolor{RubineRed}{\small\texttt{#1}}}
\newcommand{\keywordJE}[2]{\keyword{#1}(\en{\textcolor{RubineRed!60}{#2}})}

\newcommand{\br}{\vskip0.5\baselineskip}

\usepackage[object=vectorian]{pgfornament}
\newcommand{\sectionline}{%
  \noindent
  \begin{center}
    {\color{lightgray}
      \resizebox{0.5\linewidth}{1ex}
      {{%
            {\begin{tikzpicture}
                  \node  (C) at (0,0) {};
                  \node (D) at (9,0) {};
                  \path (C) to [ornament=85] (D);
                \end{tikzpicture}}}}}%
  \end{center}%
}

\renewcommand{\labelitemii}{$\circ$}

\newcommand{\refbook}[1]{\small ref: #1}

% === toc ===

\usepackage{tocloft}
\renewcommand{\cftsecfont}{\rmfamily}
\renewcommand{\cftsecpagefont}{\rmfamily}
\setcounter{secnumdepth}{0}

\addtocontents{toc}{\protect\thispagestyle{empty}}
\pagestyle{empty}

% === hyperlink ===

\definecolor{oxfordblue}{rgb}{0.0, 0.13, 0.28}

% 「%」は以降の内容を「改行コードも含めて」無視するコマンド
\usepackage[%
  dvipdfmx,% 欧文ではコメントアウトする
  pdfencoding=auto, psdextra,% 数学記号を含める
  setpagesize=false,%
  bookmarks=true,%
  bookmarksdepth=tocdepth,%
  bookmarksnumbered=true,%
  colorlinks=true,%
  allcolors=oxfordblue,%
  linkcolor=MidnightBlue,%
  pdftitle={},%
  pdfsubject={},%
  pdfauthor={},%
  pdfkeywords={}%
]{hyperref}
% PDFのしおり機能の日本語文字化けを防ぐ((u)pLaTeXのときのみかく)
\usepackage{pxjahyper}
% ref: https://tex.stackexchange.com/questions/251491/math-symbol-in-section-heading
\pdfstringdefDisableCommands{\def\varepsilon{\textepsilon}}


% ---

\begin{document}

\maketitle
\tableofcontents

\sectionline
\section{記号化}

文を記号化することにより、文の長さや内容に煩わされることなく、文の構造を把握することが容易となり、「思考の節約」になる

\br

もともとの文は忘れて、記号で表された文の間の関係を調べる分野のことを\keyword{記号論理学}という

\br

記号論理学は、
\begin{itemize}
  \item 主張(命題)を扱う\keyword{命題論理学}
  \item 「すべての〜」とか「ある〜」とかを含む文を扱う\keyword{述語論理学}
\end{itemize}
に分かれている

\sectionline
\section{命題論理の法則}

\begin{theorem}{結合法則}
  \begin{align*}
    (p \land q) \land r & \equiv p \land (q \land r) \\
    (p \lor q) \lor r   & \equiv p \lor (q \lor r)
  \end{align*}
\end{theorem}

\keyword{結合法則}は、「どこから計算しても同じ」という性質を支えるもの

\sectionline

記号論理では、ある法則が成り立つとき、
\begin{shaded}
  その法則の$\land$を$\lor$に、そして、$\lor$を$\land$に置き換えた法則が成り立つ
\end{shaded}
という原理があり、\keyword{双対性}と呼ばれている

\br

\begin{shaded}
  \keyword{双対性}は、2つのことがら・概念が、ちょうどお互いに鏡で写し合っているような対称性を持つ状況
\end{shaded}

\keyword{双対性}は数学のいろんな分野で登場する

\sectionline

\begin{theorem}{冪等法則}
  \begin{align*}
    p \land p & \equiv p \\
    p \lor p  & \equiv p
  \end{align*}
\end{theorem}

これらを繰り返して適用すると、
\begin{align*}
  p \land \cdots \land p & \equiv p \\
  p \lor \cdots \lor p   & \equiv p
\end{align*}
であることが容易にわかる

これは、AND(あるいはOR)を「何度繰り返しても同値」であることを示している

\br

$\land$をかけ算(積)と見なすと、$  p \land \cdots \land$は$p$の累乗である

昔は、累乗のことを「冪」と呼んだので、「冪等法則」の名称もここから来ている

\sectionline

\begin{theorem}{交換法則}
  \begin{align*}
    p \land q & \equiv q \land p \\
    p \lor q  & \equiv q \lor p
  \end{align*}
\end{theorem}

$p$と$q$の順序が交換できることを示している

\sectionline

\begin{theorem}{分配法則}
  \begin{align*}
    p \land (q \lor r) & \equiv (p \land q) \lor (p \land r) \\
    p \lor (q \land r) & \equiv (p \lor q) \land (p \lor r)
  \end{align*}
\end{theorem}

交換法則を考慮すると、分配法則は右から分配することもできる

\begin{align*}
  p \land (q \lor r) & \equiv (q \lor r) \land p \\
  p \lor (q \land r) & \equiv (q \land r) \lor p
\end{align*}

\sectionline

\begin{theorem}{吸収法則}
  \begin{align*}
    p \land (p \lor q) & \equiv p \\
    p \lor (p \land q) & \equiv p
  \end{align*}
\end{theorem}

分配法則によく似ているが、分配する方と分配される方のどちらにも$p$が入っている

このような状況では$q$の影響がなくなって、命題が$p$と同値になるというのが\keyword{吸収法則}

\sectionline

\begin{theorem}{ド・モルガンの法則(命題論理)}
  \begin{align*}
    \neg (p \land q) & \equiv \neg p \lor \neg q  \\
    \neg (p \lor q)  & \equiv \neg p \land \neg q
  \end{align*}
\end{theorem}

\keyword{ド・モルガンの法則}は、ANDおよびORの否定がどうなるかを述べたもの

命題の否定を作るときにはなくてはならない重要な公式

\sectionline

これらの法則を前提にすると、真理表を使用せずに、\keyword{同値変形}という方法で、2つの命題が同値であることを確かめることができる

\sectionline
\section{恒真命題と恒偽命題}

同値変形をしていく場合に、真理値が一定な値をとる命題を考えると、便利であることがわかってくる

\begin{definition}{恒真命題}
  真理値を1しかとらない命題を\keyword{恒真命題}と呼び、$I$で表す
\end{definition}

\begin{definition}{恒偽命題}
  真理値を0しかとらない命題を\keyword{恒偽命題}と呼び、$O$で表す
\end{definition}

\sectionline

恒真命題と恒偽命題の定義から、明らかに次が成り立つ

\begin{theorem}{恒真命題と恒偽命題の関係}
  \begin{align*}
    \neg I & \equiv O \\
    \neg O & \equiv I
  \end{align*}
\end{theorem}

なぜなら、否定をとるというのは、真理値について0を1にし、1を0にする操作だから

\sectionline

\begin{theorem}{恒真命題の性質}
  \begin{align*}
    p \land I & \equiv p \\
    p \lor I  & \equiv I
  \end{align*}
\end{theorem}

\begin{theorem}{恒偽命題の性質}
  \begin{align*}
    p \land O & \equiv O \\
    p \lor O  & \equiv p
  \end{align*}
\end{theorem}

これらの性質において、
\begin{itemize}
  \item $\land$を$\lor$に
  \item $\lor$を$\land$に
  \item $I$を$O$に
  \item $O$を$I$に
\end{itemize}
置き換えると、
\begin{align*}
  p \land I \equiv p & \quad\leftrightarrow\quad p \lor O \equiv p  \\
  p \lor I \equiv I  & \quad\leftrightarrow\quad p \land O \equiv O
\end{align*}
という対応が得られ、恒真命題と恒偽命題が\keyword{双対的}であることがわかる

\sectionline
\section{矛盾法則と排中法則}

「命題とその否定命題は同時に成り立たない」というのが\keyword{矛盾法則}

\begin{theorem}{矛盾法則}
  \begin{align*}
    p \land \neg p & \equiv O
  \end{align*}
\end{theorem}

矛盾法則とは双対的に、\keyword{排中法則}は、「命題とその否定命題のどちらかは常に成り立つ」ということを表している

\begin{theorem}{排中法則}
  \begin{align*}
    p \lor \neg p & \equiv I
  \end{align*}
\end{theorem}

\sectionline

否定を含む論理式の同値変形において、矛盾法則、排中法則、恒真命題の性質、恒偽命題の性質を用いると、次のような2つのステップで、式をより単純な形にすることができる

\begin{enumerate}
  \item 矛盾法則や排中法則により、命題とその否定命題のペアは、恒真命題$I$や恒偽命題$O$に置き換えることができる
  \item 恒真命題の性質や恒偽命題の性質により、恒真命題$I$と恒偽命題$O$は、式をより簡単にする
\end{enumerate}

\sectionline
\section{ならば}

\begin{definition}{ならば}
  命題$p, \, q$に対して、$\neg p \lor q$という命題を$p \to q$と書いて、「$p$ならば$q$」と読む
\end{definition}

\sectionline
\section{必要条件と十分条件}

\begin{definition}{必要条件と十分条件}
  命題$p, \, q$に対して、命題$p \to q$が常に正しいとき、$p \Rightarrow q$と書き、
  \begin{itemize}
    \item $p$は$q$の\keyword{必要条件}である
    \item $q$は$p$の\keyword{十分条件}である
  \end{itemize}
  と呼ぶ
\end{definition}

\begin{definition}{必要十分条件}
  $p \Rightarrow q$であり、$q \Rightarrow p$であるとき、$p \Leftrightarrow q$と書き、
  \begin{itemize}
    \item $p$は$q$の\keyword{必要十分条件}である
    \item $q$は$p$の\keyword{必要十分条件}である
  \end{itemize}
  と呼ぶ
\end{definition}

\sectionline
\section{三段論法}

「ならば」を用いた有名な議論の方法として、\keyword{仮言三段論法}がある

これは、「$A$ならば$B$」という主張と「$B$ならば$C$」という主張から、「$A$ならば$C$」という主張を導くことができるというもの

\sectionline
\section{逆と対偶}

対偶$\neg q \to \neg p$と、もとの命題$p \to q$は同値である

\begin{equation}
  \begin{WithArrows}
    & \phantom{\equiv} \neg q \to \neg p \Arrow{$\to$の定義} \\
    & \equiv (\neg \neg q) \lor \neg p \Arrow{反射法則} \\
    & \equiv q \lor \neg p \Arrow{交換法則} \\
    & \equiv \neg p \lor q \Arrow{$\to$の定義} \\
    & \equiv p \to q
  \end{WithArrows}
\end{equation}

\sectionline

「晴れるならば、外出する」はまともな主張だが、その対偶「外出しないならば、晴れない」というのは、少し違和感を感じる

\br

これは、「外出しない」という原因によって「晴れない」という結果が導かれるととらえてしまうから

\br

あくまで、論理の「ならば」は、「外出しない」という事実があるときに、「晴れない」という事実があるという状態を表すもの

\br

「〜ならば〜」というのは、
\begin{shaded}
  原因と結果という因果関係ではなく、2つの状態の間の事実関係である
\end{shaded}
と思っておくとよい

\sectionline

$\neg p \to \neg q$は、$p \to q$の\keyword{裏}と呼ばれることもある

\begin{itemize}
  \item $(\neg p \to \neg q) \equiv (\neg\neg p) \lor \neg q \equiv p \lor \neg q$
  \item $(p \to q) \equiv (\neg p \lor q)$
\end{itemize}
であるため、裏$\neg p \to \neg q$と元の命題$p \to q$は特に関係がない

\sectionline
\section{2つの同値}

\begin{definition}{同値}
  2つの命題$p, \, q$に対して、真理値がすべて等しい(真理表が一致する)ということを、$p$と$q$は\keyword{同値}であると呼び、
  \begin{equation*}
    p \equiv q
  \end{equation*}
  と表す
\end{definition}

一方、同値にはもう1つの定義がある

\begin{definition}{同値}
  命題$p$と命題$q$がお互いに必要十分条件であるとき、言いかえると、$p \Rightarrow q$かつ$q \Rightarrow p$であるとき、$p$と$q$は\keyword{同値}であると呼び、
  \begin{equation*}
    p \Leftrightarrow q
  \end{equation*}
  と表す
\end{definition}

この2つの同値$\equiv$と$\Leftrightarrow$は、実は同じ内容を表している

\br

「$p \Rightarrow q$かつ$q \Rightarrow p$」であるというのは、

\begin{shaded}
  命題$p \to q$および命題$q \to p$の真理値がすべて1である
\end{shaded}

ということだから、「$p$と$q$の真理値が等しいこと」と「$p \to q$と$q \to p$の真理値がどちらも1であること」は一致している

\br

したがって、2つの同値$\equiv$と$\Leftrightarrow$は同じ内容を表していることがわかる


\end{document}
