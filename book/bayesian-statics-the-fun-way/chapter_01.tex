\documentclass[../../book_bayesian-statics-the-fun-way]{subfiles}

\begin{document}

\chapter{ベイズ的思考と日常の推論}

\section{ベイズ的推論の思考プロセス}

\begin{enumerate}
  \item データを観察する
  \item 仮説を立てる
  \item データに基づいて自分の考えを更新する
\end{enumerate}

\section{データを観察する}

確率論では、複数の出来事(\keyword{事象})が組み合わされる確率に注目する場合、それらの出来事をカンマで区切る
\begin{equation*}
  P(\text{\bfseries 光}, \text{\bfseries 空中の円盤}) = \text{\bfseries 極めて低い}
\end{equation*}
この数式は「光と空中の円盤が観測される確率は極めて低い」と読める

\section{事前の信念と条件付き確率}

\keyword{事前の信念}とは、
\begin{emphabox}
  生まれてからの経験(観測したデータ)によって積み上げられた様々な考えの集まり
\end{emphabox}

光が見えるのと同時に空中に円盤が見えるという出来事は、地球上では稀にしか起こらない
\begin{multline*}
  P(\text{\bfseries 光}, \text{\bfseries 空中の円盤} \mid \text{\bfseries 地球上での経験})\\ = \text{\bfseries 極めて低い}
\end{multline*}
この数式は「地球上での経験に基づくと、光と空中の円盤が同時に観測される確率は極めて低い」と読める

\br

このような確率を、
\begin{emphabox}
  ある条件のもとで、ある出来事が起こる確率
\end{emphabox}
という意味で、\keyword{条件付き確率}と呼ぶ

\sectionline

確率を$P$で表すのと同様に、出来事や条件にも短い変数名を使う

データを$D$とすると、
\begin{equation*}
  D = \text{\bfseries 光}, \text{\bfseries 空中の円盤}
\end{equation*}

事前の信念を$X$とすると、
\begin{equation*}
  X = \text{\bfseries 地球上での経験}
\end{equation*}

これで先ほどの数式は、次のように書ける
\begin{equation*}
  P(D \mid X) = \text{\bfseries 極めて低い}
\end{equation*}

しかし実際には、事前の信念は暗黙に仮定する

\section{仮説を立てる}

目撃したものを説明するには、何らかの\keywordJE{仮説}{hypothesis}を立てる必要がある

\br

\keyword{仮説}とは、
\begin{emphabox}
  世界を何らかの形で理解する方法
\end{emphabox}
であり、その理解に基づいて世界の振る舞いに関する何らかの\keyword{予測}が導かれるもの

\br

たとえば、世界に関する基本的な考え方はすべて仮説である
\begin{itemize}
  \item 地球が自転していると考えるなら、太陽が繰り返し昇ったり沈んだりすると予測される
\end{itemize}

\sectionline

先ほどの証拠(光と空中の円盤)を見て、次のような仮説を立てる
\begin{equation*}
  H_1 = \text{\bfseries UFOがいる}
\end{equation*}

今の状況を逆向きに考えると、「UFOがいたら、何が見えると予想されるか?」と問いかけることができる

そしてその答えは、「光と空中の円盤」となる

つまり、
\begin{emphabox}
  \begin{center}
    仮説はデータを説明する
  \end{center}
\end{emphabox}

\br

$H_1$からはデータ$D$が予測されるため、この仮説のもとでこのデータを観察する場合、このデータが観測される確率は高くなる
\begin{equation*}
  P(D \mid H_1, X) > P(D \mid X)
\end{equation*}
この数式は「光と空中の円盤を目撃する確率は、それがUFOであるという私の考えを前提とした場合、はるかに高くなる」と読める

\section{日々の発言に仮説を見出す}

日常の言い回しと確率の間には関係性がある

\begin{itemize}
  \item 「驚きだ」:事前の経験に基づくと、そのデータが観測される確率は低い
  \item 「筋が通っている」:事前の経験に基づくと、そのデータが観測される確率は高い
\end{itemize}

\section{さらなる証拠を集めて考えを更新する}

もっと信頼できる結論を引き出すには、さらにデータを集める必要がある

\br

さらに観察を続けると、円盤がワイヤーで吊り下げられているのが見え、誰かが「カット!」と叫んだ

この新たなデータにより、もう一つの仮説が立てられる
\begin{equation*}
  H_2 = \text{\bfseries 映画を撮影している}
\end{equation*}
もう一つの考えられる説明が得られたので、\keyword{対立仮説}を立てたことになる

\end{document}
