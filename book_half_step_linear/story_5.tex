\documentclass[../book_half_step_linear]{subfiles}

\begin{document}

\section{行列の積を解釈する}

\subsection{まずは基本的な定義を形式的に把握}

ベクトルは数が一方向にならんだものだったが、\keyword{行列}は数が平面的にならんだもの

\br

行列に対してベクトルと同じようにスカラー倍や和の演算を定義して、行列同士を行き来できるようになる

さらに行列には掛け算、つまり積を定義できる

積は行列のサイズを変えうるため、スカラー倍や和とは異なるもの

\br

$L \times M$行列$A$と$M \times N$行列$B$の積を考える

これらの積により、$L \times N$行列$C$が作られる

作られる行列$C$の$l$行目かつ$n$列目の要素は、左側の行列$A$の$l$行目の「行」すべての要素と、右側の行列$B$の$n$列目の「列」すべての要素の掛け算、そしてその総和で作られる

\br

なお、左側の「列」のサイズと右側の「行」のサイズが一致していないと、積が定義されないこともわかる

今の例では左側の行列$A$のサイズが$L \times M$、右側の$B$が$M \times N$なので、サイズ$M$が一致している

\br

また、できあがる行列の行のサイズは左側の行のサイズと一致し、列のサイズは右側の列のサイズと一致する

今の例では$L \times N$になる

\sectionline
\subsection{途中の全経路を考えることで積を与える}

行列積$C=AB$に対して、左辺の行列$C$の$l$行$n$列成分を、「行列$A$の$l$(行目)を左端、行列$B$の$n$(列目)を右端とする経路を足し合わせたもの」と解釈する

実際に、1つ目の経路を$a_{l1}b_{1n}$、2つ目を$a_{l2}b_{2n}$などとしていくと、これらの足し算が$C_{ln}$を与えることがわかる

なお、どの経路を使っても$l$と$n$を結ぶことができるので、途中の経由点$m$としては$1$から$M$までを、つまりすべての経路を考える

$c_{ln}$を$l$と$n$を結ぶ経路と考えることで、記号的な行列積の定義を「途中の経路の総和」のようなイメージで捉えられる

\br

なお、確率的な現象を扱う場合には、行列の要素に「確率」の意味合いが出てくるため、この経路にさらに深い意味合いをもたせることができる

\sectionline
\subsection{観測装置を経由して捉える}

ブラケット記号を使って行列積を捉えてみる

ここでは例として$2\times 2$行列$A$と$2\times 1$行列$\ket{\vb*{x}}$の行列積を考える

$\ket{\vb*{x}}$は列ベクトルだが、$2\times 1$行列としても捉えられる

なお、行列積の結果としては$2\times 1$行列、つまり列ベクトルが出てくるはず

\br

まず、$A$を「2つの行ベクトル$\bra{\vb*{a}_1}$と$\bra{\vb*{a}_2}$がならんだもの」と解釈する
\begin{equation*}
  A = \begin{bmatrix}
    a_{11} & a_{12} \\
    a_{21} & a_{22}
  \end{bmatrix} = \mqty[\bra{\vb*{a}_1} \\ \bra{\vb*{a}_2}]
\end{equation*}
すると、「(要素が行ベクトルである)列ベクトル」を考えられる

\br

ここで一次結合のときの話を思い出すと、列ベクトルはある基底を用いた場合の係数をならべたものだった

つまり、その基底に関する座標である

今は素朴に\keyword{標準基底}と呼ばれるもの、すなわち、
\begin{equation*}
  \Ket{\vb*{e}_1} = \mqty[1 \\ 0], \quad \Ket{\vb*{e}_2} = \mqty[0 \\ 1]
\end{equation*}
を基底とする

標準基底$\ket{\vb*{e}_i}$は$i$番目の要素だけ$1$、それ以外は$0$の列ベクトル

これで一次結合の形が出る
\begin{equation*}
  A = \mqty[\bra{\vb*{a}_1} \\ \bra{\vb*{a}_2}] = \mqty[\Ket{\vb*{a}_1} \\ 0] + \mqty[0 \\ \Ket{\vb*{a}_2}] = \Ket{\vb*{e}_1} \bra{\vb*{a}_1} + \Ket{\vb*{e}_2} \bra{\vb*{a}_2}
\end{equation*}
内積として解釈されるのを避けるため、$\bra{\vb*{a}_i}$を$\ket{\vb*{e}_i}$の右側に書いた

\br

この形で書いた行列$A$と$\ket{\vb*{x}}$の積を計算すると、
\begin{align*}
  A\ket{\vb*{x}} & = \biggl( \Ket{\vb*{e}_1} \bra{\vb*{a}_1} + \Ket{\vb*{e}_2} \bra{\vb*{a}_2} \biggr) \ket{\vb*{x}}                             \\
                 & = \biggl( \Braket{\vb*{a}_1 | \vb*{x}} \biggr) \Ket{\vb*{e}_1} + \biggl( \Braket{\vb*{a}_2 | \vb*{x}} \biggr) \Ket{\vb*{e}_2} \\
                 & = \mqty[\Braket{\vb*{a}_1 | \vb*{x}}                                                                                          \\  \Braket{\vb*{a}_2 | \vb*{x}}] \\
                 & = \begin{bmatrix}
                       a_{11}x_1 + a_{12}x_2 \\
                       a_{21}x_1 + a_{22}x_2
                     \end{bmatrix}
\end{align*}
途中で$\braket{\vb*{a}_1|\vb*{x}}$がスカラー、つまり単なる値なので$\ket{\vb*{e}_1}$の前に出せることなども使った

\br

ここで、内積$\braket{\vb*{a}_i|\vb*{x}}$に注目しよう

内積の解釈の一つに、素朴に転置した行ベクトルと列ベクトルの掛け算というものがあった

ただ、状態$\ket{\vb*{x}}$を観測装置$\bra{\vb*{a}_1}$で観測する、という解釈もできるのだった

つまり、1つの状態$\ket{\vb*{x}}$を、2つの観測装置$\bra{\vb*{a}_1}$と$\bra{\vb*{a}_2}$で観測して、それぞれの結果を使ってまた一次結合をとったベクトルに写す、と解釈できる

\sectionline

ベクトルは、注目している対象の情報を含んでいる

そして、行列は「ベクトルをほかのベクトルに変換すること」、つまり「情報を処理すること」に対応する

\sectionline
\subsection{行列に割り算はない}

通常の実数の計算には「割り算」がある

何かを掛け算したものの、元に戻したいときには割り算が使える

\br

しかし、行列には割り算が定義されていない

\keyword{逆行列}はあるが、これは割り算ではなく、掛け算をすると単位行列を与える特別な関係にある行列である

\sectionline
\subsection{逆行列と、逆行列が存在する条件}

$D\times D$行列$A$に対して、
\begin{equation*}
  AX = XA = I
\end{equation*}
を満たす行列が「存在する場合」を考える

\br

ここで$I$は単位行列で、対角成分はすべて$1$、それ以外は$0$である

なお、単位行列の各要素を、
\begin{equation*}
  I_{ij} = \delta_{ij} = \begin{cases}
    1 & (i=j)              \\
    0 & (\text{otherwise})
  \end{cases}
\end{equation*}
と書くこともある

ここで使われている$\delta_{ij}$は\keyword{クロネッカーのデルタ}の記号

\br

式$AX = XA = I$が満たされる場合に、$X$を$A$の\keyword{逆行列}と呼び、$A^{-1}$と書く

なお、$A^{-1}$の逆行列は$\left( A^{-1} \right)^{-1} = A$である

\br

また、$AX$と$XA$の両方の掛け算が成立するためには、$A$の行と列のサイズが同じ、つまり\keyword{正方行列}である必要がある

\br

ここで、正方行列だからといって逆行列が必ずしも存在するわけではない、ということに注意

行列が逆行列をもつとき、\keyword{正則}であると言う

\br

行列が正則であるための条件として、たとえば行列の各列(各行でも可)をベクトルとみなしたとき、それらが一次独立である、というものがある

この条件を言い換えた別の条件もいくつかある

\sectionline
\subsection{行列の転置}

行列$A=[a_{ij}]$に対して、その\keyword{転置行列}は$A^\top=[a_{ji}]$である

成分の順番が変わっていて、列と行をばたんと入れ替えるイメージ

\br

$N\times 1$行列、つまり列ベクトルの転置は、$1\times N$行列、つまり行ベクトルになる

\end{document}
