\documentclass[../book_half_step_linear]{subfiles}

\begin{document}

\section{内積で近さを測る}

\subsection{ベクトル同士の関係性を知る方法}

集合だけだと身動きできないが、演算によって互いを行き来できるようになった

ただし、2つのベクトルを取り出したときに、それらが似ているかどうかを議論するためには道具が少し必要となる

それがベクトルの\keyword{内積}である

\br

矢印で考えた場合、内積は次のように定義された
\begin{enumerate}
  \item $\overrightarrow{x}$と$\overrightarrow{y}$のなす角を$\theta$とする
  \item ベクトル$\overrightarrow{x}$の大きさを$|\overrightarrow{x}|$、ベクトル$\overrightarrow{y}$の大きさを$|\overrightarrow{y}|$とする
  \item $\overrightarrow{x}$と$\overrightarrow{y}$の内積を$\overrightarrow{x} \cdot \overrightarrow{y} = |\overrightarrow{x}||\overrightarrow{y}|\cos \theta$とする
\end{enumerate}
矢印で記述できる場合にはこれでも大丈夫だが、想像できないような4次元以上の高次元では、角度$\theta$から出発するわけにはいかない

そのため、順番を逆にして定義していく

\sectionline
\subsection{内積はスカラー値を与える関数}



\end{document}
