\documentclass[../book_half_step_linear]{subfiles}

\begin{document}

\section{はじめに}

\subsection{数式を眺める視点を、いろいろと}

\keyword{行列}にはベクトルをうまく操作するための装置としての役割もある

ベクトルを別のベクトルに変換するものとしての行列、という見方もできる

その先に、関数を別の関数に変換するものを考え、これが行列とつながり、さらに時間発展する系の記述ともつながる…と話は続く

\sectionline

\subsection{半歩先から見える景色を}

線形代数は便利な道具でもあり、世界を捉えるための思考方法でもある

入力に対して出力を対応させるという少し抽象的いな「コト」を、数値がならんだベクトルや行列という具体的な「モノ」で表現する、それを可能にするのが線形代数

関数という「曲がってうねる形」を、具体的な数値のならびに書き下せること、さらには、一つの対象をさまざまに表現できること、線形代数が教えてくれるこれらは、現実世界の問題をどのように数学の言葉で記述して、どのように計算機で処理していくのかを考えるうえで、とても役立つ

\end{document}
