\documentclass[../book_half_step_linear]{subfiles}

\begin{document}

\section{どちらの基底が好み?}

\subsection{多くの場合によい形がある}

これまで見てきたように、「基底は一つではない」、さらには「内積やノルム、距離も一つではない」と、何でもありの感じだった

そのなかで便利なものを選んで議論することが大切

\sectionline
\subsection{描けない角度を内積を使って定義}

先ほどは互いの距離を内積から定義したが、今度は互いが作る角度を考える

\br

矢印ならば簡単にイメージできる

そもそも2つ矢印のなす角度を用いて内積を定義することもできた

\br

しかし、4次元以上になると矢印による解釈を使えないので、順番を変えて内積から…という話だった

そのため、素朴には内積およびベクトルのノルムを使って、逆に角度を定義してあげる

つまり、
\begin{equation*}
  \cos \theta = \frac{\Braket{\vb*{a}_1|\vb*{a}_2}}{\norm{\vb*{a}_1}\norm{\vb*{a}_2}}
\end{equation*}
で角度$\theta$を定義する

\sectionline
\subsection{高次元空間でも$90^\circ$は直交を意味する}

矢印を思い描けない高次元空間でも、角度が$90^\circ$などだと2つのベクトルは\keyword{直交}すると言う

「など」と書いたのは$-90^\circ$でも直交であり、ほかにもたくさん直交する角度があるから

そのため、
\begin{equation*}
  \cos \theta = \frac{\Braket{\vb*{a}_1|\vb*{a}_2}}{\norm{\vb*{a}_1}\norm{\vb*{a}_2}}
\end{equation*}
で定義される左辺がゼロになるときに直交と考えるのがよさそう

\br

つまり、2つのベクトル$\bra{\vb*{a}_i}$と$\bra{\vb*{a}_j}$に対して、
\begin{equation*}
  \Braket{\vb*{a}_i|\vb*{a}_j} = 0 \quad \text{もしくは}\quad \Braket{\vb*{a}_j|\vb*{a}_i} =0
\end{equation*}
のとき、これらのベクトルは直交の関係にある

ここで、内積の性質から$i$と$j$の左右を入れ替えても値は変わらないことを利用した

\sectionline
\subsection{直交していると計算がすごく簡単になる}

基底とは、考えたい空間を生成する「必要十分なもの」、つまり一次独立なベクトルの組だった

直交している必要はないが、直交していれば互いに一次結合の形で表現できないので、確実に基底になっている

\br

なぜ直感的に直交する基底のほうが「よい」と感じるのか、その理由を考えてみる

今は素朴にブラ$\bra{\cdot}$を行ベクトル、ケット$\ket{\cdot}$を列ベクトルと考える

\sectionline

基底として「直交してはいないが、一次独立」の$\left\{ \ket{\vb*{a}_1}, \ket{\vb*{a}_2} \right\}$を用いるとする

そして、この基底を用いてベクトル$\ket{\vb*{x}}$を、
\begin{equation*}
  \ket{\vb*{x}} = c_1 \ket{\vb*{a}_1} + c_2 \ket{\vb*{a}_2}
\end{equation*}
と表現しておく

「基底はいろいろあるが、基底を決めれば表現は一つ」なので、$c_1$と$c_2$は一意に定まる

これらを求めるために、左から$\bra{\vb*{a}_1}$と$\bra{\vb*{a}_2}$をかけ算しよう
\begin{align*}
  \Braket{\vb*{a}_1|\vb*{x}} & = c_1 \Braket{\vb*{a}_1|\vb*{a}_1} + c_2 \Braket{\vb*{a}_1|\vb*{a}_2} \\
  \Braket{\vb*{a}_2|\vb*{x}} & = c_1 \Braket{\vb*{a}_2|\vb*{a}_1} + c_2 \Braket{\vb*{a}_2|\vb*{a}_2}
\end{align*}
ブラケット記号が閉じている$\Braket{\vb*{a}_1|\vb*{a}_2}$や$\Braket{\vb*{a}_1|\vb*{x}}$などの形の部分はスカラーであり、具体的なベクトルが与えられれば簡単に計算できる(内積を計算するだけ)

すると、式が2つ、未知変数も2つなので、連立方程式を解けば$c_1$と$c_2$が求まる

\br

今は基底が2つだけなので、基底から作られる線形空間の次元は2、つまり平面である

もし基底の数が$D$個なら、$D$次元空間が作られ、式の数も$D$個、未知変数も$D$個になる

よって、$D$元連立一次方程式を解くことになる

\sectionline

次に基底として「直交している」$\left\{ \ket{\vb*{u}_1}, \ket{\vb*{u}_2} \right\}$を用いる

$\ket{\vb*{x}}$を基底で表現する一次結合を考えてみると、
\begin{equation*}
  \ket{\vb*{x}} = d_1 \ket{\vb*{u}_1} + d_2 \ket{\vb*{u}_2}
\end{equation*}
なお、先ほどと同じ$c_1$と$c_2$を使っているが、基底が違うので、別のものだと捉えるように注意

\br

左から$\bra{\vb*{u}_1}$と$\bra{\vb*{u}_2}$をかけ算すると、
\begin{align*}
  \Braket{\vb*{u}_1|\vb*{x}} & = c_1 \Braket{\vb*{u}_1|\vb*{u}_1} + c_2 \Braket{\vb*{u}_1|\vb*{u}_2} \\
  \Braket{\vb*{u}_2|\vb*{x}} & = c_1 \Braket{\vb*{u}_2|\vb*{u}_1} + c_2 \Braket{\vb*{u}_2|\vb*{u}_2}
\end{align*}
直交しているので$\Braket{\vb*{u}_1|\vb*{u}_2} = 0$、$\Braket{\vb*{u}_2|\vb*{u}_1} = 0$などが成立して、
\begin{align*}
  c_1 & = \frac{\Braket{\vb*{u}_1|\vb*{x}}}{\Braket{\vb*{u}_1|\vb*{u}_1}} \\
  c_2 & = \frac{\Braket{\vb*{u}_2|\vb*{x}}}{\Braket{\vb*{u}_2|\vb*{u}_2}}
\end{align*}
という式が得られる

内積を簡単に計算できるのは先ほどと同様だが、今回は連立方程式を解く必要がない

この意味で、直交する基底、すなわち\keyword{直交基底}は便利で「よい」と言える

\end{document}
