\documentclass[../book_half_step_linear]{subfiles}

\begin{document}

\section{どちらの基底が好み?}

\subsection{多くの場合によい形がある}

これまで見てきたように、「基底は一つではない」、さらには「内積やノルム、距離も一つではない」と、何でもありの感じだった

そのなかで便利なものを選んで議論することが大切

\sectionline
\subsection{描けない角度を内積を使って定義}

先ほどは互いの距離を内積から定義したが、今度は互いが作る角度を考える

\br

矢印ならば簡単にイメージできる

そもそも2つ矢印のなす角度を用いて内積を定義することもできた

\br

しかし、4次元以上になると矢印による解釈を使えないので、順番を変えて内積から…という話だった

そのため、素朴には内積およびベクトルのノルムを使って、逆に角度を定義してあげる

つまり、
\begin{equation*}
  \cos \theta = \frac{\Braket{\vb*{a}_1|\vb*{a}_2}}{\norm{\vb*{a}_1}\norm{\vb*{a}_2}}
\end{equation*}
で角度$\theta$を定義する

\sectionline
\subsection{高次元空間でも$90^\circ$は直交を意味する}

矢印を思い描けない高次元空間でも、角度が$90^\circ$などだと2つのベクトルは\keyword{直交}すると言う

「など」と書いたのは$-90^\circ$でも直交であり、ほかにもたくさん直交する角度があるから

そのため、
\begin{equation*}
  \cos \theta = \frac{\Braket{\vb*{a}_1|\vb*{a}_2}}{\norm{\vb*{a}_1}\norm{\vb*{a}_2}}
\end{equation*}
で定義される左辺がゼロになるときに直交と考えるのがよさそう

\br

つまり、2つのベクトル$\bra{\vb*{a}_i}$と$\bra{\vb*{a}_j}$に対して、
\begin{equation*}
  \Braket{\vb*{a}_i|\vb*{a}_j} = 0 \quad \text{もしくは}\quad \Braket{\vb*{a}_j|\vb*{a}_i} =0
\end{equation*}
のとき、これらのベクトルは直交の関係にある

ここで、内積の性質から$i$と$j$の左右を入れ替えても値は変わらないことを利用した

\sectionline
\subsection{直交していると計算がすごく簡単になる}

基底とは、考えたい空間を生成する「必要十分なもの」、つまり一次独立なベクトルの組だった

直交している必要はないが、直交していれば互いに一次結合の形で表現できないので、確実に基底になっている

\br

なぜ直感的に直交する基底のほうが「よい」と感じるのか、その理由を考えてみる

今は素朴にブラ$\bra{\cdot}$を行ベクトル、ケット$\ket{\cdot}$を列ベクトルと考える

\sectionline

基底として「直交してはいないが、一次独立」の$\left\{ \ket{\vb*{a}_1}, \ket{\vb*{a}_2} \right\}$を用いるとする

そして、この基底を用いてベクトル$\ket{\vb*{x}}$を、
\begin{equation*}
  \ket{\vb*{x}} = c_1 \ket{\vb*{a}_1} + c_2 \ket{\vb*{a}_2}
\end{equation*}
と表現しておく

「基底はいろいろあるが、基底を決めれば表現は一つ」なので、$c_1$と$c_2$は一意に定まる

これらを求めるために、左から$\bra{\vb*{a}_1}$と$\bra{\vb*{a}_2}$をかけ算しよう
\begin{align*}
  \Braket{\vb*{a}_1|\vb*{x}} & = c_1 \Braket{\vb*{a}_1|\vb*{a}_1} + c_2 \Braket{\vb*{a}_1|\vb*{a}_2} \\
  \Braket{\vb*{a}_2|\vb*{x}} & = c_1 \Braket{\vb*{a}_2|\vb*{a}_1} + c_2 \Braket{\vb*{a}_2|\vb*{a}_2}
\end{align*}
ブラケット記号が閉じている$\Braket{\vb*{a}_1|\vb*{a}_2}$や$\Braket{\vb*{a}_1|\vb*{x}}$などの形の部分はスカラーであり、具体的なベクトルが与えられれば簡単に計算できる(内積を計算するだけ)

すると、式が2つ、未知変数も2つなので、連立方程式を解けば$c_1$と$c_2$が求まる

\br

今は基底が2つだけなので、基底から作られる線形空間の次元は2、つまり平面である

もし基底の数が$D$個なら、$D$次元空間が作られ、式の数も$D$個、未知変数も$D$個になる

よって、$D$元連立一次方程式を解くことになる

\sectionline

次に基底として「直交している」$\left\{ \ket{\vb*{u}_1}, \ket{\vb*{u}_2} \right\}$を用いる

$\ket{\vb*{x}}$を基底で表現する一次結合を考えてみると、
\begin{equation*}
  \ket{\vb*{x}} = d_1 \ket{\vb*{u}_1} + d_2 \ket{\vb*{u}_2}
\end{equation*}
なお、先ほどと同じ$c_1$と$c_2$を使っているが、基底が違うので、別のものだと捉えるように注意

\br

左から$\bra{\vb*{u}_1}$と$\bra{\vb*{u}_2}$をかけ算すると、
\begin{align*}
  \Braket{\vb*{u}_1|\vb*{x}} & = c_1 \Braket{\vb*{u}_1|\vb*{u}_1} + c_2 \Braket{\vb*{u}_1|\vb*{u}_2} \\
  \Braket{\vb*{u}_2|\vb*{x}} & = c_1 \Braket{\vb*{u}_2|\vb*{u}_1} + c_2 \Braket{\vb*{u}_2|\vb*{u}_2}
\end{align*}
直交しているので$\Braket{\vb*{u}_1|\vb*{u}_2} = 0$、$\Braket{\vb*{u}_2|\vb*{u}_1} = 0$などが成立して、
\begin{align*}
  c_1 & = \frac{\Braket{\vb*{u}_1|\vb*{x}}}{\Braket{\vb*{u}_1|\vb*{u}_1}} \\
  c_2 & = \frac{\Braket{\vb*{u}_2|\vb*{x}}}{\Braket{\vb*{u}_2|\vb*{u}_2}}
\end{align*}
という式が得られる

内積を簡単に計算できるのは先ほどと同様だが、今回は連立方程式を解く必要がない

この意味で、直交する基底、すなわち\keyword{直交基底}は便利で「よい」と言える

\section{直交は作れる}

\subsection{直交するように係数を選ぶ}

\keyword{シュミットの直交化法}と呼ばれる方法により、直交基底を作ることもできる

\br

例として、次の3つのベクトルを生成元とする線形空間を考える
\begin{equation*}
  \ket{\vb*{a}_1} = \mqty[1 \\ 1 \\ 0], \quad \ket{\vb*{a}_2} = \mqty[1 \\ 0 \\ 1], \quad \ket{\vb*{a}_3} = \mqty[0 \\ 2 \\ -1]
\end{equation*}
まず1つ、ベクトル$\ket{\vb*{a}_1}$を選ぶ

次に$\ket{\vb*{a}_1}$に直交する基底を作りたいのだが、何でもよいわけではない

生成元から作られる空間を考えたいので、ここでは$\ket{\vb*{a}_2}$を材料に使う

$c \in \mathbb{R} $として、
\begin{equation*}
  \Ket{\tilde{\vb*{u}}_2} = \Ket{\vb*{a}_2} + c \Ket{\vb*{a}_1}
\end{equation*}
を作る

この$\Ket{\tilde{\vb*{u}}_2}$を$\ket{\vb*{a}_1}$と直交させたいので、左から$\bra{\vb*{a}_1}$を掛け算する
\begin{equation*}
  \Braket{\vb*{a}_1|\tilde{\vb*{u}}_2} = \Braket{\vb*{a}_1|\vb*{a}_2} + c \Braket{\vb*{a}_1|\vb*{a}_1}
\end{equation*}
この左辺をゼロにしたいわけなので、
\begin{equation*}
  c = -\frac{\Braket{\vb*{a}_1|\vb*{a}_2}}{\Braket{\vb*{a}_1|\vb*{a}_1}}
\end{equation*}
と選べばよいことがわかる

\sectionline
\subsection{ノルムを揃えておくと便利}

分母$\Braket{\vb*{a}_1|\vb*{a}_1}$の部分はベクトル$\vb*{a}_1$の内積で、この平方根がノルム、つまりベクトルの大きさである

ここを初めから$1$にしておくと分母が消えてくれて、計算が簡単になりそう

もし最初から$\ket{\vb*{a}_1}$のノルムが$1$であれば、次のようになる
\begin{equation*}
  c = -\Braket{\vb*{a}_1|\vb*{a}_2}
\end{equation*}
ベクトルのノルムを$1$に揃えておくことを\keyword{正規化}と呼ぶ

\br

$\ket{\vb*{a}_1}$を正規化したベクトルを$\ket{\vb*{u}_1}$として、$\bra{\vb*{a}_1}$の代わりに$\bra{\vb*{u}_1}$を使って$c$を求めておく
\begin{equation*}
  c = -\Braket{\vb*{u}_1|\vb*{a}_2}
\end{equation*}
すると、$\Ket{\tilde{\vb*{u}}_2} = \Ket{\vb*{a}_2} + c \Ket{\vb*{a}_1}$の代わりに
\begin{equation*}
  \Ket{\tilde{\vb*{u}}_2} = \Ket{\vb*{a}_2} - \Bigl( \Braket{\vb*{u}_1|\vb*{a}_2} \Bigr) \Ket{\vb*{u}_1}
\end{equation*}
を考えればよいことになる

この式だけで、考えたい空間上において$\vb*{u}_1$に直交するベクトルを作ることができた

\br

さらにこの作業を続けるときに、求まった$\Ket{\tilde{\vb*{u}}_2}$をまた正規化して、$\ket{\vb*{u}_2}$を作っておく

次に作るベクトル$\Ket{\tilde{\vb*{u}}_3}$は、$\ket{\vb*{u}_1}$と$\ket{\vb*{u}_2}$の両方に直交する必要がある

考え方は上の計算と同じで、係数を追加して、その係数を求める方程式を立てるという流れ

\br

念のため次のステップまで進んでおくと、まずは、
\begin{equation*}
  \Ket{\tilde{\vb*{u}}_3} = \Ket{\vb*{a}_3} + c_1 \Ket{\vb*{u}_1} + c_2 \Ket{\vb*{u}_2}
\end{equation*}
とする

これに左から$\bra{\vb*{u}_1}$と$\bra{\vb*{u}_2}$をかけ算した場合のそれぞれにおいて、左辺がゼロになればよい

式が2つ出てきて、未知変数も$c_1$と$c_2$の2つあるので、解ける

ただ、実際にはそもそも$\ket{\vb*{u}_2}$は$\ket{\vb*{u}_1}$に直交しているので、もっと簡単に計算を進められる

\sectionline
\subsection{シュミットの直交化法のまとめ}

今、考えたい空間の生成元を$\ket{\vb*{a}_d}, \, d = 1, 2, \ldots, D$とする
\begin{enumerate}
  \item $\ket{\vb*{a}_1}$を正規化して最初の基底を作る:$\ket{\vb*{u}_1} = \dfrac{1}{\sqrt{\Braket{\vb*{a}_1|\vb*{a}_2}}}\Ket{\vb*{a}_1}$
  \item まずは$d=1$として、以下の手順3~5を繰り返す
  \item $d+1$番目の基底の候補を作る:$\Ket{\tilde{\vb*{u}}_{d+1}} = \Ket{\vb*{a}_{d+1}} - \displaystyle\sum_{d'=1}^{d} \Bigl( \Braket{\vb*{u}_{d'}|\vb*{a}_{d+1}}\Bigr) \Ket{\vb*{u}_{d'}}$
  \item 基底の候補$\Ket{\tilde{\vb*{u}}_{d+1}}$を正規化して$\ket{\vb*{u}_{d+1}}$を作り、これを$d+1$個目の基底とする
  \item $d$を1つ増やして、次の基底の計算へと進む
\end{enumerate}
このように、互いに直交しつつノルムが1となっている基底のことを\keyword{正規直交基底}と呼ぶ

この正規直交基底を作るための手順は、あとで関数を考えるときにも使う

\sectionline
\subsection{成分を抜いたら残らないこともある}

実は以下のベクトルは一次従属であるため、「基底」とは呼べない
\begin{equation*}
  \ket{\vb*{a}_1} = \mqty[1 \\ 1 \\ 0], \quad \ket{\vb*{a}_2} = \mqty[1 \\ 0 \\ 1], \quad \ket{\vb*{a}_3} = \mqty[0 \\ 2 \\ -1]
\end{equation*}
あとで見るように、この中から2つを選んで組を作ると基底となる

3次元空間中に2つのベクトルなので、作られる線形空間は平面

\br

与えられたベクトルの組、つまり生成元が、どのような空間を作るのか、基底なのか、それとも余分なものが含まれるのか…\keyword{行列}の概念はこの判断と密接に関係する

ただ、このあとの計算で見るように、シュミットの直交化法で実際に正規直交基底を構成することでも、余分なものを削ることができる

\sectionline
\subsection{具体的な計算で余分なものが消える}

次の例で計算を進めてみる
\begin{equation*}
  \ket{\vb*{a}_1} = \mqty[1 \\ 1 \\ 0], \quad \ket{\vb*{a}_2} = \mqty[1 \\ 0 \\ 1], \quad \ket{\vb*{a}_3} = \mqty[1 \\ 2 \\ -1]
\end{equation*}

$\braket{\vb*{a}_1|\vb*{a}_1} = 2$なので、正規化すると、
\begin{equation*}
  \ket{\vb*{u}_1} = \frac{1}{\sqrt{2}} \mqty[1 \\ 1 \\ 0]
\end{equation*}
が得られる

次のステップは、
\begin{align*}
  \Braket{\vb*{u}_1|\vb*{a}_2} & = \mqty[\frac{1}{\sqrt{2}} & \frac{1}{\sqrt{2}} & 0] \mqty[1 \\ 0 \\ 1] = \frac{1}{\sqrt{2}} \cdot 1 = \frac{1}{\sqrt{2}} \\
\end{align*}
より、
\begin{align*}
  \ket{\tilde{\vb*{u}}_2} & = \ket{\vb*{a}_2} - \Bigl(\Braket{\vb*{u}_1|\vb*{a}_2}\Bigr) \ket{\vb*{u}_1} \\
                          & = \mqty[1                                                                    \\ 0 \\ 1] - \frac{1}{\sqrt{2}} \cdot \dfrac{1}{\sqrt{2}} \mqty[1 \\ 1 \\ 0] \\
                          & = \mqty[1                                                                    \\ 0 \\ 1] - \frac{1}{2} \mqty[1 \\ 1 \\ 0] \\
                          & = \frac{1}{2} \mqty[1                                                        \\ -1 \\ 2]
\end{align*}
である

\br

次のステップのために正規化しておこう

$\braket{\tilde{\vb*{u}}_2|\tilde{\vb*{u}}_2} = \dfrac{3}{2}$なので、次のようになる
\begin{equation*}
  \ket{\vb*{u}_2} = \frac{1}{\sqrt{6}} \mqty[1 \\ -1 \\ 2]
\end{equation*}

\br

次に得られるはずのベクトル$\ket{\tilde{\vb*{u}}_3}$は以下のように計算できる
\begin{align*}
  \Braket{\vb*{u}_1|\vb*{a}_3} & = \mqty[\frac{1}{\sqrt{2}}                                        & \frac{1}{\sqrt{2}}  & 0]              \mqty[1     \\ 2 \\ -1] \\
                               & = \dfrac{1}{\sqrt{2}} + \dfrac{2}{\sqrt{2}}                                                                           \\
                               & = \dfrac{3}{\sqrt{2}}                                                                                                 \\
  \Braket{\vb*{u}_2|\vb*{a}_3} & = \mqty[\frac{1}{\sqrt{6}}                                        & -\frac{1}{\sqrt{6}} & \frac{2}{\sqrt{6}}] \mqty[1 \\ 2 \\ -1] \\
                               & = \dfrac{1}{\sqrt{6}} - \dfrac{2}{\sqrt{6}} - \dfrac{2}{\sqrt{6}}                                                     \\
                               & = -\dfrac{3}{\sqrt{6}}
\end{align*}
より、
\begin{align*}
  \ket{\tilde{\vb*{u}}_3} & = \ket{\vb*{a}_3} - \Bigl(\Braket{\vb*{u}_1|\vb*{a}_3}\Bigr) \ket{\vb*{u}_1} - \Bigl(\Braket{\vb*{u}_2|\vb*{a}_3}\Bigr) \ket{\vb*{u}_2} \\
                          & = \mqty[1                                                                                                                               \\ 2 \\ -1] - \frac{3}{\sqrt{2}} \mqty[\frac{1}{\sqrt{2}} \\ \frac{1}{\sqrt{2}} \\ 0] + \frac{3}{\sqrt{6}} \mqty[\frac{1}{\sqrt{6}} \\ -\frac{1}{\sqrt{6}} \\ \frac{2}{\sqrt{6}}] \\
                          & = \mqty[1                                                                                                                               \\ 2 \\ -1] - \mqty[\frac{3}{2} \\ \frac{3}{2} \\ 0] + \mqty[\frac{3}{6} \\ -\frac{3}{6} \\ 1] \\
                          & = \mqty[\frac{6}{6} - \frac{9}{6} + \frac{3}{6}                                                                                         \\ \frac{12}{6} - \frac{9}{6} - \frac{3}{6} \\ -1 - 0 + 1] \\
                          & = \mqty[0                                                                                                                               \\ 0 \\ 0 ]
\end{align*}
最後に残ったのはゼロベクトル

つまり、これまでに得られたものを抜くと何も残らない、ということ

\br

この時点までで、3つの生成元から作られる線形空間は2次元、つまり平面であり、基底は2つであることがわかる

その基底として、ここで作った$\ket{\vb*{u}_1}$と$\ket{\vb*{u}_2}$を使える

\br

なお、消えてしまった$\ket{\vb*{a}_3}$が不要ということではない

今回は$\ket{\vb*{a}_1}$から出発したが、ほかのベクトルから始めることもできる

そうすると別の基底が求まる

そこでは$\ket{\vb*{a}_3}$が、$\ket{\vb*{a}_1}$か$\ket{\vb*{a}_2}$のどちらかの代わりに活躍する

ただし、どんな基底を使っても、結果として作られる平面は同一のもの

\end{document}
