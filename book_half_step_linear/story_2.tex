\documentclass[../book_half_step_linear]{subfiles}

\begin{document}

\section{無駄をはぶく}

\subsection{よい矢印、余分な矢印}

ある矢印$\vb*{x}$を、他の矢印の一次結合の形で書きたいとき、
\begin{itemize}
  \item 2次元系を考えているから2つあれば十分、3つは冗長
  \item 「平行」なものが2つだと不十分
\end{itemize}
などが言える

無駄なものをはぶく、必要最低限で済ます、線形代数にもそれを表すための概念がきちんと用意されている

\sectionline
\subsection{したがうことは、お互いさま}

\keyword{一次従属}は、\keyword{線形従属}とも呼ばれる

「従属」という言葉からわかるように、何かが何かにしたがっている

たとえば、互いをスカラー倍だけで表現できているベクトル$\vb*{a}_1, \vb*{a}_2$を考える
\begin{align*}
  \vb*{a}_1 = -\vb*{a}_2 , \quad \vb*{a}_2 = -\vb*{a}_1
\end{align*}
自分自身をほかの矢印を使って表現できているので、$\vb*{a}_1$は$\vb*{a}_2$にしたがっているし、逆も然り

また、$\vb*{a}_1$と$\vb*{a}_2$の一次結合で表せるベクトル$\vb*{a}_3$は、この2つの矢印$\vb*{a}_1, \vb*{a}_2$にしたがっている
\begin{equation*}
  \vb*{a}_3 = 2\vb*{a}_1 + \vb*{a}_2
\end{equation*}

\sectionline
\subsection{従属は「組」に対する概念}

ここで大切なのは、何かしらの「組」を考えたときに「それらが従属の関係にある」かどうかを判断できること

何かが何かにしたがっていれば、逆のことも言える

たとえば、$\vb*{a}_3 = 2\vb*{a}_1 + \vb*{a}_2$は、
\begin{equation*}
  \vb*{a}_2 = \vb*{a}_3 - 2\vb*{a}_1
\end{equation*}
とも書ける

ほかをしたがえているように見えて、実は自分がしたがっていて…という関係にある

\br

ベクトルの組を考え、どれか1つのベクトルがほかのベクトルの一次結合で表せるときに、それらのベクトルの組は\keyword{一次従属}である、と言う

たとえば、$\{ \vb*{a}_1, \vb*{a}_2, \vb*{a}_3 \}$は一次従属である

\br

一次従属であれば、余分なものが含まれている

そこで、次に一次従属ではないものを考える

\sectionline
\subsection{従属していなければ独立}

線形空間$V$に属する$N$個のベクトル$\vb*{a}_1, \vb*{a}_2, \ldots, \vb*{a}_N$および$N$個の実数$c_1, c_2, \ldots, c_N$に対して、
\begin{equation*}
  c_1\vb*{a}_1 + c_2\vb*{a}_2 + \cdots + c_N\vb*{a}_N = \vb*{0}
\end{equation*}
が成立するのが$c_1 = c_2 = \cdots = c_N = 0$の場合に限られるとき、ベクトル$\vb*{a}_1, \vb*{a}_2, \ldots, \vb*{a}_N$は\keyword{一次独立}であると言う

\br

一次独立の場合、互いに表現できないため、無駄がないとわかる

\sectionline
\subsection{一次独立の定義を噛み砕く}

一次独立の定義に出てきた式
\begin{equation*}
  c_1\vb*{a}_1 + c_2\vb*{a}_2 + \cdots + c_N\vb*{a}_N = \vb*{0}
\end{equation*}
に対して、たとえば$c_1 \neq 0$とする

これは一次独立の条件$c_1 = c_2 = \cdots = c_N = 0$を破っている

今は$c_1 \neq 0$なので、式を
\begin{equation*}
  \vb*{a}_1 = -\frac{c_2}{c_1}\vb*{a}_2 - \cdots - \frac{c_N}{c_1}\vb*{a}_N
\end{equation*}
と変形できる

すると、$\vb*{a}_1$をほかのベクトルで表現できてしまっているので、一次従属であることがわかる

\br

$c_1$以外が$0$でない場合も同様なので、条件$c_1 = c_2 = \cdots = c_N = 0$を満たすときのみ、このような式変形ができない

これが一次独立の状況である

\sectionline
\subsection{「基底」は、必要十分なもの}

ベクトルの一次結合を使って空間を過不足なく表現できる「必要十分であるもの」、それが\keyword{基底}

\br

2次元空間の基底は、平行でない2つのベクトルである

3次元空間に埋め込まれている平面は、2つの平行でないベクトルの一次結合で表現可能なので、その2つのベクトルは、3次元空間の中にある\keyword{部分空間}の基底となる

\br

基底は、「注目している空間」を過不足なく、必要十分に表現できるもの

余分であれば削る必要がある

また、考えている基底で表現できる空間が、もっと大きな空間の部分空間になっていることもある

\sectionline
\subsection{一次独立かどうかが鍵}

$D$次元空間$\mathbb{R}^D$を考えたとき、その部分空間$V \subset \mathbb{R}^D$を作り出すベクトル$\vb*{a}_1, \vb*{a}_2, \ldots, \vb*{a}_D'$を考える

このとき、これらの生成元が一次独立ならば、$\{ \vb*{a}_1, \vb*{a}_2, \ldots, \vb*{a}_D' \}$を$V$の\keyword{基底}と言う

\br

一次従属だと、互いを互いで表現できてしまうので、余分なものがあるとわかる

基底であるかどうかの鍵は、一次独立性があるかどうかである

\br

なお、上の定義において、$D' \leq D$であることに注意

部分空間として、たとえば3次元空間中の平面を考えると、$D'=2$および$D=3$である

3次元空間中で考えても必ずしも3次元空間すべてを表現する必要はない

\br

基底と呼ぶときには、どのような線形空間を考えているのかにも注意が必要である

\section{方法を決めれば表現は「一つ」}

\subsection{基底が変われば、座標は変わる}

ベクトル$\vb*{x}=2\vb*{a}_1 + 3\vb*{a}_2$を、いわゆる「座標」で表現する場合、どのように書くだろうか?

直感的には「右に2つ、上に3つ」と簡単に捉えて、$[2,3]^\top$と考えられる

しかし、これは基底として$\{ \vb*{a}_1, \vb*{a}_2 \}$を考えていたから

\br

一次結合の係数をならべたものが「座標」だが、「座標」というのは使っている基底の情報とセットでないと意味をなさないもの

特定の表現方法、つまり基底を決めてこそ、数をならべたベクトルを作ることができる

\br

これを利用すれば、基底を変えることで目的の計算に便利なベクトルを作ることもできる

\sectionline
\subsection{表現方法はいろいろでも本質は「一つ」}

基底の選び方はたくさんあるが、基底を決めてしまえば表現方法は一つに定まる

つまり、基底が決まれば「座標」は一意に決まる

\br

表現したい矢印やベクトル(本質)は一つ

基底の選び方は表現方法の違いであり、基底を一つに決めれば、表現の仕方は一意に定まる

\sectionline
\subsection{そもそもどうやって無駄なものを知るの?}

基底は無駄をはぶいたものだが、そのためには\keyword{行基本変形}などで一次独立かどうかを調べる必要がある

\end{document}
