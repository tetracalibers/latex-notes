\documentclass[../book_half_step_linear]{subfiles}

\begin{document}

\section{「数の集まり」に「演算」を追加}

\subsection{集まるだけでは面白くないので}

数学では、要素が集まった\keyword{集合}を考えるのが基本

\br

そこにたとえば足し算の\keyword{演算}を入れると、要素間を行き来できるようになる

実数の集合を考えたとき、$7.4+6.4=13.8$のように、二つの要素を足すことで別の要素に移れる

\br

また、\keyword{関係性}まで考えるとさらに応用の幅が広がる

関係性の一つの例は「距離」

\br

ベクトルや行列と同じような「集合・演算・関係性」をもつ対象なら、その類似性を使ってベクトルや行列で扱える

\sectionline

\subsection{足し算が豊かさを与えてくれる}

ベクトルに演算を導入すると、別のベクトルと行き来できるようになる

この演算を入れたものを\keyword{線形空間}という

\sectionline

\subsection{線形空間の定義}

たとえば和を計算したときに、結果として得られた要素が考えている集合からはみ出てしまっては困る

演算で集合の要素を行き来でき、その演算の結果が想定外にならない安全な場所、というのが\keyword{線形空間}

\br

実際には、線形空間$V$は以下の性質を満たすものとして定義できる
\begin{enumerate}
  \item $c\vb*{x} \in V$(スカラー倍しても$V$からはみ出ません)
  \item $\vb*{x}+\vb*{y} \in V$(足し算でもはみ出ません)
  \item $(c_1c_2)\vb*{x}=c_1(c_2\vb*{x})$(スカラー倍は分離できます)
  \item $1\vb*{x}=\vb*{x}$($1$というスカラー倍は要素を変えません)
  \item $\vb*{x}+\vb*{y}=\vb*{y}+\vb*{x}$(足し算の順番は交換できます)
  \item $\vb*{x}+(\vb*{y}+\vb*{z})=(\vb*{x}+\vb*{y})+\vb*{z}$(前半、後半、どちらを先に計算しても同じ)
  \item $\vb*{x}+0=\vb*{x}$となるベクトル$\vb*{0}$が存在する(零元があります)
  \item $\vb*{x}+\vb*{u}=\vb*{0}$となるベクトル$\vb*{u}$が存在し、このベクトル$\vb*{u}$を$-\vb*{x}$と書く、すなわち$\vb*{x}-\vb*{x}=\vb*{0}$(逆元、つまり負符号もあります)
  \item $c_1(\vb*{x}+\vb*{y})=c_1\vb*{x}+c_1\vb*{y}$(足してからスカラー倍、スカラー倍してから足す、が同じ)
  \item $c_1\vb*{x}+c_2\vb*{x}=(c_1+c_2)\vb*{x}$(スカラー倍だけ先に計算も可能)
\end{enumerate}

\section{一次結合がすべての基本}

\subsection{組み合わせるという視点}

演算によってベクトル同士を行き来できるようになると、あるベクトルをほかのベクトルを使って表現できる

スカラー倍と和のみを使った形を\keyword{一次結合}もしくは\keyword{線形結合}という

\sectionline
\subsection{一次結合の係数を求める方法}

$\vb*{a}$と$\vb*{b}$によって$\vb*{c}$を書き表すときの係数は、一般には\keyword{連立方程式}を使って求める
\begin{equation*}
  \vb*{c} = \lambda_1 \vb*{a} + \lambda_2 \vb*{b}
\end{equation*}
から、$\vb*{c}$の各要素$c_i$に対して以下が成り立つ
\begin{equation*}
  c_i = \lambda_1 a_i + \lambda_2 b_i
\end{equation*}
ただし、連立方程式の解がない場合もある

\sectionline
\subsection{分解するという視点}

分解できる場合もあれば、できない場合もある

これは、先ほどの「組み合わせる」という視点において、一次結合を作っても一部のベクトルしか再現できない、ということ

\sectionline
\subsection{空間を生成するという視点}

$r$と$s$は実数から自由に選べるとすると、$\vb*{x}=r\vb*{a_1}+s\vb*{a_2}$でさまざまなベクトル$\vb*{x}$を表現できる

それらを集めると平面が形作られていき、実はこの平面も線形空間になっている

このように一次結合で線形空間を作ることができ、その「もと」となるベクトルのことを\keyword{生成元}という

\end{document}
