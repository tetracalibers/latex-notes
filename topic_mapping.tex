\documentclass[b5paper,12pt,notitlepage]{jsreport}

\title{写像の整理帳}
\author{tomixy}

% === color ===

% ref: https://latexcolor.com/
\definecolor{hotpink}{rgb}{1.0, 0.41, 0.71}
\definecolor{carnationpink}{rgb}{1.0, 0.65, 0.79}
\definecolor{deeppink}{rgb}{1.0, 0.08, 0.58}
\definecolor{capri}{rgb}{0.0, 0.75, 1.0}
\definecolor{rosepink}{rgb}{1.0, 0.4, 0.8}
\definecolor{princetonorange}{rgb}{1.0, 0.56, 0.0}
\definecolor{lavendermagenta}{rgb}{0.93, 0.51, 0.93}
\definecolor{malachite}{rgb}{0.04, 0.85, 0.32}
\definecolor{lawngreen}{rgb}{0.49, 0.99, 0.0}
\definecolor{periwinkle}{rgb}{0.8, 0.8, 1.0}
\definecolor{lightslategray}{rgb}{0.47, 0.53, 0.6}
\definecolor{robineggblue}{rgb}{0.0, 0.8, 0.8}
\definecolor{rosebonbon}{rgb}{0.98, 0.26, 0.62}
\definecolor{airforceblue}{rgb}{0.36, 0.54, 0.66}
\definecolor{columbiablue}{rgb}{0.61, 0.87, 1.0}
\definecolor{magnolia}{rgb}{0.97, 0.96, 1.0}
\definecolor{coolgrey}{rgb}{0.55, 0.57, 0.67}

% === box ===

\usepackage{awesomebox}

% === math ===

\usepackage{physics}
\usepackage{braket}

\usepackage{amssymb} % use \blacksquare

\usepackage{amsthm} % 定理環境とQEDコマンド
\renewcommand{\qedsymbol}{\textcolor{coolgrey}{$\blacksquare$}}

\usepackage{mathtools}
% 別の場所で参照する数式以外は番号が付かないように
\mathtoolsset{showonlyrefs=true}

\usepackage{systeme} % 連立方程式を簡単に書く
\usepackage{empheq}

\newcommand{\id}{\operatorname{id}}
\newcommand{\Id}{\operatorname{Id}}
\newcommand{\diag}{\operatorname{diag}}
\newcommand{\Ker}{\operatorname{Ker}}
\newcommand{\sgn}{\operatorname{sgn}}

\newcommand{\suchthat}{\,\, s.t. \,\,}
\newcommand{\transpose}[1]{{}^t\! #1}

% === font ===

\usepackage{amsfonts} % use \mathbb

\usepackage[T1]{fontenc}
\usepackage{lxfonts}

% monospace font
\renewcommand*\ttdefault{cmvtt}

% === layout ===

\usepackage[top=20truemm,bottom=20truemm,left=20truemm,right=60truemm,marginparwidth=40truemm,marginparsep=10truemm]{geometry} % 余白
\renewcommand{\baselinestretch}{1.25} % 行間

\usepackage{leading}

\setlength{\parindent}{0pt} % 段落始めでの字下げをしない

\usepackage{enumitem}
\newcommand{\romanlabel}{\textsf{\roman*.}}
\newcommand{\romannum}[1]{\textsf{#1}}

\usepackage[noparboxrestore]{marginnote}

\usepackage{tocloft}
% chapterのnumwidthを広くする
\setlength{\cftchapnumwidth}{5em}

\usepackage{titling}
\renewcommand{\maketitlehooka}{\textsf}

% === tikz ===

\usepackage[dvipdfmx]{graphicx}

\usepackage{tikz}
\usetikzlibrary{
  fit,
  patterns,
  decorations.pathreplacing,
  cd,
  petri,
  positioning
}

\usepackage{ifthen}
\usepackage{listofitems} % for \readlist to create arrays

\usepackage{witharrows}
\usepackage{nicematrix}

% === tcolorbox ===

\usepackage{tcolorbox}
\tcbuselibrary{listings,breakable,xparse,skins,hooks,theorems}

\newcommand{\titlegap}{\quad\\[0.1cm]}

\DeclareTColorBox{definition}{m O{} }%
{
  enhanced,
  colframe=magnolia,
  colback=magnolia!20!white,
  coltitle=black,
  fonttitle=\bfseries,
  breakable,
  sharp corners,
  title={\textcolor{Cerulean!60!black}{\faGraduationCap}\hspace{0.1em} #1},
  detach title,
  before upper={\tcbtitle\quad},
  bottom=0.5cm,
  top=0.5cm,
  right=0.5cm,
  left=0.5cm,
  #2
}

\DeclareTColorBox{theorem}{m O{} }%
{
  enhanced,
  colframe=magnolia,
  colback=magnolia!20!white,
  coltitle=black,
  fonttitle=\bfseries,
  breakable,
  sharp corners,
  title={\textcolor{magenta!70!black}{\faAnchor}\hspace{0.1em} #1},
  detach title,
  before upper={\tcbtitle\quad},
  bottom=0.5cm,
  top=0.5cm,
  right=0.5cm,
  left=0.5cm,
  #2
}

% 背景がグレー
\DeclareTColorBox{shaded}{O{} }%
{
  enhanced,
  colframe=white,
  colback=gray!10,
  breakable=true,
  sharp corners,
  detach title,
  bottom=0.25cm,
  top=0.25cm,
  right=0.25cm,
  left=0.25cm,
  #1
}

\newcommand{\ProofColor}{coolgrey}
\DeclareTColorBox{proof}{O{証明}}{%
  empty,
  title={\faBroom #1},
  attach boxed title to top left,
  sharp corners,
  boxed title style={
      empty,
      size=minimal,
      toprule=2pt,
      top=4pt,
      left=1em,
      right=1em,
      top=0.25cm,
      overlay={
          \draw[\ProofColor, double,line width=1pt] ([yshift=-1pt]frame.north west)--([yshift=-1pt]frame.north east);
        }
    },
  coltitle=\ProofColor,
  fonttitle=\bfseries,
  before=\par\medskip\noindent,
  parbox=false,
  boxsep=0pt,
  left=1em,
  right=1em,
  top=0.5cm,
  bottom=0.5cm,
  breakable,
  pad at break*=0mm,
  vfill before first,
  overlay unbroken={
      \draw[\ProofColor,line width=0.5pt]
      ([yshift=-1pt]title.north east)
      --([xshift=-0.5pt,yshift=-1pt]title.north-|frame.east)
      --([xshift=-0.5pt]frame.south east)
      --(frame.south west);
    },
  overlay first={
      \draw[\ProofColor,line width=1pt]([yshift=-1pt]title.north east)--([xshift=-0.5pt,yshift=-1pt]title.north-|frame.east)--([xshift=-0.5pt]frame.south east);
    },
  overlay middle={
      \draw[\ProofColor,line width=1pt] ([xshift=-0.5pt]frame.north east)--([xshift=-0.5pt]frame.south east);
    },
  overlay last={
      \draw[\ProofColor,line width=1pt] ([xshift=-0.5pt]frame.north east)--([xshift=-0.5pt]frame.south east)--(frame.south west);
    },%
}
\NewDocumentCommand{\patterntitle}{m}{
  \tcbox[
    enhanced,
    empty,
    boxsep=0pt,
    left=0pt,right=0pt,
    bottom=2pt,
    fonttitle=\bfseries,
    borderline south={0.5pt}{0pt}{\ProofColor},
  ]{\textcolor{\ProofColor}{#1}}
}
\renewenvironment{quote}{%
  \list{}{%
    \leftmargin0.5cm   % this is the adjusting screw
    \rightmargin\leftmargin
  }
  \item\relax
}{\endlist}
\newenvironment{subpattern}[1]{
  \patterntitle{#1}
  \begin{quote}
    }{
  \end{quote}
}

% === memo ===

\usepackage{zebra-goodies} % TODOなどの注釈

% === original ===

\newcommand{\keyword}[1]{\textcolor{RubineRed}{\textbf{#1}}}
\newcommand{\en}[1]{\textcolor{RubineRed}{\small\texttt{#1}}}
\newcommand{\keywordJE}[2]{\keyword{#1}(\en{\textcolor{RubineRed!60}{#2}})}

\newcommand{\br}{\vskip0.5\baselineskip}

\usepackage[object=vectorian]{pgfornament}
\newcommand{\sectionline}{%
  \noindent
  \begin{center}
    {\color{lightgray}
      \resizebox{0.5\linewidth}{1ex}
      {{%
            {\begin{tikzpicture}
                  \node  (C) at (0,0) {};
                  \node (D) at (9,0) {};
                  \path (C) to [ornament=85] (D);
                \end{tikzpicture}}}}}%
  \end{center}%
}

\renewcommand{\labelitemii}{$\circ$}

\newcommand{\refbook}[1]{\small ref: #1}

% === toc ===

\usepackage{tocloft}
\renewcommand{\cftsecfont}{\rmfamily}
\renewcommand{\cftsecpagefont}{\rmfamily}
\setcounter{secnumdepth}{0}

\addtocontents{toc}{\protect\thispagestyle{empty}}
\pagestyle{empty}

% === hyperlink ===

\definecolor{oxfordblue}{rgb}{0.0, 0.13, 0.28}

% 「%」は以降の内容を「改行コードも含めて」無視するコマンド
\usepackage[%
  dvipdfmx,% 欧文ではコメントアウトする
  pdfencoding=auto, psdextra,% 数学記号を含める
  setpagesize=false,%
  bookmarks=true,%
  bookmarksdepth=tocdepth,%
  bookmarksnumbered=true,%
  colorlinks=true,%
  allcolors=oxfordblue,%
  linkcolor=MidnightBlue,%
  pdftitle={},%
  pdfsubject={},%
  pdfauthor={},%
  pdfkeywords={}%
]{hyperref}
% PDFのしおり機能の日本語文字化けを防ぐ((u)pLaTeXのときのみかく)
\usepackage{pxjahyper}
% ref: https://tex.stackexchange.com/questions/251491/math-symbol-in-section-heading
\pdfstringdefDisableCommands{\def\varepsilon{\textepsilon}}


% === 参考文献 ===

\newcommand{\refbookA}{\refbook{ろんりと集合}}
\newcommand{\refbookB}{\refbook{大学数学 ほんとうに必要なのは「集合」}}
\newcommand{\refbookC}{\refbook{なっとくする 集合・写像}}
\newcommand{\refbookD}{\refbook{集合と位相をなぜ学ぶのか}}
\newcommand{\refbookE}{\refbook{図で整理!例題で納得!線形空間入門}}

% ---

\begin{document}

\maketitle
\tableofcontents

\sectionline
\section{写像}
\marginnote{\refbookA}

\keyword{写像}は、集合の間の「対応」である

\br

関数は、数を入力すると数が出力される「装置」

関数のような「対応」という考え方の対象を「数」に限定せず、「集合の要素」に一般化したものが\keyword{写像}である

\br

写像というときは、どの集合からどの集合への写像であるかをはっきりしておかなければならない

\begin{definition}{写像}
  集合$A, \, B$があったとき、$A$のすべての要素$a$に対して、$B$のある要素$b$を「ただ一つ対応」させる規則$f$が与えられたとき、$f$を$A$から$B$への\keyword{写像}と呼び、記号で
  \begin{equation*}
    f\colon A \to B
  \end{equation*}
  と表す

  このとき、集合$A$を$f$の\keyword{定義域}と呼ぶ

  また、次の集合を$f$の\keyword{値域}と呼ぶ
  \begin{equation*}
    f(A) = \{ f(a) \mid a \in A \}
  \end{equation*}
\end{definition}

「集合」と「写像」というのはそれぞれ、「対象」と「それらの間の対応」ということであり、数学において基本的な概念である

\sectionline
\section{像と逆像}
\marginnote{\refbookE p52〜55}

\begin{definition}{像}
  写像$f$により、$A$の要素$a$が$B$の要素$b$に対応しているとき、「$b$は$a$の$f$による\keyword{像}である」あるいは「$f$により$a$は$b$に写る」といい、
  \begin{gather*}
    f(a) = b \\
    f: a \mapsto b \\
    f: A \to B; a \mapsto b
  \end{gather*}
  などと書く
\end{definition}

集合の言葉で述べると、次のようにも定義できる

\begin{definition}{像と逆像}
  写像$f\colon A \to B$があるとき、$A$の部分集合$A'$に対して、
  \begin{equation*}
    f(A') = \{ f(a) \mid a \in A' \}
  \end{equation*}
  とおき、$f(A')$を$A'$の$f$による\keyword{像}と呼ぶ

  また、$B$の部分集合$B'$に対して、
  \begin{equation*}
    f^{-1}(B') = \{ a \mid f(a) \in B' \}
  \end{equation*}
  とおき、$f^{-1}(B')$を$B'$の$f$による\keyword{逆像}と呼ぶ
\end{definition}

\keyword{値域}は、定義域$A$の像$f(A)$のことにほかならない

\begin{theorem}{像と逆像の性質}
  写像$f\colon A \to B$があるとき、$A$の部分集合$A_1, A_2$と$B$の部分集合$B_1, B_2$に対して、次が成り立つ
  \begin{itemize}
    \item $A_1 \subset A_2 \implies f(A_1) \subset f(A_2)$
    \item $B_1 \subset B_2 \implies f^{-1}(B_1) \subset f^{-1}(B_2)$
  \end{itemize}
\end{theorem}

\begin{theorem}{像と逆像の性質}
  写像$f\colon A \to B$があるとき、$A$の部分集合$A_1, A_2$と$B$の部分集合$B_1, B_2$に対して、次が成り立つ
  \begin{itemize}
    \item $f(A_1 \cap A_2) \subset f(A_1) \cap f(A_2)$
    \item $f(A_1 \cup A_2) = f(A_1) \cup f(A_2)$
    \item $f^{-1}(B_1 \cap B_2) = f^{-1}(B_1) \cap f^{-1}(B_2)$
    \item $f^{-1}(B_1 \cup B_2) = f^{-1}(B_1) \cup f^{-1}(B_2)$
  \end{itemize}
\end{theorem}

\sectionline
\section{単射}

\begin{definition}{単射}
  写像$f\colon A \to B$に対して、$f$が\keyword{単射}であるとは、$A$の任意の要素$a, a'$に対して
  \begin{equation*}
    f(a) = f(a') \implies a = a'
  \end{equation*}
  が成り立つことをいう
\end{definition}

この主張の対偶
\begin{equation*}
  a \ne a' \implies f(a) \ne f(a')
\end{equation*}
を考えれば、\keyword{単射}であるということは、「異なる要素が$f$によって同じ要素に対応することはない」ということにほかならない

\sectionline
\section{全射}

\begin{definition}{全射}
  写像$f\colon A \to B$に対して、$f$が\keyword{全射}であるとは、
  \begin{equation*}
    f(A) = B
  \end{equation*}
  すなわち
  \begin{equation*}
    \forall b \in B, \exists a \in A\colon f(a) = b
  \end{equation*}
  が成り立つことをいう
\end{definition}

言い換えると、$B$への写像$f$が\keyword{全射}であるとは、$B$の要素に「対応していないものがない」ということ

\sectionline
\section{全単射}

\begin{definition}{全単射}
  集合$A$から集合$B$への写像$f$が単射かつ全射であるときは、\keyword{全単射}であるという
\end{definition}

これは、写像$f$により、集合$A$の要素と集合$B$の要素が「一対一に対応している」ことにほかならない

\sectionline

数学では、数学的構造を保つ写像が重要であり、特に、構造を保つ全単射写像のことは\keyword{同型写像}と呼ぶ

\sectionline
\section{逆写像}

\begin{definition}{逆写像}
  写像$f\colon A \to B$が全単射であるとき、対応が一対一であるので、逆向きの対応、すなわち、$B$から$A$への対応を考えることができる

  この対応により定義される写像を$f$の\keyword{逆写像}と呼び、記号で$f^{-1}$と書く
\end{definition}

\sectionline
\section{合成写像}
\marginnote{\refbookE p55〜56}

「2つの操作を続けて行う」ことは、写像の\keyword{合成}として定義される

\begin{definition}{合成写像}
  2つの写像
  \begin{align*}
    f\colon A & \to B \\
    g\colon B & \to C
  \end{align*}
  が与えられたとき、$A$の要素$a$に対して、$C$の要素$g(f(a))$を対応させる、集合$A$から集合$C$への写像のことを$f$と$g$の\keyword{合成写像}と呼び、記号で$g \circ f$と書く

  すなわち
  \begin{equation*}
    (g \circ f)(a) = g(f(a))
  \end{equation*}
  である
\end{definition}

\sectionline
\section{恒等写像}
\marginnote{\refbookE p55〜56}

「何も変えない写像」は\keyword{恒等写像}と呼ばれる

\begin{definition}{恒等写像}
  集合$A$に対して、$A$の要素$a$を同じ要素$a$に対応させる、$A$から$A$への写像
  \begin{equation*}
    A \to A; a \mapsto a
  \end{equation*}
  を$A$上の\keyword{恒等写像}といい、$I_A$や$\Id_A$、あるいは単に$\Id$と書く
\end{definition}

\sectionline
\section{単射と全射の双対性}

\begin{definition}{左逆写像}
  写像$f\colon A \to B$に対して、写像$g\colon B \to A$が存在して、
  \begin{equation*}
    g \circ f = I_A
  \end{equation*}
  を満たすとき、$g$は$f$の\keyword{左逆写像}であるという
\end{definition}

\begin{definition}{右逆写像}
  写像$f\colon A \to B$に対して、写像$g\colon B \to A$が存在して、
  \begin{equation*}
    f \circ g = I_B
  \end{equation*}
  を満たすとき、$g$は$f$の\keyword{右逆写像}であるという
\end{definition}

\begin{theorem}{全単射の特徴づけ}
  写像$f\colon A \to B$に対して、次の2つは同値になる
  \begin{enumerate}
    \item $f$は全単射である
    \item $f$の左逆写像であり、右逆写像でもある写像が存在する
  \end{enumerate}
\end{theorem}

\sectionline

「逆写像」という観点からみることにより、「単射」と「全射」は双対的な概念であることがわかる

\begin{theorem}{単射の特徴づけ}
  写像$f\colon A \to B$に対して、次の2つは同値になる
  \begin{enumerate}
    \item $f$は単射である
    \item $f$の左逆写像が存在する
  \end{enumerate}
\end{theorem}

\begin{theorem}{全射の特徴づけ}
  写像$f\colon A \to B$に対して、次の2つは同値になる
  \begin{enumerate}
    \item $f$は全射である
    \item $f$の右逆写像が存在する
  \end{enumerate}
\end{theorem}

\sectionline
\section{関数}

\begin{definition}{関数}
  写像$f\colon A \to B$に対して、集合$B$が数の集合のとき、写像$f$を\keyword{関数}と呼ぶ
\end{definition}

関数$y=f(x)$は、
\begin{itemize}
  \item 「関数」としてみれば、「$x$を入力すると$y$が出力される」
  \item 「写像」としてみれば、「$x$に対して$y$を対応させる」
\end{itemize}

\sectionline
\section{関数の単射と全射}

関数が\keyword{単射}であるとは、「同じ値を取るものがない」ということ

たとえば、\keyword{単調増加関数}と\keyword{単調減少関数}は単射

連続関数$f(x)$が単射であるのは、グラフに山や谷がないとき

\sectionline

関数が\keyword{全射}であるとは、関数$f(x)$を$\mathbb{R}$への写像と見なしたとき、$y$軸上に対応する$x$がない点がないということ

\end{document}
