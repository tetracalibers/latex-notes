\documentclass[b5paper,12pt]{jsarticle}

\title{Topic Note: 集合}
\author{tomixy}

% === color ===

% ref: https://latexcolor.com/
\definecolor{hotpink}{rgb}{1.0, 0.41, 0.71}
\definecolor{carnationpink}{rgb}{1.0, 0.65, 0.79}
\definecolor{deeppink}{rgb}{1.0, 0.08, 0.58}
\definecolor{capri}{rgb}{0.0, 0.75, 1.0}
\definecolor{rosepink}{rgb}{1.0, 0.4, 0.8}
\definecolor{princetonorange}{rgb}{1.0, 0.56, 0.0}
\definecolor{lavendermagenta}{rgb}{0.93, 0.51, 0.93}
\definecolor{malachite}{rgb}{0.04, 0.85, 0.32}
\definecolor{lawngreen}{rgb}{0.49, 0.99, 0.0}
\definecolor{periwinkle}{rgb}{0.8, 0.8, 1.0}
\definecolor{lightslategray}{rgb}{0.47, 0.53, 0.6}
\definecolor{robineggblue}{rgb}{0.0, 0.8, 0.8}
\definecolor{rosebonbon}{rgb}{0.98, 0.26, 0.62}
\definecolor{airforceblue}{rgb}{0.36, 0.54, 0.66}
\definecolor{columbiablue}{rgb}{0.61, 0.87, 1.0}
\definecolor{magnolia}{rgb}{0.97, 0.96, 1.0}
\definecolor{coolgrey}{rgb}{0.55, 0.57, 0.67}

% === box ===

\usepackage{awesomebox}

% === math ===

\usepackage{physics}
\usepackage{braket}

\usepackage{amssymb} % use \blacksquare

\usepackage{amsthm} % 定理環境とQEDコマンド
\renewcommand{\qedsymbol}{\textcolor{coolgrey}{$\blacksquare$}}

\usepackage{mathtools}
% 別の場所で参照する数式以外は番号が付かないように
\mathtoolsset{showonlyrefs=true}

\usepackage{systeme} % 連立方程式を簡単に書く
\usepackage{empheq}

\newcommand{\id}{\operatorname{id}}
\newcommand{\Id}{\operatorname{Id}}
\newcommand{\diag}{\operatorname{diag}}
\newcommand{\Ker}{\operatorname{Ker}}
\newcommand{\sgn}{\operatorname{sgn}}

\newcommand{\suchthat}{\,\, s.t. \,\,}
\newcommand{\transpose}[1]{{}^t\! #1}

% === font ===

\usepackage{amsfonts} % use \mathbb

\usepackage[T1]{fontenc}
\usepackage{lxfonts}

% monospace font
\renewcommand*\ttdefault{cmvtt}

% === layout ===

\usepackage[top=20truemm,bottom=20truemm,left=20truemm,right=60truemm,marginparwidth=40truemm,marginparsep=10truemm]{geometry} % 余白
\renewcommand{\baselinestretch}{1.25} % 行間

\usepackage{leading}

\setlength{\parindent}{0pt} % 段落始めでの字下げをしない

\usepackage{enumitem}
\newcommand{\romanlabel}{\textsf{\roman*.}}
\newcommand{\romannum}[1]{\textsf{#1}}

\usepackage[noparboxrestore]{marginnote}

\usepackage{tocloft}
% chapterのnumwidthを広くする
\setlength{\cftchapnumwidth}{5em}

\usepackage{titling}
\renewcommand{\maketitlehooka}{\textsf}

% === tikz ===

\usepackage[dvipdfmx]{graphicx}

\usepackage{tikz}
\usetikzlibrary{
  fit,
  patterns,
  decorations.pathreplacing,
  cd,
  petri,
  positioning
}

\usepackage{ifthen}
\usepackage{listofitems} % for \readlist to create arrays

\usepackage{witharrows}
\usepackage{nicematrix}

% === tcolorbox ===

\usepackage{tcolorbox}
\tcbuselibrary{listings,breakable,xparse,skins,hooks,theorems}

\newcommand{\titlegap}{\quad\\[0.1cm]}

\DeclareTColorBox{definition}{m O{} }%
{
  enhanced,
  colframe=magnolia,
  colback=magnolia!20!white,
  coltitle=black,
  fonttitle=\bfseries,
  breakable,
  sharp corners,
  title={\textcolor{Cerulean!60!black}{\faGraduationCap}\hspace{0.1em} #1},
  detach title,
  before upper={\tcbtitle\quad},
  bottom=0.5cm,
  top=0.5cm,
  right=0.5cm,
  left=0.5cm,
  #2
}

\DeclareTColorBox{theorem}{m O{} }%
{
  enhanced,
  colframe=magnolia,
  colback=magnolia!20!white,
  coltitle=black,
  fonttitle=\bfseries,
  breakable,
  sharp corners,
  title={\textcolor{magenta!70!black}{\faAnchor}\hspace{0.1em} #1},
  detach title,
  before upper={\tcbtitle\quad},
  bottom=0.5cm,
  top=0.5cm,
  right=0.5cm,
  left=0.5cm,
  #2
}

% 背景がグレー
\DeclareTColorBox{shaded}{O{} }%
{
  enhanced,
  colframe=white,
  colback=gray!10,
  breakable=true,
  sharp corners,
  detach title,
  bottom=0.25cm,
  top=0.25cm,
  right=0.25cm,
  left=0.25cm,
  #1
}

\newcommand{\ProofColor}{coolgrey}
\DeclareTColorBox{proof}{O{証明}}{%
  empty,
  title={\faBroom #1},
  attach boxed title to top left,
  sharp corners,
  boxed title style={
      empty,
      size=minimal,
      toprule=2pt,
      top=4pt,
      left=1em,
      right=1em,
      top=0.25cm,
      overlay={
          \draw[\ProofColor, double,line width=1pt] ([yshift=-1pt]frame.north west)--([yshift=-1pt]frame.north east);
        }
    },
  coltitle=\ProofColor,
  fonttitle=\bfseries,
  before=\par\medskip\noindent,
  parbox=false,
  boxsep=0pt,
  left=1em,
  right=1em,
  top=0.5cm,
  bottom=0.5cm,
  breakable,
  pad at break*=0mm,
  vfill before first,
  overlay unbroken={
      \draw[\ProofColor,line width=0.5pt]
      ([yshift=-1pt]title.north east)
      --([xshift=-0.5pt,yshift=-1pt]title.north-|frame.east)
      --([xshift=-0.5pt]frame.south east)
      --(frame.south west);
    },
  overlay first={
      \draw[\ProofColor,line width=1pt]([yshift=-1pt]title.north east)--([xshift=-0.5pt,yshift=-1pt]title.north-|frame.east)--([xshift=-0.5pt]frame.south east);
    },
  overlay middle={
      \draw[\ProofColor,line width=1pt] ([xshift=-0.5pt]frame.north east)--([xshift=-0.5pt]frame.south east);
    },
  overlay last={
      \draw[\ProofColor,line width=1pt] ([xshift=-0.5pt]frame.north east)--([xshift=-0.5pt]frame.south east)--(frame.south west);
    },%
}
\NewDocumentCommand{\patterntitle}{m}{
  \tcbox[
    enhanced,
    empty,
    boxsep=0pt,
    left=0pt,right=0pt,
    bottom=2pt,
    fonttitle=\bfseries,
    borderline south={0.5pt}{0pt}{\ProofColor},
  ]{\textcolor{\ProofColor}{#1}}
}
\renewenvironment{quote}{%
  \list{}{%
    \leftmargin0.5cm   % this is the adjusting screw
    \rightmargin\leftmargin
  }
  \item\relax
}{\endlist}
\newenvironment{subpattern}[1]{
  \patterntitle{#1}
  \begin{quote}
    }{
  \end{quote}
}

% === memo ===

\usepackage{zebra-goodies} % TODOなどの注釈

% === original ===

\newcommand{\keyword}[1]{\textcolor{RubineRed}{\textbf{#1}}}
\newcommand{\en}[1]{\textcolor{RubineRed}{\small\texttt{#1}}}
\newcommand{\keywordJE}[2]{\keyword{#1}(\en{\textcolor{RubineRed!60}{#2}})}

\newcommand{\br}{\vskip0.5\baselineskip}

\usepackage[object=vectorian]{pgfornament}
\newcommand{\sectionline}{%
  \noindent
  \begin{center}
    {\color{lightgray}
      \resizebox{0.5\linewidth}{1ex}
      {{%
            {\begin{tikzpicture}
                  \node  (C) at (0,0) {};
                  \node (D) at (9,0) {};
                  \path (C) to [ornament=85] (D);
                \end{tikzpicture}}}}}%
  \end{center}%
}

\renewcommand{\labelitemii}{$\circ$}

\newcommand{\refbook}[1]{\small ref: #1}

% === toc ===

\usepackage{tocloft}
\renewcommand{\cftsecfont}{\rmfamily}
\renewcommand{\cftsecpagefont}{\rmfamily}
\setcounter{secnumdepth}{0}

\addtocontents{toc}{\protect\thispagestyle{empty}}
\pagestyle{empty}

% === hyperlink ===

\definecolor{oxfordblue}{rgb}{0.0, 0.13, 0.28}

% 「%」は以降の内容を「改行コードも含めて」無視するコマンド
\usepackage[%
  dvipdfmx,% 欧文ではコメントアウトする
  pdfencoding=auto, psdextra,% 数学記号を含める
  setpagesize=false,%
  bookmarks=true,%
  bookmarksdepth=tocdepth,%
  bookmarksnumbered=true,%
  colorlinks=true,%
  allcolors=oxfordblue,%
  linkcolor=MidnightBlue,%
  pdftitle={},%
  pdfsubject={},%
  pdfauthor={},%
  pdfkeywords={}%
]{hyperref}
% PDFのしおり機能の日本語文字化けを防ぐ((u)pLaTeXのときのみかく)
\usepackage{pxjahyper}
% ref: https://tex.stackexchange.com/questions/251491/math-symbol-in-section-heading
\pdfstringdefDisableCommands{\def\varepsilon{\textepsilon}}


% === 参考文献 ===

\newcommand{\refbookA}{\refbook{ろんりと集合}}
\newcommand{\refbookB}{\refbook{大学数学 ほんとうに必要なのは「集合」}}

% ---

\begin{document}

\maketitle
\tableofcontents

\sectionline
\section{集合}

\marginnote{\refbookA}

\keyword{集合}とは「ものの集まり」のことであり、その「ものの集まり」に入っているか、あるいは、入っていないかが客観的に判断できるもの

\sectionline
\marginnote{\refbookB}

「何かしらの対象」と「何かしらの集まり」としておけば、汎用性が高いまま抽象的な議論ができる点が集合を勉強する意義

\begin{definition}{集合}
  何かしらの対象の集まりを\keyword{集合}といい、その集合に入る何かしらの対象を\keyword{元}という
\end{definition}

\sectionline
\section{集合の要素}

\marginnote{\refbookA}

集合を構成する個々の「もの」を、その集合の\keyword{要素}あるいは\keyword{元}と呼ぶ

\br

$x$が集合$A$の要素であるとき、$x$は$A$に\keyword{含まれる}、あるいは\keyword{属する}と言い、記号では$x \in A$と書く


\sectionline
\marginnote{\refbookB}

任意の対象は、「ある集合$A$の元」か「ある集合$A$の元でない」かどちらかが考えられる

\begin{definition}{属する}
  集合$A$があるとする。
  このとき、ある対象$a$が集合$A$に入ることを$a \in A$と表し、$a$が集合$A$に入らないことを$a \notin A$と表す
\end{definition}

\sectionline
\section{集合の表記法}

\marginnote{\refbookA}

次のような2つの方法がある
\begin{itemize}
  \item $\{x_1, x_2, \cdots \}$(集合を書き並べる方法:\keyword{外延的記法})
  \item $\{x \mid x \text{は条件〜を満たす}\}$(要素になる条件を書く方法:\keyword{内包的記法})
\end{itemize}

集合では、このように、要素を括弧$\{\}$で囲んで記述する

\sectionline
\section{集合の「等しい」}

集合$A$と集合$B$が\keyword{等しい}とは、
\begin{shaded}
  $A$の要素がすべて$B$の要素であり、かつ、$B$の要素がすべて$A$の要素である
\end{shaded}
ことを言う

集合$A$と集合$B$が等しいとき、$A = B$と書く

\sectionline
\section{有限集合と無限集合}

集合に含まれる要素の個数が有限個のとき\keyword{有限集合}といい、無限個のとき\keyword{無限集合}と呼ぶ

\sectionline
\section{空集合}

「要素が1つもない集まり」も、1つの集合とみなして、\keyword{空集合}と呼び、記号$\emptyset$で表す

\marginnote{\refbookB}

\begin{definition}{空集合}
  何も含まれていない集まりのことを\keyword{空集合}といい、$\phi $で表す
\end{definition}

\sectionline
\section{部分集合}

\marginnote{\refbookA}

2つの集合$A$と$B$に対して、$A$は$B$の\keyword{部分集合}である($A$は$B$に\keyword{含まれる})とは、
\begin{shaded}
  $A$のすべての要素が$B$の要素になっている
\end{shaded}
ことを言い、記号では$A \subset B$と書く

\br

「$A \subset B$かつ$B \subset A$である」ことは、$A = B$であることに他ならない

\sectionline
\section{共通部分}

いくつかの集合があったとき、それらの「共通の部分」、すなわち、
\begin{shaded}
  それらの共通の要素を集めてできた集合
\end{shaded}
のことを\keyword{共通部分}という

共通部分には$\cap$という記号が用いられる

\sectionline

たとえば、2つの集合$A,\,B$に対して、$A$と$B$のどちらにも含まれている要素の全体からなる集合を$A$と$B$の\keyword{共通部分}と呼び、記号では$A \cap B$と書く

\br

すなわち、
\begin{equation*}
  A \cap B = \{x \mid x \in A \land x \in B\}
\end{equation*}

\sectionline

有限個の集合でも同様に、集合$A_1, A_2, \cdots , A_n$に対して、すべての$A_i$に含まれている要素の全体からなる集合を、集合$A_1, A_2, \cdots , A_n$の\keyword{共通部分}と呼び、記号で
\begin{equation*}
  A_1 \cap A_2 \cap \cdots \cap A_n \quad \text{あるいは} \quad \bigcap_{i=1}^n A_i
\end{equation*}
と書く

\br

すなわち、
\begin{align*}
  A_1 \cap A_2 \cap \cdots \cap A_n
   & = \{x \mid \forall A_i , c \in A_i \}               \\
   & = \{x \mid x \in A_1 \land \cdots \land x \in A_n\}
\end{align*}

\sectionline

いくつかの集合があって、それらのどの2つも共通部分をもたないとき、それらは\keyword{互いに素}であるという

\sectionline
\section{和集合}

いくつかの集合があったとき、
\begin{shaded}
  それらの集合をすべて集めてできた集合
\end{shaded}
のことを\keyword{和集合}という

和集合には$\cup$という記号が用いられる

\sectionline

たとえば、2つの集合$A,\,B$に対して、$A$と$B$のどちらかに含まれている要素の全体からなる集合を$A$と$B$の\keyword{和集合}と呼び、記号では$A \cup B$と書く

\br

すなわち、
\begin{equation*}
  A \cup B = \{x \mid x \in A \lor x \in B\}
\end{equation*}

\sectionline

有限個の集合でも同様に、集合$A_1, A_2, \cdots , A_n$に対して、ある$A_i$に含まれている要素の全体からなる集合を、集合$A_1, A_2, \cdots , A_n$の\keyword{和集合}と呼び、記号で
\begin{equation*}
  A_1 \cup A_2 \cup \cdots \cup A_n \quad \text{あるいは} \quad \bigcup_{i=1}^n A_i
\end{equation*}
と書く

\br

すなわち、
\begin{align*}
  A_1 \cup A_2 \cup \cdots \cup A_n
   & = \{x \mid \exists A_i , c \in A_i \}             \\
   & = \{x \mid x \in A_1 \lor \cdots \lor x \in A_n\}
\end{align*}

\sectionline
\section{集合と論理の間の対応関係}

「集合」と「論理」は対応しているため、論理で登場した法則は集合に対しても成り立つ

\begin{theorem}{冪等法則}
  \begin{align*}
    A \cap A & = A \\
    A \cup A & = A
  \end{align*}
\end{theorem}

\begin{theorem}{交換法則}
  \begin{align*}
    A \cap B & = B \cap A \\
    A \cup B & = B \cup A
  \end{align*}
\end{theorem}

\begin{theorem}{結合法則}
  \begin{align*}
    A \cap (B \cap C) & = (A \cap B) \cap C \\
    A \cup (B \cup C) & = (A \cup B) \cup C
  \end{align*}
\end{theorem}

\begin{theorem}{分配法則}
  \begin{align*}
    A \cap (B \cup C) & = (A \cap B) \cup (A \cap C) \\
    A \cup (B \cap C) & = (A \cup B) \cap (A \cup C)
  \end{align*}
\end{theorem}

\begin{theorem}{吸収法則}
  \begin{align*}
    A \cap (A \cup B) & = A \\
    A \cup (A \cap B) & = A
  \end{align*}
\end{theorem}

たとえば、交換法則の証明は次のようになる
\begin{equation}
  \begin{WithArrows}
    & \phantom{=} A \cap B \Arrow{$\cap$の定義}\\
    & =\{x \mid x \in A \land x \in B\} \Arrow{論理の交換法則} \\
    &=\{x \mid x \in B \land x \in A\} \Arrow{$\cap$の定義} \\
    & = B \cap A
  \end{WithArrows}
\end{equation}

この証明を見てみると、「集合の性質」と「論理の性質」が対応していることがわかる

\sectionline

「集合」というのは、内包的記法により
\begin{equation*}
  \text{集合} = \{x \mid x \text{は〜である}\}
\end{equation*}
という形で表現できるが、「$x$は〜である」というのは、「論理」の命題関数である

\br

すなわち、命題関数$p(x)$を用いて、
\begin{equation*}
  \text{集合} = \{x \mid p(x)\}
\end{equation*}
と書ける

\br

このとき、$\cap$と$\cup$の定義から、
\begin{align*}
  \{x \mid p(x)\} \cap \{x \mid q(x)\} & = \{x \mid p(x) \land q(x)\} \\
  \{x \mid p(x)\} \cup \{x \mid q(x)\} & = \{x \mid p(x) \lor q(x)\}
\end{align*}
となる

\br

さらに、次の2つの主張は同値である
\begin{itemize}
  \item $p(x) \equiv q(x)$
  \item $\{x \mid p(x)\} = \{x \mid q(x)\}$
\end{itemize}

\br

もっと一般に、次の2つの主張が同値であることが確かめられる
\begin{itemize}
  \item $p(x) \Rightarrow q(x)$
  \item $\{x \mid p(x)\} \subset \{x \mid q(x)\}$
\end{itemize}

\sectionline
\section{全体集合と補集合}

集合にも、論理の「否定」に対するものがある

それが\keyword{補集合}というもの

\br

集合の場合は「〜でない」という要素を集めてくる必要があるので、「どこまでの範囲」の中で集めるかということをあらかじめ設定しておかなければならない

その「どこまでの範囲」として、あらかじめ定められた1つの集合のことを\keyword{全体集合}という

\sectionline

枠組みとなる集合を1つ固定して、扱う集合をその部分集合に限るとき、その枠組みとなる集合を\keyword{全体集合}という

全体集合は$\Omega $という記号を用いることが多い

\br

また、全体集合$\Omega $が定まっているとき、$\Omega$の部分集合$A$に対して、$A$に含まれていない$\Omega$の要素の全体からなる集合を$A$の\keyword{補集合}と呼び、記号では$\overline{A}$あるいは$A^c$と書く
\begin{align*}
  A^c & = \{x \mid x \in \Omega \land x \notin A\} \\
      & = \Omega - A
\end{align*}

\sectionline

補集合を用いると、論理の反射法則とド・モルガンの法則に対応する、集合の法則が得られる

\begin{theorem}{反射法則}
  \begin{equation*}
    (A^c)^c = A
  \end{equation*}
\end{theorem}

\begin{theorem}{ド・モルガンの法則}
  \begin{align*}
    (A \cap B)^c & = A^c \cup B^c \\
    (A \cup B)^c & = A^c \cap B^c
  \end{align*}
\end{theorem}

補集合は論理の「否定」に対応している

\sectionline

全体集合という枠組みの設定のもとで、「空集合」と「全体集合」は双対的な概念であることがわかる

\begin{theorem}{空集合の性質}
  \begin{align*}
    A \cap \emptyset & = \emptyset \\
    A \cup \emptyset & = A
  \end{align*}
\end{theorem}

\begin{theorem}{全体集合の性質}
  \begin{align*}
    A \cap \Omega & = A      \\
    A \cup \Omega & = \Omega
  \end{align*}
\end{theorem}

これらの性質において、
\begin{itemize}
  \item $\cap$を$\cup$に
  \item $\cup$を$\cap$に
  \item $\emptyset$を$\Omega$に
  \item $\Omega$を$\emptyset$に
\end{itemize}
置き換えると、
\begin{itemize}
  \item $A \cap \emptyset = \emptyset \quad \leftrightarrow \quad A \cup \Omega = \Omega$
  \item $A \cup \emptyset = A \quad \leftrightarrow \quad A \cap \Omega = A$
\end{itemize}
という対応が得られ、空集合と全体集合が双対的であることがわかる

\sectionline

空集合の性質は恒偽命題の性質に対応し、全体集合の性質は恒真命題の性質に対応する

\br

つまり、
\begin{shaded}
  空集合$\emptyset$が論理の恒偽命題$O$に対応し、全体集合$\Omega$が論理の恒真命題$I$に対応している
\end{shaded}

実際、
\begin{align*}
  \Omega    & = \{x \in \Omega \mid I\} \\
  \emptyset & = \{x \in \Omega \mid O\}
\end{align*}
ということ

この証明も、対応する論理の法則を用いれば容易に得られる

\br

また、「空集合と全体集合の双対性」は「恒偽命題と恒真命題の双対性」に対応している

\sectionline

補集合については、次の性質が定義からわかる

\begin{theorem}{補集合の性質}
  \begin{align*}
    A \cap A^c & = \emptyset \\
    A \cup A^c & = \Omega
  \end{align*}
\end{theorem}

これらの性質はそれぞれ、論理の矛盾法則と排中法則に対応している

\sectionline

「集合」と「論理」は、双対性を備えた単純できれいな構造を持ち、それらの間には双対的な関係が成り立っている

\sectionline
\section{直積集合}

2つの集合$A, \, B$に対して、$A$の要素$a$と$B$の要素$b$の組$(a, b)$をすべて集めてできた集合のことを$A \times B$と書き、$A$と$B$の\keyword{直積集合}、あるいは単に\keyword{直積}という

\begin{equation*}
  A \times B = \{(a, b) \mid a \in A \land b \in B\}
\end{equation*}

有限個の集合でも同様に、有限個の集合$A_1, A_2, \cdots , A_n$に対して、$A_1$の要素$a_1$、$A_2$の要素$a_2$、$\cdots$、$A_n$の要素$a_n$の組$(a_1, a_2, \cdots , a_n)$をすべて集めてできた集合を$A_1 \times A_2 \times \cdots \times A_n$と書き、$A_1, A_2, \cdots , A_n$の\keyword{直積集合}、あるいは単に\keyword{直積}という

\begin{equation*}
  A_1 \times A_2 \times \cdots \times A_n = \{(a_1, a_2, \cdots , a_n) \mid a_i \in A_i\}
\end{equation*}

\sectionline

集合$A$の$n$個の直積$A \times A \times \cdots \times A$のことを$A^n$と書く

たとえば、
\begin{align*}
  A^2 & = A \times A          \\
  A^3 & = A \times A \times A
\end{align*}
などであり、このような記述は、「平面$\mathbb{R}^2$」や「空間$\mathbb{R}^3$」のように使われる

\sectionline
\section{同値関係と商集合}

等式$a=b$や不等式$a < b$のような2つの要素$a$と$b$の間にある性質を表示したものを\keyword{関係}という

\begin{definition}{関係}
  集合$A$の\keyword{関係}とは、$A \times A$の部分集合$R$のこと

  このとき、2つの要素$a$と$b$に対して、$(a, b) \in R$であるとき、$a$と$b$は関係$R$をもつという
\end{definition}

\begin{definition}{同値関係}
  集合$A$に対して、$A$上の関係$\sim $が次の3つの性質を満たすとき、$\sim$は$A$上の\keyword{同値関係}と呼ぶ
  \begin{description}
    \item[反射法則] $a \sim a$
    \item[対称法則] $a \sim b \Rightarrow b \sim a$
    \item[推移法則] $a \sim b \land b \sim c \Rightarrow a \sim c$
  \end{description}
  ここで、$a, b, c$は$A$の任意の要素である
\end{definition}

\begin{definition}{同値類}
  集合$A$上の同値関係$\sim$があるとき、$a \in A$に対して、$a$と同値な$A$の要素をすべて集めた集合を$a$の\keyword{同値類}と呼び、記号では$[a]$と書く
  \begin{equation*}
    [a] = \{b \in A \mid b \sim a\}
  \end{equation*}
\end{definition}

\begin{definition}{商集合}
  同値類をすべて集めた集合のことを、$A$の同値関係$\sim$による\keyword{商集合}と呼び、記号では$A/\sim $と書く
  \begin{equation*}
    A/\sim = \{[a] \mid a \in A\}
  \end{equation*}
\end{definition}

\begin{definition}{代表元}
  同値類$[a]$に含まれる要素のことを、同値類$[a]$の\keyword{代表元}と呼ぶ
\end{definition}

たとえば、平面や空間内のベクトルは同値類である

「平行移動で重なり合う2つの矢線ベクトルは同値である」とした同値類のことをベクトル$\vb*{a}$と呼んでいる

\end{document}
