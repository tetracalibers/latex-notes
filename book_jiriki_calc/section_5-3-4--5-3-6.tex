\documentclass[../book_jiriki_calc]{subfiles}

\begin{document}

\section{平面の幾何 --- 余弦定理}

2つの地点の間の距離を知りたくても、その間に山や池があって直接距離を測るのが難しいことがある

こんなとき、第3の地点からの角度を含めた測量データがあれば、知りたい距離を計算できることがある

\sectionline

AB間の距離$n$は、別の2つの辺の長さ$l,\,m$とその間の角度$\theta$から計算できる(AとBの間に池があってもかまわない)

この計算方法を与えるのが\keyword{余弦定理}である
\begin{equation*}
  n^2 = l^2 + m^2 - 2lm\cos\theta
\end{equation*}
この式を書き換えると、3辺の情報から、角度を知る公式にもなる
\begin{equation*}
  \cos\theta = \frac{l^2 + m^2 - n^2}{2lm}
\end{equation*}

\br

余弦定理は、三平方の定理(ピタゴラスの定理)を拡張した定理と見なせる

実際、$\theta = 90^\circ$のとき$\cos\theta = 0$であるから、余弦定理はピタゴラスの定理に一致する
\begin{equation*}
  n^2 = l^2 + m^2
\end{equation*}

\sectionline

余弦定理を座標で書き表してみる

まず、この3頂点が$xy$平面にあるものと思って、ベクトルで表示する
\begin{align*}
  \overrightarrow{OA} & = (a,b)     \\
  \overrightarrow{OB} & = (p,q)     \\
  \overrightarrow{BA} & = (a-p,b-q)
\end{align*}
とおくと、各辺の長さは、
\begin{align*}
  l & = \sqrt{a^2 + b^2}         \\
  m & = \sqrt{p^2 + q^2}         \\
  n & = \sqrt{(a-p)^2 + (b-q)^2}
\end{align*}
と表せる

これを使って、$\cos\theta = \dfrac{l^2 + m^2 - n^2}{2lm}$の分子を計算すると、多くの項が打ち消し合って、
\begin{multline*}
  l^2 + m^2 - n^2 \\ = a^2 + b^2 + p^2 + q^2 - ((a-p)^2 - (b-q)^2)
  \\ = 2ap + 2bq
\end{multline*}
となるので、余弦定理は、
\begin{align}
  \cos\theta               & = \dfrac{2(ap+bq)}{2lm} \\
  \therefore \quad ap + bq & = lm\cos\theta
\end{align}
と書き換えられる

\sectionline

この式の両辺をあらためて観察してみる

\begin{equation*}
  ap + bq = lm\cos\theta
\end{equation*}

左辺に現れる$a, b, p, q$という数は、座標系を決めないと値が定まらない

たとえば、三角形が地面に描かれているとする

地面の上なので、好きな点を原点にとり、好きな方向を$x$軸に選び、それと垂直に$y$軸を定める

そうするとベクトル$\overrightarrow{OA}$や$\overrightarrow{OB}$の$x$成分や$y$成分である$a, b, p, q$の値が定まるが、別の座標系をとれば、ベクトルの成分$a, b, p, q$は別の値になる

\br

しかし、右辺は、辺の長さや角度という三角形の幾何に固有な量だけで表されている

$ap+bq$は、どんな座標系でも同じ値になる、\keyword{三角形に内在的な量}なのだ

\br

この重要な量$ap+bq$を、2つのベクトル$\overrightarrow{OA}$と$\overrightarrow{OB}$の\keyword{内積}あるいは\keyword{スカラー積}あるいは\keyword{ドット積}という

$ap+bq$は、\keyword{座標系のとり方に依存しない}がゆえに重要な量である

\sectionline

次の2つの定義が一致するというのが、
\begin{equation*}
  ap + bq = lm\cos\theta
\end{equation*}
という等式の意味である

\br

\paragraph{内積の座標による定義}\quad

ベクトル$(a,b)$と$(p,q)$の内積を
\begin{equation*}
  (a,b)\cdot(p,q) = ap + bq
\end{equation*}
と定義する

\br

\paragraph{内積の幾何的な定義}\quad

ベクトル$\overrightarrow{OA}$と$\overrightarrow{OB}$の内積を
\begin{equation*}
  \overrightarrow{OA}\cdot\overrightarrow{OB} = |\overrightarrow{OA}||\overrightarrow{OB}|\cos\theta
\end{equation*}
と定義する

ここで、$\theta$は$\overrightarrow{OA}$と$\overrightarrow{OB}$のなす角を表す

\sectionline

ベクトルの直交について、次の関係が成り立つ
\begin{align*}
  \overrightarrow{OA}\text{と}\overrightarrow{OB}\text{が直交する}
   & \Longleftrightarrow \cos\theta = 0                                  \\
   & \Longleftrightarrow \overrightarrow{OA}\cdot\overrightarrow{OB} = 0
\end{align*}

\section{直線と平面と空間}

\subparagraph{平面の中の直線を理解する}\quad

2次元平面の中の直線を、複数の見方でとらえてみる

\sectionline

\paragraph{直線の幾何的性質}

\begin{enumerate}
  \item 相異なる2点を通る直線は唯一つ存在する
  \item 与えられた点を通り、与えられた直線に平行な直線は唯一つ存在する
  \item 与えられた点を通り、与えられた直線に垂直な直線は唯一つ存在する
\end{enumerate}

\sectionline

直線の幾何的性質2を座標で表すと、次のような形になる

\br

\paragraph{直線のパラメータ表示}

\begin{equation*}
  (x,y) = (x_0,y_0) + t(a,b)
\end{equation*}

\br

$t=0$で与えられた点$(x_0,y_0)$を通り、$t$が実数全体を動くと、ベクトル$(a,b)\neq (0,0)$に平行な直線になる

$t$を\keyword{パラメータ(媒介変数)}という

\sectionline

直線の幾何的性質3を座標で表すとどうなるだろうか?

\br

点$(x_0,y_0)$を通り、位置ベクトル$(x,y)$--$(x_0,y_0)$がベクトル$(a,b)$に垂直であるような点$(x,y)$を集めて得られる図形が、幾何的性質3で記述した直線である

このように得られる直線に対し、$(a,b)$を\keyword{法線ベクトル}という

\br

2つのベクトルが直交するための必要十分条件は、その内積が0になることであるから、
\begin{gather}
  a(x-x_0) + b(y-y_0)  = 0 \\
  ax + by             = ax_0 + by_0
\end{gather}
$a,b$は与えられた直線の方向のデータ、$x_0,y_0$は与えられた点の座標であるから、右辺は定数である

\br

よって、$c=ax_0+by_0$とおくと、次の1次方程式の解$(x,y)$の描く図形が幾何的性質3で記述した直線となる

\br

\paragraph{直線の方程式による表示}

\begin{equation*}
  ax + by = c
\end{equation*}

\br

$b\neq 0$ならば、次のように傾きが$-\dfrac{a}{b}$、$y$切片が$\dfrac{c}{b}$の直線を表す式として表せる

\br

\paragraph{直線のグラフ表示}

\begin{equation*}
  y = -\frac{a}{b}x + \frac{c}{b}
\end{equation*}

\br

\sectionline

\subparagraph{3次元空間の中の平面を理解する}\quad

3次元空間の中で直線ではなく平面を記述したい場合について考える

\br

まず、直線の幾何的性質2は次のように言い換えられる

\br

\paragraph{平面の幾何的性質}

\begin{enumerate}[start=2]
  \item 与えられた点を通り、与えられた交差する2本の直線に平行な平面は唯一つ存在する
\end{enumerate}

\br

2つのベクトル$\overrightarrow{u}$と$\overrightarrow{v}$に対して、$a\overrightarrow{u}+b\overrightarrow{v}=\overrightarrow{0}$となるような実数$a$と$b$が$0$に限るとき、この$\overrightarrow{u}$と$\overrightarrow{v}$は\keyword{一次独立}であるという

\br

「交差する2本の直線に平行」と言うかわりに、「一次独立な2つのベクトルに平行」と言い換えても同じことである

\sectionline

3次元空間における2つの直線は、交わっていなくても対応する2つの方向ベクトルが垂直なとき、\keyword{垂直}と言うことにする

\br

そうすると、直線の幾何的性質3で述べた「与えられた点を通り、与えられた直線に垂直な直線」は、3次元空間の中では無数にあり、それを全部集めると平面になる

そのため、直線の幾何的性質3は、3次元空間の中では次の形に言い直すことになる

\br

\paragraph{平面の幾何的性質}

\begin{enumerate}[start=3]
  \item 与えられた点を通り、与えられた直線に垂直な平面は唯一つ存在する
\end{enumerate}

\br

逆に、空間の中に平面が先に与えられたとき、次の主張が成り立つ

\br

\paragraph{平面の幾何的性質}

\begin{enumerate}[label={\arabic*'},start=3]
  \item 与えられた点を通り、与えられた平面に垂直な直線は唯一つ存在する
\end{enumerate}
  
\br

平面の幾何的性質3と3'は、言葉だけを見ると「直線」と「平面」が入れ替わっている

この関係は\keyword{直交条件に関する双対性}として、ラグランジュの未定乗数法の背景となる

\sectionline

平面の幾何的性質3'を座標で表すことで、平面の方程式を導くことができる

\br

平面の幾何的性質3'において、あらかじめ与えられたデータは点$(x_0,y_0,z_0)$と\keyword{法線ベクトル}$(a,b,c) \neq (0,0,0)$である

そうすると、平面の幾何的性質3'で得られた平面上の任意の点$(x,y,z)$に対して、ベクトル$(a,b,c)$とベクトル$(x-x_0,y-y_0,z-z_0)$が直交することから、その内積は$0$、すなわち、
\begin{equation*}
  a(x-x_0) + b(y-y_0) + c(z-z_0) = 0
\end{equation*}

$d=ax_0+by_0+cz_0$とおくと、この平面の方程式は、
\begin{equation}
  ax + by + cz = d
\end{equation}
となることが証明できた

\br

\paragraph{平面の方程式}

\begin{equation*}
  ax + by + cz = d
\end{equation*}

$c\neq 0$ならば、次のように表せる

\begin{equation*}
  z = -\frac{a}{c}x - \frac{b}{c}y + \frac{d}{c}
\end{equation*}

\section{接線と接平面}

\subparagraph{平面内の曲線と接線}

1変数関数$f(x)$のグラフ$y=f(x)$上の1点$(x_0,y_0)$に接する直線$y=ax+b$をどのようにして求められるか考えてみる

\br

より一般の状況(関数のグラフに限らず、図形などの局所的な近似に応用する場合)を想定して、$ax+b$を$g(x)$と書いておく

\br

「接する」とは、点を共有して、傾きが同じということ

そのため、$y=f(x)$と$y=g(x)$のグラフが点$(x_0,y_0)$で接するための条件は、
\begin{enumerate}
  \item 点$(x_0,y_0)$を共有する:$f(x_0)=g(x_0) =y_0$
  \item 点$(x_0,y_0)$における傾きが等しい:$f'(x_0)=g'(x_0)$
\end{enumerate}
いま、$g(x)$が1次式$ax+b$だとすると、$g(x_0)=ax_0+b$、$g'(x_0)=a$であるから、この条件は、
\begin{enumerate}
  \item $f(x_0)=ax_0+b=y_0$
  \item $f'(x_0)=a$
\end{enumerate}
これを$a,b$について解くと、
\begin{align*}
  a & = f'(x_0) \\
  b & = y_0 - ax_0 = y_0 - f'(x_0)x_0
\end{align*}
よって、グラフ$y=f(x)$の点$(x_0,y_0) = (x_0,f(x_0))$における接線の方程式は、
\begin{align*}
  y & = g(x) = ax+b \\
    & = f'(x_0)x + \left(y_0 - f'(x_0)x_0\right) \\
    & = f'(x_0)(x-x_0) + y_0
\end{align*}
となる

\sectionline

\subparagraph{3次元内の曲線と接平面}

「曲面に対する接平面」は「曲線に対する接線」の次元を上げたものとみなすことができる

\br

関数$f(x,y)$を地点$(x,y)$の標高と考えると、この局面は地形を表しているとも解釈できる

山の斜面の1つの地点Pで接するような板(接平面)を数式で表すためには、どう考えればよいだろうか?

\sectionline

まず、2つの曲面$z=f(x,y)$と$z=g(x,y)$が1点Pで接するための条件を考える

接点Pの座標を$(x_0,y_0,z_0)$とすると、Pはグラフ上の点なので、$z_0=f(x_0,y_0)$である

\br

$z=f(x,y)$のグラフ(曲面)に$z=g(x,y)$のグラフ(曲面)が「接している」とすると、
\begin{enumerate}
  \item 点Pを共有する:$f(x_0,y_0)=g(x_0,y_0)=z_0$
  \item 点Pにおける曲面の傾きが等しい:(偏微分の言葉で表すと…?)
\end{enumerate}

そもそも、「曲面の傾きが等しい」とはどういう意味だろうか?

\sectionline

\subparagraph{「曲面の傾きが等しい」の意味}

ある地点の「傾き」というと、あらゆる方向の傾きを扱いたいわけだが、まずは、特別な方向の傾きを考えてみる

\br

たとえば、$x$が東西方向の座標で、$y$が南北方向の座標であるとする

\br

東西方向の傾斜は、$y$座標を$y=y_0$と一定にし、$x$を動かしたときの標高の変化率であるから、これはまさに$x$に関する偏微分である

したがって、それが一致するという条件は次の等式で表される
\begin{equation*}
  \dfrac{\partial f}{\partial x}(x_0,y_0) = \dfrac{\partial g}{\partial x}(x_0,y_0)
\end{equation*}

\br

同じように、南北方向の傾斜は、$x$座標を$x=x_0$と一定にし、$y$を動かしたときの標高の変化率であるから、それが一致するという条件は、$y$に関する偏微分が一致するという条件になる
\begin{equation*}
  \dfrac{\partial f}{\partial y}(x_0,y_0) = \dfrac{\partial g}{\partial y}(x_0,y_0)
\end{equation*}

\sectionline

\subparagraph{接平面の方程式}

以上をまとめると、2つの曲面$z=f(x,y)$と$z=g(x,y)$が点Pで接するならば、
\begin{align}
  f(x_0,y_0) &= g(x_0,y_0) =z_0 \\
  \dfrac{\partial f}{\partial x}(x_0,y_0) &= \dfrac{\partial g}{\partial x}(x_0,y_0) \\
  \dfrac{\partial f}{\partial y}(x_0,y_0) &= \dfrac{\partial g}{\partial y}(x_0,y_0)
\end{align}
が成り立つ

\br

$g(x,y)$が1次式$ax+by+d$で表されるとすると、上述の3つの条件式は、
\begin{align*}
  f(x_0,y_0) &= ax_0+by_0+d = z_0 \\
  \dfrac{\partial f}{\partial x}(x_0,y_0) & = a \\
  \dfrac{\partial f}{\partial y}(x_0,y_0) & = b
\end{align*}
これを$a,b,d$について解くと、
\begin{align*}
  a & = \dfrac{\partial f}{\partial x}(x_0,y_0) \\
  b & = \dfrac{\partial f}{\partial y}(x_0,y_0) \\
  d & = z_0 - a x_0 - b y_0 \\
    & = z_0 - \dfrac{\partial f}{\partial x}(x_0,y_0)x_0 - \dfrac{\partial f}{\partial y}(x_0,y_0)y_0
\end{align*}
となるので、$a,b,d$の値はすべて決まり、点Pにおける\keyword{接平面}の方程式は、
\begin{multline*}
  z = ax + by + d \\
    = \dfrac{\partial f}{\partial x}(x_0,y_0)x + \dfrac{\partial f}{\partial y}(x_0,y_0)y \\ + \left(z_0 - \dfrac{\partial f}{\partial x}(x_0,y_0)x_0 - \dfrac{\partial f}{\partial y}(x_0,y_0)y_0\right) \\
    = \dfrac{\partial f}{\partial x}(x_0,y_0)(x-x_0) + \dfrac{\partial f}{\partial y}(x_0,y_0)(y-y_0) + z_0
\end{multline*}
という形になる

\br

つまり、$z=f(x_0,y_0)$、$\dfrac{\partial f}{\partial x}(x_0,y_0)$、$\dfrac{\partial f}{\partial y}(x_0,y_0)$の値だけで、接平面が決まってしまうということである

これは、東西方向と南北方向の勾配の情報だけで、曲面に接する「板」の傾きが決まることに対応している

\sectionline

\paragraph{接平面の方程式(その1)}

$z_0=f(x_0,y_0)$とすると、グラフ$z=f(x,y)$上の点$(x_0,y_0,z_0)$における接平面は次の方程式で与えられる
\begin{equation*}
  z = \dfrac{\partial f}{\partial x}(x_0,y_0)(x-x_0) + \dfrac{\partial f}{\partial y}(x_0,y_0)(y-y_0) + z_0
\end{equation*}

\sectionline

変数が3個以上ある場合、グラフを描こうとすると、空間の次元が$4$以上になってしまい、目で見るのが難しくなってしまう

このようなときは、次元に依存しない概念に置き換えるのが便利である

\br

以前、三角形は地面に描かれていても空中に浮かんでいても、その幾何的な性質が変わらないことに着目し、「ベクトルの内積は座標や次元によらない概念である」ことを示した

そこで、ベクトルの内積を用いて、接平面の方程式をより汎用的な形に書き換える

\br

以下のように$a,b,p,q$を当てはめると、
\begin{align*}
  a & = \dfrac{\partial f}{\partial x}(x_0,y_0) \\
  b & = \dfrac{\partial f}{\partial y}(x_0,y_0) \\
  p & = x-x_0 \\
  q & = y-y_0
\end{align*}
接平面の方程式は、ベクトル$(a,b)$とベクトル$(p,q)$の内積となる
\begin{equation*}
  z - z_0 = ap + bq
\end{equation*}

ここで、ベクトル$(a,b)$と$(p,q)$の意味を幾何的に考えてみる

\br

\subparagraph{勾配ベクトルによる表現}\quad

関数$f(x,y)$の\keyword{勾配ベクトル}を
\begin{equation*}
  \nabla f(x,y) = \mathrm{grad} f(x,y) = \left(\dfrac{\partial f}{\partial x},\dfrac{\partial f}{\partial y}\right)
\end{equation*}
と定義する

同様に、$n$変数関数$g(x_1,x_2,\ldots,x_n)$の勾配ベクトルは、$n$次元ベクトルとして定義される

\br

勾配ベクトルは多変数関数の局所的な振る舞いに関する重要な情報を持っている

\br

勾配ベクトルの定義を踏まえると、
\begin{itemize}
  \item $(a,b)$は勾配ベクトル$\nabla f(x_0,y_0)$
  \item $(p,q)$は位置ベクトル$(x-x_0,y-y_0)$
\end{itemize}
と解釈できる

よって、接平面の方程式は、勾配ベクトルと位置ベクトルとの内積として、次のように書き表せる

\sectionline

\paragraph{接平面の方程式(その2)}

\begin{equation*}
  z = z_0 + \nabla f(x_0,y_0)\cdot(x-x_0,y-y_0)
\end{equation*}

\sectionline

\subparagraph{局所的な近似式}

接平面は、接点$(x_0,y_0,z_0) = (x_0,y_0,f(x_0,y_0))$を通り、この点で曲面$z=f(x,y)$と「接している」平面である

\br

したがって、点$(x,y)$が$(x_0,y_0)$に近いとき、2変数関数$f(x,y)$の値は、接平面上の$z$座標である次の値
\begin{equation*}
  f(x_0,y_0) + \nabla f(x_0,y_0)\cdot(x-x_0,y-y_0)
\end{equation*}
で近似できることになる

この式は、展開すると$x,y$の1次式となるので、$f(x,y)$の\keyword{一次近似}あるいは\keyword{線形近似}と言う

\br

内積を用いると、変数の個数が何個であっても、多変数関数の局所的な近似式を同じ形で表示できる

\end{document}
