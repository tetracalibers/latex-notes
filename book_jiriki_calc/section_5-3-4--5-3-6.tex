\documentclass[../book_jiriki_calc]{subfiles}

\begin{document}

\section{平面の幾何 --- 余弦定理}

2つの地点の間の距離を知りたくても、その間に山や池があって直接距離を測るのが難しいことがある

こんなとき、第3の地点からの角度を含めた測量データがあれば、知りたい距離を計算できることがある

\sectionline

AB間の距離$n$は、別の2つの辺の長さ$l,\,m$とその間の角度$\theta$から計算できる(AとBの間に池があってもかまわない)

この計算方法を与えるのが\keyword{余弦定理}である
\begin{equation*}
  n^2 = l^2 + m^2 - 2lm\cos\theta
\end{equation*}
この式を書き換えると、3辺の情報から、角度を知る公式にもなる
\begin{equation*}
  \cos\theta = \frac{l^2 + m^2 - n^2}{2lm}
\end{equation*}

\vskip\baselineskip

余弦定理は、三平方の定理(ピタゴラスの定理)を拡張した定理と見なせる

実際、$\theta = 90^\circ$のとき$\cos\theta = 0$であるから、余弦定理はピタゴラスの定理に一致する
\begin{equation*}
  n^2 = l^2 + m^2
\end{equation*}

\sectionline

余弦定理を座標で書き表してみる

まず、この3頂点が$xy$平面にあるものと思って、ベクトルで表示する
\begin{align*}
  \overrightarrow{OA} & = (a,b)     \\
  \overrightarrow{OB} & = (p,q)     \\
  \overrightarrow{BA} & = (a-p,b-q)
\end{align*}
とおくと、各辺の長さは、
\begin{align*}
  l & = \sqrt{a^2 + b^2}         \\
  m & = \sqrt{p^2 + q^2}         \\
  n & = \sqrt{(a-p)^2 + (b-q)^2}
\end{align*}
と表せる

これを使って、$\cos\theta = \dfrac{l^2 + m^2 - n^2}{2lm}$の分子を計算すると、多くの項が打ち消し合って、
\begin{multline*}
  l^2 + m^2 - n^2 \\ = a^2 + b^2 + p^2 + q^2 - ((a-p)^2 - (b-q)^2)
  \\ = 2ap + 2bq
\end{multline*}
となるので、余弦定理は、
\begin{align}
  \cos\theta               & = \dfrac{2(ap+bq)}{2lm} \\
  \therefore \quad ap + bq & = lm\cos\theta
\end{align}
と書き換えられる

\sectionline

この式の両辺をあらためて観察してみる

\begin{equation*}
  ap + bq = lm\cos\theta
\end{equation*}

左辺に現れる$a, b, p, q$という数は、座標系を決めないと値が定まらない

たとえば、三角形が地面に描かれているとする

地面の上なので、好きな点を原点にとり、好きな方向を$x$軸に選び、それと垂直に$y$軸を定める

そうするとベクトル$\overrightarrow{OA}$や$\overrightarrow{OB}$の$x$成分や$y$成分である$a, b, p, q$の値が定まるが、別の座標系をとれば、ベクトルの成分$a, b, p, q$は別の値になる

\vskip\baselineskip

しかし、右辺は、辺の長さや角度という三角形の幾何に固有な量だけで表されている

$ap+bq$は、どんな座標系でも同じ値になる、\keyword{三角形に内在的な量}なのだ

\vskip\baselineskip

この重要な量$ap+bq$を、2つのベクトル$\overrightarrow{OA}$と$\overrightarrow{OB}$の\keyword{内積}あるいは\keyword{スカラー積}あるいは\keyword{ドット積}という

$ap+bq$は、\keyword{座標系のとり方に依存しない}がゆえに重要な量である

\sectionline

次の2つの定義が一致するというのが、
\begin{equation*}
  ap + bq = lm\cos\theta
\end{equation*}
という等式の意味である

\vskip\baselineskip

\paragraph{内積の座標による定義}\quad

ベクトル$(a,b)$と$(p,q)$の内積を
\begin{equation*}
  (a,b)\cdot(p,q) = ap + bq
\end{equation*}
と定義する

\vskip\baselineskip

\paragraph{内積の幾何的な定義}\quad

ベクトル$\overrightarrow{OA}$と$\overrightarrow{OB}$の内積を
\begin{equation*}
  \overrightarrow{OA}\cdot\overrightarrow{OB} = |\overrightarrow{OA}||\overrightarrow{OB}|\cos\theta
\end{equation*}
と定義する

ここで、$\theta$は$\overrightarrow{OA}$と$\overrightarrow{OB}$のなす角を表す

\sectionline

ベクトルの直交について、次の関係が成り立つ
\begin{align*}
  \overrightarrow{OA}\text{と}\overrightarrow{OB}\text{が直交する}
   & \Longleftrightarrow \cos\theta = 0                                  \\
   & \Longleftrightarrow \overrightarrow{OA}\cdot\overrightarrow{OB} = 0
\end{align*}

\section{直線と平面と空間}

\subparagraph{平面の中の直線を理解する}\quad

2次元平面の中の直線を、複数の見方でとらえてみる

\sectionline

\paragraph{直線の幾何的性質}

\begin{enumerate}
  \item 相異なる2点を通る直線は唯一つ存在する
  \item 与えられた点を通り、与えられた直線に平行な直線は唯一つ存在する
  \item 与えられた点を通り、与えられた直線に垂直な直線は唯一つ存在する
\end{enumerate}

\sectionline

直線の幾何的性質2を座標で表すと、次のような形になる

\vskip\baselineskip

\paragraph{直線のパラメータ表示}

\begin{equation*}
  (x,y) = (x_0,y_0) + t(a,b)
\end{equation*}

\vskip\baselineskip

$t=0$で与えられた点$(x_0,y_0)$を通り、$t$が実数全体を動くと、ベクトル$(a,b)\neq (0,0)$に平行な直線になる

$t$を\keyword{パラメータ(媒介変数)}という

\sectionline

直線の幾何的性質3を座標で表すとどうなるだろうか?

\vskip\baselineskip

点$(x_0,y_0)$を通り、位置ベクトル$(x,y)$--$(x_0,y_0)$がベクトル$(a,b)$に垂直であるような点$(x,y)$を集めて得られる図形が、幾何的性質3で記述した直線である

このように得られる直線に対し、$(a,b)$を\keyword{法線ベクトル}という

\vskip\baselineskip

2つのベクトルが直交するための必要十分条件は、その内積が0になることであるから、
\begin{gather}
  a(x-x_0) + b(y-y_0)  = 0 \\
  ax + by             = ax_0 + by_0
\end{gather}
$a,b$は与えられた直線の方向のデータ、$x_0,y_0$は与えられた点の座標であるから、右辺は定数である

\vskip\baselineskip

よって、$c=ax_0+by_0$とおくと、次の1次方程式の解$(x,y)$の描く図形が幾何的性質3で記述した直線となる

\vskip\baselineskip

\paragraph{直線の方程式による表示}

\begin{equation*}
  ax + by = c
\end{equation*}

\vskip\baselineskip

$b\neq 0$ならば、次のように傾きが$-\dfrac{a}{b}$、$y$切片が$\dfrac{c}{b}$の直線を表す式として表せる

\vskip\baselineskip

\paragraph{直線のグラフ表示}

\begin{equation*}
  y = -\frac{a}{b}x + \frac{c}{b}
\end{equation*}

\vskip\baselineskip

\sectionline

\subparagraph{3次元空間の中の平面を理解する}\quad

3次元空間の中で直線ではなく平面を記述したい場合について考える

\vskip\baselineskip

まず、直線の幾何的性質2は次のように言い換えられる

\vskip\baselineskip

\paragraph{平面の幾何的性質}

\begin{enumerate}[start=2]
  \item 与えられた点を通り、与えられた交差する2本の直線に平行な平面は唯一つ存在する
\end{enumerate}

\vskip\baselineskip

2つのベクトル$\overrightarrow{u}$と$\overrightarrow{v}$に対して、$a\overrightarrow{u}+b\overrightarrow{v}=\overrightarrow{0}$となるような実数$a$と$b$が$0$に限るとき、この$\overrightarrow{u}$と$\overrightarrow{v}$は\keyword{一次独立}であるという

\vskip\baselineskip

「交差する2本の直線に平行」と言うかわりに、「一次独立な2つのベクトルに平行」と言い換えても同じことである

\sectionline

3次元空間における2つの直線は、交わっていなくても対応する2つの方向ベクトルが垂直なとき、\keyword{垂直}と言うことにする

\vskip\baselineskip

そうすると、直線の幾何的性質3で述べた「与えられた点を通り、与えられた直線に垂直な直線」は、3次元空間の中では無数にあり、それを全部集めると平面になる

そのため、直線の幾何的性質3は、3次元空間の中では次の形に言い直すことになる

\vskip\baselineskip

\paragraph{平面の幾何的性質}

\begin{enumerate}[start=3]
  \item 与えられた点を通り、与えられた直線に垂直な平面は唯一つ存在する
\end{enumerate}

\vskip\baselineskip

逆に、空間の中に平面が先に与えられたとき、次の主張が成り立つ

\vskip\baselineskip

\paragraph{平面の幾何的性質}

\begin{enumerate}[label={\arabic*'},start=3]
  \item 与えられた点を通り、与えられた平面に垂直な直線は唯一つ存在する
\end{enumerate}
  
\vskip\baselineskip

平面の幾何的性質3と3'は、言葉だけを見ると「直線」と「平面」が入れ替わっている

この関係は\keyword{直交条件に関する双対性}として、ラグランジュの未定乗数法の背景となる

\sectionline

平面の幾何的性質3'を座標で表すことで、平面の方程式を導くことができる

\vskip\baselineskip

平面の幾何的性質3'において、あらかじめ与えられたデータは点$(x_0,y_0,z_0)$と\keyword{法線ベクトル}$(a,b,c) \neq (0,0,0)$である

そうすると、平面の幾何的性質3'で得られた平面上の任意の点$(x,y,z)$に対して、ベクトル$(a,b,c)$とベクトル$(x-x_0,y-y_0,z-z_0)$が直交することから、その内積は$0$、すなわち、
\begin{equation*}
  a(x-x_0) + b(y-y_0) + c(z-z_0) = 0
\end{equation*}

$d=ax_0+by_0+cz_0$とおくと、この平面の方程式は、
\begin{equation}
  ax + by + cz = d
\end{equation}
となることが証明できた

\vskip\baselineskip

\paragraph{平面の方程式}

\begin{equation*}
  ax + by + cz = d
\end{equation*}

$c\neq 0$ならば、次のように表せる

\begin{equation*}
  z = -\frac{a}{c}x - \frac{b}{c}y + \frac{d}{c}
\end{equation*}

\section{接線と接平面}



\end{document}
