\documentclass[../book_jiriki_calc]{subfiles}

\begin{document}

\section*{近似と誤差}

曲線を局所的に近似する場合、接線で近似するのが最初のステップになる

次のステップとして、接線からの乖離を正確に知りたい

たとえば道路が急カーブしているときは、直線ではなく円弧で近似する方がより正確になる

関数のグラフの各点で、その曲がり方を表す円弧の半径を求めるのには2階の微分を使う

\vskip\baselineskip

実用上は、1階微分と2階微分を用いると、多くの場合、局所的に十分良い近似ができるが、それでも微小な乖離は生じる

この微小な誤差は、3階微分を使うと評価できる

これを続け、1階微分だけではなく、2階、3階、$\ldots$と高階の微分を用い、必要な精度を実現するためには近似をどのように行えばよいかを指し示すのが\keyword{テイラー展開}とその\keyword{剰余項}である

\sectionline

誤差評価を行う際には、範囲をきちんと意識する必要がある
\begin{itemize}
  \item 翌日の天気が予測できても、1ヶ月先の天気予報は難しい
  \item 坂道の勾配を見て100m先の高低差は推測できても、10km先の高低差はわからない
\end{itemize}

\section*{誤差と誤差率}

測定値や何らかの概算値が真の値とどれくらい異なるかは、
\begin{equation*}
  \text{誤差} = \left|\text{測定値} - \text{真の値}\right|
\end{equation*}
という絶対量で表された

\vskip\baselineskip

一方、相対的な比率として定義される、
\begin{equation*}
  \text{誤差率} = \left|\frac{\text{誤差}}{\text{真の値}}\right| = \left|\frac{\text{測定値} - \text{真の値}}{\text{真の値}}\right|
\end{equation*}
も大事な視点である

\vskip\baselineskip

実用上は、分母を「測定値」に取り換えた、
\begin{equation*}
  P = \left|\frac{\text{誤差}}{\text{測定値}}\right| = \left|\frac{\text{測定値} - \text{真の値}}{\text{測定値}}\right|
\end{equation*}
で代用することもある

\vskip\baselineskip

$P$が小さいときは誤差率として代用できることは、次のように確認できる

\sectionline

\paragraph{定理}

$P<\dfrac{1}{101}$ならば、誤差率は$1\%$未満

\sectionline

\subparagraph{証明}

$t=\dfrac{\text{真の値}}{\text{測定値}}$とおくと、
\begin{align*}
  \text{誤差} & = \left|\text{測定値} - \text{真の値}\right|  \\
            & = \left|\text{測定値} - t\text{測定値}\right| \\
            & = \left|1-t\right|\text{測定値}
\end{align*}
より、
\begin{gather}
  P          = \left|\frac{\text{測定値} - \text{真の値}}{\text{測定値}}\right| = \left|1-t\right| \\
  \text{誤差率} = \left|\frac{\text{誤差}}{\text{真の値}}\right| = \left|\frac{\left|1-t\right|\text{測定値}}{\text{真の値}}\right| =\left|\dfrac{1-t}{t}\right|
\end{gather}
と書き表せる

\vskip\baselineskip

$P<\dfrac{1}{101}$ならば、
\begin{align*}
  -P  & > -\frac{1}{101}                    \\
  1-P & > 1-\frac{1}{101} = \frac{100}{101}
\end{align*}
であり、$t$について、三角不等式より、
\begin{gather}
  1 - \left| 1-t\right| = 1 - \left| t-1\right| \leq \left| 1+ (t-1)\right| = t \\
  \dfrac{100}{101} < 1-P \leq t
\end{gather}
これを用いると、
\begin{align*}
  \text{誤差率} & = \left|\frac{1-t}{t}\right|                                                                                                                                        \\
             & < \left|\dfrac{1-\dfrac{100}{101}}{\dfrac{100}{101}}\right| = \dfrac{\dfrac{1}{101}}{\dfrac{100}{101}} = \dfrac{\dfrac{1}{101}\cdot 101}{\dfrac{100}{101}\cdot 101} \\
             & = \dfrac{1}{100}
\end{align*}
として、誤差率は$1\%$未満であることが示された$\qed$

\sectionline

真の値が分からなくとも、何か別の情報や論理から、誤差を「上から評価する」すなわち、「誤差が〜以下である」という形の評価式が得られることがある

\section*{弧長の近似と誤差評価}

一般に、ある時点で誤差が生じると、その後の誤差が増幅して予期しない間違いが生じることがある

したがって、概算が信頼できるとするためには、\keyword{誤差評価}という別の論理が必要になる

誤差評価を行う際に用いるトリックが、\keyword{存在定理}である

\section*{中間値の定理}

たとえば、今朝7時の気温が$22^\circ$Cで、正午には$30^\circ$Cに上がったとすると、午前中に$27^\circ$Cになる瞬間が必ずある

これが\keyword{中間値の定理}である

\sectionline

\paragraph{定理:中間値の定理}

$a \leq x \leq b$で定義された連続関数$f(x)$を考える

$f(a)$と$f(b)$の間にある任意の実数$T$を1つ選ぶと、$f(c)=T$となる実数$c$が$a$と$b$の間に必ず存在する

\sectionline

\subparagraph{例:時刻と気温}

$f(x)$が時刻$x$における気温を表すとすると、気温は時刻が経過するとともに連続的に動くため、$f(x)$は連続関数である

$a$を7時、$b$を12時とすると、$f(a)=22^\circ$C、$f(b)=30^\circ$Cであり、$T=27$は$22\leq T\leq 30$を満たしている

中間値の定理は、$f(c)=27^\circ$Cとなる時刻$c$が7時から12時の間に必ず存在するということを述べている

\sectionline

いつかは分からないけれど、「$27^\circ$Cになる瞬間があったのは確かである」

このようなタイプの定理を\keyword{存在定理}という

7時から正午まで、一度も温度計を見ていなくても、その間の情報が皆無ではないということになる

\vskip\baselineskip

探し物をするときでも、「この部屋にあるかどうかすらわからない」と思って探すのと、「この部屋にあることは確実だ」と信じて探すのでは大きな差がある

\keyword{どこかには確実に存在する}という存在定理は、上手く使うと決定的な証拠になることがある

\sectionline

\subparagraph{例:ゴムひもの動かない点}

たとえば1本のゴムひもを両手で持って、左右に引っ張るとする

そうすると、ゴムひもの中で、まったく動かない点が必ず存在する

左右均等に引っ張れば、真ん中の点が動かない

左右均等に引っ張らなくても、必ず動かない点がある

\sectionline

このことは、中間値の定理から説明できる

\vskip\baselineskip

ゴムひもを数直線上に置き、左端と右端の座標をそれぞれ$a,\,b$とする

ゴムひも内のある点の座標を$x$とすると、$a\leq x\leq b$である

\vskip\baselineskip

両手の間隔を広げたとき、この点の行き先の座標を$g(x)$とすると、
\begin{itemize}
  \item 左端は元の位置よりも左に動くので、$g(a)<a$
  \item 右端は元の位置よりも右に動くので、$g(b)>b$
\end{itemize}

そこで、$f(x)=g(x)-x$とおくと、次の不等式が成り立つ
\begin{align*}
  f(a) & = g(a) - a < 0 \\
  f(b) & = g(b) - b > 0
\end{align*}
そうすると、中間値の定理より、$f(c)=0$となる$c$が$a$と$b$の間に存在する

\vskip\baselineskip

$f(c)=0$というのは、$g(c)=c$、つまり動かした後の座標$g(c)$と元の座標$c$が一致するということなので、$c$は動かない点である

\vskip\baselineskip

こうして、手を広げたとき、ゴムひもの中で必ず動かない点があることが示された

\sectionline

動かない点があることを保証する定理を\keyword{不動点定理}という

不動点定理は存在定理の一種であり、「どこに不動点があるのか」は明示しないが、「どこかにある」ことを保証する

経済学やゲーム理論でも、均衡した状態がどこかに存在するということが、不動点定理から説明できることがある

\section*{平均値の定理}

微分を含んだ存在定理として、\keyword{平均値の定理}がある

\sectionline

\paragraph{定理:平均値の定理}

$f(x)$は$a \leq x \leq b$で定義された関数で、微分可能とする

このとき、
\begin{equation*}
  f'(c) = \frac{f(b) - f(a)}{b - a}
\end{equation*}
となる$c$が$a$と$b$の間に存在する

\sectionline

点Pの座標を$(a,\,f(a))$、点Qの座標を$(b,\,f(b))$とすると、$\dfrac{f(b) - f(a)}{b - a}$は線分PQの傾きである

$f'(c)$は接線の傾きなので、平均値の定理は、線分PQと平行な接線が必ずある、と述べている

\sectionline

\subparagraph{例:速度の平均値}

オリンピックの100m走で、ぴったり10秒で走り切った選手がいるとする

この選手の10秒間の平均速度は秒速10mであり、この平均速度は線分PQの傾きに対応する

\vskip\baselineskip

もちろん、この選手は10秒間同じスピードで走っているわけではない

加速や減速の数値はわからなくても、ぴったり秒速10mになった瞬間がこの10秒の中に必ず存在することは保証する、というのが平均値の定理の意味になる

\sectionline

\paragraph{定理(再掲)}

$f'(x)=0$がすべての$x$で成り立てば、関数$f(x)$は定数である

\vskip\baselineskip

\subparagraph{証明}

$f'(x)=0$がすべての$x$で成り立てば、平均値の定理の左辺$f'(c)$が$0$となるので、
\begin{align*}
  \dfrac{f(b) - f(a)}{b - a} & = 0 \\\\
  f(b) - f(a)                & = 0 \\
\end{align*}
となり、$f(a)=f(b)$がすべての実数$a,\,b$に対して成り立つことがわかる

したがって、$f(x)$は定数である$\qed$

\sectionline

\subparagraph{平均値の定理の証明}\quad

平均値の定理は、「閉区間上の連続関数が最大値・最小値をとる」ことを用いて証明できる

\begin{equation*}
  g(x) = f(x) - Ax , \quad A    = \dfrac{f(b) - f(a)}{b - a}
\end{equation*}
とおくと、$g(x)$は$a \leq x \leq b$で最大値・最小値をとる

一方、$A$の選び方から、$g(a)=g(b)$である
\begin{align*}
  g(a) - g(b) & = \left(f(a) - Aa\right) - \left( f(b) - Ab \right) \\
              & = f(a) - f(b) - A(b-a)                              \\
              & = f(a) - f(b) - \dfrac{f(b) - f(a)}{b - a}(b-a)     \\
              & = 0
\end{align*}
したがって、$g(x)$の最大値または最小値のうち、少なくとも一方は$a \leq x \leq b$の両端$a,\,b$とは異なる点$c$で実現される

ここで、

\vskip\baselineskip

\paragraph{定理(再掲)}

$a < x < b$で定義された、微分可能な関数$f(x)$が$x=c$で最大値または最小値を取るならば、$f'(c)=0$である

\vskip\baselineskip

という定理を思い出すと、$a < c < b$なので$g'(c) = 0$となる
\begin{align}
  g'(c) = f'(c) - A = 0
\end{align}
より、$f'(c) = A$が示された$\qed$

\sectionline

\begin{itemize}
  \item 中間値の定理は、微分は無関係で、連続関数のとる値に関する存在定理
  \item 平均値の定理は、1階の微分に関する存在定理
\end{itemize}

もう一歩踏み込んで、2階の微分、3階の微分、$\ldots$と高階の微分に関する存在定理を考えていくと、次に述べるテイラー展開の定理になる


\end{document}
