\documentclass[../book_jiriki_calc]{subfiles}

\begin{document}

\section{大きな数を感覚的に捉える方法}

分子も分母も無限大に近づく中で、分母に比べて分子を無視できるという等式

\begin{equation}
  \lim_{n\to \infty }\frac{n^2}{n^3}=0
\end{equation}

分子を1辺$n$の正方形の面積(2次元)、分母を1辺$n$の立方体の体積(3次元)とみなせば、「次数が高くなると巨大な数が現れやすい」という性質の1つの姿と思うこともできる

\section{収束や発散の速さ}

$a_1, a_2, a_3, \cdots$という数列があったときに、その初めの$N$個の和を次のように表す

\begin{equation}
  \sum_{k=1}^{N}a_k = a_1 + a_2 + a_3 + \cdots + a_N
\end{equation}

2個や3個の和なら$a_1 + a_2$や$a_1 + a_2 + a_3$と書けるが、たとえば$N$が$10^{16}$というようなときには、すべての項を書き出せないため、このように記法を決めておく

\sectionline

等比級数$A_N$の収束の速さを考える

\begin{equation}
  A_N = \sum_{k=1}^{N}\dfrac{1}{2^k} = \dfrac{1}{2} + \dfrac{1}{4} + \cdots + \dfrac{1}{2^N}
\end{equation}

この和に$\dfrac{1}{2^N}$を足すと、$1$になる

たとえば、
\begin{equation}
  \dfrac{1}{2} + \dfrac{1}{4} + \dfrac{1}{8} + \dfrac{1}{8} = 1
\end{equation}

なので、$N$個の総和は、
\begin{equation}
  1 - \dfrac{1}{2^N}
\end{equation}

$\dfrac{1}{2^N}$がどれくらい小さいかがわかれば、$A_N$の和が$1$にどれくらい近いかがわかる

\br

では、$N=10^{16}$のとき、$\dfrac{1}{2^N}$はどれくらいの大きさか?

まず、$N=10$の場合を考えると、
\begin{equation}
  2^{10} = 1024 \fallingdotseq 1000
\end{equation}
より、
\begin{align}
  \dfrac{1}{2^{10}}     & \fallingdotseq 0.001 \\
  1 - \dfrac{1}{2^{10}} & \fallingdotseq 0.999
\end{align}
というふうに、小数点以下に$9$が3つ連続して並ぶ

$N=10^2$のときは、
\begin{equation}
  2^{100} = (2^{10})^{10} = (1024)^{10} \fallingdotseq (1000)^{10} = 10^{30}
\end{equation}
から、$9$がおよそ30個並ぶ($1024 > 1000$なので、「少なくとも」30個並ぶ)

$N=10^{16}$のときは、
\begin{equation}
  2^{10^{16}} = (2^{10})^{10^{15}} \fallingdotseq 10^{30 \times 10^{15}}
\end{equation}
から、$3 \times 10^{15} = 3000$兆以上の$9$が並ぶ

このように、$N$を大きくしていくと、$2^N$はきわめて$1$に近い数になる

$N \to \infty$としたときに、等比級数$A_N$は$1$に収束する

\section{誤差評価}

測定や推定で何らかの値を得たとき、\keyword{誤差}は次のように定義される
\begin{equation}
  \text{誤差} = \left| \text{測定値 or 推定値} - \text{真の値} \right|
\end{equation}

誤差が論理的に小さいと言えれば、その測定値や推定値は一定の安心感を持って\keyword{近似値}として使うことができる

\br

誤差がある数$\varepsilon$より小さくなるという不等式
\begin{equation}
  \left| \text{測定値 or 推定値} - \text{真の値} \right| < \varepsilon
\end{equation}
を\keyword{誤差評価}という

\br

$N$が$10^{16}$のとき、右辺は小数点以下に少なくとも$3\times 10^{15}$個の$0$が並ぶ
\begin{equation}
  \left| A_N - 1 \right| < 0.00\ldots 0 \ldots
\end{equation}
この不等式は、$N=10^{16}$のときの$A_N$が、極限値である$1$をどの程度の精度で近似しているかを表す誤差評価と考えることもでき、この誤差評価は、相対的な収束の速さを数値的に表すものでもある(右辺の小数点以下に並ぶ$0$の数で速さがわかる)

\end{document}
