\documentclass[../book_jiriki_calc]{subfiles}

\begin{document}

\section{グラフを描いて2変数関数を見る}

2変数関数$f(x, y)$が与えられたとき、変数$x,\,y$を自由に動かして点$(x, y, f(x,y))$を$xyz$空間でプロットして得られる曲面を\keyword{$z= f(x, y)$のグラフ}という。

\vskip\baselineskip

$f(x,y)$が地点$(x, y)$の標高の場合は、この$z=f(x,y)$のグラフが表す曲面はこの野山の地表にほかならない。
\begin{itemize}
  \item 2変数関数$f(x, y)$をグラフで可視化すると、野山の形状になる
  \item 野山の形状から標高を考えると、2変数関数$f(x, y)$になる
\end{itemize}

\sectionline

\subparagraph{$f(x,y)=x^2+y^2$のグラフ}\quad

このグラフは、壺のような形になっている

この形状を、2通りの見方で理解してみる

\vskip\baselineskip

\subparagraph{断面図}

「曲面を見る」堅実な方法は、\keyword{断面図(切り口)}を順に見ること

\begin{itemize}
  \item $y=0$とすると、断面が$xz$平面内の放物線$z=x^2$になる
  \item $y=1$とすると、$z=x^2+1$となり、これは$z=x^2$のグラフを$1$だけ高くした放物線
  \item $y=2$とすると、$z=x^2+4$となり、放物線がさらに高くなる
\end{itemize}

こうして、$y=\text{定数}$とした断面図をつなぎ合わせることで曲面の姿をつかむことができる

\vskip\baselineskip

\subparagraph{回転対称性}

\keyword{回転対称性}を利用するという巧妙な方法もある

三平方の定理から、$x^2+y^2$は原点から点$(x, y)$までの距離の2乗

したがって、点$(x, y)$が原点を中心とする半径$R$の円周上にあれば、$f(x,y)$の値はいつでも$R^2$であり、$z=f(x,y)$のグラフは$z$軸に関して回転させても形が変わらない曲面になっている

\vskip\baselineskip

このようにして、$z=x^2+y^2$のグラフが放物線を$z$軸に関してぐるっと回した壺のような曲面になっている様子が見えてくる

\sectionline

\subparagraph{$f(x,y)=x^2-y^2$のグラフ}\quad

断面図(切り口)を順に見ていく

\vskip\baselineskip

$xz$平面における断面は、$y=0$を代入すればわかる

すると$z=f(x,0)=x^2$で、$xz$平面において下に凸の放物線になる

\vskip\baselineskip

今度は$x$を止めて、$y$を動かす

$xyz$空間の中で、$x=\text{定数}$は$yz$平面に平行な平面となる

たとえば$x=0,1,2,\cdots$とすると、順に$z=-y^2, z=1-y^2, z=4-y^2,\cdots$となり、いずれも上に凸の放物線になる

\vskip\baselineskip

下に凸の放物線(吊り下げたひも)の各点に、上に凸の放物線(針金)を順に貼り付けていくと、$z=x^2-y^2$のグラフが得られる

\sectionline

曲面$z=x^2-y^2$の局所的な形状は、身近なところにもあちこちに現れている

\vskip\baselineskip

日本の数学用語では、このグラフの原点を、山の峠にちなんで\keyword{峠点}とよぶ

西洋では、乗馬にちなんで\keyword{鞍点}とよぶ

乗馬するとき、馬の背の凹んでいるところに鞍を置いてまたがるが、その形状は峠とそっくり

\vskip\baselineskip

山が連なっているような山脈を越えて向こう側に行きたいとすると、できるだけ登りが少ない経路を選ぶだろう

このような往来によって踏み固められてできた道が「山脈越えの道」(グラフでは$x=0$の場合の放物線)

\vskip\baselineskip

旅人が山脈越えの道を登っていくと、峠はその道沿いではいちばん高い地点になっている

峠で左右を見ると、山が続いている($x=\text{定数}$の場合の放物線)

いま登ってきた山脈越えの道と垂直に交わっている尾根道があるかもしれない(尾根道は$y=0$の場合の放物線)

尾根道沿いに歩けば、峠はその前後ではいちばん低い場所になっている

\vskip\baselineskip

これは特定の峠の話ではなく、峠の形状の普遍的な性質を示している

\section{等高線を用いて2変数関数を見る}

等高線は、2変数関数の増減を手軽に紙の上で可視化する方法

\vskip\baselineskip

2変数の関数$f(x, y)$が与えられたとき、
\begin{equation*}
  f(x, y) = \text{定数}
\end{equation*}
となるような点$(x, y)$を集めて得られる曲線を\keyword{等高線}という

\sectionline

2次元空間を可視化する方法をまとめると、

\begin{description}
  \item[グラフ(曲面)を立体模型で表す]
  \item[グラフを鳥瞰図として平面に書く] 3次元空間の中で視点を1つ選んで、$z=f(x,y)$の曲面を「風景」として平面に描く
  \item[$f(x,y)$の大きさに応じて濃淡をつける] 2次元の図で濃淡という「次元」を加えて2変数関数を見る
  \item[$f(x,y)=\text{一定}$の曲線(等高線)を描く] 2次元の図で2変数関数を表す(次元を増やす必要がない)
\end{description}

\end{document}
