\documentclass[../book_jiriki_calc]{subfiles}

\begin{document}

\section{関数の全体を見る}

$y=x^4$と$y=x^2$の差異を考えてみる

\br

$x$が大きい場合の例として、たとえば$x=10$のときは、
\begin{align}
  y=x^4 & = 10^4 = 10000 \\
  y=x^2 & = 10^2 = 100
\end{align}
$x$をもっと大きくすると、$x^2$も$x^4$も大きな数になるが、この2つだけを比較すると、$x^4$の方がはるかに大きいので$x^2$は相対的に無視できそうだといえる

\br

$x$が$0$に近い場合の例として、たとえば$x=0.1$のときは、
\begin{align}
  y=x^4 & = 0.1^4 = 0.0001 \\
  y=x^2 & = 0.1^2 = 0.01
\end{align}
$x=0$の近くでグラフを書くと、$y=x^4$の方は$x$軸すれすれになる

このように、$x$が$0$に近いときは$x^2$も$x^4$も小さな数だが、この2つだけを比較すると、$x^2$の方が相対的に大きいので$x^4$は無視できるくらい小さそうだといえる

\br

3つ以上のものを比較するときも、\keyword{圧倒的に大きいものが1つだけあれば残りを無視しよう}という考え方ができる

微分や積分における極限にも、この考え方が用いられる

\sectionline

厳密な論理体系である数学では、どういう基準で何を無視するかというルールを明確に決めてから議論を積み重ねることになる

その一方で、論理的な証明を導く際には、\keyword{効くものと無視できるものを区別する}という直観が役立つ

\section{二項係数の4つの側面}

\keyword{パスカルの三角形}を帰納的に定義する

\begin{itemize}
  \item 1段目(最上段)は$1$とする
  \item $n$段目に$n$個の数を定めたとして、$n+1$段目は線で繋がっているすぐ上の数を足し合わせて定めることにする
\end{itemize}

このとき、次の4つの数が一致する
\begin{enumerate}
  \item $(a+b)^n$の展開式における$a^{n-k}b^k$の係数
  \item パスカルの三角形の$n+1$段目、左から$k+1$番目の数
  \item パスカルの三角形で$n+1$段目、左から$k+1$番目の地点から最上段に登る最短経路の個数
  \item ${}_n C_k = \binom{n}{k} = \dfrac{n!}{k!(n-k)!}$
\end{enumerate}

\sectionline

\subparagraph{1と4の一致}

次のように$(a+b)^n$を$n$個の積を並べたものとして表すことで示す
\begin{equation}
  (a+b)^n = \underbrace{(a+b)\times (a+b)\times \cdots \times (a+b)}_{n}
\end{equation}

この展開式において、たとえば$a^{n-k}b^k$という項がどこから生じるかを考える

\br

右辺を展開するとき、$a+b$のそれぞれの項で$a$か$b$の2通りの選択肢があるため、展開すると全部で$2^n$通りの単項式の和になる

展開したときに$a^{n-k}b^k$という項に寄与するのは、$n$個の$(a+b)$の中から$b$を$k$個選ぶ場合の数なので、${}_n C_k$通りある

\br

こうして、${}_n C_k$通りの$a^{n-k}b^k$が現れることから、$a^{n-k}b^k$の係数は${}_n C_k$となる

\sectionline

\subparagraph{2と3の一致}

ある地点から最上段に行くには、その地点の左上か右上に移動するしかない

そのため、この地点から最上段に登る最短経路の個数は、そのすぐ左上の地点を経由して最上段に登る最短経路の個数と、そのすぐ右上の地点を経由して最上段に登る最短経路の個数の和になる(和の法則)

これはまさにパスカルの三角形を定義したときのルールであり、パスカルの三角形の$n+1$段目、左から$k+1$番目の数は、$n$段目かつ左から$k$番目の数と、$n$段目かつ左から$k+1$番目の数の和になる

\sectionline

\subparagraph{1と2の一致}

次のように書くことで見えてくる
\begin{align}
  (a+b)^n & = (a+b)^{n - 1}\times (a+b)                     \\
          & = (a+b)^{n - 1}\times a + (a+b)^{n - 1}\times b
\end{align}
これは、$(a+b)^{n - 1}$の各項に、$a$と$b$をそれぞれかけて足し合わせる計算になっている

\br

$(a+b)^{n - 1}$を展開すると、それぞれの項は、$a$を$(n-1)-k$回、$b$を$k$回かけた形になる
\begin{equation}
  (a+b)^{n-1} = \sum_{k=0}^{n-1} {}_{n-1}C_k \cdot a^{n-k-1}b^k
\end{equation}

\br

$(a+b)^{n - 1}$の各項に$a$をかけると、$a$の指数が1増えるので、$a^{n-k}$の項が現れる
\begin{align}
  (a+b)^{n-1} \times a & = \sum_{k=0}^{n-1} {}_{n-1}C_k a^{n-k}b^k           \\
                       & = a^n + \sum_{k=1}^{n-1} {}_{n-1}C_{k-1} a^{n-k}b^k
\end{align}

同様に、$(a+b)^{n - 1}$の各項に$b$をかけると、$b$の指数が1増えるので、$b^{k+1}$の項が現れる
\begin{align}
  (a+b)^{n-1} \times b & = \sum_{k=0}^{n-1} {}_{n-1}C_{k} a^{n-k-1}b^{k+1} \\
                       & = \sum_{k=1}^{n} {}_{n-1}C_k a^{n-k} b^k + b^n
\end{align}

\br

よって、これらを足し合わせると、
\begin{align}
  (a+b)^n & = a^n + \sum_{k=1}^{n-1} \left( {}_{n-1}C_{k-1} + {}_{n-1}C_k \right) a^{n-k} b^k + b^n
\end{align}

\br

${}_{n-1}C_{k-1} + {}_{n-1}C_k$の部分は、それぞれ
\begin{itemize}
  \item ${}_{n-1}C_{k-1}$はパスカルの三角形の$n$段目、左から$k$番目の数
  \item ${}_{n-1}C_k$はパスカルの三角形の$n$段目、左から$k+1$番目の数
\end{itemize}
を表すので、これらの和($a^{n-k}b^k$の係数)がパスカルの三角形の$n+1$段目、左から$k+1$番目の数になることがいえる

(注:段数は$1$から始まるので、$n+1$段目が$(a+b)^n$の展開に対応する。同様に、左から数える番号も$1$始まりなので、$k+1$番目が$a^{n-k}b^k$の係数に対応する)

\sectionline

\subparagraph{一般の展開公式}

1と4の一致から、一般の展開公式は次のように書ける
\begin{equation}
  (a+b)^n = \sum_{k=0}^{n} \dfrac{n!}{k!(n-k)!} a^{n-k}b^k
\end{equation}
この展開公式を\keyword{二項展開}という

\br

そして、$a^{n-k}b^k$の係数
\begin{equation}
  {}_n C_k = \dfrac{n!}{k!(n-k)!} = \dfrac{n(n-1)\cdots (n-k+1)}{k(k-1)\cdots 1}
\end{equation}
は\keyword{二項係数}という

\sectionline

二項展開において、$a=1$の場合を考えると、
\begin{equation}
  (1+b)^n = 1 + nb + \dfrac{n(n-1)}{2!}b^2 + \dfrac{n(n-1)(n-2)}{3!}b^3 + \cdots
\end{equation}
ここで、二項係数はいつでも正なので、$b > 0$ならば、上の二項展開に現れる項はすべて正となる

\br

特に、$n \geq 2$ならば、次の不等式が成り立つ
\begin{equation}
  (1+b)^n > 1 + nb
\end{equation}

\section{$100!$と$10^{100}$はどちらが大きいか?}

$n=100$のとき、\keyword{$10^n$はとても大きい数だが、$n!$と比べたら取るに足らない}ことを示す($n!$を概算するスターリングの公式は使わずに)

\sectionline

$100!$において、$10$から先をすべて$10$に置き換える

$10$から$100$までの数は$91$個あるので、
\begin{equation}
  100! > 10^{91}
\end{equation}
がわかる

\br

より精密に評価するために、$10$から$99$までの90個の数の積を10個ずつまとめてみると、
\begin{gather}
  10^{10} < 10 \cdot 11 \cdots 19 < 10^{20}  \\
  10^{20} < 20 \cdot 21 \cdots 29 < 10^{30}  \\
  \vdots                                     \\
  10^{90} < 90 \cdot 91 \cdots 99 < 10^{100}
\end{gather}
さらに、これを縦にかけ合わせて、
\begin{equation}
  10^{10} \cdots 90^{10} < 10 \cdots 99 < 20^{10} \cdots 100^{10}
\end{equation}
左辺は、
\begin{align}
  10^{10} \cdots 90^{10} & = (10\cdots 90)^{10}                                                                  \\
                         & = \left( (10 \cdot 1) \cdot (10 \cdot 2) \cdot \cdots \cdot (10 \cdot 9) \right)^{10} \\
                         & = \left( 10^9 \cdot (1 \cdot 2 \cdot \cdots \cdot 9) \right)^{10}                     \\
                         & = \left( 10^9 \cdot 9! \right)^{10}                                                   \\
                         & = 10^{90} \cdot (9!)^{10}
\end{align}
右辺は、
\begin{align}
  20^{10} \cdots 100^{10} & = (20\cdots 100)^{10}                                                      \\
                          & = \left( (10 \cdot 2) \cdot (10 \cdot 3) \cdots (10 \cdot 10) \right)^{10} \\
                          & = \left( 10^{9} \cdot (1 \cdot 2 \cdots 9 \cdot 10) \right)^{10}           \\
                          & = \left( 10^{9} \cdot 9! \cdot 10 \right)^{10}                             \\
                          & = \left( 10^{10} \cdot 9! \right)^{10}                                     \\
                          & = 10^{100} \cdot (9!)^{10}
\end{align}
なので、
\begin{equation}
  10^{90} \cdot (9!)^{10} < 10 \cdots 99 < 10^{100} \cdot (9!)^{10}
\end{equation}
ここで、
\begin{equation}
  100! = 9! \cdot 10 \cdots 99 \cdot 100
\end{equation}
と表し、後回しにしていた$9! \cdot 100$を不等式の各項にかけると、
\begin{gather}
  10^{90} \cdot (9!)^{10} \cdot 10^{2} \cdot 9! < 100! < 10^{100} \cdot (9!)^{10} \cdot 10^{2} \cdot 9! \\
  10^{92} \cdot (9!)^{11} < 100! < 10^{102} \cdot (9!)^{11}
\end{gather}

\br

ところで、$9! = 362880 \fallingdotseq 3.6 \times 10^5$から、
\begin{gather}
  3 \times 10^5 < 9! < 4 \times 10^5
\end{gather}
と粗く評価しておき、この式の両辺を$11$乗すると、
\begin{gather}
  3^{11} \times 10^{55} < (9!)^{11} < 4^{11} \times 10^{55}
\end{gather}
ここで、次のように考えると、$10^5 < 3^{11}$という大まかな不等式が成り立つ
\begin{align}
  10^5 & = 3^5 \times \dfrac{10^5}{3^5}                 \\
       & = 3^5 \times \left( \dfrac{10}{3} \right)^5    \\
       & \fallingdotseq 3^5 \times \left( 3.3 \right)^5 \\
       & = 3^5 \times 3^{0.3 \times 5}                  \\
       & = 3^{10} \times 3^{0.3}                        \\
       & < 3^{11}
\end{align}
また、次のように考えることで、$4^{11} < 10^7$という大まかな不等式も成り立つ
\begin{align}
  4^{11} & = 2^{22}                                                   \\
         & = 2^{10} \times 2^{10} \times 2^2                          \\
         & = 1024^2 \times 4                                          \\
         & = \left( 1000 \times \dfrac{1024}{1000} \right)^2 \times 4 \\
         & = \left( 10^3 \times 1.024 \right)^2 \times 4              \\
         & = 10^6 \times 1.024^2 \times 4                             \\
         & < 10^6 \times 1.1^2 \times 4                               \\
         & < 10^7
\end{align}
これらを使うと、$(9!)^{11}$に関する不等式の左辺と右辺は、
\begin{gather}
  3^{11} \times 10^{55} < (9!)^{11} < 4^{11} \times 10^{55} \\
  10^5 \times 10^{55} < (9!)^{11} < 10^7 \times 10^{55}     \\
  10^{60} < (9!)^{11} < 10^{62}
\end{gather}
したがって、$100!$に関する不等式の左辺と右辺は、
\begin{gather}
  10^{92} \times (9!)^{11} < 100! < 10^{102} \times (9!)^{11} \\
  10^{92} \times 10^{60} < 100! < 10^{102} \times 10^{62}     \\
  10^{152} < 100! < 10^{164}
\end{gather}
これで、大まかな手計算で$100!$の大きさを評価できた

特に、$10^{152} < 100!$から、$100!$が$10^{100}$よりもはるかに大きいことがわかる

\section{$\cos x$のテイラー展開}

単項式におまじないの係数をつけて足したり引いたりした関数を考える
\begin{equation}
  y = 1 - \dfrac{x^2}{2!} + \dfrac{x^4}{4!} - \dfrac{x^6}{6!} + \dfrac{x^8}{8!} - \cdots
\end{equation}

\br

この無限級数を途中で打ち切らずに、ずっと続けることを考えてみる

$x$を固定しておくと、$n$が大きいとき、分子のべき乗$x^n$と分母の階乗$n!$では$n!$の方が圧倒的に大きくなり、この無限級数は収束する

\br

この無限級数をプロットしたグラフは、$|x|$が小さいところでは$y=\cos x$のグラフとほぼ重なり合う

原点から少し離れたところでも、多項式の項の個数を増やすとよく近似できる

\br

このように、調べたい関数を多項式で近似し、局所的に近似の精度を上げるときは多項式の項を増やすという形で定理を形式化したものが\keyword{テイラー展開}である

\end{document}
