\documentclass[../book_jiriki_calc]{subfiles}

\begin{document}

\section{微分と積分は何を捉えているか}

\keyword{微分}は、「微小な変化でどのような変動が起こるか」を分析する数学の手法

数学では、\keyword{極限}という概念を用いて「無限小レベルの変化」として厳密に微分を定義する

\sectionline

現実の事象を解明しようとすると、その事象に関わる要因は1つではなく、複数の要因が絡み合っていることが多い

複数の要因が絡み合っている状況を数量的に表すのが\keyword{多変数関数}

\sectionline

変数が2個以上あると、「変数を少し動かす」といってもいろいろな動かし方がある

その動かし方を精密に扱うのが\keyword{偏微分}

\sectionline

\keyword{積分}は「そこにある量を算出する道具」

「そこにある量」を小分けして合算して求める考え方(\keyword{区分求積法})が積分論の主軸

\section{本書の内容}

数学では、相対的に無視できるものを切り捨てるという操作を論理的に積み重ねて、\keyword{無限}という概念に向き合う

\sectionline

近似をしたつもりなのに大きな誤差が生じているとすると、その兆候はどこかに現れているはず

局所近似の誤差の兆候は\keyword{高階の微分}に現れる

これを逆に突き詰めると\keyword{テイラー展開}という概念に到達する

\end{document}
