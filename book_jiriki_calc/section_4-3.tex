\documentclass[../book_jiriki_calc]{subfiles}

\begin{document}

\section{位置の変化で微分を感じる}

「傾き」としての微分は歩いているときにも感じることができる

\vskip\baselineskip

まっすぐな坂道があって、坂道の出発点から水平方向に$x$だけ進んだ地点の標高が$f(x)$だとする

標高$f(x)$は$x$の関数だと思うことができ、坂道を真横から見ると、$y=f(x)$のグラフとみなせる

\vskip\baselineskip

$f(x+h)-f(x)$は地点$x$から水平に$h$だけ進んだときの標高の差となるので、$\dfrac{f(x+h)-f(x)}{h}$はこの地点のおおよその勾配となる

一方、$f(x)$が微分可能ならば、$h$が十分に小さいとき、この値は微分$f'(x)$に近い値になっているだろう

\vskip\baselineskip

つまり、坂道の勾配として、標高の「微分を感じている」ことになる

\vskip\baselineskip

\paragraph{微分を感じる例}

坂道において、$f(x)$を出発点から水平に$x$だけ離れた地点の標高とすると、$f'(x)$はその地点における勾配を表す

\sectionline

坂道の勾配は、位置によって異なる

$x$座標が増える方向に歩いているとき、ある地点$x$における勾配が$f'(x)$というのは、次のように感じることができる
\begin{itemize}
  \item $f'(x)>0$:登り坂
  \item $f'(x)<0$:下り坂
  \item $\left|f'(x)\right|$が大きい:急勾配
\end{itemize}

\section{時間の変化で微分を感じる}

時が経つにつれて変化する量は、時刻を変数とする関数で表される

たとえば、時とともに何かものが動くときは、その位置の座標は時刻を変数とする関数で記述できる

\vskip\baselineskip

ここでは、このような時刻を変数として位置を表す例を考える

\vskip\baselineskip

\subparagraph{位置の微分}

数直線上で物体が動いていて、時刻$t$におけるその位置をその座標$f(t)$で表すとする

\vskip\baselineskip

ここで、微分の定義において、極限を取る前の
\begin{equation}
  \frac{f(t+h)-f(t)}{h}
\end{equation}
という値の意味に注目する

\vskip\baselineskip

分子は時刻$t$から時刻$t+h$の間に進んだ距離で、それをその間にかかった時間$h$で割っていることから、これは時間間隔$h$での\keyword{平均速度}を表している

\vskip\baselineskip

したがって、時間間隔$h$を$0$に近づけたときの極限、すなわち位置の微分$f'(t)$は、時刻$t$における\keyword{(瞬間)速度}を表していると理解できる

\vskip\baselineskip

\paragraph{微分を感じる例}

位置の微分$f'(t)$は、時刻$t$における速度である

\sectionline

\subparagraph{位置の2階微分}

速度は、時刻とともに変わっていく

速度の時間変化を見るために、速度$f'(t)$を時刻$t$の関数とみなすと、これは位置$f(t)$の導関数である

\vskip\baselineskip

速度$f'(t)$をさらに微分するということは、$f(t)$の2階微分$f''(t)$を考えることになる

これにも名前がついていて、\keyword{加速度}という

加速度$f''(t)$は、速度の変化を表す量である

\vskip\baselineskip

\paragraph{微分を感じる例}

位置の2階微分$f''(t)$は、時刻$t$における加速度である

\sectionline

\subparagraph{運動の記述}

\keyword{運動}という言葉は、物理学では「物体が時々刻々と位置を変える」という"motion"の意味で使われる

\vskip\baselineskip

先ほどは、数直線上という1次元的な位置の変化を考えたが、今度は次元を上げて、平面上あるいは空間の中における「運動」を考えてみる

\vskip\baselineskip

そのために、座標を用いて時々刻々と変わる位置を記述することにする

たとえば2次元の運動の場合、時刻$t$における位置を\keyword{位置ベクトル}として、
\begin{equation}
  \left( x(t), y(t) \right)
\end{equation}
とベクトルで表す

3次元空間の場合には、もう1つ$z$座標を用いる

\vskip\baselineskip

位置ベクトルの$x$成分、$y$成分をそれぞれ微分して得られるベクトル
\begin{equation}
  \left( x'(t), y'(t) \right) = \left( \dfrac{dx}{dt}(t), \dfrac{dy}{dt}(t) \right)
\end{equation}
を\keyword{速度ベクトル}という

速度ベクトルは大きさだけではなく、どちらの方向に進んでいるかという向きの情報も持っている

\vskip\baselineskip

これに対して、速度ベクトルの大きさを\keyword{速さ}といい、向きの情報を含む「速度」と区別した用語を使う
\begin{equation}
  \text{速さ} = \sqrt{x'(t)^2 + y'(t)^2}
\end{equation}

\vskip\baselineskip

\keyword{加速度ベクトル}は、速度ベクトルを微分した次のベクトルになる
\begin{equation}
  \left( x''(t), y''(t) \right) = \left( \dfrac{d^2x}{dt^2}(t), \dfrac{d^2y}{dt^2}(t) \right)
\end{equation}

\sectionline

物理法則は、座標とは無関係に成り立っている

一方、座標系を使うことで、次元が高い場合でも、座標成分ごとに微分すれば速度ベクトルや加速度ベクトルを求めることができるため、計算上の便利さがある

\section{経済学における微分}

何かの消費量が$q$であるとき、そのことによって得られる満足感やありがたみ(の総量)を仮想的に数値化して\keyword{効用}と呼ぶ

効用は、消費量$q$の関数とみなして\keyword{効用関数}とよび、
\begin{equation}
  U=U(q)
\end{equation}
と表記する

\vskip\baselineskip

この考え方には、そもそも満足度を数値化できるのだろうか?という批判がある

そのため、現代の経済学では、効用の絶対的な大きさには意味がなく、「どちらが好きか」という個人の好み(\keyword{選好})を描写する表現であるという考え方が使われている

$p$が$q$と同じ程度かそれ以上に好きならば、$U(p) \geq U(q)$を満たすような関数$U(q)$を、この\keyword{選好を描写する効用関数}という

\vskip\baselineskip

同じ選好を描写する効用関数$U(q)$は無数にあるが、どれを使っても結論が変わらない性質は、その選好から導かれる性質と考えることができる

\vskip\baselineskip

たとえば、効用関数の微分
\begin{equation}
  \frac{dU}{dq}(q) = U'(q)
\end{equation}
の符号は、その選好を描写する効用関数のどれを使っても変わらない

効用関数の微分$U'(q)$を、経済学では\keyword{限界効用}とよぶ

\sectionline

\subparagraph{限界効用漸減の法則}

たとえば、喉が渇いているうちは、少し水を飲めるだけでも嬉しいと感じるが、何杯も飲むとありがたみが薄れてくる

\vskip\baselineskip

このような「最初は嬉しいが、そのうち飽きてくる」という経験的事実を\keyword{限界効用漸減の法則}という

\vskip\baselineskip

この性質は、効用関数$U(q)$の微分(限界効用)および2階微分を用いて、
\begin{itemize}
  \item ありがたいと思う:$U'(q) \geq 0$
  \item だんだん飽きてくる:$U''(q) \leq 0$
\end{itemize}
と表される

\sectionline

\subparagraph{「ありがたみ」の数式化}

まず、水を「ありがたいと思う」を数式化してみる

\vskip\baselineskip

たとえば、すでに$q$の分量だけ水を飲んだ後、追加で少量の水を$h$だけ飲んだとすると、
\begin{equation}
  U(q+h)>U(q)
\end{equation}
が「ありがたい」という選好を描写する不等式になる

したがって、$h>0$のとき、
\begin{equation}
  \frac{U(q+h)-U(q)}{h}>0
\end{equation}
となるので、$h\to 0$としたときの極限である$U'(q)$は、$U'(q) \geq 0$を満たすことになる

\vskip\baselineskip

このようにして、「水をありがたいと思う」ことから、限界効用$U'(q)$の性質$U'(q) \geq 0$が導かれた

\sectionline

\subparagraph{「飽き」の数式化}

「だんだん飽きてくる」を選好で説明するには、効用関数$U(q)$は$p$と$q$のどちらが好きかというだけではなく、好みをもう少し精密に描写している必要がある

\vskip\baselineskip

その1つのアプローチに、「飽きてくる」ということを「他のものに目移りする」というように、他のものとの比較をするというものがある

すなわち、複数のものに対する選好を考え、それを複数の変数を持つ効用関数で描写する

\vskip\baselineskip

ここでは1変数のままで、以下のように一定量を追加して消費したときの選好があると仮定して話を進める

すなわち、今までの消費量が$q<p$のとき、同じ量$h$の追加であっても、$q$しか飲んでいないときと比べて、すでに$p$というたくさんの量を飲んだ後では「ありがたみが薄れる」ということを、次の不等式で描写してみる
\begin{equation}
  U(q+h)-U(q)>U(p+h)-U(p)
\end{equation}
このような不等式を満たす効用関数$U(q)$は無数にあるが、どれを使っても$U''(q) \leq 0$となる

\vskip\baselineskip

このことを確かめるために、まず不等式の両辺を$h>0$で割って、$h$を$0$に近づけた極限を取ると、$U'(q) \geq U'(p)$となることがわかる

\vskip\baselineskip

次に、$s>0$として$p=q+s$とおくと、$U'(q) \geq U'(p)$より、
\begin{align}
  U'(q)         & \geq U'(q+s) \\
  U'(q+s)-U'(q) & \leq 0
\end{align}
となるので、両辺を$s$で割って、$s \to 0$の極限を取ると、
\begin{align}
  U''(q) = \lim_{s \to 0} \frac{U'(q+s)-U'(q)}{s} \leq 0
\end{align}
となる

\vskip\baselineskip

これで、「だんだん飽きてくる」という限界効用漸減の法則から、効用関数の2階微分の不等式$U''(q) \leq 0$が導かれた

\vskip\baselineskip

逆に「やみつきになる」場合は、限界効用漸減の法則とは正反対で、$U''(q) \geq 0$となる

\sectionline

このように、何かを消費したときに、「ありがたい」とか「飽きてくる」という感情を描写する効用関数はどれを使っても、その微分や2階微分の符号に特徴が現れることになる

\vskip\baselineskip

\paragraph{微分を感じる例}

効用関数の微分(限界効用)$U'(q)$の符号は、消費量が$q$の時点で追加で消費することに対する「ありがたみ」を表し、2階微分$U''(q)$の符号は「飽き」や「やみつき」の傾向を表す

\sectionline

\subparagraph{微分の符号とグラフの形状}

「最初はありがたいが、たくさんあるとだんだん飽きてくる」という効用関数をグラフに表すと、グラフは右上がりで上に凸になる
\begin{enumerate}
  \item 限界効用が正:「ありがたみを感じる」ということでグラフは右上がり
  \item 2階微分が負:「だんだん飽きてくる(関数の増加率がだんだん減ってくる)」ということでグラフは上に凸
\end{enumerate}

\section{微分がつねに$0$ならば定数である}

標語的に言えば、\keyword{無限小レベルで変化がなければ、大域的に変化がない}ということ

\sectionline

\paragraph{定理}

実数全体で定義された関数$f(x)$について、すべての$x$で$f'(x)=0$ならば、その関数$f(x)$は定数である

\sectionline

この定理は、\keyword{平均値の定理}という一種の「不動点定理」から導かれる

\begin{itemize}
  \item どの時刻でも速度が$0$ならば、実は動いていない(位置が一定)
  \item 限界効用が$0$ならば、そもそもこの人はそのことに無関心(効用関数$U(q)$が消費量$q$によらずに一定)
\end{itemize}

具体例に当てはめると当たり前に思えるが、よく見ると\keyword{局所的な性質から大域的な性質を導いている}ことがわかる

\sectionline

\paragraph{定理}

実数全体で定義された関数$g(x)$について、すべての$x$で$g'(x)=a$ならば、$g(x) = ax+g(0)$である

\sectionline

\subparagraph{証明}

この定理は、前述の定理から導かれる

新たな関数として$f(x) = g(x) - ax$とおいてみると、
\begin{equation}
  f'(x) = g'(x) - a = a - a = 0
\end{equation}
となるので、$f(x)$は定数である

特に、$f(x) = f(0)$がすべての$x$に対して成り立つ

\vskip\baselineskip

$f(x) = g(x) - ax$だったことを思い出すと、$f(0)= g(0)$となるので、
\begin{align}
  g(x) -ax              & =f(0) = g(0) \\
  \therefore \quad g(x) & = ax + g(0)
\end{align}
が示される$\qed$

\sectionline

ここで取り上げた2つの定理は、もっとも簡単な\keyword{微分方程式}を解いたとみなすこともできる


\end{document}
