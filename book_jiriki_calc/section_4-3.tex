\documentclass[../book_jiriki_calc]{subfiles}

\begin{document}

\section*{位置の変化で微分を感じる}

「傾き」としての微分は歩いているときにも感じることができる

\vskip\baselineskip

まっすぐな坂道があって、坂道の出発点から水平方向に$x$だけ進んだ地点の標高が$f(x)$だとする

標高$f(x)$は$x$の関数だと思うことができ、坂道を真横から見ると、$y=f(x)$のグラフとみなせる

\vskip\baselineskip

$f(x+h)-f(x)$は地点$x$から水平に$h$だけ進んだときの標高の差となるので、$\dfrac{f(x+h)-f(x)}{h}$はこの地点のおおよその勾配となる

一方、$f(x)$が微分可能ならば、$h$が十分に小さいとき、この値は微分$f'(x)$に近い値になっているだろう

\vskip\baselineskip

つまり、坂道の勾配として、標高の「微分を感じている」ことになる

\sectionline

\paragraph{微分を感じる例}

坂道において、$f(x)$を出発点から水平に$x$だけ離れた地点の標高とすると、$f'(x)$はその地点における勾配を表す

\sectionline

坂道の勾配は、位置によって異なる

$x$座標が増える方向に歩いているとき、ある地点$x$における勾配が$f'(x)$というのは、次のように感じることができる
\begin{itemize}
  \item $f'(x)>0$:登り坂
  \item $f'(x)<0$:下り坂
  \item $\left|f'(x)\right|$が大きい:急勾配
\end{itemize}

\end{document}
