\documentclass[../book_jiriki_calc]{subfiles}

\begin{document}

\section{合成関数の微分}

$x=g(t)$を$f(x)$に代入すると、$t$を変数とする関数$f(g(t))$が得られる

この関数$f(g(t))$を、関数$f(x)$と関数$g(t)$の\keyword{合成関数}という

\sectionline

\paragraph{定理:合成関数の微分(連鎖律)}\quad

関数$f(x)$と関数$g(t)$の合成関数$f(g(t))$を$F(t)$と書くと、
\begin{equation}
  F'(t) = f'(g(t)) \cdot g'(t)
\end{equation}
が成り立つ

\sectionline

\keyword{代入してから微分 $\neq$ 微分してから代入}であることに注意

\begin{itemize}
  \item $F'(t)$:代入してから微分($x=g(t)$を$f(x)$に代入した関数$f(g(t))$を微分)
  \item $f'(g(t))$:微分してから代入($f(x)$を微分した$f'(x)$に$x=g(t)$を代入)
\end{itemize}

後者に$g'(t)$をかけて初めて前者と一致する、というのが合成関数の微分公式の趣旨

\sectionline

\subparagraph{連鎖率の感覚}

勾配が一定の坂道を登っている状況を考える

\begin{itemize}
  \item 水平方向の速度は、「単位時間あたりにどれだけ進むか」を表している
  \item 坂道の勾配は、「単位距離進むごとにどれだけ登るか」を表している
\end{itemize}

よって、上下方向の速度「単位時間あたりにどれだけ登るか」を求めるには、水平方向の速度と勾配をかければよい
\begin{equation}
  \text{上下方向の速度} = \text{水平方向の速度} \times \text{勾配}
\end{equation}

この計算式を、微分を用いて表現する

\br

水平方向に座標$x$をとり、その標高を$f(x)$とする

そうすると、微分$f'(x)$はこの地点での坂道の勾配となる

\br

一方、ある人が時刻$t$に、$x$座標では$x=g(t)$の地点にいるとすると、微分$g'(t)$はその人の水平方向の速度となる

\br

このとき、合成関数$F(t) = f(g(t))$は時刻$t$にこの人がいる地点の標高となり、その微分$F'(t)$は、時刻$t$における上下方向の速度を表すことになる

\br

よって、上下方向の速度を求める式は、
\begin{equation}
  F'(t) = f'(g(t)) \cdot g'(t)
\end{equation}
これは、合成関数の微分の公式になっている

\section{積の微分(ライプニッツの法則)}

2つの関数が与えられたとすると、それらを足せば1つの関数になり、またそれらをかけても別の関数が得られる

2つの関数をかけたとき、その微分がどうなるか?

\sectionline

\paragraph{定理:積の微分(ライプニッツの法則)}\quad

2つの関数$f(t),\, g(t)$の積$f(t)g(t)$の微分は、
\begin{equation}
  (f(t)g(t))' = f'(t)g(t) + f(t)g'(t)
\end{equation}
である

\sectionline

\subparagraph{ライプニッツの法則の可視化}\quad

時刻$t$のときに高さ$g(t)$、底辺の長さ$f(t)$の長方形を考える

この長方形の面積は$f(t)g(t)$である

\br

時々刻々と$f(t)$も$g(t)$も変わる、それに応じて長方形の形も変わり、面積$f(t)g(t)$が変化する

ライプニッツの法則の左辺$(f(t)g(t))'$は、この面積の変化率を表している

\br

まず、微分の定義に戻って、積$f(t)g(t)$の微分を書き下すと、
\begin{align}
  (f(t)g(t))' & = \lim_{h \to 0} \frac{f(t+h)g(t+h) - f(t)g(t)}{h}
\end{align}
右辺の分子は、時刻$t+h$での長方形の面積と、時刻$t$での長方形の面積の差になっている

\br

時刻$t$から$t+h$になったとき、高さも底辺の長さも増加する

このとき、長方形の面積の変化を、次の3つの部分に分けて考える
\begin{itemize}
  \item 高さの変化による面積の増加(縦長の帯)
  \item 底辺の長さの変化による面積の増加(横長の帯)
  \item 高さと底辺の長さの変化による面積の増加(小さな四角形)
\end{itemize}

\sectionline

\subparagraph{高さの変化によって増えた部分}\quad

縦長の帯は、幅$h$、高さ$g(t+h)-g(t)$の縦長の長方形になっている

\br

$h$が$0$に近づくと、$\dfrac{g(t+h)-g(t)}{h}$は$g(t)$の微分$g'(t)$に近づくことから、$h$が$0$に近ければ、
\begin{equation}
  g(t+h)-g(t) \approx hg'(t)
\end{equation}
となるので、縦長の帯の高さは$hg'(t)$で近似できる

\br

よって、
\begin{equation}
  \text{縦長の帯の面積} \approx f(t) \cdot hg'(t) = hf(t)g'(t)
\end{equation}

\sectionline

\subparagraph{底辺の長さの変化によって増えた部分}\quad

横長の帯は、幅$h$、高さ$f(t+h)-f(t)$の横長の長方形になっている

先ほどと同様に高さの近似を考えて、
\begin{equation}
  \text{横長の帯の面積} \approx g(t) \cdot hf'(t) = hf(t)g'(t)
\end{equation}

\sectionline

\subparagraph{高さと底辺の長さの変化によって増えた部分}\quad

小さな四角形は、幅$f(t+h)-f(t)$、高さ$g(t+h)-g(t)$の長方形になっている

幅と高さについて、これまでと同様に近似を考えると、
\begin{align}
  \text{小さな四角形の面積} & \approx (hf'(t))\times (hg'(t)) \\
                   & = h^2f'(t)g'(t)
\end{align}

$h$が小さいとき、縦長の帯と横長の帯は、縦か横のいずれかだけが小さくなるが、小さな四角形は縦も横も小さい

この違いを反映して、右辺には$h^2$という、$h$よりもはるかに小さい係数が現れている

この項は、$h$を$0$に近づける極限の計算の中では実は寄与しない

\sectionline

まとめると、長方形の面積の変化は、縦長の帯の面積と横長の帯の面積と小さな四角形の面積の和になるので、
\begin{multline}
  f(t+h)g(t+h) - f(t)g(t) \\ = hf(t)g'(t) + hf(t)g'(t) + h^2f'(t)g'(t)
  \\ = h\left( f(t)g'(t) + f'(t)g(t) + hf'(t)g'(t)\right)
\end{multline}

よって、積の微分は、
\begin{align}
   & \phantom{==} (f(t)g(t))'                                                      \\
   & = \lim_{h \to 0} \frac{f(t+h)g(t+h) - f(t)g(t)}{h}                            \\
   & = \lim_{h \to 0} \frac{h\left( f(t)g'(t) + f'(t)g(t) + hf'(t)g'(t)\right)}{h} \\
   & = \lim_{h \to 0} \left( f(t)g'(t) + f'(t)g'(t)+ hf'(t)g'(t)\right)            \\
   & = f(t)g'(t) + f'(t)g(t)
\end{align}
となり、ライプニッツの法則が導かれた

\sectionline

このように図形を用いると、ライプニッツの法則は、
\begin{multline}
  \text{長方形の面積の変化} \\ = \text{縦長の帯の寄与} + \text{横長の帯の寄与}
\end{multline}
という形で「目に見える」ようになる

\section{商の微分}

\paragraph{定理:商の微分}\quad

\begin{equation}
  \left( \frac{f(t)}{g(t)} \right)' = \frac{f'(t)g(t) - f(t)g'(t)}{(g(t))^2}
\end{equation}

\sectionline

\subparagraph{証明}

左辺の$\dfrac{f(t)}{g(t)}$を$G(t)$とおいておく

$G(t)$の微分$G'(t)$を求めるのが目標

\br

$G(t)=\dfrac{f(t)}{g(t)}$の分母を払うと、
\begin{equation}
  f(t) \times G(t) = g(t)
\end{equation}
この両辺を微分すると、ライプニッツの法則を使って、
\begin{equation}
  f(t)G'(t) + f'(t)G(t) = g'(t)
\end{equation}
移項して、$f(t)$で割れば、
\begin{align}
  G'(t) & = \dfrac{g'(t) - f'(t)G(t)}{f(t)}                       \\
        & = \dfrac{g'(t) - f'(t) \times \dfrac{f(t)}{g(t)}}{f(t)} \\
        & = \dfrac{g'(t)f(t) - f'(t)g(t)}{(f(t))^2}
\end{align}
となり、商の微分の公式が導かれた$\qed$

\end{document}
