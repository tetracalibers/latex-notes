\documentclass[../book_jiriki_calc]{subfiles}

\begin{document}

\section*{合成関数の微分}

$x=g(t)$を$f(x)$に代入すると、$t$を変数とする関数$f(g(t))$が得られる

この関数$f(g(t))$を、関数$f(x)$と関数$g(t)$の\keyword{合成関数}という

\sectionline

\paragraph{定理:合成関数の微分(連鎖律)}\quad

関数$f(x)$と関数$g(t)$の合成関数$f(g(t))$を$F(t)$と書くと、
\begin{equation}
  F'(t) = f'(g(t)) \cdot g'(t)
\end{equation}
が成り立つ

\sectionline

\keyword{代入してから微分 $\neq$ 微分してから代入}であることに注意

\begin{itemize}
  \item $F'(t)$:代入してから微分($x=g(t)$を$f(x)$に代入した関数$f(g(t))$を微分)
  \item $f'(g(t))$:微分してから代入($f(x)$を微分した$f'(x)$に$x=g(t)$を代入)
\end{itemize}

後者に$g'(t)$をかけて初めて前者と一致する、というのが合成関数の微分公式の趣旨

\sectionline

\subparagraph{連鎖率の感覚}

勾配が一定の坂道を登っている状況を考える

\begin{itemize}
  \item 水平方向の速度は、「単位時間あたりにどれだけ進むか」を表している
  \item 坂道の勾配は、「単位距離進むごとにどれだけ登るか」を表している
\end{itemize}

よって、上下方向の速度「単位時間あたりにどれだけ登るか」を求めるには、水平方向の速度と勾配をかければよい
\begin{equation}
  \text{上下方向の速度} = \text{水平方向の速度} \times \text{勾配}
\end{equation}

\vskip\baselineskip

この計算式を、微分を用いて表現する

\vskip\baselineskip

水平方向に座標$x$をとり、その標高を$f(x)$とする

そうすると、微分$f'(x)$はこの地点での坂道の勾配となる

\vskip\baselineskip

一方、ある人が時刻$t$に、$x$座標では$x=g(t)$の地点にいるとすると、微分$g'(t)$はその人の水平方向の速度となる

\vskip\baselineskip

このとき、合成関数$F(t) = f(g(t))$は時刻$t$にこの人がいる地点の標高となり、その微分$F'(t)$は、時刻$t$における上下方向の速度を表すことになる

\vskip\baselineskip

よって、上下方向の速度を求める式は、
\begin{equation}
  F'(t) = f'(g(t)) \cdot g'(t)
\end{equation}
これは、合成関数の微分の公式になっている

\end{document}
