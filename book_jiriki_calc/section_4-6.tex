\documentclass[../book_jiriki_calc]{subfiles}

\begin{document}

\section*{角速度}

遠くに電車が走っているのが見えたとする

距離はわからなくても、見える角度は時々刻々と変化していく

こんな状況を正確に言い表すために\keyword{角速度}という概念がある

\sectionline

基準点と、それを通る基準線(基準の方向)をあらかじめ決めておく

注目している点が、基準点から見て基準の方向から(左回りに測って)角度$\theta$の位置にあるとする

\vskip\baselineskip

この角度$\theta$は時間$t$によって変化するので、$t$を変数とする関数という意味で、$\theta (t)$と表す

このとき、微分
\begin{equation}
  \dfrac{d\theta(t)}{dt}
\end{equation}
を時刻$t$における\keyword{角速度}という

\sectionline

どこから見るか、すなわち基準点をどこにとるかによって、角速度は変わる

一方、基準線の方向については、どのように選んでも角速度に影響しない

\vskip\baselineskip

実際、基準線の方向を変えても、角度$\theta(t)$には時刻によらない定数が付け加わるだけであるから、その微分である角速度には影響しないことになる

\section*{三角関数の微分}

円周上を一定の速さで進むことを\keyword{等速円運動}という

等速円運動では、円の中心から見ると角速度が一定になっている

\sectionline

ここでは計算を簡単にするため、半径$1$の円周上を速さ$1$で左回りに動くことを考える

速さ$1$というのは、経過した時間が$t$ならば、円弧を長さ$t$だけ進むということ(経過時間と進んだ距離が等しい=その比が$1$になる)

円の中心から見た角度も、弧度法で$t$だけ増える

\vskip\baselineskip

この等速円運動を$xy$座標を用いて表す

時刻$t=0$のときに$x$軸上の点$(1,0)$を出発すると、時刻$t$の位置$P$は、原点を中心に角度$t$だけ円周上を左回りに進んだ点として
\begin{equation}
  (x(t), y(t)) = (\cos t, \sin t)
\end{equation}
と座標表示される

\vskip\baselineskip

この運動の速さは$1$で一定だが、速度ベクトルの向きは時刻とともに変わる

速度ベクトルは、$x$座標、$y$座標それぞれについて微分すればよいので、時刻$t$において、
\begin{equation}
  \left(\dfrac{dx(t)}{dt}, \dfrac{dy(t)}{dt}\right) = \left(\dfrac{d}{dt}\cos t, \dfrac{d}{dt}\sin t\right)
\end{equation}
で与えられる

\vskip\baselineskip

\subparagraph{角速度から三角関数の微分を導く}\quad

半径$1$の円周上を動くという条件は、
\begin{equation}
  x^2(t) + y^2(t) = 1
\end{equation}
と表される

両辺を微分すると、積の微分に関するライプニッツの法則より、
\begin{equation}
  2 \left(x(t)x'(t) + y(t)y'(t)\right) = 0
\end{equation}
となり、内積が$0$であることから、速度ベクトルは位置ベクトルに直交していることがわかる

\vskip\baselineskip

また、速さが$1$なので、速度ベクトルの大きさは$1$である

\vskip\baselineskip

したがって、この速度ベクトルは大きさが$1$で、向きは位置ベクトルを$\dfrac{\pi}{2}$だけ左に回転させた方向を向いていることになり、
\begin{align}
  \left(\dfrac{dx(t)}{dt}, \dfrac{dy(t)}{dt}\right) & = \left(\cos\left(t+\dfrac{\pi}{2}\right), \sin\left(t+\dfrac{\pi}{2}\right)\right) \\
                                                    & = \left(-\sin t, \cos t\right)
\end{align}
がわかる

\vskip\baselineskip

速度ベクトルの2通りの表現が得られたので、
\begin{align}
  \dfrac{d}{dt}\cos t & = -\sin t \\
  \dfrac{d}{dt}\sin t & = \cos t
\end{align}
という、三角関数の微分の公式が導かれた

\end{document}
