\documentclass[../book_jiriki_calc]{subfiles}

\begin{document}

\section*{角速度}

遠くに電車が走っているのが見えたとする

距離はわからなくても、見える角度は時々刻々と変化していく

こんな状況を正確に言い表すために\keyword{角速度}という概念がある

\sectionline

基準点と、それを通る基準線(基準の方向)をあらかじめ決めておく

注目している点が、基準点から見て基準の方向から(左回りに測って)角度$\theta$の位置にあるとする

\vskip\baselineskip

この角度$\theta$は時間$t$によって変化するので、$t$を変数とする関数という意味で、$\theta (t)$と表す

このとき、微分
\begin{equation}
  \dfrac{d\theta(t)}{dt}
\end{equation}
を時刻$t$における\keyword{角速度}という

\sectionline

どこから見るか、すなわち基準点をどこにとるかによって、角速度は変わる

一方、基準線の方向については、どのように選んでも角速度に影響しない

\vskip\baselineskip

実際、基準線の方向を変えても、角度$\theta(t)$には時刻によらない定数が付け加わるだけであるから、その微分である角速度には影響しないことになる

\end{document}
