\documentclass[../book_jiriki_calc]{subfiles}

\begin{document}

\section{角速度}

遠くに電車が走っているのが見えたとする

距離はわからなくても、見える角度は時々刻々と変化していく

こんな状況を正確に言い表すために\keyword{角速度}という概念がある

\sectionline

基準点と、それを通る基準線(基準の方向)をあらかじめ決めておく

注目している点が、基準点から見て基準の方向から(左回りに測って)角度$\theta$の位置にあるとする

\vskip\baselineskip

この角度$\theta$は時間$t$によって変化するので、$t$を変数とする関数という意味で、$\theta (t)$と表す

このとき、微分
\begin{equation}
  \dfrac{d\theta(t)}{dt}
\end{equation}
を時刻$t$における\keyword{角速度}という

\sectionline

どこから見るか、すなわち基準点をどこにとるかによって、角速度は変わる

一方、基準線の方向については、どのように選んでも角速度に影響しない

\vskip\baselineskip

実際、基準線の方向を変えても、角度$\theta(t)$には時刻によらない定数が付け加わるだけであるから、その微分である角速度には影響しないことになる

\section{三角関数の微分}

円周上を一定の速さで進むことを\keyword{等速円運動}という

等速円運動では、円の中心から見ると角速度が一定になっている

\sectionline

ここでは計算を簡単にするため、半径$1$の円周上を速さ$1$で左回りに動くことを考える

速さ$1$というのは、経過した時間が$t$ならば、円弧を長さ$t$だけ進むということ(経過時間と進んだ距離が等しい=その比が$1$になる)

円の中心から見た角度も、弧度法で$t$だけ増える

\vskip\baselineskip

この等速円運動を$xy$座標を用いて表す

時刻$t=0$のときに$x$軸上の点$(1,0)$を出発すると、時刻$t$の位置$P$は、原点を中心に角度$t$だけ円周上を左回りに進んだ点として
\begin{equation}
  (x(t), y(t)) = (\cos t, \sin t)
\end{equation}
と座標表示される

\vskip\baselineskip

この運動の速さは$1$で一定だが、速度ベクトルの向きは時刻とともに変わる

速度ベクトルは、$x$座標、$y$座標それぞれについて微分すればよいので、時刻$t$において、
\begin{equation}
  \left(\dfrac{dx(t)}{dt}, \dfrac{dy(t)}{dt}\right) = \left(\dfrac{d}{dt}\cos t, \dfrac{d}{dt}\sin t\right)
\end{equation}
で与えられる

\vskip\baselineskip

\subparagraph{角速度から三角関数の微分を導く}\quad

半径$1$の円周上を動くという条件は、
\begin{equation}
  x^2(t) + y^2(t) = 1
\end{equation}
と表される

両辺を微分すると、積の微分に関するライプニッツの法則より、
\begin{equation}
  2 \left(x(t)x'(t) + y(t)y'(t)\right) = 0
\end{equation}
となり、内積が$0$であることから、速度ベクトルは位置ベクトルに直交していることがわかる

\vskip\baselineskip

また、速さが$1$なので、速度ベクトルの大きさは$1$である

\vskip\baselineskip

したがって、この速度ベクトルは大きさが$1$で、向きは位置ベクトルを$\dfrac{\pi}{2}$だけ左に回転させた方向を向いていることになり、
\begin{align}
  \left(\dfrac{dx(t)}{dt}, \dfrac{dy(t)}{dt}\right) & = \left(\cos\left(t+\dfrac{\pi}{2}\right), \sin\left(t+\dfrac{\pi}{2}\right)\right) \\
                                                    & = \left(-\sin t, \cos t\right)
\end{align}
がわかる

\vskip\baselineskip

速度ベクトルの2通りの表現が得られたので、
\begin{align}
  \dfrac{d}{dt}\cos t & = -\sin t \\
  \dfrac{d}{dt}\sin t & = \cos t
\end{align}
という、三角関数の微分の公式が導かれた

\section{オイラーの公式と三角関数のテイラー展開}

$F(t)$が実数値の関数$f(t),\, g(t)$を用いて
\begin{equation}
  F(t) = f(t) + ig(t)
\end{equation}
と表されるとき、$F(t)$を\keyword{複素数値の関数}という

ここで、$i$は虚数単位で、$i^2 = -1$である

\sectionline

$F(t) = \cos t + i\sin t$という複素数値の関数を考える

実数であっても複素数であっても定数倍は微分の外に出せることに留意して、$F(t)$の微分を計算する
\begin{align}
  \dfrac{dF(t)}{dt} & = \dfrac{d}{dt}\cos t + i\dfrac{d}{dt}\sin t \\
                    & = -\sin t + i\cos t                          \\
                    & = i\cos t + i^2\sin t                        \\
                    & = i(\cos t + i\sin t)                        \\
                    & = iF(t)
\end{align}
両辺を見比べると、
\begin{equation*}
  F'(t) = iF(t)
\end{equation*}
という微分方程式が得られたことになる

\sectionline

ところで、関数$F(t)$が
\begin{itemize}
  \item 微分方程式$F'(t) = \lambda F(t)$
  \item 初期条件$F(0) = 1$
\end{itemize}
を満たすならば、$F(t)$は
\begin{equation}
  F(t) = e^{\lambda t}
\end{equation}
という形になっていることを、以前示した

\vskip\baselineskip

指数関数$e^{\lambda t}$は、無限級数表示$\displaystyle\sum_{n=0}^{\infty}\dfrac{(\lambda t)^n}{n!}$を用いると、$\lambda$が複素数のときでも意味を持つ

そうすると、この定理は、定数$\lambda$が複素数で$F(t)$が複素数値の関数の場合にも成り立つ

\vskip\baselineskip

上述の$F(t) = \cos t + i\sin t$は、$F'(t) = iF(t)$と$F(0) = 1$を満たすので、$\lambda = i$の場合に対応する

\sectionline

\paragraph{定理:オイラーの公式}

\begin{equation*}
  e^{it} = \cos t + i\sin t
\end{equation*}

\sectionline

この公式を用いると、三角関数の性質は、指数関数のさまざまな性質から導ける

\vskip\baselineskip

\subparagraph{三角関数の加法公式}\quad

指数法則
\begin{equation*}
  e^{i(a+b)} = e^{ia}e^{ib}
\end{equation*}
すなわち、
\begin{equation*}
  \cos(a+b) + i\sin(a+b) = (\cos a + i\sin a)(\cos b + i\sin b)
\end{equation*}
の右辺を展開して、実部と虚部を比較することで、
\begin{align*}
  \cos(a+b) & = \cos a\cos b - \sin a\sin b \\
  \sin(a+b) & = \sin a\cos b + \cos a\sin b
\end{align*}
が得られる

\vskip\baselineskip

\subparagraph{三角関数の3倍角の公式}\quad

指数法則
\begin{equation*}
  e^{i3t} = (e^{it})^3
\end{equation*}
において、$x=it$とすると、
\begin{equation*}
  \cos 3t + i\sin 3t = e^{i3t} = (e^{it})^3 = (\cos t + i\sin t)^3
\end{equation*}
右辺を展開して、実部と虚部を比較することで、
\begin{align*}
  \cos 3t & = 4\cos^3 t - 3\cos t \\
  \sin 3t & = 3\sin t - 4\sin^3 t
\end{align*}
が得られる

\vskip\baselineskip

\subparagraph{$e, \pi, i$の関係式}\quad

オイラーの公式において、$t = \pi$とすれば、
\begin{equation*}
  e^{i\pi} = -1
\end{equation*}
という等式が成り立つ

これは、数学における3つの重要な数$e, \pi, i$の間に成り立つ美しい関係式である

\vskip\baselineskip

\subparagraph{三角関数のテイラー展開}\quad

$e^x$のべき級数展開に$x = it$を代入すると、
\begin{align}
  e^{it} & = 1 + it + \dfrac{(it)^2}{2!} + \dfrac{(it)^3}{3!} + \dfrac{(it)^4}{4!} + \cdots                                \\
         & = 1 + it - \dfrac{t^2}{2!} - i\dfrac{t^3}{3!} + \dfrac{t^4}{4!} + i\dfrac{t^5}{5!} - \cdots                     \\
         & = 1 - \dfrac{t^2}{2!} + \dfrac{t^4}{4!} - \cdots + i\left(t - \dfrac{t^3}{3!} + \dfrac{t^5}{5!} - \cdots\right)
\end{align}
となる

この実部と虚部を、$e^{it} = \cos t + i\sin t$の実部や虚部と比べると、
\begin{align}
  \cos t & = 1 - \dfrac{t^2}{2!} + \dfrac{t^4}{4!} - \cdots = \sum_{n=0}^{\infty}\dfrac{(-1)^n}{(2n)!}t^{2n}     \\
  \sin t & = t - \dfrac{t^3}{3!} + \dfrac{t^5}{5!} - \cdots = \sum_{n=0}^{\infty}\dfrac{(-1)^n}{(2n+1)!}t^{2n+1}
\end{align}
という、三角関数のテイラー展開が得られる

\end{document}
