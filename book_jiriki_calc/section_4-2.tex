\documentclass[../book_jiriki_calc]{subfiles}

\begin{document}

\section*{微分しても変わらない不思議な関数}

この式をぼんやりと眺めていると、
\begin{equation}
  \dfrac{d}{dx}x^n = nx^{n-1}
\end{equation}

\begin{itemize}
  \item 左辺における$\dfrac{d}{dx}$という記号に呼応して、右辺では$n$が飛び出すというふうにも見える
  \item 左辺では$x$の$n$乗だったものが、右辺では$n-1$乗になっている
\end{itemize}

\sectionline

$x^n$を$n$の階乗で割った$\dfrac{x^n}{n!}$という関数を考える

\vskip\baselineskip

この関数を微分すると、$\dfrac{1}{n!}$は微分の外に出せる
\begin{equation}
  \dfrac{d}{dx}\left(\dfrac{x^n}{n!}\right) = \dfrac{1}{n!}\left(\dfrac{d}{dx}x^n\right) = \dfrac{nx^{n-1}}{n!} = \dfrac{x^{n-1}}{(n-1)!}
\end{equation}

この式では、左辺と右辺で似た形が現れている

文字は左辺の$n$から右辺の$n-1$に化けるが、形は同じ

\vskip\baselineskip

$n$に具体的な数を入れて確かめてみる
\begin{itemize}
  \item $n=0$のとき、$\dfrac{d}{dx}\left(\dfrac{x^0}{0!}\right) = 0$
  \item $n=1$のとき、$\dfrac{d}{dx}\left(\dfrac{x^1}{1!}\right) = \dfrac{x^0}{0!}$
  \item $n=2$のとき、$\dfrac{d}{dx}\left(\dfrac{x^2}{2!}\right) = \dfrac{x^1}{1!}$
  \item $n=3$のとき、$\dfrac{d}{dx}\left(\dfrac{x^3}{3!}\right) = \dfrac{x^2}{2!}$
  \item $n=4$のとき、$\dfrac{d}{dx}\left(\dfrac{x^4}{4!}\right) = \dfrac{x^3}{3!}$
  \item $n=5$のとき、$\dfrac{d}{dx}\left(\dfrac{x^5}{5!}\right) = \dfrac{x^4}{4!}$
\end{itemize}
微分すると斜め右下にまったく同じ形の式が現れるというパターンが続く

上のリストでは$n=5$で止めているが、たとえば$n=100$までいっても同じパターンが続く

\vskip\baselineskip

そこで、$\dfrac{x^n}{n!}$を$n=0$から順に全部足すことを考え、それを$f(x)$とおく
\begin{align}
  f(x)              & = \dfrac{x^0}{0!} + \dfrac{x^1}{1!} + \dfrac{x^2}{2!} + \dfrac{x^3}{3!} + \cdots \\
  \dfrac{d}{dx}f(x) & = 0 + \dfrac{x^0}{0!} + \dfrac{x^1}{1!} + \dfrac{x^2}{2!} + \cdots
\end{align}
下の式は1個右にずれているので、途中で打ち切れば1個足りなくなるが、無限に足すと、上の式と下の式はぴったり一致している

したがって、
\begin{equation}
  \dfrac{d}{dx}f(x) = f(x)
\end{equation}
が成り立つことがわかる

つまり、\keyword{関数$f(x)$は微分したものが自分自身になっている!}

\vskip\baselineskip

いま無限級数として定義した関数$f(x)$を何通りかの記法で表しておく
\begin{align}
  f(x) & = \sum_{n=0}^{\infty} \dfrac{x^n}{n!}                                            \\
       & = \dfrac{x^0}{0!} + \dfrac{x^1}{1!} + \dfrac{x^2}{2!} + \dfrac{x^3}{3!} + \cdots \\
       & = 1 + x + \dfrac{x^2}{2} + \dfrac{x^3}{6} + \cdots
\end{align}
後にこの関数は、\keyword{指数関数}として$e^x$と書くことになる

\section*{ネイピアの数}

次の関数に$x=0$と$x=1$を代入してみる
\begin{equation}
  f(x) = 1 + x + \dfrac{1}{2!}x^2 + \dfrac{1}{3!}x^3 + \cdots
\end{equation}

\sectionline

\subparagraph{$x=0$を代入すると}

最初の$1$だけが残り、
\begin{equation}
  f(0) = 1
\end{equation}

\sectionline

\subparagraph{$x=1$を代入すると}

$1$を何乗しても$1$であるから、
\begin{equation}
  f(1) = 1 + 1 + \dfrac{1}{2!} + \dfrac{1}{3!} + \cdots
\end{equation}

\vskip\baselineskip

この$f(1)$の数値はどのくらいになるだろうか?
\begin{enumerate}
  \item 第1項は$1$
  \item 第2項も$1$
  \item 第3項は$0.5$
  \item 次は前の項を$3$で割るわけだから$0.166\ldots$
  \item 次はさらに$4$で割るから$0.041\ldots$
  \item 次はさらにそれを$5$で割って$0.008\ldots$
\end{enumerate}
ここまでの$6$項の和で$2.716\ldots$となる

\vskip\baselineskip

加える項は急速に$0$に近づく

項が100個くらいまで進むと、次に加える$\dfrac{1}{100!}$は小数点以下に$0$が150個以上並ぶくらい小さな数になる($10^{152}<100!<10^{164}$という不等式より)

\vskip\baselineskip

このように、無限級数$f(1)$は収束がとても速く、
\begin{equation}
  f(1) = 2.71828\ldots
\end{equation}
という数になる

\sectionline

\paragraph{定理}

\begin{equation}
  \lim_{n\to\infty}\left(1+\dfrac{1}{n}\right)^n = \sum_{k=0}^{\infty} \dfrac{1}{k!}
\end{equation}

\vskip\baselineskip

\subparagraph{証明のスケッチ}

二項展開を用いて、
\begin{align}
  \left(1+\dfrac{1}{n}\right)^n
   & = \sum_{k=0}^{n} \frac{n!}{k!(n-k)!} \cdot \frac{1}{n^k}
  %& = 1 + 1 + \dfrac{1}{2!}\left(1-\dfrac{1}{n}\right)                                             \\\\
  %& \phantom{=====} + \dfrac{1}{3!}\left(1-\dfrac{1}{n}\right)\left(1-\dfrac{2}{n}\right) + \cdots
\end{align}

ここで、$k=2$以降の各項は次のように展開する
\begin{align}
  \frac{n!}{2!(n-2)!} \cdot \frac{1}{n^2}
   & = \frac{n(n-1)}{2!} \cdot \frac{1}{n^2}       \\
   & = \dfrac{1}{2!} \cdot \dfrac{n-1}{n}          \\
   & = \frac{1}{2!} \left( 1 - \frac{1}{n} \right)
\end{align}

\begin{align}
  \frac{n!}{3!(n-3)!} \cdot \frac{1}{n^3}
   & = \frac{n(n-1)(n-2)}{3!} \cdot \frac{1}{n^3}                                 \\
   & = \dfrac{1}{3!} \cdot \dfrac{n-1}{n} \cdot \dfrac{n-2}{n}                    \\
   & = \frac{1}{3!} \left( 1 - \frac{1}{n} \right) \left( 1 - \frac{2}{n} \right)
\end{align}

これらを用いると、
\begin{multline}
  \left(1+\dfrac{1}{n}\right)^n = 1 + 1 + \dfrac{1}{2!}\left(1-\dfrac{1}{n}\right) \\
  + \dfrac{1}{3!}\left(1-\dfrac{1}{n}\right)\left(1-\dfrac{2}{n}\right) + \cdots
\end{multline}
$n$が大きくなると$\dfrac{1}{n}$は$0$に近づくので、$1-\dfrac{1}{n}$は$1$に近づき、
\begin{equation}
  \left(1+\dfrac{1}{n}\right)^n = 1 + 1 + \dfrac{1}{2!} + \dfrac{1}{3!} + \cdots
\end{equation}
となる$\qed$

\section*{無限級数$\displaystyle\sum_{n=0}^{\infty} \dfrac{x^n}{n!}$の収束}

$n$を大きくすると$n!$は急速に大きくなるので、$x=1$のときには無限級数$\displaystyle\sum_{n=0}^{\infty} \dfrac{x^n}{n!}=\sum_{n=0}^{\infty} \dfrac{1}{n!}$が収束することは納得できる

\vskip\baselineskip

では、$x>1$のときもこの無限級数は収束するといえるのだろうか?

\sectionline

そもそも数列の各項が$0$に近づかないと、その数列の総和は収束しないため、まず次の問いを考える
(以下では$x$を固定しておく)

\vskip\baselineskip

\paragraph{問題}

$n$をどんどん大きくしたとき、$\dfrac{x^n}{n!}$は$0$に近づくか?

\vskip\baselineskip

この問いは、$x^n$と$n!$の大きさを比べようという問題である

たとえば$n=100$とすると、実は$100!$の方が$10^{100}$よりも圧倒的に大きくなることをすでに示している

\vskip\baselineskip

$n=100$に限らず、「$x$を止めたとき、$x^n$と$n!$の比である$\dfrac{x^n}{n!}$は、$n$を大きくすると分母が圧倒的に大きくなり、比は$0$に近づく」ことが同様の議論で示される

\sectionline

無限級数の各項が$0$に近づいたとしても、「塵も積もれば山となる」(足し合わせると発散する)ことも起こり得る

\vskip\baselineskip

では、次の問題はどうだろうか?

\vskip\baselineskip

\paragraph{問題}

無限級数$\displaystyle\sum_{n=0}^{\infty} \dfrac{x^n}{n!}$は収束するか?

\vskip\baselineskip

実はこの無限級数は、等比級数$\displaystyle\sum_{n=0}^{\infty} \dfrac{1}{2^n}$よりももっと速く収束する

\vskip\baselineskip

\subparagraph{証明のスケッチ}\quad

$x$は固定して、$n$に関する和を考える

\vskip\baselineskip

整数$n$が十分に大きければ、
\begin{equation}
  \dfrac{\left|x\right|^n}{n!} < \dfrac{1}{2^n}
\end{equation}

これは、「無限級数$\displaystyle\sum_{n=0}^{\infty} \dfrac{x^n}{n!}$が等比級数$\displaystyle\sum_{n=0}^{\infty} \dfrac{1}{2^n}$より速く収束する」という1つの表現

\vskip\baselineskip

正確には、$8x^2+1$より大きいすべての自然数$n$に対して、
\begin{equation}
  \dfrac{\left|x\right|^n}{n!} < \dfrac{1}{2^n}
\end{equation}
が成り立つ

このことがいえれば、$8x^2$より大きい整数$N$に対して、無限級数$\displaystyle\sum_{n=N+1}^{\infty} \dfrac{x^n}{n!}$は次のように等比級数$\displaystyle\sum_{n=N+1}^{\infty} \dfrac{1}{2^n}$より速く収束する
\begin{align}
  \left| \sum_{n=N+1}^{\infty} \dfrac{x^n}{n!} \right|
   & \leq \sum_{n=N+1}^{\infty} \dfrac{\left| x \right|^n}{2^n} \\
   & < \sum_{n=N+1}^{\infty} \dfrac{1}{2^n}                     \\
   & = \dfrac{1}{2^{N}}
\end{align}

上の計算のうち、$\displaystyle\left| \sum_{n=N+1}^{\infty} \dfrac{x^n}{n!} \right| \leq \sum_{n=N+1}^{\infty} \dfrac{\left| x \right|^n}{2^n}$では、次のような三角不等式を利用している
\begin{align}
  \left| a_1 + a_2 + \dots + a_m \right|   & \leq |a_1| + |a_2| + \dots + |a_m| \\
  \left| \sum_{n=N+1}^{\infty} a_n \right| & \leq \sum_{n=N+1}^{\infty} |a_n|
\end{align}

\vskip\baselineskip

そこで、無限級数$\displaystyle\sum_{n=0}^{\infty} \dfrac{x^n}{n!}$を、$n=N$までの有限和と、$n=N+1$からの無限級数に分けて考える
\begin{equation}
  \sum_{n=0}^{\infty} \dfrac{x^n}{n!} = \sum_{n=0}^{N} \dfrac{x^n}{n!} + \sum_{n=N+1}^{\infty} \dfrac{x^n}{n!}
\end{equation}
このように考えると、左辺の無限級数が、右辺の有限和に収束することがわかる

\vskip\baselineskip

\subparagraph{不等式$\dfrac{\left|x\right|^n}{n!} < \dfrac{1}{2^n}$の証明}\quad\\

一般に$A \leq 0$のとき、$n > 2A^2 + 1$ならば、
\begin{equation}
  A^n < n!
\end{equation}
という不等式が成り立つことを示す

\vskip\baselineskip

$A=2|x|$の場合$\left( 2|x| \right)^n < n!$が、$\dfrac{\left|x\right|^n}{n!} < \dfrac{1}{2^n}$となる

\vskip\baselineskip

$n$が偶数($=2m$)の場合、$n > 2A^2$の$n$を$2m$に置き換えることで、$m > A^2$となり、
\begin{align}
  n! & = (2m)! = 2m \cdot (2m-1) \cdots 2 \cdot 1   \\
     & > m \cdot m \cdots m = m^m = m^{\frac{n}{2}} \\
     & > \left( A^2 \right)^{\frac{n}{2}} = A^n
\end{align}
が成り立つ

\vskip\baselineskip

$n$が奇数の場合、$n-1$は偶数なので、偶数の場合の結果から$(n-1)! > A^{n-1}$がいえる

さらに、$n > 2A^2 + 1 > A$なので、
\begin{align}
  n! & = n \cdot (n-1)!        \\
     & > n \cdot A^{n-1}       \\
     & > A \cdot A^{n-1} = A^n
\end{align}
となり、いずれの場合も$A^n < n!$が成り立つ$\qed$

\end{document}
