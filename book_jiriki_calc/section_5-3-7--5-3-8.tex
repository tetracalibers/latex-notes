\documentclass[../book_jiriki_calc]{subfiles}

\begin{document}

\section{臨界点と極大・極小}

\keyword{極大}(または\keyword{極小})とは、局所的に最大(または最小)であることを意味する

多変数関数の極大・極小については、個々の計算よりも、極大値や極小値をとる点で一般的にどのような性質が成り立っているのかを分析することを重視する

\sectionline

\subparagraph{極大・極小と最大・最小}

たとえば、$z=f(x,y)$のグラフ(曲面)を野山の地形と想像してみる

関数$f(x, y)$は地点$(x,y)$の標高を表している

\vskip\baselineskip

このとき、山頂はその付近ではもっとも高いので、$f(x, y)$は極大値をとる

山が1つしかなければ、この山頂で$f(x,y)$は最大値をとるが、実際には遠くにどんな高い山があるかわからない

同様に、谷底はそのあたりではいちばん低い場所であるけれども、遠くにもっと低いところがあるかもしれない

\vskip\baselineskip

遠くの情報がわからないとしても、この付近での山頂や谷底を探そう、というのが\keyword{極大・極小}問題

\vskip\baselineskip

一方、\keyword{最大・最小}は大域的な問題であり、それを判定するには遠くの情報も必要となるので、ずっと難しい問題になる

\sectionline

\subparagraph{極大・極小と接平面の向き}

野山の各地点でその斜面に接するように板を当て、その板の向きがどうなっているかを想像してみる

\vskip\baselineskip

この接している板(接平面)は、山頂や谷底では水平になっているはずである

もし接している板が水平でなければ、その付近には必ず、より高い・低い方向があるため、そこは山頂や谷底とは呼べなくなる

\vskip\baselineskip

逆に言うと、点$(x_0, y_0)$で$f(x, y)$が極大あるいは極小であれば、この点における接平面は水平である

\vskip\baselineskip

水平な平面を座標で表すと、$z=\text{定数}$という形になる

したがって、接平面の方程式において、$x$の係数、$y$の係数が$0$になっているはずである

\vskip\baselineskip

よって、極大あるいは極小になる点では、
\begin{equation*}
  \dfrac{\partial f}{\partial x}(x_0, y_0) = \dfrac{\partial f}{\partial y}(x_0, y_0) = 0
\end{equation*}
この式は、勾配ベクトル$\nabla f(x_0, y_0)$が零ベクトルになることを意味している

\sectionline

\paragraph{定理}

点$(x_0, y_0)$で$f(x, y)$が極大または極小ならば、
\begin{equation*}
  \nabla f(x_0, y_0) = (0,0)
\end{equation*}
が成り立つ

\sectionline

勾配ベクトル$\nabla f(x_0, y_0)$が零ベクトルとなる点$(x_0, y_0)$を、関数$f(x, y)$の\keyword{臨界点}あるいは\keyword{停留点}と言う

$xy$平面上の点$(x_0, y_0)$だけではなく、$xyz$空間内のグラフ$z=f(x, y)$上の点$(x_0, y_0, f(x_0, y_0))$のことも臨界点とよぶことがある

\vskip\baselineskip

定理の逆は必ずしも成り立たない

すなわち、臨界点だからと言って極大点・極小点とは限らない

\end{document}
