\documentclass[../book_jiriki_calc]{subfiles}

\begin{document}

\section*{微分方程式とは?}

未知の関数を$f(x)$とおいて、$f(x)$が満たすべき条件を等式で書き下したものが「関数に対する方程式」

関数に対する方程式の中に、$f(x)$の微分$f'(x)$や$f''(x)$が含まれているとき、その方程式を\keyword{微分方程式}という

\section*{もっとも簡単な微分方程式$f'(x)=0$}

定数関数の微分は恒等的に$0$になる

逆に、「微分$f'(x)$が恒等的に$0$ならば$f(x)$は定数である」という定理も示した

\vskip\baselineskip

この定理の仮定は、未知関数$f(x)$が微分方程式
\begin{equation}
  \dfrac{df}{dx}(x) = 0
\end{equation}
を満たしているということ

つまり、この定理は、「微分方程式$\dfrac{df}{dx}(x) = 0$の解は定数関数$f(x)=C$である」ことを主張している

\vskip\baselineskip

「どの点でも勾配がなければ、実は、その道は水平だ(高さが一定だ)」というのは、実生活では当たり前に見える

数学としては、無限小レベルの条件である微分方程式から、その解の大域的な性質を記述していることになる

\sectionline

以前述べた定理「すべての$x$で$f'(x)=a$ならば、$f(x)=ax+f(0)$である」も、微分方程式の言葉で記述できる

すなわち、$a$を定数とするとき、初期条件$f(0)=C$の下で、未知関数$f(x)$に関する微分方程式
\begin{equation}
  \dfrac{df}{dx}(x) = a
\end{equation}
の解が
\begin{equation}
  f(x) = ax + C
\end{equation}
であるという主張になる

\end{document}
