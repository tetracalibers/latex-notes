\documentclass[../book_jiriki_calc]{subfiles}

\begin{document}

\section{微分方程式とは?}

未知の関数を$f(x)$とおいて、$f(x)$が満たすべき条件を等式で書き下したものが「関数に対する方程式」

関数に対する方程式の中に、$f(x)$の微分$f'(x)$や$f''(x)$が含まれているとき、その方程式を\keyword{微分方程式}という

\section{もっとも簡単な微分方程式$f'(x)=0$}

定数関数の微分は恒等的に$0$になる

逆に、「微分$f'(x)$が恒等的に$0$ならば$f(x)$は定数である」という定理も示した

\br

この定理の仮定は、未知関数$f(x)$が微分方程式
\begin{equation}
  \dfrac{df}{dx}(x) = 0
\end{equation}
を満たしているということ

つまり、この定理は、「微分方程式$\dfrac{df}{dx}(x) = 0$の解は定数関数$f(x)=C$である」ことを主張している

\br

「どの点でも勾配がなければ、実は、その道は水平だ(高さが一定だ)」というのは、実生活では当たり前に見える

数学としては、無限小レベルの条件である微分方程式から、その解の大域的な性質を記述していることになる

\sectionline

以前述べた定理「すべての$x$で$f'(x)=a$ならば、$f(x)=ax+f(0)$である」も、微分方程式の言葉で記述できる

すなわち、$a$を定数とするとき、初期条件$f(0)=C$の下で、未知関数$f(x)$に関する微分方程式
\begin{equation}
  \dfrac{df}{dx}(x) = a
\end{equation}
の解が
\begin{equation}
  f(x) = ax + C
\end{equation}
であるという主張になる

\section{$f'(x)=\lambda f(x)$という微分方程式を解く}

指数関数$e^x$は$\displaystyle\sum_{n=0}^{\infty}\dfrac{x^n}{n!}$というべき級数展開を用いると導関数が自分自身になる、すなわち$(e^x)'=e^x$が成り立つことを以前示した

\br

その逆は成り立つだろうか?

これは「$f'(x)=f(x)$という微分方程式を解く」問題である

\sectionline

\paragraph{定理}

$\lambda$を定数とする

実数全体で定義された関数$F(t)$が
\begin{itemize}
  \item 微分方程式 $F'(t) = \lambda F(t)$
  \item 初期条件 $F(0) = A$
\end{itemize}
を満たすならば、$F(t) = Ae^{\lambda t}$である

\sectionline

\subparagraph{解を1つ見つける}\quad

まず、指数関数$e^{\lambda t}$が微分方程式$F'(t)=\lambda F(t)$を満たしていることを確かめる

\br

$e^{\lambda t}$を$f(x)=e^x$と$g(t)=\lambda t$の合成関数と見なすと、
\begin{itemize}
  \item $f(x)=e^x$に対しては$f'(x)=e^x$
  \item $g(t)=\lambda t$に対しては$g'(t)=\lambda$
\end{itemize}
が成り立つので、合成関数の微分の公式より、
\begin{align}
  \dfrac{d}{dt}e^{\lambda t} & = \frac{d}{dt} f(g(t))        \\
                             & = f'(g(t)) \cdot g'(t)        \\
                             & = e^{g(t)} \cdot \lambda      \\
                             & = e^{\lambda t} \cdot \lambda
\end{align}
となり、確かに$e^{\lambda t}$は微分方程式を満たす関数である

\sectionline

\subparagraph{すべての解を見つける}\quad

では、$e^{\lambda t}$と異なるタイプの解は存在するだろうか?

\br

微分方程式$F'(t)=\lambda F(t)$を満たす未知の解$F(t)$を既知の解$e^{\lambda t}$と比較するため、割り算してみる

\br

定理の結論は、
\begin{equation}
  \dfrac{\text{未知の解}}{\text{既知の解}} = \dfrac{F(t)}{e^{\lambda t}}
\end{equation}
が$t$によらない定数$A$になるということ

\br

そこで、割り算したものを$t$で微分してみる

商の微分の公式を使うと、
\begin{align}
  \dfrac{d}{dt}\left(\dfrac{F(t)}{e^{\lambda t}}\right) = \dfrac{F'(t)e^{\lambda t} - F(t)\lambda e^{\lambda t}}{(e^{\lambda t})^2}
\end{align}
ここで、$F'(t)=\lambda F(t)$であり、$(e^{\lambda t})'=\lambda e^{\lambda t}$なので、
\begin{equation}
  \dfrac{d}{dt}\left(\dfrac{F(t)}{e^{\lambda t}}\right) = \dfrac{\lambda F(t)e^{\lambda t} - \lambda F(t)e^{\lambda t}}{e^{2\lambda t}}
\end{equation}
この分子は、同じものどうしの引き算なので$0$になる
\begin{equation}
  \dfrac{d}{dt}\left(\dfrac{F(t)}{e^{\lambda t}}\right) = 0
\end{equation}

$t$で微分すると$0$になるということは、この関数$\dfrac{F(t)}{e^{\lambda t}}$は定数関数である

\br

$\dfrac{F(t)}{e^{\lambda t}}$の値は$t$によらないので、特に、$t=0$における値とも同じになる

初期条件$F(0)=A$を思い出すと、
\begin{equation}
  \dfrac{F(t)}{e^{\lambda t}} = \dfrac{F(0)}{e^{\lambda \cdot 0}} = \dfrac{A}{e^0} = A
\end{equation}
よって、
\begin{equation}
  F(t) = Ae^{\lambda t}
\end{equation}
これで、初期条件$F(0)=A$を満たす微分方程式$F'(t)=\lambda F(t)$の解は、$F(t)=Ae^{\lambda t}$のみであることが示された$\qed$

\sectionline

このように、指数関数の性質である
\begin{enumerate}
  \item 無限級数表示
  \item 指数法則
  \item 微分方程式
\end{enumerate}
は、同じ関数の3つの異なる側面を表している

\br

ここでは、微分方程式を解くことによって、3から1や2の性質を復元できることを確かめた

\sectionline

\subparagraph{パラメータ$\lambda$の符号}\quad

微分方程式$F'(t)=\lambda F(t)$のパラメータ$\lambda$は、その解の挙動に大事な役割を持つ

\br

初期条件$A>0$とすると、$F(t) = Ae^{\lambda t}$のグラフの形状は$\lambda$の符号によって異なる

\br

現象を記述するとき、ある量が変数$t$を用いて$Ae^{\lambda t}$という形で(近似的に)表されることがよくある

こういった場合、その量は
\begin{itemize}
  \item $\lambda>0$のとき、\keyword{指数的に増大する}(ネズミ算式に増える)
  \item $\lambda<0$のとき、\keyword{指数的に減少する}
\end{itemize}
と言うことがある

\sectionline

\subparagraph{$A=\lambda=1$の場合と自然対数}\quad

$A=\lambda=1$の場合の指数関数$e^t$は、$t$が決まれば$e^t$の値が定まり、そのグラフ$y=e^t$は単調増加になっている

\br

逆に、グラフを見ると、$y>0$に対して$y=e^t$となる実数$t$が1つ定まることがわかる

この値を$y$の\keyword{自然対数}といい、$t = \log y$と表記する

\end{document}
