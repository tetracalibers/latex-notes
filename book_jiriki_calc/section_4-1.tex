\documentclass[../book_jiriki_calc]{subfiles}

\begin{document}

\section{関数の局所的な様子を見る}

簡単な関数のグラフは拡大していくと急に様子が変わったりせず、むしろ、だんだん安定したものになると考えられる

局所的な部分を拡大すると安定した姿になるとき、その様子を数学的にとらえる概念が\keyword{微分}

\br

ものによっては、拡大するとどんどん見え方が変わるものもある

拡大を何度繰り返しても同じ複雑さを保つ数学的構造(フラクタル)も自然界には現れる

\br

拡大すれば何でも簡単になるわけではないが、微分では、拡大したとき安定していく「素直」なものを主な対象とする

つまり、\keyword{微分は局所を分析するのに強力な手法だが、万能ではない}

\section{微分の定義}

\keyword{関数}は変化の法則性をとらえる数学的言語

数$x$に対して数$f(x)$が定まるとき、$f(x)$を変数$x$の関数という

\sectionline

座標$(x,f(x))$を$xy$平面でプロットした曲線を関数$f(x)$の\keyword{グラフ}という

これは、$x$座標の点$x$における高さが$f(x)$となる曲線

\sectionline

この曲線の局所的な様子を見るのに、変数$x$を$x+h$に動かしてみる

そうすると、関数の値は$f(x)$から$f(x+h)$に変わる

\br

「素直」な関数のグラフをどんどん拡大すると、拡大部分はだんだん直線のように見えるだろう、と考えられる

$h$が小さいとき、斜めの曲線がほぼ一定の傾きの直線に見えるというのは、関数の値の変化量$f(x+h)-f(x)$が$h$にほぼ正比例するということ

\br

式で表すと、$x$から$x+h$の区間のグラフを直線とみなしたときの勾配
\begin{equation}
  \dfrac{f(x+h)-f(x)}{h}
\end{equation}
は、$h$が$0$に近づくとある1つの数に近づく、すなわち、収束するはずである

\sectionline

\paragraph{定義}

$h$を$0$に近づけると、$\dfrac{f(x+h)-f(x)}{h}$がある数に収束するとき、$f(x)$は$x$において\keyword{微分可能}であるという

このとき、極限値を
\begin{equation}
  f'(x) = \lim_{h \to 0} \dfrac{f(x+h)-f(x)}{h}
\end{equation}
と書き、$f(x)$の\keyword{微分}または\keyword{微分係数(微係数)}という

\sectionline

\subparagraph{定数関数の微分}

「収束する」ことを「限りなく近づく」と言うこともある

日常的な言葉だと「限りなく近づく」には「その値に達していない」というニュアンスを感じるが、数学では、最初からずっと同じ値のときも「収束する」場合に含める

\br

$f(x)$が$x$の値によらないとき、$f(x)$を\keyword{定数関数}という

このときは$h$がどんな数でも$f(x+h)-f(x)=0$となるので、定数関数の微分は$0$である

\sectionline

\subparagraph{微分係数が定まらない例}

\begin{equation}
  \lim_{h\to 0}\dfrac{f(x+h)-f(x)}{h}
\end{equation}
が収束しない状況の例として、$y=\left|x\right|$を考える

$f(x)=|x|$の場合、$x=0$で
\begin{equation}
  \lim_{h\to 0}\dfrac{f(x+h)-f(x)}{h}
\end{equation}
を計算しようとすると、

$h>0$のときは
\begin{equation}
  \dfrac{f(h)-f(0)}{h} = \dfrac{h}{h} = 1
\end{equation}
$h<0$のときは
\begin{equation}
  \dfrac{f(h)-f(0)}{h} = \dfrac{-h}{h} = -1
\end{equation}
となり、$h$を正から$0$に近づけるときと、負から$0$に近づけるときとで、$\dfrac{f(h)-f(0)}{h}$の極限の値が異なってしまうので、微分係数$f'(0)$が定まらない

\sectionline

\paragraph{定理}

$a<x<b$で定義された、微分可能な関数$f(x)$が$x=c$で最大値または最小値をとるならば、$f'(c)=0$である

\sectionline

\subparagraph{$f'(c)$が最大値となる場合の証明}

\begin{equation}
  f'(c) = \lim_{h\to 0}\dfrac{f(c+h)-f(c)}{h}
\end{equation}
において、$f(c)$が最大値であることから、
\begin{align}
  f(c)        & \geq f(c+h) \\
  f(c+h)-f(c) & \leq 0
\end{align}
したがって、$h>0$のときは、
\begin{equation}
  \dfrac{f(c+h)-f(c)}{h} \leq 0
\end{equation}
となり、$h$を正の側から$0$に近づけた極限値として$f'(c) \leq 0$が成り立つ

一方、$h<0$のときは、
\begin{equation}
  \dfrac{f(c+h)-f(c)}{h} \geq 0
\end{equation}
となり、$h$を負の側から$0$に近づけた極限値として$f'(c) \geq 0$が成り立つ

$f'(c) \leq 0$かつ$f'(c) \geq 0$なので、$f'(c)=0$が導かれた$\qed$

\sectionline

\subparagraph{$f'(c)$が最小値となる場合の証明}

$f'(c)$が最大値となる場合と同様に示される$\qed$

\section{導関数}

$x$を止めて考えると、$f(x)$の微分は1つの数
\begin{equation}
  \dfrac{df}{dx}(x) = \lim_{h\to 0}\dfrac{f(x+h)-f(x)}{h}
\end{equation}

\br

また別の視点として、
\begin{itemize}
  \item $x$に数を与えると、何か1個、数が出てくる
  \item また別の$x$に対しては、別の数が出る
\end{itemize}
そう思うと、$x$から$\dfrac{df}{dx}(x)$への対応は1つの関数を与えていると考えることができる

\br

このように、$\dfrac{df}{dx}(x)$を$x$の関数と見たとき、それを$f(x)$の\keyword{導関数}という

\sectionline

「微分」と「導関数」は視点の違いで使い分けられる言葉

\begin{itemize}
  \item $x$を止めて$\dfrac{df}{dx}(x)$という1個の数(微分係数)に注目するのか
  \item $x$を変数と思って$\dfrac{df}{dx}(x)$を関数とみなす(導関数として扱う)のか
\end{itemize}

後者の立場に立って、$\dfrac{df}{dx}(x)$を関数だと思うと、さらに微分を考えることができる

\sectionline

微分できないからといってそこで終わりではない

\br

たとえば、関数概念を拡張した\keyword{超関数}の理論は、極限
\begin{equation}
  \lim_{h\to 0}\dfrac{f(x+h)-f(x)}{h}
\end{equation}
が存在しない場合にも、より広く「微分」という概念をとらえる枠組みを与えるもの

\section{単項式$x^n$の微分}

$f(x+h)=(x+h)^n$の二項展開
\begin{equation}
  f(x+h) = x^n + nx^{n-1}h + {}_n C_2 x^{n-2}h^2 + \cdots + h^n
\end{equation}
を用いると、
\begin{align}
  f(x+h)-f(x) & = (x+h)^n - x^n                                      \\
              & = x^n + nx^{n-1}h + {}_n C_2 x^{n-2}h^2              \\
              & \phantom{=====} + \cdots + h^n - x^n                 \\
              & = nx^{n-1}h + {}_n C_2 x^{n-2}h^2 + \cdots + h^{n-1}
\end{align}
上の式変形で、最初の$x^n$は最後の$-x^n$と相殺されている

両辺を$h$で割ると、
\begin{align}
  \dfrac{f(x+h)-f(x)}{h}
   & = \dfrac{nx^{n-1}h + {}_n C_2 x^{n-2}h^2 + \cdots + h^{n-1}}{h} \\
   & = nx^{n-1} + {}_n C_2 x^{n-2}h + \cdots + h^{n-1}
\end{align}

$h$が$0$に近づくと、
\begin{itemize}
  \item $h$に無関係な最初の項$nx^{n-1}$はそのまま残る
  \item 次の$h$の項は$0$に近づく
  \item その後の$h^2, h^3, \cdots, h^{n-1}$の項はさらに速く$0$に近づく
\end{itemize}

というわけで、$h$を$0$に近づけると$nx^{n-1}$に収束し、
\begin{equation}
  \dfrac{d}{dx}x^n = nx^{n-1}
\end{equation}
が成り立つ

\end{document}
