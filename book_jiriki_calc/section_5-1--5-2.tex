\documentclass[../book_jiriki_calc]{subfiles}

\begin{document}

\section{偏微分 --- 多変数関数の微分}

ものごとには通常、単一の要因だけではなく、複数の要因が絡みあっている

さまざまな要因が関係する現象を数量的に分析するためには、1つの変数だけでなく、複数の変数を含む関数を使う

このようにいくつもの変数があって、それによって値が定まるような関数を\keyword{多変数関数}という

\sectionline

複数の要因が絡む状況を判断する際には、すべての要因を同時に考えるのではなく、まず1つの要因に着目し、次に視点を変えて別の要因を考え、そして最後に、個別に考察した要因を統合して考えることがある

\br

\keyword{偏微分}のアイデアも、そのアプローチに似ている

多変数関数の偏微分では、1つの変数に注目し、それ以外の変数をいったん固定して定義する

そして、多変数関数の局所的な様子を分析するためには、各変数ごとに得られた偏微分の情報をどのように統合するかが重要になる

\section{多変数関数の微分をイメージする}

\subparagraph{1変数関数の例:道の標高}\quad

1変数関数の場合、坂道の勾配は水平方向の座標を変数とする標高の微分だった

これは「道」という1次元の例

\br

\subparagraph{2変数関数の例:野山の標高}\quad

状況を変えて、野山にいるとする

平面図で位置を指定するためには、たとえば東に$x$メートル、北に$y$メートルといった具合に、2つの変数があればよい

その地点の高さは2変数関数$f(x, y)$として表される

この関数が極大となる地点は山頂に対応する

さらに、斜面の勾配は偏微分によって記述できる

野山の形状は、偏微分という抽象的な概念を「感じられる」身近な例

\br

\subparagraph{3変数関数の例:温度や気圧}\quad

3次元空間の各点での温度や気圧などは、3変数関数の例となる

\br

\subparagraph{多変数関数の例:効用関数}\quad

2種類以上のモノ(経済学では財)を消費する際の効用関数も、多変数関数とみなすことができる

\br

たとえばジュースの量を$x$だけ飲み、お菓子の量を$y$だけ食べることと、$x'$だけ飲んで$y'$だけ食べることのどちらが好きかの選好を描写するのに2変数の関数を使う

前者$(x, y)$の選択の方が後者$(x', y')$の選択よりも好ましい場合には、$f(x, y) > f(x', y')$という不等式を満たしているとする

このとき、それぞれの財に関する変化率(限界効用)は偏微分で表される

\br

ジュースだけの場合は、喉が渇いているときにジュースを飲むと嬉しいが、たくさん飲むと飽きてくるといった経験則が、適切な仮定を満たす選好を描写する効用関数の1階微分(限界効用)や2階微分で表現された

一方、お菓子も合わせて食べるとどうだろうか?

ジュースとお菓子を適切な比率で組み合わせると一層楽しそうだし、飽きにくくなるかもしれない

逆に「相性」が悪い組み合わせもありそう

\br

複数の要因に対しても、選好を描写する効用関数は無数にあるが、その共通の性質は、その選好から導かれると考えられる

偏微分には、複数の財の「相性」や「相乗効果」といった、\keyword{多変数関数ならではの現象}についての大事な情報も反映されている

\section{偏微分の定義}

2変数関数$f(x, y)$の$x$に関する\keyword{偏微分}は、「$y$を止めて$x$に関して微分する」という意味で、
\begin{equation*}
  \frac{\partial f}{\partial x} = \lim_{h \to 0} \frac{f(x + h, y) - f(x, y)}{h}
\end{equation*}
と定義される

$y$を止めると、$x$だけが変数となるので、$x$の1変数関数と思って普通に微分する

\br

$\dfrac{\partial f}{\partial x}(x,y)$を変数$x$に関する\keyword{編微分係数}とよぶこともある

\br

逆に、$x$を止め、$y$だけを動かして微分することで、
\begin{equation*}
  \frac{\partial f}{\partial y} = \lim_{h \to 0} \frac{f(x, y + h) - f(x, y)}{h}
\end{equation*}
という$y$に関する偏微分が定義される

\br

\subparagraph{偏微分の記号}

偏微分の記号にはさまざまな流儀があり、以下の記号はすべて同じ意味で使う
\begin{equation*}
  \frac{\partial f}{\partial x}(x, y) = \dfrac{\partial f}{\partial x} =f_x = f_x(x, y)
\end{equation*}

\br

変数$(x,y)$を省略した$\dfrac{\partial f}{\partial x}$や$f_x$という記法は文字数が少なくて便利だが、「$y$を止めて」という約束が記号に反映されていない

異なる解釈が生じる可能性があるときには注意が必要

\sectionline

偏微分で混乱する大きな原因は、何を止めているのかが不明瞭になること

一定にするものを変えると、偏微分の意味も値も異なってしまう可能性がある

\br

たとえば、先ほどの効用関数の例では、$\dfrac{\partial f}{\partial x}$はジュースを飲む量$x$を増やしたときの効用の変化率を表す

しかし、何を一定にしているかを明示しないと、その内容がまったく異なることになる
\begin{itemize}
  \item お菓子の量を一定にして、ジュースの量$x$を増やす
  \item 予算を一定にして、ジュースの量$x$を増やす
\end{itemize}
前者では単純にジュースの量$x$が増えるのを好む一方で、後者では予算が一定であるため、ジュースの量$x$を増やすとお菓子の量が減ることになる

そのため、後者ではジュースの量を増やすのを好まない人もいる

この場合は、効用関数を$x$に関して微分したときの結果も異なってくる

\br

偏微分では、微分する変数だけではなく、その際に\keyword{いったん固定している変数は何であるか}を意識して式や文章を見る必要がある

\sectionline

偏微分を計算するときは微分していない変数をいったん止めるが、偏微分を行った後は$x$も$y$も自由に動かせる

\br

偏微分$f_x(x, y)$や$f_y(x, y)$は、$x$と$y$を与えると1つの数が決めるという意味で、再び$x, \, y$の関数と見なすことができる

このように$x,\,y$の関数と見なすときは、\keyword{偏導関数}とよぶ

\br

関数だと思えば、さらに偏微分を繰り返すことができる

こうして偏微分を2回繰り返した2階の偏微分には、いくつか可能性がある

\begin{itemize}
  \item $f$を$x$で微分すると$f_x = \dfrac{\partial f}{\partial x}$
        \begin{itemize}
          \item $f_x$を$x$で微分すると$f_{xx} = \dfrac{\partial^2 f}{\partial x^2}$
          \item $f_x$を$y$で微分すると$f_{xy} = \dfrac{\partial^2 f}{\partial x \partial y}$
        \end{itemize}
  \item $f$を$y$で微分すると$f_y = \dfrac{\partial f}{\partial y}$
        \begin{itemize}
          \item $f_y$を$x$で微分すると$f_{yx} = \dfrac{\partial^2 f}{\partial y \partial x}$
          \item $f_y$を$y$で微分すると$f_{yy} = \dfrac{\partial^2 f}{\partial y^2}$
        \end{itemize}
\end{itemize}

2階の偏微分$f_{xy}$と$f_{yx}$の違いは、どちらを先に偏微分するかという点だが、多くの場合はこの順序を気にする必要はない

「素直」な関数ならば偏微分の順序が交換でき、$f_{xy} = f_{yx}$が成り立つ

\end{document}
