\documentclass[../book_jiriki_calc]{subfiles}

\begin{document}

\section{偏微分 --- 多変数関数の微分}

ものごとには通常、単一の要因だけではなく、複数の要因が絡みあっている

さまざまな要因が関係する現象を数量的に分析するためには、1つの変数だけでなく、複数の変数を含む関数を使う

このようにいくつもの変数があって、それによって値が定まるような関数を\keyword{多変数関数}という

\sectionline

複数の要因が絡む状況を判断する際には、すべての要因を同時に考えるのではなく、まず1つの要因に着目し、次に視点を変えて別の要因を考え、そして最後に、個別に考察した要因を統合して考えることがある

\vskip\baselineskip

\keyword{偏微分}のアイデアも、そのアプローチに似ている

多変数関数の偏微分では、1つの変数に注目し、それ以外の変数をいったん固定して定義する

そして、多変数関数の局所的な様子を分析するためには、各変数ごとに得られた偏微分の情報をどのように統合するかが重要になる

\end{document}
