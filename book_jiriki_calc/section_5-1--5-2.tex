\documentclass[../book_jiriki_calc]{subfiles}

\begin{document}

\section{偏微分 --- 多変数関数の微分}

ものごとには通常、単一の要因だけではなく、複数の要因が絡みあっている

さまざまな要因が関係する現象を数量的に分析するためには、1つの変数だけでなく、複数の変数を含む関数を使う

このようにいくつもの変数があって、それによって値が定まるような関数を\keyword{多変数関数}という

\sectionline

複数の要因が絡む状況を判断する際には、すべての要因を同時に考えるのではなく、まず1つの要因に着目し、次に視点を変えて別の要因を考え、そして最後に、個別に考察した要因を統合して考えることがある

\vskip\baselineskip

\keyword{偏微分}のアイデアも、そのアプローチに似ている

多変数関数の偏微分では、1つの変数に注目し、それ以外の変数をいったん固定して定義する

そして、多変数関数の局所的な様子を分析するためには、各変数ごとに得られた偏微分の情報をどのように統合するかが重要になる

\section{多変数関数の微分をイメージする}

\subparagraph{1変数関数の例:道の標高}\quad

1変数関数の場合、坂道の勾配は水平方向の座標を変数とする標高の微分だった

これは「道」という1次元の例

\vskip\baselineskip

\subparagraph{2変数関数の例:野山の標高}\quad

状況を変えて、野山にいるとする

平面図で位置を指定するためには、たとえば東に$x$メートル、北に$y$メートルといった具合に、2つの変数があればよい

その地点の高さは2変数関数$f(x, y)$として表される

この関数が極大となる地点は山頂に対応する

さらに、斜面の勾配は偏微分によって記述できる

野山の形状は、偏微分という抽象的な概念を「感じられる」身近な例

\vskip\baselineskip

\subparagraph{3変数関数の例:温度や気圧}\quad

3次元空間の各点での温度や気圧などは、3変数関数の例となる

\vskip\baselineskip

\subparagraph{多変数関数の例:効用関数}\quad

2種類以上のモノ(経済学では財)を消費する際の効用関数も、多変数関数とみなすことができる

\vskip\baselineskip

たとえばジュースの量を$x$だけ飲み、お菓子の量を$y$だけ食べることと、$x'$だけ飲んで$y'$だけ食べることのどちらが好きかの選好を描写するのに2変数の関数を使う

前者$(x, y)$の選択の方が後者$(x', y')$の選択よりも好ましい場合には、$f(x, y) > f(x', y')$という不等式を満たしているとする

このとき、それぞれの財に関する変化率(限界効用)は偏微分で表される

\vskip\baselineskip

ジュースだけの場合は、喉が渇いているときにジュースを飲むと嬉しいが、たくさん飲むと飽きてくるといった経験則が、適切な仮定を満たす選好を描写する効用関数の1階微分(限界効用)や2階微分で表現された

一方、お菓子も合わせて食べるとどうだろうか?

ジュースとお菓子を適切な比率で組み合わせると一層楽しそうだし、飽きにくくなるかもしれない

逆に「相性」が悪い組み合わせもありそう

\vskip\baselineskip

複数の要因に対しても、選好を描写する効用関数は無数にあるが、その共通の性質は、その選好から導かれると考えられる

偏微分には、複数の財の「相性」や「相乗効果」といった、\keyword{多変数関数ならではの現象}についての大事な情報も反映されている

\end{document}
