\documentclass[../book_jiriki_calc]{subfiles}

\begin{document}

\section*{近似と誤差}

曲線を局所的に近似する場合、接線で近似するのが最初のステップになる

次のステップとして、接線からの乖離を正確に知りたい

たとえば道路が急カーブしているときは、直線ではなく円弧で近似する方がより正確になる

関数のグラフの各点で、その曲がり方を表す円弧の半径を求めるのには2階の微分を使う

\vskip\baselineskip

実用上は、1階微分と2階微分を用いると、多くの場合、局所的に十分良い近似ができるが、それでも微小な乖離は生じる

この微小な誤差は、3階微分を使うと評価できる

これを続け、1階微分だけではなく、2階、3階、$\ldots$と高階の微分を用い、必要な精度を実現するためには近似をどのように行えばよいかを指し示すのが\keyword{テイラー展開}とその\keyword{剰余項}である

\sectionline

誤差評価を行う際には、範囲をきちんと意識する必要がある
\begin{itemize}
  \item 翌日の天気が予測できても、1ヶ月先の天気予報は難しい
  \item 坂道の勾配を見て100m先の高低差は推測できても、10km先の高低差はわからない
\end{itemize}

\section*{誤差と誤差率}

測定値や何らかの概算値が真の値とどれくらい異なるかは、
\begin{equation*}
  \text{誤差} = \left|\text{測定値} - \text{真の値}\right|
\end{equation*}
という絶対量で表された

\vskip\baselineskip

一方、相対的な比率として定義される、
\begin{equation*}
  \text{誤差率} = \left|\frac{\text{誤差}}{\text{真の値}}\right| = \left|\frac{\text{測定値} - \text{真の値}}{\text{真の値}}\right|
\end{equation*}
も大事な視点である

\vskip\baselineskip

実用上は、分母を「測定値」に取り換えた、
\begin{equation*}
  P = \left|\frac{\text{誤差}}{\text{測定値}}\right| = \left|\frac{\text{測定値} - \text{真の値}}{\text{測定値}}\right|
\end{equation*}
で代用することもある

\vskip\baselineskip

$P$が小さいときは誤差率として代用できることは、次のように確認できる

\sectionline

\paragraph{定理}

$P<\dfrac{1}{101}$ならば、誤差率は$1\%$未満

\sectionline

\subparagraph{証明}

$t=\dfrac{\text{真の値}}{\text{測定値}}$とおくと、
\begin{align*}
  \text{誤差} & = \left|\text{測定値} - \text{真の値}\right|  \\
            & = \left|\text{測定値} - t\text{測定値}\right| \\
            & = \left|1-t\right|\text{測定値}
\end{align*}
より、
\begin{gather}
  P          = \left|\frac{\text{測定値} - \text{真の値}}{\text{測定値}}\right| = \left|1-t\right| \\
  \text{誤差率} = \left|\frac{\text{誤差}}{\text{真の値}}\right| = \left|\frac{\left|1-t\right|\text{測定値}}{\text{真の値}}\right| =\left|\dfrac{1-t}{t}\right|
\end{gather}
と書き表せる

\vskip\baselineskip

$P<\dfrac{1}{101}$ならば、
\begin{align*}
  -P  & > -\frac{1}{101}                    \\
  1-P & > 1-\frac{1}{101} = \frac{100}{101}
\end{align*}
であり、$t$について、三角不等式より、
\begin{gather}
  1 - \left| 1-t\right| = 1 - \left| t-1\right| \leq \left| 1+ (t-1)\right| = t \\
  \dfrac{100}{101} < 1-P \leq t
\end{gather}
これを用いると、
\begin{align*}
  \text{誤差率} & = \left|\frac{1-t}{t}\right|                                                                                                                                        \\
             & < \left|\dfrac{1-\dfrac{100}{101}}{\dfrac{100}{101}}\right| = \dfrac{\dfrac{1}{101}}{\dfrac{100}{101}} = \dfrac{\dfrac{1}{101}\cdot 101}{\dfrac{100}{101}\cdot 101} \\
             & = \dfrac{1}{100}
\end{align*}
として、誤差率は$1\%$未満であることが示された$\qed$

\sectionline

真の値が分からなくとも、何か別の情報や論理から、誤差を「上から評価する」すなわち、「誤差が〜以下である」という形の評価式が得られることがある

\section*{弧長の近似と誤差評価}

一般に、ある時点で誤差が生じると、その後の誤差が増幅して予期しない間違いが生じることがある

したがって、概算が信頼できるとするためには、\keyword{誤差評価}という別の論理が必要になる

誤差評価を行う際に用いるトリックが、\keyword{存在定理}である

\end{document}
