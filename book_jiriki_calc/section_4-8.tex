\documentclass[../book_jiriki_calc]{subfiles}

\begin{document}

\section*{近似と誤差}

曲線を局所的に近似する場合、接線で近似するのが最初のステップになる

次のステップとして、接線からの乖離を正確に知りたい

たとえば道路が急カーブしているときは、直線ではなく円弧で近似する方がより正確になる

関数のグラフの各点で、その曲がり方を表す円弧の半径を求めるのには2階の微分を使う

\vskip\baselineskip

実用上は、1階微分と2階微分を用いると、多くの場合、局所的に十分良い近似ができるが、それでも微小な乖離は生じる

この微小な誤差は、3階微分を使うと評価できる

これを続け、1階微分だけではなく、2階、3階、$\ldots$と高階の微分を用い、必要な精度を実現するためには近似をどのように行えばよいかを指し示すのが\keyword{テイラー展開}とその\keyword{剰余項}である

\sectionline

誤差評価を行う際には、範囲をきちんと意識する必要がある
\begin{itemize}
  \item 翌日の天気が予測できても、1ヶ月先の天気予報は難しい
  \item 坂道の勾配を見て100m先の高低差は推測できても、10km先の高低差はわからない
\end{itemize}

\end{document}
