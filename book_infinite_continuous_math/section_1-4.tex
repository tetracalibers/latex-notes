\documentclass[../book_infinite_continuous_math]{subfiles}

\begin{document}

\section{テイラーの定理の内容}

\subsection{関数とは、そもそも何か?}

関数が$x$を$y$に変換する機能だとすれば、ブラックボックス(仕掛けの見えない変換装置)ではなく誰でも使いやすい形でボックスの中身、つまり仕掛けが提示されたほうがずっと理解しやすいだろうし、関数の機能の解析も容易になる

そこで、数学ではそれら関数の機能を「数式」で表現してきた

\br

たとえば、3次の多項式で表される関数
\begin{equation*}
  y = f(x) = x^3 - 3x + 1
\end{equation*}
が与えられたとき、この対応規則はきっちりと明示されている

別の言葉で言えば、「$x$を加工し$y$にする手続きが示されている」といってもよい

\sectionline
\subsection{微分可能な関数のマクローリン展開}

しかし、指数関数・対数関数・三角関数のように、「多項式」で表せない関数もある

$\sin 2$の値はいくつですか?といわれても、入力$2$をどう加工して出力するのかは明確ではない

私たちは三角関数を$y=\sin x$と書くことによって、その正体をつかんだような気になっているが、実はこれは単にラベルを貼ったにすぎないので、三角関数は相変わらずブラックボックスのままである

\br

私たちが具体的にその値を計算できる関数の代表が多項式(関数)

三角関数や対数関数は式で表されているような気がするが、実はそれは関数にラベルを貼っただけと考えるべきで、具体的に計算可能な関数とは多項式だと思ったほうがよい

\br

実はテイラーの定理は、このブラックボックスの仕組みをはっきりさせる1つの方法を与えている

\begin{oframed}
  \paragraph{マクローリンの定理}
  何回でも微分できる関数について、
  \begin{multline*}
    f(x) = f(0) + f'(0)x + \dfrac{f''(0)}{2!}x^2 \\ + \cdots + \dfrac{1}{(n-1)!}f^{(n-1)}(0)x^{n-1} + R_n
  \end{multline*}
  が成り立つ

  ただし、
  \begin{equation*}
    R_n = \dfrac{1}{n!}f^{(n)}(c)x^n
  \end{equation*}
  である
\end{oframed}

この定理はただ単にテイラーの定理で$a=0, b=x$とおいたものに他ならないが、大変おもしろい定理である

\br

この式は最後の$R_n$を除いて、$x$の多項式の形をしている

関数が決まればその$r$回目の導関数の$0$での値は定数として決まってしまうから、確かに$R_n$は多項式

\br

とくに、$n$をどんどん大きくしていくとき、最後の項(これを\keyword{剰余項}という)が$R_n \to 0$を満たすときは、この項をとってしまい、$f(x)$を
\begin{equation*}
  f(x) = f(0) + f'(0)x + \dfrac{f''(0)}{2!}x^2 + \dfrac{f'''(0)}{3!}x^3 + \cdots
\end{equation*}
と無限級数の形で書いて、これを$f(x)$の\keyword{マクローリン展開}という

少し荒っぽくいえば、何回でも微分できる関数は無限次元の多項式で表すことができるということ

\br

例として、$f(x) = \sin x$をマクローリン展開してみる
\begin{equation*}
  (\sin x)' = \cos x, \quad (\cos x)' = -\sin x
\end{equation*}
を使って導関数の$x=0$における値を計算していけば、$\sin 0 = 0, \, \cos 0 = 1$などより、
\begin{equation*}
  \sin x = x - \dfrac{x^3}{3!} + \dfrac{x^5}{5!} - \dfrac{x^7}{7!} + \cdots
\end{equation*}
となる(剰余項は$0$に収束することは知られている)

\br

テイラーの定理(マクローリン展開)を使うことにより、ブラックボックスであった$\sin x$という関数の仕掛けを多項式で表すことができた

この仕掛けを使えば、$\sin 2$の値を必要な精度で計算することができる

\end{document}
