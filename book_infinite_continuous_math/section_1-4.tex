\documentclass[../book_infinite_continuous_math]{subfiles}

\begin{document}

\section{テイラーの定理の内容}

\subsection{関数とは、そもそも何か?}

関数が$x$を$y$に変換する機能だとすれば、ブラックボックス(仕掛けの見えない変換装置)ではなく誰でも使いやすい形でボックスの中身、つまり仕掛けが提示されたほうがずっと理解しやすいだろうし、関数の機能の解析も容易になる

そこで、数学ではそれら関数の機能を「数式」で表現してきた

\br

たとえば、3次の多項式で表される関数
\begin{equation*}
  y = f(x) = x^3 - 3x + 1
\end{equation*}
が与えられたとき、この対応規則はきっちりと明示されている

別の言葉で言えば、「$x$を加工し$y$にする手続きが示されている」といってもよい

\sectionline
\subsection{微分可能な関数のマクローリン展開}

しかし、指数関数・対数関数・三角関数のように、「多項式」で表せない関数もある

$\sin 2$の値はいくつですか?といわれても、入力$2$をどう加工して出力するのかは明確ではない

私たちは三角関数を$y=\sin x$と書くことによって、その正体をつかんだような気になっているが、実はこれは単にラベルを貼ったにすぎないので、三角関数は相変わらずブラックボックスのままである

\br

私たちが具体的にその値を計算できる関数の代表が多項式(関数)

三角関数や対数関数は式で表されているような気がするが、実はそれは関数にラベルを貼っただけと考えるべきで、具体的に計算可能な関数とは多項式だと思ったほうがよい

\br

実はテイラーの定理は、このブラックボックスの仕組みをはっきりさせる1つの方法を与えている

\begin{oframed}
  \paragraph{マクローリンの定理}
  何回でも微分できる関数について、
  \begin{multline*}
    f(x) = f(0) + f'(0)x + \dfrac{f''(0)}{2!}x^2 \\ + \cdots + \dfrac{1}{(n-1)!}f^{(n-1)}(0)x^{n-1} + R_n
  \end{multline*}
  が成り立つ

  ただし、
  \begin{equation*}
    R_n = \dfrac{1}{n!}f^{(n)}(c)x^n
  \end{equation*}
  である
\end{oframed}

\end{document}
