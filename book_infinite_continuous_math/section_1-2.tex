\documentclass[../book_infinite_continuous_math]{subfiles}

\begin{document}

\section{微分係数と微分、導関数}

\subsection{リミット($\lim$)をやめてしまおう?}

微分係数の定義の式でリミットをとってしまう
\begin{equation*}
  \dfrac{f(a+h)-f(a)}{h} = f'(a)
\end{equation*}
もちろんこの式は正しくない

左辺は$h$を変化させると変化するが、右辺は$h$に無関係な定数

\br

しかし、$h$が非常に小さい数($0$に近い数)のとき、右辺はほとんど左辺に等しくなるという意味で
\begin{equation*}
  \dfrac{f(a+h)-f(a)}{h} \thicksim f'(a)
\end{equation*}
と書けばよい

\br

これがリミットの直感的な、あるいは実用的な意味に他ならない
わずかな違いは(それがいくらでも小さくとれるなら)無視してもいい、これが微分積分学の実用的な部分では大きな役割を果たす

\br

しかし、ほぼ等しいというのはやはり少し気になるので、わずかな違いを表す記号$\varepsilon$を使って書いてみる
\begin{equation*}
  \dfrac{f(a+h)-f(a)}{h} = f'(a) + \varepsilon
\end{equation*}
$\varepsilon$は右辺と左辺の違い(差)を表す数で、この数は$h$を小さくすればいくらでも小さくなる

もう少し正確には、「$h$を$0$に近づければ、$\varepsilon$も$0$に近づく」あるいは「$h \to 0$なら、$\varepsilon \to 0$」となる

\br

このリミットをとってしまって、差$\varepsilon$を使って表した微分係数の式には大きな利点がある

それは分母を払うことができるということ

\br

リミットのついた分数式はリミットが邪魔をして分母を払うことができないが、この式ではその制限がなくなった

そこでこの式の分母を払ってみる
\begin{equation*}
  f(a+h)-f(a) = (f'(a) + \varepsilon)h
\end{equation*}
あるいは
\begin{equation*}
  f(a+h) = f(a) + f'(a)h + \varepsilon h
\end{equation*}

\sectionline
\subsection{微分の数式を日本語に翻訳すると?}

$h$は$x$が$a$から変化した量で、これを$x$の\keyword{変化量}という

同様に、$f(a+h) - f(a)$は$x$が$a$から$h$だけ変化したときの$y$の変化した量で、これを$y$の\keyword{変化量}($y$の\keyword{増分})という

この変化量という言葉を使えば、先ほどの数式は次のように翻訳できる

\br

\begin{screen}
  関数が微分できるとき、$y$の変化量は
  \begin{itemize}
    \item $x$の変化量$h$に正比例する部分
    \item 誤差の部分
  \end{itemize}
  に分けられる

  正比例部分の比例定数は\keyword{微分係数}で、誤差の部分は$x$の変化量$h$を小さくすればいくらでも小さくできる
\end{screen}

\br

もう少しかみ砕いてしまうと、

\br

\begin{screen}
  \keyword{微分}できる関数とは、$x$の変化量がほんのわずかなときは、$y$の変化量が$x$の変化量$h$に正比例すると見なせる関数である

  わずかな違いは無視する!
\end{screen}

\br

ということになる

これは微分できることの意味に他ならない

\br

正比例関数は関数の中でも最も簡単なもので、それは比例定数を決めれば決まってしまう

したがって、すべての関数を局所的に正比例関数と見なせれば、変化の分析はとても見通しがよいものになる

そのために、関数全体ではなく、ごく一部分だけを取り出して考えてみよう、これが微分の出発点である

\sectionline
\subsection{微分という名詞と微分するという動詞}

微分ができる関数の$x=a$近くでの正比例部分、すなわち$f'(a)h$だけを取り出した関数(正比例関数)を、元の関数$y=f(x)$の$x=a$での\keyword{微分}(名詞)という

これは$x$の変化量$h$を変数とし、比例定数を$f'(a)$とする正比例関数である

この微分という名前の新しい正比例関数を、新しい変数記号$dx, dy$を使って次のように表す
\begin{equation*}
  dy = f'(a)dx
\end{equation*}

ここで、変数記号$dx$は$x$の変化量$h$と同じもので、$dx$を$x$の\keyword{微分}、$dy$を$y$の\keyword{微分}という

これは単なる変数記号で、新しい記号を使うのは元の変数$x, y$と区別するため

\br

点$(a, f(a))$を原点とし、新しい変数記号による座標軸$dx,dy$を設定したとき、関数の各点$(a, f(a))$に付随して現れる正比例関数を(名詞で)\keyword{微分}と呼んでいる

一方、関数の導関数を求めることは、関数を(動詞で)\keyword{微分する}という

\sectionline
\subsection{微分と導関数との関係}

微分という名の新しい正比例関数はその変数を$dx,dy$と書くことにしたため、$x=a$での微分を$a$ではなく$x$について書いても変数を混同することはない

そこで、関数$y=f(x)$に対して、
\begin{equation*}
  dy = f'(x)dx
\end{equation*}
を元の関数の\keyword{微分}と呼ぶことにする

ただし、本来の意味の微分は、この$x$に特定の値$x=a$を代入した$dy = f'(a)dx$のことをいうので注意

\br

微分とは$x=a$を固定して初めて本体の意味をもつもの

関数$dy = f'(x)dx$はある種の簡略記法だが、この記法は大変に便利で、関数の微分を求めるにはその関数の\keyword{導関数}を計算すればよい

\sectionline
\subsection{微分と関数の変化の様子との関係}

たとえば、微分$dy$が$0$になって消えている場所を考えると、これは、そのような場所では$x$が変化しても$y$は変化しないということを表している

$y$が変化しないということは、$y$が定数であるということに他ならない

\br

$dy=0$は比例定数が$0$である正比例関数

したがってそのような場所では
\begin{equation*}
  f'(a) = 0
\end{equation*}
となっていて、そのグラフは$x$軸に平行な直線

これが\keyword{極大}・\keyword{極小}を求めるとき、方程式$f'(x) = 0$を解いた理由である

この方程式の解$x=a$ではたしかに
\begin{equation*}
  dy = f'(a)dx = 0 \cdot dx = 0
\end{equation*}
となっていて、$y$の値は$x$が多少変化してもほとんど変化しない

\br

グラフで表すと、$y=\text{定数}$のグラフは$x$軸に平行な直線

これはグラフの\keyword{接線}であり、その点での微分$dy=0$を表している

これが微分を使った関数の解析の典型的な方法である

\br

微分の式$dy = f'(x)dx$の両辺を$x$の微分$dx$で割ると、
\begin{equation*}
  \dfrac{dy}{dx} = f'(x)
\end{equation*}
という式になり、微分の比が導関数を与えることがわかる

この式は、導関数を\keyword{微分商}とも呼ぶことの理由の1つ

したがって、微分とはこの式の分母を払ってもよいことの根拠を与えていると考えることもできる

\end{document}
