\documentclass[../book_infinite_continuous_math]{subfiles}

\begin{document}

\section{ロルの定理を証明してみる}

\subsection{ロルの定理とその証明}

\begin{oframed}
  \paragraph{ロルの定理(再掲)}
  関数$f(x)$は閉区間$[a, b]$で連続、開区間$(a, b)$で微分可能とする

  $f(a) = f(b)=0$のとき、
  \begin{equation*}
    f'(c) = 0 \quad (a < c < b)
  \end{equation*}
  となる$c$が少なくとも1つ存在する
\end{oframed}

\paragraph{ロルの定理の証明}\quad

$f(x)$が区間$[a, b]$で定数($=0$)なら、すべての$c$($a < c < b$)で$f'(c) = 0$だから定理は成り立つ

したがって、$f(x)$は定数でないとしてよい

\br

この区間での$f(x)$の最大値を$f(c)$とする

$f(a) = f(b) = 0$なので、最大値$f(c)$は、$f(c) \geq 0$である

最大値$f(c)$が$0$になるときは$f(x) \leq 0$なので、$-f(x)$を改めて$f(x)$と考えると、$f(x) \geq 0$となり、$f(x)$は定数関数ではないので、$f(x) > 0$である

したがって、最大値$f(c) > 0$で$a < c < b$としてよい

\br

$c$での微分係数
\begin{equation*}
  f'(c) = \lim_{h \to 0} \dfrac{f(c+h) - f(c)}{h}
\end{equation*}
を考える

$h >0$としてこの極限値をとると、$f(c)$が最大値であるから$f(c+h) - f(c) \leq 0$より、
\begin{equation*}
  \dfrac{f(c+h) - f(c)}{h} \leq 0
\end{equation*}
したがって、分子は負または$0$、分母は正だから、
\begin{equation*}
  f'(c) = \lim_{h \to 0} \dfrac{f(c+h) - f(c)}{h} \leq 0
\end{equation*}
である

\br

一方、同じ極限値を$h<0$としてとると、分子の符号は変わらないから、
\begin{equation*}
  \dfrac{f(c+h) - f(c)}{h} \geq 0
\end{equation*}
したがって、
\begin{equation*}
  f'(c) = \lim_{h \to 0} \dfrac{f(c+h) - f(c)}{h} \geq 0
\end{equation*}
である

\br

すなわち、
\begin{equation*}
  f'(c) \leq 0 \text{かつ} f'(c) \geq 0
\end{equation*}
となり、
\begin{equation*}
  f'(c) = 0
\end{equation*}
である$\qed$

\sectionline
\subsection{ロルの定理の物理的な意味}



\end{document}
