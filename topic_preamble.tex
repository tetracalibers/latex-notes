% === color ===

\usepackage[dvipsnames]{xcolor}

% ref: https://latexcolor.com/
\definecolor{hotpink}{rgb}{1.0, 0.41, 0.71}
\definecolor{carnationpink}{rgb}{1.0, 0.65, 0.79}
\definecolor{deeppink}{rgb}{1.0, 0.08, 0.58}
\definecolor{capri}{rgb}{0.0, 0.75, 1.0}
\definecolor{rosepink}{rgb}{1.0, 0.4, 0.8}
\definecolor{princetonorange}{rgb}{1.0, 0.56, 0.0}
\definecolor{lavendermagenta}{rgb}{0.93, 0.51, 0.93}
\definecolor{malachite}{rgb}{0.04, 0.85, 0.32}
\definecolor{lawngreen}{rgb}{0.49, 0.99, 0.0}
\definecolor{periwinkle}{rgb}{0.8, 0.8, 1.0}
\definecolor{lightslategray}{rgb}{0.47, 0.53, 0.6}
\definecolor{robineggblue}{rgb}{0.0, 0.8, 0.8}
\definecolor{rosebonbon}{rgb}{0.98, 0.26, 0.62}
\definecolor{airforceblue}{rgb}{0.36, 0.54, 0.66}
\definecolor{columbiablue}{rgb}{0.61, 0.87, 1.0}
\definecolor{magnolia}{rgb}{0.97, 0.96, 1.0}

% === box ===

\usepackage{awesomebox}

% === math ===

\usepackage{physics}
\usepackage{braket}

\usepackage{amsthm} % 定理環境とQEDコマンド

\usepackage{mathtools}
% 別の場所で参照する数式以外は番号が付かないように
\mathtoolsset{showonlyrefs=true}

% === font ===

\usepackage[T1]{fontenc}

% normal font, math font
%\usepackage[light,math]{anttor}
\usepackage{gfsartemisia}

% monospace font
\usepackage[scaled]{beramono}

% === layout ===

\usepackage[top=20truemm,bottom=20truemm,left=20truemm,right=60truemm,marginparwidth=40truemm,marginparsep=10truemm]{geometry} % 余白
\renewcommand{\baselinestretch}{1.25} % 行間

\usepackage{leading}

\setlength{\parindent}{0pt} % 段落始めでの字下げをしない

\usepackage{enumitem}
\usepackage{marginnote}

% === tikz ===

\usepackage[dvipdfmx]{graphicx}

\usepackage{witharrows}

\usepackage{froufrou} % セクションを区切る装飾
\setfroufrou{dinkus}

% === tcolorbox ===

\usepackage{tcolorbox}
\tcbuselibrary{listings,breakable,xparse,skins,hooks,theorems}

\newcommand{\titlegap}{\quad\\[0.1cm]}

\DeclareTColorBox{definition}{m O{} }%
{
  enhanced,
  colframe=magnolia,
  colback=magnolia!20!white,
  coltitle=black,
  fonttitle=\bfseries,
  breakable=true,
  sharp corners,
  title={\textcolor{Cerulean!60!black}{\faGraduationCap}\hspace{0.1em} #1},
  detach title,
  before upper={\tcbtitle\quad},
  bottom=0.5cm,
  top=0.5cm,
  right=0.5cm,
  left=0.5cm,
  #2
}

\DeclareTColorBox{theorem}{m O{} }%
{
  enhanced,
  colframe=magnolia,
  colback=magnolia!20!white,
  coltitle=black,
  fonttitle=\bfseries,
  breakable=true,
  sharp corners,
  title={\textcolor{magenta!70!black}{\faAnchor}\hspace{0.1em} #1},
  detach title,
  before upper={\tcbtitle\quad},
  bottom=0.5cm,
  top=0.5cm,
  right=0.5cm,
  left=0.5cm,
  #2
}

% 背景がグレー
\DeclareTColorBox{shaded}{O{} }%
{
  enhanced,
  colframe=white,
  colback=gray!10,
  breakable=true,
  sharp corners,
  detach title,
  bottom=0.25cm,
  top=0.25cm,
  right=0.25cm,
  left=0.25cm,
  #1
}

% === memo ===

\usepackage[enable]{easy-todo}

% === original ===

\newcommand{\keyword}[1]{\textcolor{RubineRed}{\textbf{#1}}}

\newcommand{\br}{\vskip0.5\baselineskip}
\newcommand{\sectionline}{\froufrou}

\renewcommand{\labelitemii}{$\circ$}

% === toc ===

\usepackage{tocloft}
\renewcommand{\cftsecfont}{\rmfamily}
\renewcommand{\cftsecpagefont}{\rmfamily}
\setcounter{secnumdepth}{0}

\addtocontents{toc}{\protect\thispagestyle{empty}}
\pagestyle{empty}

% === hyperlink ===

\definecolor{oxfordblue}{rgb}{0.0, 0.13, 0.28}

% 「%」は以降の内容を「改行コードも含めて」無視するコマンド
\usepackage[%
  dvipdfmx,% 欧文ではコメントアウトする
  pdfencoding=auto, psdextra,% 数学記号を含める
  setpagesize=false,%
  bookmarks=true,%
  bookmarksdepth=tocdepth,%
  bookmarksnumbered=true,%
  colorlinks=true,%
  allcolors=oxfordblue,%
  pdftitle={},%
  pdfsubject={},%
  pdfauthor={},%
  pdfkeywords={}%
]{hyperref}
% PDFのしおり機能の日本語文字化けを防ぐ((u)pLaTeXのときのみかく)
\usepackage{pxjahyper}
% ref: https://tex.stackexchange.com/questions/251491/math-symbol-in-section-heading
\pdfstringdefDisableCommands{\def\varepsilon{\textepsilon}}
