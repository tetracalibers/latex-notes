\documentclass[b5paper,17pt,twocolumn]{jsarticle}

\usepackage{subfiles}

\title{読書ノート:地力をつける微分積分}
\author{tomixy}

% === color ===

\usepackage[dvipsnames]{xcolor}

% === math ===

\usepackage{amsthm} % 定理環境とQEDコマンド

\usepackage{mathtools}
% 別の場所で参照する数式以外は番号が付かないように
\mathtoolsset{showonlyrefs=true}

% === font ===

\usepackage[T1]{fontenc}

% normal font, math font
%\usepackage[light,math]{anttor}
\usepackage{gfsartemisia}

% monospace font
\usepackage[scaled]{beramono}

% === layout ===

\usepackage[top=15truemm,bottom=30truemm,left=20truemm,right=20truemm]{geometry} % 余白
\renewcommand{\baselinestretch}{1.25} % 行間

\setlength{\parindent}{0pt} % 段落始めでの字下げをしない

% === tikz ===

\usepackage{froufrou} % セクションを区切る装飾
\setfroufrou{dinkus}

% === original ===

\newcommand{\keyword}[1]{\textcolor{RubineRed}{\textbf{#1}}}

\newcommand{\sectionline}{\vspace{-\baselineskip}\froufrou\vspace{-\baselineskip}}

% ---

\begin{document}

\maketitle

\subfile{book_jiriki_calc/section_1-1--1-4}
\subfile{book_jiriki_calc/section_2-1--2-4}
\subfile{book_jiriki_calc/section_3-1--3-4}
\subfile{book_jiriki_calc/section_4-1}
\subfile{book_jiriki_calc/section_4-2}
\subfile{book_jiriki_calc/section_4-3}
\subfile{book_jiriki_calc/section_4-4}
\subfile{book_jiriki_calc/section_4-5}
\subfile{book_jiriki_calc/section_4-6}

\end{document}
