\documentclass[../book_why-set-theory-why-topology]{subfiles}

\begin{document}

\section{熱伝導方程式とフーリエ級数}

細長い棒状の物体があったとする

時刻$t$における、左端から右に向かって距離$x$の位置での、この物体の温度を$u(x,t)$と表す

このとき、$u$の変化は、次の\keyword{熱伝導方程式}に従う

\begin{equation*}
  \frac{\partial u}{\partial t} = \frac{\partial^2 u}{\partial x^2}
\end{equation*}

\sectionline

棒の両端での温度をゼロに保ち、初期状態での温度が位置$x$の関数$f(x)$で与えられているものとして、温度$u$のその後の変化を追っていく

\br

計算を簡単にするため、棒の長さが円周率$\pi$となるように長さの単位を選ぶことにする

\br

すると、棒の左端では$x=0$、右端では$x=\pi$であるから、両端の温度がいつでもゼロであるという条件は
\begin{equation*}
  u(0,t) = u(\pi,t) = 0 \quad (t \geq 0)
\end{equation*}
と表現される

この条件を熱伝導方程式に対する\keyword{境界条件}という

\br

また、最初の温度が$f(x)$で与えられるという条件は、
\begin{equation*}
  u(x,0) = f(x) \quad (0 \leq x \leq \pi)
\end{equation*}
と表現できる

この条件を熱伝導方程式に対する\keyword{初期条件}という

\br

熱伝導方程式を、これらの境界条件と初期条件のもとで解けばよい

\sectionline

熱伝導方程式と境界条件は、次の性質をもつ

\begin{itemize}
  \item $u_1$と$u_2$が熱伝導方程式と境界条件を満たすとき、和$u_1 + u_2$もまた熱伝導方程式と境界条件を満たす
  \item $u$が熱伝導方程式と境界条件を満たすとき、その任意の定数倍$ku$もまた熱伝導方程式と境界条件を満たす
\end{itemize}

熱伝導方程式のこの性質を、\keyword{重ね合わせの原理}という

\sectionline

フーリエは、重ね合わせの原理を活用して、

\begin{enumerate}
  \item 方程式と境界条件を満たす関数のうち、なるべく簡単なものを求める
  \item 得られた簡単な形の関数を足し合わせることで、初期条件を満たす関数を作る
\end{enumerate}

という二段構えで問題を解こうとした

\br

まず、熱伝導方程式の変数分離型の解のうち、境界条件を満たすのは、
\begin{equation*}
  e^{-n^2 t} \sin(nx) \quad (n=1,2,\ldots)
\end{equation*}
という形のものとその定数倍だけであることがわかる

\begin{leftbar}
  注:ただし、棒の長さが$\pi$ではなく、一般に長さ$L$の棒であれば、ここでの解は
  \begin{equation*}
    e^{-\frac{n^2\pi^2}{L^2}t} \sin\left(\frac{n\pi}{L} x\right) \quad (n=1,2,\ldots)
  \end{equation*}
  という形になる
\end{leftbar}

\br

重ね合わせの原理により、この関数に定数$b_n$をかけて、たとえば$1 \leq n \leq N$の範囲で足し合わせた
\begin{equation*}
  \sum_{n=1}^N b_n e^{-n^2 t} \sin(nx)
\end{equation*}
という関数も、方程式と境界条件を満たす

\br

ここでフーリエは大胆にも、和の上限$N$をなくした無限和
\begin{equation*}
  u(x,t) = \sum_{n=1}^\infty b_n e^{-n^2 t} \sin(nx)
\end{equation*}
も方程式と境界条件を満たすだろうと主張した

\br

そして、初期条件を満たす関数$u$を求めるためには、上の式の右辺で$t=0$とした式を関数$f(x)$と等値した
\begin{equation*}
  f(x) = \sum_{n=1}^\infty b_n \sin(nx)
\end{equation*}
が成り立つように定数$b_n$を定めればよいと考えた

\br

この式が有限和
\begin{equation*}
  f(x) = \sum_{n=1}^N b_n \sin(nx)
\end{equation*}
であった場合には、両辺に$\sin kx$をかけて0から$\pi$まで積分すれば、定積分の公式
\begin{equation*}
  \int_0^\pi \sin nx \sin kx \, dx = \begin{cases}
    0             & (n \neq k) \\
    \frac{\pi}{2} & (n = k)
  \end{cases}
\end{equation*}
によって、
\begin{equation*}
  \int_0^\pi f(x) \sin kx \, dx = \frac{\pi}{2} b_k
\end{equation*}
となり、$b_n$は積分
\begin{equation*}
  b_n = \frac{2}{\pi} \int_0^\pi f(x) \sin nx \, dx
\end{equation*}
で求められる

\br

無限和の場合も同様に、与えられた関数$f(x)$に対して、この式で$b_n$を定めれば、無限級数が成立するだろうとフーリエは主張した

\sectionline

こうして、熱伝導方程式を解く方法の一環として、関数$f(x)$から得られる積分値を係数とした三角関数の和で$f(x)$を表示する、という着想にフーリエは至った

\br

$f(x)$からこの方法で得られる三角関数の和を、$f(x)$の\keyword{フーリエ級数}という

\begin{leftbar}
  注:ここでは、$f(x)$が区間の両端でゼロになる場合だけを扱ったため、フーリエ級数がサイン関数だけの和になった \\
  一般には、区間の両端で値が一致しない関数や、原点でグラフが対称でない関数も含めて考えるために、サイン関数とコサイン関数の和を考える
\end{leftbar}

\sectionline
\section{フーリエの理論の問題点}

一般に、$-\pi$から$\pi$までの実数$x$に対して定義された関数$f(x)$に対して、$a_n$と$b_n$を
\begin{align*}
  a_n & = \frac{1}{\pi} \int_{-\pi}^\pi f(x) \cos nx \, dx \\
  b_n & = \frac{1}{\pi} \int_{-\pi}^\pi f(x) \sin nx \, dx
\end{align*}
のように定めれば、$f(x)$の\keyword{フーリエ級数}は
\begin{equation*}
  f(x) \sim \dfrac{a_0}{2} + \sum_{n=1}^\infty \left( a_n \cos nx + b_n \sin nx \right)
\end{equation*}
という形を持つことになる

\br

フーリエ級数は、収束するかどうかわからない上、収束しても値が$f(x)$と一致するかどうかわからないため、一般には、この式の両辺を等号で結ぶことはできない

\br

そこで、左辺の関数のフーリエ級数は右辺の級数だよ、ということを表現するために、等号の代わりに$\sim$記号を用いる

\sectionline

フーリエの議論の一番の問題点は、無限級数の\keyword{項別積分}の可能性にあった

\br

仮に、関数$f(x)$が本当に三角関数の和として
\begin{equation*}
  f(x) = \dfrac{a_0}{2} + \sum_{n=1}^\infty \left( a_n \cos nx + b_n \sin nx \right)
\end{equation*}
の形で与えられているとしても、このことから、たとえば
\begin{equation*}
  a_n = \frac{1}{\pi} \int_{-\pi}^\pi f(x) \cos nx \, dx
\end{equation*}
を結論づけるには、無限和と積分の順序交換、すなわち\keyword{項別積分}

\end{document}
