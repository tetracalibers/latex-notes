\documentclass[../../../topic_statistics]{subfiles}

\begin{document}

\sectionline
\section{データのばらつきの指標:偏差}
%\marginnote{}

代表値はデータを1つの値で要約する指標であり、データのばらつきや偏りは表現しきれない。

そこで、新たにデータのばらつきを表す指標を考える。

\br

各データが、平均からどれくらい離れているかを表す指標を\keywordJE{偏差}{deviation}という。

\begin{definition}{偏差}
  $N$個の観測値$x_1, \ldots, x_N$の平均値を$\overline{x}$とするとき、各観測値$x_i$の\keyword{偏差}は次のように定義される。
  \begin{equation*}
    d_i \coloneq x_i - \overline{x}
  \end{equation*}
  ここで、$d_i$は$i$番目のデータの偏差を表す。
\end{definition}

\subsection{偏差の平均値で全体をみる}

全データの偏差$d_1, \ldots, d_N$の平均値を求めることで、データ全体が平均からどれくらい離れて分布しているか(どれくらいばらついているか)を表すことができそうである。

\br

しかし、偏差の平均値は、次のように常に0になってしまう。
\begin{align*}
  \frac{1}{N} \sum_{i=1}^{N} d_i
  &= \frac{1}{N} \sum_{i=1}^{N} (x_i - \overline{x}) \\
  &= \frac{1}{N} \left( \sum_{i=1}^N x_i - \sum_{i=1}^{N} \overline{x} \right)
  = \frac{1}{N} \left( \sum_{i=1}^{N} x_i - N \overline{x} \right) \\
  &= \sum_{i=1}^{N} \frac{1}{N}x_i - \overline{x}
  = \overline{x} - \overline{x}
  = 0
\end{align*}

\br

そこで、単なる平均との差ではなく、平均との距離を考えることにする。

偏差に絶対値をつけたものの平均を\keyword{平均偏差}という。

\begin{definition}{平均偏差}
  $N$個の観測値$x_1,\dots,x_N$の平均値を$\overline{x}$とするとき、\keyword{平均偏差}を次のように定義する。
  \begin{equation*}
    d \coloneq \frac{1}{N} \sum_{i=1}^{N} |x_i - \overline{x}|
  \end{equation*}
  
\end{definition}

\sectionline
\section{データのばらつきの指標:分散と標準偏差}
%\marginnote{}

平均偏差では、データと平均値の距離として絶対値を用いたが、絶対値は次のような理由で計算が面倒である。
\begin{itemize}
  \item 絶対値は微分できない点がある
  \item 正負を判定する条件分岐処理が入り、コンピュータでの計算速度が落ちる
\end{itemize}

そこで、絶対値の代わりに二乗を用いた、\keywordJE{分散}{variance}という指標を定義する。

\begin{definition}{分散}
  $N$個の観測値$x_1, \ldots, x_N$の平均値を$\overline{x}$とするとき、\keyword{分散}を次のように定義する。
  \begin{equation*}
    \sigma^2 \coloneq \frac{1}{N} \sum_{i=1}^{N} (x_i - \overline{x})^2
  \end{equation*}
\end{definition}

\subsection{もとのデータと同じ単位を持ったばらつきの指標}

分散では二乗を用いるため、単位に注意が必要である。

もとのデータの単位が$[x]$であれば、分散の単位は$[x]^2$となる。

たとえば、点数を表すデータを扱っているとすると、その分散の単位は「点$^2$」となり、直観的に理解しづらい。

\br

そこで、単位をもとのデータと揃えるために、分散の平方根をとった形がよく用いられる。

分散の平方根を\keywordJE{標準偏差}{standard deviation}という。

\begin{definition}{標準偏差}
  分散$\sigma^2$の平方根をとったものを\keyword{標準偏差}として定義する。
  \begin{equation*}
    \sigma \coloneq \sqrt{\sigma^2}
  \end{equation*}
\end{definition}

\sectionline
\section{分散公式}

分散は、次のように計算することもできる。
\begin{equation*}
  \text{\bfseries 分散} = \text{\bfseries データの二乗平均} - \text{\bfseries 平均の二乗}
\end{equation*}

\begin{theorem}{分散公式}
  $N$個の観測値$x_1, \ldots, x_N$の平均を$\overline{x}$、分散を$\sigma^2$とすると、次の関係が成り立つ。
  \begin{equation*}
    \sigma^2 = \frac{1}{N} \sum_{i=1}^{N} x_i^2 - \overline{x}^2
  \end{equation*}
\end{theorem}

\begin{proof}
  分散の定義に基づいて、次のように計算する。
  \begin{align*}
    \sigma^2 &= \frac{1}{N} \sum_{i=1}^{N} (x_i - \overline{x})^2 \\
    &= \frac{1}{N} \sum_{i=1}^{N} (x_i^2 - 2x_i\overline{x} + \overline{x}^2) \\
    &= \frac{1}{N} \sum_{i=1}^{N} x_i^2 - 2\overline{x}\frac{1}{N} \sum_{i=1}^{N} x_i + \overline{x}^2 \\
    &= \frac{1}{N} \sum_{i=1}^{N} x_i^2 - 2\overline{x}^2 + \overline{x}^2 \\
    &= \frac{1}{N} \sum_{i=1}^{N} x_i^2 - \overline{x}^2
  \end{align*}
\end{proof}

\end{document}
