\documentclass[../../../topic_statistics]{subfiles}

\begin{document}

\sectionline
\section{度数分布}
%\marginnote{}

より細かくデータがどのように分布しているかを知りたいときは、次のような手順でデータを整理する。
\begin{enumerate}
  \item データがとる値をいくつかの区間に分ける
  \item 各区間にいくつのデータが入っているかを数える
\end{enumerate}

このとき、区間を\keywordJE{階級}{class}といい、各階級に属しているデータの数を\keywordJE{度数}{frequency}という。

\br

各階級ごとに、度数などの値を表にまとめたものを\keyword{度数分布表}という。

\begin{table}[htbp]
  \centering
  \caption*{\bfseries 度数分布表}
  \begin{tabular}{|c|c|c|c|c|c|}
    \hline
    階級 & 階級値 & 度数 & 累積度数 & 相対度数 & 累積相対度数 \\
    \hline
    $a_0 \sim a_1$ & $x_1$ & $f_1$ & $F_1$ & $\frac{f_1}{N}$ & $\frac{F_1}{N}$ \\
    $a_1 \sim a_2$ & $x_2$ & $f_2$ & $F_2$ & $\frac{f_2}{N}$ & $\frac{F_2}{N}$ \\
    \vdots & \vdots & \vdots & \vdots & \vdots & \vdots \\
    $a_{n-1} \sim a_n$ & $x_n$ & $f_n$ & $F_n$ & $\frac{f_n}{N}$ & $\frac{F_n}{N}$ \\
    \hline
  \end{tabular}
\end{table}

\subsection{階級値}

各階級を代表する値を\keyword{階級値}といい、通常は各階級の中央の値を階級値として用いる。
\begin{equation*}
  x_i = \frac{a_{i-1} + a_i}{2}
\end{equation*}

\subsection{累積度数}

その階級までに属するデータの個数を\keyword{累積度数}という。
\begin{equation*}
  F_i = \sum_{j=1}^{i} f_j
\end{equation*}

\subsection{相対度数}

全データ数に対して、その階級に属するデータ数がどのくらいの割合を占めているのかを\keyword{相対度数}という。
\begin{equation*}
  \frac{f_i}{N}
\end{equation*}

\subsection{累積相対度数}

全データ数に対して、その階級までに属するデータ数がどのくらいの割合を占めているのかを\keyword{累積相対度数}という。
\begin{equation*}
  \frac{F_i}{N}
\end{equation*}

\end{document}
