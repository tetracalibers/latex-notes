\documentclass[../../../topic_statistics]{subfiles}

\begin{document}

\sectionline
\section{データの標準化}
%\marginnote{}

たとえば、平均点が30点のテストでとった60点と、平均点が80点のテストでとった60点とでは、相対的な出来が異なる。
このように、点数というデータはそのテストの平均や分散によって評価が変わってしまう。

そのため、平均や分散に依存せずにデータの相対的な位置関係がわかるようにできたら便利である。

\br

特に、平均が0、標準偏差が1になるようにデータを変換することを\keywordJE{標準化}{standardization}という。

\br

$y_i = ax_i + b$というデータの変換を考えよう。

これは、データを$a$倍して$b$だけシフトする変換である。

\br

このとき、平均と標準偏差は次のように変化する。
\begin{itemize}
  \item 平均は$a$倍され、$b$だけ増える
  \item 標準偏差は$a$倍される(分散が$a^2$倍される)
\end{itemize}

数式で表すと、
\begin{align*}
  \overline{y} &= a \overline{x} + b \\
  \sigma_y &= a \sigma_x
\end{align*}

そこで、平均$\overline{y}$が0、標準偏差$\sigma_y$が1になるように、$a$と$b$を次のように設定する。
\begin{equation*}
  a = \frac{1}{\sigma_x}, \quad b = -\frac{\overline{x}}{\sigma_x}
\end{equation*}
このようにすると、たしかに$\overline{y} = 0$、$\sigma_y = 1$となる。

\br

このとき、変換後のデータ$y_i$は次のように表される。
\begin{equation*}
  y_i = ax_i + b = \frac{x_i - \overline{x}}{\sigma_x}
\end{equation*}

\begin{definition}{標準化}
  各データから平均を引き、標準偏差で割ることで、平均が0、標準偏差が1になるように変換することを\keyword{標準化}という。
  
  各データ$x_i$を標準化したデータを$y_i$とすると、次の関係が成り立つ。
  \begin{equation*}
    y_i = \frac{x_i - \overline{x}}{\sigma_x}
  \end{equation*}
\end{definition}

\end{document}
