\documentclass[../../../topic_statistics]{subfiles}

\begin{document}

\sectionline
\section{データの変換による平均と分散の変化}
%\marginnote{}

データの変換(スケーリングやシフト)を行うと、平均や分散はどのように変化するのだろうか。

\subsection{データのスケーリング}

まず、データを定数$a$倍する変換を考える。

すなわち、各データ$x_i$を$y_i = ax_i$に変換すると、平均と分散は次のように変化する。
\begin{align*}
  \overline{y} &= a \overline{x} \\
  \sigma_y^2 &= a^2 \sigma_x^2
\end{align*}

\begin{theorem}{データのスケーリングによる平均と分散の変化}
  観測値が$a$倍されると、平均は$a$倍、分散は$a^2$倍される。
\end{theorem}

\begin{proof}
  各データ$x_i$を$y_i = ax_i$に変換すると、平均は次のように変化する。
  \begin{align*}
    \overline{y} &= \frac{1}{N} \sum_{i=1}^{N} ay_i = a \cdot \frac{1}{N} \sum_{i=1}^{N} y_i = a \overline{y}
  \end{align*}

  また、分散は次のように変化する。
  \begin{align*}
    \sigma_y^2 &= \frac{1}{N} \sum_{i=1}^{N} (ax_i - a\overline{x})^2 = \frac{1}{N} \sum_{i=1}^N (a(x_i - \overline{x}))^2 \\
    &= a^2 \cdot \frac{1}{N} \sum_{i=1}^{N} (x_i - \overline{x})^2 = a^2 \sigma_x^2
  \end{align*}
\end{proof}

\subsection{データのシフト}

次に、データを定数$b$だけシフトする変換を考える。

すなわち、各データ$x_i$を$y_i = x_i + b$に変換すると、平均と分散は次のように変化する。
\begin{align*}
  \overline{y} &= \overline{x} + b \\
  \sigma_y^2 &= \sigma_x^2
\end{align*}

\begin{theorem}{データのシフトによる平均と分散の変化}
  観測値に$b$を加えると、平均は$b$だけ増えるが、分散は変化しない。
\end{theorem}

\begin{proof}
  各データ$x_i$を$y_i = x_i + b$に変換すると、平均は次のように変化する。
  \begin{align*}
    \overline{y} &= \frac{1}{N} \sum_{i=1}^{N} (x_i + b) = \frac{1}{N} \sum_{i=1}^{N} x_i + b = \overline{x} + b
  \end{align*}

  また、分散は次のように変化しない。
  \begin{align*}
    \sigma_y^2 &= \frac{1}{N} \sum_{i=1}^{N} (x_i + b - (\overline{x} + b))^2 = \frac{1}{N} \sum_{i=1}^{N} (x_i - \overline{x})^2 = \sigma_x^2
  \end{align*}
\end{proof}

\br

このように、データの変換によって平均と分散は異なる影響を受けることがわかる。

\end{document}
