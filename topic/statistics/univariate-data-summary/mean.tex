\documentclass[../../../topic_statistics]{subfiles}

\begin{document}

\sectionline
\section{データの中心の指標:平均値}
%\marginnote{}

「データを1つの値で要約するならばこれ」といった指標を\keyword{代表値}という。

最もよく使われる代表値が\keywordJE{平均値}{mean}であり、データの中心を表す指標として広く用いられる。

\br

平均値は、データをすべて足し合わせて、データの数で割ることで求まる。

\begin{definition}{平均}
  $N$個の観測値$x_1, \ldots, x_N$の総和をデータのサイズ$N$で割ったものを\keyword{平均値}という。
  \begin{equation*}
    \overline{x} \coloneq \frac{1}{N} \sum_{i=1}^{N} x_i
  \end{equation*}
\end{definition}

\end{document}
