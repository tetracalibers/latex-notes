\documentclass[../../../topic_statistics]{subfiles}

\begin{document}

\sectionline
\section{条件つき確率}

事象$A$が起こったときに事象$B$が起こる確率を\keyword{条件つき確率}という。

\begin{definition}{条件つき確率}
  事象$A$が起こったときに事象$B$が起こる確率を$P(B|A)$あるいは$P_A(B)$と表し、これを$A$が起こったときの$B$が起こる\keyword{条件つき確率}という。
\end{definition}

条件つき確率では、標本空間「全体」ではなく、その一部分である「Aが起きた場合」に限定して考える。

その中でBも起こる割合だから、「AかつB」の確率を「A」の中での割合でみればよい。
\begin{equation*}
  P(B|A) = \frac{P(A \cap B)}{P(A)}
\end{equation*}

\begin{theorem}{条件つき確率}
  事象$A$が起こったときの事象$B$が起こる条件つき確率は、
  \begin{equation*}
    P(B|A) = \frac{P(A \cap B)}{P(A)}
  \end{equation*}
\end{theorem}

\end{document}
