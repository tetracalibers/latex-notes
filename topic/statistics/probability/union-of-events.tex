\documentclass[../../../topic_statistics]{subfiles}

\begin{document}

\sectionline
\section{和事象の確率}

確率は論理の拡張であると捉えると、ANDやORといった論理演算を確率に当てはめて考えることができる。
ここからは、複数の出来事(事象)が組み合わされた場合の確率について考えていこう。

\br

まずはOR「または」で組み合わされた事象の確率を考えてみる。

$A$または$B$が起こる事象は、$A \cup B$と表すことができる。
このような事象を\keyword{和事象}という。

\br

$A$または$B$が起こる場合が$n(A \cup B)$通りあるとすると、その確率は、次の割合で表される。
\begin{equation*}
  P(A \cup B) = \frac{n(A \cup B)}{n(U)}
\end{equation*}

\br

\begin{center}
  % ref: https://doratex.hatenablog.jp/entry/20191211/1576017170
  \begin{tikzpicture}[set label/.style={fill=white,circle,inner sep=2}]
    \def\radius{2}
    \def\ratio{0.6}
    \def\centerA{180:\ratio*\radius}
    \def\centerB{0:\ratio*\radius}
    \def\circleA{(\centerA) circle [radius=\radius]}
    \def\circleB{(\centerB) circle [radius=\radius]}
    % 座標設定
    \coordinate (O) at (0,0);
    \coordinate (NE) at (4,3);
    \coordinate (SW) at ($(O)!-1!(NE)$);
    % 円の描画
    \filldraw[carnationpink, fill opacity=0.5, line width=1pt] \circleA;
    \filldraw[SkyBlue, fill opacity=0.5, line width=1pt] \circleB;
    % ラベルの描画
    \node[set label,text=Rhodamine] at ($(\centerA) + (90:\radius)$) {$A$};
    \node[set label,text=Cerulean] at ($(\centerB) + (90:\radius)$) {$B$};
    \node[color=white] at (O) {$A\cap B$};
  \end{tikzpicture}
\end{center}

\br

ここで、$n(A \cup B)$は、\hyperref[thm:rule-of-sum]{場合の数の和の法則}より、$n(A)$と$n(B)$の和で求まると考えられる。

しかし、$A$と$B$の重なっている部分は二重に数えてしまうので、$n(A \cap B)$を引く必要がある。
\begin{equation*}
  n(A \cup B) = n(A) + n(B) - n(A \cap B)
\end{equation*}

よって、和事象$A \cup B$の確率は、
\begin{align*}
  P(A \cup B) & = \frac{n(A \cup B)}{n(U)}                                         \\
              & = \frac{n(A) + n(B) - n(A \cap B)}{n(U)}                           \\
              & = \frac{n(A)}{n(U)} + \frac{n(B)}{n(U)} - \frac{n(A \cap B)}{n(U)} \\
              & = P(A) + P(B) - P(A \cap B)
\end{align*}
として求められる。

\begin{theorem}{和事象の確率}
  和事象$A \cup B$の確率は、
  \begin{equation*}
    P(A \cup B) = P(A) + P(B) - P(A \cap B)
  \end{equation*}
\end{theorem}

\sectionline
\section{排反な事象と和事象の確率}

2つの事象(出来事)が\keyword{互いに排反}であるとは、
\begin{emphabox}
  \begin{spacebox}
    \begin{center}
      一方の出来事が起こると、もう一方の出来事は起こりえない
    \end{center}
  \end{spacebox}
\end{emphabox}
という状況を指す。

\br

\begin{center}
  \begin{tikzpicture}[set label/.style={fill=white,circle,inner sep=2}]
    \def\radius{2}
    \def\gap{2.5} % 円の中心の間隔を大きくして重ならないように
    % 円の中心座標(左と右に大きく離す)
    \coordinate (centerA) at (-\gap, 0);
    \coordinate (centerB) at (\gap, 0);
    % 円の描画
    \filldraw[carnationpink, fill opacity=0.5, line width=1pt] (centerA) circle[radius=\radius];
    \filldraw[SkyBlue, fill opacity=0.5, line width=1pt] (centerB) circle[radius=\radius];
    % ラベルの描画
    \node[set label,text=Rhodamine] at ($(centerA) + (90:\radius)$) {$A$};
    \node[set label,text=Cerulean] at ($(centerB) + (90:\radius)$) {$B$};
\end{tikzpicture}
\end{center}

\br

和事象の確率において、事象$A$と$B$が互いに排反であるなら、$n(A \cap B) = 0$となるので、
\begin{equation*}
  P(A \cup B) = P(A) + P(B)
\end{equation*}
が成り立つ。これを\keyword{確率の加法定理}という。

\begin{theorem}{確率の加法定理}
  互いに排反な事象$A,B$の和事象$A \cup B$の確率は、
  \begin{equation*}
    P(A \cup B) = P(A) + P(B)
  \end{equation*}
\end{theorem}

\end{document}
