\documentclass[../../../topic_statistics]{subfiles}

\begin{document}

\sectionline
\section{論理から確率へ}
%\marginnote{}

\keywordJE{確率}{probability}は、論理を拡張したものと捉えることができる。

\br

論理では真と偽という2つの値があり、これらは確信(絶対的な信念)に対応する。

なにかが真であるというのは、それが「正しい」と完全に確信しているという意味である。

\br

しかし、私たちが行う決定には、ほぼ必ず、確信のなさがある程度伴っている。

確率を使えば、論理を拡張して、「真と偽の間の不確実な値」を扱うことができる。

\subsection{確率における真と偽}

真は1、偽は0で表現することが多いので、確率の定義もそれに倣うことにする。

\br

$X$である確率を$P(X)$とすると、
\begin{itemize}
  \item $P(X) = 0$:$X$が偽
  \item $P(X) = 1$:$X$が真
\end{itemize}

0と1の間には無限個の数が存在し、どちらの確信の方が強いかによってこの値が揺らぐ。

\begin{itemize}
  \item 0の方に近い値は、ある事柄$X$が偽である確信の方が強いという意味
  \item 1の方に近い値は、ある事柄$X$が真である確信の方が強いという意味
  \item 0.5という値は、ある事柄$X$の真偽にまったく確信が持てないという意味
\end{itemize}

\subsection{確率における否定}

論理で重要なものとして、\keyword{否定}がある。

\begin{itemize}
  \item 「真でない」とは「偽である」という意味
  \item 「偽でない」とは「真である」という意味
\end{itemize}

確率にもこのような性質を与えたいので、$X$である確率と$X$でない確率を足すと1になるようにする。
\begin{equation*}
  P(X) + P(\lnot X) = 1
\end{equation*}
ここで、記号$\lnot$は否定を表す。

\br

この論理を使えば、$X$でない確率を次のように表現できる。
\begin{equation*}
  P(\lnot X) = 1 - P(X)
\end{equation*}

このとき、$P(X) = 1$であれば$P(\lnot X) = 0$となり、基本的な論理法則と合致する。

\end{document}
