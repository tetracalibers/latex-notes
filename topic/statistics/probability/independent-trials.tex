\documentclass[../../../topic_statistics]{subfiles}

\begin{document}

\sectionline
\section{独立な事象と積事象の確率}

2つの事象(出来事)が\keyword{互いに独立}であるとは、
\begin{emphabox}
  \begin{spacebox}
    \begin{center}
      一方の出来事の結果が、もう一方の出来事の結果に影響を与えない
    \end{center}
  \end{spacebox}
\end{emphabox}
ということである。

\br

このとき、$A$が起きたかどうかが$B$の起きやすさに影響しないので、
\begin{equation*}
  P(B|A) = P(B)
\end{equation*}
が成り立つ。

よって、確率の乗法定理を次のように書き換えられる。
\begin{equation*}
  P(A \cap B) = P(A) \cdot P(B)
\end{equation*}

\begin{theorem}{独立な事象の確率}
  互いに独立な事象$A,B$の積事象$A \cap B$の確率は、
  \begin{equation*}
    P(A \cap B) = P(A) \cdot P(B)
  \end{equation*}
\end{theorem}

\end{document}
