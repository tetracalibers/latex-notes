\documentclass[../../../topic_statistics]{subfiles}

\begin{document}

\sectionline
\section{記述統計から推測統計へ}
%\marginnote{}

ここまで扱ってきた、代表値やばらつきの指標、度数分布などは、全データをもとに算出されるものだった。

すべてのデータをもとに、それらを整理することでデータ全体の性質を分析する方法論は、\keyword{記述統計}と呼ばれる。

\br

しかし、調べたい対象の規模によっては、それらすべてからデータを集めるのは不可能な場合もある。

たとえば、日本人の平均身長を求めようとしても、日本人全員に身長を聞いて回ることは現実的ではない。

\br

そこで、一部のデータだけを集めて、そこからデータ全体の性質を推測する方法論として、\keyword{推測統計}がある。

\sectionline
\section{母集団からの無作為抽出}

推測統計では、調べたいデータ全体を\keyword{母集団}といい、その一部分だけを使って推測を行う。

推測に使う、一部分のデータを\keyword{標本}という。

\br

そして、母集団から標本を取り出すことを、\keyword{標本抽出}という。

\br

たとえば、日本人の平均身長を推測するためのデータを集めたいとする。

友達に身長を聞いて回ってデータを集めるのも標本抽出といえるが、これでは同性の身長データが比較的多く集まってしまうなど、偏りが生じてしまう。

\br

ある特定の対象に偏ることなくデータを集めるためには、ランダムに聞いて回る必要がある。

ランダムに標本を抽出することを、\keyword{無作為抽出}という。

\end{document}
