\documentclass[../../../topic_calculus]{subfiles}

\begin{document}

\sectionline
\section{偏微分:一つずつ考えるアプローチ}
%\marginnote{\refbookA}

複数の要因が絡む状況を判断する際には、すべての要因を同時に考えるのではなく、まず1つの要因に着目し、次に視点を変えて別の要因を考え、そして最後に、個別に考察した要因を統合して考えることがある。

\br

\keyword{偏微分}のアイデアも、そのアプローチに似ている。

\begin{emphabox}
  \begin{spacebox}
    \begin{center}
    1つの変数を変化させるときは、他の変数は一定にしておく
  \end{center}
  \end{spacebox}
\end{emphabox}

多変数関数の偏微分では、1つの変数に注目し、それ以外の変数をいったん固定した状態で微分する。

\br

このように1つの変数に偏った微分ということで、\keywordJE{偏微分}{partial derivative}と名付けられている。
偏微分の英語訳\en{partial derivative}は、「部分的な」微分という意味である。

\end{document}
