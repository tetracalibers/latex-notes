\documentclass[../../../topic_calculus]{subfiles}

\begin{document}

\sectionline
\section{高次の偏微分}

多変数関数を1回偏微分すると、偏導関数という多変数関数が得られる。

その偏導関数をさらに偏微分して…というように、「偏導関数の偏導関数」を繰り返し考えていくことができる。

\subsection{同じ変数に関して偏微分を繰り返す場合}

たとえば、$x$に関する偏微分を2回、3回…と繰り返して得られる高次偏導関数は、次のように表記する。
\begin{equation*}
  \dfrac{\partial^2 f}{\partial x^2}, \quad
  \dfrac{\partial^3 f}{\partial x^3}, \quad \ldots
\end{equation*}

\subsection{異なる変数に関して偏微分を繰り返す場合}

$x$で偏微分してから$y$で偏微分する場合は、どのように表記すればよいだろうか?

\br

$x$に関する偏導関数$\dfrac{\partial f}{\partial x}$を、$y$に関して偏微分すると考えて、
\begin{equation*}
  \dfrac{\partial}{\partial y}\left(\dfrac{\partial f}{\partial x}\right)
\end{equation*}
これをひとまとめにして、次のように表記する。
\begin{equation*}
  \dfrac{\partial^2 f}{\partial y \partial x}
\end{equation*}

\end{document}
