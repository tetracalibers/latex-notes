\documentclass[../../../topic_calculus]{subfiles}

\begin{document}

\sectionline
\section{多変数関数のテイラー展開}

一変数関数の\hyperref[thm:taylor-expansion]{テイラー展開}は、次のようなものだった。

\begin{review}
  $f(x)$の$x = x_0$周りにおけるテイラー展開は、
  \begin{equation*}
  f(x) = f(x_0) + \frac{df}{dx}(x_0)(x - x_0) + \frac{1}{2!}\frac{d^2 f}{dx^2}(x_0)(x - x_0)^2 + \cdots
\end{equation*}
\end{review}

この式は、$x=x_0$の付近では、関数$f(x)$を右辺の多項式で近似できることを表している。

\br

ここで、$x_0$は$x$に近い点として、$x = x_0 + \Delta x$とおくと、$x - x_0 = \Delta x$となることも用いて、
\begin{equation*}
  f(x_0 + \Delta x) = f(x_0) + \frac{df}{dx}(x_0)\Delta x + \frac{1}{2!}\frac{d^2 f}{dx^2}(x_0)(\Delta x)^2 + \cdots
\end{equation*}

この式は、$x_0$の関数とみなせるので、$x_0$を変数$x$として書き換えておこう。

\begin{emphabox}
  \begin{spacebox}
    \begin{center}
      テイラー展開から、変数を微小変化させたときの関数の値が求まる
      \begin{equation*}
  f(x + \Delta x) = f(x) + \frac{df}{dx}(x)\Delta x + \frac{1}{2!}\frac{d^2 f}{dx^2}(x)(\Delta x)^2 + \cdots
\end{equation*}
    \end{center}
  \end{spacebox}
\end{emphabox}

さらに移項により、次のように書き換えられる。
\begin{equation*}
  f(x + \Delta x) - f(x) = \frac{df}{dx}(x)\Delta x + \frac{1}{2!}\frac{d^2 f}{dx^2}(x)(\Delta x)^2 + \cdots
\end{equation*}
この式の左辺は、$x$を$\Delta x$だけ微小変化させたときの関数$f(x)$の変化量を表している。
\begin{equation*}
  \Delta f = f(x + \Delta x) - f(x)
\end{equation*}

つまり、テイラー展開によって、変数を微小変化させたときの関数の変化量を表すことができる。

\begin{emphabox}
  \begin{spacebox}
    \begin{center}
      テイラー展開から、変数を微小変化させたときの関数の変化量が導かれる
      \begin{equation*}
  \Delta f = \frac{df}{dx}(x)\Delta x + \frac{1}{2!}\frac{d^2 f}{dx^2}(x)(\Delta x)^2 + \cdots
\end{equation*}
    \end{center}
  \end{spacebox}
\end{emphabox}

\subsection{二変数関数への拡張}

二変数関数$f(x,y)$においても同様に、$x$と$y$の値を微小変化させたときの関数の値
\begin{equation*}
  f(x + \Delta x, y + \Delta y)
\end{equation*}
を求めることを目指す。

\br

ここで、一変数関数の議論に帰着させるために、
\begin{equation*}
  F(s) = f(x + s\Delta x, y + s\Delta y)
\end{equation*}
という関数を考える。
これは$s$に関する一変数関数であり、$s=1$の場合を考えれば、目的の$f(x + \Delta x, y + \Delta y)$の値が得られる。

\br

この一変数関数$F(s)$を$s=0$の周りでテイラー展開すると、
\begin{equation*}
  F(s) = F(0) + F'(0)s + \frac{1}{2!}F''(0)s^2 + \cdots
\end{equation*}

\br

1階微分$F'(s)$は、$s$が出てくるたびに偏微分して、それらを足し合わせることで、
\begin{equation*}
  F'(s) = \frac{dF}{ds} = \frac{\partial f}{\partial x}(x + s\Delta x, y + s\Delta y)\Delta x + \frac{\partial f}{\partial y}(x + s\Delta x, y + s\Delta y)\Delta y
\end{equation*}
よって、$s = 0$での微分係数は、
\begin{equation*}
  F'(0) = \frac{\partial f}{\partial x}(x, y)\Delta x + \frac{\partial f}{\partial y}(x, y)\Delta y
\end{equation*}

\br

2階微分$F''(s)$は、1階微分$F'(s)$をさらに微分したものである。

1階微分$F'(s)$を次のように項に分けて考える。
\begin{equation*}
  F'(s) = F'_x(s) + F'_y(s)
\end{equation*}

まず、$F'(s)$のうち、
\begin{equation*}
  F'_x(s) \coloneqq \frac{\partial f}{\partial x}(x + s\Delta x, y + s\Delta y)\Delta x
\end{equation*}
の部分を2階微分すると、$s$が出てくるたびに偏微分することに注意して、
\begin{align*}
  F''_x(s) &= \frac{\partial}{\partial x}\left(\frac{\partial f}{\partial x}(x + s\Delta x, y + s\Delta y)\Delta x\right) + \frac{\partial}{\partial y}\left(\frac{\partial f}{\partial x}(x + s\Delta x, y + s\Delta y)\Delta x\right) \\
  &= \frac{\partial^2 f}{\partial x^2}(x + s\Delta x, y + s\Delta y)(\Delta x)^2 + \frac{\partial^2 f}{\partial y \partial x}(x + s\Delta x, y + s\Delta y)\Delta x \Delta y
\end{align*}

次に、$F'(s)$のうち、
\begin{equation*}
  F'_y(s) \coloneqq \frac{\partial f}{\partial y}(x + s\Delta x, y + s\Delta y)\Delta y
\end{equation*}
の部分を2階微分すると、同様に、
\begin{align*}
  F''_y(s) &= \frac{\partial}{\partial x}\left(\frac{\partial f}{\partial y}(x + s\Delta x, y + s\Delta y)\Delta y\right) + \frac{\partial}{\partial y}\left(\frac{\partial f}{\partial y}(x + s\Delta x, y + s\Delta y)\Delta y\right) \\
  &= \frac{\partial^2 f}{\partial x \partial y}(x + s\Delta x, y + s\Delta y)\Delta x \Delta y + \frac{\partial^2 f}{\partial y^2}(x + s\Delta x, y + s\Delta y)(\Delta y)^2
\end{align*}

よって、$s = 0$での2階微分は、
\begin{align*}
  F''(0) &= F''_x(0) + F''_y(0) \\
  &= \frac{\partial^2 f}{\partial x^2}(x, y)(\Delta x)^2 + \frac{\partial^2 f}{\partial y \partial x}(x, y)\Delta x \Delta y + \frac{\partial^2 f}{\partial x \partial y}(x, y)\Delta x \Delta y + \frac{\partial^2 f}{\partial y^2}(x, y)(\Delta y)^2
\end{align*}

偏微分が交換できる場合(偏導関数がともに連続な場合)は、次のようにまとめられる。
\begin{equation*}
  F''(0) = \frac{\partial^2 f}{\partial x^2}(x, y)(\Delta x)^2 + 2\frac{\partial^2 f}{\partial x \partial y}(x, y)\Delta x \Delta y + \frac{\partial^2 f}{\partial y^2}(x, y)(\Delta y)^2
\end{equation*}

\br

1階微分と2階微分の結果を、$F(s)$のテイラー展開の式に代入すると、
\begin{multline*}
  F(s) = F(0) + \left(\frac{\partial f}{\partial x}(x, y)\Delta x + \frac{\partial f}{\partial y}(x, y)\Delta y\right)s \\ + \frac{1}{2!}\left(\frac{\partial^2 f}{\partial x^2}(x, y)(\Delta x)^2 + 2\frac{\partial^2 f}{\partial x \partial y}(x, y)\Delta x \Delta y + \frac{\partial^2 f}{\partial y^2}(x, y)(\Delta y)^2\right)s^2 + \cdots
\end{multline*}
$s = 1$とおくことで、2変数関数のテイラー展開が得られる。
\begin{multline*}
  f(x + \Delta x, y + \Delta y) = f(x, y) + \frac{\partial f}{\partial x}(x, y)\Delta x + \frac{\partial f}{\partial y}(x, y)\Delta y \\ + \frac{1}{2!}\left(\frac{\partial^2 f}{\partial x^2}(x, y)(\Delta x)^2 + 2\frac{\partial^2 f}{\partial x \partial y}(x, y)\Delta x \Delta y + \frac{\partial^2 f}{\partial y^2}(x, y)(\Delta y)^2\right) + \cdots
\end{multline*}

\br

移項して、関数値の変化量$\Delta f = f(x + \Delta x, y + \Delta y) - f(x, y)$の形でまとめておこう。

\begin{emphabox}
  \begin{spacebox}
    \begin{center}
      テイラー展開から、変数を微小変化させたときの関数の変化量が導かれる
      \begin{multline*}
        \Delta f = \frac{\partial f}{\partial x}(x, y)\Delta x + \frac{\partial f}{\partial y}(x, y)\Delta y \\ + \frac{1}{2!}\left(\frac{\partial^2 f}{\partial x^2}(x, y)(\Delta x)^2 + 2\frac{\partial^2 f}{\partial x \partial y}(x, y)\Delta x \Delta y + \frac{\partial^2 f}{\partial y^2}(x, y)(\Delta y)^2\right) + \cdots
      \end{multline*}
    \end{center}
  \end{spacebox}
\end{emphabox}

\subsection{全微分との関係}

変数部分を省略して$\Delta f$を記述すると、
\begin{equation*}
  \Delta f = \frac{\partial f}{\partial x}\Delta x + \frac{\partial f}{\partial y}\Delta y + \frac{1}{2!}\left(\frac{\partial^2 f}{\partial x^2}(\Delta x)^2 + 2\frac{\partial^2 f}{\partial x \partial y}\Delta x \Delta y + \frac{\partial^2 f}{\partial y^2}(\Delta y)^2\right) + \cdots
\end{equation*}

ここで、$(\Delta x)^2, \Delta x \Delta y, (\Delta y)^2$などは、高次の微小量である。

\br

このように、2階以上の微分によって現れる項は高次の微小量を含んでいるため、1階微分で現れた項だけ残すと、
\begin{equation*}
  \Delta f \approx \frac{\partial f}{\partial x}\Delta x + \frac{\partial f}{\partial y}\Delta y
\end{equation*}
という近似式が得られる。そしてこの式は、まさに関数$f(x,y)$の\hyperref[sec:total-derivative]{全微分}になっている。

\br

つまり、\keyword{全微分}とは、高次の微小量を無視した近似である。

$\Delta x, \Delta y$が小さくなればなるほど、\hyperref[sec:diff-higher-infinitesimals]{高次の微小量}はそれ以上にはるかに小さくなり、無視できるほどになる。

\br

また、全微分は\hyperref[thm:tangent-plane-equation]{接平面の方程式}から導かれた。

$\Delta x, \Delta y$を小さくするということは、$x,y$の変化量(グラフ上の点の移動の幅)を小さくとることであり、これは、
\begin{emphabox}
  \begin{spacebox}
    \begin{center}
      ある点の付近では、接平面で多変数関数のグラフ(曲面)を近似できる
    \end{center}
  \end{spacebox}
\end{emphabox}
ことの根拠となっている。

\end{document}
