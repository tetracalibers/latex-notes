\documentclass[../../../topic_calculus]{subfiles}

\begin{document}

\sectionline
\section{方向微分係数と勾配ベクトル}

野山に雪が積もっていてスキーをするとしよう。

\begin{itemize}
  \item 直滑降では、最大傾斜の方向(ボールが転がり出す方向)を意識する
  \item 斜面を登るときは、高さが一定の方向にスキー板を向けて滑らないように横歩きする
\end{itemize}

高さが一定になる方向は、最大傾斜の方向(ボールが転がり出す方向)と垂直になっているはずである。
そうでなければ、スキー板が滑って降下が始まってしまう。

\br

野山の形状を$f(x,y)$のグラフとたとえて、このことを数式によって説明してみよう。

\subsection{内積による全微分の表現}

$z=f(x,y)$の全微分
\begin{equation*}
  \Delta z = \frac{\partial f}{\partial x}\Delta x + \frac{\partial f}{\partial y}\Delta y
\end{equation*}
は、次のようにベクトルの内積として捉えられる。
\begin{equation*}
  \Delta z = \begin{pmatrix} \dfrac{\partial f}{\partial x} \\ \dfrac{\partial f}{\partial y} \end{pmatrix} \cdot \begin{pmatrix} \Delta x \\ \Delta y \end{pmatrix}
\end{equation*}

ここで、偏導関数$\dfrac{\partial f}{\partial x},\dfrac{\partial f}{\partial y}$を並べたベクトルは、\keywordJE{勾配ベクトル}{gradient}と呼ばれる。

\begin{definition}{勾配ベクトル}\label{def:gradient}
  $f(x,y)$の$x,y$に関する偏導関数を並べたベクトルを、\keyword{勾配ベクトル}として次のように定義する。
  \begin{equation}
    \nabla f(x,y) \coloneqq \begin{pmatrix} \dfrac{\partial f}{\partial x} \\ \dfrac{\partial f}{\partial y} \end{pmatrix}
  \end{equation}
\end{definition}

すると、全微分の式は、次のように簡潔にまとめられる。
\begin{equation*}
  \Delta z = \nabla f(x,y) \cdot \begin{pmatrix} \Delta x \\ \Delta y \end{pmatrix}
\end{equation*}

\subsection{勾配ベクトルと高さが一定になる方向の直交}

ここで、$x,y$が変化しても関数$f(x,y)$の値$z$が変化しないような点、すなわち$\Delta z = 0$となる点を考えよう。

このとき、全微分の式は、
\begin{equation*}
  \nabla f \cdot \begin{pmatrix} \Delta x \\ \Delta y \end{pmatrix} = 0
\end{equation*}
となり、これは内積が0という式、すなわち\keyword{直交}を表す。

\br

こうして、
\begin{emphabox}
  \begin{spacebox}
    \begin{center}
      高さが一定($\Delta z = 0$)になる方向$\begin{pmatrix} \Delta x \\ \Delta y \end{pmatrix}$は、勾配ベクトル$\nabla f$と垂直
    \end{center}
  \end{spacebox}
\end{emphabox}
になっていることがわかった。

\subsection{勾配ベクトルと最大傾斜の方向}

たとえば、斜面にボールを置いたとき、ボールは最大傾斜の方向(最も下っていく方向)に転がっていくことは直観的にわかる。

\br

この最大傾斜の方向(ボールが転がり出す方向)が$\nabla f$になっていれば、
\begin{equation*}
  \nabla f \cdot \begin{pmatrix} \Delta x \\ \Delta y \end{pmatrix} = 0
\end{equation*}
という直交を表す式が、冒頭で述べた次の直観に当てはまることになる。

\begin{emphabox}
  \begin{spacebox}
    \begin{center}
      高さが一定になる方向は、最大傾斜の方向と垂直
    \end{center}
  \end{spacebox}
\end{emphabox}

\subsection{角度による方向微分係数}

傾斜を記述するには、傾きの大きさだけでなく、どの方向に傾斜しているかという情報も必要になる。

\br

地点$(x,y)$から出発して、微小に$(\Delta x, \Delta y)$だけ進んだときの関数$f(x,y)$の変化を調べてみよう。

\begin{center}
  \begin{tikzpicture}
        \def\vx{2}
        \def\vy{1.5}

        \coordinate (A) at (0,0);
        \coordinate (B) at (\vx, \vy);
        \coordinate (C) at (\vx, 0);

        % 角度
        \pic [fill=Emerald!30,angle eccentricity=1.3, angle radius=2em, "$\theta$"{text=Emerald}] {angle = C--A--B};
        
        % x軸方向のベクトル
        \draw[vector, very thick, Rhodamine] (C) -- (A) node[below, pos=0.4] {$\Delta x$};
        % y軸方向のベクトル
        \draw[vector, very thick, Cerulean] (B) -- (C) node[right, midway] {$\Delta y$};

        % ベクトル
        \draw[thick, BurntOrange] (B) -- (A) node[midway, above left] {$h$};
      \end{tikzpicture}
\end{center}

$\Delta x, \Delta y$は、方向$\theta$と距離$h$を用いて、次のように表される。
\begin{equation*}
  \Delta x = h \cos \theta , \quad
  \Delta y = h \sin \theta
\end{equation*}

このとき、関数$f(x,y)$の変化量は、$\Delta x, \Delta y$が微小であることから、
\begin{equation*}
  \Delta f = \frac{\partial f}{\partial x}\Delta x + \frac{\partial f}{\partial y}\Delta y
\end{equation*}
と見積もることができる。

\br

傾きは、この変化量を移動距離$h$で割って、$h \to 0$の極限をとることで求められる。
\begin{equation*}
  \lim_{h \to 0} \frac{\Delta f}{h}
= \lim_{h \to 0} \left( \frac{\partial f}{\partial x}\frac{\Delta x}{h} + \frac{\partial f}{\partial y}\frac{\Delta y}{h} \right)
= \frac{\partial f}{\partial x}\cos \theta + \frac{\partial f}{\partial y}\sin \theta
\end{equation*}

\br

この式が角度$\theta$方向の傾きであり、これを\keyword{$\theta$方向微分係数}と呼ぶ。
\begin{equation*}
  \frac{\partial f}{\partial x}\cos \theta + \frac{\partial f}{\partial y}\sin \theta
\end{equation*}

\subsection{ベクトルによる方向微分係数}

ここで、方向$\theta$を用いて、次のベクトルを考える。
\begin{equation*}
  \vb*{u} = \begin{pmatrix} \cos \theta \\ \sin \theta \end{pmatrix}
\end{equation*}
$\cos \theta^2 + \sin \theta^2 = 1$の関係から、$\vb*{u}$は長さ1のベクトルである。
\begin{equation*}
  \| \vb*{u} \| = \sqrt{\cos^2 \theta + \sin^2 \theta} = 1
\end{equation*}

\br

このベクトル$\vb*{u}$と勾配ベクトルを用いると、$\Delta x = h \cos \theta ,\Delta y = h sin \theta$の関係をまとめて次のように書ける。
\begin{equation*}
  \begin{pmatrix} \Delta x \\ \Delta y \end{pmatrix} = h \begin{pmatrix} \cos \theta \\ \sin \theta \end{pmatrix} = h \vb*{u}
\end{equation*}

\br

また、$\theta$方向微分係数は、$\nabla f$と$\vb*{u}$の内積として書くことができる。
\begin{equation*}
  \frac{\partial f}{\partial x}\cos \theta + \frac{\partial f}{\partial y}\sin \theta
= \nabla f \cdot \vb*{u}
\end{equation*}

\br

このように$\vb*{u}$で書き表された傾きを、\keyword{$\vb*{u}$方向微分係数}と呼ぶ。
\begin{equation*}
  \nabla f \cdot \vb*{u}
\end{equation*}

\subsection{最大傾斜}

方向$\vb*{u}$を変えたとき、$\vb*{u}$方向微分係数の最大値はどうなるだろうか?

\br

コーシー・シュワルツの不等式を用いると、
\begin{equation*}
  \| \nabla f \cdot \vb*{u} \| \leq \| \nabla f \| \cdot \| \vb*{u} \|
\end{equation*}
$\| \vb*{u} \| = 1$より、
\begin{equation*}
  \| \nabla f \cdot \vb*{u} \| \leq \| \nabla f \|
\end{equation*}

\br

つまり、傾きの大きさの最大値は$\| \nabla f \|$である。

また、等号が成立する場合を考えると、傾きが最大となる方向$\vb*{u}$は、
\begin{equation*}
  \vb*{u} = \frac{\nabla f}{\| \nabla f \|}
\end{equation*}
である。

\br

まとめると、
\begin{emphabox}
  \begin{spacebox}
    \begin{center}
      勾配ベクトル$\nabla f$は最大傾斜の方向を向いたベクトルで、\\
      その長さ$\| \nabla f \|$が傾きの最大値を与える
    \end{center}
  \end{spacebox}
\end{emphabox}
となり、スキーによる直観と矛盾ない性質が確認できた。

\end{document}
