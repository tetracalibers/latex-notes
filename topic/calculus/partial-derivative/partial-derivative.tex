\documentclass[../../../topic_calculus]{subfiles}

\begin{document}

\sectionline
\section{ある変数に関する偏微分係数}
%\marginnote{\refbookB p21〜23}

2変数関数$z = f(x, y)$において、$x$方向の傾きを考えてみる。

このとき、$y$は定数として固定する。
\begin{equation*}
  y = y_0
\end{equation*}

この平面$y=y_0$は、$x$軸と$z$軸に平行な平面である。

この平面で関数のグラフを切り、その切り口に現れた関数のグラフを微分することを考える。

\begin{center}
  \scalebox{1.2}{
    \begin{tikzpicture}[bullet/.style={circle,fill,inner sep=1pt},
 declare function={f(\x,\y)=2-0.5*pow(\x-1.25,2)-0.5*pow(\y-1,2);}]
 
 \begin{axis}[
    view={150}{45},
    colormap/bone,
    axis lines=middle,%
    zmax=2.2,zmin=0,
    xmin=-0.2,xmax=2.4,
    ymin=-0.2,ymax=2,%
    xlabel=$x$,ylabel=$y$,zlabel=$z$,
    xtick=\empty,ytick=\empty,ztick=\empty
  ]

  % 関数の曲面(平面の後ろ側)
  \addplot3[
    surf,shader=interp,
    domain=0.6:2,domain y=0.5:1.2,
    opacity=0.7
  ] {f(x,y)};
  
  % 目盛りと補助線
  \draw[dashed] (1.75,0,0) node[above left]{$x_0$} -- (1.75,1.2,0)
  node[bullet] (b1) {}  -- (0,1.2,0) node[above right]{$y_0$}
  (1.75,1.2,0) -- (1.75,1.2,{f(1.75,1.2)})node[bullet] {};
  
  % 切り口の平面
  \draw[opacity=0.5,fill=SkyBlue] (2,1.2,0) -- (0.6,1.2,0)
 -- (0.6,1.2,2.2) -- (2,1.2,2.2) -- cycle;

 % 関数の曲面(平面の前側)
  \addplot3[surf,shader=interp,domain=0.6:2,domain y=1.2:1.9,opacity=0.7] 
   {f(x,y)};
   
  % 切り口のグラフ
  \addplot3[thick,Rhodamine,domain=0.6:2,samples y=1]  ({x},1.2,{f(x,1.2)}); 
 \end{axis}

 % 点
 \node[anchor=north west] at (b1) {$(x_0,y_0)$}; 
\end{tikzpicture}
  }
\end{center}

切り口として現れるグラフは、$y = y_0$と$z=f(x, y)$の交線で、
\begin{equation*}
  \begin{cases} 
    x = x_0 \\ 
    z = f(x_0, y_0) 
  \end{cases}
\end{equation*}
という連立方程式を解いて得られる。

\br

この2式は、代入により次のような形にまとめられ、これが切り口を表している。
\begin{equation*}
  z = f(x, y_0)
\end{equation*}

切り口となる関数$z = f(x, y_0)$の$x = x_0$での接線の傾きが、求めたい$x$方向の傾きである。

\begin{center}
  \scalebox{1.2}{
    \begin{tikzpicture}[bullet/.style={circle,fill,inner sep=1pt},
 declare function={f(\x,\y)=2-0.5*pow(\x-1.25,2)-0.5*pow(\y-1,2);}]
 
 \begin{axis}[
    view={150}{45},
    colormap/bone,
    axis lines=middle,%
    zmax=2.2,zmin=0,
    xmin=-0.2,xmax=2.4,
    ymin=-0.2,ymax=2,%
    xlabel=$x$,ylabel=$y$,zlabel=$z$,
    xtick=\empty,ytick=\empty,ztick=\empty
  ]

  % 関数の曲面(平面の後ろ側)
  \addplot3[
    surf,shader=interp,
    domain=0.6:2,domain y=0.5:1.2,
    opacity=0.7
  ] {f(x,y)};
  
  % 目盛りと補助線
  \draw[dashed] (1.75,0,0) node[above left]{$x_0$} -- (1.75,1.2,0)
  node[bullet] (b1) {}  -- (0,1.2,0) node[above right]{$y_0$}
  (1.75,1.2,0) -- (1.75,1.2,{f(1.75,1.2)})node[bullet] {};
  
  % 切り口の平面
  \draw[opacity=0.5,fill=SkyBlue] (2,1.2,0) -- (0.6,1.2,0)
 -- (0.6,1.2,2.2) -- (2,1.2,2.2) -- cycle;

 % 関数の曲面(平面の前側)
  \addplot3[surf,shader=interp,domain=0.6:2,domain y=1.2:1.9,opacity=0.7] 
   {f(x,y)};
   
  % 切り口のグラフ
  \addplot3[thick,Rhodamine,domain=0.6:2,samples y=1]  ({x},1.2,{f(x,1.2)}); 
  
    % 接線
  \draw[thick, orange] (2,1.2,{f(2,1.2)}) -- (0.75,1.2,{f(1.75,1.2)+0.5}) coordinate[pos=0.2] (aux1);
 \end{axis}
 
  % 偏微分のラベル
  \draw[-Straight Barb, orange, densely dashed] (aux1) to[bend right] ++ (-1,1) node[above,align=center]{$\partial_xf(x,y)|_{x=x_0,y=y_0}$};
 
 % 点
 \node[anchor=north west] at (b1) {$(x_0,y_0)$}; 
\end{tikzpicture}
  }
\end{center}

切り口となる関数は$x$の1変数関数にすぎないので、$x$に関して普通に微分すればよい。

\br

$h$を微小量とし、$x = x_0$から少しだけ移動した点を$x = x_0 + h$とすると、次のように接線の傾きが計算できる。
\begin{equation*}
  \lim_{h \to 0} \frac{f(x_0 + h, y_0) - f(x_0, y_0)}{h}
\end{equation*}
この式を、関数$f(x,y)$の$(x_0,y_0)$における$x$に関する\keyword{偏微分係数}という。

\br

偏微分の場合は、通常の微分記号$\dfrac{d}{dx}$の代わりに、$\dfrac{\partial}{\partial x}$という記号を用いる。
\begin{equation*}
  \dfrac{\partial f}{\partial x}(x_0,y_0) = \lim_{h \to 0} \frac{f(x_0 + h, y_0) - f(x_0, y_0)}{h}
\end{equation*}

\sectionline
\section{ある変数に関する偏導関数}
%\marginnote{\refbookB p24}

偏微分係数は、$(x_0,y_0)$という値を1つ決めたときに、$\dfrac{\partial f}{\partial x}(x_0,y_0)$という値が1つ決まるという式である。

\br

$x$軸と$y$軸に平行な平面$y=y_0$は無数にあるので、$y_0$を変えれば、その切り口に現れる関数のグラフも異なるものになる。

また、切り口に現れる関数のグラフの傾きは、各点によって異なるので、$x_0$を変えれば、偏微分係数も異なる値になる。

\br

つまり、見方を変えれば、偏微分係数は$x_0$と$y_0$の2変数関数である。

\br

そこで、入力$(x_0,y_0)$を変数$(x,y)$に置き換えて、
\begin{equation*}
  \dfrac{\partial f}{\partial x}(x,y) = \lim_{h \to 0} \frac{f(x + h, y) - f(x, y)}{h}
\end{equation*}
という2変数関数を新たに考える。
これを$x$に関する\keyword{偏導関数}という。

\sectionline
\section{偏微分の記号}
%\marginnote{\refbookB p24〜25}

偏導関数の記号にはさまざまな表記法があるが、どれも同じものである。

\br

\begin{table}[h]
\renewcommand{\arraystretch}{1.5} % 行の高さを1.5倍に
\centering
\begin{tabular}{ll}
$\dfrac{\partial f(x,y)}{\partial x}$ & 微小量の変化の比 \\ 
$f_x(x,y)$ & 微分の省略形 $f'(x)$ の代わり(何に関する偏微分かを下に添えた) \\
$\partial_x f(x,y)$ & 偏微分するという操作を関数に施す \\
$\left(\dfrac{\partial f }{\partial x}\right)_y$ & 関数の変数を省略した形(止めている他の変数を下に添えた) \\
\end{tabular}
\end{table}


\end{document}
