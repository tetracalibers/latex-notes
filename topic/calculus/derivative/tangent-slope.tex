\documentclass[../../../topic_calculus]{subfiles}

\begin{document}

\sectionline
\section{接線の傾きとしての導関数}

傾きは位置$x$の関数$f'(x)$としたが、この関数がどのような関数なのか、結局傾きを計算する方法がわかっていない。

直線の傾きは$x$と$y$の増加率の比として定義されているから、まずはそれぞれの増加率を数式で表現しよう。

\begin{center}
  \scalebox{2}{
    \begin{tikzpicture}
      \def\xmin{-1};
      \def\xmax{3};
      \def\ymin{-1};
      \def\ymax{3};
      \def\fn#1{exp(0.5*#1) - 0.5};
      \def\dfn#1{0.5*exp(0.5*#1)}; % \fnの導関数
      \def\xi{0.75};
      \def\xj{1.5};

      % よく使う点の座標
      \coordinate (O) at (0,0);
      \coordinate (A) at (\xi, {\fn{\xi}});
      \coordinate (B) at (\xj, {\fn{\xi}});
      \coordinate (C) at (\xj, {\fn{\xj}});
      \coordinate (D) at (A |- C);

      % 座標軸
      \draw[axis] (\xmin,0) -- (\xmax,0) node [right]{$x$};
      \draw[axis] (0,\ymin) -- (0,\ymax) node [above]{$y$};

      % 原点
      \node at (O) [below left]{$O$};

      % 傾きを表す三角形
      \draw[fill=myPurple, myPurple!80!gray, opacity=0.5] (A) --(B) -- (C) -- cycle;

      \begin{scope}
        \clip (\xmin,\ymin) rectangle (\xmax,\ymax);
        % 接線
        \draw[orange] plot[domain=\xmin:\xmax] (\x,{\fn{\xi} + \dfn{\xi}*(\x-\xi)});
        % グラフ
        \draw[magenta,thick] plot[domain=\xmin:\xmax] (\x,{\fn{\x}});
      \end{scope}

      % x軸上の目盛り
      \node (X2) at (\xj,0) [below, scale=0.5]{$\strut x + \Delta x$};
      \node (X1) at (\xi,0) [below, scale=0.5, baseline = (X2.base)]{$\strut x$};

      % y軸上の目盛り
      \node at (0,{\fn{\xi}}) [left, scale=0.5]{$f(x)$};
      \node at (0,{\fn{\xj}}) [left, scale=0.5]{$f(x + \Delta x)$};

      % x軸からの補助線
      \draw[auxline, thin, lightgray] (\xi,0) -- (A);
      \draw[auxline, thin, lightgray] (\xj,0) -- (B);

      % y軸からの補助線
      \draw[auxline, thin, lightgray] (0,{\fn{\xi}}) -- (A);
      \draw[auxline, thin, lightgray] (0,{\fn{\xj}}) --(C);

      % \Delta xを表す矢印
      \draw[<->] ($(A)-(0,0.1)$) -- ($(B)-(0,0.1)$) node [midway, below]{$\Delta x$};
      % \Delta yを表す矢印
      \draw[<->] ($(A)+(-0.1,0)$) -- ($(D)+(-0.1, 0)$) node [midway, left]{$\Delta y$};
    \end{tikzpicture}
  }
\end{center}

この図から、$y$の増加率$\Delta y$は次のように表せることがわかる。

\begin{equation}
  \Delta y = f(x + \Delta x) - f(x)
\end{equation}

この両辺を$\Delta x$で割ると、$x$の増加率$\Delta x$と$y$の増加率$\Delta y$の比率が表せる。

\begin{equation}
  \frac{\Delta y}{\Delta x} = \frac{f(x + \Delta x) - f(x)}{\Delta x}
\end{equation}

図では$\Delta x$には幅があるが、この幅を限りなく$0$に近づけると、幅というより点になる。

つまり、$\Delta x \rightarrow 0$とすれば、$\dfrac{\Delta y}{\Delta x}$は任意の点$x$での接線の傾きとなる。

「任意の点$x$での傾き」も$x$の関数であり、この関数を導関数と呼ぶ。

\begin{definition}{導関数}
  関数$f(x)$の任意の点$x$における接線の傾き(増加の速さ)を表す関数を導関数といい、次のように定義する。
  \begin{equation}
    f'(x) = \lim_{\Delta x \to 0} \frac{f(x + \Delta x) - f(x)}{\Delta x}
  \end{equation}
\end{definition}

\end{document}
