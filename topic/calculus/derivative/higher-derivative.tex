\documentclass[../../../topic_calculus]{subfiles}

\begin{document}

\section{高階微分}

関数$f(x)$を微分したもの$f'(x)$をさらに微分して、その結果をさらに微分して…というように、「導関数の導関数」を繰り返し考えていくことを高階微分という。

\br

まずは、2回微分した場合について定義しよう。

$f(x)$を2回微分したものは、ニュートン記法では$f''(x)$と表される。

ライプニッツ記法で表現するには、次のように考えるとよい。

\begin{equation}
  \dfrac{d}{dx} \left( \dfrac{d}{dx}f(x) \right) = \left( \dfrac{d}{dx} \right)^2 f(x) = \dfrac{d^2}{dx^2} f(x)
\end{equation}

\begin{definition}{二階微分(二階導関数)}
  関数$f(x)$を微分して得られた導関数$f'(x)$をさらに微分することを\keyword{二階微分}といい、その結果得られた導関数を\keyword{二階導関数}という。\\
  二階導関数は、次のように表記される。
  \begin{equation}
    f''(x) = \dfrac{d^2}{dx^2} f(x)
  \end{equation}
\end{definition}

$n$階微分も同様に定義される。

$n$が大きな値になると、プライム記号をつける表記では$f''''''''(x)$のようになってわかりづらいので、$f^{(n)}(x)$のようにプライムの数$n$を添える記法がよく使われる。

\begin{definition}{$n$階微分($n$階導関数)}
  関数$f(x)$を$n$回微分することを\keyword{$n$階微分}といい、その結果得られた導関数を\keyword{$n$階導関数}という。\\
  $n$階導関数は、次のように表記される。
  \begin{equation}
    f^{(n)}(x) = \dfrac{d^n}{dx^n} f(x)
  \end{equation}
\end{definition}

\subsection{冪関数の高階微分}

$n$次の冪関数$f(x)=x^n$を$k$回微分すると、次のようになる。

\begin{align}
  f(x)       & = x^n                               \\
  f'(x)      & = nx^{n-1}                          \\
  f''(x)     & = n(n-1)x^{n-2}                     \\
  f'''(x)    & = n(n-1)(n-2)x^{n-3}                \\
  \vdots     &                                     \\
  f^{(k)}(x) & = n(n-1)(n-2)\cdots(n-(k-1))x^{n-k} \\
             & = n(n-1)(n-2)\cdots(n-k+1)x^{n-k}
\end{align}

ここで、$k=n$とすると、

\begin{align}
  f^{(n)}(x) & = n(n-1)(n-2)\cdots(n-n+1)x^{n-n} \\
             & = n(n-1)(n-2)\cdots 1 \cdot x^0   \\
             & = n(n-1)(n-2)\cdots 1             \\
             & = n!
\end{align}

となり、$n$階微分した時点で定数$n!$になるので、これ以上微分すると$0$になる。

\begin{equation}
  f^{(n+1)}(x) = 0
\end{equation}

\begin{theorem}{$n$次冪関数の高階微分}
  \titlegap
  $n$次冪関数$f(x) = x^n$の$n$階微分は$n!$となり、$n+1$回以上微分すると$0$になる。
  \LARGE
  \begin{equation}
    f(x) = x^n \quad \Longrightarrow \quad \begin{aligned}
      f^{(n)}(x)   & = n! \\
      f^{(n+1)}(x) & = 0
    \end{aligned}
  \end{equation}
\end{theorem}

\subsection{指数関数の高階微分}

ネイピア数を底とする指数関数$f(x)=e^x$は、何度微分しても変わらない関数である。

\begin{align}
  f(x)       & = e^x \\
  f'(x)      & = e^x \\
  f''(x)     & = e^x \\
  f'''(x)    & = e^x \\
  \vdots     &       \\
  f^{(n)}(x) & = e^x
\end{align}

\begin{theorem}{ネイピア数を底とする指数関数の高階微分}
  \titlegap
  $e$を底とする指数関数$f(x) = e^x$の$n$階微分は変わらず$e^x$となる。
  \LARGE
  \begin{equation}
    f(x) = e^{x} \quad \Longrightarrow \quad f^{(n)}(x) = e^{x}
  \end{equation}
\end{theorem}

指数が$k$倍されている場合$f(x)=e^{kx}$は、微分するたびに$k$が前に落ちてきて、$n$階微分すると$k^n$が前につくことになる。

\begin{align}
  f(x)       & = e^{kx}     \\
  f'(x)      & = ke^{kx}    \\
  f''(x)     & = k^2 e^{kx} \\
  f'''(x)    & = k^3 e^{kx} \\
  \vdots     &              \\
  f^{(n)}(x) & = k^ne^{kx}
\end{align}

\begin{theorem}{ネイピア数を底とする指数関数の高階微分(指数が定数倍されている場合)}
  $e$を底とし、指数が定数$k$倍された指数関数$f(x) = e^{kx}$の$n$階微分は$k^n e^{kx}$となる。
  \begin{equation}
    f(x) = e^{kx} \quad \Longrightarrow \quad f^{(n)}(x) = k^n e^{kx}
  \end{equation}
\end{theorem}

\end{document}
