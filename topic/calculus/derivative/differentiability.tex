\documentclass[../../../topic_calculus]{subfiles}

\begin{document}

\sectionline
\section{不連続点と微分可能性}

点$x$において連続な関数であれば、幅$\Delta x$を小さくすれば、その間の変化量$\Delta y$も小さくなるはずである。

\begin{center}
  \scalebox{2}{
    \begin{tikzpicture}
      \def\xmin{-1};
      \def\xmax{3};
      \def\ymin{-1};
      \def\ymax{3};
      \def\fn#1{exp(0.5*#1) - 0.5};
      \def\dfn#1{0.5*exp(0.5*#1)}; % \fnの導関数
      \def\xi{0.75};
      \def\xj{1};

      % よく使う点の座標
      \coordinate (O) at (0,0);
      \coordinate (A) at (\xi, {\fn{\xi}});
      \coordinate (B) at (\xj, {\fn{\xi}});
      \coordinate (C) at (\xj, {\fn{\xj}});
      \coordinate (D) at (A |- C);

      % 座標軸
      \draw[axis] (\xmin,0) -- (\xmax,0) node [right]{$x$};
      \draw[axis] (0,\ymin) -- (0,\ymax) node [above]{$y$};

      % 原点
      \node at (O) [below left]{$O$};

      % 傾きを表す三角形
      \draw[fill=myPurple, myPurple!80!gray, opacity=0.5] (A) --(B) -- (C) -- cycle;

      \begin{scope}
        \clip (\xmin,\ymin) rectangle (\xmax,\ymax);
        % 接線
        \draw[orange] plot[domain=\xmin:\xmax] (\x,{\fn{\xi} + \dfn{\xi}*(\x-\xi)});
        % グラフ
        \draw[magenta,thick] plot[domain=\xmin:\xmax] (\x,{\fn{\x}});
      \end{scope}

      % x軸上の目盛り
      \node (X2) at (\xj,0.1) [below right, scale=0.5]{$\strut x + \Delta x$};
      \node (X1) at (\xi,0.1) [below, scale=0.5, baseline = (X2.base)]{$\strut x$};

      % y軸上の目盛り
      \node at (0,{\fn{\xi}}) [left, scale=0.5]{$f(x)$};
      \node at (0,{\fn{\xj}}) [left, scale=0.5]{$f(x + \Delta x)$};

      % x軸からの補助線
      \draw[auxline, thin, lightgray] (\xi,0) -- (A);
      \draw[auxline, thin, lightgray] (\xj,0) -- (B);

      % y軸からの補助線
      \draw[auxline, thin, lightgray] (0,{\fn{\xi}}) -- (A);
      \draw[auxline, thin, lightgray] (0,{\fn{\xj}}) --(C);

      % \Delta xを表す矢印
      \draw[<->] ($(A)-(0,0.1)$) -- ($(B)-(0,0.1)$) node [midway, below]{$\Delta x$};
      % \Delta yを表す矢印
      \draw[<->] ($(A)+(-0.1,0)$) -- ($(D)+(-0.1, 0)$) node [midway, left]{$\Delta y$};
    \end{tikzpicture}
  }
\end{center}

しかし、不連続な点について考える場合は、そうはいかない。

下の図を見ると、$\Delta x$の幅を小さくしても、$\Delta y$は不連続点での関数の値の差の分までしか小さくならない。

\begin{figure}[H]
  \begin{minipage}{0.5\hsize}
    \centering
    \scalebox{1.5}{\begin{tikzpicture}
        \def\xmin{-1};
        \def\xmax{3};
        \def\ymin{-1};
        \def\ymax{3};
        \def\fnA#1{0.5*sin(deg(0.5*pi*#1))+1};
        \def\fnB#1{-0.5*cos(deg(0.5*pi*#1+0.25))+2};
        \def\xi{1};
        \def\xj{1.5};

        % よく使う点の座標
        \coordinate (O) at (0,0);
        \coordinate (A) at (\xi, {\fnA{\xi}});
        \coordinate (B) at (\xj, {\fnB{\xj}});
        \coordinate (C) at (\xj, {\fnA{\xi}});
        \coordinate (D) at (A |- B);

        % 原点
        \node at (O) [below left]{$O$};

        % 座標軸
        \draw[axis] (\xmin,0) -- (\xmax,0) node[right] {$x$};
        \draw[axis] (0,\ymin) -- (0,\ymax) node[above] {$y$};

        % 関数の描画
        \draw[domain=\xmin:\xi, samples=100, magenta,thick, smooth] plot (\x, {\fnA{\x}});
        \draw[domain=\xi:\xmax, samples=100, magenta,thick, smooth] plot (\x, {\fnB{\x}});

        % x軸上の目盛り
        \node (X2) at (\xj,0.15) [below right, scale=0.75]{$\strut x + \Delta x$};
        \node (X1) at (\xi,0.15) [below left, scale=0.75, baseline = (X2.base)]{$\strut x$};

        % y軸上の目盛り
        \node at (0,{\fnA{\xi}}) [left, scale=0.75]{$f(x)$};
        \node at (0,{\fnB{\xj}}) [left, scale=0.75]{$f(x + \Delta x)$};

        % x軸からの補助線
        \draw[auxline, thin, lightgray] (\xi,0) -- (A);
        \draw[auxline, thin, lightgray] (\xj,0) -- (B);

        % y軸からの補助線
        \draw[auxline, thin, lightgray] (0,{\fnA{\xi}}) -- (A);
        \draw[auxline, thin, lightgray] (0,{\fnB{\xj}}) --(B);

        % \Delta xを表す矢印
        \draw[<->] ($(A)-(0,0.1)$) -- ($(C)-(0,0.1)$) node [midway, below]{$\Delta x$};
        % \Delta yを表す矢印
        \draw[<->] ($(A)+(-0.1,0)$) -- ($(D)+(-0.1, 0)$) node [midway, left]{$\Delta y$};
      \end{tikzpicture}
    }
  \end{minipage}%
  \begin{minipage}{0.5\hsize}
    \centering
    \scalebox{1.5}{
      \begin{tikzpicture}
        \def\xmin{-1};
        \def\xmax{3};
        \def\ymin{-1};
        \def\ymax{3};
        \def\fnA#1{0.5*sin(deg(0.5*pi*#1))+1};
        \def\fnB#1{-0.5*cos(deg(0.5*pi*#1+0.25))+2};
        \def\xi{1};
        \def\xj{1};

        % よく使う点の座標
        \coordinate (O) at (0,0);
        \coordinate (A) at (\xi, {\fnA{\xi}});
        \coordinate (B) at (\xj, {\fnB{\xj}});
        \coordinate (C) at (\xj, {\fnA{\xi}});
        \coordinate (D) at (A |- B);

        % 原点
        \node at (O) [below left]{$O$};

        % 座標軸
        \draw[axis] (\xmin,0) -- (\xmax,0) node[right] {$x$};
        \draw[axis] (0,\ymin) -- (0,\ymax) node[above] {$y$};

        % 関数の描画
        \draw[domain=\xmin:\xi, samples=100, magenta,thick, smooth] plot (\x, {\fnA{\x}});
        \draw[domain=\xi:\xmax, samples=100, magenta,thick, smooth] plot (\x, {\fnB{\x}});

        % x軸上の目盛り
        \node (X2) at (\xj,0) [below, scale=0.6]{$ (x+ \Delta x) \simeq x$};

        % y軸上の目盛り
        \node at (0,{\fnA{\xi}}) [left, scale=0.75]{$f(x)$};
        \node at (0,{\fnB{\xj}}) [left, scale=0.75]{$f(x + \Delta x)$};

        % x軸からの補助線
        \draw[auxline, thin, lightgray] (\xi,0) -- (A);
        \draw[auxline, thin, lightgray] (\xj,0) -- (B);

        % y軸からの補助線
        \draw[auxline, thin, lightgray] (0,{\fnA{\xi}}) -- (A);
        \draw[auxline, thin, lightgray] (0,{\fnB{\xj}}) --(B);

        % \Delta xを表す矢印
        \draw ($(A)-(0,0.1)$) -- ($(C)-(0,0.1)$) node [midway, below]{$\Delta x$};
        % \Delta yを表す矢印
        \draw[<->] ($(A)+(-0.1,0)$) -- ($(D)+(-0.1, 0)$) node [midway, left]{$\Delta y$};
      \end{tikzpicture}
    }
  \end{minipage}
\end{figure}

このような不連続点においては、どんなに拡大しても、関数のグラフが直線にぴったりと重なることはない。

「拡大すれば直線に近似できる」というのが微分の考え方だが、不連続点ではこの考え方を適用できないのだ。

\br

関数の不連続点においては、微分という計算を考えることがそもそもできない。

ある点での関数のグラフが直線に重なる(微分可能である)ためには、$\Delta x \to 0$としたときに$\Delta y \to 0$となる必要がある。

\end{document}
