\documentclass[../../../topic_calculus]{subfiles}

\begin{document}

\sectionline
\section{定数関数の微分}

常に一定の値$c$を返す定数関数$f(x) = c$の微分はどうなるだろうか。

関数のグラフを描いて考えてみよう。

\begin{center}
  \begin{tikzpicture}
    \def\xmin{-1};
    \def\xmax{3};
    \def\ymin{-1};
    \def\ymax{3};
    \def\c{1.2}

    % よく使う点の座標
    \coordinate (O) at (0,0);

    % 座標軸
    \draw[axis] (\xmin,0) -- (\xmax,0) node [right]{$x$};
    \draw[axis] (0,\ymin) -- (0,\ymax) node [above]{$y$};

    % 原点
    \node at (O) [below left]{$O$};

    % グラフ
    \draw[magenta,thick] (\xmin,\c) -- (\xmax,\c) node [right]{$y = c$};

    % y軸上の目盛り
    \node at (0,{\c}) [above left]{$c$};
  \end{tikzpicture}
\end{center}

定数関数のグラフは、$x$軸に対して平行な直線であり、この直線の傾きは見るからに$0$である。

実際、導関数の定義に従って計算することで、定数関数の導関数は$0$になることを確かめられる。

\begin{review}
  導関数の定義
  \begin{equation}
    f'(x) = \lim_{\Delta x \to 0} \frac{f(x + \Delta x) - f(x)}{\Delta x}
  \end{equation}
\end{review}

どの点$x$においても$f(x)$が$c$を返すということは、$f(x+\Delta x)$も$c$であるため、

\begin{align}
  f'(x) & = \lim_{\Delta x \to 0} \dfrac{c - c}{\Delta x} \\
        & = \lim_{\Delta x \to 0} \dfrac{0}{\Delta x}     \\
        & = 0
\end{align}

となり、定数関数$f(x) = c$の微分の結果は$c$に依存せず、常に$0$になる。

\begin{theorem}{定数関数の微分}
  常に定数$c$の値をとる定数関数$f(x) = c$は、微分すると$0$になる。
  \begin{equation}
    \dfrac{d}{dx} c = 0
  \end{equation}
\end{theorem}

\end{document}
