\documentclass[../../../topic_calculus]{subfiles}

\begin{document}

\sectionline
\section{微分と高次の微小量}

まずは、基本的な例として、冪関数$y=x^n$の微分を考えてみよう。

\subsection{$y=x^2$の微分}

$y=f(x)=x^2$において、$x$を$dx$だけ微小変化させると、$y$は$dy$だけ変化するとする。

すると、微分の関係式は$y + dy = f(x + dx) = (x+dx)^2$となるが、これを次のように展開して考える。

\begin{equation}
  y + dy = (x + dx)(x + dx)
\end{equation}

右辺の$(x+dx)(x+dx)$からは、

\begin{itemize}
  \item $x^2$の項が1つ
  \item $xdx$の項が2つ
  \item $dx^2$の項が1つ
\end{itemize}

現れることになる。

数式で表すと、

\begin{equation}
  \eqnmarkbox[magenta]{Y1}{y} + dy = \eqnmarkbox[magenta]{Y2}{x^2} + 2xdx + dx^2
\end{equation}
\annotatetwo{below}{Y1}{Y2}{\bfseries 同じ}

ここで$y=x^2$なので、左辺の$y$と右辺の$x^2$は相殺される。

\begin{equation}
  dy = 2xdx + \fitLabelMath[BlueGreen][BlueGreen!40]{dx^2}{高次の微小量}
\end{equation}

さらに、$dx^2$の項は無視することができる。

なぜなら、$dx$を小さくすると、$dx^2$は$dx$とは比べ物にならないくらい小さくなってしまうからだ。

\begin{center}
  \begin{tikzpicture}
    \begin{scope}[local bounding box=left]
      \def\width{4}
      \def\dx{0.6}

      % 座標
      \coordinate (A) at (0,0);
      \coordinate (B) at (\width,0);
      \coordinate (C) at (\width,\width);
      \coordinate (D) at (0,\width);
      \coordinate (Cx) at ($(C)+(\dx,0)$);
      \coordinate (Cy) at ($(C)+(0,\dx)$);
      \coordinate (Cxy) at ($(C)+(\dx,\dx)$);
      \coordinate (Bx) at ($(B)+(\dx,0)$);
      \coordinate (Dy) at ($(D)+(0,\dx)$);

      % % 点(デバッグ用)
      % \fill (A) circle[radius=2pt] node[below left] {A};
      % \fill (B) circle[radius=2pt] node[below right] {B};
      % \fill (C) circle[radius=2pt] node[above right] {C};
      % \fill (D) circle[radius=2pt] node[above left] {D};
      % \fill (Cx) circle[radius=2pt] node[above right] {Cx};
      % \fill (Cy) circle[radius=2pt] node[above right] {Cy};
      % \fill (Cxy) circle[radius=2pt] node[above right] {Cxy};
      % \fill (Bx) circle[radius=2pt] node[below right] {Bx};
      % \fill (Dy) circle[radius=2pt] node[above left] {Dy};

      % 長方形
      \draw [fill=lightgray!20, draw=lightgray!80!gray] (A) rectangle (C) node[pos=.5] {$x^2$};
      \draw [fill=magenta!30, draw=magenta!70!gray] (B) rectangle (Cx) node[pos=.5, rotate=-90] {$xdx$};
      \draw [fill=magenta!30, draw=magenta!70!gray] (D) rectangle (Cy) node[pos=.5] {$xdx$};
      \draw [fill=BlueGreen!30, draw=BlueGreen!70!gray] (C) rectangle (Cxy) node[pos=.5] {$dx^2$};

      % 辺の長さを表す矢印
      \def\s{1em} % 辺と矢印の隙間
      \def\h{0.01} % 矢印の矢同士の隙間
      \draw[<->, thick, gray] ($(A)+(\h,-\s)$) -- ($(B)-(\h,\s)$) node[midway,below] {$x$};
      \draw[<->, thick, gray] ($(A)+(-\s,\h)$) -- ($(D)-(\s,\h)$) node[midway,left] {$x$};
      \draw[<->, thick, magenta!80] ($(B)+(\h,-\s)$) -- ($(Bx)-(\h,\s)$) node[midway,below] {$dx$};
      \draw[<->, thick, magenta!80] ($(D)+(-\s,\h)$) -- ($(Dy)-(\s,\h)$) node[midway,left] {$dx$};
    \end{scope}

    \begin{scope}[xshift=0.5\textwidth, local bounding box=right]
      \def\width{4}
      \def\dx{0.1}

      % 座標
      \coordinate (A) at (0,0);
      \coordinate (B) at (\width,0);
      \coordinate (C) at (\width,\width);
      \coordinate (D) at (0,\width);
      \coordinate (Cx) at ($(C)+(\dx,0)$);
      \coordinate (Cy) at ($(C)+(0,\dx)$);
      \coordinate (Cxy) at ($(C)+(\dx,\dx)$);
      \coordinate (Bx) at ($(B)+(\dx,0)$);
      \coordinate (Dy) at ($(D)+(0,\dx)$);

      % % 点(デバッグ用)
      % \fill (A) circle[radius=2pt] node[below left] {A};
      % \fill (B) circle[radius=2pt] node[below right] {B};
      % \fill (C) circle[radius=2pt] node[above right] {C};
      % \fill (D) circle[radius=2pt] node[above left] {D};
      % \fill (Cx) circle[radius=2pt] node[above right] {Cx};
      % \fill (Cy) circle[radius=2pt] node[above right] {Cy};
      % \fill (Cxy) circle[radius=2pt] node[above right] {Cxy};
      % \fill (Bx) circle[radius=2pt] node[below right] {Bx};
      % \fill (Dy) circle[radius=2pt] node[above left] {Dy};

      % 長方形
      \draw [fill=lightgray!20, draw=lightgray!80!gray] (A) rectangle (C) node[pos=.5] {$x^2$};
      \draw [fill=magenta!30, draw=magenta!70!gray] (B) rectangle (Cx);
      \draw [fill=magenta!30, draw=magenta!70!gray] (D) rectangle (Cy);
      \draw [fill=BlueGreen!30, draw=BlueGreen!70!gray] (C) rectangle (Cxy);
    \end{scope}

    \draw[->, thick] ($(left.east)+(1em, 0)$) -- ($(left-|right.west)-(1em,0)$) node[pos=.5, above] {$dx \to 0$};
  \end{tikzpicture}
\end{center}

というわけで、次のような式が得られる。

\begin{equation}
  dy = 2xdx
\end{equation}

よって、$y=x^2$の導関数は、$y'=2x$となることがわかった。

\begin{equation}
  \frac{dy}{dx} = 2x
\end{equation}

\subsection{$y=x^3$の微分}

同じように、$y=x^3$の微分を考えてみよう。

\begin{equation}
  y + dy = (x + dx)(x + dx)(x + dx)
\end{equation}

右辺の$(x+dx)(x+dx)(x+dx)$からは、

\begin{itemize}
  \item $x^3$の項が1つ
  \item $x^2dx$の項が3つ
  \item $dx^3$の項が1つ
\end{itemize}

現れることになる。

\begin{equation}
  \eqnmarkbox[magenta]{Y1}{y} + dy = \eqnmarkbox[magenta]{Y2}{x^3} + 3x^2dx + dx^3
\end{equation}
\annotatetwo{below}{Y1}{Y2}{\bfseries 同じ}

ここで$y=x^3$なので、左辺の$y$と右辺の$x^3$は相殺される。

\begin{equation}
  dy = 3x^2dx + \fitLabelMath[BlueGreen][BlueGreen!40]{dx^3}{高次の微小量}
\end{equation}

さらにここでは、$dx^3$の項を無視することができる。

次の図を見てみよう。

各辺$dx$の立方体は、$dx$を小さくすると、ほぼ点にしか見えないほど小さくなる。

つまり、各辺$dx$の立方体の体積$dx^3$は、考慮する必要がない。

\begin{center}
  \tdplotsetmaincoords{60}{125}
  \begin{tikzpicture}[
      tdplot_main_coords,
      grid/.style={very thin,gray},
      axis/.style={->,blue,thick},
      cube/.style={very thick,fill=lightgray!20, draw=lightgray!80!gray},
      cube_dx/.style={very thick,fill=magenta!30, draw=magenta!70!gray, opacity=0.6},
      cube_dx3/.style={very thick,fill=BlueGreen!30, draw=BlueGreen!70!gray, opacity=0.6},
      cube hidden/.style={thick, dashed, draw=lightgray!80!gray}
    ]
    \begin{scope}[local bounding box=left]
      \def\size{3}
      \def\dx{0.6}

      % 頂点の座標
      \coordinate (O) at (0,0,0);
      \coordinate (Axy) at (\size, 0, 0);
      \coordinate (Bxy) at (\size, \size, 0);
      \coordinate (Cxy) at (0, \size, 0);
      \coordinate (Ayz) at (0, 0, \size);
      \coordinate (Byz) at (0, \size, \size);
      \coordinate (Axz) at (\size, 0, \size);
      \coordinate (Axyz) at (\size, \size, \size);

      %%% 元の立方体
      % draw the front-right of the cube
      \draw[cube] (Axy) -- (Bxy) -- (Axyz) -- (Axz) -- cycle;
      % draw the front-left of the cube
      \draw[cube] (Cxy) -- (Bxy) -- (Axyz) -- (Byz) -- cycle;
      % draw the top of the cube
      \draw[cube] (Ayz) -- (Byz) -- (Axyz) -- (Axz) -- cycle;

      %%% x軸方向にdxだけ拡張
      % 底面
      \draw[cube_dx] (Axy) -- ($(Axy)+(\dx,0,0)$) -- ($(Bxy)+(\dx,0,0)$) -- (Bxy) -- cycle;
      % 後ろの面
      \draw[cube_dx] (Axz) -- (Axy) -- ($(Axy)+(\dx,0,0)$) -- ($(Axz)+(\dx,0,0)$) -- cycle;
      % 前面
      \draw[cube_dx] (Axyz) -- (Bxy) -- ($(Bxy)+(\dx,0,0)$) -- ($(Axyz)+(\dx,0,0)$) -- cycle;
      % 上面
      \draw[cube_dx] (Axz) -- ($(Axz)+(\dx,0,0)$) -- ($(Axyz)+(\dx,0,0)$) -- (Axyz) -- cycle;
      % 正方形
      \draw[cube_dx] ($(Axy)+(\dx,0,0)$) -- ($(Bxy)+(\dx,0,0)$) -- ($(Axyz)+(\dx,0,0)$) -- ($(Axz)+(\dx,0,0)$) -- cycle;

      %%% y軸方向にdxだけ拡張
      % 底面
      \draw[cube_dx] (Bxy) -- ($(Bxy)+(0,\dx,0)$) -- ($(Cxy)+(0,\dx,0)$) -- (Cxy) -- cycle;
      % 後ろの面
      \draw[cube_dx] (Cxy) -- ($(Cxy)+(0,\dx,0)$) -- ($(Byz)+(0,\dx,0)$) -- (Byz) -- cycle;
      % 前面
      \draw[cube_dx] (Axyz) -- ($(Axyz)+(0,\dx,0)$) -- ($(Bxy)+(0,\dx,0)$) -- (Bxy) -- cycle;
      % 上面
      \draw[cube_dx] (Axyz) -- ($(Axyz)+(0,\dx,0)$) -- ($(Byz)+(0,\dx,0)$) -- (Byz) -- cycle;
      % 正方形
      \draw[cube_dx] ($(Axyz)+(0,\dx,0)$) -- ($(Bxy)+(0,\dx,0)$) -- ($(Cxy)+(0,\dx,0)$) -- ($(Byz)+(0,\dx,0)$) -- cycle;

      %%% z軸方向にdxだけ拡張
      % 後ろの面
      \draw[cube_dx] (Ayz) -- ($(Ayz)+(0,0,\dx)$) -- ($(Byz)+(0,0,\dx)$) -- (Byz) -- cycle;
      % 左の面
      \draw[cube_dx] (Ayz) -- ($(Ayz)+(0,0,\dx)$) -- ($(Axz)+(0,0,\dx)$) -- (Axz) -- cycle;
      % 右の面
      \draw[cube_dx] (Byz) -- ($(Byz)+(0,0,\dx)$) -- ($(Axyz)+(0,0,\dx)$) -- (Axyz) -- cycle;
      % 前の面
      \draw[cube_dx] (Axz) -- ($(Axz)+(0,0,\dx)$) -- ($(Axyz)+(0,0,\dx)$) -- (Axyz) -- cycle;
      % 正方形
      \draw[cube_dx] ($(Ayz)+(0,0,\dx)$) -- ($(Byz)+(0,0,\dx)$) -- ($(Axyz)+(0,0,\dx)$) -- ($(Axz)+(0,0,\dx)$) -- cycle;

      %%% 高次の微小量
      \draw[cube_dx3] (Axyz) -- ($(Axyz)+(\dx,0,0)$) -- ($(Axyz)+(\dx,0,\dx)$) -- ($(Axyz)+(0,0,\dx)$) -- cycle;
      \draw[cube_dx3] (Axyz) -- ($(Axyz)+(0,\dx,0)$) -- ($(Axyz)+(0,\dx,\dx)$) -- ($(Axyz)+(0,0,\dx)$) -- cycle;
      \draw[cube_dx3] (Axyz) -- ($(Axyz)+(\dx,0,0)$) -- ($(Axyz)+(\dx,\dx,0)$) -- ($(Axyz)+(0,\dx,0)$) -- cycle; % 底面
      \draw[cube_dx3] ($(Axyz)+(\dx,0,0)$) -- ($(Axyz)+(\dx,\dx,0)$) -- ($(Axyz)+(\dx,\dx,\dx)$) -- ($(Axyz)+(\dx,0,\dx)$) -- cycle;
      \draw[cube_dx3] ($(Axyz)+(0,\dx,0)$) -- ($(Axyz)+(\dx,\dx,0)$) -- ($(Axyz)+(\dx,\dx,\dx)$) -- ($(Axyz)+(0,\dx,\dx)$) -- cycle;
      \draw[cube_dx3] ($(Axyz)+(0,0,\dx)$) -- ($(Axyz)+(\dx,0,\dx)$) -- ($(Axyz)+(\dx,\dx,\dx)$) -- ($(Axyz)+(0,\dx,\dx)$) -- cycle; % 上面

      %%% 辺の長さを表す矢印
      % 辺と矢印の隙間
      \def\s{0.4}
      % dxを表す矢印
      \draw[<->, thick, magenta!80] ($(Axz)+(\s,0,0.15)$) -- ($(Axz)+(\s,0,\dx + 0.15)$) node[midway,left] {$dx$};
      % xを表す矢印
      \draw[<->, thick, gray] ($(Axy)+(\dx + \s,0,0.1)$) -- ($(Axz)+(\dx + \s,0,0.1)$) node[midway,left] {$x$};

      % draw dashed lines to represent hidden edges
      \draw[cube hidden] (O) -- (Axy);
      \draw[cube hidden] (O) -- (Cxy);
      \draw[cube hidden] (O) -- (Ayz);

      % % 座標軸(デバッグ用)
      % \draw[axis] (0,0,0) -- (3,0,0) node[anchor=west]{$x$};
      % \draw[axis] (0,0,0) -- (0,3,0) node[anchor=west]{$y$};
      % \draw[axis] (0,0,0) -- (0,0,3) node[anchor=west]{$z$};
      % % 点(デバッグ用)
      % \fill (O) circle[radius=2pt] node[above left] {O};
      % \fill (Axy) circle[radius=2pt] node[below left] {Axy};
      % \fill (Bxy) circle[radius=2pt] node[below right] {Bxy};
      % \fill (Cxy) circle[radius=2pt] node[above right] {Cxy};
      % \fill (Ayz) circle[radius=2pt] node[above left] {Ayz};
      % \fill (Byz) circle[radius=2pt] node[above right] {Byz};
      % \fill (Axz) circle[radius=2pt] node[above left] {Axz};
      % \fill (Axyz) circle[radius=2pt] node[above right] {Axyz};
    \end{scope}

    \begin{scope}[xshift=0.5\textwidth, local bounding box=right]
      \def\size{3}
      \def\dx{0.15}

      % 頂点の座標
      \coordinate (O) at (0,0,0);
      \coordinate (Axy) at (\size, 0, 0);
      \coordinate (Bxy) at (\size, \size, 0);
      \coordinate (Cxy) at (0, \size, 0);
      \coordinate (Ayz) at (0, 0, \size);
      \coordinate (Byz) at (0, \size, \size);
      \coordinate (Axz) at (\size, 0, \size);
      \coordinate (Axyz) at (\size, \size, \size);

      %%% 元の立方体
      % draw the front-right of the cube
      \draw[cube] (Axy) -- (Bxy) -- (Axyz) -- (Axz) -- cycle;
      % draw the front-left of the cube
      \draw[cube] (Cxy) -- (Bxy) -- (Axyz) -- (Byz) -- cycle;
      % draw the top of the cube
      \draw[cube] (Ayz) -- (Byz) -- (Axyz) -- (Axz) -- cycle;

      %%% x軸方向にdxだけ拡張
      % 底面
      \draw[cube_dx] (Axy) -- ($(Axy)+(\dx,0,0)$) -- ($(Bxy)+(\dx,0,0)$) -- (Bxy) -- cycle;
      % 後ろの面
      \draw[cube_dx] (Axz) -- (Axy) -- ($(Axy)+(\dx,0,0)$) -- ($(Axz)+(\dx,0,0)$) -- cycle;
      % 前面
      \draw[cube_dx] (Axyz) -- (Bxy) -- ($(Bxy)+(\dx,0,0)$) -- ($(Axyz)+(\dx,0,0)$) -- cycle;
      % 上面
      \draw[cube_dx] (Axz) -- ($(Axz)+(\dx,0,0)$) -- ($(Axyz)+(\dx,0,0)$) -- (Axyz) -- cycle;
      % 正方形
      \draw[cube_dx] ($(Axy)+(\dx,0,0)$) -- ($(Bxy)+(\dx,0,0)$) -- ($(Axyz)+(\dx,0,0)$) -- ($(Axz)+(\dx,0,0)$) -- cycle;

      %%% y軸方向にdxだけ拡張
      % 底面
      \draw[cube_dx] (Bxy) -- ($(Bxy)+(0,\dx,0)$) -- ($(Cxy)+(0,\dx,0)$) -- (Cxy) -- cycle;
      % 後ろの面
      \draw[cube_dx] (Cxy) -- ($(Cxy)+(0,\dx,0)$) -- ($(Byz)+(0,\dx,0)$) -- (Byz) -- cycle;
      % 前面
      \draw[cube_dx] (Axyz) -- ($(Axyz)+(0,\dx,0)$) -- ($(Bxy)+(0,\dx,0)$) -- (Bxy) -- cycle;
      % 上面
      \draw[cube_dx] (Axyz) -- ($(Axyz)+(0,\dx,0)$) -- ($(Byz)+(0,\dx,0)$) -- (Byz) -- cycle;
      % 正方形
      \draw[cube_dx] ($(Axyz)+(0,\dx,0)$) -- ($(Bxy)+(0,\dx,0)$) -- ($(Cxy)+(0,\dx,0)$) -- ($(Byz)+(0,\dx,0)$) -- cycle;

      %%% z軸方向にdxだけ拡張
      % 後ろの面
      \draw[cube_dx] (Ayz) -- ($(Ayz)+(0,0,\dx)$) -- ($(Byz)+(0,0,\dx)$) -- (Byz) -- cycle;
      % 左の面
      \draw[cube_dx] (Ayz) -- ($(Ayz)+(0,0,\dx)$) -- ($(Axz)+(0,0,\dx)$) -- (Axz) -- cycle;
      % 右の面
      \draw[cube_dx] (Byz) -- ($(Byz)+(0,0,\dx)$) -- ($(Axyz)+(0,0,\dx)$) -- (Axyz) -- cycle;
      % 前の面
      \draw[cube_dx] (Axz) -- ($(Axz)+(0,0,\dx)$) -- ($(Axyz)+(0,0,\dx)$) -- (Axyz) -- cycle;
      % 正方形
      \draw[cube_dx] ($(Ayz)+(0,0,\dx)$) -- ($(Byz)+(0,0,\dx)$) -- ($(Axyz)+(0,0,\dx)$) -- ($(Axz)+(0,0,\dx)$) -- cycle;

      %%% 高次の微小量
      \draw[cube_dx3] (Axyz) -- ($(Axyz)+(\dx,0,0)$) -- ($(Axyz)+(\dx,0,\dx)$) -- ($(Axyz)+(0,0,\dx)$) -- cycle;
      \draw[cube_dx3] (Axyz) -- ($(Axyz)+(0,\dx,0)$) -- ($(Axyz)+(0,\dx,\dx)$) -- ($(Axyz)+(0,0,\dx)$) -- cycle;
      \draw[cube_dx3] (Axyz) -- ($(Axyz)+(\dx,0,0)$) -- ($(Axyz)+(\dx,\dx,0)$) -- ($(Axyz)+(0,\dx,0)$) -- cycle; % 底面
      \draw[cube_dx3] ($(Axyz)+(\dx,0,0)$) -- ($(Axyz)+(\dx,\dx,0)$) -- ($(Axyz)+(\dx,\dx,\dx)$) -- ($(Axyz)+(\dx,0,\dx)$) -- cycle;
      \draw[cube_dx3] ($(Axyz)+(0,\dx,0)$) -- ($(Axyz)+(\dx,\dx,0)$) -- ($(Axyz)+(\dx,\dx,\dx)$) -- ($(Axyz)+(0,\dx,\dx)$) -- cycle;
      \draw[cube_dx3] ($(Axyz)+(0,0,\dx)$) -- ($(Axyz)+(\dx,0,\dx)$) -- ($(Axyz)+(\dx,\dx,\dx)$) -- ($(Axyz)+(0,\dx,\dx)$) -- cycle; % 上面

      % draw dashed lines to represent hidden edges
      \draw[cube hidden] (O) -- (Axy);
      \draw[cube hidden] (O) -- (Cxy);
      \draw[cube hidden] (O) -- (Ayz);

      % % 座標軸(デバッグ用)
      % \draw[axis] (0,0,0) -- (3,0,0) node[anchor=west]{$x$};
      % \draw[axis] (0,0,0) -- (0,3,0) node[anchor=west]{$y$};
      % \draw[axis] (0,0,0) -- (0,0,3) node[anchor=west]{$z$};
      % % 点(デバッグ用)
      % \fill (O) circle[radius=2pt] node[above left] {O};
      % \fill (Axy) circle[radius=2pt] node[below left] {Axy};
      % \fill (Bxy) circle[radius=2pt] node[below right] {Bxy};
      % \fill (Cxy) circle[radius=2pt] node[above right] {Cxy};
      % \fill (Ayz) circle[radius=2pt] node[above left] {Ayz};
      % \fill (Byz) circle[radius=2pt] node[above right] {Byz};
      % \fill (Axz) circle[radius=2pt] node[above left] {Axz};
      % \fill (Axyz) circle[radius=2pt] node[above right] {Axyz};
    \end{scope}

    \draw[->, thick] ($(left.east)+(1em, 0)$) -- ($(left-|right.west)-(1em,0)$) node[pos=.5, above] {$dx \to 0$};
  \end{tikzpicture}
\end{center}

というわけで、$y=x^3$の導関数は、$y'=3x^2$となることがわかった。

\begin{equation}
  \frac{dy}{dx} = 3x^2
\end{equation}

\subsection{$y=x^n$の微分($n$が自然数の場合)}

$n$が自然数だとすると、$y=x^n$の微分は、$y=x^2$や$y=x^3$の場合と同じように考えられる。

\begin{equation}
  y + dy = \underbrace{(x+dx)(x+dx) \cdots (x+dx)}_{n\text{個}}
\end{equation}

右辺の$(x+dx)(x+dx) \cdots (x+dx)$を展開しようすると、次のような3種類のかけ算が発生する。

\begin{itemize}
  \item $x$どうしのかけ算
  \item $x$と$dx$のかけ算
  \item $dx$どうしのかけ算
\end{itemize}

つまり、右辺からは、

\begin{itemize}
  \item $x^n$の項が1つ
  \item $x^{n-1}dx$の項が$n$個
  \item $dx^n$の項が1つ
\end{itemize}

という項が現れることになる。

そして、$x^n$は左辺の$y$と相殺され、$dx^n$の項は高次の微小量として無視できる。

すると、残るのは次のような式になるだろう。

\begin{equation}
  dy = nx^{n-1}dx
\end{equation}

この式は、$y=\alpha x$という直線の式によく似ている。

高次の$dx$の項$dx^n$を無視し、1次の$dx$の項だけ残したのは、微分という計算が微小範囲における直線での近似であるからだ。

あくまでも微小範囲での直線の式であることを表すために、$x, y$を$dx, dy$として、$dy=\alpha dx$という形の式になっていると考えればよい。

\begin{theorem}{自然数の冪を持つ冪関数の導関数}
  $n$が自然数のとき、$y=x^n$の導関数は次のようになる。
  \begin{equation}
    \frac{dy}{dx} = nx^{n-1}
  \end{equation}
\end{theorem}

\end{document}
