\documentclass[../../../topic_calculus]{subfiles}

\begin{document}

\sectionline
\section{接線:拡大したら直線に近似できる}

関数$y=f(x)$について、引数の値を$x=x_0$からわずかに増加させて、$x=x_0+\Delta x$にした場合の出力の変化を考える。

\begin{center}
  \scalebox{1.75}{
    \begin{tikzpicture}
      \def\xmin{-1};
      \def\xmax{3};
      \def\ymin{-1};
      \def\ymax{3};
      \def\fn#1{exp(0.5*#1) - 0.5};
      \def\dfn#1{0.5*exp(0.5*#1)}; % \fnの導関数
      \def\xi{0.75};
      \def\xj{1.5};

      % よく使う点の座標
      \coordinate (O) at (0,0);
      \coordinate (A) at (\xi, {\fn{\xi}});
      \coordinate (B) at (\xj, {\fn{\xi}});
      \coordinate (C) at (\xj, {\fn{\xj}});
      \coordinate (D) at (A |- C);

      % 座標軸
      \draw[axis] (\xmin,0) -- (\xmax,0) node [right]{$x$};
      \draw[axis] (0,\ymin) -- (0,\ymax) node [above]{$y$};

      % 原点
      \node at (O) [below left]{$O$};

      % 傾きを表す三角形
      \draw[fill=myPurple, myPurple!80!gray, opacity=0.5] (A) --(B) -- (C) -- cycle;

      \begin{scope}
        \clip (\xmin,\ymin) rectangle (\xmax,\ymax);
        % 接線
        \draw[orange] plot[domain=\xmin:\xmax] (\x,{\fn{\xi} + \dfn{\xi}*(\x-\xi)});
        % グラフ
        \draw[magenta,thick] plot[domain=\xmin:\xmax] (\x,{\fn{\x}});
      \end{scope}

      % x軸上の目盛り
      \node (X2) at (\xj,0) [below, scale=0.5]{$\strut x_0 + \Delta x$};
      \node (X1) at (\xi,0) [below, scale=0.5, baseline = (X2.base)]{$\strut x_0$};

      % y軸上の目盛り
      \node at (0,{\fn{\xi}}) [left, scale=0.5]{$f(x_0)$};
      \node at (0,{\fn{\xj}}) [left, scale=0.5]{$f(x_0 + \Delta x)$};

      % x軸からの補助線
      \draw[auxline, thin, lightgray] (\xi,0) -- (A);
      \draw[auxline, thin, lightgray] (\xj,0) -- (B);

      % y軸からの補助線
      \draw[auxline, thin, lightgray] (0,{\fn{\xi}}) -- (A);
      \draw[auxline, thin, lightgray] (0,{\fn{\xj}}) --(C);

      % \Delta xを表す矢印
      \draw[<->] ($(A)-(0,0.1)$) -- ($(B)-(0,0.1)$) node [midway, below]{$\Delta x$};
    \end{tikzpicture}
  }
\end{center}

このとき、増分の幅$\Delta x$を狭くしていく($\Delta x$の値を小さくしていく)と、$x=x_0$付近において、関数$y=f(x)$のグラフは直線にほとんど重なるようになる。

\begin{center}
  \scalebox{2}{
    \begin{tikzpicture}[spy using outlines={circle, magnification=4, size=2cm, connect spies}]
      \def\xmin{-1};
      \def\xmax{3};
      \def\ymin{-1};
      \def\ymax{3};
      \def\fn#1{exp(0.5*#1) - 0.5};
      \def\dfn#1{0.5*exp(0.5*#1)}; % \fnの導関数
      \def\xi{0.75};
      \def\xj{0.95};

      % よく使う点の座標
      \coordinate (O) at (0,0);
      \coordinate (A) at (\xi, {\fn{\xi}});
      \coordinate (B) at (\xj, {\fn{\xi}});
      \coordinate (C) at (\xj, {\fn{\xj}});
      \coordinate (D) at (A |- C);

      % 座標軸
      \draw[axis] (\xmin,0) -- (\xmax,0) node [right]{$x$};
      \draw[axis] (0,\ymin) -- (0,\ymax) node [above]{$y$};

      % 原点
      \node at (O) [below left]{$O$};

      % x軸からの補助線
      \draw[dotted, thin, lightgray] (\xi,0) -- (A);
      \draw[dotted, thin, lightgray] (\xj,0) -- (B);
      % y軸からの補助線
      \draw[dotted, thin, lightgray] (0,{\fn{\xi}}) -- (A);
      \draw[dotted, thin, lightgray] (0,{\fn{\xj}}) --(C);

      % 傾きを表す三角形
      \draw[fill=myPurple, myPurple!80!gray, opacity=0.5] (A) --(B) -- (C) -- cycle;

      \begin{scope}
        \clip (\xmin,\ymin) rectangle (\xmax,\ymax);
        % 接線
        \draw[orange] plot[domain=\xmin:\xmax] (\x,{\fn{\xi} + \dfn{\xi}*(\x-\xi)});
        % グラフ
        \draw[magenta,thick] plot[domain=\xmin:\xmax] (\x,{\fn{\x}});
      \end{scope}

      % x軸上の目盛り
      \node (X2) at (\xj,0) [below right, scale=0.5]{$\strut x_0 + \Delta x$};
      \node (X1) at (\xi,0) [below, scale=0.5, baseline = (X2.base)]{$\strut x_0$};

      % y軸上の目盛り
      \node at (0,{\fn{\xi}}) [left, scale=0.5]{$f(x_0)$};
      \node at (0,{\fn{\xj}}) [left, scale=0.5]{$f(x_0 + \Delta x)$};

      \spy [Aquamarine] on ($(A)!.5!(C)$) in node [left] at (3.5,-1.25);
    \end{tikzpicture}
  }
\end{center}

このように、関数$f(x)$は、ある点$x_0$の付近では、

\begin{equation}
  f(x) \simeq a(x - x_0) +b
\end{equation}

という直線に近似することができる。

\br

ここで、$f(x_0)$の値を考えると、

\begin{align}
  f(x_0) & = a(x_0 - x_0) + b \\
         & = a\cdot 0 + b     \\
         & = b
\end{align}

であるから、実は$b=f(x_0)$である。

\br

一方、$a$はこの直線の傾きを表す。

そもそも、傾きとは、$x$が増加したとき、$y$がどれだけ急に(速く)増加するかを表す量である。

関数のグラフを見ると、急激に上下する箇所もあれば、なだらかに変化する箇所もある。

つまり、ある点でグラフにぴったりと沿う直線(接線)を見つけたとしても、その傾きは場所によって異なる。

そこで、「傾きは位置$x$の関数」とみなして、次のように表現しよう。

\begin{equation}
  a = f'(x)
\end{equation}

これで、先ほどの直線の式を完成させることができる。

\begin{theorem}{関数の各点での直線による近似}\label{thm:pointwise-linear-approx}
  関数$f(x)$は、ある点$x_0$の付近では、
  \begin{equation}
    f(x) \simeq f(x_0) + f'(x)(x - x_0)
  \end{equation}
  という傾き$f'(x)$の直線に近似できる。
\end{theorem}

\end{document}
