\documentclass[../../../topic_calculus]{subfiles}

\begin{document}

\sectionline
\section{合成関数の微分}

合成関数の微分の一般的な式は、いろいろな関数の微分を考える上で重要な公式である。

\subsection{関数の微小変化量}

関数$f(x)$において、変数$x$を$dx$だけ微小変化させた式は、これまで何度も登場した。

\begin{equation}
  f(x + dx) = f(x) + \fitLabelMath{f'(x)dx}{増えた分}
\end{equation}

この式は、「$x$を$dx$だけ微小変化させることで、関数$f$の値は$f'(x)dx$だけ増加した」と捉えることもできる。

言い換えれば、関数$f$の微小変化量は$f'(x)dx$だということだ。

変化量という観点で眺めるには、次のように移項した式がわかりやすいかもしれない。

\begin{equation}
  \fitLabelMath{f(x + dx) - f(x)}{区間$dx$での変化} = \fitLabelMath{f'(x)dx}{変化量}
\end{equation}

関数$f$の微小変化量$f'(x)dx$を、$df$と表すことにしよう。

\subsection{合成関数の微分の関係式}

今回はさらに、$t= f(x)$を関数$g(t)$に放り込むことを考える。

$g(t)$についても、次のような微分の関係式が成り立つはずだ。

\begin{equation}
  g(t + dt) = g(t) + g'(t)dt
\end{equation}

合成関数$g(f(x))$を作るため、$t=f$(引数$(x)$を省略して書いた関数$f(x)$)を代入する。

\begin{equation}
  g(f + df) = g(f) + g'(f)df
\end{equation}

$f$を$f(x)$に、$df$を$f'(x)dx$に書き戻すと、

\begin{equation}
  g(f(x) + f'(x)dx) = g(f(x)) + g'(f(x))f'(x)dx
\end{equation}

となり、左辺の$g()$の中身$f(x) + f'(x)dx$は$f(x + dx)$と書き換えられるので、次の式を得る。

\begin{equation}
  g(f(x + dx)) = \origFn{g(f(x))} + \derivFn{g'(f(x))f'(x)}dx
\end{equation}

\begin{theorem}{合成関数の微分(ニュートン記法による表現)}
  合成関数$g(f(x))$の微分は、次の式で表される。
  \begin{equation}
    \left( g(f(x)) \right)' = f'(x)g'(f(x))
  \end{equation}
\end{theorem}

\subsection{連鎖律としての表現}

ニュートン記法による表現はなかなかに覚えづらい式に見えるが、ライプニッツ記法を使って書き直すと、実は単純な関係式になっている。

\begin{itemize}
  \item $\left( g(f(x)) \right)'$は、$g(f(x))$を$x$で微分したもの:$\dfrac{d}{dx} g(f(x))$
  \item $f'(x)$は、$f(x)$を$x$で微分したもの:$\dfrac{d}{dx} f(x)$
  \item $g'(f(x))$は、$g(t)$を$t$で微分したもの$\dfrac{d}{dt} g(t)$に、$t=f(x)$に代入したもの:$\dfrac{d}{df}g(f(x))$
\end{itemize}

として書き直すと、

\begin{equation}
  \dfrac{d}{dx} g(f(x)) = \dfrac{d}{dx}f(x) \cdot \dfrac{d}{df}g(f(x))
\end{equation}

さらに、引数を省略して書くと、

\begin{equation}
  \frac{dg}{dx} = \frac{df}{dx} \cdot \frac{dg}{df}
\end{equation}

これは、$df$を約分できると考えたら、当たり前の式になっている。

\begin{equation}
  \frac{dg}{dx} = \frac{\cancel{df}}{dx} \cdot \frac{dg}{\cancel{df}}
\end{equation}

\begin{theorem}{合成関数の微分(連鎖律:ライプニッツ記法による表現)}
  $y=f(x)$、$z=g(y)$という関係があるとき、次の式が成り立つ。
  \begin{equation}
    \frac{dz}{dx} = \frac{dz}{dy} \cdot \frac{dy}{dx}
  \end{equation}
  これは、$x$が微小変化すると$y$も微小変化し、さらに連鎖して$z$も微小変化するという関係から、\keyword{連鎖律}と呼ばれる。
\end{theorem}

\end{document}
