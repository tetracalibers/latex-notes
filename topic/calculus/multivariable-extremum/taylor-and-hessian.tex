\documentclass[../../../topic_calculus]{subfiles}

\begin{document}

\sectionline
\section{ヘッシアンによる極値の判定}

勾配ベクトルが零ベクトルになるという条件は、極値の判定条件として十分ではない。

ここで、\keyword{テイラー展開}を用いることで、より詳細な情報を得ることができる。

\begin{review}
  テイラー展開から、変数を微小変化させたときの関数の変化量が導かれる。
  \begin{equation*}
    \Delta f = \frac{\partial f}{\partial x}\Delta x + \frac{\partial f}{\partial y}\Delta y + \frac{1}{2!}\left(\frac{\partial^2 f}{\partial x^2}(\Delta x)^2 + 2\frac{\partial^2 f}{\partial x \partial y}\Delta x \Delta y + \frac{\partial^2 f}{\partial y^2}(\Delta y)^2\right) + \cdots
  \end{equation*}
\end{review}

$\Delta x, \Delta y$を臨界点からの微小な移動量とする。

臨界点では勾配ベクトルが零ベクトルであるため、一次の項は消えてしまう。
\begin{equation*}
  \frac{\partial f}{\partial x}\Delta x + \frac{\partial f}{\partial y}\Delta y = 0
\end{equation*}

よって、関数の変化量は次のようになる。
\begin{equation*}
  \Delta f = \frac{1}{2!}\left(\frac{\partial^2 f}{\partial x^2}(\Delta x)^2 + 2\frac{\partial^2 f}{\partial x \partial y}\Delta x \Delta y + \frac{\partial^2 f}{\partial y^2}(\Delta y)^2\right) + \cdots
\end{equation*}

\br

点$(x,y)$から微小変化させた付近を調べるので、3次以降の高次の項は省略して議論を進めよう。

ここで、2次の項の各係数を次のようにおく。
\begin{equation*}
  a = \frac{1}{2!}\frac{\partial^2 f}{\partial x^2}, \quad
  b = \frac{\partial^2 f}{\partial x \partial y}, \quad
  c = \frac{1}{2!}\frac{\partial^2 f}{\partial y^2}
\end{equation*}

すると、関数の変化量$\Delta f$は、
\begin{equation*}
  \Delta f = a(\Delta x)^2 + b\Delta x \Delta y + c(\Delta y)^2
\end{equation*}
という2次式としてみることができる。

\subsection{2次式の判別式による場合分け}

さらに、$(\Delta y)^2$でくくって、$t = \dfrac{\Delta x}{\Delta y}$とおくと、
\begin{align*}
  \Delta f &= (\Delta y)^2\left(a \left(\frac{\Delta x}{\Delta y}\right)^2 + b\frac{\Delta x}{\Delta y} + c\right) \\
  &= (\Delta y)^2(at^2 + bt + c)
\end{align*}

$(\Delta y)^2$は常に正であるので、$\Delta f$の符号は、$at^2 + bt + c$の符号によって決まる。

\br

ここで、次の性質を覚えておこう。
\begin{itemize}
  \item $a > 0$のとき、$at^2 + bt + c$は下に凸の放物線
  \item $a < 0$のとき、$at^2 + bt + c$は上に凸の放物線
\end{itemize}

$at^2 + bt + c$全体の符号は、\keyword{平方完成}を行うことで明らかになる。
\begin{align*}
  at^2 + bt + c & = a \left(t^2 + \frac{b}{a}t + \frac{c}{a}\right) \\
  &= a\left(t + \frac{b}{2a}\right)^2 - \frac{b^2 - 4ac}{4a} \\
  &= a\left(t + \frac{b}{2a}\right)^2 + \frac{1}{4a}\left(4ac - b^2\right)
\end{align*}

\begin{handout}
  乗法公式$(x + k) ^2 = x^2 + 2kx + k^2$を利用した次の形を、二次式の\keyword{平方完成}という。
\begin{equation*}
  (x + k) ^2 - k^2 = x^2 - 2kx
\end{equation*}

\begin{center}
  \begin{tikzpicture}
    \begin{scope}[local bounding box=leftTermA]
      \fill[SkyBlue] (0,0) rectangle +(1,1) node[white, midway] {$kx$};
      \fill[SkyBlue] (1,1) rectangle +(1,1) node[white, midway] {$kx$};
      \fill[Rhodamine] (0,1) rectangle +(1,1) node[white, midway] {$x^2$};
    \end{scope}

    \begin{scope}[local bounding box=rightTermA, shift={(3.5,0)}]
      \fill[SkyBlue] (0,0) rectangle +(1,1) node[white, midway] {$kx$};
      \fill[SkyBlue] (1,1) rectangle +(1,1) node[white, midway] {$kx$};
      \fill[lightslategray] (1,0) rectangle +(1,1) node[white, midway] {$k^2$};
      \fill[Rhodamine] (0,1) rectangle +(1,1) node[white, midway] {$x^2$};
    \end{scope}

    \begin{scope}[local bounding box=rightTermB, shift={(7,0)}]
      \draw[SkyBlue,dashed] (0,0) rectangle +(1,1) node[white, midway] {$kx$};
      \draw[SkyBlue,dashed] (1,1) rectangle +(1,1) node[white, midway] {$kx$};
      \fill[lightslategray] (1,0) rectangle +(1,1) node[white, midway] {$k^2$};
      \draw[Rhodamine,dashed] (0,1) rectangle +(1,1) node[white, midway] {$x^2$};
    \end{scope}

    % leftTermA = rightTermA + rightTermB
    \path (leftTermA.east) -- (rightTermA.west) node[midway] {$=$};
    \path (rightTermA.east) -- (rightTermB.west) node[midway] {$-$};
  \end{tikzpicture}
\end{center}
\end{handout}

\br

$\left(t + \dfrac{b}{2a}\right)^2$は常に正なので、結局$at^2 + bt + c$の符号は、
\begin{equation*}
  D = 4ac - b^2
\end{equation*}
の部分によって決まる。これは2次式の\keyword{判別式}の$-1$倍である。

\br

したがって、$\Delta f$の符号は次のように分類できる。
\begin{itemize}
  \item $D < 0$のとき、$at^2 + bt + c$は$t$軸と2点で交わるので、$\Delta f$は正にも負にもなる(\keyword{鞍点})
  \item $D > 0$のとき、$at^2 + bt + c$は$t$軸と交わらないので、$\Delta f$の正負が常に決まる
    \begin{itemize}
      \item $a > 0$のとき、$\Delta f > 0$なので、\keyword{極小値}をとる
      \item $a < 0$のとき、$\Delta f < 0$なので、\keyword{極大値}をとる
    \end{itemize}
\end{itemize}

\subsection{ヘッセ行列とヘッシアン}

ここで、係数$a,b,c$は次のようにおかれていた。
\begin{equation*}
  a = \frac{1}{2!}\frac{\partial^2 f}{\partial x^2}, \quad
  b = \frac{\partial^2 f}{\partial x \partial y}, \quad
  c = \frac{1}{2!}\frac{\partial^2 f}{\partial y^2}
\end{equation*}

これらを$D = 4ac - b^2$の式に代入すると、
\begin{align*}
  D &= 4\left(\frac{1}{2!}\frac{\partial^2 f}{\partial x^2}\right)\left(\frac{1}{2!}\frac{\partial^2 f}{\partial y^2}\right) - \left(\frac{\partial^2 f}{\partial x \partial y}\right)^2 \\
  & = \frac{\partial^2 f}{\partial x^2} \frac{\partial^2 f}{\partial y^2} - \left(\frac{\partial^2 f}{\partial x \partial y}\right)^2
\end{align*}

\br

この$D$は、次のような行列の行列式になっている。
\begin{equation*}
  D = \det\begin{pmatrix}
    \dfrac{\partial^2 f}{\partial x^2} & \dfrac{\partial^2 f}{\partial x \partial y} \\
    \dfrac{\partial^2 f}{\partial x \partial y} & \dfrac{\partial^2 f}{\partial y^2}
  \end{pmatrix}
\end{equation*}

ここで現れた行列は\keyword{ヘッセ行列}、その行列式は\keyword{ヘッシアン}と呼ばれる。

\br

先ほどの場合分けは、ヘッシアンと係数$a$の符号を用いて、次のようにまとめることができる。

\begin{theorem}{ヘッシアンによる極値の判定}
  $D$をヘッシアンとすると、関数$f(x,y)$の極大・極小は次のように判定できる。
  \begin{itemize}
    \item $\dfrac{\partial^2 f}{\partial x^2} > 0$かつ$D > 0$のとき、$\Delta f$は必ず正となり、点$(x,y)$において$f$は\keyword{極小値}をとる。
    \item $\dfrac{\partial^2 f}{\partial x^2} < 0$かつ$D > 0$のとき、$\Delta f$は必ず負となり、点$(x,y)$において$f$は\keyword{極大値}をとる。
    \item $D < 0$のとき、$\Delta f$は正にも負にもなるので、極大にも極小にもならない。
  \end{itemize}
\end{theorem}

\end{document}
