\documentclass[../../../topic_calculus]{subfiles}

\begin{document}

\sectionline
\section{極大・極小と最大・最小}

偏微分を駆使することで、多変数関数の局所的な性質を調べることができる。

特に、局所的に最大・最小であることを意味する言葉が、\keyword{極大・極小}である。

\br

たとえば、$z=f(x,y)$のグラフ(曲面)を野山の地形と想像してみる。

つまり、関数$f(x, y)$は地点$(x,y)$の標高を表している。

\br

このとき、山頂はその付近ではもっとも高いので、$f(x, y)$は極大値をとる。

山が1つしかなければ、この山頂で$f(x,y)$は最大値をとるが、実際には遠くにどんな高い山があるかわからない。

同様に、谷底はそのあたりではいちばん低い場所であるけれども、遠くにもっと低いところがあるかもしれない。

\br

このように、遠くの情報がわからないとしても、この付近での山頂や谷底を探そう、というのが\keyword{極大・極小}問題である。

\begin{emphabox}
  \begin{spacebox}
    \begin{center}
      \keyword{極大・極小}とは、「この付近では」最大・最小であること
    \end{center}
  \end{spacebox}
\end{emphabox}

\br

一方、\keyword{最大・最小}は大域的な問題であり、それを判定するには遠くの情報も必要となるので、ずっと難しい問題になる。

\end{document}
