\documentclass[../../../topic_calculus]{subfiles}

\begin{document}

\sectionline
\section{複数の変数によって定まる関数}
%\marginnote{\refbookA p155}

ものごとは通常、単一の要因だけではなく、複数の要因が絡みあっている。

さまざまな要因が関係する現象を数量的に分析するためには、1つの変数だけでなく、複数の変数を含む関数を使う。

\br

このようにいくつもの変数があって、それによって値が定まるような関数を\keyword{多変数関数}という。

\end{document}
