\documentclass[../../../topic_probability-statistics]{subfiles}

\begin{document}

\sectionline
\section{独立な試行の確率}
\marginnote{\refbookD p33}

2つの試行が互いに他方の結果に影響を及ぼさないとき、これらの試行は\keyword{独立}であるという

\br

このとき、$A$が起きたかどうかが$B$の起きやすさに影響しないので、
\begin{equation*}
  P(B|A) = P(B)
\end{equation*}
が成り立つ

よって、確率の乗法定理は次のように書き換えられる
\begin{equation*}
  P(A \cap B) = P(A) \cdot P(B)
\end{equation*}

\begin{theorem}{独立な試行の確率}
  2つの独立な試行$S,T$を行うとき、「試行$S$では事象$A$が起こり、試行$T$では事象$B$が起こる」という事象を$C$とすると、事象$C$の確率は、
  \begin{equation*}
    P(C) = P(A) \cdot P(B)
  \end{equation*}
\end{theorem}

\end{document}
