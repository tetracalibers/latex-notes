\documentclass[../../../topic_probability-statistics]{subfiles}

\begin{document}

\sectionline
\section{積事象の確率と確率の乗法定理}
\marginnote{\refbookD p30〜31、38〜39}

$A$かつ$B$が起こる場合が$n(A \cap B)$通りあるとすると、その確率は、次のような割合で表される
\begin{equation*}
  P(A \cap B) = \frac{n(A \cap B)}{n(U)}
\end{equation*}

\br

一方、積事象の確率は、次のように分けて考えることもできる

\begin{enumerate}
  \item 全体のうち$A$が起こる(この確率は$P(A)$)
  \item $A$が起こったとき、$B$が起こる(この確率は$P(B|A)$)
\end{enumerate}

全体のうち$A$が起こる場合の数を$n(A)$、$A$が起こった場合のうち$B$が起こる場合の数を$n(B|A)$とすると、場合の数の積の法則より、
\begin{equation*}
  n(A \cap B) = n(A) \cdot n(B|A)
\end{equation*}

よって、積事象$A \cap B$の確率は、
\begin{align*}
  P(A \cap B) & = \frac{n(A \cap B)}{n(U)}                    \\
              & = \frac{n(A) \cdot n(B|A)}{n(U)}              \\
              & = \frac{n(A)}{n(U)} \cdot \frac{n(B|A)}{n(U)} \\
              & = P(A) \cdot P(B|A)
\end{align*}
という形で表すことができる

\begin{theorem}{確率の乗法定理}
  積事象$A \cap B$の確率は、
  \begin{equation*}
    P(A \cap B) = P(A) \cdot P(B|A)
  \end{equation*}
\end{theorem}

\end{document}
