\documentclass[../../../topic_probability-statistics]{subfiles}

\begin{document}

\sectionline
\section{離散型確率変数の確率分布}
\marginnote{\refbookB p77〜79}

\begin{definition}{確率関数}
  離散型確率変数$X$が定数$x_i$という値を取る確率を次のようにおく
  \begin{equation*}
    P(X = x_i) = p_i
  \end{equation*}

  ただし、各確率$p_i$は0以上の値であり、その総和は1であるとする
  \begin{equation*}
    \sum_{i} p_i = 1, \quad p_i \geq 0
  \end{equation*}

  このとき、$P(X=x_i) = p_i$は$x_i$の関数なので、それぞれの値に対する確率を関数$f(x)$を用いて、次のように表すことができる
  \begin{equation*}
    f(x_i) = P(X = x_i)
  \end{equation*}

  この関数$f(x)$を$X$の\keyword{確率関数}といい、$x_i$と$p_i$との対応$f(x_i) = p_i$を$X$の\keyword{確率分布}という
\end{definition}

\sectionline
\section{連続型確率変数の確率分布}
\marginnote{\refbookB p79〜81}

\begin{definition}{確率密度関数}
  連続型確率変数$X$に対して、次の条件を満たす関数$f(x)$の存在を仮定する

  \todo{}
\end{definition}

\end{document}
