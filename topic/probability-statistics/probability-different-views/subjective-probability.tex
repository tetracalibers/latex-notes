\documentclass[../../../topic_probability-statistics]{subfiles}

\begin{document}

\sectionline
\section{主観確率とベイズ統計学}
\marginnote{\refbookB p63〜64}

確率を考えるにあたって、頻度的立場だけで十分とは言い切れない

\begin{enumerate}[label=\romanlabel]
  \item 本当にサイコロが均一な材料で作られ、完全な立方体になっているのか?
\end{enumerate}

…もしそうでないなら、頻度的立場は意味がないのではないか?

\begin{enumerate}[label=\romanlabel, resume]
  \item サイコロを投げて出る目は、投げた瞬間に決まっているのではないか?
\end{enumerate}

…もしそうだとすると、サイコロを手にしたときから出る目は決まっているので、そもそも確率なんて存在しないのではないか?

\begin{enumerate}[label=\romanlabel, resume]
  \item サイコロの目が出る確率なんて主観的なものでもよいのではないか?
\end{enumerate}

…たとえば、100回投げて1が30回出たら、その確率は$\dfrac{30}{100}$としてもよいのではないか?

\sectionline

(\romannum{iii})のような、「確率は主観的なものでよい」という立場の統計学は\keyword{ベイズ統計学}と呼ばれている

\br

この立場では、今までの情報、知識や経験などによって得られた確率を与え、これを\keyword{主観確率}という

\br

主観確率では、全く起こっていない、あるいはほとんど起こっていない事象や実験ごとに統計的規則が変わってしまうような事象の分析も可能になる

\end{document}
