\documentclass[../../../topic_probability-statistics]{subfiles}

\begin{document}

\sectionline
\section{公理論的立場の確率}
\marginnote{\refbookB p64〜67}

化学や物理では、ある現象を考えるとき、議論がしやすいように\keyword{理想状態}というものを考える

それと同じように、確率も理想化された状態で考えることにする

\br

サイコロでいえば、そのサイコロの根拠(均一な材料か、完全な立方体なのか、etc.)を問うのではなく、最初から理想化されたサイコロを考えるようにする

\br

そして、現実の問題と理想化された問題との間を\keyword{統計的検定}を使ってつなぐことにする

\begin{equation*}
  \text{\bfseries 現実の問題} \quad \xleftrightarrow{\quad\text{\small\bfseries 統計的検定}\quad} \quad \text{\bfseries 理想化された問題}
\end{equation*}

\br

確率を理想化された数学の世界で考えるために、確率をある公理を満たすものとして定義する

確率をこのように考える立場を\keyword{公理論的立場}という

\end{document}
