\documentclass[../../../topic_probability-statistics]{subfiles}

\begin{document}

\sectionline
\section{古典的立場の確率}
\marginnote{\refbookB p61〜62}

サイコロを1回投げると、1から6のいずれかの目が出る

\br

ここで、サイコロの目が1となる確率は、全体が6通りで、1が出る場合は1通りしかないため、$\dfrac{1}{6}$と考える

\br

確率をこのように考えることを\keyword{古典的立場}あるいは\keyword{組合せ的}という

\begin{definition}{古典的立場による確率}
  全体で$n$通りの場合があり、そのうちある事象$A$が起こる場合の数が$a$通りあるとき、事象$A$の起こる\keyword{確率}を次のように定義する
  \begin{equation*}
    P(A) = \frac{a}{n}
  \end{equation*}
  このように定義された確率を\keyword{算術的確率}あるいは\keyword{先験的確率}という
\end{definition}

\end{document}
