\documentclass[../../../topic_probability-statistics]{subfiles}

\begin{document}

\sectionline
\section{頻度的立場の確率}
\marginnote{\refbookB p62〜63}

実際には、6回サイコロを投げたときに、必ず1が1回出るわけではない

\br

サイコロを何回も(膨大な回数を繰り返し)投げれば、やがて1が出る確率は$\dfrac{1}{6}$に近づいていく

\br

一般的に述べると、$n$回中$k$回だけ1が出た場合の割合$\dfrac{k}{n}$は、$n$が大きくなるにつれて一定値$\dfrac{1}{6}$に近づいていく

確率に対するこのような考え方を\keyword{頻度的立場}という

\begin{definition}{頻度的立場による確率}
  試行を$n$回繰り返して行った場合に、ある事象$A$の起こった回数を$k(n)$とする

  \br

  試行回数$n$を増やしていくとき、割合$\dfrac{k(n)}{n}$が一定値$p$に近づくならば、$p$を事象$A$の起こる\keyword{確率}と定義する
  \begin{equation*}
    P(A) = p = \lim_{n \to \infty} \frac{k(n)}{n}
  \end{equation*}

  このように定義される確率を\keyword{統計的確率}あるいは\keyword{経験的確率}という
\end{definition}

\br

この立場の客観性を保証するものは、多数回の試行あるいは大量データによる結果であり、理論的な根拠になっているものは\keyword{大数の法則}である

\end{document}
