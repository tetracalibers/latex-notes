\documentclass[../../../topic_probability-statistics]{subfiles}

\begin{document}

%\sectionline
\section{中央値}
%\marginnote{}

データを大きさの順に並べたときにちょうど中央に位置する値を\keywordJE{中央値}{median}という。

\begin{definition}{中央値}
  $N$個の観測値$x_1, \ldots, x_N$を大きさの順に並べ替えたものを
  \begin{equation*}
    x_{(1)} \leq \cdots \leq x_{(N)}
  \end{equation*}
  とするとき、\keyword{中央値}は次のように定義される。
  \begin{equation*}
    \tilde{x} \coloneq
    \begin{cases}
      x_{(k+1)} & (N = 2k + 1) \\
      \dfrac{x_{(k)} + x_{(k+1)}}{2} & (N = 2k)
    \end{cases}
  \end{equation*}
\end{definition}

\end{document}
