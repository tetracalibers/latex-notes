\documentclass[../../../topic_probability-statistics]{subfiles}

\begin{document}

\sectionline
\section{標本空間}
\marginnote{\refbookB p65}

実験や観測を行うときに起こりうるすべての結果を集合に閉じ込める

\begin{definition}{標本空間}
  起こりうるすべての結果からなる集合を\keyword{標本空間}といい、$\Omega$や$U$などと表す
\end{definition}

\begin{definition}{根元事象}
  標本空間$\Omega$の元、すなわち、標本空間の中の各結果を\keyword{根元事象}という
\end{definition}

\end{document}
