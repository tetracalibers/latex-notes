\documentclass[../../../topic_probability-statistics]{subfiles}

\begin{document}

\sectionline
\section{事象と集合の同一視}
\marginnote{\refbookB p67}

\keyword{事象}という概念を導入することにより、実験や観測による結果を\keyword{集合}に対応させることができた
\begin{equation*}
  \text{\bfseries 観測結果(事象)} \quad \longleftrightarrow \quad \text{\bfseries 集合}
\end{equation*}
いわば、確率のもととなる事柄を集合に閉じ込めたことになる

\br

このような集合を使って、確率を定義することができる

\sectionline
\section{定義をつくる:集合演算に関する閉性の保証}
\marginnote{\refbookB p68}

確率が定義される事象の全体$\mathscr{A}$を考える

\br

このとき、確率を数学の世界に閉じ込めるため、$\mathscr{A}$に集合の演算に関して閉じていることを要求する

\br

具体的には、$\mathscr{A}$に対して次の3つを要求する
\begin{enumerate}[label=\romanlabel]
  \item 空事象$\emptyset$と全事象$\Omega$は$\mathscr{A}$に含まれる
  \item 事象$A,B$が$\mathscr{A}$に属するとき、その和事象$A \cup B$、積事象$A \cap B$、また、$A$の余事象$A^c$も$\mathscr{A}$に属する
  \item $A_1,A_2,\ldots,A_n,\ldots$が$\mathscr{A}$に属せば、$\displaystyle\bigcup_{n=1}^\infty A_n$も$\mathscr{A}$に属する
\end{enumerate}

ここで、$\displaystyle\bigcup_{n=1}^\infty A_n$は、「$A_1,A_2,\ldots,A_n,\ldots$のいずれかが起こる」という事象を表している

\br

たとえば、サイコロを投げる試行において、
\begin{itemize}
  \item 「いつかは1の目が出る」という事象を$A$
  \item 「$n$回目に初めて1の目が出る」という事象を$A_n$
\end{itemize}
とすると、
\begin{equation*}
  A = \bigcup_{n=1}^\infty A_n
\end{equation*}
と表される

\sectionline
\section{集合族と$\sigma$-加法族}
\marginnote{\refbookB p68〜69}

$\mathscr{A}$は事象の全体、つまり、部分集合の全体なので、集合の集まりである

このような「集合の集合」を\keyword{集合族}という

\br

そして、先ほどの3つの条件を満たす集合族$\mathscr{A}$を\keyword{$\sigma$-加法族}という

\sectionline
\section{定義をつくる:どの立場の確率でも成り立つ性質の抽出}
\marginnote{\refbookB p69〜70}

ある事象$A$の確率$P(A)$が満たすべき条件を考える

\subsection{確率の値のとりうる範囲}

確率$P(A)$は、事象$A$が起こる可能性を表すので、
\begin{itemize}
  \item 事象が全く起こらないとき:$P(A) =0$
  \item 事象が必ず起こるとき:$P(A) = 1$
\end{itemize}
と考えることができる

\br

そこで、確率$P(A)$は次の範囲の値をとりうるものとする
\begin{equation*}
  0 \leq P(A) \leq 1
\end{equation*}

しかし、この不等式だけでは、$P(A)$が0や1になるのはどんな場合なのかを示すことはできない

\br

「事象が必ず起こるときの確率は1」「事象が全く起こらないときの確率は0」という、直観的には当たり前の事実も確率の定義に含める必要がある

\subsection{全事象の確率}

事象$A$がいつも起こるときは、事象$A$は起こりうるすべての場合を含んでいることになるので、$A$は全事象である

そこで、全事象を$\Omega$とし、次の条件を定義に追加する
\begin{equation*}
  P(\Omega) = 1
\end{equation*}

\subsection{空事象の確率}

同様に、「事象が全く起こらないときの確率は0」という事実は、次のように表される
\begin{equation*}
  P(\emptyset) = 0
\end{equation*}
しかし、実はこの式は確率の定義の他の条件から導出できるので、定義には加えないことにする

\subsection{排反な事象の確率}

最後に、「互いに排反な事象は別々に起こる」という性質も、確率の定義に含めることが重要である

\br

$A$または$B$が起こる確率は、
\begin{equation*}
  P(A \cup B) = P(A) + P(B) - P(A \cap B)
\end{equation*}
であるが、$A$と$B$が互いに排反であるときは、
\begin{equation*}
  P(A \cup B) = P(A) + P(B)
\end{equation*}
これと同様のことが、事象が増えても成り立つことを定義として要求する
\begin{equation*}
  P(A_1 \cup A_2 \cup \cdots \cup A_n \cdots) = P(A_1) + P(A_2) + \cdots + P(A_n) + \cdots
\end{equation*}

\sectionline
\section{確率と確率空間の定義}
\marginnote{\refbookB p70}

以上の議論から、確率を次のように定義する

\begin{definition}{公理論的立場の確率と確率空間}
  標本空間$\Omega$の部分空間を$A$とし、$\mathscr{A}$を$\sigma$-加法族とする

  このとき、次の3つの性質を満たす関数$P(\cdot)$を事象$A$の\keyword{確率}、あるいは$(\Omega, \mathscr{A})$上の\keyword{確率}という
  \begin{enumerate}[label=\romanlabel]
    \item 任意の$A \in \mathscr{A}$に対して、$0 \leq P(A) \leq 1$
    \item $P(\Omega) = 1$
    \item \keyword{完全加法性}:任意の互いに排反な事象$A_1, \ldots, A_n, \ldots$に対して、$\displaystyle P\left(\bigcup_{i=1}^\infty A_i\right) = \sum_{i=1}^\infty P(A_i)$
  \end{enumerate}

  なお、3つの組$(\Omega, \mathscr{A}, P)$を\keyword{確率空間}という
\end{definition}

\end{document}
