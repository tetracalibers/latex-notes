\documentclass[../../../topic_deep-learning]{subfiles}

\begin{document}

\sectionline
\section{パーセプトロン}
\marginnote{\refbookC p21〜}

\keyword{パーセプトロン}は、複数の信号を入力として受け取り、一つの信号を出力するアルゴリズム

\br

ここでいう\keyword{信号}とは、電流や川のような「流れ」をもつものをイメージするとよい

電流が導線を流れ、電子を先に送り出すように、パーセプトロンの信号も流れを作り、情報を先へと伝達していく

\br

ただし、パーセプトロンの信号は、「流す/流さない」のニ値の値である

ここでは、0を「流さない」、1を「流す」とする

\subsection{一つの入力を受け取るパーセプトロン}

$x_1$は入力信号、$y_1$は出力信号、$w_1$は重みを表す

図中の円は\keyword{ニューロン}や\keyword{ノード}と呼ばれる

\begin{center}
  \begin{tikzpicture}[x=2.2cm, y=1.4cm]
    % ノード
    \node[node in] (N1-1) at (1, 0) {$x_1$};
    \node[node out] (N2-1) at (2, 0) {$y_1$};

    % 接続(矢印)
    \draw[connect arrow] (N1-1) -- (N2-1) node[midway, above,Plum] {$w_1$};

    % ラベル
    \node[above=1em] at (N1-1) {input};
    \node[above=1em] at (N2-1) {output};
  \end{tikzpicture}
\end{center}

入力信号は、ニューロンに送られる際に、それぞれに固有の重みが乗算される

ニューロンでは、送られた信号の総和が計算され、その総和がある限界値(閾値$\theta$)を超えた場合にのみ1を出力する(ニューロンが\keyword{発火}する)

\begin{equation*}
  y = \begin{cases}
    0 & (w_1 x_1 \leq \theta) \\
    1 & (w_1 x_1 > \theta)
  \end{cases}
\end{equation*}

\subsection{二つの入力を受け取るパーセプトロン}

パーセプトロンは、複数ある入力信号のそれぞれに固有の重みを持つ

そしてその重みは、各信号の重要性をコントロールする要素として働く

つまり、重みが大きければ大きいほど、その重みに対応する信号の重要性が高くなる

\begin{center}
  \begin{tikzpicture}[x=2.2cm,y=1.4cm]
    \message{^^JNeural network with uniform arrows}
    \readlist\Nnod{2,1}

    \foreachitem \N \in \Nnod{ % loop over layers
      \def\lay{\Ncnt} % alias of index of current layer
      \pgfmathsetmacro\prev{int(\Ncnt-1)} % previous layer index
      \message{^^J Layer \lay, N=\N, prev=\prev ->}

      % NODES
      \foreach \i [evaluate={\y=\N/2-\i; \x=\lay; \n=\nstyle;}] in {1,...,\N}{ % nodes
          \message{N\lay-\i, }
          \ifnum \lay=1
            \node[node \n] (N\lay-\i) at (\x,\y) {$x_\i$};
          \else
            \node[node \n] (N\lay-\i) at (\x,\y) {$y_\i$};
          \fi
        }

      % CONNECTIONS
      \foreach \i in {1,...,\N}{ % nodes in current layer
          \ifnum\lay>1
          \setAngles{N\prev-1}{N\lay-\i}{N\prev-\Nnod[\prev]} % for receiving node

          \ifnum\Nnod[\prev]=1
          \foreach \j in {1}{
              \pgfmathsetmacro\ang{\angmin}
              \setAngles{N\lay-1}{N\prev-\j}{N\lay-\N}
              \ifnum\N=1
                \pgfmathsetmacro\angout{\angmin}
              \else
                \pgfmathsetmacro\angout{\angmin+(\dang-360)*(\i-1)/(\N-1)}
              \fi
              \draw[connect arrow] (N\prev-\j.{\angout}) -- (N\lay-\i.{\ang}) node [midway,above,Plum] {$w_\j^{(\lay)}$};
            }
          \else
          \foreach \j [evaluate={\ang=\angmin+\dang*(\j-1)/(\Nnod[\prev]-1);}] in {1,...,\Nnod[\prev]}{
              \setAngles{N\lay-1}{N\prev-\j}{N\lay-\N}
              \ifnum\N=1
                \pgfmathsetmacro\angout{\angmin}
              \else
                \pgfmathsetmacro\angout{\angmin+(\dang-360)*(\i-1)/(\N-1)}
              \fi
              \draw[connect arrow] (N\prev-\j.{\angout}) -- (N\lay-\i.{\ang}) node [midway,above,Plum] {$w_\j$};
            }
          \fi
          \fi
        }
    }
  \end{tikzpicture}
\end{center}

\begin{equation*}
  y = \begin{cases}
    0 & (w_1 x_1 + w_2 x_2 \leq \theta) \\
    1 & (w_1 x_1 + w_2 x_2 > \theta)
  \end{cases}
\end{equation*}

\sectionline
\section{ベクトルと行列を用いた表現}

\subsection{複数の出力}

次のような場合は、それぞれ同じ入力を受け取る3つの独立したコンポーネントとして動作する

\begin{center}
  \begin{tikzpicture}[x=2.2cm,y=1.4cm]
    \message{^^JNeural network with uniform arrows}
    \readlist\Nnod{1,3}

    \foreachitem \N \in \Nnod{ % loop over layers
      \def\lay{\Ncnt} % alias of index of current layer
      \pgfmathsetmacro\prev{int(\Ncnt-1)} % previous layer index
      \message{^^J Layer \lay, N=\N, prev=\prev ->}

      % NODES
      \foreach \i [evaluate={\y=\N/2-\i; \x=\lay; \n=\nstyle;}] in {1,...,\N}{ % nodes
          \message{N\lay-\i, }
          \ifnum \lay=1
            \node[node \n] (N\lay-\i) at (\x,\y) {$x_\i$};
          \else
            \node[node \n] (N\lay-\i) at (\x,\y) {$y_\i$};
          \fi
        }

      % CONNECTIONS
      \foreach \i in {1,...,\N}{ % nodes in current layer
          \ifnum\lay>1
          \setAngles{N\prev-1}{N\lay-\i}{N\prev-\Nnod[\prev]} % for receiving node

          \ifnum\Nnod[\prev]=1
          \foreach \j in {1}{
              \pgfmathsetmacro\ang{\angmin}
              \setAngles{N\lay-1}{N\prev-\j}{N\lay-\N}
              \ifnum\N=1
                \pgfmathsetmacro\angout{\angmin}
              \else
                \pgfmathsetmacro\angout{\angmin+(\dang-360)*(\i-1)/(\N-1)}
              \fi
              \draw[connect arrow] (N\prev-\j.{\angout}) -- (N\lay-\i.{\ang}) node [midway,above,Plum] {$w_\i$};
            }
          \else
          \foreach \j [evaluate={\ang=\angmin+\dang*(\j-1)/(\Nnod[\prev]-1);}] in {1,...,\Nnod[\prev]}{
              \setAngles{N\lay-1}{N\prev-\j}{N\lay-\N}
              \ifnum\N=1
                \pgfmathsetmacro\angout{\angmin}
              \else
                \pgfmathsetmacro\angout{\angmin+(\dang-360)*(\i-1)/(\N-1)}
              \fi

              % ref: https://tex.stackexchange.com/questions/745061/how-to-use-a-variable-in-hsl-model-color-definition-for-a-color-wheel-with-xco
              \definecolor{arrowcolor}{Hsb}{\inteval{180+\i*36},1,0.8} % color of arrow

              \ifnum\i=\j
                \draw[connect arrow,arrowcolor!60] (N\prev-\j.{\angout}) -- (N\lay-\i.{\ang}) node [midway,above,arrowcolor] {$w_\j$};
              \else
                \draw[connect arrow,arrowcolor!60] (N\prev-\j.{\angout}) -- (N\lay-\i.{\ang});
              \fi
            }
          \fi
          \fi
        }
    }
  \end{tikzpicture}
\end{center}

\begin{equation*}
  \begin{pmatrix}
    y_1 \\
    y_2 \\
    y_3
  \end{pmatrix} = \begin{pmatrix}
    w_1 x_1 \\
    w_2 x_1 \\
    w_3 x_1
  \end{pmatrix}
\end{equation*}

\subsection{複数の入力と出力}

\begin{center}
  \begin{tikzpicture}[x=2.2cm,y=1.4cm]
    \message{^^JNeural network with uniform arrows}
    \readlist\Nnod{3,3}

    \foreachitem \N \in \Nnod{ % loop over layers
      \def\lay{\Ncnt} % alias of index of current layer
      \pgfmathsetmacro\prev{int(\Ncnt-1)} % previous layer index
      \message{^^J Layer \lay, N=\N, prev=\prev ->}

      % NODES
      \foreach \i [evaluate={\y=\N/2-\i; \x=\lay; \n=\nstyle;}] in {1,...,\N}{ % nodes
          \message{N\lay-\i, }
          \ifnum \lay=1
            \node[node \n] (N\lay-\i) at (\x,\y) {$x_\i$};
          \else
            \node[node \n] (N\lay-\i) at (\x,\y) {$y_\i$};
          \fi
        }

      % CONNECTIONS
      \foreach \i in {1,...,\N}{ % nodes in current layer
          \ifnum\lay>1
          \setAngles{N\prev-1}{N\lay-\i}{N\prev-\Nnod[\prev]} % for receiving node

          \ifnum\Nnod[\prev]=1
          \foreach \j in {1}{
              \pgfmathsetmacro\ang{\angmin}
              \setAngles{N\lay-1}{N\prev-\j}{N\lay-\N}
              \ifnum\N=1
                \pgfmathsetmacro\angout{\angmin}
              \else
                \pgfmathsetmacro\angout{\angmin+(\dang-360)*(\i-1)/(\N-1)}
              \fi
              \draw[connect arrow] (N\prev-\j.{\angout}) -- (N\lay-\i.{\ang});
            }
          \else
          \foreach \j [evaluate={\ang=\angmin+\dang*(\j-1)/(\Nnod[\prev]-1);}] in {1,...,\Nnod[\prev]}{
              \setAngles{N\lay-1}{N\prev-\j}{N\lay-\N}
              \ifnum\N=1
                \pgfmathsetmacro\angout{\angmin}
              \else
                \pgfmathsetmacro\angout{\angmin+(\dang-360)*(\i-1)/(\N-1)}
              \fi

              % ref: https://tex.stackexchange.com/questions/745061/how-to-use-a-variable-in-hsl-model-color-definition-for-a-color-wheel-with-xco
              \definecolor{arrowcolor}{Hsb}{\inteval{180+\i*36},1,0.8} % color of arrow
              \draw[connect arrow,arrowcolor] (N\prev-\j.{\angout}) -- (N\lay-\i.{\ang});
            }
          \fi
          \fi
        }
    }
  \end{tikzpicture}
\end{center}

「3つの重みが各出力ノードに渡される」という見方で、3つの独立した内積として考えてみる

\begin{center}
  \begin{tikzpicture}[x=2.2cm,y=1.4cm]
    \message{^^JNeural network with uniform arrows}
    \readlist\Nnod{3,3}

    \foreachitem \N \in \Nnod{ % loop over layers
      \def\lay{\Ncnt} % alias of index of current layer
      \pgfmathsetmacro\prev{int(\Ncnt-1)} % previous layer index
      \message{^^J Layer \lay, N=\N, prev=\prev ->}

      % NODES
      \foreach \i [evaluate={\y=\N/2-\i; \x=\lay; \n=\nstyle;}] in {1,...,\N}{ % nodes
          \message{N\lay-\i, }
          \ifnum \lay=1
            \node[node \n] (N\lay-\i) at (\x,\y) {$x_\i$};
          \else
            \node[node \n] (N\lay-\i) at (\x,\y) {$y_\i$};
          \fi
        }

      % CONNECTIONS
      \foreach \i in {1,...,\N}{ % nodes in current layer
          \ifnum\lay>1
          \setAngles{N\prev-1}{N\lay-\i}{N\prev-\Nnod[\prev]} % for receiving node

          \ifnum\Nnod[\prev]=1
          \foreach \j in {1}{
              \pgfmathsetmacro\ang{\angmin}
              \setAngles{N\lay-1}{N\prev-\j}{N\lay-\N}
              \ifnum\N=1
                \pgfmathsetmacro\angout{\angmin}
              \else
                \pgfmathsetmacro\angout{\angmin+(\dang-360)*(\i-1)/(\N-1)}
              \fi
              \draw[connect arrow] (N\prev-\j.{\angout}) -- (N\lay-\i.{\ang});
            }
          \else
          \foreach \j [evaluate={\ang=\angmin+\dang*(\j-1)/(\Nnod[\prev]-1);}] in {1,...,\Nnod[\prev]}{
              \setAngles{N\lay-1}{N\prev-\j}{N\lay-\N}
              \ifnum\N=1
                \pgfmathsetmacro\angout{\angmin}
              \else
                \pgfmathsetmacro\angout{\angmin+(\dang-360)*(\i-1)/(\N-1)}
              \fi

              % ref: https://tex.stackexchange.com/questions/745061/how-to-use-a-variable-in-hsl-model-color-definition-for-a-color-wheel-with-xco
              \definecolor{arrowcolor}{Hsb}{\inteval{180+\i*36},1,0.8} % color of arrow

              \ifnum\i=1
                \draw[connect arrow,arrowcolor] (N\prev-\j.{\angout}) -- (N\lay-\i.{\ang}) node [midway,above,arrowcolor] {$w_{1\j}$};
              \else
                \draw[connect arrow,arrowcolor!40] (N\prev-\j.{\angout}) -- (N\lay-\i.{\ang});
              \fi
            }
          \fi
          \fi
        }
    }
  \end{tikzpicture}
\end{center}

\begin{center}
  \begin{tikzpicture}[x=2.2cm,y=1.4cm]
    \message{^^JNeural network with uniform arrows}
    \readlist\Nnod{3,3}

    \foreachitem \N \in \Nnod{ % loop over layers
      \def\lay{\Ncnt} % alias of index of current layer
      \pgfmathsetmacro\prev{int(\Ncnt-1)} % previous layer index
      \message{^^J Layer \lay, N=\N, prev=\prev ->}

      % NODES
      \foreach \i [evaluate={\y=\N/2-\i; \x=\lay; \n=\nstyle;}] in {1,...,\N}{ % nodes
          \message{N\lay-\i, }
          \ifnum \lay=1
            \node[node \n] (N\lay-\i) at (\x,\y) {$x_\i$};
          \else
            \node[node \n] (N\lay-\i) at (\x,\y) {$y_\i$};
          \fi
        }

      % CONNECTIONS
      \foreach \i in {1,...,\N}{ % nodes in current layer
          \ifnum\lay>1
          \setAngles{N\prev-1}{N\lay-\i}{N\prev-\Nnod[\prev]} % for receiving node

          \ifnum\Nnod[\prev]=1
          \foreach \j in {1}{
              \pgfmathsetmacro\ang{\angmin}
              \setAngles{N\lay-1}{N\prev-\j}{N\lay-\N}
              \ifnum\N=1
                \pgfmathsetmacro\angout{\angmin}
              \else
                \pgfmathsetmacro\angout{\angmin+(\dang-360)*(\i-1)/(\N-1)}
              \fi
              \draw[connect arrow] (N\prev-\j.{\angout}) -- (N\lay-\i.{\ang});
            }
          \else
          \foreach \j [evaluate={\ang=\angmin+\dang*(\j-1)/(\Nnod[\prev]-1);}] in {1,...,\Nnod[\prev]}{
              \setAngles{N\lay-1}{N\prev-\j}{N\lay-\N}
              \ifnum\N=1
                \pgfmathsetmacro\angout{\angmin}
              \else
                \pgfmathsetmacro\angout{\angmin+(\dang-360)*(\i-1)/(\N-1)}
              \fi

              % ref: https://tex.stackexchange.com/questions/745061/how-to-use-a-variable-in-hsl-model-color-definition-for-a-color-wheel-with-xco
              \definecolor{arrowcolor}{Hsb}{\inteval{180+\i*36},1,0.8} % color of arrow

              \ifnum\i=2
                \draw[connect arrow,arrowcolor] (N\prev-\j.{\angout}) -- (N\lay-\i.{\ang}) node [midway,above,arrowcolor] {$w_{2\j}$};
              \else
                \draw[connect arrow,arrowcolor!40] (N\prev-\j.{\angout}) -- (N\lay-\i.{\ang});
              \fi
            }
          \fi
          \fi
        }
    }
  \end{tikzpicture}
\end{center}

\begin{center}
  \begin{tikzpicture}[x=2.2cm,y=1.4cm]
    \message{^^JNeural network with uniform arrows}
    \readlist\Nnod{3,3}

    \foreachitem \N \in \Nnod{ % loop over layers
      \def\lay{\Ncnt} % alias of index of current layer
      \pgfmathsetmacro\prev{int(\Ncnt-1)} % previous layer index
      \message{^^J Layer \lay, N=\N, prev=\prev ->}

      % NODES
      \foreach \i [evaluate={\y=\N/2-\i; \x=\lay; \n=\nstyle;}] in {1,...,\N}{ % nodes
          \message{N\lay-\i, }
          \ifnum \lay=1
            \node[node \n] (N\lay-\i) at (\x,\y) {$x_\i$};
          \else
            \node[node \n] (N\lay-\i) at (\x,\y) {$y_\i$};
          \fi
        }

      % CONNECTIONS
      \foreach \i in {1,...,\N}{ % nodes in current layer
          \ifnum\lay>1
          \setAngles{N\prev-1}{N\lay-\i}{N\prev-\Nnod[\prev]} % for receiving node

          \ifnum\Nnod[\prev]=1
          \foreach \j in {1}{
              \pgfmathsetmacro\ang{\angmin}
              \setAngles{N\lay-1}{N\prev-\j}{N\lay-\N}
              \ifnum\N=1
                \pgfmathsetmacro\angout{\angmin}
              \else
                \pgfmathsetmacro\angout{\angmin+(\dang-360)*(\i-1)/(\N-1)}
              \fi
              \draw[connect arrow] (N\prev-\j.{\angout}) -- (N\lay-\i.{\ang});
            }
          \else
          \foreach \j [evaluate={\ang=\angmin+\dang*(\j-1)/(\Nnod[\prev]-1);}] in {1,...,\Nnod[\prev]}{
              \setAngles{N\lay-1}{N\prev-\j}{N\lay-\N}
              \ifnum\N=1
                \pgfmathsetmacro\angout{\angmin}
              \else
                \pgfmathsetmacro\angout{\angmin+(\dang-360)*(\i-1)/(\N-1)}
              \fi

              % ref: https://tex.stackexchange.com/questions/745061/how-to-use-a-variable-in-hsl-model-color-definition-for-a-color-wheel-with-xco
              \definecolor{arrowcolor}{Hsb}{\inteval{180+\i*36},1,0.8} % color of arrow

              \ifnum\i=3
                \draw[connect arrow,arrowcolor] (N\prev-\j.{\angout}) -- (N\lay-\i.{\ang}) node [midway,above,arrowcolor] {$w_{3\j}$};
              \else
                \draw[connect arrow,arrowcolor!40] (N\prev-\j.{\angout}) -- (N\lay-\i.{\ang});
              \fi
            }
          \fi
          \fi
        }
    }
  \end{tikzpicture}
\end{center}

\begin{align*}
  \begin{pmatrix}
    y_1 \\
    y_2 \\
    y_3
  \end{pmatrix} & = \begin{pmatrix}
                      w_{11} x_1 + w_{12} x_2 + w_{13} x_3 \\
                      w_{21} x_1 + w_{22} x_2 + w_{23} x_3 \\
                      w_{31} x_1 + w_{32} x_2 + w_{33} x_3
                    \end{pmatrix} \\
                  & = \begin{pmatrix}
                        w_{11} & w_{12} & w_{13} \\
                        w_{21} & w_{22} & w_{23} \\
                        w_{31} & w_{32} & w_{33}
                      \end{pmatrix}
  \begin{pmatrix}
    x_1 \\
    x_2 \\
    x_3
  \end{pmatrix}                                         \\
\end{align*}

\end{document}
