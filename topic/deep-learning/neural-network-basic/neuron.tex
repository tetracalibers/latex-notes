\documentclass[../../../topic_deep-learning]{subfiles}

\begin{document}

\sectionline
\section{ニューロン}
\marginnote{\refbookA p48〜}

\keyword{ニューロン}は動物の脳の基本ユニットである

脳はほとんどがニューロンで作られており、
\begin{shaded}
  電気信号がニューロンから別のニューロンに渡される
\end{shaded}
ことの連鎖で、たとえば光、音、熱などを感知することができる

\subsection{ニューロンのモデル化}

ニューロンはまず電気的な入力を受け取り、その後に別の電気信号を放出する

これは、
\begin{shaded}
  入力を受け取り、処理を行い、何かを出力する
\end{shaded}
という\keyword{関数}としてモデル化できる

\sectionline

観察によれば、ニューロンは入力にすぐには反応せずに、入力が大きくなって出力を引き起こすまで、入力を抑制する

この現象を、科学者たちは
\begin{shaded}
  入力が\keyword{閾値}に達するとニューロンが\keyword{発火}する
\end{shaded}
と表現する

直観的には、ニューロンは小さなノイズ信号は通過させず、強く意図的な信号のみを通過させる

\br

入力信号を受け取り、出力信号を生成するが、何らかの閾値を考慮した関数を\keyword{活性化関数}という

単純な\keyword{階段関数}も活性化関数だが、境界が滑らかな\keyword{シグモイド関数}などの方がよく使われる



\end{document}
