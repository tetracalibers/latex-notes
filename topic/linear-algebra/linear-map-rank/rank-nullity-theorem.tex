\documentclass[../../../topic_linear-algebra]{subfiles}

\begin{document}

\sectionline
\section{次元定理}
\marginnote{\refbookA p101 \\ \refbookF p82〜83}

連立方程式$A\vb*{x}=\vb*{b}$の\hyperref[sec:degrees-of-freedom]{解の自由度}は、
\begin{equation*}
  \text{解の自由度} = (\text{変数の個数}) - \rank(A)
\end{equation*}
で表された

この関係は、$\vb*{b} =\vb*{0}$、すなわち斉次形の場合にも成り立つ

\br

そこで、変数の個数を$n$とおくと、次のようにも書き換えられる
\begin{equation*}
  \rank(A) = n - (A\vb*{x} =\vb*{0}\text{の解の自由度})
\end{equation*}

\br

線型方程式と階数に関するこの関係を、線形写像と次元の言葉で言い換えたい

\br

次のような線形写像
\begin{equation*}
  \begin{array}{llll}
    f\colon & \mathbb{R}^n          & \longrightarrow & \mathbb{R}^m          \\
            & \rotatebox{90}{$\in$} &                 & \rotatebox{90}{$\in$} \\
            & \vb*{x}               & \longmapsto     & A\vb*{x}
  \end{array}
\end{equation*}
を考えると、
\begin{itemize}
  \item 写像$f$は、行列$A$に対応する
  \item 変数の個数は、$\vb*{x}$の動く空間$\mathbb{R}^n$の次元$n$に対応する
  \item $A\vb*{x} = \vb*{0}$の解の自由度は、写像$f$で$\vb*{0}$になってしまうものの次元に対応する
\end{itemize}
という関係が読み取れる

\br

ここで、写像$f$で$\vb*{0}$になってしまう\keyword{縮退}するものは、写像$f$の\keyword{核}$\Ker(f)$である

このことを用いて関係式を表現し直すと、次のようになる
\begin{equation*}
  \rank(f) = n - \dim \Ker(f)
\end{equation*}

\begin{theorem}{線形写像の次元定理}\label{thm:rank-nullity-theorem}
  $f\colon \mathbb{R}^n \to \mathbb{R}^m$を線形写像とすると、次が成り立つ
  \begin{equation*}
    \rank(f) = n - \dim \Ker(f)
  \end{equation*}
\end{theorem}

\begin{proof}
  $A$を$f$の表現行列とし、$\rank(f) = r$とする

  このとき、$\Ker(f)$の次元は$A\vb*{x} = \vb*{0}$の解空間の自由度$n - r$と一致するため、
  \begin{align*}
    \dim \Ker(f)              & = n - r            \\
                              & = n - \rank(f)     \\
    \therefore \quad \rank(f) & = n - \dim \Ker(f)
  \end{align*}
  となり、定理が成り立つ $\qed$
\end{proof}

\end{document}
