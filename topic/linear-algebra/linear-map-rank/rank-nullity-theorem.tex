\documentclass[../../../topic_linear-algebra]{subfiles}

\begin{document}

\sectionline
\section{次元定理}
\marginnote{\refbookA p101 \\ \refbookF p82〜83}

連立方程式$A\vb*{x}=\vb*{b}$の\hyperref[sec:degrees-of-freedom]{解の自由度}は、
\begin{equation*}
  \text{\bfseries 解の自由度} = (\text{\bfseries 変数の個数}) - \rank A
\end{equation*}
で表された。

そして、この\hyperref[sec:nullity-solution-degrees]{解の自由度は$\Ker A$の次元と一致}する。
\begin{equation*}
  \dim \Ker A = (\text{\bfseries 変数の個数}) - \rank A
\end{equation*}

ここで、次のような線形写像
\begin{equation*}
  \begin{array}{llll}
    f\colon & \mathbb{R}^n          & \longrightarrow & \mathbb{R}^m          \\
            & \rotatebox{90}{$\in$} &                 & \rotatebox{90}{$\in$} \\
            & \vb*{x}               & \longmapsto     & A\vb*{x}
  \end{array}
\end{equation*}
を考えると、次のように対応づけられる。
\begin{itemize}
  \item 行列$A$は、線形写像$f$を表す
  \item 変数の個数は、$\vb*{x}$の動く空間$\mathbb{R}^n$の次元$n$に対応する
\end{itemize}

このように行列$A$を線形写像$f$に対応させると、
\begin{equation*}
  \dim \Ker f = n - \rank f
\end{equation*}

さらに、\hyperref[def:rank-of-linear-map]{線形写像の階数の定義}から、
\begin{equation*}
  \dim \Ker f = n - \dim \Im f
\end{equation*}
という、次元に関する関係式が得られる。これを\keyword{次元定理}という。

\begin{theorem}{線形写像の次元定理}\label{thm:rank-nullity-theorem}
  $f\colon \mathbb{R}^n \to \mathbb{R}^m$を線形写像とすると、次が成り立つ。
  \begin{equation*}
    \rank(f) = \dim \Im f = n - \dim \Ker(f)
  \end{equation*}
\end{theorem}

\subsection{写像の視点}

$f$を$n$次元線型空間から$m$次元線型空間への線形写像とすると、
\begin{equation*}
  n - \dim \Ker f = \dim \Im f
\end{equation*}
という次元定理の式は、次のように読める。

\begin{emphabox}
  \begin{spacebox}
    \begin{center}
      元の$n$次元空間から、$\Ker f$の次元分が潰れて、\\
      残ったのが$\Im f$の次元分
    \end{center}
  \end{spacebox}
\end{emphabox}

\begin{center}
  \begin{tikzpicture}[scale=1.4]
    \coordinate (Vtop) at (0,4);
    \coordinate (Vbottom) at (0,0);
    \coordinate (Wtop) at (3,4);
    \coordinate (Wbottom) at (3,0);
    \coordinate (Ker-top) at (0,1.5);
    \coordinate (Im-top) at (3,2.5);

    % Nodes
    \node[above] at (Vtop) {\large $K^n$};
    \node[above] at (Wtop) {\large $K^m$};

    % Dotted arrows
    \draw[densely dashed, Rhodamine] (Vtop) -- (Im-top);
    \draw[densely dashed, Rhodamine] (Ker-top) -- (Wbottom);
    \draw[densely dashed, Cerulean] (Vbottom) -- (Wbottom);

    \draw[line width=2pt, SkyBlue] (Ker-top) -- (Vbottom);
    \draw[line width=2pt, lightslategray] (Wtop) -- (Im-top);

    \draw[line width=4pt, line cap=round, carnationpink] (Vtop) -- (Ker-top);
    \draw[line width=4pt, line cap=round, carnationpink] (Im-top) -- (Wbottom);

    % points
    \fill[SkyBlue] (Wbottom) circle (2pt);

    \draw[Cerulean, <->, Straight Barb-Straight Barb, very thick] ([xshift=-1.5ex]Ker-top) -- ([xshift=-1.5ex]Vbottom) node[midway,left, align=center] {$\dim\Ker f$};
    \draw[Rhodamine, <->, Straight Barb-Straight Barb, very thick] ([xshift=-1.5ex]Vtop) -- ([xshift=-1.5ex]Ker-top) node[midway,left, align=center] {$n - \dim\Ker f$};
    \draw[Rhodamine, <->, Straight Barb-Straight Barb, very thick, shorten <= 0.25ex] ([xshift=1.5ex]Im-top) -- ([xshift=1.5ex]Wbottom) node[midway,right=0.5ex, align=center] {$\dim\Im f$};
  \end{tikzpicture}
\end{center}

\sectionline
\section{次元定理と単射性・全射性}

$m \times n$型行列$A$によって表現される線形写像$f$が単射・全射かどうかは、$m$と$n$の大小によって決まる。

\subsection{$A$が横長($m < n$)の場合}

このとき、$A$が表現する線形写像$f$は単射ではない。

\begin{enumerate}
  \item $\Im A$は行き先の$m$次元空間の一部なので、$\dim \Im A \leq m$
  \item $m < n$より、$\dim \Im A < n$
  \item 次元定理より、$\dim \Ker A > 0$
\end{enumerate}

\hyperref[sec:injectivity-condition-kernel]{潰れる部分$\Ker A$が$\{ \vb*{o} \}$でない}ことは、単射ではないことを意味する。

\begin{emphabox}
  \begin{spacebox}
    \begin{center}
      表現行列$A$が横長($m < n$)だと単射にはなれない
    \end{center}
  \end{spacebox}
\end{emphabox}

\subsection{$A$が縦長($m > n$)の場合}

このとき、$A$が表現する線形写像$f$は全射ではない。

\begin{enumerate}
  \item 次元は$0$以上なので、$\dim \Ker A \geq 0$
  \item 次元定理より、$\dim \Im A \leq n$
  \item $m > n$より、$\dim \Im A < m$
\end{enumerate}

\hyperref[sec:image-and-surjectivity]{$\Im A$が写り先の空間全体をカバーしていない}ことは、全射ではないことを意味する。

\begin{emphabox}
  \begin{spacebox}
    \begin{center}
      表現行列$A$が縦長($m > n$)だと全射にはなれない
    \end{center}
  \end{spacebox}
\end{emphabox}

\end{document}
