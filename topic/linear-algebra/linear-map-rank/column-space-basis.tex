\documentclass[../../../topic_linear-algebra]{subfiles}

\begin{document}

\sectionline
\section{線形写像の像空間の基底}
\marginnote{\refbookA p96〜97}

線形写像の像空間は表現行列の列ベクトルによって張られるが、列ベクトルの集合は一般には線型独立ではない。

像空間の基底を得るためには、列ベクトルの部分集合を考えるのが自然である。

\begin{theorem}{主列ベクトルによる像空間の基底の構成}\label{thm:pivot-cols-form-basis}
  行列$A$の\hyperref[def:pivot-columns]{主列ベクトル}の集合は$\Im A$の基底である。
\end{theorem}

\begin{proof}
  \todo{\refbookA p97 定理3.1.10}
\end{proof}

\end{document}
