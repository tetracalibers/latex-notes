\documentclass[../../../topic_linear-algebra]{subfiles}

\begin{document}

\sectionline
\section{線形写像の像と列空間}\label{sec:image-and-column-space}
\marginnote{\refbookA p96〜97、\refbookL p135}

ベクトル$\vb*{a}_1, \ldots, \vb*{a}_n$の張る空間の記号を用いると、\hyperref[sec:span-of-vectors]{ベクトルの張る空間と$\Im A$に関する考察}は次のようにまとめられる。
\begin{equation*}
  \Im A = \langle \vb*{a}_1, \ldots, \vb*{a}_n \rangle
\end{equation*}

つまり、$A$の列ベクトルが張る空間が$\Im A$である。

このことから、$\Im A$を$A$の\keyword{列空間}と呼ぶこともある。

\begin{theorem*}{線形写像の像と表現行列の列空間の一致}
  線形写像$f\colon \mathbb{R}^n \to \mathbb{R}^m$の像$\Im f$は、$f$の表現行列の列ベクトルが張る空間である。
\end{theorem*}

\begin{proof}
  線形写像$f\colon \mathbb{R}^n \to \mathbb{R}^m$の表現行列を$A = (\vb*{a}_1, \vb*{a}_2, \dots, \vb*{a}_n)$とするとき、$\vb*{v} \in \mathbb{R}^n$に対して、
  \begin{equation*}
    f(\vb*{v}) = A \vb*{v} = v_1 \vb*{a}_1 + v_2 \vb*{a}_2 + \cdots + v_n \vb*{a}_n
  \end{equation*}
  なので、
  \begin{align*}
     & \phantom{\longleftrightarrow\,\,\,} \vb*{u} \in \Im f                                                                 \\
     & \Longleftrightarrow \exists \vb*{v} \in \mathbb{R}^n \suchthat \vb*{u} = f(\vb*{v})                                   \\
     & \Longleftrightarrow \exists v_1, \dots, v_n \in \mathbb{R} \suchthat \vb*{u} = v_1 \vb*{a}_1 + \cdots + v_n \vb*{a}_n \\
     & \Longleftrightarrow \vb*{u} \in \langle \vb*{a}_1, \vb*{a}_2, \dots, \vb*{a}_n \rangle
  \end{align*}
  したがって、
  \begin{equation*}
    \Im f = \Im A = \langle \vb*{a}_1, \vb*{a}_2, \dots, \vb*{a}_n \rangle
  \end{equation*}
  が成り立つ。 $\qed$
\end{proof}

上述の証明の
\begin{equation*}
  \vb*{u} \in \Im f                                                               \Longleftrightarrow \exists \vb*{v} \in \mathbb{R}^n \suchthat \vb*{u} = f(\vb*{v})
\end{equation*}

という変形に着目すると、この定理は次のように線型方程式の文脈で言い換えられる。

\begin{theorem*}{線形写像の像空間と方程式の解の存在}
  $\vb*{b} \in \mathbb{R}^m$に対して
  \begin{equation*}
    \vb*{b} \in \Im A \Longleftrightarrow \text{方程式} A \vb*{x} = \vb*{b} \text{が解を持つ}
  \end{equation*}
\end{theorem*}

$\vb*{b} \in \mathbb{R}^m$が$\Im A$に属するかどうかを調べるためには\hyperref[thm:augmented-rank-solution-condition]{階数による判定条件}が使える。

\end{document}
