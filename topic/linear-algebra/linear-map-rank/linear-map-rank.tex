\documentclass[../../../topic_linear-algebra]{subfiles}

\usepackage{xr-hyper}
\externaldocument{../../../.tex_intermediates/topic_linear-algebra}

\begin{document}

\sectionline
\section{線形写像の階数}
\marginnote{\refbookA p100}

次の定理は、行列の階数のさらに本質的な意味を明らかにし、行列の階数が行変形の仕方によらずに決まることを念押しするような定理である。

\begin{theorem*}{行列の階数と像空間の次元の一致}
  行列の階数は像空間の次元である。

  すなわち、$A$を$m \times n$型行列とするとき、
  \begin{equation*}
    \rank A = \dim \Im A
  \end{equation*}
\end{theorem*}

\begin{proof}
  \thmref{thm:pivot-cols-form-basis}で示したように、$A$の主列ベクトル$\vb*{a}_{i_1}, \vb*{a}_{i_2}, \dots, \vb*{a}_{i_r}$は$\Im A$の基底を成す。

  よってその個数$r = \rank A$は$\Im A$の次元である。 $\qed$
\end{proof}

\br

この定理から、線形写像に対して、像空間の次元をその階数と定める。

\begin{definition}{線形写像の階数}\label{def:rank-of-linear-map}
  $f\colon \mathbb{R}^n \to \mathbb{R}^m$を線形写像とするとき、$f$の\keyword{階数}を
  \begin{equation*}
    \rank f = \dim \Im f
  \end{equation*}
  と定義する。
\end{definition}

\end{document}
