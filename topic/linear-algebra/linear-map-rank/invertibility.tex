\documentclass[../../../topic_linear-algebra]{subfiles}

\begin{document}

\sectionline
\section{核空間の次元による正則の判定}

次元定理から、次のような正則判定法が得られる。

\begin{theorem}{核空間の次元による正則判定}\label{thm:invertibility-by-kernel}
  $n$次正方行列$A$に対して、
  \begin{align*}
    A \text{が正則行列} &\Longleftrightarrow \Ker A = \{ \vb*{o} \} \\
    &\Longleftrightarrow \dim \Ker A = 0
  \end{align*}
\end{theorem}

\begin{proof}
  \hyperref[thm:invertible-iff-full-rank]{階数による正則判定}より、$A$が正則であることは、
  \begin{equation*}
    \rank A = n
  \end{equation*}
  であることと同値である。
  
  ここで、次元定理より、
  \begin{equation*}
    \rank A + \dim \Ker A = n
  \end{equation*}
  $\rank A =n$を代入し、整理すると、
  \begin{equation*}
    \dim \Ker A = 0
  \end{equation*}
  が得られる。 $\qed$
\end{proof}

\end{document}
