\documentclass[../../../topic_linear-algebra]{subfiles}

\begin{document}

\sectionline
\section{正則の判定}

ここまでの議論により、さまざまな正則判定法が得られる。

\subsection{階数による正則判定}
\marginnote{\refbookA p71}

\hyperref[thm:linear-pigeonhole]{線形代数における鳩の巣原理}から、次のことがいえる。

\begin{theorem}{階数による正則の判定}\label{thm:invertible-iff-full-rank}
  $n$次正方行列$A$に対して、
  \begin{equation*}
    A \text{が正則行列} \Longleftrightarrow \rank A = n
  \end{equation*}
\end{theorem}

この定理は、線形変換$f$(もしくは正方行列$A$)が\keyword{正則}かどうかについて、\keyword{階数}という1つの数値で判定できることを示している。

\subsection{列ベクトルの線型独立性による正則の判定}
\marginnote{\refbookA p71}

\begin{theorem}{列ベクトルの線型独立性による正則の判定}\label{thm:invertible-iff-col-indep}
  $\vb*{a}_1, \dots, \vb*{a}_n$を列ベクトルとする$n$次正方行列$A$に対して、次が成り立つ。
  \begin{equation*}
    A \text{が正則行列} \Longleftrightarrow \vb*{a}_1, \dots, \vb*{a}_n \text{が線型独立}
  \end{equation*}
\end{theorem}

\begin{proof}
  $\vb*{a}_1, \dots, \vb*{a}_n \in \mathbb{R}^n$が線型独立であることは、\hyperref[thm:lin-indep-iff-rank-n]{$\rank A = n$と同値}である。
  
  \hyperref[thm:invertible-iff-full-rank]{階数による正則判定}より、$\rank A =n$は$A$が正則行列であることと同値である $\qed$
\end{proof}

\subsection{核空間の次元による正則の判定}

次元定理から、次のような正則判定法が得られる。

\begin{theorem}{核空間の次元による正則判定}\label{thm:invertibility-by-kernel}
  $n$次正方行列$A$に対して、
  \begin{align*}
    A \text{が正則行列} &\Longleftrightarrow \Ker A = \{ \vb*{o} \} \\
    &\Longleftrightarrow \dim \Ker A = 0
  \end{align*}
\end{theorem}

\begin{proof}
  \hyperref[thm:invertible-iff-full-rank]{階数による正則判定}より、$A$が正則であることは、
  \begin{equation*}
    \rank A = n
  \end{equation*}
  であることと同値である。
  
  ここで、次元定理より、
  \begin{equation*}
    \rank A + \dim \Ker A = n
  \end{equation*}
  $\rank A =n$を代入し、整理すると、
  \begin{equation*}
    \dim \Ker A = 0
  \end{equation*}
  が得られる。 $\qed$
\end{proof}

\end{document}
