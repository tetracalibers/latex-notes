\documentclass[../../../topic_linear-algebra]{subfiles}

\begin{document}

\sectionline
\section{ランク1行列による圧縮}

1つのグレースケール画像を行列$A \in \mathbb{R}^{m \times n}$に見立てると、画像のファイルサイズは$mn$に比例する。
単純に行列の成分を保存しようとすると、膨大な記憶容量が必要となる。

\br

そこで、圧縮に向けた一つの考え方として、$m \times n$型行列$A$を、
\begin{emphabox}
  \begin{spacebox}
    \begin{center}
      いくつかの「縦ベクトルと横ベクトルの積」の和として近似的に表現する
    \end{center}
  \end{spacebox}
\end{emphabox}
ことを考える。

\subsection{ランク1行列}

ここで重要なのは次の事実である。
\begin{emphabox}
  \begin{spacebox}
    \begin{center}
      「縦ベクトルと横ベクトルの積」は階数が1の行列
    \end{center}
  \end{spacebox}
\end{emphabox}

たとえば、
\begin{equation*}
  \vb*{u} = \begin{pmatrix}
    1 \\
    2 \\
    3
  \end{pmatrix}, \quad
  \vb*{v}^\top = \begin{pmatrix}
    4 & 5
  \end{pmatrix}
\end{equation*}
とすると、これらの積は次のように計算される。
\begin{equation*}
  \vb*{u} \vb*{v}^\top = \begin{pmatrix}
    1 \cdot 4 & 1 \cdot 5 \\
    2 \cdot 4 & 2 \cdot 5 \\
    3 \cdot 4 & 3 \cdot 5
  \end{pmatrix} = \begin{pmatrix}
    4 & 5 \\
    8 & 10 \\
    12 & 15
  \end{pmatrix}
\end{equation*}
各行は、最初の行$(4,5)$のスカラー倍となっていることに注目しよう。

つまり、独立な行は$(4,5)$だけであり、他の行はこの行の線形結合で表現できる。

独立な行が1つしかないので、この行列の階数は1である。

\br

このことは、一般的な成分表示で考えることもできる。
\begin{equation*}
  \vb*{u} = \begin{pmatrix}
    u_1 \\
    \vdots \\
    u_m
  \end{pmatrix}, \quad
  \vb*{v}^\top = \begin{pmatrix}
    v_1 & \cdots & v_n
  \end{pmatrix}
\end{equation*}
とおくと、これらの積は次のように計算される。
\begin{equation*}
  \vb*{u} \vb*{v}^\top = \begin{pmatrix}
    u_1 v_1 & u_1 v_2 & \cdots & u_1 v_n \\
    u_2 v_1 & u_2 v_2 & \cdots & u_2 v_n \\
    \vdots  & \vdots  & \ddots & \vdots  \\
    u_m v_1 & u_m v_2 & \cdots & u_m v_n
  \end{pmatrix}
\end{equation*}

ここで、$i$行目を取り出すと、
\begin{equation*}
  u_i \cdot \begin{pmatrix}
    v_1 & \cdots & v_n
  \end{pmatrix} = u_i \vb*{v}^\top
\end{equation*}
となっているので、すべての行は$\vb*{v}^\top$のスカラー倍で表現できることがわかる。

\subsection{低ランク近似による圧縮}

縦ベクトルと横ベクトルの積は階数1の行列となることから、もし$k$個の和で近似するならば、近似した行列の階数は$k$以下となる。

もし、$k$個の和で良い近似になっているのであれば、
\begin{itemize}
  \item $m$次元ベクトル(縦ベクトル)を$k$本
  \item $n$次元ベクトル(横ベクトル)を$k$本
\end{itemize}
保持すればよいことになる。

\br

これで、$mn$個の成分を保持する必要はなくなり、$k(m+n)$個の成分を保持すれば十分となる。

\br

たとえば、$m=1000, n=1000$の行列を$k=50$個の和でうまく近似できるとしたら、
\begin{equation*}
  \frac{k(m+n)}{mn} = \frac{50(1000+1000)}{1000 \times 1000} = 0.1
\end{equation*}
より、ファイルサイズを$90\%$削減できることがわかる。

\br

このようなアイデアを、工学では\keyword{圧縮}といい、数学的には行列の\keyword{低ランク近似}という。

\sectionline
\section{特異値分解による低ランク近似}

さて、$O$でない任意の行列$A$は、次のように特異値分解できた。
\begin{equation*}
  A = \sigma_1 \vb*{u}_1 \vb*{v}_1^\top + \cdots + \sigma_k \vb*{u}_k \vb*{v}_k^\top + \cdots + \sigma_r \vb*{u}_r \vb*{v}_r^\top
\end{equation*}
これは、「縦ベクトルと横ベクトルの積」の和の形になっている。

\br

さらに、特異値$\sigma_i$は大きい順に並んでいるので、ランク$k$の行列の中で、
\begin{equation*}
  A_k = \sigma_1 \vb*{u}_1 \vb*{v}_1^\top + \cdots + \sigma_k \vb*{u}_k \vb*{v}_k^\top
\end{equation*}
が$A$に最も近い近似となるのではないか?と予想できる。

\br

ここで、「$A$との近さ」を測るためには、行列に関する\keyword{ノルム}を考える必要がある。

\end{document}
