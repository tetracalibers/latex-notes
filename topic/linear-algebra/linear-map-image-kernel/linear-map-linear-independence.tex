\documentclass[../../../topic_linear-map-image-kernel]{subfiles}

\begin{document}

\sectionline
\section{線形写像とベクトルの線型独立性}
\marginnote{\refbookA p65〜}

\begin{theorem}{線形写像とベクトルの線形独立性}
  $f\colon \mathbb{R}^n \to \mathbb{R}^m$を線形写像、$\vb*{v}_1, \vb*{v}_2, \dots, \vb*{v}_n \in \mathbb{R}^n$とする
  \begin{enumerate}[label=\romanlabel]
    \item $\{ f(\vb*{v}_1), \dots, f(\vb*{v}_n) \}$が線型独立ならば、$\{ \vb*{v}_1, \dots, \vb*{v}_n \}$は線型独立
    \item $\{\vb*{v}_1, \dots, \vb*{v}_n\}$が線形従属ならば、$\{ f(\vb*{v}_1),  \dots, f(\vb*{v}_n) \}$は線形従属
  \end{enumerate}
\end{theorem}

\begin{proof}
  \todo{\refbookA p65 問2.11}
\end{proof}

\romannum{ii}は、平行なベクトルを線型写像で写した結果、平行でなくなったりはしないということを述べている

\sectionline

\end{document}
