\documentclass[../../../topic_linear-algebra]{subfiles}

\begin{document}

\sectionline
\section{外積}

3次元直交座標空間において、
\begin{itemize}
  \item $x$軸と$y$軸の両方に垂直なのは$z$軸
  \item $y$軸と$z$軸の両方に垂直なのは$x$軸
  \item $z$軸と$x$軸の両方に垂直なのは$y$軸
\end{itemize}
という関係がある。

\br

そこで、$x$軸方向の単位ベクトルを$\vb*{e}_x$, $y$軸方向の単位ベクトルを$\vb*{e}_y$, $z$軸方向の単位ベクトルを$\vb*{e}_z$とすると、
\begin{align*}
  \vb*{e}_x \times \vb*{e}_y &= \vb*{e}_z \\
  \vb*{e}_y \times \vb*{e}_z &= \vb*{e}_x \\
  \vb*{e}_z \times \vb*{e}_x &= \vb*{e}_y
\end{align*}
これが正規直交ベクトル同士の外積である。

\begin{equation*}
  \vb*{e}_x = \begin{pmatrix}
    1 \\
    0 \\
    0
  \end{pmatrix} , \quad
  \vb*{e}_y = \begin{pmatrix}
    0 \\
    1 \\
    0
  \end{pmatrix} , \quad
  \vb*{e}_z = \begin{pmatrix}
    0 \\
    0 \\
    1
  \end{pmatrix}
\end{equation*}

ここで、たとえば、
\begin{equation*}
  \vb*{e}_y \times \vb*{e}_x = \begin{pmatrix}
    0 \cdot 0 - 1 \cdot 0 \\
    1 \cdot 0 - 0 \cdot 0 \\
    0 \cdot 0 - 0 \cdot 1
  \end{pmatrix} = \begin{pmatrix}
    0 \\
    0 \\
    -1
  \end{pmatrix} = -\vb*{e}_z = - (\vb*{e}_x \times \vb*{e}_y)
\end{equation*}

\br

また、$x$軸と$x$軸の両方に垂直なベクトルは、零ベクトルしかない。
\begin{equation*}
  \vb*{e}_x \times \vb*{e}_x = \begin{pmatrix}
    0 \cdot 0 - 0 \cdot 0 \\
    0 \cdot 0 - 0 \cdot 0 \\
    0 \cdot 0 - 0 \cdot 0
  \end{pmatrix} = \vb*{o}
\end{equation*}
このように、平行なベクトル同士の外積は零ベクトルになる。

\br

正規直交ベクトルどうしの内積は、次のように表された。
\begin{equation*}
  \vb*{e}_i \cdot \vb*{e}_j = \delta_{ij} = \begin{cases}
    1 & (i = j) \\
    0 & (i \neq j)
  \end{cases}
\end{equation*}

一方、正規直交ベクトルに対する外積は、次の性質を持つ。
\begin{equation*}
  \vb*{e}_i \times \vb*{e}_j = \begin{cases}
    0 & (i = j) \\
    - \vb*{e}_j \times \vb*{e}_i & (i \neq j)
  \end{cases}
\end{equation*}

\end{document}
