\documentclass[../../../topic_linear-algebra]{subfiles}

\usepackage{xr-hyper}
\externaldocument{../../../.tex_intermediates/topic_linear-algebra}

\begin{document}

\sectionline
\section{線形写像}
\marginnote{
  \refweb{線形代数の基礎のキソ}{https://www1.econ.hit-u.ac.jp/kawahira/courses/kiso/01-senkei.pdf}
}

$U$と$V$をベクトル空間とする。

写像$f \colon U \to V$が与えられたとき、これは$U$の出来事、構造、その他もろもろの情報を$V$に投影していると考えられる。

\br

このとき、その「写り方」にはどのような性質を期待するべきであろうか?

もっとも素朴な期待は、$U$はベクトル空間なのだから、写った先$f(U)$でもベクトル空間としての代数的構造(ベクトルどうしの和・定数倍に関する関係式)が保存されるという状況である。

\begin{mindflow}
  \placeholder{線形写像の定義}
\end{mindflow}

たとえば、和の保存
\begin{equation*}
  f(\vb*{u}_1 + \vb*{u}_2) = f(\vb*{u}_1) + f(\vb*{u}_2)
\end{equation*}
は、次のように解釈できる。

\begin{enumerate}
  \item $U$の側で、$\vb*{u}_1$と$\vb*{u}_2$が足され、$\vb*{u}_1 + \vb*{u}_2$が得られた。
  \item この現象を$f$というレンズを通して$V$に写すと、$f(\vb*{u}_1)$と$f(\vb*{u}_2)$が足され、$f(\vb*{u}_1) + f(\vb*{u}_2)$が得られたように見える。
\end{enumerate}

\subsection{比例関数の一般化}

線形写像の特徴づけとして、「比例関数の一般化」という考え方もできる。

もっとも簡単な関数である比例関数が満たすべき性質を、高次元なりに抽象化し、実現しているのが線形写像だとも考えられる。

\subsection{線形写像と微分}

線形写像は「局所的には」ありふれている。

あらゆる微分可能な関数は、あらゆる場所で「線形写像+誤差」と局所的に表現される。

局所的に線形写像として近似するのが微分ともいえる。

\end{document}
