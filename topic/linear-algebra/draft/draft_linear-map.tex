\documentclass[../../../topic_linear-algebra]{subfiles}

\usepackage{xr-hyper}
\externaldocument{../../../.tex_intermediates/topic_linear-algebra}

\begin{document}

\sectionline
\section{線形写像}
\marginnote{
  \refweb{線形代数の基礎のキソ}{https://www1.econ.hit-u.ac.jp/kawahira/courses/kiso/01-senkei.pdf}
}

たとえば、和の保存
\begin{equation*}
  f(\vb*{u}_1 + \vb*{u}_2) = f(\vb*{u}_1) + f(\vb*{u}_2)
\end{equation*}
は、次のように解釈できる。

\begin{enumerate}
  \item $U$の側で、$\vb*{u}_1$と$\vb*{u}_2$が足され、$\vb*{u}_1 + \vb*{u}_2$が得られた。
  \item この現象を$f$というレンズを通して$V$に写すと、$f(\vb*{u}_1)$と$f(\vb*{u}_2)$が足され、$f(\vb*{u}_1) + f(\vb*{u}_2)$が得られたように見える。
\end{enumerate}

\subsection{線形変換}

特に、\keyword{線形変換}は空間$\mathbb{R}^n$からそれ自身への写像なので、$\mathbb{R}^n$内において「ベクトルが変化している」(あるいは$f$が空間$\mathbb{R}^n$に\keyword{作用}している)ニュアンスとみることができる

\end{document}
