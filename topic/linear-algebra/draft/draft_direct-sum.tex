\documentclass[../../../topic_linear-algebra]{subfiles}

\begin{document}

\sectionline
\section{積集合としての直和}
\marginnote{\refbookS p12〜13}

$V, W$が線型空間のとき、集合としての\hyperref[def:cartesian-product]{直積}
\begin{equation*}
  V \times W = \{ (x, y) \mid x \in V, y \in W \}
\end{equation*}
は、成分ごとの加法とスカラー倍
\begin{align*}
  (x_1, y_1) + (x_2, y_2) &= (x_1 + x_2, y_1 + y_2) \\
  c (x, y) &= (c x, c y)
\end{align*}
により、線型空間となる。

これを$V$と$W$の\keyword{(抽象的な)直和}といい、$V \oplus W$と表す。

\br

より一般に、$V_1, \ldots, V_n$が線型空間のとき、
\begin{equation*}
  V_1 \times \cdots \times V_n = \{ x_1 \in V_1, \ldots, x_n \in V_n \}
\end{equation*}
は、成分ごとの加法とスカラー倍により、線型空間となる。

これを$V_1, \ldots, V_n$の\keyword{(抽象的な)直和}といい、$V_1 \oplus \cdots \oplus V_n$と表す。

\br

特に、$V_1 = \cdots = V_n = V$のとき、$V_1 \oplus \cdots \oplus V_n$を$V^{\oplus n}$あるいは$V^n$と表す。

\sectionline
\section{部分空間の和としての直和}
\marginnote{\refbookS p21〜22}

\begin{theorem}{todo}
  $W_1, W_2$を$V$の部分空間とするとき、次の条件は同値である。
  \begin{enumerate}[label=\romanlabel]
    \item $W_1 \cap W_2 = \{ \vb*{o} \}$
    \item $(x_1, x_2) \in W_1 \oplus W_2$を$x_1 + x_2 \in W_1 + W_2$に写す写像$W_1 \oplus W_2 \to W_1 + W_2$は同型である
  \end{enumerate}
\end{theorem}

\begin{proof}
  \todo{\refbookS p21〜22、p30}
\end{proof}

この定理より、$W_1 \cap W_2 = \{ \vb*{o} \}$のとき、
\begin{itemize}
  \item 積集合としての直和$W_1 \oplus W_2$(抽象的な直和)
  \item 部分集合の和空間$W_1 + W_2$(部分空間としての直和)
\end{itemize}
を同一視することができる。

\br

そこで、$W_1 \cap W_2 = \{ \vb*{o} \}$のとき、和空間$W_1 + W_2$を\keyword{(部分空間としての)直和}といい、$W_1 \oplus W_2$と表す。

\br

特に、$V = W_1 \oplus W_2$のとき、$V$は$W_1$と$W_2$の\keyword{直和に分解する}という。

このとき、$W_1$は$V$の\keywordJE{直和因子}{direct summand}であるという。

また、$W_2$を$W_1$の\keywordJE{補空間}{complementary subspace}という。

\begin{mindflow}
  \todo{一般の場合の定義(\refbookS p23)}
\end{mindflow}

\end{document}
