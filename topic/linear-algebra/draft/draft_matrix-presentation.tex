\documentclass[../../../topic_linear-algebra]{subfiles}

\usepackage{xr-hyper}
\externaldocument{../../../.tex_intermediates/topic_linear-algebra}

\begin{document}

\sectionline
\section{線形写像の行列表示}
\marginnote{
  \refbookS p56〜58 \\
  \refweb{【表現行列】線形写像の行列表示を詳しく}{https://mathlandscape.com/map-matrix/}
}

有限次元線形空間の元が、基底を使えばベクトルで表せたように、有限次元線形空間の線形写像は、基底を使えば行列で表すことができる。

\br

$V, W$をそれぞれ$m,n$次元線形空間とし、それらの基底を$\{ \vb*{v}_1, \ldots, \vb*{v}_m \}$、$\{ \vb*{w}_1, \ldots, \vb*{w}_n \}$と定める。

\br

ここで、$f\colon V \to W$を線形写像とすると、\thmref{thm:linear-map-determined-by-basis}より、次のように$f$を定めることができる。
\begin{align*}
  f(\vb*{v}_1) &= a_{11}\vb*{w}_1 + a_{12}\vb*{w}_2 + \cdots + a_{1n}\vb*{w}_n \\
  f(\vb*{v}_2) &= a_{21}\vb*{w}_1 + a_{22}\vb*{w}_2 + \cdots + a_{2n}\vb*{w}_n \\
  & \ldots \\
  f(\vb*{v}_m) &= a_{m1}\vb*{w}_1 + a_{m2}\vb*{w}_2 + \cdots + a_{mn}\vb*{w}_n
\end{align*}

上の$m$個の式をまとめて、次のように書くことができる。
\begin{align*}
  &\begin{pmatrix}
    f(\vb*{v}_1) & f(\vb*{v}_2) & \cdots & f(\vb*{v}_m)
  \end{pmatrix} \\
  &= \begin{pmatrix}
    \vb*{w}_1 & \vb*{w}_2 & \cdots & \vb*{w}_n
  \end{pmatrix} \begin{pmatrix} 
  a_{11} & a_{12} & \dots  & a_{1n} \\
  a_{21} & a_{22} & \dots  & a_{2n} \\
  \vdots & \vdots & \ddots & \vdots \\
  a_{m1} & a_{m2} & \dots  & a_{mn}
\end{pmatrix} 
\end{align*}

このときの行列$A = (a_{ij})$を、$f$の\keyword{表現行列}あるいは\keyword{行列表現}という。
\begin{equation*}
  \begin{pmatrix}
    f(\vb*{v}_1) & f(\vb*{v}_2) & \cdots & f(\vb*{v}_m)
  \end{pmatrix} = \begin{pmatrix}
    \vb*{w}_1 & \vb*{w}_2 & \cdots & \vb*{w}_n
  \end{pmatrix} A
\end{equation*}

\br

行列表示を使えば、線形空間と線形写像についての問題を、ベクトルと行列についての問題に帰着させて解くことができる。

\sectionline
\section{底の変換行列}
\marginnote{\refbookS p60〜61}

\begin{mindflow}
  \todo{}
\end{mindflow}

\end{document}
