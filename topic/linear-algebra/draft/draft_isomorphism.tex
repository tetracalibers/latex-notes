\documentclass[../../../topic_linear-algebra]{subfiles}

\usepackage{xr-hyper}
\externaldocument{../../../.tex_intermediates/topic_linear-algebra}

\begin{document}

\sectionline
\section{同型写像による観測}
\marginnote{
  \refweb{線形代数の基礎のキソ}{https://www1.econ.hit-u.ac.jp/kawahira/courses/kiso/01-senkei.pdf}
}

われわれは地図上の距離を何cmか測って、実際の距離を割り出すことがある。

これは地図$V$と実際の地形$U$の精巧な対応(同型写像)$f$をもとに、$V$の性質を$U$の性質とみなして距離の「観測」を行っているのである。

\br

このような考え方は、\keyword{引き戻し}や\keyword{押し出し}といった概念として数学に深く根付いている。

\sectionline
\section{基底が定める同型}
\marginnote{\refbookS p42}

\begin{mindflow}
  \begin{itemize}
    \item 座標写像と絡めて考えれば理解できそう
    \item \url{https://chatgpt.com/c/68a7e03b-a458-832c-a18b-c40ae25f938a}
    \item \url{https://www1.econ.hit-u.ac.jp/kawahira/courses/kiso/01-senkei.pdf} 1.5.1
  \end{itemize}
\end{mindflow}

\sectionline
\section{todo}
\marginnote{\refbookS p43}

% \refbookS 例2.1.6.1
\begin{theorem}{数ベクトル空間の標準埋め込みと部分空間の同型}{standard-embedding-subspace}
  $m \leq n$を自然数とし、$K^n$の部分空間$W$を次のように定める。
  \begin{equation*}
    W = \left\{ \begin{pmatrix}
      \vb*{x}_1 \\ \vdots \\ \vb*{x}_n
    \end{pmatrix} \in K^n \,\middle|\,
    \begin{gathered} 
      \vb*{x}_{m+1} = \cdots = \vb*{x}_n = 0
    \end{gathered}
    \right\}
  \end{equation*}
  このとき、次の写像は同型である。
  \begin{equation*}
    K^m \to W \colon \begin{pmatrix}
      \vb*{x}_1 \\ \vdots \\ \vb*{x}_m
    \end{pmatrix} \mapsto \begin{pmatrix}
      \vb*{x}_1 \\ \vdots \\ \vb*{x}_m \\ 0 \\ \vdots \\ 0
    \end{pmatrix}
  \end{equation*}
\end{theorem}

\begin{mindflow}
  \begin{itemize}
    \item 部分空間の定義時に、「入れものの空間のことはあまり意識せずに…」と書いたが、その根拠となるのがこの同型?
    \item \url{https://chatgpt.com/c/68a54b60-b3f8-8333-a900-aea35036832a}
  \end{itemize}
\end{mindflow}

この同型$K^m \to W$により、部分空間$W \subset K^n$を$K^m$と同一視できる。

\sectionline
\section{todo}
\marginnote{\refbookS p44}

% \refbookS 系2.1.8.2
\begin{theorem*}{todo}
  $V$を$n$次元線形空間とし、$W$を$V$の$m$次元部分空間とする。
  \thmref{thm:standard-embedding-subspace}の同型により、$K^m$を$K^n$の部分空間と同一視する。
  このとき、次が成り立つ。
  \begin{enumerate}[label=\romanlabel]
    \item 同型$f\colon K^n \to V$で、$f(K^m) = W$を満たすものが存在する
    \item \defref*{def:restriction-of-map}$f|_{K^m}\colon K^m \to W$は同型である
  \end{enumerate}
\end{theorem*}

\begin{mindflow}
  \begin{itemize}
    \item \url{https://chatgpt.com/c/68a8f174-667c-832a-b72c-473d5daf885f}
  \end{itemize}
\end{mindflow}

\end{document}
