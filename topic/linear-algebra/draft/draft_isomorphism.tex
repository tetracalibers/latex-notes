\documentclass[../../../topic_linear-algebra]{subfiles}

\begin{document}

\sectionline
\section{同型と同一視}
\marginnote{\refbookS p42〜43}

$V$と$W$が同型なら、$V$と$W$は線型空間としては本質的に同じものと考えることができる。

$W$が未知の線型空間でも、既知の線型空間$V$と同型なら、$W$のことも$V$と同じようによくわかることになる。

\br

既知の線型空間として、数ベクトル空間$K^n$を考えることが多い。

\sectionline
\section{基底を写す線形写像}
\marginnote{\refbookS p40〜41 \\ \refbookA p103}

% \refbookA 命題3.2.13、\refbookS 命題2.1.3
\begin{theorem}{基底を写す線形写像の存在}
  $V$を線型空間とし、$\vb*{v}_1, \ldots, \vb*{v}_n$を$V$の基底とする。
  $W$を線型空間とし、$\vb*{w}_1, \ldots, \vb*{w}_n \in W$を任意に与えるとき、次を満たす線形写像$f\colon V \to W$が一意的に存在する。
  \begin{equation*}
    f(\vb*{v}_i) = \vb*{w}_i \quad (i = 1, \ldots, n)
  \end{equation*}
\end{theorem}

\begin{proof}
  \begin{subpattern}{\bfseries $f$の存在}
    任意の$\vb*{v}\in V$は、基底$\{\vb*{v}_i\}_{i=1}^n$により一意的に
    \begin{equation*}
      \vb*{v} = \sum_{i=1}^n a_i \vb*{v}_i \quad (a_i \in K)
    \end{equation*}
    と表すことができる。
    
    そこで、$f$を次のように定める。
    \begin{equation*}
      f(\vb*{v}) = \sum_{i=1}^n a_i\vb*{w}_i
    \end{equation*}
    
    このとき、もし$\vb*{v}=\displaystyle\sum_{i=1}^n b_i\vb*{v}_i$とも表せるなら、基底による表示の一意性から$a_i=b_i$($i=1,\dots,n$)である。
    
    よって、$\displaystyle\sum_{i=1}^n a_i\vb*{w}_i=\sum_{i=1}^n b_i\vb*{w}_i$となり、$f$の定義は一意に定まる。$\qed$
  \end{subpattern}
  
  \begin{subpattern}{\bfseries $f$の線形性}
    $\displaystyle\vb*{a}=\sum_{i=1}^n a_i\vb*{v}_i,\ \vb*{b}=\sum_{i=1}^n b_i\vb*{v}_i$とし、$c_1,c_2$をスカラーとする。

    このとき、$\displaystyle c_1\vb*{a}+c_2\vb*{b}=\sum_{i=1}^n(c_1 a_i+c_2 b_i)\vb*{v}_i$であるから、$f$の定義より、
    \begin{align*}
      f(c_1\vb*{a}+c_2\vb*{b})
        &= \sum_{i=1}^n(c_1 a_i+c_2 b_i)\vb*{w}_i \\
        &= c_1\sum_{i=1}^n a_i\vb*{w}_i + c_2\sum_{i=1}^n b_i\vb*{w}_i \\
        &= c_1 f(\vb*{a})+c_2 f(\vb*{b})
    \end{align*}
    よって$f$は線形である。$\qed$
  \end{subpattern}
  
  \begin{subpattern}{\bfseries $f$の一意性}
    $f(\vb*{v}_i)=\vb*{w}_i$を満たす線形写像$g\colon V\to W$を任意にとる。
    
    任意の$\displaystyle\vb*{v}=\sum_{i=1}^n a_i\vb*{v}_i$に対し、$f$の線形性より、
    \begin{equation*}
      g(\vb*{v}) = g\left(\sum_{i=1}^n a_i\vb*{v}_i\right)
      = \sum_{i=1}^n a_i g(\vb*{v}_i)
      = \sum_{i=1}^n a_i\vb*{w}_i
      = f(\vb*{v})
    \end{equation*}
    したがって、$g=f$である。 $\qed$
  \end{subpattern}
\end{proof}

この定理の条件を満たすものとして、証明では線形写像$f$を次のように構成した。
\begin{equation*}
  f(\vb*{v}) = \sum_{i=1}^{n} a_i \vb*{w}_i
\end{equation*}
この$f$を、$V$の\keyword{基底}$\{ \vb*{v}_1, \ldots, \vb*{v}_n \}$を$\vb*{w}_1, \ldots, \vb*{w}_n \in W$に\keyword{写す}線形写像とよぶ。

\br

特に、$V = K^n$で、$V$の基底として\keyword{標準基底}$\{ \vb*{e}_1, \ldots, \vb*{e}_n \}$を選んだときは、この$f$を$\vb*{w}_1, \ldots, \vb*{w}_n \in W$が\keyword{定める}線形写像ともよぶ。

\sectionline
\section{基底が定める同型}
\marginnote{\refbookS p42}

\begin{mindflow}
  座標写像と絡めて考えれば理解できそう
\end{mindflow}

$V$を線型空間とするとき、$\vb*{v}_1, \ldots, \vb*{v}_n \in V$が基底であるとは、$\vb*{v}_1, \ldots, \vb*{v}_n$が定める線形写像$f\colon K^n \to V$が\keyword{同型}であるということである。

\br

$\vb*{v}_1, \ldots, \vb*{v}_n \in V$が基底であるとき、$\vb*{v}_1, \ldots, \vb*{v}_n$が定める線形写像$f\colon K^n \to V$を、基底$\vb*{v}_1, \ldots, \vb*{v}_n$が定める同型という。

\br

どんな線型空間$V$でも、$V$の基底があれば、数ベクトル空間から$V$への同型が定まるため、$V$の元を数ベクトルを使って表すことができる。

\begin{emphabox}
  \begin{spacebox}
    \begin{center}
      $V$の基底を\keywordJE{とる}{take}ことで、\\
      $V$の元を数ベクトルを使って表すことができる
    \end{center}
  \end{spacebox}
\end{emphabox}

\sectionline
\section{todo}
\marginnote{\refbookS p44}

\begin{mindflow}
  「線形同型写像による基底の保存」の逆も成り立つことを示すのがこの定理
\end{mindflow}

% \refbookS 命題2.1.7
\begin{theorem}{todo}
  $V$を線形空間とし、$\vb*{v}_1, \ldots, \vb*{v}_n$を$V$の基底とする。
  線形写像$f \colon V \to W$に対し、次の条件は同値である。
  \begin{enumerate}[label=\romanlabel]
    \item $f\colon V \to W$は同型である
    \item $f(\vb*{v}_1), \ldots, f(\vb*{v}_n)$は$W$の基底をなす
  \end{enumerate}
\end{theorem}

\begin{equation*}
  \begin{tikzcd}[every label/.append style = {font = \small}]
    K_n \arrow[r,"g"]\arrow[rd, "f \circ g"'] & V \arrow[d,"f"]\\
    & W
  \end{tikzcd}
\end{equation*}

\sectionline
\section{todo}
\marginnote{\refbookS p43}

% \refbookS 例2.1.6.1
\begin{theorem}{todo}
  $m \leq n$を自然数とし、$K^n$の部分空間$W$を次のように定める。
  \begin{equation*}
    W = \left\{ \begin{pmatrix}
      \vb*{x}_1 \\ \vdots \\ \vb*{x}_n
    \end{pmatrix} \in K^n \,\middle|\,
    \begin{gathered} 
      \vb*{x}_{m+1} = \cdots = \vb*{x}_n = 0
    \end{gathered}
    \right\}
  \end{equation*}
  このとき、次の写像は同型である。
  \begin{equation*}
    K^m \to W \colon \begin{pmatrix}
      \vb*{x}_1 \\ \vdots \\ \vb*{x}_m
    \end{pmatrix} \mapsto \begin{pmatrix}
      \vb*{x}_1 \\ \vdots \\ \vb*{x}_m \\ 0 \\ \vdots \\ 0
    \end{pmatrix}
  \end{equation*}
\end{theorem}

この同型$K^m \to W$により、部分空間$W \subset K^n$を$K^m$と同一視できる。

\begin{mindflow}
  部分空間の定義時に、「入れものの空間のことはあまり意識せずに…」と書いたが、その根拠となるのがこの同型?
  
  以下も参照:
  
  \url{https://chatgpt.com/c/68a54b60-b3f8-8333-a900-aea35036832a}
\end{mindflow}

\end{document}
