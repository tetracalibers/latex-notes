\documentclass[../../../topic_linear-algebra]{subfiles}

\begin{document}

\sectionline
\section{線形部分空間の定義}
\marginnote{\refbookA p93〜94}

$\mathbb{R}^n$の部分集合であって、ベクトル演算で閉じた集合について考える

原点を含み直線や平面などを一般化した概念である

\begin{definition}{線形部分空間}
  $\mathbb{R}^n$のベクトルからなる空集合でない集合$V$は、次が成り立つとき\keyword{線形部分空間}あるいは簡単に\keyword{部分空間}であるという
  \begin{enumerate}[label=\romanlabel]
    \item すべての$\vb*{u}, \vb*{v} \in V$に対して$\vb*{u} + \vb*{v} \in V$が成り立つ
    \item すべての$c \in \mathbb{R},\,\vb*{u} \in V$に対して$c \vb*{u} \in V$が成り立つ
  \end{enumerate}
\end{definition}

\subsection{線形部分空間の例:$\mathbb{R}^n$自身}

たとえば、$\mathbb{R}^n$自身は明らかに$\mathbb{R}^n$の部分空間である

\subsection{線形部分空間の例:零ベクトルだけからなる部分集合}

零ベクトル$\vb*{0}$だけからなる部分集合$\{ \vb*{0} \}$も部分空間である

\br

$V$は空集合でないので、ある$\vb*{v} \in V$をとるとき、線形部分空間の定義\romannum{ii}より
\begin{equation*}
  0 \cdot \vb*{v} = \vb*{0} \in V
\end{equation*}
よって部分空間は必ず$\vb*{0}$を含む

\subsection{線形部分空間の例:ベクトルが張る空間}

\begin{theorem}{ベクトルが張る空間は線形部分空間}
  $\vb*{v}_1, \vb*{v}_2, \dots, \vb*{v}_k \in \mathbb{R}^n$が張る空間$\langle \vb*{v}_1, \vb*{v}_2, \dots, \vb*{v}_k \rangle$は部分空間である
\end{theorem}

\begin{proof}
  \todo{\refbookA p94 命題3.1.2}
\end{proof}

\br

たとえば$\mathbb{R}^3$において座標を$(x, y, z)$とするとき、$xy$平面は$\mathbb{R}^3$の部分空間である

\begin{definition}{座標部分空間}
  $\{1, 2, \dots, n\}$の部分集合$I$に対して、$x_i \, (i \in I)$以外の座標がすべて0である部分集合は$\mathbb{R}^n$の部分集合である

  このようなものを\keyword{座標部分空間}といい、$\mathbb{R}^I$と書く
  \begin{equation*}
    \mathbb{R}^I = \langle \vb*{e}_i \mid i \in I \rangle
  \end{equation*}
  と表すこともできる
\end{definition}

\br

\begin{theorem}{部分空間の張る空間は部分空間}
  $V \subset \mathbb{R}^n$を部分空間、$\vb*{v}_1, \vb*{v}_2, \dots, \vb*{v}_k \in V$とすると、
  \begin{equation*}
    \langle \vb*{v}_1, \vb*{v}_2, \dots, \vb*{v}_k \rangle \subset V
  \end{equation*}
\end{theorem}

\begin{proof}
  \todo{\refbookA p94 命題3.1.4}
\end{proof}

\subsection{線形部分空間の例:共通部分}
\marginnote{\refbookC p22}

\begin{theorem}{線形部分空間の共通部分は部分空間}
  $V,\,W$を$\mathbb{R}^n$の部分空間とするとき、\keyword{共通部分}$V \cap W$は$\mathbb{R}^n$の部分空間である
\end{theorem}

\begin{proof}
  \begin{subpattern}{\bfseries 和について}
    $\vb*{a}, \vb*{b} \in V \cap W$とすると、共通部分の定義より、$\vb*{a}$と$\vb*{b}$はどちらも$V$と$W$の両方に属していることになる

    つまり、$\vb*{a}, \vb*{b} \in V$かつ$\vb*{a}, \vb*{b} \in W$である

    \br

    $V$も$W$も部分空間なので、部分空間の定義より、
    \begin{align*}
      \vb*{a} + \vb*{b} & \in V \\
      \vb*{a} + \vb*{b} & \in W
    \end{align*}

    $\vb*{a} + \vb*{b}$が$V$と$W$の両方に属していることから、$\vb*{a} + \vb*{b}$は$V \cap W$に属する

    よって、$V \cap W$は和について閉じている $\qed$
  \end{subpattern}

  \begin{subpattern}{\bfseries スカラー倍について}
    $\vb*{a} \in V \cap W$と$c \in \mathbb{R}$をとる

    共通部分の定義より、$\vb*{a}$は$V$と$W$の両方に属しているので、部分空間の定義より
    \begin{align*}
      c \vb*{a} & \in V \\
      c \vb*{a} & \in W
    \end{align*}

    よって、$c \vb*{a}$は$V \cap W$に属するため、$V \cap W$はスカラー倍について閉じている $\qed$
  \end{subpattern}
\end{proof}

\subsection{線形部分空間の例:和空間}

\begin{theorem}{線形部分空間の和は部分空間}
  $V,\,W$を$\mathbb{R}^n$の部分空間とするとき、\keyword{和空間}
  \begin{equation*}
    V + W \coloneq \{ \vb*{v} + \vb*{w} \mid \vb*{v} \in V, \vb*{w} \in W \}
  \end{equation*}
  は$\mathbb{R}^n$の部分空間である
\end{theorem}

\begin{proof}
  \begin{subpattern}{\bfseries 和について}
    $\vb*{a}_1, \vb*{a}_2 \in V, \, \vb*{b}_1, \vb*{b}_2 \in W$とする

    $V$と$W$は部分空間なので、部分空間の定義より
    \begin{equation*}
      \vb*{a}_1 + \vb*{a}_2 \in V, \quad \vb*{b}_1 + \vb*{b}_2 \in W
    \end{equation*}

    一方、和空間の定義より、$\vb*{a}_1 + \vb*{b}_1,\, \vb*{a}_2 + \vb*{b}_2$はそれぞれ$V+W$の元である

    これらの元の和をとったときに、その和も$V + W$に属していれば、和空間は和について閉じているといえる
    \begin{align*}
      (\vb*{a}_1 + \vb*{b}_1) + (\vb*{a}_2 + \vb*{b}_2) & = (\vb*{a}_1 + \vb*{a}_2) + (\vb*{b}_1 + \vb*{b}_2) \\
                                                        & \in V + W
    \end{align*}
    上式で、和空間は和について閉じていることが示された $\qed$
  \end{subpattern}

  \begin{subpattern}{\bfseries スカラー倍について}
    $\vb*{a} \in V, \, \vb*{b} \in W$と$c \in \mathbb{R}$をとる

    $V$と$W$は部分空間なので、部分空間の定義より
    \begin{align*}
      c \vb*{a} & \in V \\
      c \vb*{b} & \in W
    \end{align*}

    一方、和空間の定義より、$\vb*{a} + \vb*{b}$は$V + W$の元である

    この元をスカラー倍したときに、そのスカラー倍も$V + W$に属していれば、和空間はスカラー倍について閉じているといえる
    \begin{align*}
      c(\vb*{a} + \vb*{b}) & = c \vb*{a} + c \vb*{b} \\
                           & \in V + W
    \end{align*}
    上式で、和空間はスカラー倍について閉じていることが示された $\qed$
  \end{subpattern}
\end{proof}

\subsection{線形部分空間の例:線形写像の核空間}
\marginnote{\refbookC p71〜72}

\begin{theorem}{部分空間の零ベクトルと線形写像}\label{thm:linear-map-zero-preserving}
  部分空間$V,\,W$の間の線形写像$f\colon V \to W$に対して、$V$の零ベクトルを$\vb*{0}_V$、$W$の零ベクトルを$\vb*{0}_W$とすると、
  \begin{equation*}
    f(\vb*{0}_V) = \vb*{0}_W
  \end{equation*}
\end{theorem}

\begin{proof}
  任意の$\vb*{v} \in V,\, \vb*{w} \in W$に対して、
  \begin{align*}
    0  \cdot \vb*{v} & = \vb*{0}_V \\
    0 \cdot \vb*{w}  & = \vb*{0}_W
  \end{align*}
  が成り立つ

  \br

  $f(\vb*{0}_V)$は、$f$の線形性により、次のように変形できる
  \begin{equation*}
    f(\vb*{0}_V) = f(0 \cdot \vb*{v}) = 0 \cdot f(\vb*{v})
  \end{equation*}

  ここで、$f(\vb*{v})$は、$f$による$\vb*{v} \in V$の像であるので、$W$に属する

  そこで、$\vb*{w} = f(\vb*{v})$とおくと、
  \begin{align*}
    f(\vb*{0}_V) & = 0 \cdot f(\vb*{v}) \\
                 & = 0 \cdot \vb*{w}    \\
                 & = \vb*{0}_W
  \end{align*}
  となり、目標としていた式が示された $\qed$
\end{proof}

\marginnote{\refbookC p82}

\begin{theorem}{線形写像の核空間は部分空間}
  線形写像$f\colon V \to W$の核$\Ker(f)$は$V$の部分空間である
\end{theorem}

\begin{proof}
  \hyperref[thm:linear-map-zero-preserving]{前述の定理}の主張$f(\vb*{0}_V) = \vb*{0}_W$より、零ベクトルは核空間に属する
  \begin{equation*}
    \vb*{0} \in \Ker(f)
  \end{equation*}

  \begin{subpattern}{\bfseries 和について}
    $\vb*{u}, \vb*{v} \in \Ker(f)$とすると、$f(\vb*{u}) = \vb*{0}$かつ$f(\vb*{v}) = \vb*{0}$である

    よって、$f$の線形性より
    \begin{align*}
      f(\vb*{u} + \vb*{v}) & = f(\vb*{u}) + f(\vb*{v})     \\
                           & = \vb*{0} + \vb*{0} = \vb*{0}
    \end{align*}
    したがって、$\vb*{u} + \vb*{v} \in \Ker(f)$である $\qed$

  \end{subpattern}

  \begin{subpattern}{\bfseries スカラー倍について}
    $\vb*{u} \in \Ker(f)$と$c \in \mathbb{R}$をとると、$f(\vb*{u}) = \vb*{0}$である

    よって、$f$の線形性より
    \begin{align*}
      f(c \vb*{u}) & = c f(\vb*{u})              \\
                   & = c \cdot \vb*{0} = \vb*{0}
    \end{align*}
    したがって、$c \vb*{u} \in \Ker(f)$である $\qed$
  \end{subpattern}
\end{proof}

\sectionline
\section{線形写像の核空間}
\marginnote{\refbookA p94〜95}

すでに学んだように、斉次形方程式$A\vb*{x} = \vb*{0}$の解の自由度を$d$とすると、基本解$\vb*{u}_1, \vb*{u}_2, \dots, \vb*{u}_d \in \Ker(A)$が存在して、任意の$\vb*{u} \in \Ker(A)$に対して
\begin{equation*}
  \vb*{u} = c_1 \vb*{u}_1 + c_2 \vb*{u}_2 + \cdots + c_d \vb*{u}_d
\end{equation*}
を満たす$c_1, c_2, \dots, c_d \in \mathbb{R}$が一意的に定まる

\br

\todo{\refbookA p95}

\sectionline
\section{基底の定義}
\marginnote{\refbookA p96}

核空間の場合を参考にして、部分空間のパラメータ表示を与えるために基準として固定するベクトルの集合を定式化すると、\keyword{基底}という概念になる

\begin{definition}{基底}
  $V$を$\mathbb{R}^n$の部分空間とする

  ベクトルの集合$\{ \vb*{v}_1, \vb*{v}_2, \dots, \vb*{v}_k \} \subset V$は、次を満たすとき$V$の\keyword{基底}であるという
  \begin{enumerate}[label=\romanlabel]
    \item $\{ \vb*{v}_1, \vb*{v}_2, \dots, \vb*{v}_k \}$は線型独立である
    \item $V = \langle \vb*{v}_1, \vb*{v}_2, \dots, \vb*{v}_k \rangle$
  \end{enumerate}
\end{definition}

たとえば、基本ベクトルの集合$\{ \vb*{e}_1, \vb*{e}_2, \dots, \vb*{e}_n \}$は$\mathbb{R}^n$の基底であり、これを$\mathbb{R}^n$の\keyword{標準基底}という

\sectionline

核空間について先ほど述べたことは、基底の言葉で言い換えると次のようになる

\begin{theorem}{斉次形方程式の基本解と核空間の基底}
  $A$を$m \times n$型行列とし、$\vb*{u}_1, \vb*{u}_2, \dots, \vb*{u}_d$を$A\vb*{x} = \vb*{0}$の基本解とするとき、$\{ \vb*{u}_1, \vb*{u}_2, \dots, \vb*{u}_d \}$は$\Ker(A)$の基底である
\end{theorem}

\sectionline
\section{線形写像の像空間}
\marginnote{\refbookA p96〜97}

\todo{\refbookA p96〜97}

\end{document}
