\documentclass[../../../topic_linear-algebra]{subfiles}

\begin{document}

\sectionline
\section{基底と次元}
\marginnote{\refbookA p96、p99〜100 \\ \refbookC p33〜35}

部分空間のパラメータ表示を与えるために基準として固定するベクトルの集合を定式化すると、\keyword{基底}という概念になる

\br

\keyword{基底}は、座標空間の「座標軸」に相当するものであり、部分空間を生成する独立なベクトルの集合として定義される

\begin{definition}{基底}
  $V$を$\mathbb{R}^n$の部分空間とする

  ベクトルの集合$\{ \vb*{v}_1, \vb*{v}_2, \dots, \vb*{v}_k \} \subset V$は、次を満たすとき$V$の\keyword{基底}であるという
  \begin{enumerate}[label=\romanlabel]
    \item $\{ \vb*{v}_1, \vb*{v}_2, \dots, \vb*{v}_k \}$は線型独立である
    \item $V = \langle \vb*{v}_1, \vb*{v}_2, \dots, \vb*{v}_k \rangle$
  \end{enumerate}
\end{definition}

線形空間$V$の基底$\{ \vb*{v}_1, \vb*{v}_2, \dots, \vb*{v}_k \}$を1つ見つけたら、ベクトルの個数を数えて、$V$の\keyword{次元}が$k$であるとする

\begin{definition}{次元}
  $V$を線形空間とする

  $V$の基底をなすベクトルの個数を$V$の\keyword{次元}といい、$\dim V$と書く

  また、$\dim\{\vb*{0}\} = 0$と定義する
\end{definition}

\subsection{基底の例:標準基底}
\marginnote{\refbookC p35}

たとえば、基本ベクトルの集合$\{ \vb*{e}_1, \vb*{e}_2, \dots, \vb*{e}_n \}$は$\mathbb{R}^n$の基底であり、これを$\mathbb{R}^n$の\keyword{標準基底}という

標準基底$\{ \vb*{e}_1, \vb*{e}_2, \dots, \vb*{e}_n \}$は$n$個のベクトルからなるため、$\mathbb{R}^n$の次元は$n$である

\begin{theorem}{数ベクトル空間の標準基底}
  数ベクトル空間$K^n$において、基本ベクトルの集合$\{ \vb*{e}_1, \vb*{e}_2, \dots, \vb*{e}_n \}$は$K^n$の基底である
\end{theorem}

\begin{proof}
  \begin{subpattern}{\bfseries 部分空間を生成すること}
    任意のベクトル$\vb*{v} \in K^n$は、次のように表せる
    \begin{equation*}
      \vb*{v} = v_1 \vb*{e}_1 + v_2 \vb*{e}_2 + \cdots + v_n \vb*{e}_n
    \end{equation*}
    したがって、$K^n$は$\{ \vb*{e}_1, \vb*{e}_2, \dots, \vb*{e}_n \}$によって生成される $\qed$
  \end{subpattern}

  \begin{subpattern}{\bfseries 線型独立であること}
    $\vb*{e}_1, \vb*{e}_2, \dots, \vb*{e}_n$の線形関係式
    \begin{equation*}
      c_1 \vb*{e}_1 + c_2 \vb*{e}_2 + \cdots + c_n \vb*{e}_n = \vb*{0}
    \end{equation*}
    を考える

    このとき、左辺は
    \begin{equation*}
      c_1 \vb*{e}_1 + c_2 \vb*{e}_2 + \cdots + c_n \vb*{e}_n = \begin{pmatrix}
        c_1    \\
        c_2    \\
        \vdots \\
        c_n
      \end{pmatrix}
    \end{equation*}
    と書き換えられるので、これが零ベクトルになるためには、
    \begin{equation*}
      c_1 = 0, \quad c_2 = 0, \quad \cdots, \quad c_n = 0
    \end{equation*}
    でなければならない

    よって、$\{ \vb*{e}_1, \vb*{e}_2, \dots, \vb*{e}_n \}$は線型独立である $\qed$
  \end{subpattern}
\end{proof}

\sectionline

基底と次元を定義するにあたって、次の保証が必要になる

\begin{enumerate}[label=\romanlabel]
  \item 任意の部分空間に、基底の定義を満たす有限個のベクトルが存在すること(基底の存在)
  \item 任意の部分空間に対して、基底をなすベクトルの個数が、基底の選び方によらず一定であること(次元の不変性)
\end{enumerate}

\sectionline
\section{基底の存在}
\marginnote{\refbookA p98〜99 \\ \refbookC p36〜37}

基底の構成と存在を示すために、次の補題を用いる

\begin{theorem}{線型独立なベクトルの延長}\label{thm:extend-indep-outside-subspace}
  $V$を$K^n$の$\{\vb*{0} \}$でない部分空間とする

  このとき、$V$の線型独立なベクトル$\vb*{a}_1, \vb*{a}_2, \dots, \vb*{a}_m$と、$V$に入らないベクトル$\vb*{a}$は線型独立である
\end{theorem}

\begin{proof}
  $\vb*{a}, \vb*{a}_1, \vb*{a}_2, \dots, \vb*{a}_m$が線型従属であるとする

  すると、\hyperref[thm:dep-vec-is-lincomb]{定理「線形結合によるベクトルの表現」}より、$\vb*{a}$は$\vb*{a}_1, \vb*{a}_2, \dots, \vb*{a}_m$の線形結合で表され、$V$に入り、矛盾する

  よって、$\vb*{a}, \vb*{a}_1, \vb*{a}_2, \dots, \vb*{a}_m$は線型独立である $\qed$
\end{proof}

この定理は、\hyperref[def:span-of-vectors]{ベクトルの集合が張る空間}の記号を用いると、次のように簡潔にまとめられる

\begin{theorem}{線型独立なベクトルの延長}
  $\{ \vb*{v}_1, \dots, \vb*{v}_k \}$が線型独立であって、$\vb*{v}_{k+1} \notin \langle \vb*{v}_1, \dots, \vb*{v}_k \rangle$ならば、$\{ \vb*{v}_1, \dots, \vb*{v}_k, \vb*{v}_{k+1} \}$は線型独立である
\end{theorem}

\sectionline

$K^n$の$\{\vb*{0} \}$でない部分空間$V$の線型独立なベクトルは、$V$の基底に拡張できる

\begin{theorem}{基底の存在}
  $K^n$の$\{\vb*{0} \}$でない部分空間$V$には基底が存在する
\end{theorem}

\begin{proof}
  $V \neq  \{\vb*{0} \}$なので、$V$には少なくとも1つのベクトル$\vb*{v}_1 \neq \vb*{0}$が存在する

  \hyperref[thm:single-vec-indep-iff-nonzero]{定理「単一ベクトルの線型独立性と零ベクトル」}より、$\{\vb*{v}_1 \}$は線型独立である

  \br

  このとき、$\langle \vb*{v}_1 \rangle \subset V$であるが、もしも$\langle \vb*{v}_1 \rangle = V$ならば、$\{\vb*{v}_1 \}$は$V$の基底である

  \br

  $\langle \vb*{v}_1 \rangle \subsetneq V$ならば、$\vb*{v}_2 \subsetneq \langle \vb*{v}_1 \rangle$であるベクトルを$V$から選ぶことができる

  \hyperref[thm:extend-indep-outside-subspace]{補題「線型独立なベクトルの延長」}より、$\{\vb*{v}_1, \vb*{v}_2 \}$は線型独立である

  \br

  このとき、$\langle \vb*{v}_1, \vb*{v}_2 \rangle \subset V$であるが、もしも$\langle \vb*{v}_1, \vb*{v}_2 \rangle = V$ならば、$\{\vb*{v}_1, \vb*{v}_2 \}$は$V$の基底である

  \br

  $\langle \vb*{v}_1, \vb*{v}_2 \rangle \subsetneq V$ならば、$\vb*{v}_3 \subsetneq \langle \vb*{v}_1, \vb*{v}_2 \rangle$であるベクトルを$V$から選ぶことができる

  \hyperref[thm:extend-indep-outside-subspace]{補題「線型独立なベクトルの延長」}より、$\{\vb*{v}_1, \vb*{v}_2, \vb*{v}_3 \}$は線型独立である

  \br

  以下同様に続けると、$\langle \vb*{v}_1, \vb*{v}_2, \dots, \vb*{v}_k \rangle = V$となるまで、$V$に属するベクトルを選び続けることができる

  \br

  ここで線型独立なベクトルを繰り返し選ぶ操作が無限に続かないこと(有限値$k$が存在すること)は、\hyperref[thm:finite-dependency]{有限従属性定理}により、$K^n$の中には$n$個を超える線型独立なベクトルの集合は存在しないことから保証される $\qed$
\end{proof}

\sectionline
\section{次元の不変性}
\marginnote{\refbookA p99 \\ \refbookC p37〜38}

\begin{theorem}{次元の不変性}
  $K^n$の部分空間$V$の基底をなすベクトルの個数(次元)は一定である

  つまり、$\{ \vb*{v}_1, \dots, \vb*{v}_k \}$と$\{ \vb*{u}_1, \dots, \vb*{u}_l \}$がともに$V$の基底ならば、$k = l$である
\end{theorem}

\begin{proof}
  $\vb*{u}_1, \vb*{u}_2, \dots, \vb*{u}_l \in \langle \vb*{v}_1, \vb*{v}_2, \dots, \vb*{v}_k \rangle$であり、$\vb*{u}_1, \vb*{u}_2, \dots, \vb*{u}_l$は線型独立であるから、\hyperref[thm:abstract-finite-dependency]{有限従属性定理の抽象版}より、$l \leq k$である

  同様にして$k \leq l$も成り立つので、$k = l$である $\qed$
\end{proof}

\sectionline
\section{次元の性質}
\marginnote{\refbookA p100}

\begin{theorem}{線形独立なベクトルの最大個数と空間の次元}
  線形空間$V$中の線型独立なベクトルの最大個数は$\dim V$と等しい
\end{theorem}

\begin{proof}
  $V$の基底を$\{ \vb*{v}_1, \vb*{v}_2, \dots, \vb*{v}_k \}$とすると、$V$には$k$個の線型独立なベクトルが存在する

  また、$V = \langle \vb*{v}_1, \vb*{v}_2, \dots, \vb*{v}_k \rangle$であるため、\hyperref[thm:abstract-finite-dependency]{有限従属性定理の抽象版}より、$V$中の線型独立なベクトルの個数は$k$を超えることはない

  つまり、$k$は$V$に含まれる線型独立なベクトルの最大個数である $\qed$
\end{proof}

\sectionline

\begin{theorem}{線形空間を生成するベクトルの最小個数と次元}
  線形空間$V$を張るベクトルの最小個数は$\dim V$と等しい
\end{theorem}

\begin{proof}
  \todo{\refbookA p100 問3.3}
\end{proof}

\sectionline
\section{線形写像の核空間と基底}
\marginnote{\refbookA p94〜95}

斉次形方程式$A\vb*{x} = \vb*{0}$の解の自由度を$d$とすると、基本解$\vb*{u}_1, \vb*{u}_2, \dots, \vb*{u}_d \in \Ker(A)$が存在して、任意の$\vb*{u} \in \Ker(A)$に対して
\begin{equation*}
  \vb*{u} = c_1 \vb*{u}_1 + c_2 \vb*{u}_2 + \cdots + c_d \vb*{u}_d
\end{equation*}
を満たす$c_1, c_2, \dots, c_d \in \mathbb{R}$が一意的に定まる

\br

このことは、基底の言葉で言い換えると次のようになる

\begin{theorem}{斉次形方程式の基本解と核空間の基底}
  $A$を$m \times n$型行列とし、$\vb*{u}_1, \vb*{u}_2, \dots, \vb*{u}_d$を$A\vb*{x} = \vb*{0}$の基本解とするとき、$\{ \vb*{u}_1, \vb*{u}_2, \dots, \vb*{u}_d \}$は$\Ker(A)$の基底である
\end{theorem}

\sectionline
\section{線形写像の像空間と列空間}
\marginnote{\refbookA p96〜97}

\todo{\refbookA p96〜97}

\end{document}
