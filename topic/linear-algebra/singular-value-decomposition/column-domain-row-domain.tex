\documentclass[../../../topic_linear-algebra]{subfiles}

\begin{document}

\sectionline
\section{行空間と列空間}
\marginnote{\refbookI p30〜31}

ここでは、$A$の特異値の個数について調べていく

$A$の特異値の個数は$A$の階数に等しく、特異ベクトルは\keyword{行空間}と\keyword{列空間}の正規直交基底を成すことが示される

\begin{definition}{列空間}
  行列$A$の$n$本の列の張る$\mathbb{R}^m$の部分空間を$A$の\keyword{列空間}という
\end{definition}

\begin{definition}{行空間}
  行列$A$の$m$本の行の張る$\mathbb{R}^n$の部分空間を$A$の\keyword{行空間}という
\end{definition}

ここでは、列空間を$\mathcal{U}$、行空間を$\mathcal{V}$と表記することにする

\end{document}
