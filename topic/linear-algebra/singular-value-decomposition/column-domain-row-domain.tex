\documentclass[../../../topic_linear-algebra]{subfiles}

\usepackage{xr-hyper}
\externaldocument{../../../.tex_intermediates/topic_linear-algebra}

\begin{document}

\sectionline
\section{列空間と行空間の正規直交基底}
\marginnote{\refbookI p30〜31}

ここでは、$A$の特異値の個数は$A$の階数に等しく、特異ベクトルは\keyword{行空間}と\keyword{列空間}の正規直交基底を成すことを示す。

\begin{definition*}{列空間}
  行列$A$の$n$本の列の張る$\mathbb{R}^m$の部分空間を$A$の\keyword{列空間}という。
\end{definition*}

\begin{definition*}{行空間}
  行列$A$の$m$本の行の張る$\mathbb{R}^n$の部分空間を$A$の\keyword{行空間}という。
\end{definition*}

列空間を$\mathcal{U}$、行空間を$\mathcal{V}$と表記することにする。

\subsection{列空間の正規直交基底}

$A$の列$\vb*{a}_1,\ldots,\vb*{a}_n$の任意の線形結合を考える。
\begin{equation*}
  c_1 \vb*{a}_1 + \cdots + c_n \vb*{a}_n = \begin{pmatrix}
    \vb*{a}_1 & \cdots & \vb*{a}_n
  \end{pmatrix} \begin{pmatrix}
    c_1    \\
    \vdots \\
    c_n
  \end{pmatrix} = A \vb*{c}
\end{equation*}

ここで、\secref{sec:svd-of-A}の式
\begin{equation*}
  A = \sigma_1\vb*{u}_1\vb*{v}_1^\top + \cdots + \sigma_r \vb*{u}_r \vb*{v}_r^\top
\end{equation*}
の両辺に$\vb*{c}$を右からかけると、
\begin{align*}
  A \vb*{c} & = \sigma_1\vb*{u}_1\vb*{v}_1^\top \vb*{c} + \cdots + \sigma_r \vb*{u}_r \vb*{v}_r^\top \vb*{c} \\
            & = \sigma_1 (\vb*{v}_1, \vb*{c}) \vb*{u}_1 + \cdots + \sigma_r (\vb*{v}_r, \vb*{c}) \vb*{u}_r
\end{align*}
すなわち、$A$の列の任意の線形結合$A\vb*{c}$は、互いに直交する左特異ベクトル$\vb*{u}_1,\ldots,\vb*{u}_r$の線形結合で書ける。

\br

よって、$\vb*{a}_1,\ldots,\vb*{a}_n$の張る列空間$\mathcal{U}$は、左特異ベクトル$\vb*{u}_1,\ldots,\vb*{u}_r$を正規直交基底とする$r$次元部分空間である。

\br

このことから、
\begin{emphabox}
  \begin{spacebox}
    \begin{center}
      $r$本の列のみが線型独立である
    \end{center}
  \end{spacebox}
\end{emphabox}
ということもいえる。

\subsection{行空間の正規直交基底}

$A$の行は$A^\top$の列であるので、同様の議論を$A^\top$に対して行う。

\br

\secref{sec:svd-of-transpose-A}の式
\begin{equation*}
  A^\top = \sigma_1\vb*{v}_1\vb*{u}_1^\top + \cdots + \sigma_r \vb*{v}_r \vb*{u}_r^\top
\end{equation*}
の両辺に$\vb*{c}$を右からかけることで、
\begin{align*}
  A^\top \vb*{c} & = \sigma_1\vb*{v}_1\vb*{u}_1^\top \vb*{c} + \cdots + \sigma_r \vb*{v}_r \vb*{u}_r^\top \vb*{c} \\
                 & = \sigma_1 (\vb*{u}_1, \vb*{c}) \vb*{v}_1 + \cdots + \sigma_r (\vb*{u}_r, \vb*{c}) \vb*{v}_r
\end{align*}
となり、$A$の行の任意の線形結合$A^\top \vb*{c}$は、互いに直交する右特異ベクトル$\vb*{v}_1,\ldots,\vb*{v}_r$の線形結合で書けることがわかる。

\br

よって、$A$の行の張る行空間$\mathcal{V}$は、右特異ベクトル$\vb*{v}_1,\ldots,\vb*{v}_r$を正規直交基底とする$r$次元部分空間である。

\br

このことから、
\begin{emphabox}
  \begin{spacebox}
    \begin{center}
      $r$本の行のみが線型独立である
    \end{center}
  \end{spacebox}
\end{emphabox}
ということもいえる。

\subsection{特異値の個数と特異ベクトルによる基底}

以上の議論から、次のことがわかる。

\begin{theorem}{特異値の個数と階数}{rank-and-singular-values}
  行列$A$の階数$r$は、$A$の特異値の個数に等しい。
\end{theorem}

\begin{theorem*}{特異ベクトルと列空間・行空間の正規直交基底}
  左特異ベクトル$\{\vb*{u}_i\}_{i=1}^r$と右特異ベクトル$\{\vb*{v}_i\}_{i=1}^r$は、それぞれ列空間$\mathcal{U}$と行空間$\mathcal{V}$の正規直交基底を成す。
\end{theorem*}

\sectionline
\section{列空間と行空間への射影}
\marginnote{\refbookI p31}

特異ベクトルが列空間・行空間の正規直交基底をなすことから、これらを用いて\secref{sec:projection-matrix-expansion}を考えることができる。

\br

$\mathbb{R}^m$の列空間$\mathcal{U}$への射影行列を$P_{\mathcal{U}}$、$\mathbb{R}^n$の行空間$\mathcal{V}$への射影行列を$P_{\mathcal{V}}$とすると、
\begin{align*}
  P_{\mathcal{U}} & = \sum_{i=1}^r \vb*{u}_i \vb*{u}_i^\top \\
  P_{\mathcal{V}} & = \sum_{i=1}^r \vb*{v}_i \vb*{v}_i^\top
\end{align*}

\br

ここで、$\vb*{u}_i$は列空間$\mathcal{U}$の正規直交基底であることから、$\vb*{u}_i \in \mathcal{U}$である。

よって、\secref{sec:projection-onto-subspace}で議論したように、$P_{\mathcal{U}}$は列空間$\mathcal{U}$の元をそのまま写すので、
\begin{equation*}
  P_{\mathcal{U}}\vb*{u}_i = \vb*{u}_i
\end{equation*}

このことから、\secref{sec:svd-of-A}の式
\begin{equation*}
  A = \sigma_1\vb*{u}_1\vb*{v}_1^\top + \cdots + \sigma_r \vb*{u}_r \vb*{v}_r^\top
\end{equation*}
の両辺に左から$P_{\mathcal{U}}$をかけても変化しないことが次のように導かれる。
\begin{align*}
  P_{\mathcal{U}} A & = P_{\mathcal{U}} \left( \sigma_1\vb*{u}_1\vb*{v}_1^\top + \cdots + \sigma_r \vb*{u}_r \vb*{v}_r^\top \right)   \\
                    & = \sigma_1 P_{\mathcal{U}} \vb*{u}_1\vb*{v}_1^\top + \cdots + \sigma_r P_{\mathcal{U}} \vb*{u}_r \vb*{v}_r^\top \\
                    & = \sigma_1 \vb*{u}_1\vb*{v}_1^\top + \cdots + \sigma_r \vb*{u}_r \vb*{v}_r^\top                                 \\
                    & = A
\end{align*}

\br

行についても同様に、$P_{\mathcal{V}}\vb*{v}_i = \vb*{v}_i$であるから、\secref{sec:svd-of-transpose-A}の式
\begin{equation*}
  A^\top = \sigma_1\vb*{v}_1\vb*{u}_1^\top + \cdots + \sigma_r \vb*{v}_r \vb*{u}_r^\top
\end{equation*}
の両辺に右から$P_{\mathcal{V}}$をかけても変化しない。
\begin{align*}
  A P_{\mathcal{V}} & = \left( \sigma_1\vb*{v}_1\vb*{u}_1^\top + \cdots + \sigma_r \vb*{v}_r \vb*{u}_r^\top \right) P_{\mathcal{V}}    \\
                    & = \sigma_1 \vb*{v}_1 P_{\mathcal{V}} \vb*{u}_1^\top + \cdots + \sigma_r \vb*{v}_r P_{\mathcal{V}} \vb*{u}_r^\top \\
                    & = \sigma_1 \vb*{v}_1\vb*{u}_1^\top + \cdots + \sigma_r \vb*{v}_r \vb*{u}_r^\top                                  \\
                    & = A^\top
\end{align*}

\br

以上をまとめると、次の式が成り立つ。
\begin{align*}
  P_{\mathcal{U}} A & = A      \\
  A P_{\mathcal{V}} & = A^\top
\end{align*}

\end{document}
