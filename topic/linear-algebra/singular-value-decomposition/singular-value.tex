\documentclass[../../../topic_linear-algebra]{subfiles}

\begin{document}

\sectionline
\section{特異値と特異ベクトル}
\marginnote{\refbookI p28〜29}

スペクトル分解の拡張である特異値分解では、任意の行列がその\keyword{特異値}と\keyword{特異ベクトル}によって表せる

\begin{definition}{特異値と特異ベクトル}
  零行列ではない任意の$m \times n$行列$A$に対して、
  \begin{equation*}
    A\vb*{v}       = \sigma \vb*{u}, \quad
    A^\top \vb*{u} = \sigma \vb*{v}
  \end{equation*}
  となる正の数$\sigma$を\keyword{特異値}と呼び、
  \begin{itemize}
    \item \keyword{左特異ベクトル}:$m$次元ベクトル$\vb*{u} \, (\neq \vb*{0})$
    \item \keyword{右特異ベクトル}:$n$次元ベクトル$\vb*{v} \, (\neq \vb*{0})$
  \end{itemize}
  を合わせて\keyword{特異ベクトル}と呼ぶ
\end{definition}

\subsection{特異ベクトルと固有ベクトルの関係}

特異値と特異ベクトルの関係式
\begin{equation*}
  A\vb*{v} = \sigma \vb*{u}, \quad A^\top \vb*{u} = \sigma \vb*{v}
\end{equation*}
において、第1式の両辺に$A^\top$を左からかけると、
\begin{equation*}
  \begin{WithArrows}
    A^\top A \vb*{v} & = \sigma A^\top \vb*{u} \Arrow{第$2$式を代入} \\
    & = \sigma^2 \vb*{v}
  \end{WithArrows}
\end{equation*}
また、第2式の両辺に$A$を左からかけると、
\begin{equation*}
  \begin{WithArrows}
    A A^\top \vb*{u} & = \sigma A \vb*{v} \Arrow{第$1$式を代入} \\
    & = \sigma^2 \vb*{u}
  \end{WithArrows}
\end{equation*}

得られた結果をまとめると、
\begin{equation*}
  A A^\top \vb*{u} = \sigma^2 \vb*{u} , \quad A^\top A \vb*{v} = \sigma^2 \vb*{v}
\end{equation*}
ここで、$A$は任意の長方行列だが、\hyperref[thm:symmetric-products-of-any-matrix]{$A A^\top$と$A^\top A$は対称行列}となる

\br

すなわち、
\begin{itemize}
  \item 左特異ベクトル$\vb*{u}$は$m$次対称行列$A A^\top$の固有ベクトル
  \item 右特異ベクトル$\vb*{v}$は$n$次対称行列$A^\top A$の固有ベクトル
\end{itemize}
であり、特異値の2乗$\sigma^2$は\hyperref[thm:svd-singular-value-vector-correspondence]{$A A^\top, A^\top A$共通の固有値}である

\subsection{特異ベクトルの正規直交化}

$A$の特異値を$\sigma_1 \geq \cdots \geq \sigma_r > 0$とする

ここで、重複があってもよい

\br

対応する$r$本の左特異ベクトル$\vb*{u}_1, \ldots, \vb*{u}_r$と$r$本の右特異ベクトル$\vb*{v}_1, \ldots, \vb*{v}_r$は、どちらも\hyperref[sec:orthogonalization-eigenvectors-symmetric]{対称行列の固有ベクトルであるから、それぞれを正規直交系に}選ぶことができる

\end{document}
