\documentclass[../../../topic_linear-algebra]{subfiles}

\begin{document}

\sectionline
\section{特異値分解の行列表記}\label{sec:matrix-form-svd}
\marginnote{\refbookI p31〜32 \\ \refbookK p143〜145 \\ \refbookA p223〜228}

特異値分解の式
\begin{equation*}
  A = \sigma_1\vb*{u}_1\vb*{v}_1^\top + \cdots + \sigma_r \vb*{u}_r \vb*{v}_r^\top
\end{equation*}
は、次のように行列で表すこともできる。
\begin{align*}
  A & = \begin{pmatrix}
    \sigma_1\vb*{u}_1 & \cdots & \sigma_r\vb*{u}_r
  \end{pmatrix} \begin{pmatrix}
    \vb*{v}_1^\top \\
    \vdots \\
    \vb*{v}_r^\top
  \end{pmatrix} \\
  & = \begin{pmatrix}
    \vb*{u}_1 & \cdots & \vb*{u}_r
  \end{pmatrix} \begin{pmatrix}
                        \sigma_1 &        &           \\
                                  & \ddots &           \\
                                  &        & \sigma_r
                      \end{pmatrix} \begin{pmatrix}
    \vb*{v}_1^\top \\
    \vdots      \\
    \vb*{v}_r^\top
  \end{pmatrix}
\end{align*}
ここで、
\begin{gather*}
  U_r = \begin{pmatrix}
    \vb*{u}_1 & \cdots & \vb*{u}_r
  \end{pmatrix}, \quad
  V_r = \begin{pmatrix}
    \vb*{v}_1 & \cdots & \vb*{v}_r
  \end{pmatrix}, \\
  \Sigma_r = \begin{pmatrix}
                        \sigma_1 &        &           \\
                                  & \ddots &           \\
                                  &        & \sigma_r
                      \end{pmatrix}
\end{gather*}
とおくと、
\begin{equation*}
  A = U_r \Sigma_r V_r^\top
\end{equation*}
という、行列$A$を3つの行列を用いて分解した式として表すことができる。

この式は、$A$の\keyword{簡約された特異値分解}と呼ばれる。

\subsection{特異値分解のより一般的な形}

先ほどの式が「簡約された」特異値分解と呼ばれるということは、簡約する前のより一般的な形も考えられるということである。

\br

元々、特異値分解の式は、$A$が$\mathbb{R}^n$の正規直交基底$\{ \vb*{v}_1, \ldots, \vb*{v}_n \}$をそれぞれ
\begin{equation*}
  \sigma_1\vb*{u}_1,\ldots,\sigma_r\vb*{u}_r,\vb*{o},\ldots,\vb*{o}
\end{equation*}
に写像することから導かれた。

\br

そこで、$r+1$番以降の項も省略せずに書くと、
\begin{align*}
  A &= \sigma_1\vb*{u}_1\vb*{v}_1^\top + \cdots + \sigma_r \vb*{u}_r \vb*{v}_r^\top + \vb*{o}\vb*{v}_{r+1}^\top + \cdots + \vb*{o}\vb*{v}_n^\top \\
  & = \begin{pmatrix}
    \sigma_1\vb*{u}_1 & \cdots & \sigma_r\vb*{u}_r & \vb*{o} & \cdots & \vb*{o}
  \end{pmatrix}
  \begin{pmatrix}
    \vb*{v}_1^\top \\
    \vdots \\
    \vb*{v}_r^\top \\
    \vb*{v}_{r+1}^\top \\
    \vdots \\
    \vb*{v}_n^\top
  \end{pmatrix} \\
  & = \begin{pmatrix}
    \vb*{u}_1 & \cdots & \vb*{u}_m
  \end{pmatrix} \begin{pNiceArray}{ccc|cc}[xdots={horizontal-labels,line-style = <->},margin,columns-width =1em]
                                           \sigma_1 & & & \Block{3-2}<\large>{O} &  \\
                                           & \ddots &&& \\
                                           & & \sigma_r&& \\
                                           \hline
                                           \Block{2-3}<\large>{O} && & \Block{2-2}<\large>{O} & \\
                                           &&&&
                                         \end{pNiceArray} \begin{pmatrix}
    \vb*{v}_1^\top \\
    \vdots \\
    \vb*{v}_n^\top
  \end{pmatrix}
\end{align*}

\begin{handout}
  この式変形は、ブロック行列の積の計算に基づいている。
  
  たとえば、
  \begin{equation*}
    A = \begin{pmatrix}
      A_{11} & A_{12} 
    \end{pmatrix}, \quad
    B = \begin{pmatrix}
      B_{11} & B_{12} \\
      B_{21} & B_{22}
    \end{pmatrix}
  \end{equation*}
  とおけば、
  \begin{equation*}
    AB = \begin{pmatrix}
      A_{11}B_{11} + A_{12}B_{21} & A_{11}B_{12} + A_{12}B_{22}
    \end{pmatrix}
  \end{equation*}
  のように計算できる。
  
  \br
  
  そこで、
  \begin{gather*}
    U_1 = \begin{pmatrix}
      \vb*{u}_1 & \cdots & \vb*{u}_r
    \end{pmatrix}, \quad
    U_2 = \begin{pmatrix}
      \vb*{u}_{r+1} & \cdots & \vb*{u}_m
    \end{pmatrix}, \\
    D = \diag(\sigma_1, \ldots, \sigma_r)
  \end{gather*}
  とおくと、
  \begin{align*}
    \begin{pmatrix}
      U_1 & U_2
    \end{pmatrix} \begin{pmatrix}
      D & O \\
      O & O
    \end{pmatrix} & = \begin{pmatrix}
      U_1D + U_2O & U_1O + U_2O
    \end{pmatrix} \\
    & = \begin{pmatrix}
      U_1D & O
    \end{pmatrix} \\
    & = \begin{pmatrix}
      \sigma_1\vb*{u}_1 & \cdots & \sigma_r\vb*{u}_r & \vb*{o} & \cdots & \vb*{o}
    \end{pmatrix}
  \end{align*}
  という式変形が確かめられる。
\end{handout}

\br

ここで、
\begin{gather*}
  U = \begin{pmatrix}
    \vb*{u}_1 & \cdots & \vb*{u}_m
  \end{pmatrix}, \quad
  V = \begin{pmatrix}
    \vb*{v}_1 & \cdots & \vb*{v}_n
  \end{pmatrix}, \\
  \Sigma = \begin{pNiceArray}{ccc|cc}[xdots={horizontal-labels,line-style = <->},first-row,last-col,margin,columns-width =1em]
                                           \Hdotsfor{3}^{r} & \Hdotsfor{2}^{n-r} \\
                                           \sigma_1 & & & \Block{3-2}<\large>{O} && \Vdotsfor{3}^{r}  \\
                                           & \ddots &&& \\
                                           & & \sigma_r&& \\
                                           \hline
                                           \Block{2-3}<\large>{O} && & \Block{2-2}<\large>{O} && \Vdotsfor{2}^{m-r} \\
                                           &&&&
                                         \end{pNiceArray}
\end{gather*}
とおくと、
\begin{equation*}
  A = U \Sigma V^\top
\end{equation*}
と表せる。この式を$A$の\keyword{特異値分解}と呼ぶ。

\br

簡約された特異値分解は、特異値分解において$U$の余計な列と$\Sigma$の零行を省いたものだといえる。

\end{document}
