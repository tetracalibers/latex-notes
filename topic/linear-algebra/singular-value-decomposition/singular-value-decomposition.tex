\documentclass[../../../topic_linear-algebra]{subfiles}

\begin{document}

\sectionline
\section{特異値分解}
\marginnote{\refbookI p28〜30}

$k$本の左特異ベクトルの正規直交系$\vb*{u}_1,\ldots,\vb*{u}_k$を拡張して、$\mathbb{R}^m$の正規直交基底$\vb*{u}_1,\ldots,\vb*{u}_k,\vb*{u}_{k+1},\ldots,\vb*{u}_m$が定義できる

同様に、$k$本の右特異ベクトルの正規直交系$\vb*{v}_1,\ldots,\vb*{v}_k$を拡張して、$\mathbb{R}^n$の正規直交基底$\vb*{v}_1,\ldots,\vb*{v}_k,\vb*{v}_{k+1},\ldots,\vb*{v}_n$が定義できる

\br

$\vb*{u}_1,\ldots,\vb*{u}_n$と$\vb*{v}_1,\ldots,\vb*{v}_n$はそれぞれ$AA^\top$と$A^\top A$の固有ベクトルであり、これらに対応する共通の固有値を$\lambda_1,\ldots,\lambda_n$とおく

\br

$AA^\top$および$A^\top A$は半正定値行列であるので、その固有値はすべて零か正の数である

また、$AA^\top$および$A^\top A$は対称行列であり、対称行列の階数$r$は非零の固有値の個数に等しい

$n$個の固有値のうち、$r$個ある正の固有値は特異値の条件を満たすので、
\begin{itemize}
  \item $\lambda_1,\ldots,\lambda_r$は特異値(正の固有値)$\sigma_1,\ldots,\sigma_r$
  \item $\lambda_{r+1},\ldots,\lambda_{n}$は零の固有値
\end{itemize}
とする

\br

特異値が$r$個あることから、左特異ベクトルと特異値の組の個数、右特異ベクトルと特異値の組の個数は、どちらも$r$であることがいえる
\begin{equation*}
  k = r
\end{equation*}

以上の議論をまとめると、
\begin{align*}
  AA^\top \vb*{u}_i  & = \begin{cases}
                           \sigma_i \vb*{u}_i & (i = 1,\ldots,r)   \\
                           \vb*{0}            & (i = r+1,\ldots,m)
                         \end{cases} \\
  A^\top A \vb*{v}_i & = \begin{cases}
                           \sigma_i \vb*{v} & (i = 1,\ldots, r)   \\
                           \vb*{0}          & (i = r+1,\ldots, n)
                         \end{cases}
\end{align*}

\br

ここで、$i = 1,\ldots,r$の範囲に限っては、特異値と特異ベクトルの関係より、
\begin{align*}
  A^\top \vb*{u}_i & = \sigma_i \vb*{v}_i \\
  A\vb*{v}_i       & = \sigma_i \vb*{u}_i
\end{align*}
という形で書ける

$i >r$の場合についても同じ形で書くために、次の定理を示す

\begin{theorem}{todo}
  \begin{enumerate}[label=\romanlabel]
    \item $AA^\top \vb*{u} = \vb*{0} \Longrightarrow A^\top \vb*{u} = \vb*{0}$
    \item $A^\top A \vb*{v} = \vb*{0} \Longrightarrow A\vb*{v} = \vb*{0}$
  \end{enumerate}
\end{theorem}

\begin{proof}
  \begin{subpattern}{(\romannum{i}) $AA^\top \vb*{u} = \vb*{0}$\bfseries について}
    $AA^\top \vb*{u} = \vb*{0}$の両辺で$\vb*{u}$との内積をとって、
    \begin{equation*}
      (\vb*{u}, AA^\top \vb*{u}) = 0
    \end{equation*}
    このとき、左辺は、
    \begin{equation*}
      \begin{WithArrows}
        (\vb*{u}, AA^\top \vb*{u}) & = (\vb*{u}, A(A^\top\vb*{u})) \Arrow{外側の$A$に\\ 随伴公式を適用}  \\
        & = (A^\top \vb*{u}, A^\top \vb*{u}) \\
        & = \|A^\top \vb*{u}\|^2
      \end{WithArrows}
    \end{equation*}
    と変形できるので、
    \begin{equation*}
      \|A^\top \vb*{u}\|^2 = 0
    \end{equation*}
    が成り立つ

    \br

    ここで、内積の正値性
    \begin{equation*}
      \|A^\top \vb*{u}\|^2 = (A^\top\vb*{u}, A^\top\vb*{u}) \geq 0
    \end{equation*}
    において、等号が成立するのは、
    \begin{equation*}
      A^\top\vb*{u} = \vb*{0}
    \end{equation*}
    の場合のみである $\qed$
  \end{subpattern}

  \begin{subpattern}{(\romannum{ii}) $A^\top A \vb*{v} = \vb*{0}$\bfseries について}
    $A^\top A \vb*{v} = \vb*{0}$の両辺で$\vb*{v}$との内積をとって、
    \begin{equation*}
      (\vb*{v}, A^\top A \vb*{v}) = 0
    \end{equation*}
    このとき、左辺は、
    \begin{equation*}
      (\vb*{v}, A^\top A \vb*{v}) = (A\vb*{v}, A\vb*{v}) = \|A\vb*{v}\|^2
    \end{equation*}
    と変形できるので、
    \begin{equation*}
      \|A\vb*{v}\|^2 = 0
    \end{equation*}
    が成り立つ

    \br

    ここで、内積の正値性
    \begin{equation*}
      \|A\vb*{v}\|^2 = (A\vb*{v},A\vb*{v}) \geq 0
    \end{equation*}
    において、等号が成立するのは、
    \begin{equation*}
      A\vb*{v} = \vb*{0}
    \end{equation*}
    の場合のみである $\qed$
  \end{subpattern}
\end{proof}

\br

この定理を用いると、
\begin{align*}
  A\vb*{v}_i       & = \begin{cases}
                         \sigma_i \vb*{u}_i & (i = 1,\ldots,r)   \\
                         \vb*{0}            & (i = r+1,\ldots,m)
                       \end{cases}  \\
  A^\top \vb*{u}_i & = \begin{cases}
                         \sigma_i \vb*{v}_i & (i = 1,\ldots, r)   \\
                         \vb*{0}            & (i = r+1,\ldots, n)
                       \end{cases}
\end{align*}
とまとめられる

\br

これより、$A$は$\mathbb{R}^n$の正規直交基底$\{ \vb*{v}_1, \ldots, \vb*{v}_n \}$をそれぞれ
\begin{equation*}
  \sigma_1\vb*{u}_1,\ldots,\sigma_r\vb*{u}_r,\vb*{0},\ldots,\vb*{0}
\end{equation*}
に写像するから、\hyperref[thm:orthobasis-formula-for-rep-matrix]{正規直交基底による表現行列の展開}より、$A$は
\begin{equation*}
  A = \sigma_1\vb*{u}_1\vb*{v}_1^\top + \cdots + \sigma_r \vb*{u}_r \vb*{v}_r^\top \quad (\sigma_1 \geq \cdots \geq \sigma_r > 0)
\end{equation*}
と表すことができる

\br

同様に、$A^\top$は$\mathbb{R}^m$の正規直交基底$\{ \vb*{u}_1, \ldots, \vb*{u}_m\}$をそれぞれ
\begin{equation*}
  \sigma_1\vb*{v}_1, \ldots, \sigma_r\vb*{v}_r,\vb*{0},\ldots,\vb*{0}
\end{equation*}
に写像するから、$A^\top$は
\begin{equation*}
  A^\top = \sigma_1\vb*{v}_1\vb*{u}_1^\top + \cdots + \sigma_r \vb*{v}_r \vb*{u}_r^\top \quad (\sigma_1 \geq \cdots \geq \sigma_r > 0)
\end{equation*}
と表すことができる

\br

このように、任意の行列は、その特異値と特異ベクトルによって表すことができ、これを\keyword{特異値分解}と呼ぶ

\end{document}
