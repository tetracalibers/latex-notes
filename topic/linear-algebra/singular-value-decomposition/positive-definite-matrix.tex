\documentclass[../../../topic_linear-algebra]{subfiles}

\begin{document}

\sectionline
\section{正定値行列と半正定値行列}
\marginnote{\refbookI p29 \\ \refbookK p137〜138}

固有値がすべて0以上になる対称行列は、応用上さまざまな場面で現れる

\begin{itemize}
  \item \keyword{半正定値}行列:すべての固有値が非負(正または零)である対称行列
  \item \keyword{正定値}行列:すべての固有値が正である対称行列
\end{itemize}

\br

\begin{definition}{正定値行列}
  $A$をエルミート行列(対称行列)とし、任意のベクトル$\vb*{x} \in \mathbb{C}^n$($\vb*{x} \in \mathbb{R}^n$)に対して、
  \begin{equation*}
    (A\vb*{x}, \vb*{x}) > 0
  \end{equation*}
  が成り立つとき、$A$は\keyword{正定値行列}であるという
\end{definition}

\begin{theorem}{正定値性と固有値の正実性}
  エルミート行列$A$が正定値行列であることと、$A$のすべての固有値が正の実数であることは同値である
\end{theorem}

\begin{proof}
  \begin{subpattern}{\bfseries 正定値行列 $\Longrightarrow$ 固有値が正}
    $A$の固有値を$\lambda$、対応する固有ベクトルを$\vb*{x}$とすると、
    \begin{equation*}
      A\vb*{x} = \lambda \vb*{x}
    \end{equation*}
    両辺で$\vb*{x}$との内積をとると、
    \begin{equation*}
      (A\vb*{x}, \vb*{x}) = \lambda (\vb*{x}, \vb*{x}) = \lambda \|\vb*{x}\|^2
    \end{equation*}
    $A$が正定値行列であることから、$(A\vb*{x}, \vb*{x}) > 0$が成り立ち、
    \begin{equation*}
      \lambda \|\vb*{x}\|^2 > 0
    \end{equation*}

    \br

    ここで、固有ベクトルは零ベクトルではないので、$\|\vb*{x}\|^2 >0$である

    よって、$\lambda \|\vb*{x}\|^2 > 0$の両辺を$\|\vb*{x}\|^2$で割ることにより、
    \begin{equation*}
      \lambda > 0
    \end{equation*}
    が得られる $\qed$
  \end{subpattern}

  \begin{subpattern}{\bfseries 固有値が正 $\Longrightarrow$ 正定値行列}
    $A$の固有値を$\lambda_1,\ldots,\lambda_n >0$とする

    $A$はエルミート行列であることから、ユニタリ行列$U$を用いて次のように対角化できる
    \begin{equation*}
      A = U D U^{-1} = UDU^* = U \begin{pmatrix}
        \lambda_1 &        &           \\
                  & \ddots &           \\
                  &        & \lambda_n
      \end{pmatrix} U^*
    \end{equation*}

    \hyperref[thm:inner-product-adjoint-form]{随伴による内積の表現}より、
    \begin{equation*}
      (A\vb*{x}, \vb*{x}) = \vb*{x}^* A \vb*{x} = \vb*{x}^* U D U^* \vb*{x}
    \end{equation*}
    ここで、$\vb*{y} = U^* \vb*{x}$とおくと、
    \begin{equation*}
      \vb*{y}^* = (U^* \vb*{x})^* = \vb*{x}^* U
    \end{equation*}
    となるので、次のように書き換えられる
    \begin{equation*}
      (A\vb*{x}, \vb*{x}) = \vb*{y}^* D \vb*{y} = (D\vb*{y}, \vb*{y})
    \end{equation*}

    \br

    左辺の内積を計算すると、
    \begin{align*}
      (D\vb*{y}, \vb*{y}) & = \begin{pmatrix}
                                \lambda_1 y_1 &        &               \\
                                              & \ddots &               \\
                                              &        & \lambda_n y_n
                              \end{pmatrix} \begin{pmatrix}
                                              y_1    \\
                                              \vdots \\
                                              y_n
                                            \end{pmatrix}         \\
                          & = \lambda_1 |y_1|^2 + \cdots + \lambda_n |y_n|^2
    \end{align*}
    $\lambda_1,\ldots,\lambda_n > 0$であることから、すべての項が正になるので、
    \begin{equation*}
      (A\vb*{x}, \vb*{x}) = (D\vb*{y}, \vb*{y}) > 0
    \end{equation*}
    よって、$A$は正定値行列である $\qed$
  \end{subpattern}
\end{proof}

\br

半正定値行列は、正定値行列の条件に等号を含むようにしたものである

\begin{definition}{半正定値行列}
  $A$をエルミート行列(対称行列)とし、任意のベクトル$\vb*{x} \in \mathbb{C}^n$($\vb*{x} \in \mathbb{R}^n$)に対して、
  \begin{equation*}
    (A\vb*{x}, \vb*{x}) \geq 0
  \end{equation*}
  が成り立つとき、$A$は\keyword{半正定値行列}であるという
\end{definition}

\begin{theorem}{半正定値性と固有値の非負実性}
  エルミート行列$A$が半正定値行列であることと、$A$のすべての固有値が非負の実数であることは同値である
\end{theorem}

\sectionline
\section{対称行列を構成する行列積}
\marginnote{\refbookI p29}

\hyperref[thm:spectral-decomposition-symmetric]{スペクトル分解}は対称行列に対するものだったが、これを任意の長方行列に拡張したものが\keyword{特異値分解}である

\br

対称行列から任意の行列へ議論を拡張するにあたって、次の定理が重要となる

\begin{theorem}{自身の随伴行列との積で構成されるエルミート行列}
  $A$を任意の複素行列(長方行列)とするとき、$A^* A$および$AA^*$はエルミート行列である
\end{theorem}

\begin{proof}
  \hyperref[thm:adjoint-of-product]{積をエルミート行列にすると順序が入れ替わる}ことに注意して、
  \begin{equation*}
    (A^* A)^* = A^* (A^*)^* = A^* A
  \end{equation*}
  よって、$A^* A$はエルミート行列である

  \br

  同様に、
  \begin{equation*}
    (AA^*)^* = (A^*)^* A^* = AA^*
  \end{equation*}
  よって、$AA^*$もエルミート行列である $\qed$
\end{proof}

\br

$A$を実行列とすれば、次が成り立つ

\begin{theorem}{自身の転置行列との積で構成される対称行列}\label{thm:symmetric-products-of-any-matrix}
  $A$を任意の実行列(長方行列)とするとき、$A^\top A$および$AA^\top$は対称行列である
\end{theorem}

\sectionline

$A^* A$および$AA^*$という形の行列には、さらに固有値に関する重要な性質がある

\begin{theorem}{自身の随伴行列との積で構成される半正値行列}
  任意の行列$A$に対して、$AA^*$および$A^*A$はともに半正値行列である
\end{theorem}

\begin{proof}
  エルミート行列$AA^*$の固有ベクトルを$\vb*{u}$とし、その固有値を$\lambda \in \mathbb{C}$とすると、
  \begin{equation*}
    AA^* \vb*{u} = \lambda \vb*{u}
  \end{equation*}
  両辺で$\vb*{u}$との内積をとると、
  \begin{equation*}
    (\vb*{u}, AA^* \vb*{u}) = \lambda (\vb*{u}, \vb*{u}) = \lambda \|\vb*{u}\|^2
  \end{equation*}
  この左辺は、\hyperref[thm:adjoint-identity]{随伴公式}を用いて、
  \begin{equation*}
    \begin{WithArrows}
      (\vb*{u}, AA^* \vb*{u}) & = (\vb*{u}, A(A^*\vb*{u})) \Arrow{外側の$A$に\\ 随伴公式を適用}  \\
      & = (A^* \vb*{u}, A^* \vb*{u}) \\
      & = \|A^* \vb*{u}\|^2 \geq 0
    \end{WithArrows}
  \end{equation*}
  となるので、
  \begin{equation*}
    \|A^* \vb*{u}\|^2 = \lambda \|\vb*{u}\|^2 \geq 0
  \end{equation*}
  ここで、固有ベクトルは零ベクトルではないので、$\|\vb*{u}\|^2 >0$である

  よって、$\lambda \|\vb*{u}\|^2 \geq 0$の両辺を$\|\vb*{u}\|^2$で割ることにより、
  \begin{equation*}
    \lambda \geq 0
  \end{equation*}
  が得られる

  \br

  $A^*A$についても同様に、
  \begin{equation*}
    (\vb*{u}, A^*A \vb*{u}) = (A\vb*{u}, A\vb*{u}) = \|A\vb*{u}\|^2 \geq 0
  \end{equation*}
  から、$\lambda \geq 0$が得られる $\qed$
\end{proof}

\br

\begin{theorem}{特異値と左右特異ベクトルの対応関係}\label{thm:svd-singular-value-vector-correspondence}
  $A$を$O$でない任意の行列とするとき、$A^\top A$と$AA^\top$は共通の正の固有値$\sigma^2$を持ち、それぞれの固有ベクトル$\vb*{u},\vb*{v}$は次の関係を満たす
  \begin{equation*}
    A \vb*{v} = \sigma \vb*{u}, \quad A^\top \vb*{u} = \sigma \vb*{v}
  \end{equation*}
\end{theorem}

\begin{proof}
  \begin{subpattern}{\bfseries $AA^\top$の固有値が$\sigma^2$と仮定した場合}
    $AA^\top$の固有値が非負の固有値$\sigma^2$を持ち、対応する固有ベクトルが$\vb*{u}$であるとすると、
    \begin{equation*}
      AA^\top \vb*{u} = \sigma^2 \vb*{u}
    \end{equation*}
    この両辺に左から$A^\top$をかけて、
    \begin{equation*}
      A^\top AA^\top \vb*{u} = A^\top \sigma^2 \vb*{u}
    \end{equation*}
    ここで、$\vb*{v} = \dfrac{A^\top\vb*{u}}{\sigma}$とおくと、$A^\top \vb*{u} = \sigma \vb*{v}$となるので、
    \begin{align*}
      A^\top A\sigma\vb*{v} & = \sigma^3 \vb*{v} \\
      A^\top A \vb*{v}      & = \sigma^2 \vb*{v}
    \end{align*}
    よって、$\sigma^2$は$A^\top A$の固有値でもあり、対応する固有ベクトル$\vb*{v}$は
    \begin{equation*}
      A^\top \vb*{u} = \sigma \vb*{v}
    \end{equation*}
    を満たす $\qed$
  \end{subpattern}

  \begin{subpattern}{\bfseries $A^\top A$の固有値が$\sigma^2$と仮定した場合}
    $A^\top A$の固有値が非負の固有値$\sigma^2$を持ち、対応する固有ベクトルが$\vb*{v}$であるとすると、
    \begin{equation*}
      A^\top A \vb*{v} = \sigma^2 \vb*{v}
    \end{equation*}
    この両辺に左から$A$をかけて、
    \begin{equation*}
      AA^\top A \vb*{v} = A\sigma^2 \vb*{v}
    \end{equation*}
    ここで、$\vb*{u} = \dfrac{A\vb*{v}}{\sigma}$とおくと、$A\vb*{v} = \sigma \vb*{u}$となるので、
    \begin{align*}
      AA^\top\sigma\vb*{u} & = \sigma^3 \vb*{u} \\
      AA^\top \vb*{u}      & = \sigma^2 \vb*{u}
    \end{align*}
    よって、$\sigma^2$は$AA^\top$の固有値でもあり、対応する固有ベクトル$\vb*{u}$は
    \begin{equation*}
      A\vb*{v} = \sigma \vb*{u}
    \end{equation*}
    を満たす $\qed$
  \end{subpattern}
\end{proof}

\end{document}
