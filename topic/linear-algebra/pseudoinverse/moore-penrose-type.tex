\documentclass[../../../topic_linear-algebra]{subfiles}

\begin{document}

\sectionline
\section{ムーア・ペンローズ型一般逆行列}
\marginnote{\refbookI p37}

$O$でない$m \times n$型行列$A$が次のように特異値分解されているとする。
\begin{equation*}
  A = \sigma_1\vb*{u}_1\vb*{v}_1^\top + \cdots + \sigma_r \vb*{u}_r \vb*{v}_r^\top
\end{equation*}

このとき、次の行列を$A$の\keyword{ムーア・ペンローズ型一般逆行列}と定義する。
\begin{equation*}
  A^- = \frac{\vb*{v}_1\vb*{u}_1^\top}{\sigma_1} + \cdots + \frac{\vb*{v}_r\vb*{u}_r^\top}{\sigma_r}
\end{equation*}
$\sigma_i$が逆数に、$\vb*{u}_i\vb*{v}_i^\top$が$\vb*{v}_i\vb*{u}_i^\top$に置き換わっていることに注意しよう。

\br

\hyperref[sec:matrix-form-svd]{特異値分解の行列表記}を用いると、ムーア・ペンローズ型一般逆行列は次のように書ける。
\begin{equation*}
  A^- = V \begin{pmatrix}
                        \dfrac{1}{\sigma_1} &        &           \\
                                  & \ddots &           \\
                                  &        & \dfrac{1}{\sigma_r}
                      \end{pmatrix} U^\top
\end{equation*}

ここで、$U$と$V$は次のように定義している。
\begin{equation*}
  U = \begin{pmatrix}
    \vb*{u}_1 & \cdots & \vb*{u}_r
  \end{pmatrix}, \quad
  V = \begin{pmatrix}
    \vb*{v}_1 & \cdots & \vb*{v}_r
  \end{pmatrix}
\end{equation*}

\br

\todo{$A$が正則なら逆行列と一致}

\end{document}
