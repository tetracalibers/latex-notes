\documentclass[../../../topic_linear-algebra]{subfiles}

\begin{document}

\sectionline
\section{ムーア・ペンローズの一般逆行列}
\marginnote{\refbookI p37}

$O$でない$m \times n$型行列$A$が次のように特異値分解されているとする。
\begin{equation*}
  A = \sigma_1\vb*{u}_1\vb*{v}_1^\top + \cdots + \sigma_r \vb*{u}_r \vb*{v}_r^\top
\end{equation*}

このとき、次の行列を$A$の\keyword{ムーア・ペンローズの一般逆行列}と定義する。
\begin{equation*}
  A^+ = \frac{\vb*{v}_1\vb*{u}_1^\top}{\sigma_1} + \cdots + \frac{\vb*{v}_r\vb*{u}_r^\top}{\sigma_r}
\end{equation*}
$\sigma_i$が逆数に、$\vb*{u}_i\vb*{v}_i^\top$が$\vb*{v}_i\vb*{u}_i^\top$に置き換わっていることに注意しよう。

\br

\hyperref[sec:matrix-form-svd]{特異値分解の行列表記}を用いると、ムーア・ペンローズの一般逆行列は次のように書ける。
\begin{equation*}
  A^+ = V \begin{pmatrix}
                        \dfrac{1}{\sigma_1} &        &           \\
                                  & \ddots &           \\
                                  &        & \dfrac{1}{\sigma_r}
                      \end{pmatrix} U^\top
\end{equation*}

ここで、$U$と$V$は次のように定義している。
\begin{equation*}
  U = \begin{pmatrix}
    \vb*{u}_1 & \cdots & \vb*{u}_r
  \end{pmatrix}, \quad
  V = \begin{pmatrix}
    \vb*{v}_1 & \cdots & \vb*{v}_r
  \end{pmatrix}
\end{equation*}

\subsection{$A$が正則行列の場合}

$A$が正則行列の場合、$A$のムーア・ペンローズの一般逆行列は、$A$の逆行列に一致する。

すなわち、$A$が正則の場合は、次が成り立つ。
\begin{equation*}
  A^+ A = A A^+ = E
\end{equation*}

この意味で、ムーア・ペンローズの一般逆行列は逆行列の一般化とみなせる。

\begin{theorem}{正則行列に対するムーア・ペンローズ逆行列}
  $A$が正則行列であれば、ムーア・ペンローズの一般逆行列$A^+$は逆行列$A^{-1}$に一致する。
\end{theorem}

\begin{proof}
  $A$が正則行列であることから、$m = n$であり、\hyperref[thm:invertible-iff-full-rank]{$\rank A = n$}である。
  
  \hyperref[thm:rank-and-singular-values]{行列の階数は特異値の個数に等しい}ので、$A$は$n$個の特異値を用いて次のように特異値分解できる。
  \begin{equation*}
    A = \sum_{i=1}^{n} \sigma_i \vb*{u}_i \vb*{v}_i^\top
  \end{equation*}
  このとき、$A$のムーア・ペンローズの一般逆行列は次のように定義される。
  \begin{equation*}
    A^+ = \sum_{i=1}^{n} \frac{1}{\sigma_i} \vb*{v}_i \vb*{u}_i^\top
  \end{equation*}
  
  これらの積を計算すると、
  \begin{align*}
    A^+ A & = \left( \sum_{i=1}^{n} \frac{\vb*{v}_i \vb*{u}_i^\top}{\sigma_i} \right) \left( \sum_{j=1}^{n} \sigma_j \vb*{u}_j \vb*{v}_j^\top \right) \\
           & = \sum_{i=1}^{n} \sum_{j=1}^{n} \frac{\sigma_i}{\sigma_j} \vb*{v}_i \vb*{u}_i^\top \vb*{u}_j \vb*{v}_j^\top \\
           & = \sum_{i=1}^{n} \sum_{j=1}^{n} \frac{\sigma_i}{\sigma_j} \vb*{v}_i (\vb*{u}_i^\top \vb*{u}_j) \vb*{v}_j^\top
  \end{align*}
  ここで、特異ベクトル$\vb*{u}_i, \vb*{v}_i$はそれぞれ$\mathbb{R}^n$の正規直交基底を成すので、
  \begin{equation*}
    \vb*{u}_i^\top \vb*{u}_j = \delta_{ij}, \quad
    \vb*{v}_i^\top \vb*{v}_j = \delta_{ij}
  \end{equation*}
  よって、$A^+A$の式は次のように簡約される。
  \begin{align*}
    A^+ A & = \sum_{i=1}^{n} \sum_{j=1}^{n} \frac{\sigma_i}{\sigma_j} \vb*{v}_i (\vb*{u}_i^\top \vb*{u}_j) \vb*{v}_j^\top \\
          & = \sum_{i=1}^{n} \sum_{j=1}^{n} \frac{\sigma_i}{\sigma_j} \delta_{ij} \vb*{v}_i \vb*{v}_j^\top \\
          & = \sum_{i=1}^{n} \frac{\sigma_i}{\sigma_i} \vb*{v}_i \vb*{v}_i^\top \\
          & = \sum_{i=1}^{n} \vb*{v}_i \vb*{v}_i^\top \\
          & = E
  \end{align*}
  $A A^+$も同様に計算することができるため、$A^+$は$A$の逆行列$A^{-1}$に一致する。 $\qed$
\end{proof}

\subsection{$A$が正則でない場合}

$A$が正則でない場合は、$A^+A$や$AA^+$を計算しても単位行列にはならない。
一般に、次の式が成り立つことが知られている。
\begin{equation*}
  A^+ A = P_{\mathcal{V}}, \quad A A^+ = P_{\mathcal{U}}
\end{equation*}

\begin{theorem}{todo}
  $A^+$をムーア・ペンローズの一般逆行列とするとき、$A^+ A$は行空間$\mathcal{V}$への射影行列、$AA^+$は列空間$\mathcal{U}$への射影行列となる。
\end{theorem}

\begin{proof}
  \todo{}
  
  $A$を$m \times n$型行列とする。
  
  \begin{subpattern}{$A^+ A = P_{\mathcal{V}}$}
    \begin{align*}
      A^+ A & = \left( \sum_{i=1}^{r} \frac{\vb*{v}_i \vb*{u}_i^\top}{\sigma_i} \right) \left( \sum_{j=1}^{r} \sigma_j \vb*{u}_j \vb*{v}_j^\top \right) \\
            & = \sum_{i=1}^{r} \sum_{j=1}^{r} \frac{\sigma_i}{\sigma_j} \vb*{v}_i (\vb*{u}_i^\top \vb*{u}_j) \vb*{v}_j^\top \\
    \end{align*}
  \end{subpattern}
\end{proof}

\end{document}
