\documentclass[../../../topic_linear-algebra]{subfiles}

\begin{document}

\sectionline
\section{ムーア・ペンローズの擬似逆行列}
\marginnote{\refbookI p36〜37}

正方行列は、それが正則であれば逆行列を持つ。

これを$O$でない任意の長方行列に拡張するのが\keyword{一般逆行列}(\keyword{擬似逆行列})である。

\begin{theorem}{擬似逆行列の存在と一意性}
  $O$でない任意の$m \times n$型行列$A$に対して、以下の4つの条件を満たす$n \times m$型行列$A^+$がただ一つ存在する。
  \begin{enumerate}[label=\romanlabel]
    \item $AA^+A = A$
    \item $A^+AA^+ = A^+$
    \item $(AA^+)^\top = AA^+$
    \item $(A^+A)^\top = A^+A$
  \end{enumerate}
  この$A^+$を\keyword{ムーア・ペンローズの擬似逆行列}という。
\end{theorem}

\begin{proof}[\bfseries 存在性の証明]
    $A$の特異値分解を考える。
    \begin{gather*}
      \Sigma_r = \begin{pmatrix}
        \sigma_1 &        &           \\
                  & \ddots &           \\
                  &        & \sigma_r
      \end{pmatrix}, \quad
      \Sigma^{-1}_r = \begin{pmatrix}
        \dfrac{1}{\sigma_1} &        &           \\
                            & \ddots &           \\
                            &        & \dfrac{1}{\sigma_r}
      \end{pmatrix} \\
      \Sigma = \begin{pmatrix}
        \Sigma_r & O \\
        O & O
      \end{pmatrix}, \quad
      \Sigma^{-1} = \begin{pmatrix}
        \Sigma^{-1}_r & O \\
        O & O
      \end{pmatrix}
    \end{gather*}
    とおくと、$A$は次のように特異値分解できる。
    \begin{equation*}
      A = U \Sigma V^\top
    \end{equation*}
        
    ここで、
    \begin{equation*}
      B = V \Sigma^{-1} U^\top
    \end{equation*}
    とおくと、$B$は次のように4つの条件を満たす。
  
    \begin{subpattern}{(\romannum{i}) $AA^+A = A$}
      $U,V$はユニタリ行列であるから、$V^\top V = E$および$U^\top U = E$が成り立つ。
      \begin{align*}
        ABA &= (U \Sigma V^\top)(V \Sigma^{-1} U^\top)(U \Sigma V^\top) \\
             &= U \Sigma (V^\top V) \Sigma^{-1} (U^\top U) \Sigma V^\top \\
             &= U(\Sigma \Sigma^{-1}) \Sigma V^\top
      \end{align*}
      
      \br
      
      ここで、
      \begin{equation*}
        \Sigma \Sigma^{-1} = \begin{pmatrix}
          E_r & O \\
          O & O
        \end{pmatrix}
      \end{equation*}
      であり、これに$\Sigma$をかけると、単位行列$E_r$の部分が$\Sigma_r$に置き換わるだけとなるので、
      \begin{equation*}
        (\Sigma \Sigma^{-1}) \Sigma = \Sigma
      \end{equation*}
      が成り立つ。
      
      \br
      
      この関係を用いると、
      \begin{align*}
        ABA &= U(\Sigma \Sigma^{-1}) \Sigma V^\top \\
              &= U \Sigma V^\top \\
              &= A
      \end{align*}
      となり、条件(\romannum{i})が成り立つ。$\qed$
    \end{subpattern}
    
    \begin{subpattern}{(\romannum{ii}) $A^+AA^+ = A^+$}
      同様に、
      \begin{align*}
        B A B &= (V \Sigma^{-1} U^\top)(U \Sigma V^\top)(V \Sigma^{-1} U^\top) \\
              &= V \Sigma^{-1} (U^\top U) \Sigma (V^\top V) \Sigma^{-1} U^\top \\
              &= V(\Sigma^{-1} \Sigma) \Sigma^{-1} U^\top \\
              &= V \Sigma^{-1} U^\top \\
              &= B
      \end{align*}
      となり、条件(\romannum{ii})が成り立つ。$\qed$
    \end{subpattern}
    
    \begin{subpattern}{(\romannum{iii}) $(AA^+)^\top = AA^+$}
      まず$AB$を計算すると、
      \begin{align*}
        AB & = (U \Sigma V^\top)(V \Sigma^{-1} U^\top) \\
           & = U \Sigma (V^\top V) \Sigma^{-1} U^\top \\
           & = U(\Sigma \Sigma^{-1}) U^\top 
      \end{align*}
      よって、$(AB)^\top$は、
      \begin{align*}
        (AB)^\top & = (U(\Sigma \Sigma^{-1}) U^\top)^\top \\
                  & = U (\Sigma \Sigma^{-1}) U^\top \\
                  & = AB
      \end{align*}
      となり、条件(\romannum{iii})が成り立つ。$\qed$
    \end{subpattern}
    
    \begin{subpattern}{(\romannum{iv}) $(A^+A)^\top = A^+A$}
      まず$BA$を計算すると、
      \begin{align*}
        BA & = (V \Sigma^{-1} U^\top)(U \Sigma V^\top) \\
           & = V \Sigma^{-1} (U^\top U) \Sigma (V^\top V) \\
           & = V(\Sigma^{-1} \Sigma) V^\top
      \end{align*}
      よって、$(BA)^\top$は、
      \begin{align*}
        (BA)^\top & = (V(\Sigma^{-1} \Sigma) V^\top)^\top \\
                  & = V (\Sigma^{-1} \Sigma) V^\top \\
                  & = BA
      \end{align*}
      となり、条件(\romannum{iv})が成り立つ。$\qed$
    \end{subpattern}
\end{proof}

\subsection{ムーア・ペンローズの擬似逆行列の構成}

存在性の証明過程から、ムーア・ペンローズの擬似逆行列は次のように構成すればよいことがわかる。

\begin{definition}{ムーア・ペンローズの擬似逆行列}
  行列$A$のムーア・ペンローズの擬似逆行列$A^+$は、次のように定義される。
  \begin{equation*}
    A^+ = V \Sigma^{-1} U^\top = V \begin{pmatrix}
                        \dfrac{1}{\sigma_1} &        &           \\
                                  & \ddots &           \\
                                  &        & \dfrac{1}{\sigma_r}
                      \end{pmatrix} U^\top
  \end{equation*}
\end{definition}

\subsection{ムーア・ペンローズの擬似逆行列の一意性}

行列$A$に対して$A^+$が一意的に定まることは、次のように示される。

\br

\begin{proof}[一意性の証明]
  $B_1, B_2$がいずれもムーア・ペンローズの擬似逆行列の4つの条件
  \begin{enumerate}[label=\romanlabel]
    \item $ABA = A$
    \item $BAB = B$
    \item $(AB)^\top = AB$
    \item $(BA)^\top = BA$
  \end{enumerate}
  を満たすとすると、
  \begin{equation*}
    \begin{WithArrows}
      B_1 &\overset{\romannum{ii}}{=} B_1 A B_1 \Arrow{$B_1 A = B_1$} \\
        &\overset{\romannum{iv}}{=} (B_1 A)^\top B_1 \Arrow{$AB_2A = A$} \\
        &\overset{\romannum{ii}}{=} (B_1 A B_2 A)^\top B_1 \Arrow{積の転置の性質} \\
        &\overset{\romannum{iii}}{=} (B_2 A)^\top (B_1 A)^\top B_1 \\
        &\overset{\romannum{iv}}{=} (B_2 A)(B_1 A) B_1 \Arrow{$AB_1A = A$} \\
        &\overset{\romannum{i}}{=} B_2 A B_1
    \end{WithArrows}
  \end{equation*}
  同様の計算により、$B_2 = B_2 A B_1$も得られる。
  
  よって、$B_1 = B_2$である。$\qed$
\end{proof}

\subsection{特異値分解の展開式による表記}

$O$でない$m \times n$型行列$A$が次のように特異値分解されているとする。
\begin{equation*}
  A = \sigma_1\vb*{u}_1\vb*{v}_1^\top + \cdots + \sigma_r \vb*{u}_r \vb*{v}_r^\top
\end{equation*}

このとき、ムーア・ペンローズの擬似逆行列$A^+$は次のように表される。
\begin{equation*}
  A^+ = \frac{\vb*{v}_1\vb*{u}_1^\top}{\sigma_1} + \cdots + \frac{\vb*{v}_r\vb*{u}_r^\top}{\sigma_r}
\end{equation*}
$\sigma_i$が逆数に、$\vb*{u}_i\vb*{v}_i^\top$が$\vb*{v}_i\vb*{u}_i^\top$に置き換わっていることに注意しよう。

\end{document}
