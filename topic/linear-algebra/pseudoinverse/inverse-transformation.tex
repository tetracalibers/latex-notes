\documentclass[../../../topic_linear-algebra]{subfiles}

\usepackage{xr-hyper}
\externaldocument{../../../.tex_intermediates/topic_linear-algebra}

\begin{document}

\sectionline
\section{逆変換としての擬似逆行列}
\marginnote{\refbookI p38〜39}

ムーア・ペンローズの擬似逆行列は、線形変換の逆変換という視点でも、逆行列の一般化となっている。

\subsection{逆変換と恒等変換}

逆行列$A^{-1}$は、線形変換$A$の\keyword{逆変換}を表すものだった。

逆変換は「元に戻す」操作である。

\br

変換$A$と逆変換$A^{-1}$を合成すると、「なにもしない」という変換(\keyword{恒等変換})が得られる。

$A$してからそれを$A^{-1}$で打ち消すのだから、結局なにもしなかったことになるのである。

\br

このことを数式で表したものが、逆行列$A^{-1}$の定義式である。
\begin{equation*}
  A^{-1} A = A A^{-1} = E
\end{equation*}
単位行列$E$は、恒等変換を表す行列である。

\subsection{列空間上の逆変換}

$\vb*{x}$が列空間$\mathcal{U}$の元である場合は、\secref{sec:projection-onto-subspace}で述べたように、列空間へ射影しても変わらないので、
\begin{equation*}
  P_{\mathcal{U}} \vb*{x} = \vb*{x}
\end{equation*}
が成り立つ。

つまり、列空間$\mathcal{U}$においては、$P_{\mathcal{U}}$は「なにもしない」恒等変換を表す。

\br

このことから、
\begin{equation*}
  AA^+ = P_{\mathcal{U}}
\end{equation*}
という式は、
\begin{emphabox}
  \begin{spacebox}
    \begin{center}
      列空間$\mathcal{U}$において、$A^+$は$A$の\keyword{逆変換}を表す
    \end{center}
  \end{spacebox}
\end{emphabox}
と解釈できる。

\subsection{行空間上の逆変換}

同様に、$\vb*{x}$が行空間$\mathcal{V}$の元である場合は、\secref{sec:projection-onto-subspace}で述べたように、行空間へ射影しても変わらないので、
\begin{equation*}
  P_{\mathcal{V}} \vb*{x} = \vb*{x}
\end{equation*}
が成り立つ。

つまり、行空間$\mathcal{V}$においては、$P_{\mathcal{V}}$は「なにもしない」恒等変換を表す。

\br

このことから、
\begin{equation*}
  A^+ A = P_{\mathcal{V}}
\end{equation*}
という式は、
\begin{emphabox}
  \begin{spacebox}
    \begin{center}
      行空間$\mathcal{V}$において、$A^+$は$A$の\keyword{逆変換}を表す
    \end{center}
  \end{spacebox}
\end{emphabox}
と解釈できる。

\end{document}
