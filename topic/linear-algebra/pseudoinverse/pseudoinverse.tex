\documentclass[../../../topic_linear-algebra]{subfiles}

\begin{document}

\sectionline
\section{一般逆行列:逆行列の拡張}
\marginnote{\refbookI p36}

正方行列は、それが正則であれば逆行列を持つ。

これを$O$でない任意の長方行列に拡張するのが\keyword{一般逆行列}である。

\br

逆行列は、もとの行列との積が単位行列となるものとして定義される。

一方、一般逆行列ともとの行列との積は、列および行の張る空間への射影行列となる。

\br

正則行列はすべての列や行が線型独立であるため、それらは全空間を張る。

つまり、単位行列は全空間への射影行列である。

\br

この意味で、一般逆行列は逆行列の自然な拡張になっている。

\end{document}
