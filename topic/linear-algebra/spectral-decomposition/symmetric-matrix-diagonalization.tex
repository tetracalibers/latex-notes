\documentclass[../../../topic_linear-algebra]{subfiles}

\usepackage{xr-hyper}
\externaldocument{../../../.tex_intermediates/topic_linear-algebra}

\begin{document}

\sectionline
\section{対称行列のランクと固有値}
\marginnote{\refbookI p19}

$n$次対称行列$A$の列の任意の線形結合
\begin{equation*}
  c_1\vb*{a}_1 + \cdots + c_n\vb*{a}_n = \begin{pmatrix}
    \vb*{a}_1 & \cdots & \vb*{a}_n
  \end{pmatrix} \begin{pmatrix}
    c_1    \\
    \vdots \\
    c_n
  \end{pmatrix} = A\vb*{c}
\end{equation*}
を考える

\br

$A$の$n$個の固有値のうち、0でないものの個数を$r$とすれば、\thmref{thm:spectral-decomposition-symmetric}による$A$のスペクトル分解の式において、$\lambda_{r+1}, \ldots, \lambda_n = 0$とおいて、
\begin{align*}
  A\vb*{c} & = \lambda_1\vb*{u}_1\vb*{u}_1^\top\vb*{c} + \cdots + \lambda_r\vb*{u}_r\vb*{u}_r^\top\vb*{c}     \\
           & = \lambda_1(\vb*{u}_1^\top\vb*{c})\vb*{u}_1 + \cdots + \lambda_r(\vb*{u}_r^\top\vb*{c})\vb*{u}_r
\end{align*}
すなわち、$A$の列の任意の線形結合は、互いに直交する$\vb*{u}_1, \ldots, \vb*{u}_r$の線形結合で書ける

\br

\thmref{thm:orthogonal-set-is-independent}より、互いに直交するベクトルは線型独立であることから、
\begin{itemize}
  \item $\vb*{a}_1, \ldots, \vb*{a}_n$の張る部分空間(線形結合の集合)の次元は$r$である
  \item $n$本の列のうち、$r$本しか線型独立ではない
\end{itemize}
ということがいえる

\br

\thmref{thm:rank-equals-max-indep-cols}より、行列$A$の$n$本の列のうち、線型独立なものの個数を$A$の\keyword{ランク}あるいは\keyword{階数}というので、次のことがいえる

\begin{theorem}{対称行列における階数と非零固有値の個数}{rank-nonzero-eigenvalues-symmetric}
  $A$を対称行列とするとき、$\rank(A)$は、$A$の非零の固有値の個数に等しい
\end{theorem}

$A$は対称行列であるから、行についても同じことがいえる

\sectionline
\section{スペクトル分解による対称行列の対角化}
\marginnote{\refbookI p19〜20}

スペクトル分解の式を用いることで、対称行列の対角化について簡潔に議論できるようになる

\br

対称行列$A$のスペクトル分解の式
\begin{equation*}
  A = \lambda_1 \vb*{u}_1 \vb*{u}_1^\top + \cdots + \lambda_n \vb*{u}_n \vb*{u}_n^\top
\end{equation*}
は、次のように書き換えられる
\begin{align*}
  A & = \begin{pmatrix}
          \lambda_1\vb*{u}_1 & \cdots & \lambda_n\vb*{u}_n
        \end{pmatrix} \begin{pmatrix}
                        \vb*{u}_1^\top \\
                        \vdots         \\
                        \vb*{u}_n^\top
                      \end{pmatrix} \\
    & = \begin{pmatrix}
          \vb*{u}_1 & \cdots & \vb*{u}_n
        \end{pmatrix} \begin{pmatrix}
                        \lambda_1 &        &           \\
                                  & \ddots &           \\
                                  &        & \lambda_n
                      \end{pmatrix} \begin{pmatrix}
                                      \vb*{u}_1^\top \\
                                      \vdots         \\
                                      \vb*{u}_n^\top
                                    \end{pmatrix}     \\
    & = U \begin{pmatrix}
            \lambda_1 &        &           \\
                      & \ddots &           \\
                      &        & \lambda_n
          \end{pmatrix} U^\top
\end{align*}

ここで、
\begin{equation*}
  U = \begin{pmatrix}
    \vb*{u}_1 & \cdots & \vb*{u}_n
  \end{pmatrix}
\end{equation*}
は、\thmref{thm:unitary-iff-columns-orthonormal}より、列が正規直交系をなすことから\keyword{直交行列}である

\br

そして、$U$が直交行列であれば、\thmref{thm:transpose-of-orthogonal}より、その転置$U^\top$も直交行列である

それゆえ、直交行列の行も正規直交系をなす

\br

$A$の式の両辺に左から$U^\top$、右から$U$をかけると、
\begin{equation*}
  U^\top A U  = U^\top U \begin{pmatrix}
    \lambda_1 &        &           \\
              & \ddots &           \\
              &        & \lambda_n
  \end{pmatrix} U^\top U
\end{equation*}
直交行列の定義$U^\top U = E$より、
\begin{equation*}
  U^\top A U =\begin{pmatrix}
    \lambda_1 &        &           \\
              & \ddots &           \\
              &        & \lambda_n
  \end{pmatrix}
\end{equation*}
として、対称行列$A$は、直交行列$U$によって対角化できることがわかる

\end{document}
