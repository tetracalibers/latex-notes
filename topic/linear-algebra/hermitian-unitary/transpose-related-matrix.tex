\documentclass[../../../topic_linear-algebra]{subfiles}

\begin{document}

\sectionline
\section{転置行列と随伴行列}
\marginnote{\refbookF p275}

複素正方行列$A$の転置行列において、各成分をその共役複素数に置き換えた行列を\keyword{随伴行列}という

\begin{definition}{随伴行列}
  複素正方行列$A = (a_{ij})$に対し、$\overline{a_{ji}}$を$(i,j)$成分にもつ行列${}^t\overline{A}$を$A$の\keyword{随伴行列}といい、$A^*$と表す
\end{definition}

実数$x$の複素共役は$\overline{x} = x$であるので、$A$が実行列のときは、
\begin{equation*}
  A^* = {}^t A
\end{equation*}
すなわち、
\begin{shaded}
  実行列の世界では、随伴行列は転置行列
\end{shaded}
にすぎない

\sectionline

転置行列と複素共役の性質から、次の性質が成り立つ

\begin{theorem}{積に対するエルミート共役の順序反転性}
  複素行列$A\,B$の積$AB$が定義できるとき、
  \begin{equation*}
    (AB)^* = B^* A^*
  \end{equation*}
\end{theorem}

\begin{proof}
  \todo{}
\end{proof}

\sectionline
\section{対称行列とエルミート行列}
\marginnote{\refbookF p275〜276}

\begin{definition}{エルミート行列}
  複素正方行列$A$が次を満たすとき、$A$を\keyword{エルミート行列}という
  \begin{equation*}
    A^* = A
  \end{equation*}
\end{definition}

$A$が実正方行列のときは、
\begin{equation*}
  A\text{がエルミート行列} \Longleftrightarrow {}^tA = A
\end{equation*}
となり、このような$A$は\hyperref[def:symmetric-matrix]{対称行列}、あるいは\keyword{実対称行列}と呼ばれる

\sectionline
\section{直交行列とユニタリ行列}
\marginnote{\refbookF p275〜276}

\begin{definition}{ユニタリ行列}
  複素正方行列$A$が次を満たすとき、$A$を\keyword{ユニタリ行列}という
  \begin{equation*}
    A^* = A^{-1}
  \end{equation*}
\end{definition}

$A$が実正方行列のときは、
\begin{equation*}
  A\text{がユニタリ行列} \Longleftrightarrow {}^tA = A^{-1}
\end{equation*}
となり、このような$A$は\keyword{直交行列}と呼ばれる

\begin{definition}{直交行列}
  実正方行列$A$が次を満たすとき、$A$を\keyword{直交行列}という
  \begin{equation*}
    {}^t A = A^{-1}
  \end{equation*}
\end{definition}

直交行列という名前の由来は、次のように考えられる

\br

$A$を$n$個の列ベクトルを横一列に並べたものとみなし、
\begin{equation*}
  A = (\vb*{a}_1, \vb*{a}_2, \ldots, \vb*{a}_n)
\end{equation*}
とおくと、${}^t A = A^{-1}$、すなわち${}^tAA = E$は、
\begin{equation*}
  \begin{pmatrix}
    {}^t \vb*{a}_1 \\
    {}^t \vb*{a}_2 \\
    \vdots         \\
    {}^t \vb*{a}_n
  \end{pmatrix} \left(
  \vb*{a}_1, \vb*{a}_2, \ldots, \vb*{a}_n
  \right) = \begin{pmatrix}
    1      & 0      & \cdots & 0      \\
    0      & 1      & \cdots & 0      \\
    \vdots & \vdots & \ddots & \vdots \\
    0      & 0      & \cdots & 1
  \end{pmatrix}
\end{equation*}
と表される

\br

これは、ベクトル$\vb*{a}_1, \vb*{a}_2, \ldots, \vb*{a}_n$が、次の性質
\begin{equation*}
  {}^t \vb*{a}_i \vb*{a}_j =(\vb*{a}_i, \vb*{a}_j) = \delta_{ij}
\end{equation*}
を満たすことを意味する

\br

すなわち、直交行列$A$の列ベクトル$\vb*{a}_1, \vb*{a}_2, \ldots, \vb*{a}_n$は、互いに直交する単位ベクトルである

\end{document}
