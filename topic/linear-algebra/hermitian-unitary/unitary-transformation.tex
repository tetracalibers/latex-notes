\documentclass[../../../topic_linear-algebra]{subfiles}

\begin{document}

\sectionline
\section{ユニタリ変換}
\marginnote{\refbookA p77〜82 \\ \refbookC p126〜131}

体$\mathbb{C}$上の計量空間において、内積を保つ線形変換を\keyword{ユニタリ変換}という

\begin{definition}{ユニタリ変換}
  体$\mathbb{C}$上の計量空間$V$における線形変換$f$が\keyword{ユニタリ変換}であるとは、任意の$\vb*{u},\,\vb*{v} \in V$に対し、
  \begin{equation*}
    (f(\vb*{u}),f(\vb*{v})) = (\vb*{u},\vb*{v})
  \end{equation*}
  が成り立つことである
\end{definition}

体$\mathbb{R}$上のユニタリ変換は、\keyword{直交変換}と呼ばれる

\subsection{ユニタリ変換とノルム}

ユニタリ変換は、ベクトルの長さを変えない変換でもある

\begin{theorem}{ユニタリ変換とノルム保存性}
  計量空間$V$における線形変換を$f$がユニタリ変換であることと、任意の$\vb*{v} \in V$に対し
  \begin{equation*}
    \|f(\vb*{v})\| = \|\vb*{v}\|
  \end{equation*}
  が成り立つことは同値である
\end{theorem}

\begin{proof}
  \begin{subpattern}{\bfseries $f$がユニタリ変換$\Longrightarrow$ $f$はノルムを保つ}
    ユニタリ変換の定義より、
    \begin{equation*}
      (f(\vb*{v}),f(\vb*{v})) = (\vb*{v},\vb*{v}) = \|\vb*{v}\|^2
    \end{equation*}
    ここで、$\|f(\vb*{v})\| = \sqrt{(f(\vb*{v}),f(\vb*{v}))}$であるから、
    \begin{equation*}
      \|f(\vb*{v})\| = \|\vb*{v}\|
    \end{equation*}
    が成り立つ $\qed$
  \end{subpattern}

  \begin{subpattern}{\bfseries $f$はノルムを保つ$\Longrightarrow$ $f$はユニタリ変換}
    任意の$\vb*{v} \in V$に対し、
    \begin{equation*}
      \|f(\vb*{v})\| = \|\vb*{v}\|
    \end{equation*}
    が成り立つというのが仮定である

    そこで、$\vb*{a},\,\vb*{b} \in V$とすると、
    \begin{equation*}
      \|\vb*{a} + \vb*{b} \| = \|f(\vb*{a}) + f(\vb*{b})\|
    \end{equation*}
    両辺を二乗して、
    \begin{equation*}
      \|\vb*{a} + \vb*{b} \|^2 = \|f(\vb*{a}) + f(\vb*{b})\|^2
    \end{equation*}
    このとき、左辺は次のように展開できる
    \begin{align*}
      \|\vb*{a} + \vb*{b} \|^2 & = (\vb*{a} + \vb*{b},\vb*{a} + \vb*{b})                      \\
                               & = (\vb*{a},\vb*{a}) + 2(\vb*{a},\vb*{b}) + (\vb*{b},\vb*{b}) \\
                               & = \|\vb*{a}\|^2 + 2(\vb*{a},\vb*{b}) + \|\vb*{b}\|^2
    \end{align*}
    右辺も同様に、
    \begin{multline*}
      \|f(\vb*{a}) + f(\vb*{b})\|^2 \\= \|f(\vb*{a})\|^2 + 2(f(\vb*{a}),f(\vb*{b})) + \|f(\vb*{b})\|^2
    \end{multline*}

    さて、仮定より、$\|f(\vb*{a})\| = \|\vb*{a}\|$と$\|f(\vb*{b})\| = \|\vb*{b}\|$が成り立つことから、
    \begin{equation*}
      \|\vb*{a} + \vb*{b} \|^2 = \|f(\vb*{a}) + f(\vb*{b})\|^2
    \end{equation*}
    という等式の両辺を展開した結果、残る項は
    \begin{equation*}
      2 (\vb*{a},\vb*{b}) = 2(f(\vb*{a}),f(\vb*{b}))
    \end{equation*}
    だけとなる

    したがって、
    \begin{equation*}
      (\vb*{a},\vb*{b}) = (f(\vb*{a}),f(\vb*{b}))
    \end{equation*}
    が成り立つので、$f$はユニタリ変換である $\qed$
  \end{subpattern}
\end{proof}

\subsection{ユニタリ変換の表現行列}

\todo{「計量同型」を学んでから}

\end{document}
