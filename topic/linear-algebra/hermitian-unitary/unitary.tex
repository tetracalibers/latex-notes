\documentclass[../../../topic_linear-algebra]{subfiles}

\begin{document}

\sectionline
\section{直交行列とユニタリ行列}
\marginnote{\refbookF p275〜276 \\ \refbookA p204}

\begin{definition}{ユニタリ行列}
  複素正方行列$A$が次を満たすとき、$A$を\keyword{ユニタリ行列}という
  \begin{equation*}
    A^* = A^{-1}
  \end{equation*}
\end{definition}

$A$が実正方行列のときは、
\begin{equation*}
  A\text{がユニタリ行列} \Longleftrightarrow {}^tA = A^{-1}
\end{equation*}
となり、このような$A$は\keyword{直交行列}と呼ばれる

\begin{definition}{直交行列}
  実正方行列$A$が次を満たすとき、$A$を\keyword{直交行列}という
  \begin{equation*}
    {}^t A = A^{-1}
  \end{equation*}
\end{definition}

直交行列という名前の由来は、次のように考えられる

\br

$A$を$n$個の列ベクトルを横一列に並べたものとみなし、
\begin{equation*}
  A = (\vb*{a}_1, \vb*{a}_2, \ldots, \vb*{a}_n)
\end{equation*}
とおくと、${}^t A = A^{-1}$、すなわち${}^tAA = E$は、
\begin{equation*}
  \begin{pmatrix}
    {}^t \vb*{a}_1 \\
    {}^t \vb*{a}_2 \\
    \vdots         \\
    {}^t \vb*{a}_n
  \end{pmatrix} \left(
  \vb*{a}_1, \vb*{a}_2, \ldots, \vb*{a}_n
  \right) = \begin{pmatrix}
    1      & 0      & \cdots & 0      \\
    0      & 1      & \cdots & 0      \\
    \vdots & \vdots & \ddots & \vdots \\
    0      & 0      & \cdots & 1
  \end{pmatrix}
\end{equation*}
と表される

\br

これは、ベクトル$\vb*{a}_1, \vb*{a}_2, \ldots, \vb*{a}_n$が、次の性質
\begin{equation*}
  {}^t \vb*{a}_i \vb*{a}_j =(\vb*{a}_i, \vb*{a}_j) = \delta_{ij}
\end{equation*}
を満たすことを意味する

\br

すなわち、直交行列$A$の列ベクトル$\vb*{a}_1, \vb*{a}_2, \ldots, \vb*{a}_n$は、互いに直交する単位ベクトルである

\br

この事実は、複素行列に対しても成立する

\begin{theorem}{todo}
  複素正方行列$U$を$U = (\vb*{u}_1, \ldots, \vb*{u}_n)$と列ベクトル分解するとき、
  \begin{equation*}
    U\text{がユニタリ行列} \Longleftrightarrow (\vb*{u}_i, \vb*{u}_j) = \delta_{ij}
  \end{equation*}
  すなわち、ユニタリ行列の列ベクトルは、互いに直交する単位ベクトルである
\end{theorem}

\sectionline
\section{ユニタリ変換}
\marginnote{\refbookA p77〜82 \\ \refbookC p126〜131}

体$\mathbb{C}$上の計量空間において、内積を保つ線形変換を\keyword{ユニタリ変換}という

\begin{definition}{ユニタリ変換}
  体$\mathbb{C}$上の計量空間$V$における線形変換$f$が\keyword{ユニタリ変換}であるとは、任意の$\vb*{u},\,\vb*{v} \in V$に対し、
  \begin{equation*}
    (f(\vb*{u}),f(\vb*{v})) = (\vb*{u},\vb*{v})
  \end{equation*}
  が成り立つことである
\end{definition}

体$\mathbb{R}$上のユニタリ変換は、\keyword{直交変換}と呼ばれる

\subsection{ユニタリ変換とノルム}

ユニタリ変換は、ベクトルの長さを変えない変換でもある

\begin{theorem}{ユニタリ変換とノルム保存性}
  計量空間$V$における線形変換を$f$がユニタリ変換であることと、任意の$\vb*{v} \in V$に対し
  \begin{equation*}
    \|f(\vb*{v})\| = \|\vb*{v}\|
  \end{equation*}
  が成り立つことは同値である
\end{theorem}

\begin{proof}
  \begin{subpattern}{\bfseries $f$がユニタリ変換$\Longrightarrow$ $f$はノルムを保つ}
    ユニタリ変換の定義より、
    \begin{equation*}
      (f(\vb*{v}),f(\vb*{v})) = (\vb*{v},\vb*{v}) = \|\vb*{v}\|^2
    \end{equation*}
    ここで、$\|f(\vb*{v})\| = \sqrt{(f(\vb*{v}),f(\vb*{v}))}$であるから、
    \begin{equation*}
      \|f(\vb*{v})\| = \|\vb*{v}\|
    \end{equation*}
    が成り立つ $\qed$
  \end{subpattern}

  \begin{subpattern}{\bfseries $f$はノルムを保つ$\Longrightarrow$ $f$はユニタリ変換}
    任意の$\vb*{v} \in V$に対し、
    \begin{equation*}
      \|f(\vb*{v})\| = \|\vb*{v}\|
    \end{equation*}
    が成り立つというのが仮定である

    そこで、$\vb*{a},\,\vb*{b} \in V$とすると、
    \begin{equation*}
      \|\vb*{a} + \vb*{b} \| = \|f(\vb*{a}) + f(\vb*{b})\|
    \end{equation*}
    両辺を二乗して、
    \begin{equation*}
      \|\vb*{a} + \vb*{b} \|^2 = \|f(\vb*{a}) + f(\vb*{b})\|^2
    \end{equation*}
    このとき、左辺は次のように展開できる
    \begin{align*}
      \|\vb*{a} + \vb*{b} \|^2 & = (\vb*{a} + \vb*{b},\vb*{a} + \vb*{b})                      \\
                               & = (\vb*{a},\vb*{a}) + 2(\vb*{a},\vb*{b}) + (\vb*{b},\vb*{b}) \\
                               & = \|\vb*{a}\|^2 + 2(\vb*{a},\vb*{b}) + \|\vb*{b}\|^2
    \end{align*}
    右辺も同様に、
    \begin{multline*}
      \|f(\vb*{a}) + f(\vb*{b})\|^2 \\= \|f(\vb*{a})\|^2 + 2(f(\vb*{a}),f(\vb*{b})) + \|f(\vb*{b})\|^2
    \end{multline*}

    さて、仮定より、$\|f(\vb*{a})\| = \|\vb*{a}\|$と$\|f(\vb*{b})\| = \|\vb*{b}\|$が成り立つことから、
    \begin{equation*}
      \|\vb*{a} + \vb*{b} \|^2 = \|f(\vb*{a}) + f(\vb*{b})\|^2
    \end{equation*}
    という等式の両辺を展開した結果、残る項は
    \begin{equation*}
      2 (\vb*{a},\vb*{b}) = 2(f(\vb*{a}),f(\vb*{b}))
    \end{equation*}
    だけとなる

    したがって、
    \begin{equation*}
      (\vb*{a},\vb*{b}) = (f(\vb*{a}),f(\vb*{b}))
    \end{equation*}
    が成り立つので、$f$はユニタリ変換である $\qed$
  \end{subpattern}
\end{proof}

\subsection{ユニタリ変換の表現行列}

ユニタリ変換の表現行列は、\keyword{ユニタリ行列}である

\begin{theorem}{todo}
  計量空間上の線形変換$f$がユニタリ変換であることと、$f$の表現行列$A$がユニタリ行列であることは同値である
\end{theorem}

\begin{proof}
  \begin{subpattern}{\bfseries $f$がユニタリ変換$\Longrightarrow$ $A$がユニタリ行列}

  \end{subpattern}

  \begin{subpattern}{\bfseries $A$がユニタリ行列$\Longrightarrow$ $f$がユニタリ変換}

  \end{subpattern}
\end{proof}

\end{document}
