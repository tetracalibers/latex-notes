\documentclass[../../../topic_linear-algebra]{subfiles}

\usepackage{xr-hyper}
\externaldocument{../../../.tex_intermediates/topic_linear-algebra}

\begin{document}

\sectionline
\section{基底を写す線形写像}
\marginnote{\refbookS p40〜41 \\ \refbookA p103}

通常の関数では、定義域のすべての点で「対応する値をどうするか」を決めなければならない。
数の対応表から点を1つ1つ打ち、最後にそれらを結ぶことで関数のグラフを描いた経験がある人も多いだろう。

\br

一方で、\keyword{線形写像}は、「\keyword{基底}での値だけを決めれば、線形性により残りが全部決まる」という性質を持つ。
線形写像を定めるには、定義域の基底がどこに行くか(基底の像)だけを指定すればよい。

\begin{emphabox}
  \begin{spacebox}
    \begin{center}
      線形写像は基底上の値だけで完全に決まる
    \end{center}
  \end{spacebox}
\end{emphabox}

この事実は、次のように示される。

% \refbookA 命題3.2.13、\refbookS 命題2.1.3
\begin{theorem}{基底を写す線形写像の存在}{linear-map-determined-by-basis}
  $V$を線型空間とし、$\vb*{v}_1, \ldots, \vb*{v}_n$を$V$の基底とする。
  $W$を線型空間とし、$\vb*{w}_1, \ldots, \vb*{w}_n \in W$を任意に与えるとき、次を満たす線形写像$f\colon V \to W$が一意的に存在する。
  \begin{equation*}
    f(\vb*{v}_i) = \vb*{w}_i \quad (i = 1, \ldots, n)
  \end{equation*}
\end{theorem}

\begin{proof}
  \begin{subpattern}{\bfseries $f$の存在}
    任意の$\vb*{v}\in V$は、基底$\{\vb*{v}_i\}_{i=1}^n$により一意的に
    \begin{equation*}
      \vb*{v} = \sum_{i=1}^n a_i \vb*{v}_i \quad (a_i \in K)
    \end{equation*}
    と表すことができる。
    
    そこで、$f$を次のように定める。
    \begin{equation*}
      f(\vb*{v}) = \sum_{i=1}^n a_i\vb*{w}_i
    \end{equation*}
    
    このとき、もし$\vb*{v}=\displaystyle\sum_{i=1}^n b_i\vb*{v}_i$とも表せるなら、基底による表示の一意性から$a_i=b_i$($i=1,\dots,n$)である。
    
    よって、$\displaystyle\sum_{i=1}^n a_i\vb*{w}_i=\sum_{i=1}^n b_i\vb*{w}_i$となり、$f$の定義は一意に定まる。$\qed$
  \end{subpattern}
  
  \begin{subpattern}{\bfseries $f$の線形性}
    $\displaystyle\vb*{a}=\sum_{i=1}^n a_i\vb*{v}_i,\ \vb*{b}=\sum_{i=1}^n b_i\vb*{v}_i$とし、$c_1,c_2$をスカラーとする。

    このとき、$\displaystyle c_1\vb*{a}+c_2\vb*{b}=\sum_{i=1}^n(c_1 a_i+c_2 b_i)\vb*{v}_i$であるから、$f$の定義より、
    \begin{align*}
      f(c_1\vb*{a}+c_2\vb*{b})
        &= \sum_{i=1}^n(c_1 a_i+c_2 b_i)\vb*{w}_i \\
        &= c_1\sum_{i=1}^n a_i\vb*{w}_i + c_2\sum_{i=1}^n b_i\vb*{w}_i \\
        &= c_1 f(\vb*{a})+c_2 f(\vb*{b})
    \end{align*}
    よって$f$は線形である。$\qed$
  \end{subpattern}
  
  \begin{subpattern}{\bfseries $f$の一意性}
    $f(\vb*{v}_i)=\vb*{w}_i$を満たす線形写像$g\colon V\to W$を任意にとる。
    
    任意の$\displaystyle\vb*{v}=\sum_{i=1}^n a_i\vb*{v}_i$に対し、$f$の線形性より、
    \begin{equation*}
      g(\vb*{v}) = g\left(\sum_{i=1}^n a_i\vb*{v}_i\right)
      = \sum_{i=1}^n a_i g(\vb*{v}_i)
      = \sum_{i=1}^n a_i\vb*{w}_i
      = f(\vb*{v})
    \end{equation*}
    したがって、$g=f$である。 $\qed$
  \end{subpattern}
\end{proof}

\br

この定理の条件を満たすものとして、証明では線形写像$f$を次のように構成した。
\begin{equation*}
  f(\vb*{v}) = \sum_{i=1}^{n} a_i \vb*{w}_i
\end{equation*}
この$f$を、$V$の\keyword{基底}$\{ \vb*{v}_1, \ldots, \vb*{v}_n \}$を$\vb*{w}_1, \ldots, \vb*{w}_n \in W$に\keyword{写す}線形写像とよぶ。

\br

特に、$V = K^n$で、$V$の基底として\keyword{標準基底}$\{ \vb*{e}_1, \ldots, \vb*{e}_n \}$を選んだときは、この$f$を$\vb*{w}_1, \ldots, \vb*{w}_n \in W$が\keyword{定める}線形写像ともよぶ。

\sectionline
\section{基底による線形写像の決定と比較}
\marginnote{
  \refweb{一般の線形空間(増補版)}{https://www.cck.dendai.ac.jp/math/support/latb.html}
}

\thmref{thm:linear-map-determined-by-basis}で述べたことから、$\{ \vb*{u}_1, \ldots, \vb*{u}_n \}$を$V$の基底とするとき、線形写像$f$に対して$f(\vb*{u}_1), \ldots, f(\vb*{u}_n)$の値が測定できれば、$f$の形を一意的に決定できる。
\begin{align*}
  f(\vb*{v}) &= f(v_1 \vb*{u}_1 + \cdots + v_n \vb*{u}_n) \\
  &= \dlinelabelmath[Cerulean]{v_1}{変数} \fitLabelMath[Rhodamine][carnationpink]{f(\vb*{u}_1)}{定数} + \cdots + \dlinelabelmath[Cerulean]{v_n}{変数} \fitLabelMath[Rhodamine][carnationpink]{f(\vb*{u}_n)}{定数}
\end{align*}

上の式において、$v_1, \ldots, v_n$は$\vb*{v}$によって決まる変数である。

$f(\vb*{u}_1), \ldots, f(\vb*{u}_n)$の値さえ決まれば、どんな$\vb*{v}$を入れても$f$の値が定まる。

これが「$f$の形が決まる」ということである。

\br

このことを利用すると、基底に関する線形写像の値を比較することで、複数の線形写像が等しいかどうかを判別できる。

\begin{theorem}{基底上の値による線型写像の同一性判定}{linear-map-equality-on-basis}
  $f, g$をともに$V$から$W$への線型写像とする。
  $f$と$g$が等しいとは、$V$のある基底$\{ \vb*{v}_1, \ldots, \vb*{v}_n \}$に対して、次が成り立つことと同値である。
  \begin{equation*}
    f(\vb*{v}_i) = g(\vb*{v}_i) \quad (i = 1, \ldots, n)
  \end{equation*}
\end{theorem}

\begin{proof}
  \begin{subpattern}{$f = g \Longrightarrow f(\vb*{v}_i) = g(\vb*{v}_i)$}
    $f = g$ならば、任意の$\vb*{v} \in V$に対して$f(\vb*{v}) = g(\vb*{v})$が成り立つ。
    
    よって、基底ベクトル$\vb*{v}_i$に対しても$f(\vb*{v}_i) = g(\vb*{v}_i)$が成り立つ。 $\qed$
  \end{subpattern}
  
  \begin{subpattern}{$f(\vb*{v}_i) = g(\vb*{v}_i) \Longrightarrow f = g$}
    $V$の任意のベクトル$\vb*{v}$は、基底ベクトルの線形結合として表される。
    \begin{equation*}
      \vb*{v} = c_1 \vb*{v}_1 + \cdots + c_n \vb*{v}_n
    \end{equation*}
    
    このとき、線型写像の線形性から、
    \begin{align*}
      f(\vb*{v}) &= f(c_1 \vb*{v}_1 + \cdots + c_n \vb*{v}_n) \\
      &= c_1 f(\vb*{v}_1) + \cdots + c_n f(\vb*{v}_n) \\
      &= c_1 g(\vb*{v}_1) + \cdots + c_n g(\vb*{v}_n) \\
      &= g(c_1 \vb*{v}_1 + \cdots + c_n \vb*{v}_n) = g(\vb*{v})
    \end{align*}
    
    よって、任意の$\vb*{v} \in V$に対して$f(\vb*{v}) = g(\vb*{v})$が成り立つので、$f = g$である。 $\qed$
  \end{subpattern}
\end{proof}

\end{document}
