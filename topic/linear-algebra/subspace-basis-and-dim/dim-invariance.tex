\documentclass[../../../topic_linear-algebra]{subfiles}

\usepackage{xr-hyper}
\externaldocument{../../../.tex_intermediates/topic_linear-algebra}

\begin{document}

\sectionline
\section{次元の不変性}
\marginnote{\refbookA p99 \\ \refbookC p37〜38}

\begin{theorem*}{次元の不変性}
  $K^n$の部分空間$V$の基底をなすベクトルの個数(次元)は一定である。

  つまり、$\{ \vb*{v}_1, \dots, \vb*{v}_k \}$と$\{ \vb*{u}_1, \dots, \vb*{u}_l \}$がともに$V$の基底ならば、$k = l$である。
\end{theorem*}

\begin{proof}
  $\vb*{u}_1, \dots, \vb*{u}_l \in \langle \vb*{v}_1, \dots, \vb*{v}_k \rangle$であり、$\vb*{u}_1, \dots, \vb*{u}_l$は線型独立であるから、\thmref{thm:abstract-finite-dependency}より、$l \leq k$である。

  同様にして$k \leq l$も成り立つので、$k = l$である。 $\qed$
\end{proof}

\sectionline
\section{線型独立なベクトルと次元}
\marginnote{\refbookA p100}

\begin{theorem*}{線形独立なベクトルの最大個数と空間の次元}
  線形空間$V$中の線型独立なベクトルの最大個数は$\dim V$と等しい。
\end{theorem*}

\begin{proof}
  $V$の基底を$\{ \vb*{v}_1,\dots, \vb*{v}_k \}$とすると、$V$には$k$個の線型独立なベクトルが存在する。

  また、$V = \langle \vb*{v}_1, \dots, \vb*{v}_k \rangle$であるため、\thmref{thm:abstract-finite-dependency}より、$V$中の線型独立なベクトルの個数は$k$を超えることはない。

  つまり、$k$は$V$に含まれる線型独立なベクトルの最大個数である。 $\qed$
\end{proof}

\sectionline

\begin{theorem*}{線形空間を生成するベクトルの最小個数と次元}
  線形空間$V$を張るベクトルの最小個数は$\dim V$と等しい。
\end{theorem*}

\begin{proof}
  \todo{\refbookA p100 問3.3}
\end{proof}

\end{document}
