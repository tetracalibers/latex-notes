\documentclass[../../../topic_linear-algebra]{subfiles}

\begin{document}

\sectionline
\section{ベクトルが張る空間}\label{sec:span-of-vectors}
\marginnote{\refbookA p6〜8 \\ \refbookL p135 \\ \refbookS p21}

$m \times n$型行列$A$で写れる範囲を$\Im A$として定義した。

$\vb*{x}$を$n$次元ベクトルとすると、$\Im A$は次のようなものといえる。
\begin{emphabox}
  \begin{spacebox}
    \begin{center}
      $\vb*{x}$をいろいろ動かしたときの、\\
      $\vb*{y} = A\vb*{x}$が動ける範囲が$\Im A$
    \end{center}
  \end{spacebox}
\end{emphabox}

ここで、$A$を列ベクトルを並べたもの$A= (\vb*{a}_1,\ldots, \vb*{a}_n)$として書き、$\vb*{x}$も成分$x_1,\ldots,x_n$で書けば、
\begin{equation*}
  \vb*{y} = \begin{pmatrix}
    \vb*{a}_1 & \cdots & \vb*{a}_n
  \end{pmatrix} \begin{pmatrix}
    x_1    \\
    \vdots \\
    x_n
  \end{pmatrix}
  = x_1 \vb*{a}_1 + \cdots + x_n \vb*{a}_n
\end{equation*}

つまり、
\begin{emphabox}
  \begin{spacebox}
    \begin{center}
      数$x_1, \ldots, x_n$をいろいろ動かしたときの、\\
      $x_1 \vb*{a}_1 + \cdots + x_n \vb*{a}_n$が動ける範囲が$\Im A$
    \end{center}
  \end{spacebox}
\end{emphabox}
であり、この線形結合が動ける範囲を「ベクトル$\vb*{a}_1, \ldots, \vb*{a}_n$の\keyword{張る空間}」という。

\begin{definition}{ベクトルが張る空間}{span-of-vectors}
  $k$個のベクトル$\vb*{a}_1, \dots, \vb*{a}_k \in \mathbb{R}^n$を与えたとき、$\vb*{a}_1, \dots, \vb*{a}_k$の線形結合全体の集合を
  \begin{equation*}
    \langle \vb*{a}_1, \dots, \vb*{a}_k \rangle \quad \text{あるいは} \quad \Span\{\vb*{a}_1, \dots, \vb*{a}_k\}
  \end{equation*}
  によって表し、これを$\vb*{a}_1, \dots, \vb*{a}_k$が\keyword{張る空間}という。
\end{definition}

\begin{definition*}{生成系}
  $V = \langle \vb*{v}_1, \dots, \vb*{v}_k \rangle$であるとき、$\vb*{v}_1, \dots, \vb*{v}_k$を$V$の\keywordJE{生成系}{system of generators}という。
  
  このとき、$V$は$\vb*{v}_1, \dots, \vb*{v}_k$によって\keywordJE{生成される}{generated}ともいう。
\end{definition*}

\subsection{ベクトルが張る空間は部分空間}

\begin{theorem*}{ベクトルが張る空間は部分空間}
  $\vb*{v}_1,\dots, \vb*{v}_k \in \mathbb{R}^n$が張る空間$\langle \vb*{v}_1, \dots, \vb*{v}_k \rangle$は部分空間である
\end{theorem*}

\begin{proof}
  \todo{\refbookA p94 命題3.1.2}
\end{proof}

\br

\begin{theorem*}{部分空間の張る空間は部分空間}
  $V \subset \mathbb{R}^n$を部分空間、$\vb*{v}_1, \dots, \vb*{v}_k \in V$とすると、
  \begin{equation*}
    \langle \vb*{v}_1, \dots, \vb*{v}_k \rangle \subset V
  \end{equation*}
\end{theorem*}

\begin{proof}
  \todo{\refbookA p94 命題3.1.4}
\end{proof}

\subsection{ベクトルが張る空間の幾何的解釈}

ベクトル$\vb*{a}_1, \ldots, \vb*{a}_n$の張る空間$\langle \vb*{a}_1, \ldots, \vb*{a}_n \rangle$は、$\vb*{a}_1, \ldots, \vb*{a}_n$で定まる平面の一般化といえる。
(ここで、点は0次元平面、直線は1次元平面と考える。)

\begin{itemize}
  \item $\vb*{a}_1, \ldots, \vb*{a}_n$がすべて$\vb*{o}$なら、$\vb*{o}$ただ一点が$\langle \vb*{a}_1, \ldots, \vb*{a}_n \rangle$
  \item $\vb*{a}_1, \ldots, \vb*{a}_n$がすべて一直線上にあれば、その直線が$\langle \vb*{a}_1, \ldots, \vb*{a}_n \rangle$
  \item $\vb*{a}_1, \ldots, \vb*{a}_n$がすべて平面上にあれば、その平面が$\langle \vb*{a}_1, \ldots, \vb*{a}_n \rangle$
\end{itemize}

\subsection{ベクトルが張る空間と有限従属性}

\begin{theorem}{有限従属性定理の抽象版}{abstract-finite-dependency}
  $\vb*{v}_1, \vb*{v}_2, \dots, \vb*{v}_k \in \mathbb{R}^n$とする

  $\langle \vb*{v}_1, \vb*{v}_2, \dots, \vb*{v}_k \rangle$に含まれる$k$個よりも多い個数のベクトルの集合は線形従属である
\end{theorem}

\begin{proof}
  \todo{\refbookA p41 (問 1.14)}
\end{proof}

\end{document}
