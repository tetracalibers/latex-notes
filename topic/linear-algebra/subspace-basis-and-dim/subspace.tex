\documentclass[../../../topic_linear-algebra]{subfiles}

\usepackage{xr-hyper}
\externaldocument{../../../.tex_intermediates/topic_linear-algebra}

\begin{document}

\sectionline
\section{部分空間}
\marginnote{\refbookS p17〜18 \\ \refbookA p93〜94、p99}

$m > n$の場合、$m \times n$型行列$A$は、写し先の空間をカバーしきれない写像を表していた。(\secref{sec:linear-map-m-greater-n})

つまり、写った結果が空間の一部、\keyword{部分空間}になるということである。

\br

そこで、線型空間$V$の部分集合であって、それ自体もまた線形空間になるような集合について考える。
これは、原点を含む直線や平面などを一般化した概念である。

\begin{definition*}{部分空間}
  $V$を線形空間とする。
  $W$が$V$の\keywordJE{部分空間}{subspace}であるとは、$W$が$V$の部分集合であり、次の条件を満たすことをいう。
  \begin{enumerate}[label=\romanlabel]
    \item 任意の$\vb*{u}, \vb*{v} \in W$に対して、$\vb*{u} + \vb*{v} \in W$
    \item 任意の$\vb*{u} \in W$、任意のスカラー$c$に対して、$c \vb*{u} \in W$
    \item $V$の零元$\vb*{o}$は$W$の元
  \end{enumerate}
\end{definition*}

条件(\romannum{i})を満たすことを、$W$は加法で\keywordJE{閉じている}{closed}という。

条件(\romannum{ii})を満たすことを、$W$はスカラー倍で\keyword{閉じている}という。

\br

また、空集合は条件(\romannum{i})、(\romannum{ii})は満たすが、条件(\romannum{iii})を満たさない。

\defref{def:vector-space-axioms}には零元の存在が含まれているため、空集合は線型空間ではない。

空集合を排除するために、条件(\romannum{iii})が必須となる。

\br

入れものの空間$V$のことはあまり意識せずに、集合$W$とそのベクトル演算に着目して、ある$V$の線形部分空間のことを単に\keyword{線形空間}と呼ぶこともある。

\subsection{部分空間の図示}

部分空間を視覚的に表すには、箱を使うと便利である。

\begin{center}
  \begin{tikzpicture}[node distance=3cm]
    \node [subspace, rectangle split draw splits=false, label=above:{$V$}, draw=carnationpink, thick, rectangle split part fill={carnationpink!50, SkyBlue!50}]        (A)    {\phantom{$\vb*{a}$} \nodepart{second} $W$};
    \node [subspace, rectangle split draw splits=false, right of=A, label=above:{$V$}, draw=carnationpink, thick, rectangle split part fill={carnationpink!50, SkyBlue!50}]    (B)    {$\vb*{a}$ \nodepart{second} $\vb*{b}$};
  \end{tikzpicture}
\end{center}

\begin{itemize}
  \item 左図:$V$の部分空間が$W$である
  \item 右図:$\vb*{a},\vb*{b} \in V$が、$\vb*{b} \in W$であり$\vb*{a} \notin W$である
\end{itemize}

\subsection{部分空間の例:$V$自身}

$V \supseteq V$であり、$V$は線形空間であるから、$V$自身も$V$の部分空間といえる。

\subsection{部分空間の例:零ベクトルだけからなる集合}

零ベクトル$\vb*{o}$だけからなる部分集合$\{ \vb*{o} \}$も部分空間である。

\begin{theorem}{部分空間における零ベクトルの包含}{subspace-contains-zero}
  部分空間は必ず零ベクトル$\vb*{o}$を含む。
\end{theorem}

\begin{proof}
  $V$は空集合でないので、ある$\vb*{v} \in V$をとるとき、線形部分空間の定義(\romannum{ii})より
  \begin{equation*}
    0 \cdot \vb*{v} = \vb*{o} \in V
  \end{equation*}
  よって部分空間は必ず$\vb*{o}$を含む。 $\qed$
\end{proof}

\subsection{部分空間の例:$n-1$次平面}

たとえば、$\mathbb{R}^3$において座標を$(x, y, z)$とするとき、$xy$平面は$\mathbb{R}^3$の部分空間である。

\begin{definition}{座標部分空間}{coordinate-subspace}
  $\{1, 2, \dots, n\}$の部分集合$I$に対して、$x_i \, (i \in I)$以外の座標がすべて0である部分集合は$\mathbb{R}^n$の部分集合である。

  このようなものを\keyword{座標部分空間}といい、$\mathbb{R}^I$と書く。
  \begin{equation*}
    \mathbb{R}^I = \langle \vb*{e}_i \mid i \in I \rangle
  \end{equation*}
  と表すこともできる。
\end{definition}

\subsection{部分空間の例:線形写像の像}
\marginnote{\refbookC p82}

\begin{theorem*}{線形写像の像は部分空間}
  線形写像$f\colon V \to W$の像$\Im(f)$は$W$の部分空間である。
\end{theorem*}

\begin{proof}
  \begin{subpattern}{\bfseries 和について}
    $\vb*{u}, \vb*{v} \in \Im(f)$とすると、$\vb*{u} = f(\vb*{v}_1),\, \vb*{v} = f(\vb*{v}_2)$とおける。

    よって、$f$の線形性より、
    \begin{align*}
      \vb*{u} + \vb*{v} & = f(\vb*{v}_1) + f(\vb*{v}_2) \\
                        & = f(\vb*{v}_1 + \vb*{v}_2)
    \end{align*}
    となり、$\Im(f)$は和について閉じている。 $\qed$
  \end{subpattern}

  \begin{subpattern}{\bfseries スカラー倍について}
    $\vb*{u} \in \Im(f)$と$c \in \mathbb{R}$をとると、$\vb*{u} = f(\vb*{v})$とおける。

    よって、$f$の線形性より、
    \begin{align*}
      c \vb*{u} & = c f(\vb*{v}) \\
                & = f(c \vb*{v})
    \end{align*}
    となり、$\Im(f)$はスカラー倍について閉じている。 $\qed$
  \end{subpattern}
\end{proof}

\subsection{部分空間の例:線形写像の核}
\marginnote{\refbookC p71〜72、p82}

\begin{theorem}{部分空間の零ベクトルと線形写像}{linear-map-zero-preserving}
  部分空間$V,\,W$の間の線形写像$f\colon V \to W$に対して、$V$の零ベクトルを$\vb*{o}_V$、$W$の零ベクトルを$\vb*{o}_W$とすると、
  \begin{equation*}
    f(\vb*{o}_V) = \vb*{o}_W
  \end{equation*}
\end{theorem}

\begin{proof}
  任意の$\vb*{v} \in V,\, \vb*{w} \in W$に対して、
  \begin{align*}
    0  \cdot \vb*{v} & = \vb*{o}_V \\
    0 \cdot \vb*{w}  & = \vb*{o}_W
  \end{align*}
  が成り立つ。

  \br

  $f(\vb*{o}_V)$は、$f$の線形性により、次のように変形できる。
  \begin{equation*}
    f(\vb*{o}_V) = f(0 \cdot \vb*{v}) = 0 \cdot f(\vb*{v})
  \end{equation*}

  ここで、$f(\vb*{v})$は、$f$による$\vb*{v} \in V$の像であるので、$W$に属する。

  そこで、$\vb*{w} = f(\vb*{v})$とおくと、
  \begin{align*}
    f(\vb*{o}_V) & = 0 \cdot f(\vb*{v}) \\
                 & = 0 \cdot \vb*{w}    \\
                 & = \vb*{o}_W
  \end{align*}
  となり、目標としていた式が示された。 $\qed$
\end{proof}

\br

\begin{theorem*}{線形写像の核は部分空間}
  線形写像$f\colon V \to W$の核$\Ker(f)$は$V$の部分空間である。
\end{theorem*}

\begin{proof}
  \thmref{thm:linear-map-zero-preserving}の主張$f(\vb*{o}_V) = \vb*{o}_W$より、零ベクトルは核空間に属する。
  \begin{equation*}
    \vb*{o} \in \Ker(f)
  \end{equation*}

  \begin{subpattern}{\bfseries 和について}
    $\vb*{u}, \vb*{v} \in \Ker(f)$とすると、$f(\vb*{u}) = \vb*{o}$かつ$f(\vb*{v}) = \vb*{o}$である。

    よって、$f$の線形性より、
    \begin{align*}
      f(\vb*{u} + \vb*{v}) & = f(\vb*{u}) + f(\vb*{v})     \\
                           & = \vb*{o} + \vb*{o} = \vb*{o}
    \end{align*}
    したがって、$\vb*{u} + \vb*{v} \in \Ker(f)$である。 $\qed$
  \end{subpattern}

  \begin{subpattern}{\bfseries スカラー倍について}
    $\vb*{u} \in \Ker(f)$と$c \in \mathbb{R}$をとると、$f(\vb*{u}) = \vb*{o}$である。

    よって、$f$の線形性より、
    \begin{align*}
      f(c \vb*{u}) & = c f(\vb*{u})              \\
                   & = c \cdot \vb*{o} = \vb*{o}
    \end{align*}
    したがって、$c \vb*{u} \in \Ker(f)$である。 $\qed$
  \end{subpattern}
\end{proof}

\end{document}
