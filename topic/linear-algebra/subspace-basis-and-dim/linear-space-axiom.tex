\documentclass[../../../topic_linear-algebra]{subfiles}

\begin{document}

\sectionline
\section{抽象化された線形空間}
\marginnote{\refbookS p6〜7 \\ \refbookH p158〜163 \\ \refbookC p143〜166 \\ \refbookG p25〜27}

線形代数の理論は、線型独立性や線形写像を基礎にしている。

\br

これらは線形結合、すなわちベクトルの\keyword{和}と\keyword{スカラー倍}を用いて定義された。

任意のベクトルは線形結合で表され、線形写像は線形結合を保つ写像として定義される。

\br

そこで、和とスカラー倍が定義された集合なら、線形写像の理論を適用できるのではないか?という発想が生まれる。

あえて抽象的に議論することで、多項式、関数、数列なども、線形代数の枠組みで扱うことができるようになる。

\subsection{線形空間の公理}

和とスカラー倍が定義された一般の集合を、\keywordJE{線形空間}{linear space}として定義する。

\br

そして、その集合の元を\keywordJE{ベクトル}{vector}と呼ぶ。

和とスカラー倍が定義されていれば、数ベクトルと同様に、線形結合によりその元を表すことができるからだ。

この意味で、線形空間は\keywordJE{ベクトル空間}{vector space}と呼ばれることもある。

\begin{definition}{線形空間の公理}{vector-space-axioms}
  集合$V$が\keyword{$K$線型空間}(あるいは\keyword{$K$上の線形空間})であるとは、$V$に加法と、$K$の元によるスカラー倍が定義されていて、次の条件が満たされていることである。
  \begin{enumerate}[label=\romanlabel]
    \item 交換法則:$\vb*{a} + \vb*{b} = \vb*{b} + \vb*{a}$
    \item 結合法則:$(\vb*{a} + \vb*{b}) + \vb*{c} = \vb*{a} + (\vb*{b} + \vb*{c})$、$k(l\vb*{a}) = (kl)\vb*{a}$
    \item 分配法則:$k(\vb*{a} + \vb*{b}) = k\vb*{a} + k\vb*{b}$、$(k + l)\vb*{a} = k\vb*{a} + l\vb*{a}$
    \item $1 \vb*{a} = \vb*{a}$($1$は体$K$の乗法に関する単位元)
    \item 零元の存在:$\vb*{o}$と書かれる特別な元が存在し、任意の$\vb*{a} \in V$に対して$\vb*{a} + \vb*{o} = \vb*{a}$
    \item 和に関する逆元の存在:任意の$\vb*{a} \in V$に対して$-\vb*{a}$と書かれる特別な元が存在し、$\vb*{a} + (-\vb*{a}) = (-\vb*{a}) + \vb*{a} = \vb*{o}$
  \end{enumerate}
\end{definition}

線形空間は必ず零元を含むことから、線形空間は\keywordJE{空集合}{empty set}ではないことに注意しよう。

\end{document}
