\documentclass[../../../topic_linear-algebra]{subfiles}

\begin{document}

\sectionline
\section{線形写像の核空間の基底}
\marginnote{\refbookA p94〜95}

斉次形方程式$A\vb*{x} = \vb*{o}$の解の自由度を$d$とすると、基本解$\vb*{u}_1, \dots, \vb*{u}_d \in \Ker A$が存在して、任意の$\vb*{u} \in \Ker A$に対し、
\begin{equation*}
  \vb*{u} = c_1 \vb*{u}_1 + \cdots + c_d \vb*{u}_d
\end{equation*}
を満たす$c_1, \dots, c_d \in \mathbb{R}$が一意的に定まる。

\br

このことは、基底の言葉で言い換えると次のようになる。

\begin{theorem}{斉次形方程式の基本解と核空間の基底}
  $A$を$m \times n$型行列とし、$\vb*{u}_1, \dots, \vb*{u}_d$を$A\vb*{x} = \vb*{o}$の基本解とするとき、$\{ \vb*{u}_1, \dots, \vb*{u}_d \}$は$\Ker A$の基底である。
\end{theorem}

\end{document}
