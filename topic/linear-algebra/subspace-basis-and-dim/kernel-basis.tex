\documentclass[../../../topic_linear-algebra]{subfiles}

\begin{document}

\sectionline
\section{線形写像の核空間の基底}
\marginnote{\refbookA p94〜95}

斉次形方程式$A\vb*{x} = \vb*{o}$の解の自由度を$d$とすると、基本解$\vb*{u}_1, \dots, \vb*{u}_d \in \Ker A$が存在して、任意の$\vb*{u} \in \Ker A$に対し、
\begin{equation*}
  \vb*{u} = c_1 \vb*{u}_1 + \cdots + c_d \vb*{u}_d
\end{equation*}
を満たす$c_1, \dots, c_d \in \mathbb{R}$が一意的に定まる。

\br

このことは、基底の言葉で言い換えると次のようになる。

\begin{theorem}{斉次形方程式の基本解と核空間の基底}
  $A$を$m \times n$型行列とし、$\vb*{u}_1, \dots, \vb*{u}_d$を$A\vb*{x} = \vb*{o}$の基本解とするとき、$\{ \vb*{u}_1, \dots, \vb*{u}_d \}$は$\Ker A$の基底である。
\end{theorem}

\br

つまり、基本解$\vb*{u}_1, \dots, \vb*{u}_d$を基準として固定すれば、$\Ker A$の元を1つ指定することは、パラメータの値の組
\begin{equation*}
  \begin{pmatrix}
    t_1    \\
    \vdots \\
    t_d
  \end{pmatrix} \in \mathbb{R}^d
\end{equation*}
を指定することと同じである。

\subsection{解のパラメータの空間と座標部分空間}

斉次形方程式$A\vb*{x} = \vb*{o}$の主変数を$x_{i_1}, \dots, x_{i_r}$、自由変数を$x_{j_1}, \dots, x_{j_d}$とすると、解のパラメータの空間は\hyperref[def:coordinate-subspace]{座標部分空間}$\mathbb{R}^{\{ j_1, \dots, j_d \}}$である。

\br

そして、そのパラメータ付けは、
\begin{equation*}
  \mathbb{R}^{\{ j_1, \dots, j_d \}} \ni \sum_{k=1}^d t_k \vb*{e}_{j_k} \longmapsto \sum_{k=1}^d t_k \vb*{u}_k \in \Ker A
\end{equation*}
によって与えられる。

\end{document}
