\documentclass[../../../topic_linear-algebra]{subfiles}

\begin{document}

\sectionline
\section{行列の階数}
\marginnote{\refbookA p28〜29}

行階段行列に変形することで、重要な量が読み取れる

\begin{definition}{行列の階数}
  行列$A$を行階段行列に変形したとき、零行でない行の個数を$A$の\keyword{階数}(rank)と呼び、$\rank(A)$と書く
\end{definition}

変形の結果として得られる行階段行列は1通りとは限らないし、変形の途中の掃き出しの手順も1通りとは限らないが、
\begin{shaded}
  階数は$A$のみによって定まる値である
\end{shaded}
ことが後に証明できる

\sectionline

$A$が$m \times n$型ならば、行は$m$個なので、$\rank(A)$は0以上$m$以下の整数である

\br

行階段行列において、零行でない行の個数は主成分の個数と一致するので、階数は行階段行列に変形したときの主成分の個数でもある

\br

行基本行列の主成分は各列に高々1つなので、主成分の個数は列の個数$n$を超えない

\br

したがって、次の重要な評価が成り立つ
\begin{equation*}
  0 \leq \rank(A) \leq \min(m,n)
\end{equation*}

\end{document}
