\documentclass[../../../topic_linear-algebra]{subfiles}

\begin{document}

\sectionline
\section{行列の階数}
\marginnote{\refbookA p28〜29}

ここで、主成分の個数$r$に名前をつけておこう。

\br

行階段行列において、主成分とは、零行でない行の中で一番左にある0でない成分のことを指す。

つまり、行階段行列の主成分の個数$r$は、零行でない行の数と一致する。

\begin{definition*}{行列の階数}
  行列$A$を行階段行列に変形したとき、零行でない行の個数を$A$の\keyword{階数}あるいは\keyword{ランク}といい、$\rank A$と書く。
\end{definition*}

零行でない行の個数は、既約行階段行列まで変形しなくても、行階段行列の時点で読み取れることに注意しよう。

\br

変形の結果として得られる行階段行列は1通りとは限らないし、変形の途中の掃き出しの手順も1通りとは限らないが、
\begin{emphabox}
  \begin{spacebox}
    \begin{center}
      階数$\rank A$は$A$のみによって定まる値である
    \end{center}
  \end{spacebox}
\end{emphabox}
ことが後に証明できる。

\subsection{階数のとりうる値の範囲}

$A$が$m \times n$型ならば、行は$m$個なので、$\rank A$は0以上$m$以下の整数である。

\br

また、階数は行階段行列に変形したときの主成分の個数でもあり、行基本行列の主成分は各列に高々1つなので、主成分の個数は列の個数$n$を超えることはない。

\br

したがって、次の重要な評価が成り立つ。

\begin{theorem}{行列の階数の範囲}{rank-bounds}
  $m \times n$型の行列$A$の階数に対して、次の不等式が成り立つ。
  \begin{equation*}
    0 \leq \rank(A) \leq \min(m,n)
  \end{equation*}
\end{theorem}

\end{document}
