\documentclass[../../../topic_linear-algebra]{subfiles}

\usepackage{xr-hyper}
\externaldocument{../../../.tex_intermediates/topic_linear-algebra}

\begin{document}

\sectionline
\section{一般解のパラメータ表示}\label{sec:general-solution-parametric-form}
\marginnote{\refbookA p33〜36}

解が1つに定まらない場合は、解の全体像を知ることが方程式を「解く」ことになる。

無数個の解が存在する場合、解の集合が直線を成していたり、もっと高い次元の図形になっていることがある。

\subsection{主変数と自由変数}

係数行列$A$の$n$個の列が、$n$個の変数に対応していることを思い出そう。

\begin{definition*}{主変数と自由変数}
  行列$A$を行基本変形により行階段形にしたとき、\secref{sec:def-pivot}がある列に対応する変数を\keyword{主変数}と呼び、それ以外の変数を\keyword{自由変数}と呼ぶ。
\end{definition*}

\br

たとえば、次のような既約行階段形に変形した拡大係数行列を考える。

\begin{equation*}
  \tilde{A_\circ} =\begin{pNiceArray}{ccccc|c}[first-row,code-for-first-row=\color{Rhodamine},last-row,code-for-last-row=\color{Cerulean},xdots={horizontal-labels,line-style ={ <->,thick}},margin]
    1 & & 3 & 4 & &                           \\
    1 & 2 & 0 & 0 & -1 & -3 \\
    0 & 0 & 1 & 0 & 2 & 1 \\
    0 & 0 & 0 & 1 & 1 & 2 \\
    0 & 0 & 0 & 0 & 0 & 0\\
    \Hdotsfor{5}_{\displaystyle 5}
    \CodeAfter
    \tikz \draw[thick,Rhodamine] (1-1) circle (0.5em);
    \tikz \draw[thick,Rhodamine] (2-3) circle (0.5em);
    \tikz \draw[thick,Rhodamine] (3-4) circle (0.5em);
  \end{pNiceArray}
\end{equation*}

変数を使って方程式の形に直すと、
\begin{equation*}
  \left\{
  \begin{NiceArray}{rrrrrrrrrrc}[margin]
    x_1                     & +                    & 2x_2                    & \color{lightgray}{+} & \color{lightgray}{0x_3} & \color{lightgray}{+} & \color{lightgray}{0x_4} & - & x_5  & = & -3 \\
    \color{lightgray}{0x_1} & \color{lightgray}{+} & \color{lightgray}{0x_2} & +                    & x_3                     & \color{lightgray}{+} & \color{lightgray}{0x_4} & + & 2x_5 & = & 1  \\
    \color{lightgray}{0x_1} & \color{lightgray}{+} & \color{lightgray}{0x_2} & \color{lightgray}{+} & \color{lightgray}{0x_3} & \color{lightgray}{+} & x_4                     & + & x_5  & = & 2
    \CodeAfter
    \tikz \draw[thick,Rhodamine] (1-1) circle (0.75em);
    \tikz \draw[thick,Rhodamine] (2-5) circle (0.75em);
    \tikz \draw[thick,Rhodamine] (3-7) circle (0.75em);
  \end{NiceArray}
  \right.
\end{equation*}

\br

主成分がある列は$1,3,4$列なので、主変数は$x_1, x_3, x_4$である。

それ以外の$x_2, x_5$は自由変数となる。

\subsection{自由変数とパラメータ}\label{sec:free-vars-and-params}

先ほどの方程式
\begin{center}
  \systeme{
    x_1 + 2x_2 - x_5 = -3,
    x_3 + 2x_5 = 1,
    x_4 + x_5 = 2
  }
\end{center}
において、自由変数を含む項を左辺に移行すれば、
\begin{center}
  \systeme{
    x_1 = -3 - 2x_2 + x_5,
    x_3 = 1 - 2x_5,
    x_4 = 2 - x_5
  }
\end{center}
となる。

\br

自由変数$x_2,x_5$に任意の値を代入したときの主変数$x_1,x_3,x_4$の値はすべてこの方程式の解になる。

自由変数の値は定まらないので、任意の値を取りうる文字として表すしかない。

\br

そこで、
\begin{equation*}
  x_2 = t_1, \quad x_5 =  t_2
\end{equation*}
とおけば、

\begin{center}
  \systeme{
    x_1 = -3 - 2t_1 + t_2,
    x_3 = 1 - 2t_2,
    x_4 = 2 - t_2
  }
\end{center}

すなわち、

\begin{center}
  \systeme{
    x_1 = -3 - 2t_1 + t_2,
    x_2 = t_1,
    x_3 = 1 - 2t_2,
    x_4 = 2 - t_2,
    x_5 = t_2
  }
\end{center}

と書ける。

\br

これをベクトル形に直すことで、一般的な解のパラメータ表示が得られる。

\begin{equation*}
  \vb*{x} = \begin{pmatrix}
    -3 \\
    0  \\
    1  \\
    2  \\
    0
  \end{pmatrix} + t_1 \begin{pmatrix}
    -2 \\
    1  \\
    0  \\
    0  \\
    0
  \end{pmatrix} + t_2 \begin{pmatrix}
    1  \\
    0  \\
    -2 \\
    -1 \\
    1
  \end{pmatrix}
\end{equation*}

\subsection{各行の方程式から得られる一般解}

先ほどの具体例を一般化して考えてみよう。

\br

次のように変形された係数拡大行列のうち、係数行列部分において主成分を含む列を$j_1, j_2, \dots, j_r$とする。
\begin{equation*}
  \tilde{A}\text{\bfseries の変形後} = \begin{pNiceArray}{ccccccc|c}[first-row,code-for-first-row=\color{Rhodamine},last-row,code-for-last-row=\color{Cerulean},xdots={horizontal-labels,line-style ={ <->,thick}},margin]
    \CodeBefore
    \rectanglecolor{carnationpink!40}{1-1}{1-7}
    \rectanglecolor{carnationpink!40}{2-3}{2-7}
    \rectanglecolor{carnationpink!40}{4-5}{4-7}
    \Body
    j_1    &        & j_2    & \cdots & j_r    &        &        &                    \\
    1      & \star  & 0      & \cdots & 0      & \star     & \star& b_1 \\
    0      & 0      & 1      & \cdots & 0      & \star     & \star& b_2 \\
    \vdots & \vdots & \vdots & \ddots & \vdots & \vdots      & \vdots     & \vdots \\
    0      & 0      & 0      & \cdots & 1      & \star      & \star  & b_r \\
    0      & 0      & 0      & \cdots & 0      & 0      & 0   & 0\\
    \vdots & \vdots & \vdots & \ddots & \vdots & \vdots& \vdots & \vdots                  \\
    0      & 0      & 0      & \cdots & 0      & 0      & 0& 0\\
    \Hdotsfor{7}_{$n$}
  \end{pNiceArray}
\end{equation*}

すると、各行の方程式は次のような形になる。
\begin{gather*}
  x_{j_i} + \sum_k \star x_k = b_i \\
  (k > j_i , k \notin \{j_1, j_2, \dots, j_r\})
\end{gather*}

ここで、$x_k$は$j_i$列よりも右にある、$\star$に対応する変数である。

\br

既約行階段形では、主成分を含む列の主成分以外の要素はすべて0であるため、$\star$に対応する自由変数のみが残る。

そのため、$k \notin \{j_1, j_2, \dots, j_r\}$という条件によって、$\star$だけを残すようにしている。

\br

移項して主変数$x_{j_i}$について解いた形にすると、各行から得られる解は、
\begin{equation*}
  x_{j_i} = b_i - \sum_k \star x_k
\end{equation*}
となる。

\br

この式から、$x_{j_1}, x_{j_2}, \dots, x_{j_r}$以外の自由変数$x_k$に勝手な数を与えるごとに、主変数$x_{j_1}, x_{j_2}, \dots, x_{j_r}$が定まることがわかる。

\br

このような自由変数は$n-r$個あるので、$A\vb*{x} = \vb*{b}$の解は、$n-r$個のパラメータを用いて表すことができる。

\subsection{基本解と特殊解}\label{sec:fundamental-and-particular-solutions}
\marginnote{\refbookB p103}

自由変数$\star$をパラメータ$t_i$とおき、各行の方程式から得られる解を縦に並べてベクトルの形にまとめると、
\begin{equation*}
  \vb*{x} = \vb*{b}' - \sum_{i=1}^{n-r} t_i \vb*{u}_i
\end{equation*}
という形の一般解の表示が得られる。

\br

ここで、パラメータ$t_i$をかけた列ベクトル$\vb*{u}_i$を連立方程式の\keyword{基本解}という。

また、パラメータをかけていない列ベクトル$\vb*{b}'$は、変形後の拡大係数行列$(A' \mid \vb*{b}')$の定数項部分$\vb*{b}'$から得られるベクトルであり、これを\keyword{特殊解}という。

\end{document}
