\documentclass[../../../topic_linear-algebra]{subfiles}

\begin{document}

\sectionline
\section{解の自由度}\label{sec:degrees-of-freedom}
\marginnote{\refbookB p113〜114 \\ \refbookF p69}

連立方程式は、解が存在する場合、$n - r$個のパラメータを用いて一般解を表現できた。

\br

パラメータの個数は、自由変数の個数でもあり、基本解の個数でもある。

パラメータの個数だけ、自由に値を決めることができる未知数が方程式に含まれているということである。

\br

そこで、解を表すパラメータの個数を\keyword{解の自由度}と呼ぶ。
\begin{align*}
  \text{\bfseries 解の自由度} & = \text{\bfseries 変数の個数} - \rank(A) \\
                         & = n - r
\end{align*}

解の自由度は、解全体のなす集合の大きさ、すなわち何次元の空間なのかを表している。

\end{document}
