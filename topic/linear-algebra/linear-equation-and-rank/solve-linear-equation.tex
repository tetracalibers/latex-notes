\documentclass[../../../topic_linear-algebra]{subfiles}

\begin{document}

\sectionline
\section{連立一次方程式を解く}
\marginnote{\refbookA p25}

方程式を解くということは、次のような問題に答えることである

\begin{enumerate}[label=\Alph*.]
  \item 解は存在するか?
  \item 解が存在する場合、それはただ1つの解か?
  \item 解が複数存在する場合は、どれくらい多く存在するのか?
  \item 解全体の集合をいかにしてわかりやすく表示できるか?
\end{enumerate}

\sectionline
\section{拡大係数行列}
\marginnote{\refbookA p31〜32}

$A$を$m$行$n$列の行列、$\vb*{b} \in \mathbb{R}^m$とし、線形方程式
\begin{equation*}
  A\vb*{x} = \vb*{b}
\end{equation*}
を考える

これは、$n$個の文字に関する$m$本の連立方程式である

$\vb*{x}$は未知数$x_1, x_2, \dots, x_n$を成分とするベクトルである

\br

このとき、$A$は方程式の\keyword{係数行列}と呼ばれる

$A$の右端に列ベクトル$\vb*{b}$を追加して得られる$m$行$(n+1)$列の行列
\begin{equation*}
  \tilde{A} = (A \mid \vb*{b})
\end{equation*}
を考えて、これを\keyword{拡大係数行列}という

\sectionline
\section{斉次形}

$\vb*{b}=\vb*{0}$の場合、つまり
\begin{equation*}
  A\vb*{x} = \vb*{0}
\end{equation*}
の形の線形連立方程式は\keyword{斉次形}であるという

\br

斉次形の場合は$\vb*{x} = \vb*{0}$が明らかに解になっていて、これを\keyword{自明解}という

したがって、自明解以外に解が存在するかどうかが基本的な問題である

\end{document}
