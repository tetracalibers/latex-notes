\documentclass[../../../topic_linear-algebra]{subfiles}

\begin{document}

\sectionline
\section{解の存在条件}
\marginnote{\refbookB p110〜111}

連立一次方程式$A\vb*{x} = \vb*{b}$の解の存在条件について、階数を用いて議論することができる。

\br

まず、行基本変形によって得られる方程式の解は、元の方程式の解と同じであった。

そこで、次のように変形した拡大係数行列をもとに考えると、
\begin{equation*}
  \tilde{A}\text{\bfseries の変形後} = \begin{pNiceArray}{ccccccc|c}[first-row,code-for-first-row=\color{Rhodamine},last-row,code-for-last-row=\color{Cerulean},xdots={horizontal-labels,line-style ={ <->,thick}},margin]
    \CodeBefore
    \rectanglecolor{carnationpink!40}{1-1}{1-7}
    \rectanglecolor{carnationpink!40}{2-3}{2-7}
    \rectanglecolor{carnationpink!40}{4-5}{4-7}
    \Body
    j_1    &        & j_2    & \cdots & j_r    &        &        &                    \\
    1      & \star  & 0      & \cdots & 0      & \star     & \star& b_1 \\
    0      & 0      & 1      & \cdots & 0      & \star     & \star& b_2 \\
    \vdots & \vdots & \vdots & \ddots & \vdots & \vdots      & \vdots     & \vdots \\
    0      & 0      & 0      & \cdots & 1      & \star      & \star  & b_r \\
    0      & 0      & 0      & \cdots & 0      & 0      & 0   & b_{r+1} \\
    \vdots & \vdots & \vdots & \ddots & \vdots & \vdots& \vdots & \vdots                  \\
    0      & 0      & 0      & \cdots & 0      & 0      & 0& b_m\\
    \Hdotsfor{7}_{$n$}
  \end{pNiceArray}
\end{equation*}
この方程式の解が存在するのは、
\begin{equation*}
  b_{r+1} = \cdots = b_m = 0
\end{equation*}
の場合のみであることをすでに考察した。

\br

ここで、拡大係数行列$\tilde{A}$は$A$の右端に1列追加して得られるので、零行でない行の個数、すなわち階数を考えると、$\rank \tilde{A}$は$\rank A$と等しいか、1だけ増えるかのどちらかである。

\br

$b_{r+1} = \cdots = b_m = 0$の場合、$\rank \tilde{A}$と$\rank A$は一致する。

\begin{tcolorbox}[empty, size=minimal, sidebyside, scale=0.85]
  \begin{equation*}
    \begin{pNiceArray}{ccccccc}[first-row,code-for-first-row=\color{Rhodamine},last-row,code-for-last-row=\color{Cerulean},xdots={horizontal-labels,line-style ={ <->,thick}},margin]
      \CodeBefore
      \rectanglecolor{carnationpink!40}{1-1}{1-7}
      \rectanglecolor{carnationpink!40}{2-3}{2-7}
      \rectanglecolor{carnationpink!40}{4-5}{4-7}
      \Body
      j_1    &        & j_2    & \cdots & j_r    &        &                           \\
      1      & \star  & 0      & \cdots & 0      & \star     & \star \\
      0      & 0      & 1      & \cdots & 0      & \star     & \star \\
      \vdots & \vdots & \vdots & \ddots & \vdots & \vdots      & \vdots      \\
      0      & 0      & 0      & \cdots & 1      & \star      & \star \\
      0      & 0      & 0      & \cdots & 0      & 0      & 0    \\
      \vdots & \vdots & \vdots & \ddots & \vdots & \vdots& \vdots                 \\
      0      & 0      & 0      & \cdots & 0      & 0      & 0\\
      \Hdotsfor{7}_{$n$}
    \end{pNiceArray}
  \end{equation*}%

  \tcblower

  \begin{equation*}
    \begin{pNiceArray}{ccccccc|c}[first-row,code-for-first-row=\color{Rhodamine},last-row,code-for-last-row=\color{Cerulean},xdots={horizontal-labels,line-style ={ <->,thick}},margin]
      \CodeBefore
      \rectanglecolor{carnationpink!40}{1-1}{1-8}
      \rectanglecolor{carnationpink!40}{2-3}{2-8}
      \rectanglecolor{carnationpink!40}{4-5}{4-8}
      \Body
      j_1    &        & j_2    & \cdots & j_r    &        &        &                    \\
      1      & \star  & 0      & \cdots & 0      & \star     & \star& b_1 \\
      0      & 0      & 1      & \cdots & 0      & \star     & \star& b_2 \\
      \vdots & \vdots & \vdots & \ddots & \vdots & \vdots      & \vdots     & \vdots \\
      0      & 0      & 0      & \cdots & 1      & \star      & \star  & b_r \\
      0      & 0      & 0      & \cdots & 0      & 0      & 0   & 0 \\
      \vdots & \vdots & \vdots & \ddots & \vdots & \vdots& \vdots & \vdots                  \\
      0      & 0      & 0      & \cdots & 0      & 0      & 0&0\\
      \Hdotsfor{7}_{$n$}
    \end{pNiceArray}
  \end{equation*}%
\end{tcolorbox}

\br

一方、$b_{r+1}, \ldots, b_m$のうち、1つでも0でないものがある場合は、拡大係数行列の右端の列に主成分が現れ、$\rank \tilde{A}$と$\rank A$は一致しない。

\begin{tcolorbox}[empty, size=minimal, sidebyside, scale=0.85]
  \begin{equation*}
    \begin{pNiceArray}{ccccccc}[first-row,code-for-first-row=\color{Rhodamine},last-row,code-for-last-row=\color{Cerulean},xdots={horizontal-labels,line-style ={ <->,thick}},margin]
      \CodeBefore
      \rectanglecolor{carnationpink!40}{1-1}{1-7}
      \rectanglecolor{carnationpink!40}{2-3}{2-7}
      \rectanglecolor{carnationpink!40}{4-5}{4-7}
      \Body
      j_1    &        & j_2    & \cdots & j_r    &        &                           \\
      1      & \star  & 0      & \cdots & 0      & \star     & \star \\
      0      & 0      & 1      & \cdots & 0      & \star     & \star \\
      \vdots & \vdots & \vdots & \ddots & \vdots & \vdots      & \vdots      \\
      0      & 0      & 0      & \cdots & 1      & \star      & \star \\
      0      & 0      & 0      & \cdots & 0      & 0      & 0    \\
      \vdots & \vdots & \vdots & \ddots & \vdots & \vdots& \vdots                 \\
      0      & 0      & 0      & \cdots & 0      & 0      & 0\\
      \Hdotsfor{7}_{$n$}
    \end{pNiceArray}
  \end{equation*}%

  \tcblower

  \begin{equation*}
    \begin{pNiceArray}{ccccccc|c}[first-row,code-for-first-row=\color{Rhodamine},last-row,code-for-last-row=\color{Cerulean},xdots={horizontal-labels,line-style ={ <->,thick}},margin]
      \CodeBefore
      \rectanglecolor{carnationpink!40}{1-1}{1-8}
      \rectanglecolor{carnationpink!40}{2-3}{2-8}
      \rectanglecolor{carnationpink!40}{4-5}{4-8}
      \rectanglecolor{carnationpink!40}{5-8}{5-8}
      \Body
      j_1    &        & j_2    & \cdots & j_r    &        &        &                    \\
      1      & \star  & 0      & \cdots & 0      & \star     & \star& b_1 \\
      0      & 0      & 1      & \cdots & 0      & \star     & \star& b_2 \\
      \vdots & \vdots & \vdots & \ddots & \vdots & \vdots      & \vdots     & \vdots \\
      0      & 0      & 0      & \cdots & 1      & \star      & \star  & b_r \\
      0      & 0      & 0      & \cdots & 0      & 0      & 0   & 1 \\
      \vdots & \vdots & \vdots & \ddots & \vdots & \vdots& \vdots & \vdots                  \\
      0      & 0      & 0      & \cdots & 0      & 0      & 0&0\\
      \Hdotsfor{7}_{$n$}
    \end{pNiceArray}
  \end{equation*}%
\end{tcolorbox}

$b_{r+1}, \ldots, b_m$のうち、0でないものが2つ以上ある場合も、さらに行基本変形を行うことで、右上の拡大係数行列と同じ形にできるので、
\begin{equation*}
  \rank \tilde{A} = \rank A + 1
\end{equation*}
となる。

\br

以上の考察から、連立一次方程式$A\vb*{x} = \vb*{b}$の解が存在する条件は、
\begin{emphabox}
  \begin{spacebox}
    \begin{center}
      係数行列と拡大係数行列の階数が等しい
    \end{center}
  \end{spacebox}
\end{emphabox}
ことだといえる。

\begin{theorem}{拡大係数行列と解の存在条件}\label{thm:augmented-rank-solution-condition}
  $A$を$m \times n$型行列、$\vb*{b} \in \mathbb{R}^m$とする

  $\tilde{A} = (A \mid \vb*{b})$とおくとき、
  \begin{equation*}
    \rank(\tilde{A}) = \rank(A) \Longleftrightarrow A\vb*{x} = \vb*{b} \text{に解が存在する}
  \end{equation*}
\end{theorem}

\begin{proof}
  \todo{\refbookA p31 (定理1.5.1)}
\end{proof}

\br

\begin{theorem}{解の存在条件の系}\label{thm:full-row-rank-solvable}
  $A$を$m \times n$型行列とするとき、
  \begin{equation*}
    ^{\forall}\vb*{b} \in \mathbb{R}^m, A\vb*{x} = \vb*{b} \text{の解が存在する} \Longleftrightarrow \rank(A) = m
  \end{equation*}
\end{theorem}

\begin{proof}
  \todo{\refbookA p32 (定理1.5.2, 1.5.3)}
\end{proof}


\end{document}
