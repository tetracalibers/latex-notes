\documentclass[../../../topic_linear-algebra]{subfiles}

\usepackage{xr-hyper}
\externaldocument{../../../.tex_intermediates/topic_linear-algebra}

\begin{document}

\sectionline
\section{固有値と固有ベクトル}
\marginnote{\refbookA p183〜184 \\ \refbookC p178〜179 \\ \refbookF p251〜252}

与えられた線形写像を表現する行列を単純化(\keyword{対角化})する上で、一次元不変部分空間への直和分解が本質的な役割を果たす

\br

一次元の$f$不変部分空間$W$の基底$\vb*{a}$とは、
\begin{shaded}
  ある$\lambda \in K$について$f(\vb*{a}) = \lambda \vb*{a}$
\end{shaded}
となるような$\vb*{0}$以外のベクトルだった

\begin{definition*}{固有値と固有ベクトル}
  体$K$上の線形空間$V$上の線形変換$f\colon V \to V$に対して、
  \begin{equation*}
    f(\vb*{a}) = \lambda \vb*{a} \quad (\vb*{a} \neq \vb*{0})
  \end{equation*}
  となるベクトル$\vb*{a} \in V$が存在するとき、このようなスカラー$\lambda \in K$を、線形変換$f$の\keyword{固有値}という

  また、このようなベクトル$\vb*{a}$を、$f$の固有値$\lambda$に属する\keyword{固有ベクトル}という
\end{definition*}

線形変換$f$の表現行列を$A$とすると、これは正方行列であり、$f(\vb*{a}) = A\vb*{a}$と表せる

よって、固有値と固有ベクトルの定義は、次のようにも書ける

\begin{definition*}{行列の固有値と固有ベクトル}
  正方行列$A$に対して、
  \begin{equation*}
    A\vb*{a} = \lambda \vb*{a} \quad (\vb*{a} \neq \vb*{0})
  \end{equation*}
  となるベクトル$\vb*{a}$とスカラー$\lambda$が存在するとき、このようなスカラー$\lambda$を行列$A$の\keyword{固有値}という

  また、このようなベクトル$\vb*{a}$を、行列$A$の固有値$\lambda$に属する\keyword{固有ベクトル}という
\end{definition*}

\sectionline
\section{異なる固有値に属する固有ベクトル}
\marginnote{\refbookA p186〜187 \\ \refbookF p265〜266}

\begin{theorem*}{異なる固有値に属する固有ベクトルの非一致性}
  異なる固有値$\alpha_i,\, \alpha_j \, (\alpha_i \neq \alpha_j)$に属する固有ベクトル$\vb*{p}_i, \vb*{p}_j$は異なるベクトルである
\end{theorem*}

\begin{proof}
  固有値と固有ベクトルの定義より、
  \begin{equation*}
    \left\{
    \begin{array}{rcl}
      A\vb*{p}_i & = & \alpha_i \vb*{p}_i \\
      A\vb*{p}_j & = & \alpha_j \vb*{p}_j
    \end{array}
    \right.
  \end{equation*}
  である

  もし$\vb*{p}_i = \vb*{p}_j$ならば、
  \begin{gather*}
    \alpha_i \vb*{p}_i              = \alpha_j \vb*{p}_i \\
    \therefore \quad (\alpha_i - \alpha_j) \vb*{p}_i = \vb*{0}
  \end{gather*}
  となるが、$\vb*{p}_i$は固有ベクトルであり$\vb*{0}$ではないので、$\alpha_i - \alpha_j = 0$となる

  すなわち、
  \begin{equation*}
    \alpha_i = \alpha_j
  \end{equation*}
  が成立し、これは$\alpha_i \neq \alpha_j$に反する

  よって、$\vb*{p}_i \neq \vb*{p}_j$でなければならない $\qed$
\end{proof}

この定理を発展させて、次のことがいえる

\begin{theorem}{異なる固有値に属する固有ベクトルの線型独立性}{eigenvectors-of-distinct-eigenvalues-are-independent}
  $\alpha_1, \alpha_2, \ldots, \alpha_k$が行列$A$の相異なる固有値であるとすると、それぞれに属する固有ベクトル$\vb*{p}_1, \vb*{p}_2, \ldots, \vb*{p}_k$は線型独立である
\end{theorem}

\begin{proof}
  固有値の個数$k$についての数学的帰納法によって証明する

  \br

  $k=1$のとき、$\vb*{p}_1$は固有ベクトルゆえ$\vb*{0}$ではないので、$\{\vb*{p}_1\}$は線型独立である

  \br

  $k \geq 2$として、$(k-1)$個以下の固有ベクトルについて定理の主張が成り立つと仮定する

  このとき、線形関係式
  \begin{equation*}
    c_1 \vb*{p}_1 + c_2 \vb*{p}_2 + \cdots + c_k \vb*{p}_k = \vb*{0}
  \end{equation*}
  を考える

  両辺に$A$をかけると、$A\vb*{p}_i = \alpha_i \vb*{p}_i$より、
  \begin{equation*}
    c_1 \alpha_1 \vb*{p}_1 + c_2 \alpha_2 \vb*{p}_2 + \cdots + c_k \alpha_k \vb*{p}_k = \vb*{0}
  \end{equation*}
  この等式から、初めの線形関係式の$\alpha_k$倍を引いて
  \begin{equation*}
    c_1 (\alpha_1 - \alpha_k) \vb*{p}_1 + \cdots + c_{k-1} (\alpha_{k-1} - \alpha_k) \vb*{p}_{k-1} = \vb*{0}
  \end{equation*}

  ここで、帰納法の仮定より、$\vb*{p}_1, \vb*{p}_2, \ldots, \vb*{p}_{k-1}$は線型独立であるため、係数はすべて0でなければならない
  \begin{equation*}
    c_1 (\alpha_1 - \alpha_k) = 0, \quad \ldots, \quad c_{k-1} (\alpha_{k-1} - \alpha_k) = 0
  \end{equation*}

  さらに、$\alpha_1, \alpha_2, \ldots, \alpha_k$は相異なる固有値であるため、$\alpha_i - \alpha_k \neq 0$ ($i = 1, \ldots, k-1$)である

  よって、
  \begin{equation*}
    c_1 = 0, \quad \ldots, \quad c_{k-1} = 0
  \end{equation*}
  が成り立つ

  この結果を初めの線形関係式に代入すると、
  \begin{equation*}
    c_k \vb*{p}_k = \vb*{0}
  \end{equation*}
  が残るが、$\vb*{p}_k$は固有ベクトルであり$\vb*{0}$ではないため、$c_k = 0$も成り立つ

  以上より、$\vb*{p}_1, \vb*{p}_2, \ldots, \vb*{p}_k$は線型独立である $\qed$
\end{proof}

\sectionline
\section{固有ベクトルによる行列の対角化}
\marginnote{\refbookA p184〜185 \\ \refbookF p264〜265、p267}

\secref{sec:1d-invariant-subspaces}で見たように、
\begin{equation*}
  f(\vb*{a}_i) = \lambda_i \vb*{a}_i \quad (i = 1, \dots, n)
\end{equation*}
となるような$\vb*{a}_i$を基底として用いると、線形変換$f$は次のような対角行列で表現できた
\begin{equation*}
  \begin{pmatrix}
    \lambda_1 & 0         & \cdots & 0         \\
    0         & \lambda_2 & \cdots & 0         \\
    \vdots    & \vdots    & \ddots & \vdots    \\
    0         & 0         & \cdots & \lambda_n
  \end{pmatrix}
\end{equation*}

そして、このような$\vb*{a}_i$を\keyword{固有ベクトル}、$\lambda_i$を\keyword{固有値}として定義したため、
\begin{shaded}
  行列$A$の固有ベクトルからなる基底が存在すれば、\\
  $A$は\keyword{対角化}できる
\end{shaded}
と言い換えられる

\sectionline

線形変換$f$の表現行列を$A$とすると、$A$は正方行列である

\br

たとえば基底を$A$の固有ベクトルに変換した際に、\thmref{thm:similarity-under-basis-change}より、この線形変換$f$の表現行列が$P^{-1}AP$に変化するとして、この行列$P^{-1}AP$が対角行列となる場合が、$A$が\keyword{対角化}できるということである

\begin{definition}{対角化可能}{diagonalizable}
  与えられた正方行列$A$が適当な正則行列$P$により
  \begin{equation*}
    P^{-1}AP = \begin{pNiceArray}{cccc}[margin]
      \lambda_1 & & & O \\
      & \lambda_2 & & \\
      & & \ddots & \\
      O & & & \lambda_n \\
    \end{pNiceArray}
  \end{equation*}
  と変形できるとき、$A$は\keyword{対角化可能}であるという
\end{definition}

\sectionline

行列$A$の固有ベクトルからなる基底が存在すれば、$A$は\keyword{対角化可能}である、ということを定式化しよう

\begin{theorem}{対角化可能性と固有ベクトルの線型独立性}{diagonalizable-iff-n-indep-eigenvectors}
  $n$次元正方行列$A$が対角化可能であるための必要十分条件は、線型独立な$n$個の$A$の固有ベクトルが存在することである
\end{theorem}

\begin{proof}
  \begin{subpattern}{\bfseries 線型独立な$A$の固有ベクトルが存在 $\Longrightarrow$ $A$は対角化可能}
    $A$の固有ベクトルを$\vb*{a}_i \, (i = 1, \dots, n)$、それに対応する固有値を$\alpha_i$とすると、固有値と固有ベクトルの定義より、次式が成り立つ
    \begin{equation*}
      f(\vb*{a}_i) = \alpha_i \vb*{a}_i \quad (i = 1, \dots, n)
    \end{equation*}
    仮定より$\vb*{a}_i$は線型独立であり、一次元部分空間$\{ c \vb*{a}_i \mid c \in K \}$は$\vb*{a}_i$によって張られる空間である

    よって、$\vb*{a}_i$を基底として用いることができるので、\secref{sec:1d-invariant-subspaces}に関する議論で見たように、$A$は対角行列で表現できる $\qed$
  \end{subpattern}

  \begin{subpattern}{\bfseries $A$は対角化可能 $\Longrightarrow$ 線型独立な$A$の固有ベクトルが存在}
    $A$が対角化可能であることから、ある正則行列$P$が存在して、
    \begin{equation*}
      P^{-1}AP = \begin{pNiceArray}{cccc}[margin]
        \alpha_1 & & & O \\
        & \alpha_2 & & \\
        & & \ddots & \\
        O & & & \alpha_n \\
      \end{pNiceArray}
    \end{equation*}
    が成り立つので、両辺に$P$をかけて、
    \begin{equation*}
      AP = P \begin{pNiceArray}{cccc}[margin]
        \alpha_1 & & & O \\
        & \alpha_2 & & \\
        & & \ddots & \\
        O & & & \alpha_n \\
      \end{pNiceArray}
    \end{equation*}
    が成り立つ

    ここで、$P$を$n$個の列ベクトル$\vb*{p}_1, \vb*{p}_2, \ldots, \vb*{p}_n$を横に並べたもの、すなわち、
    \begin{equation*}
      P = (\vb*{p}_1, \vb*{p}_2, \ldots, \vb*{p}_n)
    \end{equation*}
    とみなせば、上の等式は、
    \begin{equation*}
      \left\{
      \begin{array}{c}
        A\vb*{p}_1 = \alpha_1 \vb*{p}_1 \\
        A\vb*{p}_2 = \alpha_2 \vb*{p}_2 \\
        \vdots                          \\
        A\vb*{p}_n = \alpha_n \vb*{p}_n
      \end{array}
      \right.
    \end{equation*}
    という関係を意味する

    これはすなわち、$\vb*{p}_1, \vb*{p}_2, \ldots, \vb*{p}_n$がそれぞれの固有値$\alpha_1, \alpha_2, \ldots, \alpha_n$に属する$A$の固有ベクトルであることを意味する

    さらに、\thmref{thm:invertible-iff-col-indep}より、$P$は正則であるため、その列ベクトル$\vb*{p}_1, \vb*{p}_2, \ldots, \vb*{p}_n$は線型独立である $\qed$
  \end{subpattern}
\end{proof}

この定理と、\thmref{thm:eigenvectors-of-distinct-eigenvalues-are-independent}から、次の定理が得られる

\begin{theorem}{固有値の相異性と対角化可能性}{distinct-eigenvalues-imply-diagonalizable}
  $n$次正方行列$A$が異なる$n$個の固有値$\alpha_1, \ldots, \alpha_n$をもつならば、$A$は対角化可能である

  すなわち、ある$n$次正則行列$P$によって、
  \begin{equation*}
    P^{-1}AP = \begin{pNiceArray}{cccc}[margin]
      \alpha_1 & & & O \\
      & \alpha_2 & & \\
      & & \ddots & \\
      O & & & \alpha_n \\
    \end{pNiceArray}
  \end{equation*}
  が成り立つ
\end{theorem}

\begin{proof}
  $n$個の異なる固有値$\alpha_1, \ldots, \alpha_n$に属する固有ベクトル$\vb*{p}_1, \ldots, \vb*{p}_n$は線型独立である

  よって、固有ベクトルの線型独立性より、対角化可能性が導かれる $\qed$
\end{proof}

ただし、この定理の逆は成立しない

つまり、$n$次正方行列$A$が$n$個の異なる固有値を持たなくても、対角化できることがある

\br

実際、$A$がすでに対角行列になっているなら、最も単純な場合として$A = E$をとると、$A$の固有値は1だけであるが、任意の正則行列$P$に対して$P^{-1}EP$は対角行列$E$になる

\br

よって、対角化のために本質的なのは、$n$個の異なる固有値ではなく、
\begin{shaded}
  $n$個の線型独立な固有ベクトル
\end{shaded}
であるといえる

\end{document}
