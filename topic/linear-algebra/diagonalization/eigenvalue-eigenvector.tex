\documentclass[../../../topic_linear-algebra]{subfiles}

\begin{document}

\sectionline
\section{固有値と固有ベクトル}
\marginnote{\refbookA p183〜184 \\ \refbookC p178〜179 \\ \refbookF p251〜252}

与えられた線形写像を表現する行列を単純化(\keyword{対角化})する上で、一次元不変部分空間への直和分解が本質的な役割を果たす

\br

一次元の$f$不変部分空間$W$の基底$\vb*{a}$とは、
\begin{shaded}
  ある$\lambda \in K$について$f(\vb*{a}) = \lambda \vb*{a}$
\end{shaded}
となるような$\vb*{0}$以外のベクトルだった

\begin{definition}{固有値と固有ベクトル}
  体$K$上の線形空間$V$上の線形変換$f\colon V \to V$に対して、
  \begin{equation*}
    f(\vb*{a}) = \lambda \vb*{a} \quad (\vb*{a} \neq \vb*{0})
  \end{equation*}
  となるベクトル$\vb*{a} \in V$が存在するとき、このようなスカラー$\lambda \in K$を、線形変換$f$の\keyword{固有値}という

  また、このようなベクトル$\vb*{a}$を、$f$の固有値$\lambda$に属する\keyword{固有ベクトル}という
\end{definition}

線形変換$f$の表現行列を$A$とすると、これは正方行列であり、$f(\vb*{a}) = A\vb*{a}$と表せる

よって、固有値と固有ベクトルの定義は、次のようにも書ける

\begin{definition}{行列の固有値と固有ベクトル}
  正方行列$A$に対して、
  \begin{equation*}
    A\vb*{a} = \lambda \vb*{a} \quad (\vb*{a} \neq \vb*{0})
  \end{equation*}
  となるベクトル$\vb*{a}$とスカラー$\lambda$が存在するとき、このようなスカラー$\lambda$を行列$A$の\keyword{固有値}という

  また、このようなベクトル$\vb*{a}$を、行列$A$の固有値$\lambda$に属する\keyword{固有ベクトル}という
\end{definition}

\sectionline
\section{固有ベクトルによる行列の対角化}
\marginnote{\refbookA p184〜185}

\hyperref[sec:1d-invariant-subspaces]{一次元不変部分空間}に関する議論で見たように、
\begin{equation*}
  f(\vb*{a}_i) = \lambda_i \vb*{a}_i \quad (i = 1, \dots, n)
\end{equation*}
となるような$\vb*{a}_i$を基底として用いると、線形変換$f$は次のような対角行列で表現できた
\begin{equation*}
  \begin{pmatrix}
    \lambda_1 & 0         & \cdots & 0         \\
    0         & \lambda_2 & \cdots & 0         \\
    \vdots    & \vdots    & \ddots & \vdots    \\
    0         & 0         & \cdots & \lambda_n
  \end{pmatrix}
\end{equation*}

そして、このような$\vb*{a}_i$を\keyword{固有ベクトル}、$\lambda_i$を\keyword{固有値}として定義したため、このことは、
\begin{shaded}
  行列$A$の固有ベクトルからなる基底が存在すれば、
  $A$は\keyword{対角化}できる
\end{shaded}
と言い換えられる

\end{document}
