\documentclass[../../../topic_linear-algebra]{subfiles}

\begin{document}

\sectionline
\section{固有空間}
\marginnote{\refbookC p182〜185 \\ \refbookG p2 \\ \refbookA p187 \\ \refbookF p251〜252、p262}

線形空間$V$の中に、行列の固有ベクトルが「どれくらい」あるかを調べるため、各固有値$\alpha$に対して、$\alpha$の固有ベクトルと$\vb*{0}$からなる$V$の部分集合を考える

\br

$\alpha$が$A$の固有値ならば、方程式
\begin{equation*}
  (\alpha E - A)\vb*{x} = \vb*{0}
\end{equation*}
の解空間、すなわち\keyword{核空間}$\Ker(\alpha E - A)$は、固有値$\alpha$を持つ$A$の固有ベクトルのすべてと$\vb*{0}$からなる

\br

核空間は$V$の部分空間であり、これを固有値$\alpha$の\keyword{固有空間}と呼ぶ

\begin{definition}{固有空間}
  $\alpha$が$A$の固有値であるとき、
  \begin{equation*}
    W(\alpha) = \Ker(\alpha E - A)
  \end{equation*}
  を固有値$\alpha$の\keyword{固有空間}と呼ぶ
\end{definition}

\end{document}
