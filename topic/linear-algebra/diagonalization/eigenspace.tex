\documentclass[../../../topic_linear-algebra]{subfiles}

\begin{document}

\sectionline
\section{固有空間}
\marginnote{\refbookC p182〜185 \\ \refbookG p2、p4〜5 \\ \refbookA p187 \\ \refbookF p251〜252、p262、p271〜273}

線形空間$V$の中に、行列の固有ベクトルが「どれくらい」あるかを調べるため、各固有値$\alpha$に対して、$\alpha$の固有ベクトルと$\vb*{0}$からなる$V$の部分集合を考える

\br

$\alpha$が$A$の固有値ならば、方程式
\begin{equation*}
  (\alpha E - A)\vb*{x} = \vb*{0}
\end{equation*}
の解空間、すなわち\keyword{核空間}$\Ker(\alpha E - A)$は、固有値$\alpha$を持つ$A$の固有ベクトルのすべてと$\vb*{0}$からなる

\br

核空間は$V$の部分空間であり、これを固有値$\alpha$の\keyword{固有空間}と呼ぶ

\begin{definition}{固有空間}
  $\alpha$が$A$の固有値であるとき、
  \begin{equation*}
    W(\alpha) = \Ker(\alpha E - A)
  \end{equation*}
  を固有値$\alpha$の\keyword{固有空間}と呼ぶ
\end{definition}

\subsection{固有空間の次元}

\begin{theorem}{固有空間の次元と固有値の重複度}{geom-mult-leq-alg-mult}
  $A$の固有値$\alpha_i$の重複度$k_i$と、固有空間$W(\alpha_i)$の次元$\dim W(\alpha_i)$に対し、次の不等式が成立する
  \begin{equation*}
    \dim W(\alpha_i) \leq k_i \quad (1 \leq i \leq s)
  \end{equation*}
\end{theorem}

\begin{proof}
  $W(\alpha_i)$の基底$\vb*{v}_1,\ldots,\vb*{v}_m$をとる

  $\vb*{v}_1,\ldots,\vb*{v}_m,\vb*{v}_{m+1},\ldots,\vb*{v}_{n}$が$K^n$の基底となるように、$n-m$個のベクトル$\vb*{v}_{m+1},\ldots,\vb*{v}_{n}$を追加して\hyperref[thm:basis-extension]{基底を延長}する

  \br

  $P = (\vb*{v}_1,\ldots,\vb*{v}_m,\vb*{v}_{m+1},\ldots,\vb*{v}_{n})$とするとき、
  \begin{equation*}
    AP = (A\vb*{v}_1,\ldots,A\vb*{v}_m,A\vb*{v}_{m+1},\ldots,A\vb*{v}_{n})
  \end{equation*}
  ここで、$W(\alpha_i)$の基底$\vb*{v}_1,\ldots,\vb*{v}_m$は$A$の固有ベクトルであるので、固有値と固有ベクトルの定義より、
  \begin{align*}
    AP & = (\alpha_i\vb*{v}_1,\ldots,\alpha_i\vb*{v}_m,A\vb*{v}_{m+1},\ldots,A\vb*{v}_{n})                                                      \\
       & = (\vb*{v}_1,\ldots,\vb*{v}_{n})\begin{pNiceArray}{ccc|cc}[xdots={horizontal-labels,line-style = <->},first-row,last-col,margin,columns-width =1em]
                                           \Hdotsfor{3}^{m} & \Hdotsfor{2}^{n-m} \\
                                           \alpha_i & & & \Block{3-2}<\large>{B} && \Vdotsfor{3}^{m}  \\
                                           & \ddots &&& \\
                                           & & \alpha_i&& \\
                                           \hline
                                           \Block{2-3}<\large>{O} && & \Block{2-2}<\large>{C} && \Vdotsfor{2}^{n-m} \\
                                           &&&&
                                         \end{pNiceArray} \\
       & = P\begin{pmatrix}
              \alpha_iE_m & B \\
              O           & C
            \end{pmatrix}
  \end{align*}

  基底の線型独立性より、\hyperref[thm:invertible-iff-col-indep]{線型独立な列ベクトルを並べた行列$P$は正則}であるので、
  \begin{equation*}
    P^{-1}AP = \begin{pmatrix}
      \alpha_iE_m & B \\
      O           & C
    \end{pmatrix}
  \end{equation*}

  この行列の特性多項式を考えると、
  \begin{align*}
    \Phi_{P^{-1}AP}(x) & = \det(xE - P^{-1}AP)                                                                   \\
                       & = \begin{vNiceArray}{ccc|cc}[xdots={horizontal-labels,line-style = <->},margin,columns-width =1.5em]
                             x-\alpha_i & & & \Block{3-2}{-B} &   \\
                             & \ddots &&& \\
                             & & x-\alpha_i&& \\
                             \hline
                             \Block{2-3}<\large>{O} && & \Block{2-2}{xE'-C} &\\
                             &&&&
                           \end{vNiceArray} \\
                       & = \begin{vNiceArray}{ccc}[xdots={horizontal-labels,line-style = <->},margin,columns-width =1.5em]
                             x-\alpha_i & & \\
                             & \ddots & \\
                             & & x-\alpha_i
                           \end{vNiceArray} \det(xE' - C)    \\
                       & = (x-\alpha_i)^m \det(xE' - C)
  \end{align*}
  より、固有値$\alpha_i$の重複度$k_i$は$m$以上となる
  \begin{equation*}
    m \leq k_i
  \end{equation*}
  $m$は$W(\alpha)_i$の基底を構成するベクトルの個数、すなわち$\dim W(\alpha_i)$であるので、
  \begin{equation*}
    \dim W(\alpha_i) \leq k_i
  \end{equation*}
  が成り立つ $\qed$
\end{proof}

この定理の証明過程で登場した特性方程式
\begin{equation*}
  \Phi_{P^{-1}AP}(x) = (x-\alpha_i)^m \det(xE' - C)
\end{equation*}
において、$\det(xE' - C)$からも$(x-\alpha_i)$が現れれば、$\alpha_i$の重複度$k_i$は$m$より大きくなることがわかる

\sectionline

$W(\alpha)$の基底を構成するベクトルの個数は、固有値$\alpha$に属する線型独立な固有ベクトルの個数ともいえる

また、固有値$\alpha$の重複度が$k$であることは、特性方程式が$x=\alpha$を$k$重解にもつことを意味する

\br

以上をふまえると、前述の定理は、特性方程式の視点で次のように言い換えられる

\begin{theorem*}{固有値の重複度と固有ベクトルの最大数}
  正方行列$A$の特性方程式$\Phi_A(x) = 0$が$x=\alpha$を$k$重解にもつとき、固有値$\alpha$に属する線型独立な固有ベクトルは、$k$個以下しかとれない
\end{theorem*}

\end{document}
