\documentclass[../../../topic_linear-algebra]{subfiles}

\begin{document}

\sectionline
\section{特性方程式}
\marginnote{\refbookA p184、p188〜191 \\ \refbookF p258〜260}

$\lambda$が$n$次正方行列$A$の固有値であることは、
\begin{equation*}
  A\vb*{x} = \lambda \vb*{x} \quad (\vb*{x} \neq \vb*{0})
\end{equation*}
となるような$\vb*{x} \in K^n$が存在することである

\br

ここで、$A\vb*{x} = \lambda \vb*{x}$を次のように変形することができる
\begin{align*}
  A\vb*{x} - \lambda \vb*{x}  & = \vb*{0} \\
  A\vb*{x} - \lambda E\vb*{x} & = \vb*{0} \\
  (A - \lambda E)\vb*{x}      & = \vb*{0}
\end{align*}

$\vb*{x} \neq \vb*{0}$という条件により、$(A - \lambda E)\vb*{x} = \vb*{0}$は非自明な解を持つ必要がある

\begin{theorem}{固有ベクトルの斉次形方程式による定義}
  固有値$\lambda$の固有ベクトルとは、斉次形方程式
  \begin{equation*}
    (A - \lambda E)\vb*{x} = \vb*{0}
  \end{equation*}
  の非自明な解のことである
\end{theorem}

固有値を求める上で重要となるこの定理は、行列式を使って言い換えることができる

\begin{theorem}{固有値の方程式による定義}
  行列$A$の固有値$\lambda$は、$x$についての$n$次方程式
  \begin{equation*}
    \det(A - x E) = 0
  \end{equation*}
  の$K$に含まれる解である
\end{theorem}

\begin{proof}
  \todo{\refbookF p258}
\end{proof}

\end{document}
