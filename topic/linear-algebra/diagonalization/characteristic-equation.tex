\documentclass[../../../topic_linear-algebra]{subfiles}

\begin{document}

\sectionline
\section{特性方程式}
\marginnote{\refbookA p184、p188〜191 \\ \refbookF p258〜260}

$\lambda$が$n$次正方行列$A$の固有値であることは、
\begin{equation*}
  A\vb*{x} = \lambda \vb*{x} \quad (\vb*{x} \neq \vb*{0})
\end{equation*}
となるような$\vb*{x} \in K^n$が存在することである

\br

ここで、$A\vb*{x} = \lambda \vb*{x}$を次のように変形することができる
\begin{align*}
  A\vb*{x} - \lambda \vb*{x}  & = \vb*{0} \\
  A\vb*{x} - \lambda E\vb*{x} & = \vb*{0} \\
  (A - \lambda E)\vb*{x}      & = \vb*{0}
\end{align*}

$\vb*{x} \neq \vb*{0}$という条件により、$(A - \lambda E)\vb*{x} = \vb*{0}$は非自明な解を持つ必要がある

\begin{theorem}{固有ベクトルの斉次形方程式による定義}
  固有値$\lambda$の固有ベクトルとは、斉次形方程式
  \begin{equation*}
    (A - \lambda E)\vb*{x} = \vb*{0}
  \end{equation*}
  の非自明な解のことである
\end{theorem}

固有値を求める上で重要となるこの定理は、行列式を使って言い換えることができる

\begin{theorem}{固有値の方程式による定義}
  行列$A$の固有値$\lambda$は、$x$についての$n$次方程式
  \begin{equation*}
    \det(A - x E) = 0
  \end{equation*}
  の$K$に含まれる解である
\end{theorem}

\begin{proof}
  $\lambda$が$A$の固有値であることは、斉次形方程式$(A - \lambda E)\vb*{x} = \vb*{0}$が非自明解を持つことと言い換えられる

  そして、\hyperref[thm:homogeneous-solution-iff-det-zero]{斉次形方程式が非自明解を持つことは、行列式が0になることと同値}である

  すなわち、
  \begin{equation*}
    \det(A - \lambda E) = 0
  \end{equation*}
  が成り立ち、つまり$x = \lambda$は方程式$\det(A - xE) = 0$の解である $\qed$
\end{proof}

\sectionline

$n$次正方行列$A$に対し、$\det(A-xE)$は、$x$についての$n$次式になる

\br

実際、$A = (a_{ij})$とおいて、
\begin{equation*}
  \det(A - xE ) = \left|\begin{matrix}
    a_{11} - x & a_{12}     & \cdots & a_{1n}     \\
    a_{21}     & a_{22} - x & \cdots & a_{2n}     \\
    \vdots     & \vdots     & \ddots & \vdots     \\
    a_{n1}     & a_{n2}     & \cdots & a_{nn} - x
  \end{matrix}\right|
\end{equation*}
を展開することを考える

すべての列(あるいはすべての行)から、$x$を含む成分をとった場合の積が
\begin{multline*}
  (a_{11} - x)(a_{22} - x) \cdots (a_{nn} - x) \\
  = (-1)^n x^n + (-1)^{n-1}(a_{11} + a_{22} + \cdots + a_{nn})x^{n-1} \\+
  \cdots + a_{11}a_{22}\cdots a_{nn}
\end{multline*}
であり、$x$についての最高次は、ここに現れる$(-1)^n x^n$である

\todo{\refbookF p259、\refbookA p190}

\end{document}
