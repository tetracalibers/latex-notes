\documentclass[../../../topic_linear-algebra]{subfiles}

\begin{document}

\sectionline
\section{特性多項式と特性方程式}
\marginnote{\refbookA p184、p188〜191 \\ \refbookF p258〜260}

$\lambda$が$n$次正方行列$A$の固有値であることは、
\begin{equation*}
  A\vb*{x} = \lambda \vb*{x} \quad (\vb*{x} \neq \vb*{0})
\end{equation*}
となるような$\vb*{x} \in K^n$が存在することである

\br

ここで、$A\vb*{x} = \lambda \vb*{x}$を次のように変形することができる
\begin{align*}
  A\vb*{x} - \lambda \vb*{x}  & = \vb*{0} \\
  A\vb*{x} - \lambda E\vb*{x} & = \vb*{0} \\
  (A - \lambda E)\vb*{x}      & = \vb*{0}
\end{align*}

$\vb*{x} \neq \vb*{0}$という条件により、$(A - \lambda E)\vb*{x} = \vb*{0}$は非自明な解を持つ必要がある

\begin{theorem}{固有ベクトルの斉次形方程式による定義}
  固有値$\lambda$の固有ベクトルとは、斉次形方程式
  \begin{equation*}
    (A - \lambda E)\vb*{x} = \vb*{0}
  \end{equation*}
  の非自明な解のことである
\end{theorem}

固有値を求める上で重要となるこの定理は、行列式を使って言い換えることができる

\begin{theorem}{固有値の方程式による定義}
  行列$A$の固有値$\lambda$は、$x$についての$n$次方程式
  \begin{equation*}
    \det(A - x E) = 0
  \end{equation*}
  の$K$に含まれる解である
\end{theorem}

\begin{proof}
  $\lambda$が$A$の固有値であることは、斉次形方程式$(A - \lambda E)\vb*{x} = \vb*{0}$が非自明解を持つことと言い換えられる

  そして、\hyperref[thm:homogeneous-solution-iff-det-zero]{斉次形方程式が非自明解を持つことは、行列式が0になることと同値}である

  すなわち、
  \begin{equation*}
    \det(A - \lambda E) = 0
  \end{equation*}
  が成り立ち、つまり$x = \lambda$は方程式$\det(A - xE) = 0$の解である $\qed$
\end{proof}

\sectionline

$A = (a_{ij})$とおいて、
\begin{equation*}
  \det(A - xE ) = \left|\begin{matrix}
    a_{11} - x & a_{12}     & \cdots & a_{1n}     \\
    a_{21}     & a_{22} - x & \cdots & a_{2n}     \\
    \vdots     & \vdots     & \ddots & \vdots     \\
    a_{n1}     & a_{n2}     & \cdots & a_{nn} - x
  \end{matrix}\right|
\end{equation*}
を展開すると、$x$についての$n$次式になる

特に、すべての列(あるいはすべての行)から、$x$を含む成分をとった場合の積は、
\begin{equation*}
  (a_{11} - x)(a_{22} - x) \cdots (a_{nn} - x)
\end{equation*}
であるので、これを展開して現れる項を中心に考察する

\subsection{$n$次の項}

$(a_{11} - x)(a_{22} - x) \cdots (a_{nn} - x)$の各因子から、$-x$だけを選んでかけ合わせたものが
\begin{equation*}
  (-1)^n x^n
\end{equation*}
であり、これが最高次の項となる

\subsection{$n-1$次の項}

$(a_{11} - x)(a_{22} - x) \cdots (a_{nn} - x)$のうち、1つだけ$a_{ii}$を選び、残りの因子からは$-x$を選んでかけ合わせたものが
\begin{equation*}
  (-1)^{n-1} (a_{11} + a_{22} + \cdots + a_{nn}) x^{n-1}
\end{equation*}
である

これは、\hyperref[def:trace]{トレースの定義}より、
\begin{equation*}
  (-1)^{n-1} \tr(A) x^{n-1}
\end{equation*}
とも書き換えられる

\subsection{$n-2$次以下の項}

行列式では、各列から1つずつ、行に重複がないように成分を選ぶ必要がある

そして、今取り上げている行列式では$x$を含む成分が対角線上にあるので、$n-1$次の場合は、対角成分以外を選ぶことができなかった
(対角成分以外から$x$でない数$a_{ij}$を得ようとすると、同じ行もしくは列から2つ成分を選ぶことになってしまう)

\br

しかし、$n-2$次以下の項では、$x$を含まない成分を$2$個以上選ぶことができるので、対角成分以外からも成分を選ぶことができる

そのため、$n-2$次以下の項は、上の展開式以外からも現れることになり、単純に計算はできない

\subsection{定数項}

定数項は、多項式において$x=0$とおくことで得られるので、$\det(A - xE)$に$x=0$を代入した
\begin{equation*}
  \det(A)
\end{equation*}
が定数項となる

\sectionline

多項式の最高次の係数に$(-1)^n$がつくのは面倒なので、$\det(A - xE)$の代わりに、その$(-1)^n$倍である
\begin{equation*}
  \det(xE - A)
\end{equation*}
を考えることが多い

\br

実際、$\det(xE - A)$を展開すると、
\begin{equation*}
  \det(xE - A) = \left| \begin{matrix}
    x - a_{11} & -a_{12}    & \cdots & -a_{1n}    \\
    -a_{21}    & x - a_{22} & \cdots & -a_{2n}    \\
    \vdots     & \vdots     & \ddots & \vdots     \\
    -a_{n1}    & -a_{n2}    & \cdots & x - a_{nn}
  \end{matrix} \right|
\end{equation*}
となり、$x$の前に$(-1)$がつかずに済む

\sectionline

\begin{definition}{特性多項式}
  $A$を正方行列、$x$を変数として、
  \begin{equation*}
    \Phi_A(x) = \det(xE - A)
  \end{equation*}
  とおく

  これを\keyword{特性多項式}あるいは\keyword{固有多項式}と呼ぶ
\end{definition}

\begin{theorem}{特性多項式の構造}
  $A$を$n$次正方行列とすると、特性多項式は、次のような$n$次多項式である
  \begin{equation*}
    \Phi_A(x) = x^n - \tr(A) x^{n-1} + \cdots + (-1)^n \det(A)
  \end{equation*}
\end{theorem}

\begin{definition}{特性方程式}
  特性多項式$\Phi_A(x)$の根を求める方程式
  \begin{equation*}
    \Phi_A(x) = 0
  \end{equation*}
  を、\keyword{特性方程式}あるいは\keyword{固有方程式}と呼ぶ
\end{definition}

\sectionline
\section{固有値の重複度}
\marginnote{\refbookA p192 \\ \refbookF p270 \\ \refbookG p2}

たとえば、次の方程式
\begin{equation*}
  (x-2)^3 (x-1) = 0
\end{equation*}
の解は、$x=2$と$x=1$である

ここで、左辺を、
\begin{equation*}
  (x-2)(x-2)(x-2)(x-1) = 0
\end{equation*}
とみなすと、
\begin{align*}
  x & = 2 \\
  x & = 2 \\
  x & = 2 \\
  x & = 1
\end{align*}
というように解が重複していることがわかる

このように、「何回同じ解が現れるか?」を数えたものを\keyword{重複度}という

\begin{definition}{方程式の解の重複度}
  多項式$f(x)$で表される方程式$f(x) = 0$において、$f(x)$が$(x-\alpha)^m$で割り切れるが、$(x-\alpha)^{m+1}$では割り切れないような定数$\alpha$と自然数$m$が存在するとき、$\alpha$はこの方程式の\keyword{$m$重解}あるいは\keyword{$m$重根}であるといい、$m$を$\alpha$の\keyword{重複度}と呼ぶ
\end{definition}

上の定義は難しく聞こえるが、「ちょうど$m$回だけ$(x-\alpha)$がかかっている」ということの言い換えにすぎない

\br

たとえば、
\begin{equation*}
  (x-2)^3 (x-1)
\end{equation*}
を$(x-2)^3$で割ると、
\begin{equation*}
  (x - 1)
\end{equation*}
として割り切れるが、$(x-2)^4$で割ると、
\begin{equation*}
  \frac{(x-2)^3 (x-1)}{(x-2)^4} = \frac{x-1}{x-2} = \frac{A}{x-1} + \frac{B}{x-2}
\end{equation*}
というように部分分数分解できるので、余りが出ていることがわかる
(多項式の割り算における\keyword{余り}とは、$f(x) = g(x)q(x) + r(x)$の$r(x)$のことである)

\br

つまり、$f(x)$に因数$(x-\alpha)$が$m$個含まれている場合、$f(x)$は$(x-\alpha)^m$で割り切れるが、$m$個以上は含まれていないので、$(x-\alpha)^{m+1}$で割ると余りが出てしまう

これはすなわち、「ちょうど$m$回だけ$(x-\alpha)$がかかっている」ということである

\sectionline

ここまでの議論を応用して、固有値の\keyword{重複度}を定義する

\begin{definition}{固有値の重複度}\label{def:algebraic-multiplicity}
  特性多項式を因数分解して、
  \begin{equation*}
    \Phi_A(x) = (x- \alpha_1)^{k_1}\cdots (x - \alpha_s)^{k_s}
  \end{equation*}
  とする

  ここで、$\alpha_1,\ldots, \alpha_s$は相異なるものとする

  $k_i$は1以上の整数であり、これを固有値$\alpha_i$の\keyword{重複度}と呼ぶ

  $\Phi_A(x)$は$n$次多項式であるから、
  \begin{equation*}
    \sum_{i=1}^s k_i = n
  \end{equation*}
  が成り立つ
\end{definition}

ここで、特性多項式が
\begin{equation*}
  \Phi_A(x) = (x- \alpha_1)^{k_1}\cdots (x - \alpha_s)^{k_s}
\end{equation*}
と因数分解できることは、\keyword{代数学の基本定理}によって保証されている

\begin{theorem}{代数学の基本定理}
  定数でない任意の1変数多項式は、複素計数の1次式の積に分解できる
\end{theorem}

\sectionline
\section{相似な行列の特性多項式}
\marginnote{\refbookF p269〜271 \\ \refbookA p190〜191}

\hyperref[thm:determinant-multiplicativity]{行列式の乗法性}により、正方行列$A,\,B$が、ある正則行列$P$に対して
\begin{equation*}
  B = P^{-1}AP
\end{equation*}
となる($A$と$B$が\hyperref[def:similar-matrices]{相似}である)ならば、次のように$A$と$B$の特性多項式は一致する
\begin{align*}
  \det(xE - B) & = \det(xE - P^{-1}AP)       \\
               & = \det(xPP^{-1} - P^{-1}AP) \\
               & = \det(P^{-1}P(x - A))      \\
               & = \det(P^{-1}P(xE - A))     \\
               & = \det(E(xE - A))           \\
               & = \det(E)\det(xE - A)       \\
               & = \det(xE - A)
\end{align*}

\begin{theorem}{相似な行列の特性多項式}
  相似な行列の特性多項式は一致する
\end{theorem}

この事実は、すなわち次の事実を意味する

\begin{theorem}{相似な行列の固有値}
  相似な行列の固有値は重複度も含めて一致する
\end{theorem}

\sectionline

$n$次元正方行列$A = (a_{ij})$の特性多項式が、
\begin{equation*}
  \Phi_A(x) = x^n - \tr(A)x^{n-1} + \cdots + (-1)^n \det(A)
\end{equation*}
であることを思い出すと、次のことがいえる

\begin{theorem}{相似な行列のトレースと行列式}
  $A$と$B$が相似ならば、
  \begin{align*}
    \tr(A)  & = \tr(B)  \\
    \det(A) & = \det(B)
  \end{align*}
\end{theorem}

\br

さらに、$A$が\hyperref[def:diagonalizable]{対角化可能}であるときには、
\begin{equation*}
  B = P^{-1}AP = \begin{pmatrix}
    \alpha_1 & 0        & \cdots & 0        \\
    0        & \alpha_2 & \cdots & 0        \\
    \vdots   & \vdots   & \ddots & \vdots   \\
    0        & 0        & \cdots & \alpha_n
  \end{pmatrix}
\end{equation*}
という行列$B$と$A$が相似であるので、
\begin{align*}
  \tr(A)  & = \tr(B) = \alpha_1 + \alpha_2 + \cdots + \alpha_n \\
  \det(A) & = \det(B) = \alpha_1 \alpha_2 \cdots \alpha_n
\end{align*}
であることがわかる

\begin{theorem}{対角化可能行列の固有値による不変量の表現}
  行列$A$が対角化可能であるとき、
  \begin{itemize}
    \item $A$のトレースは$A$の固有値の和
    \item $A$の行列式は$A$の固有値の積
  \end{itemize}
\end{theorem}

\sectionline

さて、$A$と$B$が\hyperref[def:similar-matrices]{相似}であるとき、$A$と$B$は\hyperref[thm:similarity-under-basis-change]{1つの線形変換$f$を異なる基底によって表現}して得られた行列であるという関係にある

\br

このとき、$A$と$B$の特性多項式が一致するということは、次のように言い換えられる

\begin{theorem}{特性多項式の基底不変性}
  線形空間$V$の線形変換$f$に対して、$V$のある基底に関する表現行列$A$の特性多項式$\Phi_A(x)$は、基底の選び方によらず$f$のみによって決まる
\end{theorem}

\sectionline
\section{対角化可能な行列の特性多項式}
\marginnote{\refbookG p4}

$A$が$P$によって対角化されたとして、
\begin{equation*}
  P^{-1}AP= \diag(c_1,\ldots, c_n)
\end{equation*}
とすると、$A$の特性多項式は、
\begin{align*}
  \Phi_A(x) & = \Phi_{P^{-1}AP}(x)                          \\
            & = \det(xE - P^{-1}AP)                         \\
            & = \left| \begin{matrix}
                         x - c_1 & 0       & \cdots & 0       \\
                         0       & x - c_2 & \cdots & 0       \\
                         \vdots  & \vdots  & \ddots & \vdots  \\
                         0       & 0       & \cdots & x - c_n
                       \end{matrix} \right| \\
            & = (x - c_1)(x - c_2) \cdots (x - c_n)
\end{align*}
となる

一般の場合の特性多項式
\begin{equation*}
  (x- \alpha_1)^{k_1}\cdots (x - \alpha_s)^{k_s}
\end{equation*}
と見比べると、各$1 \leq i \leq s$に対して、$c_1,\ldots,c_n$の中には$\alpha_i$が$k_i$個あることがわかる

\br

つまり、対角成分として現れた数たち$c_1,\ldots,c_n$は、重複度を含めて、特性多項式$\Phi_A(x)$の根と一致する

\begin{theorem}{対角化と特性多項式の根}
  対角化可能な行列$A$を対角化して得られる対角行列の対角成分たちは、重複度を含めて特性多項式の根と一致する
\end{theorem}

また、\hyperref[thm:diagonalizable-iff-n-indep-eigenvectors]{対角化可能性と固有ベクトルの線型独立性}の証明過程を振り返ると、次のようにまとめられる

\begin{theorem}{対角化行列の列ベクトルと固有ベクトルの対応}\label{thm:diagonalization-columns-are-eigenvectors}
  対角化可能な行列$A$を対角化する正則行列$P$の列ベクトルはすべて$A$の固有ベクトルであり、固有値$\alpha_i$のものが$k_i$個ある

  ここで、$k_i$は$\alpha_i$の重複度である
\end{theorem}

\end{document}
