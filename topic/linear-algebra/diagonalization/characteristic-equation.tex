\documentclass[../../../topic_linear-algebra]{subfiles}

\begin{document}

\sectionline
\section{特性多項式と特性方程式}
\marginnote{\refbookA p184、p188〜191 \\ \refbookF p258〜260}

$\lambda$が$n$次正方行列$A$の固有値であることは、
\begin{equation*}
  A\vb*{x} = \lambda \vb*{x} \quad (\vb*{x} \neq \vb*{0})
\end{equation*}
となるような$\vb*{x} \in K^n$が存在することである

\br

ここで、$A\vb*{x} = \lambda \vb*{x}$を次のように変形することができる
\begin{align*}
  A\vb*{x} - \lambda \vb*{x}  & = \vb*{0} \\
  A\vb*{x} - \lambda E\vb*{x} & = \vb*{0} \\
  (A - \lambda E)\vb*{x}      & = \vb*{0}
\end{align*}

$\vb*{x} \neq \vb*{0}$という条件により、$(A - \lambda E)\vb*{x} = \vb*{0}$は非自明な解を持つ必要がある

\begin{theorem}{固有ベクトルの斉次形方程式による定義}
  固有値$\lambda$の固有ベクトルとは、斉次形方程式
  \begin{equation*}
    (A - \lambda E)\vb*{x} = \vb*{0}
  \end{equation*}
  の非自明な解のことである
\end{theorem}

固有値を求める上で重要となるこの定理は、行列式を使って言い換えることができる

\begin{theorem}{固有値の方程式による定義}
  行列$A$の固有値$\lambda$は、$x$についての$n$次方程式
  \begin{equation*}
    \det(A - x E) = 0
  \end{equation*}
  の$K$に含まれる解である
\end{theorem}

\begin{proof}
  $\lambda$が$A$の固有値であることは、斉次形方程式$(A - \lambda E)\vb*{x} = \vb*{0}$が非自明解を持つことと言い換えられる

  そして、\hyperref[thm:homogeneous-solution-iff-det-zero]{斉次形方程式が非自明解を持つことは、行列式が0になることと同値}である

  すなわち、
  \begin{equation*}
    \det(A - \lambda E) = 0
  \end{equation*}
  が成り立ち、つまり$x = \lambda$は方程式$\det(A - xE) = 0$の解である $\qed$
\end{proof}

\sectionline

$A = (a_{ij})$とおいて、
\begin{equation*}
  \det(A - xE ) = \left|\begin{matrix}
    a_{11} - x & a_{12}     & \cdots & a_{1n}     \\
    a_{21}     & a_{22} - x & \cdots & a_{2n}     \\
    \vdots     & \vdots     & \ddots & \vdots     \\
    a_{n1}     & a_{n2}     & \cdots & a_{nn} - x
  \end{matrix}\right|
\end{equation*}
を展開すると、$x$についての$n$次式になる

特に、すべての列(あるいはすべての行)から、$x$を含む成分をとった場合の積は、
\begin{equation*}
  (a_{11} - x)(a_{22} - x) \cdots (a_{nn} - x)
\end{equation*}
であるので、これを展開して現れる項を中心に考察する

\subsection{$n$次の項}

$(a_{11} - x)(a_{22} - x) \cdots (a_{nn} - x)$の各因子から、$-x$だけを選んでかけ合わせたものが
\begin{equation*}
  (-1)^n x^n
\end{equation*}
であり、これが最高次の項となる

\subsection{$n-1$次の項}

$(a_{11} - x)(a_{22} - x) \cdots (a_{nn} - x)$のうち、1つだけ$a_{ii}$を選び、残りの因子からは$-x$を選んでかけ合わせたものが
\begin{equation*}
  (-1)^{n-1} (a_{11} + a_{22} + \cdots + a_{nn}) x^{n-1}
\end{equation*}
である

これは、\hyperref[def:trace]{トレースの定義}より、
\begin{equation*}
  (-1)^{n-1} \tr(A) x^{n-1}
\end{equation*}
とも書き換えられる

\subsection{$n-2$次以下の項}

行列式では、各列から1つずつ、行に重複がないように成分を選ぶ必要がある

そして、今取り上げている行列式では$x$を含む成分が対角線上にあるので、$n-1$次の場合は、対角成分以外を選ぶことができなかった
(対角成分以外から$x$でない数$a_{ij}$を得ようとすると、同じ行もしくは列から2つ成分を選ぶことになってしまう)

\br

しかし、$n-2$次以下の項では、$x$を含まない成分を$2$個以上選ぶことができるので、対角成分以外からも成分を選ぶことができる

そのため、$n-2$次以下の項は、上の展開式以外からも現れることになり、単純に計算はできない

\subsection{定数項}

定数項は、多項式において$x=0$とおくことで得られるので、$\det(A - xE)$に$x=0$を代入した
\begin{equation*}
  \det(A)
\end{equation*}
が定数項となる

\sectionline

多項式の最高次の係数に$(-1)^n$がつくのは面倒なので、$\det(A - xE)$の代わりに、その$(-1)^n$倍である
\begin{equation*}
  \det(xE - A)
\end{equation*}
を考えることが多い

\br

実際、$\det(xE - A)$を展開すると、
\begin{equation*}
  \det(xE - A) = \left| \begin{matrix}
    x - a_{11} & -a_{12}    & \cdots & -a_{1n}    \\
    -a_{21}    & x - a_{22} & \cdots & -a_{2n}    \\
    \vdots     & \vdots     & \ddots & \vdots     \\
    -a_{n1}    & -a_{n2}    & \cdots & x - a_{nn}
  \end{matrix} \right|
\end{equation*}
となり、$x$の前に$(-1)$がつかずに済む

\sectionline

\begin{definition}{特性多項式}
  $A$を正方行列、$x$を変数として、
  \begin{equation*}
    \Phi_A(x) = \det(xE - A)
  \end{equation*}
  とおく

  これを\keyword{特性多項式}あるいは\keyword{固有多項式}と呼ぶ
\end{definition}

\begin{theorem}{特性多項式の構造}
  $A$を$n$次正方行列とすると、特性多項式は、次のような$n$次多項式である
  \begin{equation*}
    \Phi_A(x) = x^n - \tr(A) x^{n-1} + \cdots + (-1)^n \det(A)
  \end{equation*}
\end{theorem}

\begin{definition}{特性方程式}
  特性多項式$\Phi_A(x)$の根を求める方程式
  \begin{equation*}
    \Phi_A(x) = 0
  \end{equation*}
  を、\keyword{特性方程式}あるいは\keyword{固有方程式}と呼ぶ
\end{definition}

\end{document}
