\documentclass[../../../topic_linear-algebra]{subfiles}

\begin{document}

\sectionline
\section{対角化可能性}
\marginnote{\refbookA p193〜194 \\ \refbookC p186〜188 \\ \refbookF p271〜273}

次の定理は、
\begin{shaded}
  各固有値の固有空間が「可能な限り大きい」
\end{shaded}
ときに限り、対角化可能であると述べている

\begin{theorem}{固有空間次元と重複度の一致による対角化可能性}
  $A$の固有値を$\alpha_i$、その重複度を$k_i$とする

  $A$が対角化可能であることは、次と同値である
  \begin{equation*}
    \dim W(\alpha_i) = k_i \quad (1 \leq i \leq s)
  \end{equation*}
\end{theorem}

\begin{proof}
  \begin{subpattern}{\bfseries 対角化可能 $\Longrightarrow$ 固有空間の次元と重複度が一致}
    \todo{\refbookA p193}
  \end{subpattern}

  \begin{subpattern}{\bfseries 固有空間の次元と重複度が一致 $\Longrightarrow$ 対角化可能}
    \todo{\refbookA p194}
  \end{subpattern}
\end{proof}

\end{document}
