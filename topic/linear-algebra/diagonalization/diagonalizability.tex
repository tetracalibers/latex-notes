\documentclass[../../../topic_linear-algebra]{subfiles}

\usepackage{xr-hyper}
\externaldocument{../../../.tex_intermediates/topic_linear-algebra}

\begin{document}

\sectionline
\section{対角化可能性}
\marginnote{\refbookA p193〜194 \\ \refbookC p186〜188 \\ \refbookF p271〜273}

次の定理は、
\begin{shaded}
  各固有値の固有空間が「可能な限り大きい」
\end{shaded}
ときに限り、対角化可能であると述べている

\begin{theorem}{固有空間次元と重複度の一致による対角化可能性}{diagonalizable-iff-eigenspace-dim-equals-multiplicity}
  $A$の固有値を$\alpha_i$、その重複度を$k_i$とする

  $A$が対角化可能であることは、次と同値である
  \begin{equation*}
    \dim W(\alpha_i) = k_i \quad (1 \leq i \leq s)
  \end{equation*}
\end{theorem}

\begin{proof}
  \begin{subpattern}{\bfseries 対角化可能 $\Longrightarrow$ 固有空間の次元と重複度が一致}
    $A$が対角化可能であるので、正則行列$P$により$P^{-1}AP$が対角行列になる

    このとき、\thmref{thm:diagonalization-columns-are-eigenvectors}より、$P$の列ベクトルからなる$A$の固有ベクトルの集合には、固有値$\alpha_i$を持つものが$k_i$個含まれる

    \br

    各$i$に対して、$k_i$個の線型独立なベクトルが$W(\alpha_i)$に含まれることになるため、
    \begin{equation*}
      \dim W(\alpha_i) \geq k_i
    \end{equation*}
    がいえる

    \br

    一方、\thmref{thm:geom-mult-leq-alg-mult}より、
    \begin{equation*}
      \dim W(\alpha_i) \leq k_i
    \end{equation*}

    したがって、
    \begin{equation*}
      \dim W(\alpha_i) = k_i
    \end{equation*}
    が成り立つ $\qed$
  \end{subpattern}

  \begin{subpattern}{\bfseries 固有空間の次元と重複度が一致 $\Longrightarrow$ 対角化可能}
    $\dim W(\alpha_i) = k_i$が成り立つとし、$W(\alpha_i)$の基底$\mathcal{V}_i$をとる

    $\mathcal{V}_i$は$k_i$個の元からなり、これらは$W(\alpha_i)$の基底であることから、線形独立な固有ベクトルである

    \br

    さらに、\thmref{thm:eigenvectors-of-distinct-eigenvalues-are-independent}より、$i\neq j$とし$\mathcal{V}_i$と$\mathcal{V}_j$のベクトルは互いに線形独立である

    そこで、すべての$\mathcal{V}_i$を併せた集合
    \begin{equation*}
      \mathcal{V} = \bigcup_{i=1}^s \mathcal{V}_i
    \end{equation*}
    を考えると、$\mathcal{V}$のベクトルは線型独立である

    \br

    このとき、$\mathcal{V}$の元の個数は
    \begin{equation*}
      \sum_{i=1}^s k_i = n
    \end{equation*}
    である

    \br

    したがって、線型独立な$n$個の固有ベクトルが存在するため、\thmref{thm:diagonalizable-iff-n-indep-eigenvectors}より、$A$は対角化可能である $\qed$
  \end{subpattern}
\end{proof}

\sectionline

次の補題をもとに、対角化可能性を特性方程式の言葉で述べることができる

\begin{theorem*}{特性方程式の単根性と固有空間の次元}
  特性方程式$\Phi_A(x)$において$\alpha_i$が単根ならば、すなわち$k_i = 1$ならば、
  \begin{equation*}
    \dim W(\alpha_i) = 1
  \end{equation*}
\end{theorem*}

\begin{proof}
  $\alpha_i$は固有値なので、$\alpha_i \neq \vb*{0}$より、$W(\alpha_i) \neq \{\vb*{0}\}$がいえる

  これはつまり、
  \begin{equation*}
    \dim W(\alpha_i) \geq 1
  \end{equation*}
  ということだが、\thmref{thm:geom-mult-leq-alg-mult}より、
  \begin{equation*}
    \dim W(\alpha_i) \leq k_i = 1
  \end{equation*}
  も成り立つ

  したがって、
  \begin{equation*}
    \dim W(\alpha_i) = 1
  \end{equation*}
  である $\qed$
\end{proof}

\br

\begin{theorem*}{特性方程式の単根性と対角化可能性}
  特性方程式$\Phi_A(x)$が重根を持たなければ、$A$は対角化可能である
\end{theorem*}

\begin{proof}
  重根を持たないということは、各固有値の重複度$k_i$は1である

  よって、
  \begin{gather*}
    \dim W(\alpha_i) = 1
  \end{gather*}
  となり、$k_i$と$\dim W(\alpha_i)$が一致するので、$A$は対角化可能である  $\qed$
\end{proof}

\end{document}
