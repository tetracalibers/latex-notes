\documentclass[../../../topic_linear-algebra]{subfiles}

\usepackage{xr-hyper}
\externaldocument{../../../.tex_intermediates/topic_linear-algebra}

\begin{document}

\sectionline
\section{双対ペアリング}
\marginnote{
  \refbookA p120 \\
  \refweb{1次形式と双対空間}{https://www.cck.dendai.ac.jp/math/support/latb.html}
}

$V$と$(V^*)^*$の間には、線形同型写像$\iota \colon V \to (V^*)^*$が存在する。

このことから、\defref{def:linear-subspace-isomorphism}より、$V$と$(V^*)^*$は線形同型であることがいえる。

\br

このように、$V$が有限次元の場合は、$V$と$(V^*)^*$を自然に(基底によらずに)同一視することができる。

\br

ここで、\secref{sec:map-to-bidual}を考える際に登場した次の式を再解釈してみよう。
\begin{equation*}
  \Phi_{\vb*{v}}(\phi) = \phi(\vb*{v})
\end{equation*}

$V$と$(V^*)^*$の同型により、$\vb*{v} \in V$と$\Phi_{\vb*{v}} \in (V^*)^*$も同一視することができる。

そこで、$\Phi_{\vb*{v}}$を単に$\vb*{v}$と書くことにすると、次の関係が得られる。
\begin{equation*}
  \vb*{v}(\phi) = \phi(\vb*{v})
\end{equation*}

これは、$\vb*{v} \in V$と$\phi \in V^*$に対し、
\begin{emphabox}
  \begin{spacebox}
    \begin{center}
      値$\phi(\vb*{v})$をとることは、$\vb*{v}$から見ても$\phi$から見ても対等
    \end{center}
  \end{spacebox}
\end{emphabox}
であることを表している。

\br

この平等さを表すために、次のような記法を使うこともある。
\begin{equation*}
  \langle \phi, \vb*{v} \rangle = \langle \vb*{v}, \phi \rangle = \phi(\vb*{v})
\end{equation*}
この記号$\langle \cdot, \cdot \rangle$を、双対を表す\keyword{ペアリング}と呼ぶ。

\end{document}
