\documentclass[../../../topic_linear-algebra]{subfiles}

\usepackage{xr-hyper}
\externaldocument{../../../.tex_intermediates/topic_linear-algebra}

\begin{document}

\sectionline
\section{双対写像の表現行列}
\marginnote{\refbookR p238〜239 \\ \refbookA p122〜123}

双対写像の表現行列は、元の線形写像の表現行列の転置になる。

このことから、双対写像は\keyword{転置写像}とも呼ばれる。

\begin{theorem*}{双対写像の行列表現}
  $V,W$を有限次元の線形空間とし、$f\colon V \to W$を線型写像とする。また、$\dim V = n, \dim W = m$とする。
  
  $V$の基底$\vb*{v}_1,\ldots,\vb*{v}_n$、$W$の基底$\vb*{w}_1, \ldots, \vb*{w}_m$を選び、これらの双対基底をそれぞれ$\phi_1, \ldots, \phi_n$、$\psi_1, \ldots, \psi_m$とする。

  このとき、$\{\vb*{v}_i \}$、$\{ \vb*{w}_j\}$に関する$f$の表現行列を$A$とすると、$\{ \psi_j \}, \{ \phi_i \}$に関する$f^*$の表現行列は${}^tA$によって与えられる。
\end{theorem*}

\begin{proof}
  $f$の双対写像$f^*$は次のように定義される。
  \begin{equation*}
    f^*(\varphi)(\vb*{v}) = \varphi(f(\vb*{v}))
  \end{equation*}
  
  \br
  
  \secref{sec:construction-of-matrix-rep}より、$f \colon V \to W$の表現行列$A$は次のように表される。
  \begin{equation*}
    f(\vb*{v}_i) = \sum_{j=1}^m a_{ji} \vb*{w}_j \quad (1 \leq i \leq n)
  \end{equation*}
  したがって、任意の$i$に対し、
  \begin{equation*}
    \psi_k(f(\vb*{v}_i)) = \psi_k\left(\sum_{j=1}^m a_{ji} \vb*{w}_j\right) = \sum_{j=1}^m a_{ji} \psi_k(\vb*{w}_j)
  \end{equation*}
  ここで、$\{ \psi_k \}$は$\{ \vb*{w}_j \}$の双対基底なので、$\psi_k(\vb*{w}_j) = \delta_{kj}$より、
  \begin{equation*}
    \psi_k(f(\vb*{v}_i)) = a_{ki}
  \end{equation*}
  
  \br

  また、$f^*(\psi_k) \in V^*$は$V$上の線形汎関数なので、$V$の双対基底$\{ \phi_i \}$の線形結合として表せる。
  \begin{equation*}
    f^*(\psi_k) = \sum_{i=1}^n b_{ik} \phi_i \quad (1 \leq k \leq m)
  \end{equation*}
  この係数$b_{ik}$を並べた行列を$B$とすると、$B$は$f^*$の表現行列である。
  
  \br
  
  このとき、
  \begin{equation*}
    f^*(\psi_k)(\vb*{v}_i) = \psi_k(f(\vb*{v}_i)) = a_{ki}
  \end{equation*}
  であり、一方、
  \begin{equation*}
    f^*(\psi_k)(\vb*{v}_i) = \sum_{j=1}^n b_{ji} \phi_j(\vb*{v}_i) = \sum_{j=1}^n b_{ji} \delta_{ij} = b_{ki}
  \end{equation*}
  でもあるから、$b_{ki} = a_{ki}$が成り立つ。すなわち、
  \begin{equation*}
    B = {}^tA
  \end{equation*}
  である。 $\qed$
\end{proof}

\end{document}
