\documentclass[../../../topic_linear-algebra]{subfiles}

\begin{document}

\sectionline
\section{双対写像の表現行列}
\marginnote{\refbookR p238〜239 \\ \refbookA p122〜123}

線形写像の双対写像の表現行列は、元の線形写像の表現行列の転置になる。

このことから、双対写像は\keyword{転置写像}とも呼ばれる。

\begin{theorem}{双対写像の行列表現}
  $V,W$を有限次元の線形空間とし、$f\colon V \to W$を線型写像とする。また、$\dim V = n, \dim W = m$とする。
  
  $V$の基底$\vb*{v}_1,\ldots,\vb*{v}_n$、$W$の基底$\vb*{w}_1, \ldots, \vb*{w}_m$を選び、これらの双対基底をそれぞれ$\phi_1, \ldots, \phi_n$、$\psi_1, \ldots, \psi_m$とする。

  このとき、$\{\vb*{v}_i \}$、$\{ \vb*{w}_j\}$に関する$f$の表現行列を$A$とすると、$\{ \psi_j \}, \{ \phi_i \}$に関する$f^*$の表現行列は$A^\top$によって与えられる。
\end{theorem}

\begin{proof}
  $f\colon V \to W$に対して、その双対写像$f^* \colon W^* \to V^*$は次で定義される。
    \begin{equation*}
  \begin{tikzcd}[every label/.append style = {font = \normalsize}]
    V \arrow[r,"f"]\arrow[rd, "\psi \circ f"'] & W \arrow[d,"\psi"]\\
    & \mathbb{R}
  \end{tikzcd}
\end{equation*}
  
  \begin{equation*}
    f^*(\psi) = \psi \circ f \quad (\psi \in W^*)
  \end{equation*}
  
  $f^*(\psi)$は$V$上の線形汎関数なので、任意の$\vb*{v} \in V$を入力するとスカラーを返す。
  合成写像の記法より、$(\psi \circ f)(\vb*{v})$は、$\psi(f(\vb*{v}))$とも書けるので、
  \begin{equation*}
    f^*(\psi)(\vb*{v}) = \psi(f(\vb*{v}))
  \end{equation*}
  が成り立つ。
  
  \br

  一方、\hyperref[sec:construction-of-matrix-rep]{$f$の表現行列$A$は次のように構成される}。
  \begin{equation*}
    f(\vb*{v}_i) = \sum_{j=1}^m a_{ji} \vb*{w}_j \quad (1 \leq i \leq n)
  \end{equation*}
  したがって、任意の$i$に対し、
  \begin{equation*}
    \psi_k(f(\vb*{v}_i)) = \psi_k\left(\sum_{j=1}^m a_{ji} \vb*{w}_j\right) = \sum_{j=1}^m a_{ji} \psi_k(\vb*{w}_j)
  \end{equation*}
  ここで、$\{ \psi_k \}$は$\{ \vb*{w}_j \}$の双対基底なので、$\psi_k(\vb*{w}_j) = \delta_{kj}$より、
  \begin{equation*}
    \psi_k(f(\vb*{v}_i)) = a_{ki}
  \end{equation*}
  
  \br

  また、$f^*(\psi_k) \in V^*$は$V$上の線形汎関数なので、$V$の双対基底$\{ \phi_i \}$の線形結合として表せる。
  \begin{equation*}
    f^*(\psi_k) = \sum_{i=1}^n b_{ik} \phi_i
  \end{equation*}
  この係数$b_{ik}$を並べた行列を$B$とすると、$B$は$f^*$の表現行列である。
  
  \br
  
  このとき、
  \begin{equation*}
    f^*(\psi_k)(\vb*{v}_i) = \psi_k(f(\vb*{v}_i)) = a_{ki}
  \end{equation*}
  であり、一方、
  \begin{equation*}
    f^*(\psi_k)(\vb*{v}_i) = \sum_{j=1}^n b_{ji} \phi_j(\vb*{v}_i) = \sum_{j=1}^n b_{ji} \delta_{ij} = b_{ki}
  \end{equation*}
  でもあるから、$b_{ki} = a_{ki}$が成り立つ。すなわち、
  \begin{equation*}
    B = A^\top
  \end{equation*}
  である。 $\qed$
\end{proof}

\subsection{例:縦ベクトルと横ベクトルによる線形写像}

\todo{まとめ直す}

\br

$A$を$m \times n$型行列とする

\br

縦ベクトルに$A$を左からかけることによって定まる線形写像を次のように表す
\begin{equation*}
  f_A \colon \mathbb{R}^n \to \mathbb{R}^m \, (\vb*{v} \mapsto A \vb*{v})
\end{equation*}

\br

これと対照的に、横ベクトルに右から$A$をかけることによって定まる次の線形写像を\keyword{転置写像}と呼ぶ
\begin{equation*}
  f_A^* \colon {}^t\mathbb{R}^m \to {}^t\mathbb{R}^n \, (\phi \mapsto \phi A)
\end{equation*}

\br

横ベクトル$\phi A \in {}^t\mathbb{R}^n$は、次の合成写像の表現行列である
\begin{equation*}
  \mathbb{R}^n \xrightarrow{f_A} \mathbb{R}^m \xrightarrow{\phi} \mathbb{R}
\end{equation*}

\br

\begin{theorem}{転置写像と自然なペアリング}
  $A$を$m \times n$型行列とし、$\phi \in {}^t\mathbb{R}^m,\,\vb*{v} \in \mathbb{R}^n$に対して、
  \begin{equation*}
    \langle f_A^*(\phi), \vb*{v} \rangle = \langle \phi, f_A(\vb*{v}) \rangle
  \end{equation*}
\end{theorem}

\begin{proof}
  \begin{align*}
    \langle f_A^*(\phi), \vb*{v} \rangle & = (\phi A)(\vb*{v})                  \\
                                         & = \phi(A \vb*{v})                    \\
                                         & = \phi(f_A \vb({v}))                 \\
                                         & = \langle \phi, f_A(\vb*{v}) \rangle
  \end{align*}
  より、目的の等式が得られる $\qed$
\end{proof}

\br

\begin{theorem}{転置写像と座標関数}
  $A$を$m \times n$型行列とし、$y_1, \ldots, y_m \in {}^t\mathbb{R}^m$を${}^t\mathbb{R}^m$上の座標関数とするとき、
  \begin{equation*}
    f_A^*(y_i) = \sum_{j=1}^n a_{ij} x_j \quad (1 \leq i \leq m)
  \end{equation*}
\end{theorem}

\begin{proof}
  行ベクトルとしての観点から見ると、$y_i = {}^t\vb*{e}_i$として、
  \begin{equation*}
    f_A^*(y_i) = f_A^*({}^t\vb*{e}_i) = {}^t\vb*{e}_i A = \begin{pmatrix}
      a_{i1} & \cdots & a_{in}
    \end{pmatrix}
  \end{equation*}

  これは双対基底$x_j = {}^t\vb*{e}_j$を用いて、
  \begin{align*}
    f_A^*(y_i) & = \begin{pmatrix}
                     a_{i1} & \cdots & a_{in}
                   \end{pmatrix}          \\
               & = \sum_{j=1}^n a_{ij} {}^t\vb*{e}_j \\
               & = \sum_{j=1}^n a_{ij} x_j
  \end{align*}
  とも書ける $\qed$
\end{proof}

\br

\begin{theorem}{転置写像の表現行列}
  $A$を$m \times n$型行列とするとき、基底$\{y_1,\ldots y_m\},\,\{x_1,\ldots,x_n\}$に関する$f_A^*$の表現行列は${}^tA$である
\end{theorem}

\begin{proof}
  \todo{よくわからない}

  \br

  表現行列は、基底$\{y_i\}$の各元が、写像を通してどのような線形結合で$\{x_j\}$に写されるかを記述したものである

  \br

  すなわち、写像$f_A^*$の表現行列を求めることは、
  \begin{equation*}
    f_A^*(y_i) = \sum_{j=1}^n a_{ij} x_j \quad (1 \leq i \leq m)
  \end{equation*}
  において、係数$a_{ij}$を行列に並べることである

  \br

  ここで、$f_A^* \colon {}^t\mathbb{R}^m \to {}^t\mathbb{R}^n$において、
  \begin{itemize}
    \item 定義域の基底は$\{ y_1, \ldots, y_m \} \subset {}^t\mathbb{R}^m$
    \item 値域の基底は$\{ x_1, \ldots, x_n \} \subset {}^t\mathbb{R}^n$
  \end{itemize}

  \br

  先ほど示した等式
  \begin{equation*}
    f_A^*(y_i) = \sum_{j=1}^n a_{ij} x_j \quad (1 \leq i \leq m)
  \end{equation*}
  より、表現行列の第$i$列が、$f_A^*(y_i)$の係数ベクトル
  \begin{equation*}
    \begin{pmatrix}
      a_{i1} & \cdots & a_{in}
    \end{pmatrix}
  \end{equation*}
  を転置して縦ベクトルにしたものになる
\end{proof}

\end{document}
