\documentclass[../../../topic_linear-algebra]{subfiles}

\begin{document}

\sectionline
\section{双対空間と双対基底}

ここまでの話を、一般の線形空間$V$に拡張しよう。

\br

まず、$V$上の線形汎関数を次のように定義する。

\begin{definition}{線形汎関数}\label{def:linear-functional}
  $V$を$\mathbb{R}$上の線形空間とする。$V$から$\mathbb{R}$への線形写像$\phi\colon V \to \mathbb{R}^n$を$V$上の\keyword{線形汎関数}あるいは\keyword{線形形式}という。
\end{definition}

$V$から$\mathbb{R}$への線形写像、すなわち$V$上の線形汎関数全体の集合を考える。

\begin{definition}{双対空間}
  $V$上の線形汎関数全体の集合を$V$の\keyword{双対空間}といい、$V^*$と表す。
  \begin{equation*}
    V^* \coloneq \Hom(V, \mathbb{R}) = \{ \phi \colon V \to \mathbb{R} \mid \phi \text{は線形写像} \}
  \end{equation*}
\end{definition}

線形空間$V$が有限次元の場合は、選んでおいた$V$の基底に対して、\keywordJE{双対基底}{dual basis}という双対空間$V^*$の基底を考えることができる。

\begin{theorem}{双対基底の構成}\label{thm:dual-basis-construction}
  $V$を$n$次元の線形空間とし、$\{ \vb*{v}_1, \ldots, \vb*{v}_n \}$を$V$の基底とする。
  このとき、$\phi_j \in V^*$を次のように定める。
  \begin{equation*}
    \phi_j (\vb*{v}_i) = \delta_{ij}
  \end{equation*}
  このような$\phi_1, \ldots, \phi_n$は$V^*$の基底をなす。
\end{theorem}

この定理は\hyperref[thm:dual-basis-construction-Rn]{$V=\mathbb{R}^n$の場合}と同様に示すことができる。

また、この定理から次が成り立つ。

\begin{theorem}{双対空間の次元}\label{thm:dual-basis-dimension}
  $n$次元線形空間$V$の双対空間$V^*$の次元は、$V$の次元と等しい。
  \begin{equation*}
    \dim V = \dim V^* = n
  \end{equation*}
\end{theorem}

これより、$V$と$V^*$は線形同型であることがいえるが、この同型は基底に依存していることに注意しよう。

\br

一旦ここまでの話をまとめると、次のような関係が成り立っている。
\begin{equation*}
  \begin{tikzcd}
    V \arrow[r,leftrightarrow,"\textbf{?}"] \arrow[rd, "\textbf{基底による同型}"'] & (V^*)^* \\
    & V^* \arrow[u,"\textbf{基底による同型}"']
  \end{tikzcd}
\end{equation*}

\end{document}
