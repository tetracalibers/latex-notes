\documentclass[../../../topic_linear-algebra]{subfiles}

\usepackage{xr-hyper}
\externaldocument{../../../.tex_intermediates/topic_linear-algebra}

\begin{document}

\sectionline
\section{横ベクトルと座標関数}
\marginnote{\refbookA p120}

$n \times 1$型行列($n$次の縦ベクトル)全体の集合は$\mathbb{R}^n$と表された。

$1 \times n$型行列($n$次の横ベクトル)全体の集合を${}^t\mathbb{R}^n$と表すことにする。

\br

${}^t\mathbb{R}^n$の元は$1 \times n$型行列なので、$\mathbb{R}^n$から$\mathbb{R}$への線形写像(すなわち$\mathbb{R}^n$上の\keyword{線形汎関数})を表現している行列だと考えることができる。

\subsection{座標関数の表現行列}

基本ベクトルを転置したもの${}^t\vb*{e}_j \in {}^t\mathbb{R}^n$を縦ベクトル$\vb*{v} \in \mathbb{R}^n$にかけると、$\vb*{v}$の$j$番目の成分が得られる。

たとえば、$n=3,\,j=2$の場合、
\begin{equation*}
  {}^t\vb*{e}_2\begin{pmatrix}
    v_1 \\
    v_2 \\
    v_3
  \end{pmatrix} = \begin{pmatrix}
    0 & 1 & 0
  \end{pmatrix}
  \begin{pmatrix}
    v_1 \\
    v_2 \\
    v_3
  \end{pmatrix}
  = v_2
\end{equation*}
といった具合に、$2$番目の成分$v_2$が得られる。

\br

このように、ベクトル$\vb*{v} \in \mathbb{R}^n$に対して、その$j$番目の成分を返す\keyword{座標関数}を$x_j$と表記することにしよう。
\begin{equation*}
  \begin{array}{llll}
    x_j\colon & \mathbb{R}^n         & \longrightarrow & \mathbb{R}          \\
            & \rotatebox{90}{$\in$} &                 & \rotatebox{90}{$\in$} \\
            & \vb*{v}               & \longmapsto     & v_j = {}^t\vb*{e}_j \vb*{v}
  \end{array}
\end{equation*}

このとき、$x_j\colon \mathbb{R}^n \to \mathbb{R}$は$\mathbb{R}^n$上の\keyword{線形汎関数}である。

${}^t\vb*{e}_j \vb*{v}$を行列の積として見ると、横基本ベクトル${}^t\vb*{e}_j \in {}^t\mathbb{R}^n$は線形汎関数$x_j$の表現行列だと捉えることができる。

\begin{mindflow}
  \todo{「基底方向への正射影」という観点についても述べる?}
\end{mindflow}

\subsection{横ベクトルと線形汎関数の同一視}\label{sec:row-vector-functional}

任意の縦ベクトルは、基本ベクトル(標準基底)の線形結合として一意的に表現できる。
\begin{equation*}
  \ket{\vb*{v}} = \begin{pmatrix}
    v_1 \\
    \vdots \\
    v_n
  \end{pmatrix} = v_1 \vb*{e}_1 + \cdots + v_n \vb*{e}_n
\end{equation*}

同様に、任意の横ベクトルは、横基本ベクトルの線形結合として一意的に表現できる。
\begin{equation*}
  \bra{\vb*{a}} = \begin{pmatrix}
    a_1 & \cdots & a_n
  \end{pmatrix}
  = a_1 {}^t\vb*{e}_1 + \cdots + a_n {}^t\vb*{e}_n
\end{equation*}

\br

ここで、\secref{sec:inner-product-to-functional}で述べた、{横ベクトル$\bra{\vb*{a}}$は観測装置}という視点に戻って、縦ベクトルを入力したら$\vb*{a}$との内積を返す線形汎関数を$\phi$とおくと、
\begin{align*}
  \phi(\vb*{v}) &= \vb*{a}^\top \vb*{v} = \begin{pmatrix}
    a_1 & \cdots & a_n
  \end{pmatrix}
  \begin{pmatrix}
    v_1 \\
    \vdots \\
    v_n
  \end{pmatrix} \\
  &= a_1 v_1 + \cdots + a_n v_n \\
  & = a_1 {}^t\vb*{e}_1 \begin{pmatrix}
                               v_1    \\
                               \vdots \\
                               v_n
                             \end{pmatrix} + \cdots + a_n {}^t\vb*{e}_n \begin{pmatrix}
                                                                          v_1    \\
                                                                          \vdots \\
                                                                          v_n
                                                                        \end{pmatrix} \\
       & = a_1 x_1(\vb*{v}) + \cdots + a_n x_n(\vb*{v})
\end{align*}

よって、任意の線形汎関数$\phi \in (\mathbb{R}^n)^*$は、座標関数$x_1,\dots,x_n$の線型結合として表すことができる。
\begin{equation*}
  \phi = a_1 x_1 + \cdots + a_n x_n
\end{equation*}

また、$x_i$の表現行列が${}^t \vb*{e}_i$であることを思い出すと、
\begin{equation*}
  \phi = a_1 {}^t\vb*{e}_1 + \cdots + a_n {}^t\vb*{e}_n = \bra{\vb*{a}}
\end{equation*}
というように、線形汎関数$\phi$は横ベクトル$\bra{\vb*{a}}$と同一視することができる。

\br

$\{ {}^t\vb*{e}_1, \ldots, {}^t\vb*{e}_n \}$を基底としてどんな横ベクトルも表現できることは、$\{ x_1, \ldots, x_n \}$を基底としてどんな線形汎関数も表現できることに対応する。

これより、横ベクトルの空間${}^t \mathbb{R}^n$と、線形汎関数の空間$(\mathbb{R}^n)^*$は、同じ空間とみなすことができる。

\sectionline
\section{縦ベクトルと横ベクトルの双対性}
\marginnote{\refbookA p121〜122}

$\{ \vb*{u}_1, \ldots, \vb*{u}_n \}$を$\mathbb{R}^n$の基底とするとき、任意の縦ベクトル$\vb*{v} \in \mathbb{R}^n$は、
\begin{equation*}
  \vb*{v} = v_1 \vb*{u}_1 + \cdots + v_n \vb*{u}_n
\end{equation*}
という線形結合で表すことができる。

\br

ここで、$v_1, \ldots, v_n$は基底$\{ \vb*{u}_1, \ldots, \vb*{u}_n \}$に関する$\vb*{v}$の座標である。

このうち、$j$番目の座標$v_j$を取得する関数を$\phi_j\colon\mathbb{R}^n \to \mathbb{R}$と定めると、$\phi_j$は、
\begin{equation*}
  \phi_j(\vb*{u}_i) = \delta_{ij}
\end{equation*}
を満たし、$\{ \phi_1, \ldots, \phi_n \}$が$(\mathbb{R}^n)^*$の基底となる。

\br

このとき、$(\mathbb{R}^n)^*$の元(線形汎関数)を横ベクトルと同一視すると、任意の横ベクトル$\phi \in {}^t\mathbb{R}^n$は、
\begin{equation*}
  \phi = c_1 \phi_1 + \cdots + c_n \phi_n
\end{equation*}
という線形結合で表すことができる。

\br

ここで、$c_1, \ldots, c_n$は基底$\{ \phi_1, \ldots, \phi_n \}$に関する$\phi$の座標である。

このうち、$j$番目の座標$c_j$を取得する関数を$\psi_j\colon {}^t\mathbb{R}^n \to \mathbb{R}$と定めると、$\psi_j$は、
\begin{equation*}
  \psi_j(\phi_i) = \delta_{ij}
\end{equation*}
を満たし、$\{ \psi_1, \ldots, \psi_n \}$が$({}^t\mathbb{R}^n)^*$の基底となる。

\br

さて、基底を変えれば座標も変わってしまうので、$\psi_j$はあくまでも基底が$\{ \phi_1, \ldots, \phi_n \}$のときの横ベクトルの座標を返す関数である。

さらに、$\phi_j$は$\mathbb{R}^n$の基底が$\{ \vb*{u}_1, \ldots, \vb*{u}_n \}$のときの縦ベクトルの座標を返す関数である。

\br

つまり、$\psi_j$は$\mathbb{R}^n$の基底$\{ \vb*{u}_1, \ldots, \vb*{u}_n \}$に依存しているので、$\vb*{u}_j \in \mathbb{R}^n$を入力として$\psi_j$を定める関数$\iota$を考えてみる。
\begin{equation*}
  \begin{array}{lllc}
    \iota\colon & \mathbb{R}^n         & \longrightarrow & ({}^t\mathbb{R}^n)^*          \\
            & \rotatebox{90}{$\in$} &                 & \rotatebox{90}{$\in$} \\
            & \vb*{u}_j               & \longmapsto     & \psi_j
  \end{array}
\end{equation*}

$\iota$を用いると、次のように書ける。
\begin{equation*}
  \iota(\vb*{u}_j) = \psi_j
\end{equation*}

このとき、基底に対して座標は一意的であり、基底が変わると座標が変わることから、
\begin{enumerate}[label=\romanlabel]
  \item 基底$\{ \vb*{u}_j \}_{j=1}^n$を固定すれば、$\iota(\vb*{u}_j) = \psi_j$を満たす座標$\{ \psi_j \}_{j=1}^n$は一意に定まる
  \item 座標$\{ \psi_j \}_{j=1}^n$を固定すれば、$\iota(\vb*{u}_j) = \psi_j$を満たす基底$\{ \vb*{u}_j \}_{j=1}^n$は一意に定まる
\end{enumerate}
という2通りの見方ができる。

\br

このように、$\vb*{u}_j \in \mathbb{R}^n$と$\psi_j \in ({}^t\mathbb{R}^n)^*$には、「互いに測り、測られる」という対称性がある。
このような対称性を\keyword{双対性}という。

\br

この性質を意識し、${}^t\mathbb{R}^n$を$\mathbb{R}^n$の\keyword{双対空間}という。
\begin{equation*}
  \begin{tikzcd}
    \mathbb{R}^n \arrow[r,leftrightarrow,"\textbf{同一視}"] \arrow[rd, "\textbf{裏返し}"'] & ({}^t\mathbb{R}^n)^* \\
    & {}^t\mathbb{R}^n \arrow[u,"\textbf{裏返し}"']
  \end{tikzcd}
\end{equation*}

\keyword{双対}とは、「裏返しにした関係」と解釈できる。

\br

${}^t\mathbb{R}^n$が$\mathbb{R}^n$の双対空間であるとは、「横ベクトルの空間${}^t\mathbb{R}^n$を裏返しにしたもの$({}^t\mathbb{R}^n)^*$は、縦ベクトルの空間$\mathbb{R}^n$と同一視できる」ということである。

\br

逆に、$\mathbb{R}^n$は${}^t\mathbb{R}^n$の双対空間である。
「縦ベクトルの空間$\mathbb{R}^n$を裏返しにしたもの$(\mathbb{R}^n)^*$は、横ベクトルの空間${}^t\mathbb{R}^n$と同一視できる」ということでもある。

すなわち、線形汎関数の空間$(\mathbb{R}^n)^*$を横ベクトルの空間${}^t\mathbb{R}^n$と同一視できる。

\br

そこで、${}^t\mathbb{R}^n$を$(\mathbb{R}^n)^*$に書き換えると、
\begin{equation*}
  \begin{tikzcd}
    \mathbb{R}^n \arrow[r,leftrightarrow,"\textbf{同一視}"] \arrow[rd, "\textbf{裏返し}"'] & ((\mathbb{R}^n)^*)^* \\
    & (\mathbb{R}^n)^* \arrow[u,"\textbf{裏返し}"']
  \end{tikzcd}
\end{equation*}
という関係が見えてくる。
$(\mathbb{R}^n)^*$を$\mathbb{R}^n$の\keyword{双対空間}という。

表$\mathbb{R}^n$の裏は$(\mathbb{R}^n)^*$であり、裏の裏$((\mathbb{R}^n)^*)^*$は表$\mathbb{R}^n$になる。

\end{document}
