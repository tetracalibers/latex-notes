\documentclass[../../../topic_linear-algebra]{subfiles}

\begin{document}

\sectionline
\section{線形関数}
\marginnote{\refbookA p120}

横ベクトル($1 \times n$型行列)を縦ベクトル($n \times 1$型行列)にかけると、$1 \times 1$のスカラー値が得られる

\begin{equation*}
  \begin{pmatrix}
    a_1 & \cdots & a_n
  \end{pmatrix} \begin{pmatrix}
    x_1    \\
    \vdots \\
    x_n
  \end{pmatrix}
  = a_1 x_1 + \cdots + a_n x_n
\end{equation*}

\br

これは、縦ベクトルを入力とする\keyword{線形関数}($\mathbb{R}^n$から$\mathbb{R}$への線形写像)と見なすことができる

列ベクトルを$\vb*{v}$、この線形関数を$\phi$とすると、
\begin{equation*}
  \phi(\vb*{v}) = a_1 x_1 + \cdots + a_n x_n
\end{equation*}
と書ける

\sectionline
\section{横ベクトルの集合}
\marginnote{\refbookA p120}

$n \times 1$型行列($n$次の縦ベクトル)全体の集合は$\mathbb{R}^n$と表された

$1 \times n$型行列($n$次の横ベクトル)全体の集合を${}^t\mathbb{R}^n$と表すことにする

\br

${}^t\mathbb{R}^n$の元は$1 \times n$型行列なので、$\mathbb{R}^n$から$\mathbb{R}$への線形写像(すなわち$\mathbb{R}^n$上の\keyword{線形関数})を表現している行列だと考えることができる

\subsection{座標関数の表現行列}

基本ベクトルを転置したもの${}^t\vb*{e}_j$を列ベクトルにかけると、$j$番目の成分が得られる

たとえば、$n=3,\,j=2$の場合、
\begin{equation*}
  {}^t\vb*{e}_2\begin{pmatrix}
    x_1 \\
    x_2 \\
    x_3
  \end{pmatrix} = \begin{pmatrix}
    0 & 1 & 0
  \end{pmatrix}
  \begin{pmatrix}
    x_1 \\
    x_2 \\
    x_3
  \end{pmatrix}
  = x_2
\end{equation*}

このように、ベクトル$\vb*{v} \in \mathbb{R}^n$に対して、$j$番目の成分を返す関数を\keyword{座標関数}$x_j$という

\br

横基本ベクトル${}^t\vb*{e}_j \in {}^t\mathbb{R}^n$は、座標関数$x_j\colon \mathbb{R}^n \to \mathbb{R}$の表現行列になっている

\subsection{基底としての座標関数}

任意の横ベクトルは、横基本ベクトルの線形結合として一意的に表現できる

\begin{equation*}
  \begin{pmatrix}
    a_1 & \cdots & a_n
  \end{pmatrix}
  = a_1 {}^t\vb*{e}_1 + \cdots + a_n {}^t\vb*{e}_n
\end{equation*}

これを用いると、
\begin{align*}
  \phi & = \begin{pmatrix}
             a_1 & \cdots & a_n
           \end{pmatrix} \begin{pmatrix}
                           x_1    \\
                           \vdots \\
                           x_n
                         \end{pmatrix}                                                \\
       & = a_1 {}^t\vb*{e}_1 \begin{pmatrix}
                               x_1    \\
                               \vdots \\
                               x_n
                             \end{pmatrix} + \cdots + a_n {}^t\vb*{e}_n \begin{pmatrix}
                                                                          x_1    \\
                                                                          \vdots \\
                                                                          x_n
                                                                        \end{pmatrix} \\
       & = a_1 x_1 + \cdots + a_n x_n
\end{align*}
となる

\br

任意の線形関数$\phi \in {}^t\mathbb{R}^n$は、座標関数$x_1,\dots,x_n$の線型結合として
\begin{equation*}
  \phi = a_1 x_1 + \cdots + a_n x_n
\end{equation*}
のように一意的に書くことができる

\br

つまり、$\{x_1,\dots,x_n\}$は${}^t\mathbb{R}^n$の\keyword{基底}である

\br

また、縦ベクトルが基底の線形結合で表現できたのと同様に、$\phi$は横ベクトル$(a_1,\dots,a_n)$と同一視できる

\sectionline
\section{自然なペアリング}
\marginnote{\refbookA p120}

$\phi \in {}^t\mathbb{R}^n$と$\vb*{v} \in \mathbb{R}^n$に対して、
\begin{equation*}
  \langle \phi, \vb*{v} \rangle = \phi(\vb*{v})
\end{equation*}
とおく

\br

これは線形関数$\phi$に$\vb*{v}$を入力して得られる値を表しているが、$\phi$を横ベクトル、$\vb*{v}$を縦ベクトルとみれば、$\langle \phi, \vb*{v} \rangle$は行列としての積$\phi \cdot \vb*{v}$と一致している

\br

左辺の記法$\langle \phi, \vb*{v} \rangle$を用いると、見通しの良い議論ができることがある

これを\keyword{自然なペアリング}と呼ぶ

\end{document}
