\documentclass[../../../topic_linear-algebra]{subfiles}

\begin{document}

\sectionline
\section{ペアリングの記号}
\marginnote{\refbookA p120}

\begin{mindflow}
  \placeholder{改編予定}
\end{mindflow}

$\phi \in V^*$と$\vb*{v} \in V$に対して、線形関数$\phi$に$\vb*{v}$を入力して得られる値を次のように書くことがある。
\begin{equation*}
  \langle \phi, \vb*{v} \rangle \coloneq \phi(\vb*{v})
\end{equation*}
この記法を\keyword{ペアリング}と呼ぶことにする。

\subsection{例:行列の積を表すペアリング}

$V = \mathbb{R}^n$の場合、その双対空間は$V^* = {}^t\mathbb{R}^n$である。

\br

$\phi \in {}^t\mathbb{R}^n$を横ベクトル、$\vb*{v} \in \mathbb{R}^n$を縦ベクトルとみれば、$\langle \phi, \vb*{v} \rangle$は行列としての積$\phi \cdot \vb*{v}$と一致している。

\end{document}
