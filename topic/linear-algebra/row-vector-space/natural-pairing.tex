\documentclass[../../../topic_linear-algebra]{subfiles}

\begin{document}

\sectionline
\section{自然なペアリング}
\marginnote{\refbookA p120}

$\phi \in {}^t\mathbb{R}^n$と$\vb*{v} \in \mathbb{R}^n$に対して、
\begin{equation*}
  \langle \phi, \vb*{v} \rangle = \phi(\vb*{v})
\end{equation*}
とおく

\br

これは線形関数$\phi$に$\vb*{v}$を入力して得られる値を表しているが、$\phi$を横ベクトル、$\vb*{v}$を縦ベクトルとみれば、$\langle \phi, \vb*{v} \rangle$は行列としての積$\phi \cdot \vb*{v}$と一致している

\br

左辺の記法$\langle \phi, \vb*{v} \rangle$を用いると、見通しの良い議論ができることがある

これを\keyword{自然なペアリング}と呼ぶ

\end{document}
