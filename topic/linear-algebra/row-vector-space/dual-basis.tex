\documentclass[../../../topic_linear-algebra]{subfiles}

\begin{document}

\sectionline
\section{双対基底}
\marginnote{\refbookA p120〜121 \\ \refbookG p58〜59}

$\mathbb{R}^n$の基底$\{ \vb*{v}_1,\ldots,\vb*{v}_n \}$に対して、
\begin{equation*}
  \phi_i(\vb*{v}_j) = \delta_{ij} \quad (i,j=1,\ldots,n)
\end{equation*}
を満たす線形写像$\phi_i:\mathbb{R}^n \to \mathbb{R}$を考える

\br

このとき、任意のベクトル $\vb*{v} \in \mathbb{R}^n$を
\begin{equation*}
  \vb*{v} = \sum_{j=1}^n a_j \vb*{v}_j
\end{equation*}
とおくと、
\begin{equation*}
  \phi_i(\vb*{v}) = \phi_i\left(\sum_{j=1}^{n} a_j \vb*{v}_j\right) = \sum_{j=1}^{n} a_j \phi_i(\vb*{v}_j) = \sum_{j=1}^{n} a_j \delta_{ij} = a_i
\end{equation*}
となるから、$\phi_i$は基底$\{ \vb*{v}_1,\ldots,\vb*{v}_n \}$に関する第$i$座標を表す関数である

\br

任意の線形関数$\phi \in {}^t\mathbb{R}^n$は、座標関数の線型結合として一意的に表現できるから、次が成り立つ

\begin{theorem}{${}^t\mathbb{R}^n$の双対基底の存在}
  $\mathbb{R}^n$の基底$\{ \vb*{v}_1,\ldots,\vb*{v}_n \}$に対して、$\phi_i(\vb*{v}_j) = \delta_{ij}$によって$\phi_i \in {}^t\mathbb{R}^n$を定める

  このとき、任意の$\phi \in {}^t\mathbb{R}^n$を$\phi_1,\ldots,\phi_n$の線形結合
  \begin{equation*}
    \phi = \sum_{i=1}^{n} \phi(\vb*{v}_i) \phi_i
  \end{equation*}
  として一意的に書くことができる

  すなわち、$\{ \phi_1,\ldots,\phi_n \}$は${}^t \mathbb{R}^n$の基底である
\end{theorem}

\begin{proof}
  \begin{subpattern}{\bfseries $\phi_1,\ldots,\phi_n$が線型独立}
    線形関係式
    \begin{equation*}
      \sum_{j=1}^{n} c_j \phi_j = 0
    \end{equation*}
    があるとすると、
    \begin{equation*}
      \begin{WithArrows}
        \left( \displaystyle\sum_{j=1}^{n} c_j \phi_j \right)(\vb*{v}_i) & = 0 \Arrow{$\phi$の線形性} \\
        \displaystyle\sum_{j=1}^{n} c_j \phi_j(\vb*{v}_i) & = 0 \Arrow{$\phi_j(\vb*{v}_i) = \delta_{ij}$} \\
        \displaystyle\sum_{j=1}^{n} c_j \delta_{ij} & = 0
      \end{WithArrows}
    \end{equation*}
    左辺の和は、$\delta_{ij}$の定義より、$j=i$の項のみ生き残って、
    \begin{equation*}
      c_i = 0 \quad (i=1,\ldots,n)
    \end{equation*}
    が得られる $\qed$
  \end{subpattern}

  \begin{subpattern}{$\langle \phi_1, \ldots, \phi_n \rangle = {}^t\mathbb{R}^n$}
    任意の$\vb*{v} \in \mathbb{R}^n$に対して、
    \begin{equation*}
      \vb*{v} =\sum_{i=1}^{n} c_i \vb*{v}_i
    \end{equation*}
    と書く

    \br

    $\psi \in {}^t\mathbb{R}^n$を任意にとると、$\psi$と$\phi$の線形性により、
    \begin{align*}
      \psi(\vb*{v}) & = \psi\left(\sum_{i=1}^{n} c_i \vb*{v}_i\right)              \\
                    & = \sum_{i=1}^{n} c_i \psi(\vb*{v}_i)                         \\
                    & = \sum_{i=1}^{n} \phi_i(\vb*{v}) \psi(\vb*{v}_i)             \\
                    & = \sum_{i=1}^{n} \psi(\vb*{v}_i) \phi_i(\vb*{v})             \\
                    & = \left( \sum_{i=1}^n \psi(\vb*{v}_i)\phi_i\right) (\vb*{v})
    \end{align*}

    よって、
    \begin{equation*}
      \psi = \sum_{i=1}^{n} \psi(\vb*{v}_i) \phi_i
    \end{equation*}

    上式で、任意の$\psi \in {}^t\mathbb{R}^n$を$\phi_1,\ldots,\phi_n$の線形結合で書けることが示せたので、$\langle \phi_1, \ldots, \phi_n \rangle$は${}^t\mathbb{R}^n$を張ることがわかる $\qed$
  \end{subpattern}
\end{proof}

\br

$\mathbb{R}^n$の基底$\{ \vb*{v}_1,\ldots,\vb*{v}_n \}$に対して、上の定理で定まる${}^t\mathbb{R}^n$の基底$\{ \phi_1,\ldots,\phi_n \}$を\keyword{双対基底}という

\end{document}
