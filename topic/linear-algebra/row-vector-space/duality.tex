\documentclass[../../../topic_linear-algebra]{subfiles}

\begin{document}

\sectionline
\section{双対性}
\marginnote{\refbookR p237 \\ \refbookG p60〜61}

\begin{mindflow}
  \placeholder{改編予定}
\end{mindflow}

$V$と$(V^*)^*$の間には、線形同型写像$\iota \colon V \to (V^*)^*$が存在する。

このことから、\hyperref[def:linear-subspace-isomorphism]{$V$と$(V^*)^*$は線形同型}であることがいえる。

\br

このように、$V$が有限次元の場合は、$V$と$(V^*)^*$は自然に同一視することができるので、
\begin{emphabox}
  \begin{spacebox}
    \begin{center}
      双対空間$V^*$の双対空間$(V^*)^*$は$V$自身である
    \end{center}
  \end{spacebox}
\end{emphabox}
ということになる。

\br

たとえるなら、$V$と$V^*$は表と裏のような関係になっていて、裏の裏$(V^*)^*$は表$V$である。
このような関係を\keyword{双対}と呼ぶ。

\subsection{ペアリングの記号が表す双対}

$V$を$V^*$の双対と考えても、$V^*$を$V$の双対と考えてもよい。

ペアリングの記号は、この平等さを表す記法ともいえる。

\br

$\phi \in V^*$を与えたとき、それに$V$の任意の元$\vb*{v}$を入力したもの$\langle \phi, \vb*{v} \rangle$を考えることができる。
\begin{equation*}
  \langle \phi, \vb*{v} \rangle = \phi(\vb*{v})
\end{equation*}

逆に$\vb*{v} \in V$を与えたとき、$\iota(\vb*{v})$に$V^*$の任意の元$\phi$を入力したもの$\langle \vb*{v}, \phi \rangle$を考えることもできる。
\begin{equation*}
  \langle \vb*{v}, \phi \rangle = \iota(\vb*{v})(\phi)
\end{equation*}

\br

どちらから考えても、具体的な$\vb*{v}$と$\phi$を与えたときに得られるスカラーは同じである。
\begin{equation*}
  \langle \phi, \vb*{v} \rangle = \langle \vb*{v}, \phi \rangle = \phi(\vb*{v}) = \iota(\vb*{v})(\phi)
\end{equation*}

\br

また、$V^*$の双対空間$(V^*)^*$は$V$と同一視できるのだから、$V$の基底$\{ \vb*{v}_i \}_{i=1}^n$の双対基底を$\{ \phi_i \}_{i=1}^n$とするとき、$\{ \phi_i \}_{i=1}^n$の双対基底は$\{ \vb*{v}_i \}_{i=1}^n$と同一視できる。
\begin{equation*}
  \langle \phi_i, \vb*{v}_j \rangle = \langle \vb*{v}_j, \phi_i \rangle = \delta_{ij}
\end{equation*}

\end{document}
