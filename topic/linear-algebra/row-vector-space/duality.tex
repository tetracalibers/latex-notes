\documentclass[../../../topic_linear-algebra]{subfiles}

\begin{document}

\sectionline
\section{縦ベクトル空間と横ベクトル空間の双対性}
\marginnote{\refbookA p122}

$\{ \phi_i \}_{i=1}^n$を${}^t\mathbb{R}^n$の基底とする

\br

写像$\iota$を$\vb*{v}_i \mapsto \langle -, \vb*{v}_i \rangle$と定めると、
\begin{equation*}
  \psi_i(\phi_j) = \langle \phi_j, \vb*{v}_i \rangle = \delta_{ij}
\end{equation*}
を満たす$\psi_i \in ({}^t\mathbb{R}^n)^*$が、各$1 \leq i \leq n$に対して定まるといえる

\br

一方で、$\iota$が単射であることから、$\iota(\vb*{v}_i) = \psi_i$を満たす$\vb*{v}_i$が一意的に存在する

単射とは、$\iota(\vb*{v}_i) = \iota(\vb*{v}_j) \Longrightarrow \vb*{v}_i = \vb*{v}_j$
という性質であり、ある$\psi_i$に対して、$\iota(\vb*{v}_i) = \psi_i$を満たす$\vb*{v}_i$はただ一つしか存在しないことを意味する

\br

したがって、$\iota(\vb*{v}_i)$は、$\phi \in {}^t\mathbb{R}^n$に対して$\langle \phi, \vb*{v}_i \rangle$を返す線形関数である
\begin{equation*}
  \iota(\vb*{v}_i) = \psi_i(\phi) = \langle \phi, \vb*{v}_i \rangle
\end{equation*}

\br

$\{ \vb*{v}_i \}_{i=1}^n$は$\mathbb{R}^n$の基底であり、これを$\{ \phi_i \}_{i=1}^n$の\keyword{双対基底}という

\br

定義を考えると、$\{ \vb*{v}_i \}_{i=1}^n$の双対基底は$\{ \phi_i \}_{i=1}^n$になっていることがわかる

\br

このように、縦ベクトル空間と横ベクトル空間とは、表と裏のような関係になっていて、裏の裏は表である

こういう状況を\keyword{双対性}と呼ぶ

\end{document}
