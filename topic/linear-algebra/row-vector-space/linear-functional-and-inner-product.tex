\documentclass[../../../topic_linear-algebra]{subfiles}

\usepackage{xr-hyper}
\externaldocument{../../../.tex_intermediates/topic_linear-algebra}

\begin{document}

\sectionline
\section{内積から線形汎関数へ}\label{sec:inner-product-to-functional}
\marginnote{
  \refbookA p120 \\ \refbookR p232〜233 \\
  \refweb{線形代数の基礎のキソ}{https://www1.econ.hit-u.ac.jp/kawahira/courses/kiso/01-senkei.pdf}
}

横ベクトル($1 \times n$型行列)を縦ベクトル($n \times 1$型行列)にかけると、$1 \times 1$のスカラー値が得られる。
\begin{equation*}
  \begin{pmatrix}
    a_1 & \cdots & a_n
  \end{pmatrix} \begin{pmatrix}
    v_1    \\
    \vdots \\
    v_n
  \end{pmatrix}
  = a_1 v_1 + \cdots + a_n v_n
\end{equation*}

上の式は、数ベクトル空間の内積そのものである。
\begin{equation*}
  \braket{\vb*{a}|\vb*{v}} = \vb*{a}^\top\vb*{v} = a_1 v_1 + \cdots + a_n v_n
\end{equation*}

\br

さて、\hyperref[sec:inner-product-as-observer]{「観測装置としての内積」}の章で述べたように、
\begin{emphabox}
  \begin{spacebox}
    \begin{center}
      内積$\braket{\vb*{a}|\vb*{v}}$は、観測装置$\bra{\vb*{a}}$によるベクトル$\ket{\vb*{v}}$の測定結果
    \end{center}
  \end{spacebox}
\end{emphabox}
という捉え方もできる。

\br

ここで、観測装置である横ベクトル$\bra{\vb*{a}}$を、縦ベクトル$\ket{\vb*{v}}$から内積を返す関数$\phi_{\vb*{a}}$とみることにしよう。
\begin{equation*}
  \phi_{\vb*{a}}(\vb*{v}) = \braket{\vb*{a}|\vb*{v}} = a_1 v_1 + \cdots + a_n v_n
\end{equation*}

$\phi_{\vb*{a}}$は、縦ベクトル$\vb*{v}$を入力とし、スカラー値$\braket{\vb*{a}|\vb*{v}}$を返す、$\mathbb{R}^n$から$\mathbb{R}$への写像である。

\br

さらに、内積の双線形性から、$\phi_{\vb*{a}}$は線形写像であることがわかる。
\begin{align*}
  \phi_{\vb*{a}}(c_1 \vb*{v}_1 + c_2 \vb*{v}_2) &= (\vb*{a}, c_1 \vb*{v}_1 + c_2 \vb*{v}_2) \\
  &= c_1 (\vb*{a}, \vb*{v}_1) + c_2 (\vb*{a}, \vb*{v}_2) \\
  &= c_1 \phi_{\vb*{a}}(\vb*{v}_1) + c_2 \phi_{\vb*{a}}(\vb*{v}_2)
\end{align*}

この関数$\phi_{\vb*{a}}$は、\keyword{線形汎関数}と呼ばれる写像の一例である。

\begin{definition}{$\mathbb{R}^n$上の線形汎関数}
  $\mathbb{R}^n$上の関数$\phi \colon \mathbb{R}^n \to \mathbb{R}$が線形写像であるとき、$\phi$を$\mathbb{R}^n$上の\keyword{線形汎関数}あるいは\keyword{線形形式}という。
\end{definition}

\sectionline
\section{線形汎関数のベクトル表示}

$\mathbb{R}^n$上の線形汎関数は、すべて内積から定めることができる。

\begin{theorem*}{$\mathbb{R}^n$上の線形汎関数の内積による表現}
  $\mathbb{R}^n$上の任意の線形汎関数$\psi\colon \mathbb{R}^n \to \mathbb{R}$に対し、ある$\vb*{a} \in \mathbb{R}^n$がただ一つ存在して、次を満たす。
  \begin{equation*}
    \psi = \phi_{\vb*{a}} = \braket{\vb*{a}|\cdot}
  \end{equation*}
\end{theorem*}

\begin{proof}
  $\mathbb{R}^n$の標準基底を$\{\vb*{e}_1, \ldots, \vb*{e}_n\}$とする。
  
  このとき、任意のベクトル$\vb*{v} \in \mathbb{R}^n$は、次のように表される。
  \begin{equation*}
    \vb*{v} = v_1 \vb*{e}_1 + \cdots + v_n \vb*{e}_n
  \end{equation*}
  
  これに$\psi$を作用させると、線形汎関数$\psi$は線形性をもつので、
  \begin{align*}
    \psi(\vb*{v}) &= \psi(v_1 \vb*{e}_1 + \cdots + v_n \vb*{e}_n) \\
    &= v_1 \psi(\vb*{e}_1) + \cdots + v_n \psi(\vb*{e}_n) \\
    &= \begin{pmatrix}
      \psi(\vb*{e}_1) & \cdots & \psi(\vb*{e}_n)
    \end{pmatrix} \begin{pmatrix}
      v_1 \\
      \vdots \\
      v_n
    \end{pmatrix}
  \end{align*}
  
  ここで、
  \begin{equation*}
    \vb*{a} = \begin{pmatrix}
      \psi(\vb*{e}_1) & \cdots & \psi(\vb*{e}_n)
    \end{pmatrix}
  \end{equation*}
  とおけば、次が成り立つ。
  \begin{equation*}
    \psi(\vb*{v}) = \braket{\vb*{a}|\vb*{v}} = \phi_{\vb*{a}}(\vb*{v})
  \end{equation*}
  
  $\vb*{v}$は任意のベクトルなので、
  \begin{equation*}
    \psi = \phi_{\vb*{a}} = \braket{\vb*{a}|\cdot}
  \end{equation*}
  となるような$\vb*{a} \in \mathbb{R}^n$の存在が示された。
  
  \br
  
  さらに、次式を振り返ると、$\psi$が決まれば$\vb*{a}$が一意に定まることがわかる。
  \begin{equation*}
    \vb*{a} = \begin{pmatrix}
      \psi(\vb*{e}_1) & \cdots & \psi(\vb*{e}_n)
    \end{pmatrix}
  \end{equation*}
  よって、$\psi$に対して$\vb*{a}$はただ一つ存在する。 $\qed$
\end{proof}

上の定理の証明で現れた次の式は、2通りの読み方ができる。
\begin{equation*}
  \vb*{a} = \begin{pmatrix}
    \psi(\vb*{e}_1) & \cdots & \psi(\vb*{e}_n)
  \end{pmatrix}
\end{equation*}

$\psi$が決まれば、$\psi(\vb*{e}_1), \ldots, \psi(\vb*{e}_n)$の値が決まるので、$\vb*{a}$がただ一つ定まる。

逆に、\thmref{thm:linear-map-equality-on-basis}より、基底$\{\vb*{e}_1, \ldots, \vb*{e}_n\}$に対する$\psi$の値が決まれば$\psi$の形が決まるので、上の式のように$\vb*{a}$を定めれば、$\vb*{a}$に対応して$\psi$の形がただ一つに定まることになる。

\br

まとめると、
\begin{itemize}
  \item すべてのベクトル$\vb*{a}$は線形汎関数$\psi$をひとつ定める
  \item すべての線形汎関数$\psi$はベクトル$\vb*{a}$をひとつ定める
\end{itemize}

$\vb*{a}$から$\psi$への対応は一対一であり、$\psi$から$\vb*{a}$への対応も一対一である。

すなわち、$\mathbb{R}^n$のベクトルと$\mathbb{R}^n$上の線形汎関数の間には、\keyword{全単射}が存在する。

\br

全単射な対応は、本来同じものに「異なる表現を与えている」と捉えることができる。

\subsection{縦ベクトルと横ベクトルによる線形汎関数の表現}\label{sec:column-row-functional}

次の式も、先ほどの定理の証明で現れたものである。
\begin{equation*}
  \psi(\vb*{v}) = \begin{pmatrix}
    \psi(\vb*{e}_1) & \cdots & \psi(\vb*{e}_n)
  \end{pmatrix} \begin{pmatrix}
    v_1 \\
    \vdots \\
    v_n
  \end{pmatrix}
\end{equation*}

この式もまた、2通りの読み方ができる。

\br

$\vb*{a}$を横ベクトルとみるなら、
\begin{equation*}
  \psi(\vb*{v}) = \wavelabelmath[Rhodamine]{\begin{pmatrix}
    \psi(\vb*{e}_1) & \cdots & \psi(\vb*{e}_n)
  \end{pmatrix}}{$\vb*{a}$} \begin{pmatrix}
    v_1 \\
    \vdots \\
    v_n
  \end{pmatrix} = \vb*{a} \vb*{v}
\end{equation*}

この見方では、線形汎関数は横ベクトル$\vb*{a}$との「行列としての積」である。

線形汎関数を行列の積として定義すれば、「横」ベクトル$\vb*{a}$が線形汎関数の表現行列に相当すると捉えられる。

\br

一方、$\vb*{a}$を縦ベクトルとみるなら、
\begin{equation*}
  \psi(\vb*{v}) = \wavelabelmath[Cerulean]{\begin{pmatrix}
    \psi(\vb*{e}_1) & \cdots & \psi(\vb*{e}_n)
  \end{pmatrix}}{$\vb*{a}^\top$} \begin{pmatrix}
    v_1 \\
    \vdots \\
    v_n
  \end{pmatrix} = \vb*{a}^\top \vb*{v} = (\vb*{a}, \vb*{v})
\end{equation*}

この見方では、線形汎関数は縦ベクトル$\vb*{a}$との「内積」である。

線形汎関数を内積として定義すれば、「縦」ベクトル$\vb*{a}$が線形汎関数の表現行列に相当すると捉えられる。

\br

このように、線形汎関数という同じものに対して、横ベクトルと縦ベクトルは「異なる表現を与えている」とも解釈できる。

横ベクトルと縦ベクトルが\keyword{転置}という関係で結ばれていることで、この2通りの見方が可能になる。

\end{document}
