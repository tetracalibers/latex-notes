\documentclass[../../../topic_linear-algebra]{subfiles}

\usepackage{xr-hyper}
\externaldocument{../../../.tex_intermediates/topic_linear-algebra}

\begin{document}

\sectionline
\section{双対写像}
\marginnote{\refbookR p237〜238 \\ \refbookG p61〜62 \\ \refbookA p122〜123}

線形空間の間の線形写像が与えられると、双対空間の間の線形写像を定めることができる。

\subsection{数ベクトル空間の場合}

$A$を$m \times n$型行列とする。

$A$を左からかけることによって定義される線形写像を$f_A$とする。
\begin{equation*}
  \begin{array}{lclc}
    f_A \colon & \mathbb{R}^n         & \longrightarrow & \mathbb{R}^m          \\
            & \rotatebox{90}{$\in$} &                 & \rotatebox{90}{$\in$} \\
            & \vb*{v}               & \longmapsto     & A \vb*{v}
  \end{array}
\end{equation*}

一方、横ベクトルに$A$を右からかけることによって定義される線形写像を$f_A^*$とする。
\begin{equation*}
  \begin{array}{lclc}
    f_A^* \colon & {}^t\mathbb{R}^m & \longrightarrow & {}^t\mathbb{R}^n \\
            & \rotatebox{90}{$\in$} &                 & \rotatebox{90}{$\in$} \\
            & \phi               & \longmapsto     & \phi A
  \end{array}
\end{equation*}

ここで、横ベクトルの空間を線形汎関数の空間と同一視して、次のように書こう。
\begin{equation*}
  \begin{array}{lclc}
    f_A^* \colon & (\mathbb{R}^m)^* & \longrightarrow & (\mathbb{R}^n)^* \\
            & \rotatebox{90}{$\in$} &                 & \rotatebox{90}{$\in$} \\
            & \phi               & \longmapsto     & \phi A
  \end{array}
\end{equation*}

\begin{handout}[補足:ベクトルと行列の積の次元]
  $n$次元縦ベクトル$\vb*{v} \in \mathbb{R}^n$に対して$A$を左からかけたものは、$m$次元縦ベクトルとなる。
  \begin{equation*}
    \underset{\eqnmarkbox[cyan]{_dim_m}{m} \times \eqnmarkbox[magenta]{dim_n1}{n}}{A} \cdot \underset{\eqnmarkbox[magenta]{dim_n2}{n} \times \eqnmarkbox[cyan]{_dim_1}{1}}{\vb*{v}} = \underset{\eqnmarkbox[cyan]{_dim_m}{m} \times \eqnmarkbox[cyan]{_dim_1}{1}}{A \vb*{v}}
  \end{equation*}
  \annotatetwo{below}{dim_n1}{dim_n2}{\bfseries\scriptsize 同じ}

  $m$次元横ベクトル$\phi \in {}^t\mathbb{R}^m$に対して$A$を右からかけたものは、$n$次元横ベクトルとなる。
  \begin{equation*}
    \underset{\eqnmarkbox[cyan]{_dim_1}{1} \times \eqnmarkbox[magenta]{dim_m1}{m}}{\phi} \cdot \underset{\eqnmarkbox[magenta]{dim_m2}{m} \times \eqnmarkbox[cyan]{_dim_n}{n}}{A} = \underset{\eqnmarkbox[cyan]{_dim_1}{1} \times \eqnmarkbox[cyan]{_dim_n}{n}}{\phi A}
  \end{equation*}
  \annotatetwo{below}{dim_m1}{dim_m2}{\bfseries\scriptsize 同じ}
\end{handout}

\br

$\phi \in (\mathbb{R}^m)^*$は$\mathbb{R}^m$上の線形汎関数であるから、次のような関係になる。
\begin{equation*}
  \begin{tikzcd}[every label/.append style = {font = \normalsize}]
    \mathbb{R}^n \arrow[r,"f_A"] & \mathbb{R}^m \arrow[d,"\phi"]\\
    & \mathbb{R}
  \end{tikzcd}
\end{equation*}

このとき、合成写像$\phi \circ f_A$を考えることができ、その表現行列は$\phi A \in (\mathbb{R}^n)^*$となる。
\begin{equation*}
  \begin{tikzcd}[every label/.append style = {font = \normalsize}]
    \mathbb{R}^n \arrow[r,"f_A"]\arrow[rd, "\phi \circ f_A"'] & \mathbb{R}^m \arrow[d,"\phi"]\\
    & \mathbb{R}
  \end{tikzcd}
\end{equation*}

$f_{A}^*$の定義より、$\phi A$は$f_A^*(\phi)$と書くことができるから、
\begin{equation*}
  \begin{tikzcd}[every label/.append style = {font = \normalsize}]
    \mathbb{R}^n \arrow[r,"f_A"]\arrow[rd, "f_A^*(\phi)"'] & \mathbb{R}^m \arrow[d,"\phi"]\\
    & \mathbb{R}
  \end{tikzcd}
\end{equation*}

ここで、$f_{A}^*$は、$\mathbb{R}^m$上の線形汎関数$\phi$を入力として、$\mathbb{R}^n$上の線形汎関数$f_A^*(\phi)$を返す線形写像である。
\begin{equation*}
  \begin{tikzcd}[every label/.append style = {font = \normalsize}]
    \mathbb{R}^n \arrow[r,"f_A"]\arrow[rd, "f_A^*(\phi)"'{name=v}] & \mathbb{R}^m \arrow[d,"\phi"{name=w}] \arrow[from=w,to=v,out=-60, in=-70,looseness=2,"f_A^*"]\\
    & \mathbb{R}
  \end{tikzcd}
\end{equation*}

\begin{handout}[補足:$f_A^*$の線形性]
  $\phi_1, \phi_2 \in (\mathbb{R}^m)^*$と$c_1, c_2 \in \mathbb{R}$に対して、
  \begin{align*}
    f_A^*(c_1 \phi_1 + c_2 \phi_2) &= (c_1 \phi_1 + c_2 \phi_2) A \\
    &= c_1 (\phi_1 A) + c_2 (\phi_2 A) \\
    &= c_1 f_A^*(\phi_1) + c_2 f_A^*(\phi_2)
  \end{align*}
  となるので、$f_A^*$は線形写像である。
\end{handout}

このように、$(\mathbb{R}^m)^*$から$(\mathbb{R}^n)^*$への線形写像$f_A^*$を、
\begin{equation*}
  f_A^*(\phi) = \phi \circ f_A
\end{equation*}
として定めることができる。
$f_A^*$を$f_A$の\keyword{双対写像}という。

\br

$f_A^*(\phi)\colon \mathbb{R}^n \to \mathbb{R}$に$\vb*{v} \in \mathbb{R}^n$を入力すると、次の関係が導かれる。
\begin{equation*}
  f_A^*(\phi)(\vb*{v}) = (\phi \circ f_A)(\vb*{v}) = \phi(f_A(\vb*{v}))
\end{equation*}
つまり、$\vb*{v}$に$f_A^*(\phi)$を作用させることと、$\phi$に$f_A(\vb*{v})$を作用させることは同じである。

\br

この関係は、ペアリングの記号を用いて書くと対称性がわかりやすい。

\begin{theorem*}{数ベクトル空間における双対写像とペアリング}
  $\phi \in (\mathbb{R}^m)^*, \vb*{v} \in \mathbb{R}^n$に対して、次の関係が成り立つ。
  \begin{equation*}
    \langle f_A^*(\phi), \vb*{v} \rangle = \langle \phi, f_A(\vb*{v}) \rangle
  \end{equation*}
\end{theorem*}

\subsection{一般の線形空間の場合}

一般の線型空間$V, W$に対しても、同様に双対写像を定義することができる。

\br

線形空間$V,W$の間の線形写像$f \colon V \to W$が与えられたとする。

$W$上の線形汎関数を$\varphi \in W^*$とすると、次のような関係になっている。
\begin{equation*}
  \begin{tikzcd}[every label/.append style = {font = \normalsize}]
    V \arrow[r,"f"] & W \arrow[d,"\varphi"]\\
    & \mathbb{R}
  \end{tikzcd}
\end{equation*}

このとき、合成写像$\varphi \circ f$を考えることができる。
\begin{equation*}
  \begin{tikzcd}[every label/.append style = {font = \normalsize}]
    V \arrow[r,"f"]\arrow[rd, "\varphi \circ f"'] & W \arrow[d,"\varphi"]\\
    & \mathbb{R}
  \end{tikzcd}
\end{equation*}

\thmref{thm:linear-map-composition}より、線形写像の合成もまた線形写像になるので、$\varphi \circ f$は$V$上の線形汎関数である。

これを$f^*(\varphi) \in V^*$と書くことにする。
\begin{equation*}
  \begin{tikzcd}[every label/.append style = {font = \normalsize}]
    V \arrow[r,"f"]\arrow[rd, "f^*(\varphi)"'] & W \arrow[d,"\varphi"]\\
    & \mathbb{R}
  \end{tikzcd}
\end{equation*}

ここで、$f^*$は、$W^*$上の線形汎関数$\varphi$を入力として、$V^*$上の線形汎関数$f^*(\varphi)$を返す線形写像である。
\begin{equation*}
  \begin{tikzcd}[every label/.append style = {font = \normalsize}]
    V \arrow[r,"f"]\arrow[rd, "f^*(\varphi)"'{name=v}] & W \arrow[d,"\varphi"{name=w}] \arrow[from=w,to=v,out=-60, in=-70,looseness=2,"f^*"]\\
    & \mathbb{R}
  \end{tikzcd}
\end{equation*}

\begin{handout}[補足:$f^*$の線形性]
  $\varphi_1, \varphi_2 \in W^*$と$c_1, c_2 \in \mathbb{R}$とする。
  
  $\varphi_1, \varphi_2$は線形写像であるので、\defref{def:linear-map-addition-scalar}より、
  \begin{align*}
    f^*(c_1 \varphi_1 + c_2 \varphi_2)(\vb*{v}) &= (c_1 \varphi_1 + c_2 \varphi_2)f(\vb*{v}) \\
    &= c_1 \varphi_1(f(\vb*{v})) + c_2 \varphi_2(f(\vb*{v})) \\
    &= c_1 f^*(\varphi_1)(\vb*{v}) + c_2 f^*(\varphi_2)(\vb*{v})
  \end{align*}
  となるので、$f^*$は線形写像である。
  
  \br
  
  ここで、$f^*(\varphi) = \varphi \circ f$と定義したことから、
  \begin{equation*}
    f^*(\varphi)(\vb*{v}) = (\varphi \circ f)(\vb*{v}) = \varphi(f(\vb*{v}))
  \end{equation*}
  と書けることを用いている。
\end{handout}

このように、$W^*$から$V^*$への線形写像$f^*$を、
\begin{equation*}
  f^*(\varphi) = \varphi \circ f
\end{equation*}
として定めることができる。
$f^*$を$f$の\keyword{双対写像}という。

\begin{definition*}{双対写像}
  $V,W$を線形空間とし、$f \colon V \to W$を線形写像とするとき、$f$の\keyword{双対写像}$f^* \colon W^* \to V^*$を次のように定義する。
  \begin{equation*}
    f^*(\varphi) \coloneq \varphi \circ f \quad (\varphi \in W^*)
  \end{equation*}
\end{definition*}

$f^*(\varphi)\colon V \to \mathbb{R}$に$\vb*{v} \in V$を入力すると、次の関係が導かれる。
\begin{equation*}
  f^*(\varphi)(\vb*{v}) = (\varphi \circ f)(\vb*{v}) = \varphi(f(\vb*{v}))
\end{equation*}
つまり、$\vb*{v}$に$f^*(\varphi)$を作用させることと、$\varphi$に$f(\vb*{v})$を作用させることは同じである。

\begin{theorem*}{双対写像とペアリング}
  $\varphi \in W^*, \vb*{v} \in V$に対して、次の関係が成り立つ。
  \begin{equation*}
    \langle f^*(\varphi), \vb*{v} \rangle = \langle \varphi, f(\vb*{v}) \rangle
  \end{equation*}
\end{theorem*}

\end{document}
