\documentclass[../../../topic_linear-algebra]{subfiles}

\begin{document}

\sectionline
\section{線形汎関数の空間}

内積の双線形性は、任意のベクトル$\vb*{v}$に対して、
\begin{equation*}
  (c_1 \vb*{a}_1 + c_2 \vb*{a}_2, \vb*{v}) = c_1 (\vb*{a}_1, \vb*{v}) + c_2 (\vb*{a}_2, \vb*{v})
\end{equation*}
が成り立つというものだった。

\br

これは、$\mathbb{R}^n$上の線形汎関数が満たす関係式と読み替えることができる。
\begin{equation*}
  \phi_{c_1 \vb*{a}_1 + c_2 \vb*{a}_2}(\vb*{v}) = c_1 \phi_{\vb*{a}_1}(\vb*{v}) + c_2 \phi_{\vb*{a}_2}(\vb*{v})
\end{equation*}

この関係式は、$\mathbb{R}^n$上の線形汎関数の集合に、\keyword{線形空間}としての構造をもたらす。

\br

$\mathbb{R}^n$上の線形汎関数の集合を$(\mathbb{R}^n)^*$と書くことにしよう。

この集合$(\mathbb{R}^n)^*$に和とスカラー倍の演算を導入することで、$(\mathbb{R}^n)^*$を線形空間とみなすことができる。

\sectionline
\section{線形汎関数の空間の基底}
\marginnote{
  \refweb{1次形式と双対空間}{https://www.cck.dendai.ac.jp/math/support/latb.html}
}

$\mathbb{R}^n$の基底を$\{ \vb*{u}_1, \ldots, \vb*{u}_n \}$とするとき、任意のベクトル$\vb*{v} \in \mathbb{R}^n$は、
\begin{equation*}
  \vb*{v} = v_1 \vb*{u}_1 + \cdots + v_n \vb*{u}_n = \begin{pmatrix}
    \vb*{u}_1 & \cdots & \vb*{u}_n
  \end{pmatrix} \begin{pmatrix}
    v_1 \\
    \vdots \\
    v_n
  \end{pmatrix}
\end{equation*}
という線形結合で表すことができる。

ここで、$v_1, \ldots, v_n$は、基底$\{\vb*{u}_1, \ldots, \vb*{u}_n\}$に関する$\vb*{v}$の\keyword{成分}あるいは\keyword{座標}と呼ばれる。

\br

このうち第$j$座標$v_j$を取得する関数を$\phi_j$と定めよう。
\begin{equation*}
  \phi_j(\vb*{v}) = v_j
\end{equation*}
このような関数を\keyword{座標関数}と呼ぶことにする。

\br

また、$\phi_j$は線形であるため、$\mathbb{R}^n$上の線形汎関数である。

\br

\begin{handout}[補足:$\phi_j$の線形性]
  任意の$\vb*{v},\vb*{w}\in\mathbb{R}^n$が基底$\{\vb*{u}_1, \ldots, \vb*{u}_n\}$に関して次のように表せるとする。
  \begin{equation*}
    \vb*{v} = \sum_{i=1}^n v_i \vb*{u}_i, \quad \vb*{w} = \sum_{j=1}^n w_j \vb*{u}_j
  \end{equation*}
  このとき、$\phi_j$は次のように定義される。
  \begin{equation*}
    \phi_j(\vb*{v}) = v_j, \quad \phi_j(\vb*{w}) = w_j
  \end{equation*}
  
  \br
  
  ベクトルの和を考えると、
  \begin{equation*}
    \vb*{v} + \vb*{w} = \sum_{i=1}^n (v_i + w_i) \vb*{u}_i
  \end{equation*}
  より、第$j$座標は$v_j+w_j$となるので、
  \begin{equation*}
    \phi_j(\vb*{v} + \vb*{w}) = v_j + w_j = \phi_j(\vb*{v}) + \phi_j(\vb*{w})
  \end{equation*}
  
  \br
  
  ベクトルのスカラー倍を考えると、
  \begin{equation*}
    \alpha \vb*{v} = \sum_{i=1}^n (\alpha v_i) \vb*{u}_i
  \end{equation*}
  より、第$j$座標は$\alpha v_j$となるので、
  \begin{equation*}
    \phi_j(\alpha \vb*{v}) = \alpha v_j = \alpha \phi_j(\vb*{v})
  \end{equation*}
  
  \br
  
  以上より、$\phi_j\colon\mathbb{R}^n\to\mathbb{R}$は線形写像であることが示された。
\end{handout}

\br

$\phi_j$を用いると、$\vb*{v}$を表す線形結合は次のように書ける。
\begin{equation*}
  \vb*{v} = \phi_1(\vb*{v}) \vb*{u}_1 + \cdots + \phi_n(\vb*{v}) \vb*{u}_n
\end{equation*}

\br

ここで、たとえば$\vb*{v}$を$\vb*{u}_1$に置き換えた式を考える。
\begin{equation*}
  \vb*{u}_1 = \phi_1(\vb*{u}_1) \vb*{u}_1 + \cdots + \phi_n(\vb*{u}_1) \vb*{u}_n
\end{equation*}
この等式が成り立つには、
\begin{itemize}
  \item $\phi_1(\vb*{u}_1) = 1$
  \item $\phi_2(\vb*{u}_1) = 0, \ldots, \phi_n(\vb*{u}_1) = 0$
\end{itemize}
でなければならない。

右辺の$\vb*{u}_1$だけが残り、他の項が消えることで、$\vb*{u}_1 = \vb*{u}_1$という等式が成り立つ。

\br

同様に考えると、$\vb*{v}$を$\vb*{u}_i$に置き換えた式
\begin{equation*}
  \vb*{u}_i = \phi_1(\vb*{u}_i) \vb*{u}_1 + \cdots + \phi_n(\vb*{u}_i) \vb*{u}_n
\end{equation*}
が成り立つには、$\vb*{u}_i$だけが残り、他の項が消えなければならないので、
\begin{equation*}
  \phi_j (\vb*{u}_i) = \delta_{ij} = \begin{cases}
    1 & (i = j) \\
    0 & (i \neq j)
  \end{cases}
\end{equation*}
と定める必要がある。

\br

この式により、$\mathbb{R}^n$の基底$\vb*{u}_1, \ldots, \vb*{u}_n$を選べば、それらに対応する線形汎関数$\phi_1, \ldots, \phi_n$が定まることがわかる。

そしてこのとき、$\phi_1, \ldots, \phi_n$は$(\mathbb{R}^n)^*$の基底となっている。

\begin{theorem}{$\mathbb{R}^n$における基底に対応する線形汎関数の構成}{dual-basis-construction-Rn}
  $\{ \vb*{u}_1, \ldots, \vb*{u}_n \}$を$\mathbb{R}^n$の基底とするとき、$\phi_j \in (\mathbb{R}^n)^*$を次のように定める。
  \begin{equation*}
    \phi_j (\vb*{u}_i) = \delta_{ij}
  \end{equation*}
  このような$\phi_1, \ldots, \phi_n$は$(\mathbb{R}^n)^*$の基底をなす。
\end{theorem}

\begin{proof}
  \begin{subpattern}{\bfseries $\phi_1, \ldots, \phi_n$が線型独立であること}
    次のような$\phi_1, \ldots, \phi_n$の線形関係式を考える。
    \begin{equation*}
      c_1 \phi_1 + \cdots + c_n \phi_n = 0
    \end{equation*}
    
    このとき、任意の$j$に対して、
    \begin{align*}
      (c_1 \phi_1 + \cdots + c_n \phi_n)(\vb*{u}_j) &= c_1 \phi_1(\vb*{u}_j) + \cdots + c_n \phi_n(\vb*{u}_j) \\
      &= \sum_{i=1}^n c_i \phi_i(\vb*{u}_j) = \sum_{i=1}^n c_i \delta_{ij} \\
      &= c_j = 0
    \end{align*}
    が成り立たなければならない。
    
    これは$\phi_1, \ldots, \phi_n$が線型独立であることを示している。 $\qed$
  \end{subpattern}
  
  \begin{subpattern}{\bfseries $\phi_1, \ldots, \phi_n$が$(\mathbb{R}^n)^*$を張ること}
    $\psi \in (\mathbb{R}^n)^*$を任意にとると、$\vb*{u}_j$に対する値$\alpha_j = \psi(\vb*{u}_j)$が定まる。
    
    \br
    
    このとき、$\alpha_j$を係数とする$\phi_1, \ldots, \phi_n$の線形結合を作ると、
    \begin{align*}
      (\alpha_1 \phi_1 + \cdots + \alpha_n \phi_n)(\vb*{u}_j) &= \alpha_1 \phi_1(\vb*{u}_j) + \cdots + \alpha_n \phi_n(\vb*{u}_j) \\
      &= \sum_{i=1}^n \alpha_i \phi_i(\vb*{u}_j) = \sum_{i=1}^n \alpha_i \delta_{ij} = \alpha_j \\
      &= \psi(\vb*{u}_j)
    \end{align*}
    
    $\phi_j,\psi$はともに$\mathbb{R}^n$から$\mathbb{R}$への線形写像であり、$\phi_j$の線形結合もまた$(\mathbb{R}^n)^*$の元なので$\mathbb{R}^n$から$\mathbb{R}$への線形写像である。
    
    よって、\hyperref[thm:linear-map-equality-on-basis]{$\mathbb{R}^n$の基底$\{\vb*{u}_1, \ldots, \vb*{u}_n\}$に対して同じ値をとる}ことから、
    \begin{equation*}
      \psi = \alpha_1 \phi_1 + \cdots + \alpha_n \phi_n
    \end{equation*}
    がいえる。
    
    \br
    
    したがって、任意の$\psi \in (\mathbb{R}^n)^*$は$\phi_1, \ldots, \phi_n$の線形結合として表すことができるため、
    \begin{equation*}
      (\mathbb{R}^n)^* = \langle \phi_1, \ldots, \phi_n \rangle
    \end{equation*}
    が示された。$\qed$
  \end{subpattern}
\end{proof}

\subsection{線形汎関数の空間の次元}

$\mathbb{R}^n$の基底$\{ \vb*{u}_1, \ldots, \vb*{u}_n \}$と、それに対応する$(\mathbb{R}^n)^*$の基底$\{ \phi_1, \ldots, \phi_n \}$は、どちらも$n$個のベクトルの組になっている。

\begin{supplnote}
  ここでいう「ベクトル」とは、「線形空間の元」という意味である。
  $(\mathbb{R}^n)^*$も線形空間であるので、その元である線形汎関数も「ベクトル」と呼んでいる。
\end{supplnote}

基底をなすベクトルの個数は、その空間の\keyword{次元}として定義されるので、次のことがいえる。

\begin{theorem*}{$\mathbb{R}^n$とその線形汎関数の空間の次元の一致}
  $\mathbb{R}^n$上の線形汎関数の空間$(\mathbb{R}^n)^*$の次元は、$\mathbb{R}^n$の次元と等しい。
  \begin{equation*}
    \dim \mathbb{R}^n = \dim (\mathbb{R}^n)^* = n
  \end{equation*}
\end{theorem*}

また、\hyperref[thm:abstract-linear-pigeonhole]{次元が等しいことから、$\mathbb{R}^n$と$(\mathbb{R}^n)^*$は線形同型}である。

すなわち、$\mathbb{R}^n$の元(縦ベクトル)と$(\mathbb{R}^n)^*$の元($\mathbb{R}^n$上の線形汎関数)の間には、\keyword{全単射}が存在する。

\br

基底を決めれば、縦ベクトルと線形汎関数を同一視する(同じものの「異なる表現」と捉える)ことができる。

\end{document}
