\documentclass[../../../topic_linear-algebra]{subfiles}

\begin{document}

\sectionline
\section{線形汎関数}
\marginnote{\refbookA p120 \\ \refbookR p232〜233}

横ベクトル($1 \times n$型行列)を縦ベクトル($n \times 1$型行列)にかけると、$1 \times 1$のスカラー値が得られる。

\begin{equation*}
  \begin{pmatrix}
    a_1 & \cdots & a_n
  \end{pmatrix} \begin{pmatrix}
    v_1    \\
    \vdots \\
    v_n
  \end{pmatrix}
  = a_1 v_1 + \cdots + a_n v_n
\end{equation*}

\br

これは、縦ベクトルを入力とする関数($\mathbb{R}^n$から$\mathbb{R}$への線形写像)と見なすことができる。

縦ベクトルを$\vb*{v}$、この関数を$\phi$とすると、次のように書ける。
\begin{equation*}
  \phi(\vb*{v}) = a_1 v_1 + \cdots + a_n v_n
\end{equation*}

この関数$\phi$は、\keyword{線形汎関数}と呼ばれる写像の一例である。

\begin{definition}{線形汎関数}\label{def:linear-functional}
  $V$を$\mathbb{R}$上の線形空間とするとき、$V$から$\mathbb{R}$への線形写像を$V$上の\keyword{線形汎関数}あるいは\keyword{線形形式}という。
\end{definition}

\sectionline
\section{横ベクトルの集合と座標関数}
\marginnote{\refbookA p120}

$n \times 1$型行列($n$次の縦ベクトル)全体の集合は$\mathbb{R}^n$と表された。

$1 \times n$型行列($n$次の横ベクトル)全体の集合を${}^t\mathbb{R}^n$と表すことにする。

\br

${}^t\mathbb{R}^n$の元は$1 \times n$型行列なので、$\mathbb{R}^n$から$\mathbb{R}$への線形写像(すなわち$\mathbb{R}^n$上の\keyword{線形汎関数})を表現している行列だと考えることができる。

\subsection{座標関数の表現行列}

基本ベクトルを転置したもの${}^t\vb*{e}_j$を縦ベクトルにかけると、$j$番目の成分が得られる。

たとえば、$n=3,\,j=2$の場合、
\begin{equation*}
  {}^t\vb*{e}_2\begin{pmatrix}
    v_1 \\
    v_2 \\
    v_3
  \end{pmatrix} = \begin{pmatrix}
    0 & 1 & 0
  \end{pmatrix}
  \begin{pmatrix}
    v_1 \\
    v_2 \\
    v_3
  \end{pmatrix}
  = v_2
\end{equation*}
といった具合に、$j$番目の成分$v_j$が得られる。

\br

このように、ベクトル$\vb*{v} \in \mathbb{R}^n$に対して、$j$番目の成分を返す関数を\keyword{座標関数}といい、$x_j$と表記することにする。
\begin{equation*}
  \mathbb{R}^n \ni \vb*{v} = \begin{pmatrix}
    v_1 \\
    \vdots \\
    v_n
  \end{pmatrix} \mapsto v_j \in \mathbb{R}
\end{equation*}

横基本ベクトル${}^t\vb*{e}_j \in {}^t\mathbb{R}^n$は、座標関数$x_j\colon \mathbb{R}^n \to \mathbb{R}$の表現行列になっている。

\subsection{基底としての座標関数}

任意の横ベクトルは、横基本ベクトルの線形結合として一意的に表現できる。

\begin{equation*}
  \begin{pmatrix}
    a_1 & \cdots & a_n
  \end{pmatrix}
  = a_1 {}^t\vb*{e}_1 + \cdots + a_n {}^t\vb*{e}_n
\end{equation*}

これを用いると、
\begin{align*}
  \phi & = \begin{pmatrix}
             a_1 & \cdots & a_n
           \end{pmatrix} \begin{pmatrix}
                           v_1    \\
                           \vdots \\
                           v_n
                         \end{pmatrix}                                                \\
       & = a_1 {}^t\vb*{e}_1 \begin{pmatrix}
                               v_1    \\
                               \vdots \\
                               v_n
                             \end{pmatrix} + \cdots + a_n {}^t\vb*{e}_n \begin{pmatrix}
                                                                          v_1    \\
                                                                          \vdots \\
                                                                          v_n
                                                                        \end{pmatrix} \\
       & = a_1 x_1 + \cdots + a_n x_n
\end{align*}
となることから、任意の線形汎関数$\phi \in {}^t\mathbb{R}^n$は、座標関数$x_1,\dots,x_n$の線型結合として
\begin{equation*}
  \phi = a_1 x_1 + \cdots + a_n x_n
\end{equation*}
のように一意的に書くことができる。

つまり、$\{x_1,\dots,x_n\}$は${}^t\mathbb{R}^n$の\keyword{基底}である。

\br

また、縦ベクトルが基底との線形結合の係数を並べたものであるのと同様に、$\phi$は横ベクトル$(a_1,\dots,a_n)$と同一視できる。

\sectionline
\section{双対空間}
\marginnote{\refbookG p58 \\ \refbookR p233}

$V$から$\mathbb{R}$への線形写像、すなわち$V$上の\keyword{線形汎関数}全体の集合を考える。

\begin{definition}{双対空間}
  $V$上の線形汎関数全体の集合を$V$の\keyword{双対空間}といい、$V^*$と表す。
  \begin{equation*}
    V^* \coloneq \Hom(V, \mathbb{R}) = \{ \phi \colon V \to \mathbb{R} \mid \phi \text{は線形写像} \}
  \end{equation*}
\end{definition}

\hyperref[thm:hom-space]{線形写像全体の集合は線形空間}であるため、双対空間$V^*$も$\mathbb{R}$上の線形空間である。

\subsection{例:縦ベクトル空間の双対空間}

$V = \mathbb{R}^n$のとき、ベクトルの$j$番目の成分を返す座標関数$x_j\colon \mathbb{R}^n \to \mathbb{R}$は線形汎関数であるため、$V^*$の元である。
\begin{equation*}
  x_j \in V^* \quad (j = 1, \dots, n)
\end{equation*}

また、$V^*$は線形空間であるため、$V^*$の元$x_j$の線形結合も$V^*$の元である。
\begin{equation*}
  \phi = a_1 x_1 + \cdots + a_n x_n \in V^*
\end{equation*}

このとき、$\{ x_1, \dots, x_n \}$は$V^*$の基底をなす(※後述)ので、$V^*$は$x_1, \dots, x_n$の1次式として表せる関数全体の集合である。

\br

また、このような$\phi$は横ベクトル$(a_1, \dots, a_n)$と同一視できるので、$\phi$全体の集合は横ベクトルの集合${}^t\mathbb{R}^n$ともいえる。

すなわち、
\begin{emphabox}
  \begin{spacebox}
    \begin{center}
      縦ベクトルの集合$\mathbb{R}^n$の双対空間は、\\
      横ベクトルの集合${}^t\mathbb{R}^n$
    \end{center}
  \end{spacebox}
\end{emphabox}
である。

\br

\begin{handout}[補足:なぜ基底といえるのか]
  $\{ x_1, \dots, x_n \}$が$V^*$の基底をなすことは、次のように確かめられる。
  
  ここでは、各座標関数$x_j$を次のように表記する。
    \begin{equation*}
      x_j(\vb*{v}) = v_j \quad (j = 1, \dots, n)
    \end{equation*}
  
  \begin{subpattern}{\bfseries 線型独立であること}
    係数$a_1, \dots, a_n$が
    \begin{equation*}
      a_1 x_1 + \cdots + a_n x_n = 0
    \end{equation*}
    を満たすと仮定すると、任意の$\vb*{v} \in \mathbb{R}^n$に対して、
    \begin{equation*}
      (a_1 x_1 + \cdots + a_n x_n)(\vb*{v}) = 0
    \end{equation*}
    ここで、$x_j$の線形性を用いると、
    \begin{align*}
      a_1 x_1(\vb*{v}) + \cdots + a_n x_n(\vb*{v}) & = 0 \\
      a_1 v_1 + \cdots + a_n v_n & = 0
    \end{align*}
    これが任意の$\vb*{v}$について成り立つということは、$a_1 = \cdots = a_n = 0$でなければならない(さもなくば、$\vb*{v}$によって和が0にならないものを作れる)ので、${x_j}$は線形独立である。 $\qed$
  \end{subpattern}
  
  \begin{subpattern}{\bfseries 空間を張ること}
    $\phi$は線形なので、任意の縦ベクトル$\vb*{v}$に対して、
    \begin{equation*}
      \phi(\vb*{v}) = \phi \left( \sum_{j=1}^n v_j \vb*{e}_j \right) 
      = \sum_{j=1}^n v_j \phi(\vb*{e}_j)
    \end{equation*}
    ここで、各$\phi(\vb*{e}_j)$は実数なので、それを$a_j := \phi(\vb*{e}_j)$とおくと、
    \begin{equation*}
      \phi(\vb*{v}) = \sum_{j=1}^n v_j a_j = \sum_{j=1}^n a_j x_j(\vb*{v})
    \end{equation*}
    したがって、
    \begin{equation*}
      \phi = \sum_{j=1}^n a_j x_j
    \end{equation*}
    として、${x_1, \dots, x_n}$の線形結合で任意の$\phi \in V^*$を表すことができる。 $\qed$
  \end{subpattern}
\end{handout}

\end{document}
