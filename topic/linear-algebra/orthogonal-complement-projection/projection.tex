\documentclass[../../../topic_linear-algebra]{subfiles}

\begin{document}

\sectionline
\section{直交射影と反射影}\label{sec:orthogonal-projection-and-reflection}
\marginnote{\refbookI p4〜5}

$\mathbb{R}^n$上の点$P$に対して、部分空間$\mathcal{U}$上の点$Q \in \mathcal{U}$のうち、$\overrightarrow{PQ}$が$\mathcal{U}$に直交するような点$Q$を、点$P$の$\mathcal{U}$への\keyword{直交射影}あるいは\keyword{正射影}という

\br

また、$\overrightarrow{QP}$を点$Q$の$\mathcal{U}$からの\keyword{反射影}という

\br

射影前のベクトルを$\vb*{p}$、射影後のベクトルを$\vb*{q}$とすると、直交射影とは、$\vb*{q}$と$\vb*{q} - \vb*{p}$が直交するように射影することである

\begin{center}
  \tdplotsetmaincoords{70}{110} % 視点の角度設定
  \begin{tikzpicture}[tdplot_main_coords, scale=3]
    % 平面サイズと傾きの設定
    \def\planesize{1.1}
    \tdplotsetrotatedcoords{-20}{30}{0} % (theta, phi, psi):傾きの設定

    % 原点と点の定義
    \coordinate (O) at (0, 0, 0);
    \coordinate (P) at (0.75, 0.75, 1.75); % 空間上の点 P
    \coordinate (Q) at (0.75, 0.75, 0.75); % 任意の平面上の点(Pの直交射影として仮置き)

    % 座標軸
    %\draw[axis] (O) -- (2.6, 0, 0) node[anchor=north east]{$x$};
    %\draw[axis] (O) -- (0, 1.4, 0) node[anchor=north west]{$y$};
    %\draw[axis] (O) -- (0, 0, 1.8) node[anchor=south]{$z$};

    % 直角
    \draw pic[
        fill=gray!60,
        draw=gray,
        angle radius=3mm,
        angle eccentricity=1.2,
      ] {right angle = P--Q--O};

    % 傾いた平面の描画(tdplot_rotated_coords環境内)
    \begin{scope}[tdplot_rotated_coords]
      \filldraw[SkyBlue!40, draw=none, opacity=0.6,postaction={pattern=crosshatch dots}, pattern color=SkyBlue!60]
      (-\planesize,-\planesize,0) --
      ( \planesize,-\planesize,0) --
      ( \planesize, \planesize,0) --
      (-\planesize, \planesize,0) -- cycle;
      \node[SkyBlue] at (\planesize*0.85,-\planesize*0.85,0) {$\mathcal{U}$};
    \end{scope}

    \draw[vector, Rhodamine] (O) -- (P) node[midway, left=1pt]{$\vb*{p}$};
    \draw[vector, Cerulean] (O) -- (Q) node[pos=0.35, right=3pt]{$\vb*{q}$};
    \draw[vector, Orchid] (P) -- (Q) node[pos=0.4, right=1pt]{$\overrightarrow{PQ} = \vb*{q} - \vb*{p}$};

    % 点
    \filldraw[Rhodamine] (P) circle (0.5pt) node[anchor=south]{$P$};
    \filldraw[Cerulean] (Q) circle (0.5pt) node[anchor=west]{$Q$};
    \filldraw[gray] (O) circle (0.5pt) node[anchor=north east]{$O$};
  \end{tikzpicture}
\end{center}

このとき、次のような関係が成り立っている
\begin{gather*}
  \overrightarrow{OP} = \overrightarrow{OQ} + \overrightarrow{QP} \\
  \overrightarrow{OQ} \in \mathcal{U}, \quad \overrightarrow{QP} \in \mathcal{U}^\perp
\end{gather*}

ここで、$\mathcal{U}^\perp$は部分空間$\mathcal{U}$に直交するベクトルの全体であり、$\mathcal{U}$の\keyword{直交補空間}と呼ばれる

$\mathbb{R}^n$の部分空間$\mathcal{U}$の直交補空間$\mathcal{U}^\perp$も、$\mathbb{R}^n$の部分空間となる

\br

$\overrightarrow{OP}$は$\mathbb{R}^n$の任意のベクトルを表すことから、
$\mathbb{R}^n$のベクトルは、$\mathcal{U}$への射影$\overrightarrow{OQ}$と、$\mathcal{U}$からの反射影$\overrightarrow{QP}$の和として表されることがわかる

\br

このような表し方は一意的であり、$\overrightarrow{OP}$の$\mathcal{U}$と$\mathcal{U}^\perp$への\keyword{直和分解}という

\sectionline

点$Q$を$\mathcal{U}$上の別の点$Q'$に移動した場合を考える

\begin{center}
  \tdplotsetmaincoords{70}{110} % 視点の角度設定
  \begin{tikzpicture}[tdplot_main_coords, scale=3]

    % 平面サイズと傾きの設定
    \def\planesize{1.1}
    \tdplotsetrotatedcoords{-20}{30}{0} % (theta, phi, psi):傾きの設定

    % 原点と点の定義
    \coordinate (O) at (0, 0, 0);
    \coordinate (P) at (0.75, 0.75, 1.75); % 空間上の点 P
    \coordinate (Q) at (0.75, 0.75, 0.75); % 射影点 Q(平面上)
    \coordinate (Qp) at ($(Q)+(-0.1, 0.55, 0)$);   % Q':平面上の別の点(仮に設定)

    % 座標軸
    %\draw[axis] (O) -- (2.6, 0, 0) node[anchor=north east]{$x$};
    %\draw[axis] (O) -- (0, 1.4, 0) node[anchor=north west]{$y$};
    %\draw[axis] (O) -- (0, 0, 1.8) node[anchor=south]{$z$};

    % 直角マーク(P-Q-O)
    \draw pic[
        fill=gray!60,
        draw=gray,
        angle radius=3mm,
        angle eccentricity=1.2,
      ] {right angle = P--Q--O};
    % 直角マーク(Q-Q'-P)
    \draw pic[
        fill=gray!60,
        draw=gray,
        angle radius=3mm,
        angle eccentricity=1.2,
      ] {right angle = Qp--Q--P};

    % 傾いた平面(tdplot_rotated_coords環境内)
    \begin{scope}[tdplot_rotated_coords]
      \filldraw[SkyBlue!40, draw=none, opacity=0.6, postaction={pattern=crosshatch dots}, pattern color=SkyBlue!60]
      (-\planesize,-\planesize,0) --
      ( \planesize,-\planesize,0) --
      ( \planesize, \planesize,0) --
      (-\planesize, \planesize,0) -- cycle;
      \node[SkyBlue] at (\planesize*0.85,-\planesize*0.85,0) {$\mathcal{U}$};
    \end{scope}

    % ベクトル描画
    \draw[vector, Rhodamine] (O) -- (P);
    \draw[vector, Cerulean] (O) -- (Q);
    \draw[vector, Orchid] (P) -- (Q);

    % Q'関連のベクトルと点
    \draw[vector, Orchid, dashed] (P) -- (Qp) node[pos=0.5, right=2pt]{$\overrightarrow{PQ'}$};
    \draw[vector, Cerulean, dashed] (Q) -- (Qp) node[pos=0.5, above=1pt]{$\overrightarrow{QQ'}$};

    % 点描画
    \filldraw[Rhodamine] (P) circle (0.5pt) node[anchor=south]{$P$};
    \filldraw[Cerulean] (Q) circle (0.5pt) node[anchor=south east]{$Q$};
    \filldraw[Cerulean] (Qp) circle (0.5pt) node[anchor=south west]{$Q'$};
    \filldraw[gray] (O) circle (0.5pt) node[anchor=north east]{$O$};
  \end{tikzpicture}
\end{center}

このとき、三平方の定理より、
\begin{equation*}
  \| \overrightarrow{PQ'} \|^2 = \| \overrightarrow{PQ} \|^2 + \| \overrightarrow{QQ'} \|^2 > \| \overrightarrow{PQ} \|^2
\end{equation*}
となるから、
\begin{shaded}
  射影した点$Q$は、点$P$から最短となる$\mathcal{U}$上の点
\end{shaded}
であることがわかる

\end{document}
