\documentclass[../../../topic_linear-algebra]{subfiles}

\begin{document}

\sectionline
\section{直交補空間}
\marginnote{\refbookC p136〜140}

内積を導入すると、ベクトルの長さや直交性が利用できるようになる

直交性は、ベクトルだけでなく、部分空間に対しても拡張できる

\br

計量空間の部分空間に直交するベクトルの集合を、\keyword{直交補空間}と呼ぶ

\begin{definition}{直交補空間}
  計量空間$V$の部分空間$W$に対し、$W$の\keyword{直交補空間}$W^\perp$を次のように定義する
  \begin{equation*}
    W^\perp \coloneq \{ \vb*{v} \in V \mid \forall \vb*{w} \in W, (\vb*{v}, \vb*{w}) = 0 \}
  \end{equation*}
\end{definition}

\br

直交補空間もまた、計量空間の部分空間になっている

\begin{theorem}{直交補空間の部分空間性}
  計量空間$V$の部分空間$W$の直交補空間$W^\perp$は、計量空間$V$の部分空間である
\end{theorem}

\begin{proof}
  \begin{subpattern}{\bfseries 和について}
    $\vb*{a}_1, \vb*{a}_2 \in W^\perp$とすると、任意の$\vb*{b} \in W$に対して、
    \begin{equation*}
      (\vb*{a}_1 + \vb*{a}_2, \vb*{b}) = (\vb*{a}_1, \vb*{b}) + (\vb*{a}_2, \vb*{b}) = 0 + 0 = 0
    \end{equation*}
    となるので、$\vb*{a}_1 + \vb*{a}_2 \in W^\perp$である $\qed$
  \end{subpattern}

  \begin{subpattern}{\bfseries スカラー倍について}
    $\vb*{a} \in W^\perp$とすると、任意のスカラー$c \in K$と任意の$\vb*{b} \in W$に対して、
    \begin{equation*}
      (c\vb*{a}, \vb*{b}) = c(\vb*{a}, \vb*{b}) = c \cdot 0 = 0
    \end{equation*}
    となるので、$c\vb*{a} \in W^\perp$である $\qed$
  \end{subpattern}
\end{proof}

\end{document}
