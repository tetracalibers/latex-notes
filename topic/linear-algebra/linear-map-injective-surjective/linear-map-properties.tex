\documentclass[../../../topic_linear-algebra]{subfiles}

\begin{document}

\sectionline
\section{線形写像とベクトルの線型独立性}
\marginnote{\refbookA p65〜66}

\begin{theorem}{線形写像とベクトルの線形独立性}
  $f\colon \mathbb{R}^n \to \mathbb{R}^m$を線形写像、$\vb*{v}_1, \vb*{v}_2, \dots, \vb*{v}_n \in \mathbb{R}^n$とする
  \begin{enumerate}[label=\romanlabel]
    \item $\{ f(\vb*{v}_1), \dots, f(\vb*{v}_n) \}$が線型独立ならば、$\{ \vb*{v}_1, \dots, \vb*{v}_n \}$は線型独立
    \item $\{\vb*{v}_1, \dots, \vb*{v}_n\}$が線形従属ならば、$\{ f(\vb*{v}_1),  \dots, f(\vb*{v}_n) \}$は線形従属
  \end{enumerate}
\end{theorem}

\begin{proof}
  \todo{\refbookA p65 問2.11}
\end{proof}

\romannum{ii}は、平行なベクトルを線型写像で写した結果、平行でなくなったりはしないということを述べている

\sectionline

\begin{theorem}{線型写像とベクトルの線型独立性}
  線型写像$f\colon \mathbb{R}^n \to \mathbb{R}^m$に対して、次の2つは同値になる
  \begin{enumerate}[label=\romanlabel]
    \item $\vb*{v} \neq \vb*{0}$ならば、$f(\vb*{v}) \neq \vb*{0}$
    \item $\{ \vb*{v}_1, \vb*{v}_2, \dots, \vb*{v}_n \}$が線型独立ならば、$\{ f(\vb*{v}_1), f(\vb*{v}_2), \dots, f(\vb*{v}_n) \}$も線型独立
  \end{enumerate}
\end{theorem}

\begin{proof}
  \todo{\refbookA p66 命題2.3.2}
\end{proof}

\romannum{i}は、零写像と射影を除けば、$f$によってベクトルが「つぶれない」という性質を表している

\todo{\refbookA p55 例2.1.15}

\br

\romannum{ii}は、たとえば平行四辺形の像が線分や1点になったりしないことなどを意味している

\sectionline

\begin{theorem}{線形写像の単射性}
  線形写像$f$が単射であることと次は同値である
  \begin{equation*}
    f(\vb*{v}) = \vb*{0} \Longrightarrow \vb*{v} = \vb*{0}
  \end{equation*}
\end{theorem}

\begin{proof}
  \todo{\refbookA p66 命題2.3.3}
\end{proof}

\sectionline
\section{線形写像の単射性と全射性}
\marginnote{\refbookA p67〜}

線形写像$f$の単射性を表現行列$A$の言葉で述べる

\todo{\refbookA p67〜69}

% \begin{theorem}{線形写像の単射性と表現行列}

% \end{theorem}

\end{document}
