\documentclass[../../../topic_linear-algebra]{subfiles}

\begin{document}

\sectionline
\section{線形写像とベクトルの線型独立性}
\marginnote{\refbookA p65〜66}

\begin{theorem}{線形写像とベクトルの線形独立性}
  $f\colon \mathbb{R}^n \to \mathbb{R}^m$を線形写像、$\vb*{v}_1, \vb*{v}_2, \dots, \vb*{v}_n \in \mathbb{R}^n$とする
  \begin{enumerate}[label=\romanlabel]
    \item $\{ f(\vb*{v}_1), \dots, f(\vb*{v}_n) \}$が線型独立ならば、$\{ \vb*{v}_1, \dots, \vb*{v}_n \}$は線型独立
    \item $\{\vb*{v}_1, \dots, \vb*{v}_n\}$が線形従属ならば、$\{ f(\vb*{v}_1),  \dots, f(\vb*{v}_n) \}$は線形従属
  \end{enumerate}
\end{theorem}

\begin{proof}
  \todo{\refbookA p65 問2.11}
\end{proof}

\romannum{ii}は、平行なベクトルを線型写像で写した結果、平行でなくなったりはしないということを述べている

\sectionline

\begin{theorem}{線型写像とベクトルの線型独立性}
  線型写像$f\colon \mathbb{R}^n \to \mathbb{R}^m$に対して、次の2つは同値になる
  \begin{enumerate}[label=\romanlabel]
    \item $\vb*{v} \neq \vb*{0}$ならば、$f(\vb*{v}) \neq \vb*{0}$
    \item $\{ \vb*{v}_1, \vb*{v}_2, \dots, \vb*{v}_n \}$が線型独立ならば、$\{ f(\vb*{v}_1), f(\vb*{v}_2), \dots, f(\vb*{v}_n) \}$も線型独立
  \end{enumerate}
\end{theorem}

\begin{proof}
  \todo{\refbookA p66 命題2.3.2}
\end{proof}

\romannum{i}は、零写像と射影を除けば、$f$によってベクトルが「つぶれない」という性質を表している

\todo{\refbookA p55 例2.1.15}

\br

\romannum{ii}は、たとえば平行四辺形の像が線分や1点になったりしないことなどを意味している

\sectionline

\begin{theorem}{線形写像の単射性}
  線形写像$f$が単射であることと次は同値である
  \begin{equation*}
    f(\vb*{v}) = \vb*{0} \Longrightarrow \vb*{v} = \vb*{0}
  \end{equation*}
\end{theorem}

\begin{proof}
  \todo{\refbookA p66 命題2.3.3}
\end{proof}

\sectionline
\section{線形写像の単射性と全射性}
\marginnote{\refbookA p67〜68}

線形写像$f$の単射性を表現行列$A$の言葉で述べる

\begin{theorem}{線形写像の単射性と表現行列}
  線形写像$f\colon \mathbb{R}^n \to \mathbb{R}^m$の表現行列を$A$とするとき、次はすべて同値
  \begin{enumerate}[label=\romanlabel]
    \item $f$は単射
    \item $A\vb*{x} = \vb*{0}$は自明な解しか持たない
    \item $\rank(A) = n$
  \end{enumerate}
\end{theorem}

\begin{proof}
  \todo{\refbookA p67 命題2.3.4}
\end{proof}

\romannum{i}は抽象的な概念、\romannum{ii}は方程式論的な言葉、\romannum{iii}は数値的な条件であり、これらは言い換えただけで同値であると述べている。

\sectionline

単射性と対比して、全射性の理解も表現行列の言葉で整理する

\begin{theorem}{線形写像の全射性と表現行列}
  線形写像$f\colon \mathbb{R}^n \to \mathbb{R}^m$の表現行列を$A$とするとき、次はすべて同値
  \begin{enumerate}[label=\romanlabel]
    \item $f$は全射
    \item 任意の$\vb*{b} \in \mathbb{R}^m$に対して、$A\vb*{x} = \vb*{b}$には解が存在する
    \item $\rank(A) = m$
  \end{enumerate}
\end{theorem}

\begin{proof}
  \todo{\refbookA p68 命題2.3.6}
\end{proof}

\sectionline
\section{像空間と核空間}
\marginnote{\refbookA p68〜69}

線形写像$f\colon \mathbb{R}^n \to \mathbb{R}^m$の\keyword{全射性}は、$\mathbb{R}^m$の部分集合である\keyword{像空間}$\Im(f)$と関係している

$f$が全射であることは、$\Im(f) = \mathbb{R}^m$と同値である

\sectionline

一方、$f$の\keyword{単射性}と関連して、$\mathbb{R}^n$の部分集合
\begin{equation*}
  \Ker(f) = \{ \vb*{v} \in \mathbb{R}^n \mid f(\vb*{v}) = \vb*{0} \}
\end{equation*}
を考え、これを$f$の\keyword{核空間}あるいは\keyword{カーネル}と呼ぶ

\br

線形写像の単射性は、次のようにも言い換えられる

\begin{theorem}{線形写像の単射性}
  線形写像$f$が単射であることと次は同値である
  \begin{equation*}
    \Ker(f) = \{ \vb*{0} \}
  \end{equation*}
\end{theorem}

\sectionline

核空間$\Ker(f)$は、実はすでに馴染みのある概念である

\begin{theorem}{核空間と表現行列}
  線形写像$f\colon \mathbb{R}^n \to \mathbb{R}^m$の表現行列を$A$とするとき、
  \begin{equation*}
    \Ker(f) = \{ \vb*{v} \in \mathbb{R}^n \mid A\vb*{v} = \vb*{0} \}
  \end{equation*}
  と定めると、
  \begin{equation*}
    \Ker(f) = \Ker(A)
  \end{equation*}
\end{theorem}

これは、斉次形の連立線形方程式$A\vb*{x} = \vb*{0}$の\keyword{解空間}そのものである

\br

$\Ker(A)$の元は、$A\vb*{x} = \vb*{0}$の基本解を使ってパラメータ表示できる

\end{document}
