\documentclass[../../../topic_linear-algebra]{subfiles}

\begin{document}

\sectionline
\section{像空間と核空間}
\marginnote{\refbookA p68〜69}

線形写像$f\colon \mathbb{R}^n \to \mathbb{R}^m$の\keyword{全射性}は、$\mathbb{R}^m$の部分集合である\keyword{像空間}$\Im(f)$と関係している

$f$が全射であることは、$\Im(f) = \mathbb{R}^m$と同値である

\sectionline

一方、$f$の\keyword{単射性}と関連して、$\mathbb{R}^n$の部分集合
\begin{equation*}
  \Ker(f) = \{ \vb*{v} \in \mathbb{R}^n \mid f(\vb*{v}) = \vb*{0} \}
\end{equation*}
を考え、これを$f$の\keyword{核空間}あるいは\keyword{カーネル}と呼ぶ

\br

線形写像の単射性は、次のようにも言い換えられる

\begin{theorem}{線形写像の単射性}
  線形写像$f$が単射であることと次は同値である
  \begin{equation*}
    \Ker(f) = \{ \vb*{0} \}
  \end{equation*}
\end{theorem}

\sectionline

核空間$\Ker(f)$は、実はすでに馴染みのある概念である

\begin{theorem}{核空間と表現行列}
  線形写像$f\colon \mathbb{R}^n \to \mathbb{R}^m$の表現行列を$A$とするとき、
  \begin{equation*}
    \Ker(f) = \{ \vb*{v} \in \mathbb{R}^n \mid A\vb*{v} = \vb*{0} \}
  \end{equation*}
  と定めると、
  \begin{equation*}
    \Ker(f) = \Ker(A)
  \end{equation*}
\end{theorem}

これは、斉次形の連立線形方程式$A\vb*{x} = \vb*{0}$の\keyword{解空間}そのものである

\br

$\Ker(A)$の元は、$A\vb*{x} = \vb*{0}$の基本解を使ってパラメータ表示できる

\end{document}
