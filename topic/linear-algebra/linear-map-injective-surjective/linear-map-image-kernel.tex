\documentclass[../../../topic_linear-algebra]{subfiles}

\begin{document}

\sectionline
\section{線形写像の像}
\marginnote{\refbookC p79〜84}

写像の\keyword{像}を、線形写像の場合に考える

線形写像$f$の像は、$\vb*{v} \in V$に対して、$f(\vb*{v})$の値が取りうる集合である

\begin{definition}{線形写像の像}
  線形写像$f\colon V \to W$に対して、$f$による$V$の像$f(V)$を、線形写像$f$の\keyword{像}や\keyword{像空間}といい、$\Im(f)$と表記する
  \begin{equation*}
    \Im(f) = f(V) = \{ f(\vb*{v}) \in W \mid \vb*{v} \in V \} \subset W
  \end{equation*}
\end{definition}

線形写像の像は、
\begin{shaded}
  写像によって何が出力されるか
\end{shaded}
を表す

\sectionline
\section{線形写像の核}
\marginnote{\refbookC p79〜84}

線形写像$f$の\keyword{核}は、$f(\vb*{v}) = \vb*{0}$となるベクトル$\vb*{v} \in V$全体の集合として定義される

\begin{definition}{線形写像の核}
  線形写像$f\colon V \to W$に対して、$f$による$\{\vb*{0}\}$の逆像$f^{-1}(\{\vb*{0}\})$を、線形写像$f$の\keyword{核}や\keyword{核空間}、あるいは\keyword{カーネル}といい、$\Ker(f)$と表記する
  \begin{equation*}
    \Ker(f) = f^{-1}(\{\vb*{0}\}) = \{ \vb*{v} \in V \mid f(\vb*{v}) = \vb*{0} \} \subset V
  \end{equation*}
\end{definition}

線形写像の核は、
\begin{shaded}
  どのような成分が写像によって失われるか
\end{shaded}
を表す

\sectionline
\section{像空間と全射性}\label{sec:image-and-surjectivity}
\marginnote{\refbookA p68〜69}

線形写像$f\colon \mathbb{R}^n \to \mathbb{R}^m$の\keyword{全射性}は、$\mathbb{R}^m$の部分集合である\keyword{像空間}$\Im(f)$と関係している

\begin{shaded}
  全射な写像は、定義域の元の像で値域を「埋め尽くす」
\end{shaded}

ということから、$f$が全射であることは、$\Im(f) = \mathbb{R}^m$と同値だとわかる

\sectionline
\section{核空間と単射性}

線形写像$f$が単射であることは、次の条件と同値であった
\begin{equation*}
  f(\vb*{v}) = \vb*{0} \Longrightarrow \vb*{v} = \vb*{0}
\end{equation*}

この条件は、次のように言い換えることができる
\begin{equation*}
  \Ker(f) = \{ \vb*{0} \}
\end{equation*}

\begin{theorem}{線形写像の単射性と核の関係}\label{thm:injective-iff-trivial-kernel}
  $f$を線形写像とするとき、
  \begin{equation*}
    f\text{が単射} \Longleftrightarrow \Ker(f) = \{ \vb*{0} \}
  \end{equation*}
\end{theorem}

\begin{proof}
  $\Ker(f)$の定義は
  \begin{equation*}
    \Ker(f) = \{ \vb*{v} \in V \mid f(\vb*{v}) = \vb*{0} \}
  \end{equation*}

  これを踏まえて、次の2つが同値であることを示す
  \begin{enumerate}[label=\romanlabel]
    \item $f(\vb*{v}) = \vb*{0} \Longrightarrow \vb*{v} = \vb*{0}$
    \item $\Ker(f) = \{ \vb*{0} \}$
  \end{enumerate}

  \begin{subpattern}{(\romannum{i}) $\Longrightarrow$ (\romannum{ii})}
    このとき、$f(\vb*{v}) = \vb*{0}$が$\vb*{v} = \vb*{0}$を意味するので、$\Ker(f)$の元は零ベクトルのみになる

    よって、$\Ker(f) = \{ \vb*{0} \}$が成り立つ $\qed$
  \end{subpattern}

  \begin{subpattern}{(\romannum{ii}) $\Longrightarrow$ (\romannum{i})}
    $\Ker(f) = \{ \vb*{0} \}$であれば、$\Ker(f)$の元は零ベクトルのみである

    よって、$f(\vb*{v}) = \vb*{0}$が成り立つとき、$\vb*{v} = \vb*{0}$が成り立つことになる

    すなわち、$f(\vb*{v}) = \vb*{0} \Longrightarrow \vb*{v} = \vb*{0}$が成り立つ $\qed$
  \end{subpattern}
\end{proof}

\sectionline
\section{核空間と解空間}

線形写像$f\colon \mathbb{R}^n \to \mathbb{R}^m$の表現行列を$A$とするとき、
\begin{equation*}
  \Ker(f) = \{ \vb*{v} \in \mathbb{R}^n \mid A\vb*{v} = \vb*{0} \}
\end{equation*}
と定めると、$f(\vb*{v}) = A\vb*{v}$という関係から、$\Ker(f)$と$\Ker(A)$は同じものを指す

\br

これは、斉次形の連立線形方程式$A\vb*{x} = \vb*{0}$の\keyword{解空間}そのものである

$\Ker(A)$の元は、$A\vb*{x} = \vb*{0}$の基本解を使ってパラメータ表示できる

\end{document}
