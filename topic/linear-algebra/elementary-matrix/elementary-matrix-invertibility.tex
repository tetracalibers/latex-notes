\documentclass[../../../topic_linear-algebra]{subfiles}

\usepackage{xr-hyper}
\externaldocument{../../../.tex_intermediates/topic_linear-algebra}

\begin{document}

\sectionline
\section{基本行列の正則性}
\marginnote{\refbookF p62 \\ \refbookA p86}

行基本変形も列基本変形も、基本行列によって定式化できる

この考えをさらに進めるため、基本行列の性質を述べる

\begin{theorem*}{基本行列の正則性}
  基本行列は正則である
\end{theorem*}

\begin{proof}
  基本行列の表す変形を考えれば、
  \begin{align*}
    F(i,j)F(i,j)           & = E                          \\
    G(i;c)G(i;\frac{1}{c}) & = G(i;\frac{1}{c})G(i;c) = E \\
    H(i,j;c)H(i;-c)        & = H(i,;-c)H(i,j;c) = E
  \end{align*}
  が成り立つことがわかる

  したがって、基本行列は逆行列を持つので正則である $\qed$
\end{proof}

つまり、各々の基本変形は可逆の変形、すなわち逆に戻ることのできる変形である

\sectionline
\section{基本行列の積と逆行列}
\marginnote{\refbookF p62〜63 \\ \refbookA p86}

行基本変形が基本行列を左からかけることに対応することから、行基本変形とは線形写像であり、基本行列はその表現行列であるという見方もできる

\br

そのため、行基本変形の合成は、基本行列の積として表現できる

\br

このことから、行についての連続する複数の基本変形の繰り返しも可逆であることがいえる

\begin{theorem}{基本行列の積による行変形の構成}{row-operation-by-elementary-matrices}
  行列の行変形$A \to B$に対し、$B=PA$を満たす正則行列$P$が存在する

  このとき、$P$はいくつかの基本行列の積である
\end{theorem}

\begin{proof}
  行基本変形を$A \xrightarrow{\alpha_k} \cdots \xrightarrow{\alpha_1} B$と合成して得られる行変形は、$E_{\alpha_1} \cdots E_{\alpha_k}$を左からかけることで実現される

  すなわち、
  \begin{equation*}
    B = E_{\alpha_1} \cdots E_{\alpha_k} A
  \end{equation*}
  が成り立つ

  個々の基本行列$E_{\alpha_1},\ldots,E_{\alpha_k}$は正則であるので、これらの積$P = E_{\alpha_1} \cdots E_{\alpha_k}$も正則である $\qed$
\end{proof}

\br

上の証明から、正則行列$P$に対して、その逆行列を$P^{-1}$とすると、
\begin{equation*}
  P^{-1}B = P^{-1}E_{\alpha_1} \cdots E_{\alpha_k} A = P^{-1}PA = A
\end{equation*}
が成り立つことになる

\br

ここで、$B = E$の場合を考えると、$P^{-1}E = A$となるので、次のことがいえる

\begin{theorem*}{単位行列への行変形による逆行列の構成}
  正方行列$A$の単位行列への行変形$A \to E$に対応する基本変形の積は、$A$の逆行列を与える
\end{theorem*}

つまり、任意の正方行列は行基本変形だけで単位行列に変形でき、その基本行列の積から逆行列を求めることができる

\sectionline

この章で得られた定理を組み合わせると、次の定理が得られる

\begin{theorem}{基本行列の積による正則行列の表現}{invertible-as-product-of-elementary}
  任意の正則行列はいくつかの基本行列の積である
\end{theorem}

\begin{proof}
  $A$を正則行列とすると、$A$の逆行列$A^{-1}$は行変形$A \to E$に対応する基本変形の積によって与えられる

  さらに、\thmref{thm:row-operation-by-elementary-matrices}より、行変形$A \to E$に対し、
  \begin{equation*}
    E = PA
  \end{equation*}
  を満たす正則行列$P$が存在する

  この等式より、$A^{-1} = P$となり、$P$も基本行列の積であることがいえる $\qed$
\end{proof}

\end{document}
