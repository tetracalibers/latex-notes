\documentclass[../../../topic_linear-algebra]{subfiles}

\usepackage{xr-hyper}
\externaldocument{../../../.tex_intermediates/topic_linear-algebra}

\begin{document}

\sectionline
\section{階数標準形}
\marginnote{\refbookA p87〜88 \\ \refbookF p75〜78}

任意の行列$A$は、行基本変形により、次のような\secref{sec:reduced-row-echelon-form}に変形できる
\begin{equation*}
  \begin{pNiceArray}{>{\strut}cccccccc}% <-- % mandatory
    [margin, extra-margin=2pt,no-cell-nodes]
    0 & \rowcolor{carnationpink!40} 1 & * & 0 & 0 & * & * & 0 \\
    0 & 0 & 0 & \rowcolor{carnationpink!40} 1 & 0 & * & * & 0 \\
    0 & 0 & 0 & 0 & \rowcolor{carnationpink!40} 1 & * & * & 0 \\
    0 & 0 & 0 & 0 & 0 & 0 & 0 & \rowcolor{carnationpink!40} 1 \\
    0 & 0 & 0 & 0 & 0 & 0 & 0 & 0
  \end{pNiceArray}
\end{equation*}

\br

ここからさらに、列の交換によって、主成分のある列を左に集めることができる
\begin{equation*}
  \begin{pNiceArray}{ccc|cc}[xdots={horizontal-labels,line-style = <->},first-row,last-col,margin,columns-width =1em]
    \Hdotsfor{3}^{r} & \Hdotsfor{2}^{n-r} \\
    1 & & & \Block{3-2}<\large>{*} && \Vdotsfor{3}^{r}  \\
    & \ddots &&& \\
    & & 1&& \\
    \hline
    \Block{2-3}<\large>{O} && & \Block{2-2}<\large>{O} && \Vdotsfor{2}^{n-r} \\
    &&&&
  \end{pNiceArray}
\end{equation*}

\br

ここで、$r$は零行ではない行の個数、すなわち$A$の階数である

\br

さらに、列の掃き出しで、左上のブロックの成分$*$をすべて0にできる
\begin{equation*}
  \begin{pNiceArray}{ccc|cc}[xdots={horizontal-labels,line-style = <->},first-row,last-col,margin,columns-width =1em]
    \Hdotsfor{3}^{r} & \Hdotsfor{2}^{n-r} \\
    1 & & & \Block{3-2}<\large>{O} && \Vdotsfor{3}^{r}  \\
    & \ddots &&& \\
    & & 1&& \\
    \hline
    \Block{2-3}<\large>{O} && & \Block{2-2}<\large>{O} && \Vdotsfor{2}^{n-r} \\
    &&&&
  \end{pNiceArray}
\end{equation*}

\br

この形を、$A$の\keyword{階数標準形}という

この形を得るまでの過程をまとめると、次のことがいえる

\begin{theorem*}{基本変形による階数標準形の構成}
  任意の行列は、行と列の基本変形を繰り返すことで、階数標準形に変形できる
\end{theorem*}

\br

ここで、$P$を行基本変形に対応する基本行列の積、$Q$を列基本変形に対応する基本行列の積とすると、$A$の階数標準形は$PAQ$で与えられる

\br

\thmref{thm:invertible-as-product-of-elementary}より、基本行列の積は任意の正則行列を表すので、次のようにまとめられる

\begin{theorem}{正則行列による階数標準形の構成}{rank-normal-form-by-regular-matrices}
  $m \times n$型行列$A$に対して、行変形に対応する$m$次正則行列$P$、列変形に対応する$n$次正則行列$Q$が存在し、
  \begin{equation*}
    B = PAQ
  \end{equation*}
  が階数標準形となる
\end{theorem}

\end{document}
