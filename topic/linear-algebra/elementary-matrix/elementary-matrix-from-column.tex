\documentclass[../../../topic_linear-algebra]{subfiles}

\begin{document}

\sectionline
\section{列基本変形と基本行列}
\marginnote{\refbookA p87 \\ \refbookF p61〜62}

行基本変形と同様に、列に関する基本変形を考えることもできる

\begin{enumerate}[label=\romanlabel]
  \item 2つの列を交換する
  \item ある列に0でない数をかける
  \item ある列の定数倍を他の列に加える
\end{enumerate}

列に関する基本変形は、基本行列を右からかけることで実現できる

\begin{theorem}{基本行列による列基本変形の表現}
  行列$A$に列基本変形$\alpha$を行って得られる行列を$B$とすると、
  \begin{equation*}
    B = A E_\alpha
  \end{equation*}
\end{theorem}

\begin{proof}
  転置すると$A$になるような行列$A'$を考える
  \begin{equation*}
    A' = {}^t(A)
  \end{equation*}

  転置すると行と列が入れ替わるので、$A'$に「行」基本変形を施した行列を転置すれば、$A$に同じ基本変形を列に関して施した行列が得られる

  \br

  適用したい基本変形を$\alpha$とし、これを列に関して施す基本行列が$E_\alpha$なら、これを行に関して施す基本行列は${}^t(E_\alpha)$となる

  \br

  よって、
  \begin{equation*}
    B = {}^t({}^t(E_\alpha)A') = {}^t(A'){}^t({}^t(E_\alpha))  = AE_{\alpha}
  \end{equation*}
  というように、\hyperref[thm:transpose-of-product]{積の転置を取ると積の順序が入れ替わる}ことから、行基本変形の場合とは積の順序が逆転することがいえる $\qed$
\end{proof}

\end{document}
