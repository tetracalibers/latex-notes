\documentclass[../../../topic_linear-algebra]{subfiles}

\begin{document}

\sectionline
\section{行基本変形と基本行列}
\marginnote{\refbookA p85〜86 \\ \refbookF p58〜61}

基本変形を行列のかけ算によって実現することができる

\begin{definition}{基本行列}
  基本変形$\alpha$を単位行列$E$に行った結果を$E_\alpha$とするとき、$E_\alpha$を$\alpha$に対応する\keyword{基本行列}と呼ぶ
\end{definition}

行基本変形とは、次の3種類の操作であった

\begin{enumerate}[label=\romanlabel]
  \item 2つの行を交換する
  \item ある行に0でない数をかける
  \item ある行の定数倍を他の行に加える
\end{enumerate}

これらに対応して、行基本変形を表現する基本行列は、次の3種類がある

\begin{enumerate}[label=\romanlabel]
  \item $F(i,j)$:$E$の$i$行と$j$行を交換したもの($i\neq j$)
  \item $G(i;c)$:$E$の$(i,j)$成分を1から$c$に置き換えたもの($c\neq 0$)
  \item $H(i,j;c)$:$E$の$(i,j)$成分を0から$c$に置き換えたもの($i \neq j$)
\end{enumerate}

\begin{equation*}
  F(i,j) = \begin{pmatrix}
    1                                                \\
     & \ddots                                        \\
     &        & 0      & \dots & 1                   \\
     &        & \vdots &       & \vdots              \\
     &        & 1      & \dots & 0                   \\
     &        &        &       &        & \ddots     \\
     &        &        &       &        &        & 1
  \end{pmatrix}
\end{equation*}

\begin{equation*}
  G(i;c) = \begin{pmatrix}
    1                                            \\
     & \ddots                                    \\
     &        & c &                              \\
     &        &   & \ddots                       \\
     &        &   &        & \ddots              \\
     &        &   &        &        & \ddots     \\
     &        &   &        &        &        & 1
  \end{pmatrix}
\end{equation*}

\begin{equation*}
  H(i,j;c) = \begin{pmatrix}
    1                                            \\
     & \ddots                                    \\
     &        & 1 & \dots  & c                   \\
     &        &   & \ddots & \vdots              \\
     &        &   &        & 1                   \\
     &        &   &        &        & \ddots     \\
     &        &   &        &        &        & 1
  \end{pmatrix}
\end{equation*}

\sectionline

行に関する基本変形は、基本行列を左からかけることに他ならない

\begin{theorem*}{基本行列による行基本変形の表現}
  行列$A$に行基本変形$\alpha$を行って得られる行列を$B$とすると、
  \begin{equation*}
    B = E_\alpha A
  \end{equation*}
\end{theorem*}

\begin{proof}
  $\vb*{e}_k$を$k$列目が1で他が0の横ベクトルとし、$A$の$k$行目の行ベクトルを$\vb*{a}_k$とする

  \begin{subpattern}{\bfseries 行の交換}
    基本行列$F(i,j)$の$k$行目は、
    \begin{equation*}
      (F(i,j))_{k,*} = \begin{cases}
        \vb*{e}_j & (k=i)        \\
        \vb*{e}_i & (k=j)        \\
        \vb*{e}_k & (k \neq i,j)
      \end{cases}
    \end{equation*}
    よって、$F(i,j)A$の$k$行目は、
    \begin{equation*}
      (F(i,j)A)_{k,*} = \begin{cases}
        \vb*{a}_j & (k=i)        \\
        \vb*{a}_i & (k=j)        \\
        \vb*{a}_k & (k \neq i,j)
      \end{cases}
    \end{equation*}
    となり、$i$行目と$j$行目が交換されていることがわかる $\qed$
  \end{subpattern}

  \begin{subpattern}{\bfseries 行の定数倍}
    基本行列$G(i;c)$の$k$行目は、
    \begin{equation*}
      (G(i;c))_{k,*} = \begin{cases}
        c \vb*{e}_i & (k=i)      \\
        \vb*{e}_k   & (k \neq i)
      \end{cases}
    \end{equation*}
    よって、$G(i;c)A$の$k$行目は、
    \begin{equation*}
      (G(i;c)A)_{k,*} = \begin{cases}
        c \vb*{a}_i & (k=i)      \\
        \vb*{a}_k   & (k \neq i)
      \end{cases}
    \end{equation*}
    となり、$i$行目が$c$倍されていることがわかる $\qed$
  \end{subpattern}

  \begin{subpattern}{\bfseries 行の定数倍の加算}
    基本行列$H(i,j;c)$の$k$行目は、
    \begin{equation*}
      (H(i,j;c))_{k,*} = \begin{cases}
        \vb*{e}_i + c \vb*{e}_j & (k=i)        \\
        \vb*{e}_j               & (k=j)        \\
        \vb*{e}_k               & (k \neq i,j)
      \end{cases}
    \end{equation*}
    よって、$H(i,j;c)A$の$k$行目は、
    \begin{equation*}
      (H(i,j;c)A)_{k,*} = \begin{cases}
        \vb*{a}_i + c \vb*{a}_j & (k=i)        \\
        \vb*{a}_j               & (k=j)        \\
        \vb*{a}_k               & (k \neq i,j)
      \end{cases}
    \end{equation*}
    となり、$i$行目に$j$行目の$c$倍が加えられていることがわかる $\qed$
  \end{subpattern}
\end{proof}

\end{document}
