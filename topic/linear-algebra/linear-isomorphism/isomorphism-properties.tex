\documentclass[../../../topic_linear-algebra]{subfiles}

\usepackage{xr-hyper}
\externaldocument{../../../.tex_intermediates/topic_linear-algebra}

\begin{document}

\sectionline
\section{線形同型の性質}
\marginnote{\refbookC p93〜94}

ここでは、線形同型写像の恒等写像、逆写像、合成写像との関係を述べる

\subsection{線形同型と恒等写像}

\begin{theorem*}{恒等写像の線形同型性}
  恒等写像は線形同型写像である
\end{theorem*}

\begin{proof}
  恒等写像は明らかに全単射であり、線形写像でもあるため、線形同型写像である $\qed$
\end{proof}

この事実は、部分空間の線形同型に関して次のように言い換えられる

\begin{theorem*}{部分空間の自己同型性}
  部分空間$V$は$V$自身と線形同型である

  すなわち、
  \begin{equation*}
    V \cong V
  \end{equation*}
\end{theorem*}

\subsection{線形同型と逆写像}

\begin{theorem*}{線形同型写像の逆写像}
  線形同型写像の逆写像は線形同型写像である
\end{theorem*}

\begin{proof}
  \todo{\refbookC p93〜94}
\end{proof}

この事実は、部分空間の線形同型に関して次のように言い換えられる

\begin{theorem*}{線形同型性の対称性}
  部分空間$V$が部分空間$W$と線形同型なら、$W$は$V$と線形同型である

  すなわち、
  \begin{equation*}
    V \cong W \Longrightarrow W \cong V
  \end{equation*}
\end{theorem*}

\subsection{線形同型と合成写像}

\begin{theorem*}{線形同型写像の合成}
  線形同型写像の合成は線形同型写像である
\end{theorem*}

\begin{proof}
  \todo{\refbookC p94}
\end{proof}

この事実は、部分空間の線形同型に関して次のように言い換えられる

\begin{theorem*}{線形同型性の推移性}
  部分空間$V$が部分空間$W$と線形同型で、$W$が部分空間$U$と線形同型ならば、$V$は$U$と線形同型である

  すなわち、
  \begin{equation*}
    V \cong W \land W \cong U \Longrightarrow V \cong U
  \end{equation*}
\end{theorem*}

\sectionline

ここまでで登場した、部分空間の線形同型に関する性質をまとめると、

\begin{theorem*}{線形同型の同値関係としての性質}
  \begin{enumerate}[label=\romanlabel]
    \item $V \cong V$
    \item $V \cong W \Longrightarrow W \cong V$
    \item $V \cong W \land W \cong U \Longrightarrow V \cong U$
  \end{enumerate}
\end{theorem*}

となり、これらは、
\begin{shaded}
  同型$\cong$が等号$=$と同じ性質をもつ
\end{shaded}
ことを意味している

\end{document}
