\documentclass[../../../topic_linear-algebra]{subfiles}

\usepackage{xr-hyper}
\externaldocument{../../../.tex_intermediates/topic_linear-algebra}

\begin{document}

\sectionline
\section{線形同型}
\marginnote{\refbookA p101 \\ \refbookC p91〜92}

\keyword{線形同型}は、部分空間が「同じ」であることを述べた概念である

\begin{definition}{線形同型写像}\label{def:linear-isomorphism}
  $V,\,W$を線形空間とし、線形写像$f\colon V \to W$が全単射であるとき、$f$は\keyword{線形同型写像}あるいは単に\keyword{線形同型}であるという

  このとき、同型を表す記号$\cong$を用いて、
  \begin{equation*}
    f\colon V \xrightarrow{\cong} W
  \end{equation*}
  と書くこともある
\end{definition}

\begin{definition}{部分空間の線形同型}\label{def:linear-subspace-isomorphism}
  $V$と$W$の間に線形同型写像が存在するとき、$V$と$W$は\keyword{線形同型}であるとい、
  \begin{equation*}
    V \cong W
  \end{equation*}
  と書く
\end{definition}

\sectionline
\section{線形同型の性質}
\marginnote{\refbookC p93〜94}

ここでは、線形同型写像の恒等写像、逆写像、合成写像との関係を述べる

\subsection{線形同型と恒等写像}

\begin{theorem*}{恒等写像の線形同型性}
  恒等写像は線形同型写像である
\end{theorem*}

\begin{proof}
  恒等写像は明らかに全単射であり、線形写像でもあるため、線形同型写像である $\qed$
\end{proof}

この事実は、部分空間の線形同型に関して次のように言い換えられる

\begin{theorem*}{部分空間の自己同型性}
  部分空間$V$は$V$自身と線形同型である

  すなわち、
  \begin{equation*}
    V \cong V
  \end{equation*}
\end{theorem*}

\subsection{線形同型と逆写像}

\begin{theorem*}{線形同型写像の逆写像}
  線形同型写像の逆写像は線形同型写像である
\end{theorem*}

\begin{proof}
  \todo{\refbookC p93〜94}
\end{proof}

この事実は、部分空間の線形同型に関して次のように言い換えられる

\begin{theorem*}{線形同型性の対称性}
  部分空間$V$が部分空間$W$と線形同型なら、$W$は$V$と線形同型である

  すなわち、
  \begin{equation*}
    V \cong W \Longrightarrow W \cong V
  \end{equation*}
\end{theorem*}

\subsection{線形同型と合成写像}

\begin{theorem*}{線形同型写像の合成}
  線形同型写像の合成は線形同型写像である
\end{theorem*}

\begin{proof}
  \todo{\refbookC p94}
\end{proof}

この事実は、部分空間の線形同型に関して次のように言い換えられる

\begin{theorem*}{線形同型性の推移性}
  部分空間$V$が部分空間$W$と線形同型で、$W$が部分空間$U$と線形同型ならば、$V$は$U$と線形同型である

  すなわち、
  \begin{equation*}
    V \cong W \land W \cong U \Longrightarrow V \cong U
  \end{equation*}
\end{theorem*}

\sectionline

ここまでで登場した、部分空間の線形同型に関する性質をまとめると、

\begin{theorem*}{線形同型の同値関係としての性質}
  \begin{enumerate}[label=\romanlabel]
    \item $V \cong V$
    \item $V \cong W \Longrightarrow W \cong V$
    \item $V \cong W \land W \cong U \Longrightarrow V \cong U$
  \end{enumerate}
\end{theorem*}

となり、これらは、
\begin{shaded}
  同型$\cong$が等号$=$と同じ性質をもつ
\end{shaded}
ことを意味している

\sectionline
\section{線形同型写像と基底}
\marginnote{\refbookC p94}

\begin{theorem*}{線形同型写像による基底の保存}
  線形同型写像$f$によって、部分空間の基底は基底に写る
\end{theorem*}

\begin{proof}
  \thmref{thm:injective-preserves-independence}ことから、$f$の単射性により、基底の線型独立性が保たれる

  また、$f$の全射性により、基底の生成性も保たれる

  よって、$f$によって基底は基底に写る $\qed$
\end{proof}

\sectionline
\section{座標写像}
\marginnote{\refbookA p101 \\ \refbookC p94〜95}

\begin{mindflow}
  \placeholder{再編予定(\refbookS p42〜43)}
\end{mindflow}

\begin{definition}{座標写像}\label{def:coordinate-mapping}
  $V$を線形空間とし、$\mathcal{V} = \{ \vb*{v}_1, \vb*{v}_2, \dots, \vb*{v}_n \}$を$V$の基底とする

  このとき、$K^n$から$V$への線形写像$\Phi_{\mathcal{V}}\colon K^n \to V$を
  \begin{equation*}
    \Phi_{\mathcal{V}}(\vb*{x}) = \sum_{i=1}^n x_i \vb*{v}_i \quad (\vb*{x} \in (x_i)_{i=1}^n \in K^n)
  \end{equation*}
  を$\mathcal{V}$で定まる\keyword{座標写像}と呼ぶ
\end{definition}

このように定めた線形写像が\keyword{座標写像}と呼ばれる背景は、この座標写像が線形同型であることを示し、それがどんな意味を持つのかを考えることでわかる

\begin{theorem*}{線形空間の基底によって定まる線形同型写像}
  $V$を線形空間とし、$\mathcal{V} = \{ \vb*{v}_1, \vb*{v}_2, \dots, \vb*{v}_n \}$を$V$の基底とする

  このとき、$K^n$から$V$への線形写像$\Phi_{\mathcal{V}}\colon K^n \to V$を
  \begin{equation*}
    \Phi_{\mathcal{V}}(\vb*{x}) = \sum_{i=1}^n x_i \vb*{v}_i \quad (\vb*{x} \in (x_i)_{i=1}^n \in K^n)
  \end{equation*}
  と定めると、これは線形同型写像である
\end{theorem*}

\begin{proof}
  線形写像$\Phi_{\mathcal{V}}$が全単射であることを示す

  \begin{subpattern}{\bfseries 単射であること}
    基底$\{ \vb*{v}_1, \vb*{v}_2, \dots, \vb*{v}_n \}$の線型独立性は、
    \begin{equation*}
      \sum_{i=1}^n x_i \vb*{v}_i = \vb*{0}
    \end{equation*}
    で表される線形結合が、$x_i = 0$を満たすことを意味する

    $\Phi_{\mathcal{V}}$の定義をふまえると、この条件は、
    \begin{equation*}
      \Ker(\Phi_{\mathcal{V}}) = \{ \vb*{0} \}
    \end{equation*}
    と書ける

    よって、\thmref{thm:injective-iff-trivial-kernel}より、$\Phi_{\mathcal{V}}$は単射である $\qed$
  \end{subpattern}

  \begin{subpattern}{\bfseries 全射であること}
    基底$\{ \vb*{v}_1, \vb*{v}_2, \dots, \vb*{v}_n \}$が$V$を生成することは、
    \begin{align*}
      \vb*{u} \in V & \Longleftrightarrow \vb*{u} \in \langle \vb*{v}_1, \vb*{v}_2, \dots, \vb*{v}_n \rangle           \\
                    & \Longleftrightarrow \exists (x_i)_{i=1}^n \in K^n \suchthat \vb*{u} = \sum_{i=1}^n x_i \vb*{v}_i \\
                    & \Longleftrightarrow \exists \vb*{x} \in K^n \suchthat \Phi_{\mathcal{V}}(\vb*{x}) = \vb*{u}      \\
                    & \Longleftrightarrow \vb*{u} \in \Im(\Phi_{\mathcal{V}})
    \end{align*}
    という言い換えにより、
    \begin{equation*}
      V = \Im(\Phi_{\mathcal{V}})
    \end{equation*}
    を意味する

    よって、\hyperref[sec:image-and-surjectivity]{像空間と全射性の関係}により、$\Phi_{\mathcal{V}}$は全射である $\qed$
  \end{subpattern}
\end{proof}

この定理を部分空間の線形同型に関して言い換えると、次のような主張になる

\begin{theorem}{有限次元部分空間と数ベクトル空間の線形同型性}{subspace-isomorphic-to-Kn}
  任意の部分空間は次元の等しい数ベクトル空間と線形同型である
\end{theorem}

つまり、
\begin{shaded}
  和とスカラー倍だけに着目すれば、\\
  どんな部分空間も数ベクトル空間と「同じ」
\end{shaded}
ということを意味する

\br

この同型により、部分空間に\keyword{座標}を与えることができる

そしてその座標によって、ベクトルの\keyword{成分表示}が得られる

\end{document}
