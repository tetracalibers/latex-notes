\documentclass[../../../topic_linear-algebra]{subfiles}

\usepackage{xr-hyper}
\externaldocument{../../../.tex_intermediates/topic_linear-algebra}

\begin{document}

\sectionline
\section{線形同型}
\marginnote{
  \refbookA p101 \\ \refbookC p91〜92 \\ \refbookS p42〜43 \\
  \refweb{線形代数の基礎のキソ}{https://www1.econ.hit-u.ac.jp/kawahira/courses/kiso/01-senkei.pdf}
}

線形写像$f\colon V \to U$が全単射であるとき、$f$を\keywordJE{同型写像}{isomorphism}という。

\begin{definition}{線形同型写像}{linear-isomorphism}
  $V,W$を線形空間とし、線形写像$f\colon V \to W$が全単射であるとき、$f$は\keyword{線形同型写像}あるいは単に\keyword{線形同型}であるという。

  このとき、同型を表す記号$\cong$を用いて、次のように表す。
  \begin{equation*}
    f\colon V \xrightarrow{\cong} W
  \end{equation*}
\end{definition}

全単射性から、$V$のベクトル全体と$W$のベクトル全体の間に一対一の対応がつく。

また、線形性より、和・スカラー倍といった基本的な演算も対応がつく。

\br

これより、$W$は$V$を$f$という精巧なレンズで観測した像であり、実体は同じものだとも考えられる。

\begin{definition}{部分空間の線形同型}{linear-subspace-isomorphism}
  $V$と$W$の間に線形同型写像が存在するとき、$V$と$W$は\keyword{線形同型}であるといい、次のように表す。
  \begin{equation*}
    V \cong W
  \end{equation*}
\end{definition}

同型写像はふたつのベクトル空間を写しあう精巧なレンズである。

たとえば、同型写像$f\colon V \to W$があるとき、$f$を通して、$V$の性質を$W$の性質として「観測」することができる。

\br

$W$が未知の線型空間でも、既知の線型空間$V$と同型なら、$W$のことも$V$と同じようによくわかることになる。

特に、既知の線型空間として、数ベクトル空間$K^n$を考えることが多い。

\end{document}
