\documentclass[../../../topic_linear-algebra]{subfiles}

\usepackage{xr-hyper}
\externaldocument{../../../.tex_intermediates/topic_linear-algebra}

\begin{document}

\sectionline
\section{線形同型写像と基底}
\marginnote{\refbookC p94 \\ \refbookS p44}

線形写像$f\colon V \to W$において、$f$が同型写像であることと、$f$が$V$の基底を$W$の基底に写すことは同値である。

\br

基底によって、その線形空間のすべての元(ベクトル)を一意的に表すことができる。

そのため、基底の像がまた基底になることは、
\begin{enumerate}[label=\romanlabel]
  \item $f$では作れないベクトルが$W$に残ることはない($f$の像が$W$を張る)
  \item 異なるベクトルが同じベクトルに潰れることがない($f$の像は線形従属にはならない)
\end{enumerate}
ということを意味する。

ここで、(\romannum{i})は\keyword{全射}の条件、(\romannum{ii})は\keyword{単射}の条件を表しているので、基底の像がまた基底になるなら、$f$は全単射すなわち同型写像となる。

\br

逆に、同型写像は基底を基底に写す写像となる。

% \refbookS 命題2.1.7
\begin{theorem*}{基底の像による線形同型の判定}
  $V$を線形空間とし、$\vb*{v}_1, \ldots, \vb*{v}_n$を$V$の基底とする。
  線形写像$f \colon V \to W$に対し、次の条件は同値である。
  \begin{enumerate}[label=\romanlabel]
    \item $f\colon V \to W$は同型である
    \item $f(\vb*{v}_1), \ldots, f(\vb*{v}_n)$は$W$の基底をなす
  \end{enumerate}
\end{theorem*}

\begin{proof}
  \begin{subpattern}{(\romannum{i}) $\Longrightarrow$ (\romannum{ii})}
    このとき、$f$は単射でもあり、全射でもある。
    
    \thmref{thm:injective-preserves-independence}ことから、$f$の単射性により、基底の線型独立性が保たれる。

    また、\thmref{thm:surjective-iff-full-image}より、$f$の全射性は、$f$の像が$W$を張ることを意味する。

    よって、$f$による像は$W$の基底をなす。 $\qed$
  \end{subpattern}
  
  \begin{subpattern}{(\romannum{ii}) $\Longrightarrow$ (\romannum{i})}
    % https://chatgpt.com/c/68a82263-450c-8322-b3b9-3072769d8ac2
    \todo{}
  \end{subpattern}
\end{proof}

\end{document}
