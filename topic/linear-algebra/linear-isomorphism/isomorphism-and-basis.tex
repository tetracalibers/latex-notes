\documentclass[../../../topic_linear-algebra]{subfiles}

\usepackage{xr-hyper}
\externaldocument{../../../.tex_intermediates/topic_linear-algebra}

\begin{document}

\sectionline
\section{線形同型写像と基底}
\marginnote{\refbookC p94}

\begin{theorem*}{線形同型写像による基底の保存}
  線形同型写像$f$によって、部分空間の基底は基底に写る
\end{theorem*}

\begin{proof}
  \thmref{thm:injective-preserves-independence}ことから、$f$の単射性により、基底の線型独立性が保たれる

  また、$f$の全射性により、基底の生成性も保たれる

  よって、$f$によって基底は基底に写る $\qed$
\end{proof}

\end{document}
