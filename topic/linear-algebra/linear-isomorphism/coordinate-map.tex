\documentclass[../../../topic_linear-algebra]{subfiles}

\usepackage{xr-hyper}
\externaldocument{../../../.tex_intermediates/topic_linear-algebra}

\begin{document}

\sectionline
\section{座標写像}
\marginnote{\refbookA p101 \\ \refbookC p94〜95}

\begin{mindflow}
  \placeholder{再編予定(\refbookS p42〜43)}
\end{mindflow}

\begin{definition}{座標写像}{coordinate-mapping}
  $V$を線形空間とし、$\mathcal{V} = \{ \vb*{v}_1, \vb*{v}_2, \dots, \vb*{v}_n \}$を$V$の基底とする

  このとき、$K^n$から$V$への線形写像$\Phi_{\mathcal{V}}\colon K^n \to V$を
  \begin{equation*}
    \Phi_{\mathcal{V}}(\vb*{x}) = \sum_{i=1}^n x_i \vb*{v}_i \quad (\vb*{x} \in (x_i)_{i=1}^n \in K^n)
  \end{equation*}
  を$\mathcal{V}$で定まる\keyword{座標写像}と呼ぶ
\end{definition}

このように定めた線形写像が\keyword{座標写像}と呼ばれる背景は、この座標写像が線形同型であることを示し、それがどんな意味を持つのかを考えることでわかる

\begin{theorem*}{線形空間の基底によって定まる線形同型写像}
  $V$を線形空間とし、$\mathcal{V} = \{ \vb*{v}_1, \vb*{v}_2, \dots, \vb*{v}_n \}$を$V$の基底とする

  このとき、$K^n$から$V$への線形写像$\Phi_{\mathcal{V}}\colon K^n \to V$を
  \begin{equation*}
    \Phi_{\mathcal{V}}(\vb*{x}) = \sum_{i=1}^n x_i \vb*{v}_i \quad (\vb*{x} \in (x_i)_{i=1}^n \in K^n)
  \end{equation*}
  と定めると、これは線形同型写像である
\end{theorem*}

\begin{proof}
  線形写像$\Phi_{\mathcal{V}}$が全単射であることを示す

  \begin{subpattern}{\bfseries 単射であること}
    基底$\{ \vb*{v}_1, \vb*{v}_2, \dots, \vb*{v}_n \}$の線型独立性は、
    \begin{equation*}
      \sum_{i=1}^n x_i \vb*{v}_i = \vb*{0}
    \end{equation*}
    で表される線形結合が、$x_i = 0$を満たすことを意味する

    $\Phi_{\mathcal{V}}$の定義をふまえると、この条件は、
    \begin{equation*}
      \Ker(\Phi_{\mathcal{V}}) = \{ \vb*{0} \}
    \end{equation*}
    と書ける

    よって、\thmref{thm:injective-iff-trivial-kernel}より、$\Phi_{\mathcal{V}}$は単射である $\qed$
  \end{subpattern}

  \begin{subpattern}{\bfseries 全射であること}
    基底$\{ \vb*{v}_1, \vb*{v}_2, \dots, \vb*{v}_n \}$が$V$を生成することは、
    \begin{align*}
      \vb*{u} \in V & \Longleftrightarrow \vb*{u} \in \langle \vb*{v}_1, \vb*{v}_2, \dots, \vb*{v}_n \rangle           \\
                    & \Longleftrightarrow \exists (x_i)_{i=1}^n \in K^n \suchthat \vb*{u} = \sum_{i=1}^n x_i \vb*{v}_i \\
                    & \Longleftrightarrow \exists \vb*{x} \in K^n \suchthat \Phi_{\mathcal{V}}(\vb*{x}) = \vb*{u}      \\
                    & \Longleftrightarrow \vb*{u} \in \Im(\Phi_{\mathcal{V}})
    \end{align*}
    という言い換えにより、
    \begin{equation*}
      V = \Im(\Phi_{\mathcal{V}})
    \end{equation*}
    を意味する

    よって、\secref{sec:image-and-surjectivity}により、$\Phi_{\mathcal{V}}$は全射である $\qed$
  \end{subpattern}
\end{proof}

この定理を部分空間の線形同型に関して言い換えると、次のような主張になる

\begin{theorem}{有限次元部分空間と数ベクトル空間の線形同型性}{subspace-isomorphic-to-Kn}
  任意の部分空間は次元の等しい数ベクトル空間と線形同型である
\end{theorem}

つまり、
\begin{shaded}
  和とスカラー倍だけに着目すれば、\\
  どんな部分空間も数ベクトル空間と「同じ」
\end{shaded}
ということを意味する

\br

この同型により、部分空間に\keyword{座標}を与えることができる

そしてその座標によって、ベクトルの\keyword{成分表示}が得られる

\end{document}
