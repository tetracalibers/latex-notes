\documentclass[../../../topic_linear-algebra]{subfiles}

\usepackage{xr-hyper}
\externaldocument{../../../.tex_intermediates/topic_linear-algebra}

\begin{document}

\sectionline
\section{座標写像による数ベクトル空間との同型}
\marginnote{\refbookA p101 \\ \refbookC p94〜95 \\ \refbookS p42、44}

$K^n$の\keyword{座標}$(x_1,\dots,x_n)$を、次のように$V$のベクトルに送り込む写像$\Phi$を考える。
\begin{equation*}
  \Phi(x_1,\dots,x_n) = x_1 \vb*{v}_1 + \cdots + x_n \vb*{v}_n
\end{equation*}

ここで、$K^n$の座標$(x_1,\dots,x_n)$は、標準基底$\{ \vb*{e}_1, \dots, \vb*{e}_n \}$を用いたベクトルの成分表示として考えている。(\secref{sec:orthogonal-coordinates-standard-basis})

\begin{mindflow}
  \todo{座標写像が線形写像であること}
\end{mindflow}

\begin{definition}{座標写像}{coordinate-mapping}
  $V$を線形空間とし、$\mathcal{V} = \{ \vb*{v}_1, \dots, \vb*{v}_n \}$を$V$の基底とする。

  このとき、$K^n$から$V$への線形写像$\Phi_{\mathcal{V}}\colon K^n \to V$を次のように定める。
  \begin{equation*}
    \Phi_{\mathcal{V}}(x_1,\dots,x_n) = \sum_{i=1}^n x_i \vb*{v}_i \quad (x_i \in K)
  \end{equation*}
  この写像$\Phi_{\mathcal{V}}$を$\mathcal{V}$で定まる\keyword{座標写像}という。
\end{definition}

\begin{theorem}{座標写像の線形同型性}{coordinate-map-isomorphism}
  \defref*{def:coordinate-mapping}は線形同型写像である。
\end{theorem}

\begin{proof}
  $K^n$の座標$(x_1,\dots,x_n)$を$\vb*{x}$と表記し、
  線形写像$\Phi_{\mathcal{V}}$が全単射であることを示す。

  \begin{subpattern}{\bfseries 単射であること}
    基底$\{ \vb*{v}_1,\dots, \vb*{v}_n \}$の線型独立性は、次の条件を満たすことである。
    \begin{equation*}
      \sum_{i=1}^n x_i \vb*{v}_i = \vb*{o} \Longrightarrow x_i = 0 \quad (i=1,\dots,n)
    \end{equation*}

    $\Phi_{\mathcal{V}}$の定義をふまえると、上の条件は、次のように書ける。
    \begin{equation*}
      \Phi_{\mathcal{V}}(\vb*{x}) = \vb*{o} \Longrightarrow \vb*{x} = \vb*{o}
    \end{equation*}

    よって、\thmref{thm:injective-zero-test}より、$\Phi_{\mathcal{V}}$は単射である $\qed$
  \end{subpattern}

  \begin{subpattern}{\bfseries 全射であること}
    基底の定義より、$\{ \vb*{v}_1, \dots, \vb*{v}_n \}$は$V$を生成する。
    \begin{equation*}
      \langle \vb*{v}_1, \dots, \vb*{v}_n \rangle = V
    \end{equation*}
    
    $\Phi_{\mathcal{V}}$の定義をふまえると、$\Phi_{\mathcal{V}}$は$\vb*{v}_1, \dots, \vb*{v}_n$の線形結合全体、すなわち\defref*{def:span-of-vectors}を像として持つ。
    \begin{equation*}
      \Im \Phi_{\mathcal{V}} = \langle \vb*{v}_1, \dots, \vb*{v}_n \rangle
    \end{equation*}
    
    よって、
    \begin{equation*}
      V = \Im(\Phi_{\mathcal{V}})
    \end{equation*}
    が成り立つため、\thmref{thm:surjective-iff-full-image}より、$\Phi_{\mathcal{V}}$は全射である。 $\qed$
  \end{subpattern}
\end{proof}

\subsection{数ベクトル空間との同型}

\thmref{thm:coordinate-map-isomorphism}を部分空間の線形同型に関して言い換えると、次のような主張になる。

\begin{theorem}{有限次元部分空間と数ベクトル空間の線形同型性}{subspace-isomorphic-to-Kn}
  任意の部分空間$V$は、次元の等しい数ベクトル空間$K^n$と線形同型である。
\end{theorem}

このことはつまり、
\begin{emphabox}
  \begin{spacebox}
    \begin{center}
      和とスカラー倍だけに着目すれば、\\
      どんな部分空間も数ベクトル空間と「同じ」
    \end{center}
  \end{spacebox}
\end{emphabox}
ということを意味する。

\br

この同型により、部分空間に\keyword{座標}を与えることができる。

そしてその座標によって、ベクトルの\keyword{成分表示}が得られる。

\end{document}
