\documentclass[../../../topic_linear-algebra]{subfiles}

\usepackage{xr-hyper}
\externaldocument{../../../.tex_intermediates/topic_linear-algebra}

\begin{document}

\sectionline
\section{零行列と単位行列}

\begin{mindflow}
  \placeholder{再編予定}
\end{mindflow}

\begin{definition*}{零写像と零行列}
  $f\colon \mathbb{R}^n \to \mathbb{R}^m$を、すべての$\vb*{v} \in \mathbb{R}^n$に対して$f(\vb*{v}) = \vb*{o}$と定めたものは明らかに線形写像であり、これを\keyword{零写像}と呼ぶ

  その表現行列はすべての成分が0である行列である

  この行列を\keyword{零行列}と呼び、$O$で表す
\end{definition*}

$m \times n$型であることを明示するために$O_{m,n}$と書くこともある

また、$n$次正方行列の場合は、$O_n$と書く

\sectionline

\begin{definition*}{恒等写像と単位行列}
  $f\colon \mathbb{R}^n \to \mathbb{R}^n$を、すべての$\vb*{v} \in \mathbb{R}^n$に対して$f(\vb*{v}) = \vb*{v}$と定めたものは明らかに線形写像である

  これを\keyword{恒等写像}と呼び、$f = \id_{\mathbb{R}^n}$と書く

  恒等写像の表現行列は、$f(\vb*{e}_j) = \vb*{e}_j \quad (1 \leq j \leq n)$より
  \begin{equation*}
    E = (\vb*{e}_1, \vb*{e}_2, \dots, \vb*{e}_n) = \begin{pmatrix}
      1      & 0      & \dots  & 0      \\
      0      & 1      & \dots  & 0      \\
      \vdots & \vdots & \ddots & \vdots \\
      0      & 0      & \dots  & 1
    \end{pmatrix}
  \end{equation*}
  であり、これを\keyword{単位行列}と呼ぶ
\end{definition*}

単位行列は正方行列であり、$n$次であることを明示したいときは$E_n$と書く

\end{document}
