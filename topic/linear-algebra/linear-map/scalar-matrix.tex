\documentclass[../../../topic_linear-algebra]{subfiles}

\usepackage{xr-hyper}
\externaldocument{../../../.tex_intermediates/topic_linear-algebra}

\begin{document}

\sectionline
\section{行列のスカラー倍}
\marginnote{\refbookS p48 \\ \refbookA p59}

\begin{mindflow}
  \placeholder{再編予定:スカラー行列について書く}
\end{mindflow}

\begin{definition*}{行列のスカラー倍}
  $A$を行列、$c$をスカラーとするとき、$A$のすべての成分を$c$倍して得られる行列を$cA$とする
\end{definition*}

\begin{theorem*}{行列の積とスカラー倍の性質}
  行列$A,\,B$の積$AB$が定義できるとき、つまり$A$の列の個数と$B$の行の個数が同じであるとき、$c \in \mathbb{R}$に対して
  \begin{equation*}
    (cA)B = A(cB) = c(AB)
  \end{equation*}
  が成り立つ
\end{theorem*}

\end{document}
