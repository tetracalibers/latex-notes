\documentclass[../../../topic_linear-algebra]{subfiles}

\usepackage{xr-hyper}
\externaldocument{../../../.tex_intermediates/topic_linear-algebra}

\begin{document}

\sectionline
\section{スカラー行列と行列のスカラー倍}
\marginnote{\refbookA p51、p59 \\ \refbookS p39、p48}

零行列と単位行列は、次の\keywordJE{スカラー行列}{scalar matrix}の特別な場合である。
\begin{equation*}
  C = \begin{pmatrix}
      c      & 0      & \dots  & 0      \\
      0      & c      & \dots  & 0      \\
      \vdots & \vdots & \ddots & \vdots \\
      0      & 0      & \dots  & c
    \end{pmatrix} \quad (c \in K)
\end{equation*}

$c=0$の場合が零行列、$c=1$の場合が単位行列となる。

\subsection{対角成分}

スカラー行列において、$c$が並んでいる部分は\keyword{対角成分}と呼ばれる。

\br

一般に、\keywordJE{対角成分}{diagonal entry}とは、正方行列において、左上から右下に向かう対角線上にある成分のことをいう。
\begin{equation*}
  \begin{pNiceArray}{>{\strut}cccc}% <-- % mandatory
   [margin, extra-margin=2pt,no-cell-nodes]
    \cellcolor{carnationpink!60}a_{11} & a_{12} & a_{13} & a_{14} \\
    a_{21} & \cellcolor{carnationpink!60}a_{22} & a_{23} & a_{24} \\
    a_{31} & a_{32} & \cellcolor{carnationpink!60}a_{33} & a_{34} \\
    a_{41} & a_{42} & a_{43} & \cellcolor{carnationpink!60}a_{44}
  \end{pNiceArray}
\end{equation*}

\br

対角成分の添字に着目すると、$a_{11}, a_{22}, a_{33}, a_{44}$のように、行番号と列番号が等しい成分であることがわかる。
そこで、対角成分は次のように定義される。

\begin{definition*}{対角成分}
  正方行列$A = (a_{ij})$において、$i=j$となる成分$a_{ii}$を$A$の\keyword{対角成分}という。
\end{definition*}

\subsection{スカラー行列との積}

行列$A$にスカラー行列$C$をかけると、$A$のすべての成分が$c$倍される。

行列$A$のすべての成分を$c$倍した行列を、行列$A$の\keyword{スカラー倍}といい、$cA$と表す。

\begin{theorem*}{スカラー行列による積とスカラー倍の一致}
  行列$A$にスカラー行列をかけることは、$A$をスカラー倍することと同じである。
\end{theorem*}

\begin{proof}
  スカラー行列$C = (c_{ij})$は、対角成分が$c$、それ以外の成分が$0$である正方行列なので、各成分は次のように書ける。
  \begin{equation*}
    c_{ij} = \begin{cases}
      c & (i = j) \\
      0 & (i \neq j)
    \end{cases}
  \end{equation*}
  
  $C$と$A$の積$CA$の$(i,j)$成分は、
  \begin{equation*}
    (CA)_{ij} = \sum_{k=1}^{n} c_{ik} a_{kj}
  \end{equation*}
  ここで、$c_{ik}$は$i=k$のとき$c$、それ以外のとき$0$なので、$i=k$の場合の項だけが残って、
  \begin{equation*}
    (CA)_{ij} = \sum_{k=1}^{n} c_{ii} a_{ij} = c a_{ij}
  \end{equation*}
  として、$A$のすべての成分が$c$倍されたものが$CA$となる。$\qed$
\end{proof}

\br

この定理の特別な場合として、次のことがいえる。
\begin{itemize}
  \item $c=0$の場合のスカラー行列(\keyword{零行列})をかけると、$A$のすべての成分が$0$になる
  \item $c=1$の場合のスカラー行列(\keyword{単位行列})をかけても、$A$の成分は変化しない
\end{itemize}

これらは、次の2つの定理としてまとめられる。

\begin{theorem*}{零行列との積}
  零行列をかけると、すべての成分が0になる。
  
  すなわち、$A$を$m \times n$型とするとき、次が成り立つ。
  \begin{equation*}
    O_m A = A O_n = O_{m,n}
  \end{equation*}
\end{theorem*}

\begin{theorem*}{単位行列との積}
  単位行列をかけても、行列は変わらない。
  
  すなわち、$A$を$m \times n$型とするとき、次が成り立つ。
  \begin{equation*}
    E_m A = A E_n = A
  \end{equation*}
\end{theorem*}

\subsection{行列のスカラー倍の性質}

\begin{theorem*}{行列の積とスカラー倍の性質}
  行列$A,B$の積$AB$が定義できる($A$の列の個数と$B$の行の個数が同じである)とき、$c \in K$に対して次が成り立つ。
  \begin{equation*}
    (cA)B = A(cB) = c(AB)
  \end{equation*}
\end{theorem*}

\begin{proof}
  $A\in M_{m,r}(K),\,B\in M_{r,n}(K)$とし、$AB$が定義されているとする。
  
  $1\le i\le m,\,1\le j\le n$について成分を比較する。
  
  \br
  
  まず、
  \begin{equation*}
    \bigl((cA)B\bigr)_{ij}=\sum_{k=1}^{r}(c\,a_{ik})\,b_{kj}
    =c\sum_{k=1}^{r}a_{ik}b_{kj}
    =c\,(AB)_{ij}
  \end{equation*}
  が成り立つ。同様に、
  \begin{equation*}
    \bigl(A(cB)\bigr)_{ij}=\sum_{k=1}^{r}a_{ik}\,(c\,b_{kj})
    =c\sum_{k=1}^{r}a_{ik}b_{kj}
    =c\,(AB)_{ij}
  \end{equation*}
  を得る。
  
  \br
  
  以上より、任意の$i,j$で
  \begin{equation*}
    \bigl((cA)B\bigr)_{ij}=\bigl(A(cB)\bigr)_{ij}=c\,(AB)_{ij}
  \end{equation*}
  であるから、
  \begin{equation*}
    (cA)B=A(cB)=c(AB)
  \end{equation*}
  が成り立つ。$\qed$
\end{proof}

\end{document}
