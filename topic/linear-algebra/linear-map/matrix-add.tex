\documentclass[../../../topic_linear-algebra]{subfiles}

\begin{document}

\sectionline
\section{行列の和}

\begin{mindflow}
  \placeholder{再編予定}
\end{mindflow}

$A, \, B$がともに$m \times n$型行列であるとき、それぞれの$(i, j)$成分を足すことで行列の和$A + B$を定める

\begin{theorem*}{分配法則}
  積が定義できるとき、
  \begin{align*}
    A(B + C) & = AB + AC \\
    (B + C)A & = BA + CA
  \end{align*}
\end{theorem*}

\sectionline

\begin{theorem*}{線形写像の和}
  $f,g\colon \mathbb{R}^n \to \mathbb{R}^m$を線形写像とし、
  \begin{equation*}
    h(\vb*{v}) = f(\vb*{v}) + g(\vb*{v}) \quad (\vb*{v} \in \mathbb{R}^n)
  \end{equation*}
  により写像$h\colon \mathbb{R}^n \to \mathbb{R}^m$を定めるとき、$h$も線形写像である

  また、$f,g$の表現行列を$A,B$とするとき、$h$の表現行列は$A + B$である

  なお、$h = f + g$と書き、$f,g$の\keyword{和}と呼ぶ
\end{theorem*}

\begin{proof}
  \todo{\refbookA p59 (問2.5)}
\end{proof}

\end{document}
