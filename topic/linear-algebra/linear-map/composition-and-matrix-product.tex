\documentclass[../../../topic_linear-algebra]{subfiles}

\usepackage{xr-hyper}
\externaldocument{../../../.tex_intermediates/topic_linear-algebra}

\begin{document}

\sectionline
\section{行列の積と線形写像の合成}
\marginnote{\refbookA p56〜61}

\begin{mindflow}
  \placeholder{再編予定}
\end{mindflow}

% \refbookA 命題2.2.1
\begin{theorem}{線形写像の合成}{linear-map-composition}
  $\mathbb{R}^n$から$\mathbb{R}^m$への線形写像$g$と、$\mathbb{R}^m$から$\mathbb{R}^l$への線形写像$f$が与えられているとき、これらを合成して得られる写像
  \begin{equation*}
    f \circ g\colon \mathbb{R}^n \xrightarrow{g} \mathbb{R}^m \xrightarrow{f} \mathbb{R}^l
  \end{equation*}
  は、$\mathbb{R}^n$から$\mathbb{R}^l$への線形写像である
\end{theorem}

\begin{proof}
  任意の$\vb*{a},\vb*{b} \in \mathbb{R}^n$とスカラー$c_1,c_2 \in \mathbb{R}$について、次の合成写像を考える。
  \begin{equation*}
    (f \circ g)(c_1\vb*{a}+c_2\vb*{b}) = f\bigl(g(c_1\vb*{a}+c_2\vb*{b})\bigr)
  \end{equation*}
  $g$の線形性より、
  \begin{equation*}
    g(c_1\vb*{a}+c_2\vb*{b}) = c_1 g(\vb*{a}) + c_2 g(\vb*{b})
  \end{equation*}
  これを$f$に適用すると、$f$の線形性より、
  \begin{align*}
    f\bigl(c_1 g(\vb*{a}) + c_2 g(\vb*{b})\bigr) &= c_1 f(g(\vb*{a})) + c_2 f(g(\vb*{b})) \\
    &= c_1(f \circ g)(\vb*{a}) + c_2(f \circ g)(\vb*{b})
  \end{align*}
  したがって、
  \begin{equation*}
    (f \circ g)(c_1\vb*{a}+c_2\vb*{b}) = c_1(f \circ g)(\vb*{a}) + c_2(f \circ g)(\vb*{b})
  \end{equation*}
  が成り立つことから、$f \circ g \colon \mathbb{R}^n \to \mathbb{R}^l$は線形写像である。 $\qed$
\end{proof}

$f$と$g$の表現行列をそれぞれ$A = (a_{ij}), \, B = (b_{ij})$とする

$A$は$l \times m$型、$B$は$m \times n$型の行列である

\br

このとき、$f \circ g$は$l \times n$型行列で表現される

それを$C$と書くことにして、その成分を計算しよう

そのためには、基本ベクトルの写り先を見ればよい

\br

$B$を列ベクトルに分解して$B = (\vb*{b}_1, \vb*{b}_2, \dots, \vb*{b}_n)$と書くとき、
\begin{equation*}
  (f \circ g)(\vb*{e}_j) = f(g(\vb*{e}_j)) = f(\vb*{b}_j) = A \vb*{b}_j \quad (1 \leq j \leq n)
\end{equation*}
なので、
\begin{equation*}
  C = (A\vb*{b}_1, A\vb*{b}_2, \dots, A\vb*{b}_n)
\end{equation*}
となる

$C$の$(i, j)$成分は$A\vb*{b}_j$の第$i$成分なので、
\begin{equation*}
  c_{ij} = a_{i1} b_{1j} + a_{i2} b_{2j} + \cdots + a_{im} b_{mj} = \sum_{k=1}^m a_{ik} b_{kj}
\end{equation*}
により与えられる

つまり、$C$の$(i, j)$成分を計算するときは、$A$の第$i$行、$B$の第$j$列だけを見ればよい

\tikzset{highlight/.style={rectangle,
      fill=CarnationPink!30,
      rounded corners,
      fit=#1}}
\begin{equation*}
  \begin{NiceArray}{*{6}{c}@{\hspace{6mm}}*{5}{c}}[nullify-dots]
    \CodeBefore [create-cell-nodes]
    \SubMatrix({2-7}{6-last})
    \SubMatrix({7-2}{last-6})
    \SubMatrix({7-7}{last-last})
    \begin{tikzpicture}
      \node [highlight = (9-2) (9-6), inner sep=4pt] { } ;
      \node [highlight = (2-9) (6-9), inner sep=3pt] { } ;
      \node [highlight = (9-9), shape=circle, inner sep=2pt, fill=Rhodamine!35] {} ;
    \end{tikzpicture}
    \Body
     &        &        &        &        &        &        &        & j\text{\bfseries 列}                   \\
     &        &        &        &        &        & b_{11} & \Cdots & b_{1j}              & \Cdots & b_{1n} \\
     &        &        &        &        &        & \Vdots &        & \Vdots              &        & \Vdots \\
     &        &        &        &        &        &        &        & b_{kj}                                \\
     &        &        &        &        &        &        &        & \Vdots                                \\
     &        &        &        &        &        & b_{n1} & \Cdots & b_{mj}              & \Cdots & b_{mn} \\[3mm]
     & a_{11} & \Cdots &        &        & a_{1m}                                                           \\
     & \Vdots &        &        &        & \Vdots &        &        & \Vdots                                \\
    i\text{\bfseries 行}
     & a_{i1} & \Cdots & a_{ik} & \Cdots & a_{im} & \Cdots &        & c_{ij}                                \\
     & \Vdots &        &        &        & \Vdots                                                           \\
     & a_{n1} & \Cdots &        &        & a_{nm}                                                           \\
    \CodeAfter
    \tikz \draw[CarnationPink, thick, <->, shorten >=0.5em, shorten <=0.5em](9-4.north) to[bend left] (4-9.west);
    \tikz \path [decoration={text along path, text={{$\displaystyle\sum_{k=1}^m a_{ik} b_{kj}$}{}}, text color=Rhodamine, text align=center, raise=4ex}, decorate](9-4.north) to [bend left] (4-9.west);
  \end{NiceArray}
\end{equation*}

\br

このようにして得られた$l \times n$型行列$C$を$AB$と書き、$A$と$B$の\keyword{積}と呼ぶ

\sectionline

\begin{theorem*}{単位行列との積}
  $A$を$m \times n$型とするとき、次が成り立つ
  \begin{align*}
    E_mA & = A \\
    AE_n & = A
  \end{align*}
\end{theorem*}

\begin{theorem*}{零行列との積}
  $A$を$m \times n$型とするとき、次が成り立つ
  \begin{equation*}
    O_m A = A O_n = O_{m,n}
  \end{equation*}
\end{theorem*}

\sectionline

2つの行列の積が順番に依らない場合、2つの行列は\keyword{可換}であるという

\br

一般には、2つの行列は可換であるとは限らない

つまり、$AB$と$BA$は一般には異なる

\br

\todo{\refbookA p58 (例2.2.3, 2.2.4)}

\subsection{行列の積の結合法則}

\begin{theorem*}{積の結合法則}
  積$AB, \, BC$がともに定義できるとき、
  \begin{equation*}
    (AB)C = A(BC)
  \end{equation*}
\end{theorem*}

\begin{proof}[写像による証明]
  $A,\,B, \, C$がそれぞれ$q \times m, \, m \times n, \, n \times p$型行列だとする

  線形写像の合成
  \begin{equation*}
    \mathbb{R}^p \xrightarrow{h} \mathbb{R}^n \xrightarrow{g} \mathbb{R}^m \xrightarrow{f} \mathbb{R}^q
  \end{equation*}
  を考え、$f,\, g, \, h$の表現行列をそれぞれ$A, \, B, \, C$とする

  一般的な写像の合成の性質として、
  \begin{equation*}
    (f \circ g) \circ h = f \circ (g \circ h)
  \end{equation*}
  が成り立つから、
  \begin{equation*}
    (AB)C = A(BC)
  \end{equation*}
  がしたがう $\qed$
\end{proof}

\begin{proof}[積の計算規則による証明]
  $AB$の$(i,l)$成分は、
  \begin{equation*}
    (AB)_{il} = \sum_{k=1}^m a_{ik} b_{kl}
  \end{equation*}
  これを用いて、
  \begin{align*}
    ((AB)C)_{ij} & = \sum_{l=1}^n (AB)_{il} c_{lj}                                 \\
                 & = \sum_{l=1}^n \left( \sum_{k=1}^m a_{ik} b_{kl} \right) c_{lj}
  \end{align*}

  $i,\,j$はいま固定されているので、和には関係がない

  動いているのは$k,\,l$だけ

  \br

  ここで、次の書き換えができる
  \begin{align*}
    \sum_{l=1}^n \left( \sum_{k=1}^m a_{ik} b_{kl} \right) c_{lj} & = \sum_{l=1}^n \left(\sum_{k=1}^m a_{ik} b_{kl} c_{lj} \right) \\
                                                                  & = \sum_{l=1}^n \sum_{k=1}^m a_{ik} b_{kl} c_{lj}
  \end{align*}
  $\displaystyle\sum_{l=1}^n$の右にある式は$l$に関する和をとる前のものなので、$l$は止まっていると考えてよく、単純な分配法則を使っている

  また、括弧がなくても、$k$に関する和を先にとって、その後で$l$に関する和をとっていると読むことができる

  \br

  このとき、和の順番は交換してもよいので、
  \begin{align*}
    \sum_{l=1}^n \sum_{k=1}^m a_{ik} b_{kl} c_{lj} & = \sum_{k=1}^m \sum_{l=1}^n a_{ik} b_{kl} c_{lj}                \\
                                                   & = \sum_{k=1}^m a_{ik} \left( \sum_{l=1}^n b_{kl} c_{lj} \right) \\
                                                   & = \sum_{k=1}^m a_{ik} (BC)_{kj}
  \end{align*}
  先ほどと同様に、$\displaystyle\sum_{k=1}^m$の右では$k$は止まっていると考えている

  そして、この結果は、$A(BC)$の$(i, j)$である $\qed$
\end{proof}

結合法則が成り立つことが示されたので、$(AB)C$または$A(BC)$を表すとき、括弧を書かずに単に$ABC$と書いても問題ない

行列の個数が増えても同様である

\br

また、$A$が正方行列の場合は、
\begin{align*}
  A^2 & = AA  \\
  A^3 & = AAA
\end{align*}
などのように書く

\end{document}
