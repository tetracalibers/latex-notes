\documentclass[../../../topic_linear-algebra]{subfiles}

\begin{document}

\sectionline
\section{線形変換とその表現行列}
\marginnote{\refbookA p51〜56}

特に、\keyword{線形変換}は空間$\mathbb{R}^n$からそれ自身への写像なので、$\mathbb{R}^n$内において「ベクトルが変化している」(あるいは$f$が空間$\mathbb{R}^n$に\keyword{作用}している)ニュアンスとみることができる。

\br

$\mathbb{R}^n$の線形変換の表現行列は、$n$次\keyword{正方行列}である。

\begin{mindflow}
  \todo{$\mathbb{R}^n$の線形変換の具体例を紹介する}
\end{mindflow}

\end{document}
