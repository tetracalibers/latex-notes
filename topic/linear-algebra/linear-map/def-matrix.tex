\documentclass[../../../topic_linear-algebra]{subfiles}

\usepackage{xr-hyper}
\externaldocument{../../../.tex_intermediates/topic_linear-algebra}

\begin{document}

\sectionline
\section{線形写像の記述と行列}
%\marginnote{}

$K^n$の任意のベクトル$\vb*{v}$は、基本ベクトル(標準基底)の線型結合として次のように書ける。
\begin{equation*}
  \vb*{v} = \begin{pmatrix}
    v_1 \\
    \vdots \\
    v_n
  \end{pmatrix} = v_1 \vb*{e}_1 + \cdots + v_n \vb*{e}_n
  = \sum_{j=1}^n v_j \vb*{e}_j 
\end{equation*}

この$\vb*{v}$に、線形写像$f$を作用させると、$f$の線形性より、
\begin{equation*}
  f(\vb*{v})
  = v_1 f(\vb*{e}_1) + \cdots + v_n f(\vb*{e}_n)
  = \sum_{j=1}^n v_j f(\vb*{e}_j)
\end{equation*}

ここで、$v_1, \ldots, v_n$は$\vb*{v}$の成分なので、$f$の引数にどんなベクトルを入れるかによって変わる部分である。
\begin{equation*}
  f(\vb*{v}) = \dlinelabelmath[Cerulean]{v_1}{\small 引数} \fitLabelMath[Rhodamine][carnationpink]{f(\vb*{e}_1)}{\small $f$の構成要素} + \cdots + \dlinelabelmath[Cerulean]{v_n}{\small 引数} \fitLabelMath[Rhodamine][carnationpink]{f(\vb*{e}_n)}{\small $f$の構成要素}
\end{equation*}

よって、$f$自体は、基本ベクトルの像$f(\vb*{e}_1),\ldots,f(\vb*{e}_n)$だけで決まってしまう。

\subsection{行列:線形写像の簡略記法}
\marginnote{
  \refweb{行列式(Determinant)の歴史と導出}{https://kinakomoti321.hatenablog.com/entry/2025/04/10/225230}
}

$f$の構成要素($f$が表す操作)と$f$の引数($f$の操作対象)を分離して、もっと簡潔に書けないか?ということを考える。

\br

基本ベクトル$\vb*{e}_1, \ldots, \vb*{e}_n$が、$f$によって$\vb*{a}_1, \ldots, \vb*{a}_n$というベクトルに写るとしよう。

すなわち、$f(\vb*{e}_j) = \vb*{a}_j$と書き直して、
\begin{equation*}
  f(\vb*{v}) = v_1 \vb*{a}_1 + \cdots + v_n \vb*{a}_n
\end{equation*}

ここで、基本ベクトルの像$\vb*{a}_1, \ldots, \vb*{a}_n$を横に並べたものを$A$とおく。
\begin{equation*}
  A = \begin{pmatrix}
    \vb*{a}_1 & \cdots & \vb*{a}_n
  \end{pmatrix}
\end{equation*}

$A$は、縦ベクトルを横に並べたものなので、結局は縦横に数を並べたものになっている。

このような、縦横に数を並べたものを\keywordJE{行列}{matrix}という。

\br

そして、次のような演算の規則を定める。

\begin{definition*}{行列とベクトルの積}
  行列$A = \begin{pmatrix} \vb*{a}_1 & \cdots & \vb*{a}_n \end{pmatrix}$と$\vb*{v} \in K^n$との\keyword{積}を次のように定める。
  \begin{equation*}
    A\vb*{v} = v_1 \vb*{a}_1 + \cdots + v_n \vb*{a}_n
  \end{equation*}
  ここで、$v_i$は$\vb*{v}$の第$i$成分である。
\end{definition*}

行列とベクトルの積を用いると、$f(\vb*{v})$は次のように簡潔に書ける。
\begin{equation*}
  f(\vb*{v}) = \fitLabelMath[Rhodamine][carnationpink]{A}{\small 操作}\dlinelabelmath[Cerulean]{\vb*{v}}{\small 引数}
\end{equation*}
このとき、行列$A$は線形写像$f$の\keyword{表現行列}と呼ばれる。

\sectionline
\section{行列の定義}
\marginnote{\refbookA 1.4}

前節で述べたように、線形写像の簡略記法として生まれたものが、\keyword{行列}である。

ここでは、行列についての用語をいくつか定義する。

\subsection{行列:縦横に数を並べたもの}

縦ベクトルは数を縦に並べたもの、横ベクトルは数を横に並べたものだった。

縦横に数を並べたものは\keyword{行列}といい、たとえば次のように書く。
\begin{equation*}
  \tikzset{highlight/.style={rectangle,
                           rounded corners = 1mm,
                           opacity=0.6,
                           fit=#1
                           }}
  A = \begin{pNiceArray}{cccc}[last-col, first-row, create-medium-nodes]
    \CodeBefore[create-cell-nodes]
      \begin{tikzpicture}
        \node[fill=SkyBlue!60, highlight=(1-2) (last-2)]{};
        \node[fill=carnationpink!80, highlight=(2-1) (2-last)]{};
      \end{tikzpicture}
    \Body
    & \text{\bfseries \textcolor{Cerulean}{列}}  &        &        & \\
    a_{11} & a_{12} & \dots  & a_{1n} \\
    a_{21} & a_{22} & \dots  & a_{2n} & \text{\bfseries \textcolor{Rhodamine}{行}} \\
    \vdots & \vdots & \ddots & \vdots \\
    a_{m1} & a_{m2} & \dots  & a_{mn}
  \end{pNiceArray}
\end{equation*}

\br

横の数字の並びを\keyword[Rhodamine]{行}、縦の数字の並びを\keyword[Cerulean]{列}という。

\subsection{行列の型}

$A$は、$m$個の行と$n$個の列をもつ行列である。

行が$m$個、列が$n$個の行列を、\keyword{$m$行$n$列の行列}、あるいは\keyword{$m \times n$型行列}という。

\br

$m =n$の場合、すなわち$n \times n$型行列は、正方形状に数を並べたものなので\keyword{$n$次正方行列}という。

\subsection{行列の成分}

第$i$行、第$j$列にある数を$a_{ij}$と表し、これを\keyword{$(i, j)$成分}という。

\begin{equation*}
  \tikzset{highlight/.style={rectangle,
  opacity=0.6,
                           rounded corners = 1mm,
                           fit=#1
                           }}
  A = \begin{pNiceArray}{ccccc}[create-medium-nodes, last-col, first-row]
    \CodeBefore[create-cell-nodes]
      \begin{tikzpicture}
        \node[fill=SkyBlue!60, highlight=(1-3) (last-3)]{};
        \node[fill=carnationpink!80, highlight=(3-1) (3-last)]{};
      \end{tikzpicture}
    \Body
                   &      & \textcolor{Cerulean}{j}     &       & \\
      a_{11} & \dots & a_{1j} & \dots & a_{1n} \\
      \vdots &       & \vdots &       & \vdots \\
      a_{i1} & \dots & a_{ij} & \dots & a_{in} & \textcolor{Rhodamine}{i} \\
      \vdots &       & \vdots &       & \vdots \\
      a_{m1} & \dots & a_{mj} & \dots & a_{mn}
  \end{pNiceArray}
\end{equation*}

\br

行列$A$を、$a_{ij}$成分の集まりとして、次のように略記することもある。
\begin{equation*}
  A = (a_{ij})
\end{equation*}

\subsection{行列の列ベクトル}

行列$A$から、第$j$列だけを取り出して$K^m$のベクトルとしたものは、
\begin{equation*}
  \vb*{a}_j = \begin{pmatrix}
    a_{1j} \\
    a_{2j} \\
    \vdots \\
    a_{mj}
  \end{pmatrix}
\end{equation*}
であり、これを$A$の$j$番目の\keyword{列ベクトル}という。

\br

$m \times n$型行列$A$は、$m$次元の列ベクトルを横に$n$個並べたものという意味で、
\begin{equation*}
  A = \begin{pmatrix}
    \vb*{a}_1 & \cdots & \vb*{a}_n
  \end{pmatrix}
\end{equation*}
と書くこともできる。

\subsection{行列の行ベクトル}

行列$A$から、第$i$行だけを取り出して$K^n$のベクトルとしたものは、
\begin{equation*}
  \vb*{a}_i = \begin{pmatrix}
    a_{i1} & a_{i2} & \cdots & a_{in}
  \end{pmatrix}
\end{equation*}
であり、これを$A$の$i$番目の\keyword{行ベクトル}という。

\br

$m \times n$型行列$A$は、$n$次元の行ベクトルを縦に$m$個並べたものという意味で、
\begin{equation*}
  A = \begin{pmatrix}
    \vb*{a}_1 \\
    \vdots \\
    \vb*{a}_m
  \end{pmatrix}
\end{equation*}
と書くこともできる。

\end{document}
