\documentclass[../../../topic_linear-algebra]{subfiles}

\begin{document}

\sectionline
\section{行列の転置}
\marginnote{\refbookA p78 \\ \refbookF p30}

行列$A=(a_{ij})$に対し、その成分の行と列の位置を交換してできる行列を\keyword{転置行列}という

\begin{definition*}{転置行列}
  $A = (a_{ij})$を$m \times n$型行列とするとき、$(i,j)$成分が$a_{ji}$である$n \times m$型行列を$A$の\keyword{転置行列}と呼び、$\transpose{A}$と表す
\end{definition*}

文字$t$を左肩に書くのは、右肩に書くと$t$乗に見えてしまうからである

$t$乗と区別しつつ、右肩に書く流儀として、$A^T$と書く場合もある

\subsection{ベクトルの転置}

特別な場合として、$n$次の数ベクトル$\vb*{v}$を$n \times 1$型行列とみて転置したもの${}^t\vb*{v}$は$1 \times n$型行列となる

すなわち、数ベクトルの転置は\keyword{横ベクトル}になる

\br

このことを利用して、たとえば
\begin{equation*}
  \begin{pmatrix}
    v_1    \\
    v_2    \\
    \vdots \\
    v_n
  \end{pmatrix}
\end{equation*}
を${}^t (v_1,\,v_2,\,\ldots,\,v_n)$と表記することもある

\subsection{転置の性質}

\keyword{転置}は「行と列の入れ替え」であるので、明らかに次が成り立つ

\begin{theorem}{転置操作の反復不変性}{transpose-involution}
  $\transpose{A}$に対して、転置をもう一度して得られる行列は$A$と一致する
  \begin{equation*}
    {}^t ({}^t A) = {}^t\transpose{A} = A
  \end{equation*}
\end{theorem}

\sectionline

\begin{theorem}{転置と行列積の順序反転性}{transpose-of-product}
  行列$A,\,B$の積$AB$が定義できるとき、
  \begin{equation*}
    {}^t (AB) = \transpose{B} \transpose{A}
  \end{equation*}
\end{theorem}

\begin{proof}
  \todo{\refbookA p78 命題2.5.3}
\end{proof}

\sectionline

\begin{theorem}{行列の和に対する転置の分配性}{transpose-distributes-over-sum}
  $A$と$B$が同じ型の行列であるとき、
  \begin{equation*}
    {}^t(A + B) = {}^t A + {}^t B
  \end{equation*}
\end{theorem}

\begin{proof}
  \todo{}
\end{proof}

\sectionline
\section{対称行列と交代行列}
\marginnote{\refbookF p30}

正方行列$A$が「転置しても元と変わらない」としたら、$A$の成分は左上から右下にかけての対角線に関して\keyword{対称}($a_{ij} = a_{ji}$)になっている

\begin{definition}{対称行列}{symmetric-matrix}
  正方行列$A$が次を満たすとき、$A$を\keyword{対称行列}という
  \begin{equation*}
    \transpose{A} = A
  \end{equation*}
\end{definition}

\begin{definition*}{交代行列}
  正方行列$A$が次を満たすとき、$A$を\keyword{交代行列}という
  \begin{equation*}
    \transpose{A} = -A
  \end{equation*}
\end{definition*}

\end{document}
