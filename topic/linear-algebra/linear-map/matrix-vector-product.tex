\documentclass[../../../topic_linear-algebra]{subfiles}

\usepackage{xr-hyper}
\externaldocument{../../../.tex_intermediates/topic_linear-algebra}

\begin{document}

\sectionline
\section{行列とベクトルの積}
\marginnote{\refbookA p24}

\begin{mindflow}
  \placeholder{再編予定}
\end{mindflow}

行列とベクトルの積$A\vb*{v}$を考えるとき、ほとんどの場合は、$A$が1つ与えられていて$\vb*{v}$がいろいろ動くという意識が強い

それは、行列$A$のことを、ベクトルを与えて別なベクトルを作る
\begin{equation*}
  \text{入力ベクトル} \vb*{v} \rightarrow \text{出力ベクトル} A\vb*{v}
\end{equation*}
という装置、すなわち\keyword{写像}だとみなすことである

% \refbookA 命題1.4.3
\begin{theorem}{行列とベクトルの積の性質}{matrix-vector-product-linear}
  $A, B$を$m \times n$型行列、$\vb*{u}, \vb*{v} \in K^n$、$c \in K$とするとき、次が成り立つ
  \begin{enumerate}[label=\romanlabel]
    \item $A(\vb*{u} + \vb*{v}) = A\vb*{u} + A\vb*{v}$
    \item $A(c\vb*{v}) = cA\vb*{v}$
  \end{enumerate}
\end{theorem}

\begin{proof}
  \begin{subpattern}{(\romannum{i})\bfseries 和の性質}
    \begin{equation*}
      A(\vb*{u} + \vb*{v})
      = \sum_{j=1}^n (u_j + v_j) \vb*{a}_j
      = \sum_{j=1}^n u_j \vb*{a}_j + \sum_{j=1}^n v_j \vb*{a}_j
      = A\vb*{u} + A\vb*{v}
    \end{equation*}
  \end{subpattern}
  
  \begin{subpattern}{(\romannum{ii})\bfseries スカラー倍の性質}
    \begin{equation*}
      A(c\vb*{v})
      = \sum_{j=1}^n (cv_j) \vb*{a}_j
      = c \sum_{j=1}^n v_j \vb*{a}_j
      = cA\vb*{v}
    \end{equation*}
  \end{subpattern}
\end{proof}

\end{document}
