\documentclass[../../../topic_linear-algebra]{subfiles}

\usepackage{xr-hyper}
\externaldocument{../../../.tex_intermediates/topic_linear-algebra}

\begin{document}

\sectionline
\section{行列から定まる線形写像}\label{sec:linear-map-from-matrix}
\marginnote{\refbookA p24}

\secref{sec:matrix-description-linear-map}では、線形写像$f$による像$f(\vb*{v})$を、行列$A$を用いて次のように表した。
\begin{equation*}
  f(\vb*{v}) = \fitLabelMath[Rhodamine][carnationpink]{A}{\small 操作}\dlinelabelmath[Cerulean]{\vb*{v}}{\small 引数}
\end{equation*}

このように、行列とベクトルの積$A\vb*{v}$を考えるとき、「$A$が1つ与えられていて$\vb*{v}$がいろいろ動く」と解釈することが多い。

\br

これは、行列$A$のことを、ベクトルを与えて別なベクトルを作る装置、すなわち\keyword{写像}だとみなすことである。
\begin{equation*}
  \text{\bfseries 入力ベクトル} \vb*{v} \xrightarrow{\, A \times \,} \text{\bfseries 出力ベクトル} A\vb*{v}
\end{equation*}

次の定理は、この「行列$A$を左からかける」という操作が、\keyword{線形写像}であることを示している。

% \refbookA 命題1.4.3
\begin{theorem}{行列とベクトルの積の性質}{matrix-vector-product-linear}
  $A, B$を$m \times n$型行列、$\vb*{u}, \vb*{v} \in K^n$、$c \in K$とするとき、次が成り立つ。
  \begin{enumerate}[label=\romanlabel]
    \item $A(\vb*{u} + \vb*{v}) = A\vb*{u} + A\vb*{v}$
    \item $A(c\vb*{v}) = c(A\vb*{v})$
  \end{enumerate}
\end{theorem}

\begin{proof}
  \begin{subpattern}{(\romannum{i})\bfseries 和の性質}
    \begin{equation*}
      A(\vb*{u} + \vb*{v})
      = \sum_{j=1}^n (u_j + v_j) \vb*{a}_j
      = \sum_{j=1}^n u_j \vb*{a}_j + \sum_{j=1}^n v_j \vb*{a}_j
      = A\vb*{u} + A\vb*{v}
    \end{equation*}
  \end{subpattern}
  
  \begin{subpattern}{(\romannum{ii})\bfseries スカラー倍の性質}
    \begin{equation*}
      A(c\vb*{v})
      = \sum_{j=1}^n (cv_j) \vb*{a}_j
      = c \sum_{j=1}^n v_j \vb*{a}_j
      = c(A\vb*{v})
    \end{equation*}
  \end{subpattern}
\end{proof}

\subsection{行列から線形写像を作る}

\secref{sec:matrix-description-linear-map}で線形写像$f$から行列$A$を作ったのとは逆に、任意の行列から線形写像を作ることもできる。

\begin{theorem}{行列から定まる線形写像}{linear-map-from-matrix}
  $m \times n$型行列$A$に対して、
  \begin{equation*}
    f_A(\vb*{v}) = A\vb*{v} \quad (\vb*{v} \in K^n)
  \end{equation*}
  によって写像$f_A\colon K^n \to K^m$を定めれば、$f_A$は線形写像である。
\end{theorem}

\begin{proof}
  $\vb*{u}, \vb*{v} \in K^n$、$c \in K$とする。
  
  \thmref{thm:matrix-vector-product-linear}より、
  \begin{gather*}
    f_A(\vb*{u} + \vb*{v}) = A(\vb*{u} + \vb*{v}) = A\vb*{u} + A\vb*{v} = f_A(\vb*{u}) + f_A(\vb*{v}) \\
    f_A(c\vb*{v}) = A(c\vb*{v}) = cA\vb*{v} = cf_A(\vb*{v})
  \end{gather*}
  が成り立つので、$f_A$は線形写像である。$\qed$
\end{proof}

\subsection{行列と線形写像の対応}

ここまでの議論をまとめると、行列$A$と線形写像$f_A$の間には、次のような関係がある。

\begin{itemize}
  \item 行列$A$が与えられれば、線形写像$f_A$が定まる
  \item 線形写像$f_A$が与えられれば、行列$A$が定まる
\end{itemize}

このように、行列$A$と線形写像$f_A$は一対一に対応している。

このことから、
\begin{emphabox}
  \begin{spacebox}
    \begin{center}
      行列$A$と線形写像$f_A$は「同じ」ものを表す
    \end{center}
  \end{spacebox}
\end{emphabox}
とみなして議論を進めることも多い。

\br

この同一視の根拠は、\secref{sec:matrix-A-map-isomorphism}でより厳密に議論する。

\end{document}
