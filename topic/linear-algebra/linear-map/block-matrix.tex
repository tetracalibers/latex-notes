\documentclass[../../../topic_linear-algebra]{subfiles}

\usepackage{xr-hyper}
\externaldocument{../../../.tex_intermediates/topic_linear-algebra}

\begin{document}

\sectionline
\section{行列の区分け}
\marginnote{\refbookA p64}

行列を
\begin{equation*}
  A = \begin{pmatrix}
    A_{11} & A_{12} \\
    A_{21} & A_{22}
  \end{pmatrix}
\end{equation*}
のようなブロック型に区分けして計算することがよくある

\br

$A$が$m \times n$型のとき、$m = m_1 + m_2, \, n = n_1 + n_2$として、$A_{ij}$は$m_i \times n_j$型である

\br

また、$B$が$n \times l$型で、$n = n_1 + n_2, \, l = l_1 + l_2$と区分けして
\begin{equation*}
  B = \begin{pmatrix}
    B_{11} & B_{12} \\
    B_{21} & B_{22}
  \end{pmatrix}
\end{equation*}
とするとき、
\begin{align*}
  AB & = \begin{pmatrix}
           A_{11} & A_{12} \\
           A_{21} & A_{22}
         \end{pmatrix} \begin{pmatrix}
                         B_{11} & B_{12} \\
                         B_{21} & B_{22}
                       \end{pmatrix}                             \\
     & = \begin{pmatrix}
           A_{11}B_{11} + A_{12}B_{21} & A_{11}B_{12} + A_{12}B_{22} \\
           A_{21}B_{11} + A_{22}B_{21} & A_{21}B_{12} + A_{22}B_{22}
         \end{pmatrix}
\end{align*}
のように$A_{ij}$などが行列の成分であるかのようにして(ただし積の順序は変えずに)積が計算できる

ここで、$A$の列の区分けと$B$の行の区分けの仕方が同じであることが必要である

\br

3つ以上のブロックに分ける場合も同様である

\end{document}
