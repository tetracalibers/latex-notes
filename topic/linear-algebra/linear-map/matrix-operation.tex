\documentclass[../../../topic_linear-map]{subfiles}

\begin{document}

\sectionline
\section{行列の積}

\begin{theorem}{線形写像の合成}
  $\mathbb{R}^n$から$\mathbb{R}^m$への線形写像$g$と、$\mathbb{R}^m$から$\mathbb{R}^l$への線形写像$f$が与えられているとき、これらを合成して得られる写像
  \begin{equation*}
    f \circ g\colon \mathbb{R}^n \xrightarrow{g} \mathbb{R}^m \xrightarrow{f} \mathbb{R}^l
  \end{equation*}
  は、$\mathbb{R}^n$から$\mathbb{R}^l$への線形写像である
\end{theorem}

\begin{proof}
  \todo{\refbookA p56 (問2.2)}
\end{proof}

$f$と$g$の表現行列をそれぞれ$A = (a_{ij}), \, B = (b_{ij})$とする

$A$は$l \times m$型、$B$は$m \times n$型の行列である

\br

このとき、$f \circ g$は$l \times n$型行列で表現される

それを$C$と書くことにして、その成分を計算しよう

そのためには、基本ベクトルの写り先を見ればよい

\br

$B$を列ベクトルに分解して$B = (\vb*{b}_1, \vb*{b}_2, \dots, \vb*{b}_n)$と書くとき、
\begin{equation*}
  (f \circ g)(\vb*{e}_j) = f(g(\vb*{e}_j)) = f(\vb*{b}_j) = A \vb*{b}_j \quad (1 \leq j \leq n)
\end{equation*}
なので、
\begin{equation*}
  C = (A\vb*{b}_1, A\vb*{b}_2, \dots, A\vb*{b}_n)
\end{equation*}
となる

$C$の$(i, j)$成分は$A\vb*{b}_j$の第$i$成分なので、
\begin{equation*}
  c_{ij} = a_{i1} b_{1j} + a_{i2} b_{2j} + \cdots + a_{im} b_{mj} = \sum_{k=1}^m a_{ik} b_{kj}
\end{equation*}
により与えられる

つまり、$C$の$(i, j)$成分を計算するときは、$A$の第$i$行、$B$の第$j$列だけを見ればよい

\begin{equation*}
  \begin{pmatrix}
           &        &       &        \\
           &        &       &        \\
    a_{i1} & a_{i2} & \dots & a_{im} \\
           &        &       &        \\
           &        &       &
  \end{pmatrix}
  \begin{pmatrix}
     & b_{1j} & \\
     & b_{2j} & \\
     & \vdots & \\
     & b_{mj} &
  \end{pmatrix} = \begin{pmatrix}
          & \vdots                                  &       \\
    \dots & \displaystyle\sum_{k=1}^m a_{ik} b_{kj} & \dots \\
          & \vdots                                  &
  \end{pmatrix}
\end{equation*}

\br

このようにして得られた$l \times n$型行列$C$を$AB$と書き、$A$と$B$の\keyword{積}と呼ぶ

\sectionline

\begin{theorem}{単位行列との積}
  $A$を$m \times n$型とするとき、次が成り立つ
  \begin{align*}
    E_mA & = A \\
    AE_n & = A
  \end{align*}
\end{theorem}

\begin{theorem}{零行列との積}
  $A$を$m \times n$型とするとき、次が成り立つ
  \begin{equation*}
    O_m A = A O_n = O_{m,n}
  \end{equation*}
\end{theorem}

\sectionline

2つの行列の積が順番に依らない場合、2つの行列は\keyword{可換}であるという

\br

一般には、2つの行列は可換であるとは限らない

つまり、$AB$と$BA$は一般には異なる

\br

\todo{\refbookA p58 (例2.2.3, 2.2.4)}

\sectionline
\section{行列の和とスカラー倍}

\end{document}
