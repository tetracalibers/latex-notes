\documentclass[../../../topic_linear-algebra]{subfiles}

\begin{document}

\sectionline
\section{対角行列}

\begin{definition}{対角成分}
  正方行列$A = (a_{ij})$に対して、$a_{ii}$を\keyword{対角成分}と呼ぶ
\end{definition}

\begin{definition}{対角行列}
  対角成分以外の成分がすべて0である正方行列を\keyword{対角行列}と呼ぶ

  $a_{ii} = c_i \quad (1 \leq i \leq n)$である対角行列を次のように表す
  \begin{equation*}
    \diag(c_1, c_2, \ldots, c_n) = \begin{pmatrix}
      c_1    & 0      & \cdots & 0      \\
      0      & c_2    & \cdots & 0      \\
      \vdots & \vdots & \ddots & \vdots \\
      0      & 0      & \cdots & c_n
    \end{pmatrix}
  \end{equation*}
\end{definition}

\begin{theorem}{対角行列と列ベクトルのスカラー倍}
  右から対角行列をかけると、各列ベクトルがスカラー倍になる

  すなわち、$A = (\vb*{a}_1, \vb*{a}_2, \ldots, \vb*{a}_n)$とすると、
  \begin{equation*}
    A \cdot \diag(c_1, c_2, \ldots, c_n) = (c_1\vb*{a}_1, c_2\vb*{a}_2, \ldots, c_n\vb*{a}_n)
  \end{equation*}
  が成り立つ
\end{theorem}

\begin{proof}
  \todo{\refbookA p63 (問2.8)}
\end{proof}

\end{document}
