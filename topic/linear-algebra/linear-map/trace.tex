\documentclass[../../../topic_linear-algebra]{subfiles}

\begin{document}

\sectionline
\section{正方行列のトレース}
\marginnote{\refbookA p64}

\begin{definition*}{対角成分}
  正方行列$A = (a_{ij})$に対して、$a_{ii}$を\keyword{対角成分}と呼ぶ
\end{definition*}

\begin{definition}{トレース}{trace}
  正方行列$A=(a_{ij})$に対して、対角成分の和
  \begin{equation*}
    \sum_{i=1}^n a_{ii}
  \end{equation*}
  を$A$の\keyword{トレース}と呼び、$\tr(A)$と表す
\end{definition}

\begin{theorem*}{トレースの性質}
  \begin{enumerate}[label=\romanlabel]
    \item $\tr(A+B) = \tr(A) + \tr(B)$
    \item $\tr(cA) = c\tr(A)$
    \item $\tr(AB) = \tr(BA)$
  \end{enumerate}
\end{theorem*}

\begin{proof}
  \todo{\refbookA p64 問2.9}
\end{proof}

\end{document}
