\documentclass[../../../topic_linear-algebra]{subfiles}

\begin{document}

\sectionline
\section{線形写像の定義}
\marginnote{\refbookA p2 \\ \refbookS p38〜39}

\begin{definition*}{線形写像と線形性}
  写像$f\colon \mathbb{R}^n \to \mathbb{R}^m$が\keywordJE{線形写像}{linear mapping}であるとは、次の条件を満たすことをいう。
  \begin{enumerate}[label=\romanlabel]
    \item 任意の$c \in \mathbb{R} , \, \vb*{v} \in \mathbb{R}^n$に対して、$f(c\vb*{v}) = cf(\vb*{v})$
    \item 任意の$\vb*{u}, \vb*{v} \in \mathbb{R}^n$に対して、$f(\vb*{u} + \vb*{v}) = f(\vb*{u}) + f(\vb*{v})$
  \end{enumerate}
  これらの性質を写像$f$の\keyword{線形性}という。

  また、$m=n$のとき、線形写像$f\colon \mathbb{R}^n \to \mathbb{R}^n$を$\mathbb{R}^n$の\keywordJE{線形変換}{linear transformation}という。
\end{definition*}

\keyword{線形変換}は空間$\mathbb{R}^n$からそれ自身への写像なので、$\mathbb{R}^n$内において「ベクトルが変化している」(あるいは$f$が空間$\mathbb{R}^n$に\keyword{作用}している)ニュアンスとみることができる

\sectionline

$f\colon \mathbb{R}^n \to \mathbb{R}^m$を線形写像とするとき、(\romannum{i})より、
\begin{equation*}
  f(0 \cdot \vb*{v}) = 0 \cdot f(\vb*{v})
\end{equation*}
なので、
\begin{equation*}
  f(\vb*{o}) = \vb*{o}
\end{equation*}
が成り立つ

\begin{theorem*}{零ベクトルの像}
  零ベクトルは線形写像によって零ベクトルに写される
\end{theorem*}

\sectionline

$m=n=1$のときは、線形写像$f\colon \mathbb{R}^1 \to \mathbb{R}^1$は、通常の意味の関数である

このとき、\romannum{i}の性質から、
\begin{equation*}
  f(c) = f(c \cdot 1) = c \cdot f(1) \quad (c \in \mathbb{R} = \mathbb{R}^1)
\end{equation*}
が成り立つので、$a = f(1) \in \mathbb{R}$とおくと、
\begin{equation*}
  f(x) = ax
\end{equation*}
と書ける

\begin{theorem}{一次元線形写像と比例関数の同一性}{linear-map-R1-is-proportional}
  線形写像$f\colon \mathbb{R}^1 \to \mathbb{R}^1$は、$a$を\keyword{比例定数}とする\keyword{比例関数}である
\end{theorem}

\sectionline
\section{線形写像の表現行列}\label{sec:matrix-rep-of-linear-maps}

$f\colon \mathbb{R}^n \to \mathbb{R}^m$を線形写像とするとき、各基本ベクトル$\vb*{e}_j$の$f$による像を
\begin{equation*}
  f(\vb*{e}_j) = \vb*{a}_j = \begin{pmatrix}
    a_{1j} \\
    a_{2j} \\
    \vdots \\
    a_{mj}
  \end{pmatrix}
\end{equation*}
と書くとする

これらを横に並べることによって、$m$行$n$列の行列を作る
\begin{equation*}
  A = \begin{pmatrix}
    a_{11} & a_{12} & \dots  & a_{1n} \\
    a_{21} & a_{22} & \dots  & a_{2n} \\
    \vdots & \vdots & \ddots & \vdots \\
    a_{m1} & a_{m2} & \dots  & a_{mn}
  \end{pmatrix} = (\vb*{a}_1, \vb*{a}_2, \dots, \vb*{a}_n)
\end{equation*}
この行列$A$を$f$の\keyword{表現行列}という

\br

特に、$\mathbb{R}^n$の線形変換の表現行列は$n$次\keyword{正方行列}である

\sectionline

$\mathbb{R}^n$の一般のベクトル$\vb*{v}$を、基本ベクトルの線型結合として
\begin{equation*}
  \vb*{v} = \sum_{j=1}^n v_j \vb*{e}_j
\end{equation*}
と書く

このとき、$f$の線形性より、
\begin{equation*}
  f(\vb*{v}) = \sum_{j=1}^n v_j f(\vb*{e}_j) = \sum_{j=1}^n v_j \vb*{a}_j
\end{equation*}
となる

このベクトルの第$i$成分は
\begin{equation*}
  a_{i1} v_1 + a_{i2} v_2 + \cdots + a_{in} v_n
\end{equation*}
と書ける

これは$A\vb*{v}$の第$i$成分である

\br

したがって、この記法を踏まえて、次のような表記ができる

\begin{theorem*}{線形写像とその表現行列の関係}
  \begin{equation*}
    f(\vb*{v}) = A\vb*{v}
  \end{equation*}
\end{theorem*}

比例関数が比例定数$a$だけで決まるのと同じように、線形写像は表現行列$A$が与えられれば決まる

\sectionline

\begin{definition*}{零写像と零行列}
  $f\colon \mathbb{R}^n \to \mathbb{R}^m$を、すべての$\vb*{v} \in \mathbb{R}^n$に対して$f(\vb*{v}) = \vb*{o}$と定めたものは明らかに線形写像であり、これを\keyword{零写像}と呼ぶ

  その表現行列はすべての成分が0である行列である

  この行列を\keyword{零行列}と呼び、$O$で表す
\end{definition*}

$m \times n$型であることを明示するために$O_{m,n}$と書くこともある

また、$n$次正方行列の場合は、$O_n$と書く

\sectionline

\begin{definition*}{恒等写像と単位行列}
  $f\colon \mathbb{R}^n \to \mathbb{R}^n$を、すべての$\vb*{v} \in \mathbb{R}^n$に対して$f(\vb*{v}) = \vb*{v}$と定めたものは明らかに線形写像である

  これを\keyword{恒等写像}と呼び、$f = \id_{\mathbb{R}^n}$と書く

  恒等写像の表現行列は、$f(\vb*{e}_j) = \vb*{e}_j \quad (1 \leq j \leq n)$より
  \begin{equation*}
    E = (\vb*{e}_1, \vb*{e}_2, \dots, \vb*{e}_n) = \begin{pmatrix}
      1      & 0      & \dots  & 0      \\
      0      & 1      & \dots  & 0      \\
      \vdots & \vdots & \ddots & \vdots \\
      0      & 0      & \dots  & 1
    \end{pmatrix}
  \end{equation*}
  であり、これを\keyword{単位行列}と呼ぶ
\end{definition*}

単位行列は正方行列であり、$n$次であることを明示したいときは$E_n$と書く

\sectionline

線形写像$f$から行列$A$を作ったのとは逆に、任意の行列から線形写像を作ることができる

\begin{theorem*}{行列から線形写像を作る}
  $m \times n$型行列$A$に対して、
  \begin{equation*}
    f(\vb*{v}) = A\vb*{v} \quad (\vb*{v} \in \mathbb{R}^n)
  \end{equation*}
  によって写像$f\colon \mathbb{R}^n \to \mathbb{R}^m$を定めれば、$f$は線形写像である
\end{theorem*}

\begin{proof}
  行列とベクトルの積の性質より、$f$は線形写像である

  また、$f$の定義から明らかに$A$は$f$の表現行列である $\qed$
\end{proof}

\end{document}
