\documentclass[../../../topic_linear-algebra]{subfiles}

\begin{document}

\sectionline
\section{線形写像の定義}
\marginnote{\refbookA p2 \\ \refbookS p38〜39}

\begin{definition*}{線形写像と線形性}
  写像$f\colon \mathbb{R}^n \to \mathbb{R}^m$が\keywordJE{線形写像}{linear mapping}であるとは、次の条件を満たすことをいう。
  \begin{enumerate}[label=\romanlabel]
    \item 任意の$c \in \mathbb{R} , \, \vb*{v} \in \mathbb{R}^n$に対して、$f(c\vb*{v}) = cf(\vb*{v})$
    \item 任意の$\vb*{u}, \vb*{v} \in \mathbb{R}^n$に対して、$f(\vb*{u} + \vb*{v}) = f(\vb*{u}) + f(\vb*{v})$
  \end{enumerate}
  これらの性質を写像$f$の\keyword{線形性}という。

  また、$m=n$のとき、線形写像$f\colon \mathbb{R}^n \to \mathbb{R}^n$を$\mathbb{R}^n$の\keywordJE{線形変換}{linear transformation}という。
\end{definition*}

\keyword{線形変換}は空間$\mathbb{R}^n$からそれ自身への写像なので、$\mathbb{R}^n$内において「ベクトルが変化している」(あるいは$f$が空間$\mathbb{R}^n$に\keyword{作用}している)ニュアンスとみることができる

\sectionline

$f\colon \mathbb{R}^n \to \mathbb{R}^m$を線形写像とするとき、(\romannum{i})より、
\begin{equation*}
  f(0 \cdot \vb*{v}) = 0 \cdot f(\vb*{v})
\end{equation*}
なので、
\begin{equation*}
  f(\vb*{o}) = \vb*{o}
\end{equation*}
が成り立つ

\begin{theorem*}{零ベクトルの像}
  零ベクトルは線形写像によって零ベクトルに写される
\end{theorem*}

\sectionline

$m=n=1$のときは、線形写像$f\colon \mathbb{R}^1 \to \mathbb{R}^1$は、通常の意味の関数である

このとき、\romannum{i}の性質から、
\begin{equation*}
  f(c) = f(c \cdot 1) = c \cdot f(1) \quad (c \in \mathbb{R} = \mathbb{R}^1)
\end{equation*}
が成り立つので、$a = f(1) \in \mathbb{R}$とおくと、
\begin{equation*}
  f(x) = ax
\end{equation*}
と書ける

\begin{theorem}{一次元線形写像と比例関数の同一性}{linear-map-R1-is-proportional}
  線形写像$f\colon \mathbb{R}^1 \to \mathbb{R}^1$は、$a$を\keyword{比例定数}とする\keyword{比例関数}である
\end{theorem}

\end{document}
