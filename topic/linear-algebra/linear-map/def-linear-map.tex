\documentclass[../../../topic_linear-algebra]{subfiles}

\begin{document}

\sectionline
\section{線形写像と線形性}
\marginnote{
  \refbookA p2 \\ \refbookS p38〜39 \\
  \refweb{線形代数の基礎のキソ}{https://www1.econ.hit-u.ac.jp/kawahira/courses/kiso/01-senkei.pdf}
}

写像$f \colon K^n \to K^m$が与えられたとき、これは$K^n$の出来事、構造、その他もろもろの情報を$K^m$に投影していると考えられる。

このとき、その「写り方」にはどのような性質を期待するべきであろうか?

\br

ベクトルには、和とスカラー倍という2つの演算が備わっていた。

そして、和とスカラー倍の組み合わせが、線形結合として重要な役割を果たしている。

\br

そのため、写った先でも、ベクトルどうしの和・定数倍に関する関係式が保存されるという状況が望ましい。

\begin{definition*}{線形写像と線形性}
  写像$f\colon K^n \to K^m$が\keywordJE{線形写像}{linear mapping}であるとは、次の条件を満たすことをいう。
  \begin{enumerate}[label=\romanlabel]
    \item 任意の$c \in K , \, \vb*{v} \in K^n$に対して、$f(c\vb*{v}) = cf(\vb*{v})$
    \item 任意の$\vb*{u}, \vb*{v} \in K^n$に対して、$f(\vb*{u} + \vb*{v}) = f(\vb*{u}) + f(\vb*{v})$
  \end{enumerate}
  これらの性質を写像$f$の\keyword{線形性}という。

  また、$m=n$のとき、線形写像$f\colon K^n \to K^n$を$K^n$の\keywordJE{線形変換}{linear transformation}という。
\end{definition*}

$f$が線形写像であれば、たとえば$c_1 \vb*{u} + c_2 \vb*{v}$を$f$で写したときに、
\begin{equation*}
  f(c_1 \vb*{u} + c_2 \vb*{v}) = c_1 f(\vb*{u}) + c_2 f(\vb*{v})
\end{equation*}
というように、ベクトル$\vb*{u},\vb*{v}$を$f$で写したものに置き換えただけで、線形結合の形はそのまま保たれる。

\subsection{比例関数の一般化}

線形写像のひとつの解釈として、「\keyword{比例関数}の一般化」という考え方もできる。

\br

$m=n=1$の場合の線形写像$f\colon \mathbb{R} \to \mathbb{R}$は、単に数と数を対応させているので、(写像というより)関数である。
このとき、線形性(\romannum{i})から、
\begin{equation*}
  f(c) = f(c \cdot 1) = c \cdot f(1) \quad (c \in \mathbb{R})
\end{equation*}
が成り立つので、$a = f(1) \in \mathbb{R}$とおくと、次のように書ける。
\begin{equation*}
  f(x) = ax
\end{equation*}

\begin{theorem}{一次元線形写像と比例関数の同一性}{linear-map-R1-is-proportional}
  線形写像$f\colon \mathbb{R} \to \mathbb{R}$は、$a$を\keyword{比例定数}とする\keyword{比例関数}である。
\end{theorem}

もっとも簡単な関数である比例関数が満たすべき性質を抽象化し、高次元の世界で実現しているのが線形写像だとも考えられる。

\subsection{線形写像による零ベクトルの像}

$f\colon \mathbb{K}^n \to \mathbb{K}^m$を線形写像とするとき、線形性(\romannum{i})より、
\begin{equation*}
  f(0 \cdot \vb*{v}) = 0 \cdot f(\vb*{v})
\end{equation*}
よって、次が成り立つ。
\begin{equation*}
  f(\vb*{o}) = \vb*{o}
\end{equation*}

\begin{theorem*}{零ベクトルの像}
  零ベクトルは線形写像によって零ベクトルに写される。
\end{theorem*}

\subsection{局所的な線形写像}

線形写像は「局所的には」ありふれている。

\br

たとえば、あらゆる微分可能な関数は、あらゆる場所で「線形写像+誤差」と局所的に表現される。
局所的に線形写像として近似するのが微分ともいえる。

\end{document}
