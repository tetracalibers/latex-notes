\documentclass[../../../topic_linear-equation]{subfiles}

\begin{document}

\sectionline
\section{連立一次方程式を解く}
\marginnote{\refbookA p25}

方程式を解くということは、次のような問題に答えることである

\begin{enumerate}[label=\Alph*.]
  \item 解は存在するか?
  \item 解が存在する場合、それはただ1つの解か?
  \item 解が複数存在する場合は、どれくらい多く存在するのか?
  \item 解全体の集合を以下にしてわかりやすく表示できるか?
\end{enumerate}

\sectionline
\section{拡大係数行列と解の存在条件}
\marginnote{\refbookA p31〜32}

$A$を$m$行$n$列の行列、$\vb*{b} \in \mathbb{R}^m$とし、線形方程式
\begin{equation*}
  A\vb*{x} = \vb*{b}
\end{equation*}
を考える

これは、$n$個の文字に関する$m$本の連立方程式である

$\vb*{x}$は未知数$x_1, x_2, \dots, x_n$を成分とするベクトルである

\br

このとき、$A$は方程式の\keyword{係数行列}と呼ばれる

$A$の右端に列ベクトル$\vb*{b}$を追加して得られる$m$行$(n+1)$列の行列
\begin{equation*}
  \tilde{A} = (A \mid \vb*{b})
\end{equation*}
を考えて、これを\keyword{拡大係数行列}という

\sectionline

$\vb*{b}=\vb*{0}$の場合、つまり
\begin{equation*}
  A\vb*{x} = \vb*{0}
\end{equation*}
の形の線形連立方程式は\keyword{斉次形}であるという

\br

斉次形の場合は$\vb*{x} = \vb*{0}$が明らかに解になっていて、これを\keyword{自明解}という

したがって、自明解以外に解が存在するかどうかが基本的な問題である

\sectionline

まず、一般の$\vb*{b}$の場合の解の存在(問題A)について考える

\br

$\tilde{A}$は$A$の右端に1列追加して得られるので、掃き出しの過程を考えると、$\rank(\tilde{A})$は$\rank(A)$と等しいか、1だけ増えるかのどちらかであることがわかる

\begin{theorem}{解の存在条件}
  $A$を$m \times n$型行列、$\vb*{b} \in \mathbb{R}^m$とする

  $\tilde{A} = (A \mid \vb*{b})$とおくとき、
  \begin{equation*}
    \rank(\tilde{A}) = \rank(A) \Longleftrightarrow A\vb*{x} = \vb*{b} \text{に解が存在する}
  \end{equation*}
\end{theorem}

\begin{proof}
  \todo{\refbookA p31 (定理1.5.1)}
\end{proof}

\begin{theorem}{解の存在条件の系}
  $A$を$m \times n$型行列とするとき、
  \begin{equation*}
    ^{\forall}\vb*{b} \in \mathbb{R}^m, A\vb*{x} = \vb*{b} \text{の解が存在する} \Longleftrightarrow \rank(A) = m
  \end{equation*}
\end{theorem}

\begin{proof}
  \todo{\refbookA p32 (定理1.5.2, 1.5.3)}
\end{proof}

\sectionline

右端の列に主成分がない場合は、一般には無数個の解が存在する

解の集合が直線を成していたり、もっと高い次元の図形になっていることがある

\br

解が1つに定まらない場合は、解の全体像を知ることが方程式を「解く」ことになる

\sectionline
\section{一般解のパラメータ表示}
\marginnote{\refbookA p33〜36}

係数行列$A$の$n$個の列が、$n$個の変数に対応していることを思い出そう

\begin{definition}{主変数と自由変数}
  行列$A$を行基本変形により行階段形にしたとき、主成分がある列に対応する変数を\keyword{主変数}と呼び、それ以外の変数を\keyword{自由変数}と呼ぶ
\end{definition}

\sectionline

解が存在する場合には、
\begin{equation*}
  \vb*{x} = \vb*{x}_0 + t_1\vb*{u}_1 + t_2\vb*{u}_2 + \cdots + t_{n-r}\vb*{u}_{n-r}
\end{equation*}
という形の一般解の表示(問題Dの答え)が得られる

ここで、$r$は行列$A$の階数である

\sectionline

自由変数、すなわちパラメータの個数を\keyword{解の自由度}と呼ぶ

\begin{align*}
  \text{解の自由度} & = \text{(変数の個数)} - \rank(A) \\
               & = n - r
\end{align*}

これは、解全体の集合が何次元の空間なのかを表している(問題Cの答え)

\sectionline
\section{解の一意性}
\marginnote{\refbookA p37〜}

ここまでの議論で、問題Bが解決している

\begin{theorem}{解の一意性}
  $A\vb*{x} = \vb*{b}$の解が存在するとき、
  \begin{equation*}
    \text{解が一意的である} \Longleftrightarrow \rank(A) = n
  \end{equation*}
  ここで、$n$は変数の個数である
\end{theorem}

\begin{proof}
  \todo{\refbookA p37 (定理1.5.8)}
\end{proof}

斉次形の場合の非自明解の存在問題も解決している

\begin{theorem}{斉次形の非自明解の存在条件}
  斉次形の方程式$A\vb*{x} = \vb*{0}$において、
  \begin{equation*}
    \text{自明解しか存在しない} \Longleftrightarrow \rank(A) = n
  \end{equation*}
  ここで、$n$は変数の個数である
\end{theorem}

\begin{proof}
  斉次形の場合は自明解が常に存在するので、解の一意性は、それ以外の解がないということである $\qed$
\end{proof}

\sectionline

自由変数を$x_{j_1}, \dots, x_{j_{n-r}}$とするとき、一般解の表示
\begin{equation*}
  \vb*{x} = \vb*{x}_0 + t_1\vb*{u}_1 + t_2\vb*{u}_2 + \cdots + t_{n-r}\vb*{u}_{n-r}
\end{equation*}
の$j_k$番目の成分は等式
\begin{equation*}
  x_{j_k} = t_k
\end{equation*}
を意味するので、解が与えられたとき、パラメータの値は直接に読み取れる

\br

このことから、
\begin{equation*}
  \vb*{x} = \vb*{x}_0 + t_1\vb*{u}_1 + t_2\vb*{u}_2 + \cdots + t_{n-r}\vb*{u}_{n-r}
\end{equation*}
によって解を表示する際の$n-r$個のパラメータの値は一意的に定まることがわかる

この事実は、$\vb*{u}_1, \vb*{u}_2, \dots, \vb*{u}_{n-r} \in \mathbb{R}^m$が\keyword{線形独立}であると表現される

\end{document}
