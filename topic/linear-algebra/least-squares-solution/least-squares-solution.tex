\documentclass[../../../topic_linear-algebra]{subfiles}

\begin{document}

\sectionline
\section{最小二乗解}
\marginnote{
  \refbookL p161 \\
  \refbookI p50 \\
  \refweb{線形最小二乗解の射影による理解}{https://qiita.com/takseki/items/8114eb537a7fe78ca8f8} \\
  \refweb{射影行列のイメージと楽しい公式}{https://manabitimes.jp/math/2486}
}

$A\vb*{x} = \vb*{b}$の解がないなら、「せめて$A\vb*{x}$が$\vb*{b}$にできるだけ近くなるような$\vb*{x}$を求めよう」というのが\keyword{最小二乗解}のアプローチである。

\br

最小二乗解とは、次のような\keyword{残差}$J$を最小化するような$\vb*{x}$のことである。
\begin{equation*}
  J = \| A\vb*{x} - \vb*{b} \|^2
\end{equation*}

\subsection{残差の最小化と直交射影}

$\vb*{x}$を動かすと、$A\vb*{x}$は$\Im A$上のさまざまなベクトルをとる。

$m > n$の場合、$\vb*{b}$は$\vb*{x}$よりも高次元のベクトルなので、$\Im A$からはみ出してしまう。

\begin{center}
  \tdplotsetmaincoords{70}{80} % 視点の角度設定
  \begin{tikzpicture}[tdplot_main_coords, scale=3]
    % 平面サイズと傾きの設定
    \def\planesize{1.1}
    \tdplotsetrotatedcoords{-20}{30}{0} % (theta, phi, psi):傾きの設定

    % 原点と点の定義
    \coordinate (O) at (0, 0, 0);
    \coordinate (P) at (0.75, 0.75, 1.75); % 空間上の点 P
    \coordinate (Q) at (0.75, 0.75, 0.75); % 任意の平面上の点(Pの直交射影として仮置き)

    % 座標軸
    %\draw[axis] (O) -- (2.6, 0, 0) node[anchor=north east]{$x$};
    %\draw[axis] (O) -- (0, 1.4, 0) node[anchor=north west]{$y$};
    %\draw[axis] (O) -- (0, 0, 1.8) node[anchor=south]{$z$};

    % 直角
    \draw pic[
        fill=gray!60,
        draw=gray,
        angle radius=3mm,
        angle eccentricity=1.2,
      ] {right angle = P--Q--O};

    % 傾いた平面の描画(tdplot_rotated_coords環境内)
    \begin{scope}[tdplot_rotated_coords]
      \filldraw[SkyBlue!40, draw=none, opacity=0.6,postaction={pattern=crosshatch dots}, pattern color=SkyBlue!60]
      (-\planesize,-\planesize,0) --
      ( \planesize,-\planesize,0) --
      ( \planesize, \planesize,0) --
      (-\planesize, \planesize,0) -- cycle;
      \node[SkyBlue] at (\planesize*0.85,-\planesize*0.75,0) {$\Im A$};
    \end{scope}

    \draw[vector, Rhodamine] (O) -- (P) node[midway, left=1pt]{$\vb*{b}$};
    \draw[vector, Cerulean] (O) -- (Q) node[pos=0.35, right=3pt]{$A\vb*{x}$};
    \draw[vector, Orchid] (P) -- (Q) node[pos=0.4, right=1pt]{$A\vb*{x} - \vb*{b}$};
    
    % 点
    \filldraw[Rhodamine] (P) circle (0.5pt) node[anchor=south]{$P$};
    \filldraw[Cerulean] (Q) circle (0.5pt) node[anchor=west]{$Q$};
    \filldraw[gray] (O) circle (0.5pt) node[anchor=north east]{$O$};
  \end{tikzpicture}
\end{center}

ここで、\hyperref[sec:shortest-distance-projection]{点$P$から最短となる$\Im A$上の点は、$P$を$\Im A$へ直交射影した点$Q$}である。

このことから、$\| A\vb*{x} - \vb*{b} \|$が最小となるのは、
  \begin{spacebox}
    \begin{center}
      $\vb*{b}$を$\Im A$へ直交射影したベクトルが$A\vb*{x}$
    \end{center}
  \end{spacebox}
になっているときだとわかる。

\br

$\Im A$は$A$の\keyword{列空間}($A$の列ベクトルが張る空間)であるから、これまで通り$\mathcal{U}$と表すことにしよう。

すると、最小二乗解$\vb*{x}$が満たすべき条件は、次のように書ける。
\begin{equation*}
  A\vb*{x} = P_{\mathcal{U}}\vb*{b}
\end{equation*}

\begin{theorem}{最小二乗解と列空間への射影の関係}\label{thm:least-squares-projection}
  $A$の列空間を$\mathcal{U}$とし、$\mathcal{U}$への直交射影行列を$P_{\mathcal{U}}$とすると、
  線形方程式$A\vb*{x} = \vb*{b}$の最小二乗解は次の関係を満たす。
  \begin{equation*}
    A\vb*{x} = P_{\mathcal{U}}\vb*{b}
  \end{equation*}
\end{theorem}

\subsection{フルランクの場合の最小二乗解}

$n = \rank A$だと仮定する。
\hyperref[thm:rank-equals-max-indep-cols]{階数は線型独立な列の最大個数}なので、$A$の$n$個の列ベクトルは線型独立である。

\br

$A$の列ベクトルを$\vb*{a}_1, \ldots, \vb*{a}_n$とすると、これらは$\mathcal{U}$を張るベクトルであり、かつ線型独立であるので、$\mathcal{U}$の基底を成す。

よって、$\mathcal{U}$上の任意のベクトルは、これらの線型結合で表される。

\br

直交射影の定義より、$A\vb*{x} - \vb*{b}$は$\mathcal{U}$上のすべてのベクトルに直交する。

\br

\begin{center}
  \tdplotsetmaincoords{70}{80} % 視点の角度設定
  \begin{tikzpicture}[tdplot_main_coords, scale=3]
    % 平面サイズと傾きの設定
    \def\planesize{1.1}
    \tdplotsetrotatedcoords{-20}{30}{0} % (theta, phi, psi):傾きの設定

    % 原点と点の定義
    \coordinate (O) at (0, 0, 0);
    \coordinate (P) at (0.75, 0.75, 1.75); % 空間上の点 P
    \coordinate (Q) at (0.75, 0.75, 0.75); % 任意の平面上の点(Pの直交射影として仮置き)

    % 座標軸
    %\draw[axis] (O) -- (2.6, 0, 0) node[anchor=north east]{$x$};
    %\draw[axis] (O) -- (0, 1.4, 0) node[anchor=north west]{$y$};
    %\draw[axis] (O) -- (0, 0, 1.8) node[anchor=south]{$z$};

    % 直角
    \draw pic[
        fill=gray!60,
        draw=gray,
        angle radius=3mm,
        angle eccentricity=1.2,
      ] {right angle = P--Q--O};

    % 傾いた平面の描画(tdplot_rotated_coords環境内)
    \begin{scope}[tdplot_rotated_coords]
      \filldraw[SkyBlue!40, draw=none, opacity=0.6,postaction={pattern=crosshatch dots}, pattern color=SkyBlue!60]
      (-\planesize,-\planesize,0) --
      ( \planesize,-\planesize,0) --
      ( \planesize, \planesize,0) --
      (-\planesize, \planesize,0) -- cycle;
      \node[SkyBlue] at (\planesize*0.85,-\planesize*0.75,0) {$\mathcal{U}$};
    \end{scope}

    \draw[vector, Rhodamine] (O) -- (P) node[midway, left=1pt]{$\vb*{b}$};
    \draw[vector, Cerulean] (O) -- (Q) node[pos=0.35, right=3pt]{$A\vb*{x}$};
    \draw[vector, Orchid] (P) -- (Q) node[pos=0.4, right=1pt]{$A\vb*{x} - \vb*{b}$};
    
    % 点
    \filldraw[Rhodamine] (P) circle (0.5pt) node[anchor=south]{$P$};
    \filldraw[Cerulean] (Q) circle (0.5pt) node[anchor=west]{$Q$};
    \filldraw[gray] (O) circle (0.5pt) node[anchor=north east]{$O$};
  \end{tikzpicture}
\end{center}

\br

$A\vb*{x} - \vb*{b}$が$\mathcal{U}$上の任意のベクトルに直交するには、
\begin{equation*}
  \vb*{a}_i^\top (A\vb*{x} - \vb*{b}) = 0 \quad (i = 1, \ldots, n)
\end{equation*}
が成り立てばよい($\mathcal{U}$上の任意のベクトルは$\vb*{a}_1, \ldots, \vb*{a}_n$の線型結合で表されるのだから)。

この条件を行列の形に書き直すと、
\begin{equation*}
  A^\top (A\vb*{x} - \vb*{b}) = \vb*{o}
\end{equation*}
となる。

\br

\begin{handout}
  $A^\top (A\vb*{x} - \vb*{b}) = \vb*{o}$は、$\vb*{a}_1, \ldots, \vb*{a}_n$すべてとの内積を並べた形である。
\begin{equation*}
  \begin{pmatrix}
    \vb*{a}_1^\top \\
    \vdots \\
    \vb*{a}_n^\top
  \end{pmatrix} (A\vb*{x} - \vb*{b})
  = \begin{pmatrix}
    \vb*{a}_1^\top (A\vb*{x} - \vb*{b}) \\
    \vdots \\
    \vb*{a}_n^\top (A\vb*{x} - \vb*{b})
  \end{pmatrix}
  = \begin{pmatrix}
    0 \\
    \vdots \\
    0
  \end{pmatrix}
\end{equation*}
\end{handout}

これを展開すると、
\begin{gather*}
  A^\top A \vb*{x} - A^\top \vb*{b} = \vb*{o} \\
  A^\top A \vb*{x} = A^\top \vb*{b}
\end{gather*}
となるので、$A^\top A$が正則であれば、$\vb*{x} = \cdots$の形に整理できる。

\br

$n = \rank A$という仮定のもとでは、次の定理が成り立つ。

\begin{theorem}{線型独立な列ベクトルと自己共役積の正則性}
  $m \times n$型行列$A$において、$m \geq n = \rank A$(列ベクトルが線型独立である)とき、$A^\top A$は正則である。
\end{theorem}

\begin{proof}
  $A^\top A$が正則であることと同値な条件として、\hyperref[thm:invertibility-by-kernel]{$A^\top A$の核空間の次元が0}であることを示す。
  すなわち、次を示せばよい。
  \begin{equation*}
    A^\top A \vb*{x} = \vb*{o} \Longrightarrow \vb*{x} = \vb*{o}
  \end{equation*}
  
  $A^\top A \vb*{x} = \vb*{o}$の両辺に$\vb*{x}^\top$をかけると、
  \begin{align*}
    \vb*{x}^\top A^\top A \vb*{x} = \vb*{o} \\
    (A \vb*{x})^\top (A \vb*{x}) = 0 \\
    \| A \vb*{x} \|^2 = 0
  \end{align*}
  ノルムが0であるベクトルは零ベクトルのみであるから、
  \begin{equation*}
    A \vb*{x} = \vb*{o}
  \end{equation*}
  
  ここで、$A \vb*{x} = \vb*{o}$ということは、$A$の列ベクトルの線形結合として$\vb*{o}$になる$\vb*{x}$が存在することを意味する。
  
  しかし、仮定より$A$の列ベクトルは線型独立なので、$\vb*{x} = \vb*{o}$しか解がない。
  
  \br
  
  よって、$A^\top A \vb*{x} = \vb*{o}$の解空間である核空間$\Ker(A^\top A)$は、零ベクトルしか含まない。

  すなわち、$\dim \Ker(A^\top A) = 0$であり、$A^\top A$は正則である。$\qed$
\end{proof}

\br

$A^\top A$が正則であることがわかったので、$\vb*{x}$は次のように求められる。
\begin{equation*}
  \vb*{x} = (A^\top A)^{-1}A^\top \vb*{b}
\end{equation*}

これが、$m \geq n = \rank A$という条件のもとでの\keyword{最小二乗解}である。

\begin{theorem}{線型独立な列ベクトルに対する最小二乗解}
  $m \times n$型行列$A$において、$m \geq n = \rank A$(列ベクトルが線型独立である)とき、線形方程式$A\vb*{x} = \vb*{b}$の最小二乗解$\vb*{x}$は次のように表される。
  \begin{equation*}
    \vb*{x} = (A^\top A)^{-1}A^\top \vb*{b}
  \end{equation*}
\end{theorem}

\subsection{列空間への射影行列}

$\vb*{x} = (A^\top A)^{-1}A^\top \vb*{b}$の両辺に左から$A$をかけると、
\begin{equation*}
  A\vb*{x} = A(A^\top A)^{-1}A^\top \vb*{b}
\end{equation*}
この式を、\hyperref[thm:least-squares-projection]{最小二乗解と射影行列の関係}$A\vb*{x} = P_{\mathcal{U}}\vb*{b}$と見比べると、
\begin{equation*}
  P_{\mathcal{U}} = A(A^\top A)^{-1}A^\top
\end{equation*}
であることがわかる。

\begin{theorem}{列空間への直交射影の行列表現}
  $m \times n$型行列$A$において、$m \geq n = \rank A$(列ベクトルが線型独立である)とき、$A$の列空間への直交射影行列は次のように表される。
  \begin{equation*}
    P_{\mathcal{U}} = A(A^\top A)^{-1}A^\top
  \end{equation*}
\end{theorem}

\end{document}
