\documentclass[../../../topic_linear-algebra]{subfiles}

\begin{document}

\sectionline
\section{解けない線形方程式}
\marginnote{\refbookL p161}

$\vb*{x}$についての線形方程式を考える。
\begin{equation*}
  A\vb*{x} = \vb*{b}
\end{equation*}

ここで、$A$は$m \times n$型行列、$\vb*{x}$は$n$次元ベクトル、$\vb*{b}$は$m$次元ベクトルである。

\br

$m =n$で、$A$が正則であれば、その逆行列を用いて、一意な解を求めることができる。
\begin{equation*}
  \vb*{x} = A^{-1}\vb*{b}
\end{equation*}

\br

しかし、この線型方程式が解けない場合を扱うこともある。
\begin{enumerate}[label=\romanlabel]
  \item $m > n$($A$が縦長)の場合、条件式の数が未知数の数より多く、方程式は不能(解なし)
  \item $m < n$($A$が横長)の場合、条件式の数が未知数の数より少なく、方程式は不定(解が無数にある)
\end{enumerate}

このような場合には、次のようなアプローチが考えられる。
\begin{enumerate}[label=\romanlabel]
  \item 解が存在しないなら、なるべく近い解を探す
  \item 解が無数にあるなら、その中から適した解を探す
\end{enumerate}

なにをもって「近い」「適した」とすべきかは状況によって異なるが、近い解は\keyword{最小二乗解}、適した解は\keyword{最小ノルム解}とすることが多い。

\end{document}
