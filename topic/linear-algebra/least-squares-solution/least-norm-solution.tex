\documentclass[../../../topic_linear-algebra]{subfiles}

\usepackage{xr-hyper}
\externaldocument{../../../.tex_intermediates/topic_linear-algebra}

\begin{document}

\sectionline
\section{最小ノルム解}
\marginnote{
  \refweb{ノルム最小解の射影による導出}{https://qiita.com/takseki/items/a34c4cb98040a0b7d38c} \\
  \refweb{射影行列}{https://qiita.com/takseki/items/e0de89ab82d4d287a0d7} \\
  \refweb{最小ノルム解の導出と図による理解}{https://manabitimes.jp/math/2747}
}

$A\vb*{x} = \vb*{b}$の解が無数にある場合は、「その中で有用な解を選ぼう」というアプローチをとる。

このような場合、$\| \vb*{x} \|$が最小となる解を選ぶことが多い。

\br

$\| \vb*{x} \|$が最小となる解を選ぶメリットはさまざまある。
\begin{itemize}
  \item 解空間の中で「原点に最も近い」解
  \item 「変位が最小」など、物理学でも頻出する条件を満たす解
\end{itemize}

しかし、$\| \vb*{x} \|$は平方根を含むため、計算が複雑になりがちである。

\br

そのため、2乗して平方根を外したもの、すなわち$\| \vb*{x} \|^2$が最小となる解を考えることが多い。

$\| \vb*{x} \|^2$が最小となる解は、\keyword{最小ノルム解}と呼ばれる。

\subsection{核空間と直交する空間への射影}

\secref{sec:reinterpretation-of-parametric-solution}でも考察したように、$A\vb*{x} = \vb*{b}$の解は次のようにして構成することができる。

\begin{enumerate}
  \item 1つの解$\vb*{x}_0$を見つける
  \item $\Ker A$の任意のベクトル$\vb*{u}$を持ってくれば、$\vb*{x}_0 + \vb*{u}$も解である
\end{enumerate}

\br

\begin{handout}[補足:なんで$\vb*{x}_0 + \vb*{u}$も解だといえるのか?]
  $A$によって$\vb*{o}$に写ってしまうような、つまり$A\vb*{u}=\vb*{o}$となるような$\vb*{u}$すべての集合が$\Ker A$である。
  
  \br
  
  $A\vb*{x}_0 = \vb*{b}$が成り立つことと合わせると、
  \begin{equation*}
    A(\vb*{x}_0 + \vb*{u}) = A\vb*{x}_0 + A\vb*{u} = A\vb*{x}_0 + \vb*{o} = A\vb*{x}_0 = \vb*{b}
  \end{equation*}
  となり、たしかに$\vb*{x} = \vb*{x}_0 + \vb*{u}$も$A\vb*{x} = \vb*{b}$を満たすことがわかる。
\end{handout}

\br

そのため、$\Ker A$を$\vb*{x}_0$だけ平行移動したものが、$A\vb*{x} = \vb*{b}$の解空間となる。
\begin{equation*}
  \text{\bfseries 解空間} = \{ \vb*{x}_0 + \vb*{u} \mid \vb*{u} \in \Ker A \}
\end{equation*}

\br

\begin{center}
  \tdplotsetmaincoords{70}{80}
\begin{tikzpicture}[tdplot_main_coords, scale=3]

  \tdplotsetrotatedcoords{30}{-10}{20}
  \def\planesize{1.1}

  % 点の定義
  \coordinate (O) at (0, -0.25, 0);
  \coordinate (P) at (0.75, 0.5, 1.25);
  \coordinate (Q) at (P |- O);
  \coordinate (R) at (O |- P);

  % 直角
    \draw pic[
        fill=gray!60,
        draw=gray,
        angle radius=3mm,
        angle eccentricity=1.2,
      ] {right angle = P--Q--O};

  % PとQを通る直線
  \draw[thick, BurntOrange, densely dashed] ($(P)!1.75!(Q)$) -- ($(Q)!1.5!(P)$) node[above] {\bfseries $A\vb*{x} = \vb*{b}$の解空間};
  % OとRを通る直線
  \draw[thick, Orchid, densely dashed] ($(R)!1.75!(O)$) -- ($ (O)!1.5!(R) $) node[above] {$\Ker A$};
  
  % 平面 Im A
  \begin{scope}[tdplot_rotated_coords]
    \filldraw[SkyBlue!40, draw=none, opacity=0.6, postaction={pattern=crosshatch dots}, pattern color=SkyBlue!60]
      (-\planesize,-\planesize,0) --
      ( \planesize,-\planesize,0) --
      ( \planesize, \planesize,0) --
      (-\planesize, \planesize,0) -- cycle;
    \node[ProcessBlue] at (-\planesize*0.8,-\planesize*0.7,0) {$(\Ker A)^\perp$};
  \end{scope}

  % ベクトル
  \draw[vector, Rhodamine] (O) -- (P) node[midway, left]{$\vb*{x}$};
  \draw[vector, Cerulean] (O) -- (Q) node[below, midway]{$\vb*{x}_0$};
  \draw[vector, Orchid] (Q) -- (P) node[midway, right=1pt]{$\vb*{u} \in \Ker A$};
  \draw[vector, Orchid] (O) -- (R) node[midway, left]{$\vb*{u}$};
  
  % 点描画
  \filldraw[Rhodamine] (P) circle (0.5pt) node[anchor=west]{$P$};
  \filldraw[Cerulean] (Q) circle (0.5pt) node[anchor=west]{$Q$};
  \filldraw[gray] (O) circle (0.5pt) node[anchor=north east]{$O$};
\end{tikzpicture}
\end{center}

さて、$A\vb*{x} = \vb*{b}$の解
\begin{equation*}
  \vb*{x} = \vb*{x}_0 + \vb*{u} \quad (\vb*{u} \in \Ker A)
\end{equation*}
において、$\vb*{x}$のノルム(長さ)が最小となるのは、明らかに$\vb*{u} = \vb*{o}$のときである。

つまり、$\vb*{x} = \vb*{x}_0$が\keyword{最小ノルム解}となるが、$\vb*{x}_0$を使わずに最小ノルム解を表現したい。

\br

図を見ると、$\vb*{x}_0$は、$\vb*{x}$を平面$(\Ker A)^\perp$へ\keyword{直交射影}したものであることがわかるので、
\begin{emphabox}
  \begin{spacebox}
    \begin{center}
      解$\vb*{x}$のノルムを最小化するには、\\
      $\vb*{x}$を$\Ker A$と直交する空間に直交射影すればよい
    \end{center}
  \end{spacebox}
\end{emphabox}
ということがいえる。

\br

ここで、\thmref{thm:kernel-row-orthogonality}より、「$\Ker A$と直交する空間」は、$A$の行空間$\Im A^\top$である。

\br

まとめると、$\| \vb*{x} \|$が最小となるのは、
\begin{spacebox}
  \begin{center}
    $\vb*{x}_0$を\keyword{行空間}$\Im A^\top$へ直交射影したベクトルが$\vb*{x}$
  \end{center}
\end{spacebox}
となっているときだとわかる。

\br

行空間$\Im A^\top$をこれまで通り$\mathcal{V}$と表すことにすると、最小ノルム解は次のように表現できる。
\begin{equation*}
  \vb*{x} = P_{\mathcal{V}} \vb*{x}_0 \quad (A\vb*{x}_0 = \vb*{b})
\end{equation*}

\begin{theorem}{最小ノルム解と行空間への直交射影の関係}{min-norm-solution-projection}
  $A$の行空間を$\mathcal{V}$とし、$\mathcal{V}$への直交射影行列を$P_{\mathcal{V}}$とすると、
  線形方程式$A\vb*{x} = \vb*{b}$の最小ノルム解は次の関係を満たす。
  \begin{equation*}
    \vb*{x} = P_{\mathcal{V}} \vb*{x}_0 \quad (A\vb*{x}_0 = \vb*{b})
  \end{equation*}
\end{theorem}

\subsection{行空間への直交射影行列}

転置すると行と列が入れ替わるため、$A$の行空間は、$A^\top$の列空間と言い換えられる。

\br

そのため、$A$の\thmref{thm:projection-matrix-colspace}の式
\begin{equation*}
  P_{\mathcal{U}} = A(A^\top A)^{-1}A^\top
\end{equation*}
において、$A$を$A^\top$に置き換えたものが、$A$の行空間への直交射影行列となる。
\begin{equation*}
  P_{\mathcal{V}} = A^\top (AA^\top)^{-1} A
\end{equation*}

\begin{theorem*}{行空間への直交射影の行列表現}
  $m \times n$型行列$A$において、$n \geq m = \rank A$(行ベクトルが線型独立)のとき、$A$の行空間への直交射影行列は次のように表される。
  \begin{equation*}
    P_{\mathcal{V}} = A^\top (AA^\top)^{-1} A
  \end{equation*}
\end{theorem*}

\subsection{行フルランクの場合の最小ノルム解}

$\rank A = m$という条件のもとでは、$P_{\mathcal{V}}$を
\begin{equation*}
  P_{\mathcal{V}} = A^\top (AA^\top)^{-1} A
\end{equation*}
という式で表現できるので、これを$\vb*{x} = P_{\mathcal{V}} \vb*{x}_0$という関係に代入すると、
\begin{equation*}
  \begin{WithArrows}
    \vb*{x} &= A^\top (AA^\top)^{-1} A \vb*{x}_0 \Arrow{$A\vb*{x}_0 = \vb*{b}$} \\
    &= A^\top (AA^\top)^{-1} \vb*{b}
  \end{WithArrows}
\end{equation*}
という形で、最小ノルム解が求まる。

\begin{theorem*}{線型独立な行ベクトルに対する最小ノルム解}
  $m \times n$型行列$A$において、$n \geq m = \rank A$(行ベクトルが線型独立である)とき、線形方程式$A\vb*{x} = \vb*{b}$の最小ノルム解$\vb*{x}$は次のように表される。
  \begin{equation*}
    \vb*{x} = A^\top (AA^\top)^{-1} \vb*{b}
  \end{equation*}
\end{theorem*}

\end{document}
