\documentclass[../../../topic_linear-algebra]{subfiles}

\usepackage{xr-hyper}
\externaldocument{../../../.tex_intermediates/topic_linear-algebra}

\begin{document}

\sectionline
\section{射影行列の冪等性と対称性}
\marginnote{\refbookI p7}

「一度射影した点をもう一度射影しても変化しない」という性質は、次のような数式として表現できる

\begin{theorem*}{射影行列の冪等性}
  射影行列$P_{\mathcal{U}}$は\keyword{冪等}である
  \begin{equation*}
    P_{\mathcal{U}}^2 = P_{\mathcal{U}}
  \end{equation*}
\end{theorem*}

\begin{proof}
  \begin{align*}
    P_{\mathcal{U}}^2 & = \left( \sum_{i=1}^{r} \vb*{u}_i\vb*{u}_i^\top \right) \left( \sum_{j=1}^{r} \vb*{u}_j\vb*{u}_j^\top \right) \\
                      & = \sum_{i=1}^{r} \sum_{j=1}^{r} \vb*{u}_i\vb*{u}_i^\top \vb*{u}_j\vb*{u}_j^\top                               \\
                      & = \sum_{i=1}^{r} \sum_{j=1}^{r} \vb*{u}_i (\vb*{u}_i^\top \vb*{u}_j) \vb*{u}_j^\top                           \\
                      & = \sum_{i=1}^{r} \sum_{j=1}^{r} \delta_{ij}\vb*{u}_i \vb*{u}_j^\top
  \end{align*}
  ここで、$\delta_{ij}$を含むことから、$i = j$の場合のみ項が残り、
  \begin{equation*}
    P_{\mathcal{U}}^2 = \sum_{i=1}^{r} \vb*{u}_i \vb*{u}_i^\top = P_{\mathcal{U}}
  \end{equation*}
  が得られる $\qed$
\end{proof}

\br

この$P_{\mathcal{U}}^2 = P_{\mathcal{U}}$という式は、一般の(必ずしも直交射影でない)射影の定義として用いられる

\br

次の性質は、射影が直交射影であることを示すものである

\begin{theorem*}{射影行列の対称性}
  射影行列$P_{\mathcal{U}}$は、対称行列である
  \begin{equation*}
    P_{\mathcal{U}}^\top = P_{\mathcal{U}}
  \end{equation*}
\end{theorem*}

\begin{proof}
  \begin{equation*}
    P_{\mathcal{U}}^\top = \left( \sum_{i=1}^{r} \vb*{u}_i\vb*{u}_i^\top \right)^\top
  \end{equation*}

  \thmref{thm:transpose-distributes-over-sum}より、和の転置は転置の和であることを用いて、
  \begin{equation*}
    P_{\mathcal{U}}^\top = \sum_{i=1}^{r} (\vb*{u}_i\vb*{u}_i^\top)^\top
  \end{equation*}

  また、\thmref{thm:transpose-of-product}より、積の転置は転置の積だが、積の順序が入れ替わることに注意して、
  \begin{equation*}
    P_{\mathcal{U}}^\top = \sum_{i=1}^{r} (\vb*{u}_i^\top)^\top \vb*{u}_i^\top
  \end{equation*}

  \thmref{thm:transpose-involution}より、転置の転置をとると元に戻るので、
  \begin{equation*}
    P_{\mathcal{U}}^\top = \sum_{i=1}^{r} \vb*{u}_i \vb*{u}_i^\top = P_{\mathcal{U}}
  \end{equation*}
  が得られる $\qed$
\end{proof}

\sectionline

射影行列は冪等かつ対称であるが、その逆も成り立つ

\begin{theorem*}{対称性と冪等性による射影行列の特徴づけ}
  対称かつ冪等な行列は、ある部分空間への射影行列となる
\end{theorem*}

\begin{proof}
  \thmref{thm:eigenvalues-of-hermitian-are-real}より、$n$次対称行列$P$は、$n$個の実数の固有値$\lambda_1,\ldots,\lambda_n$を持つ

  これらに属する固有ベクトルを$\vb*{u}_1, \ldots, \vb*{u}_n$とすると、\thmref{thm:hermitian-eigenvectors-orthogonality}より、$\vb*{u}_i$は互いに直交する

  \br

  固有値と固有ベクトルの関係から、
  \begin{equation*}
    P\vb*{u}_i = \lambda_i \vb*{u}_i
  \end{equation*}
  両辺に左から$P$をかけると、
  \begin{gather*}
    P^2\vb*{u}_i = \lambda_i P\vb*{u}_i = \lambda_i \cdot \lambda_i \vb*{u}_i = \lambda_i^2 \vb*{u}_i \\
    \therefore \quad P^2\vb*{u}_i = \lambda_i^2 \vb*{u}_i
  \end{gather*}

  \br

  ここで、$P$は冪等なので、$P^2 = P$が成り立つ
  \begin{equation*}
    P^2\vb*{u}_i = P\vb*{u}_i = \lambda_i \vb*{u}_i
  \end{equation*}
  これを用いると、
  \begin{equation*}
    \lambda_i \vb*{u}_i = \lambda_i^2 \vb*{u}_i
  \end{equation*}
  固有ベクトル$\vb*{u}_i$は零ベクトルではないので、
  \begin{equation*}
    \lambda_i = \lambda_i^2
  \end{equation*}
  よって、
  \begin{align*}
    \lambda_i^2 - \lambda_i  & = 0     \\
    \lambda_i(\lambda_i - 1) & = 0     \\
    \lambda_i                & = 0 , 1
  \end{align*}
  すなわち、固有値は$0$か$1$のいずれかである

  \br

  そこで、
  \begin{align*}
    \lambda_1 = \cdots = \lambda_r     & = 1 \\
    \lambda_{r+1} = \cdots = \lambda_n & = 0
  \end{align*}
  とおくと、
  \begin{equation*}
    P\vb*{u}_i = \begin{cases}
      \vb*{u}_i & (i = 1,\ldots,r)   \\
      0         & (i = r+1,\ldots,n)
    \end{cases}
  \end{equation*}

  よって、$P$は$\{\vb*{u}_1, \ldots, \vb*{u}_r\}$の張る部分空間への射影行列である $\qed$
\end{proof}

\end{document}
