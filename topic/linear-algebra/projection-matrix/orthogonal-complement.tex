\documentclass[../../../topic_linear-algebra]{subfiles}

\usepackage{xr-hyper}
\externaldocument{../../../.tex_intermediates/topic_linear-algebra} % 親の .aux を参照(パスは環境に合わせて)

\begin{document}

\sectionline
\section{直交補空間}
\marginnote{\refbookC p136〜137}

内積を導入したことで、ベクトルの長さや直交性が利用できるようになった。

直交性は、ベクトルだけでなく、部分空間に対しても拡張できる。

\br

計量空間の部分空間に直交するベクトルの集合を、\keyword{直交補空間}という。

\begin{definition}{直交補空間}
  計量空間$V$の部分空間$W$に対し、$W$の\keyword{直交補空間}$W^\perp$を次のように定義する。
  \begin{equation*}
    W^\perp \coloneq \{ \vb*{v} \in V \mid \forall \vb*{w} \in W, (\vb*{v}, \vb*{w}) = 0 \}
  \end{equation*}
\end{definition}

\subsection{直交補空間は$V$の部分空間}

直交補空間もまた、計量空間の部分空間になっている。

\begin{theorem*}{直交補空間の部分空間性}
  計量空間$V$の部分空間$W$の直交補空間$W^\perp$は、計量空間$V$の部分空間である。
\end{theorem*}

\begin{proof}
  \begin{subpattern}{\bfseries 和について}
    $\vb*{a}_1, \vb*{a}_2 \in W^\perp$とすると、任意の$\vb*{b} \in W$に対して、
    \begin{equation*}
      (\vb*{a}_1 + \vb*{a}_2, \vb*{b}) = (\vb*{a}_1, \vb*{b}) + (\vb*{a}_2, \vb*{b}) = 0 + 0 = 0
    \end{equation*}
    となるので、$\vb*{a}_1 + \vb*{a}_2 \in W^\perp$である。 $\qed$
  \end{subpattern}

  \begin{subpattern}{\bfseries スカラー倍について}
    $\vb*{a} \in W^\perp$とすると、任意のスカラー$c \in K$と任意の$\vb*{b} \in W$に対して、
    \begin{equation*}
      (c\vb*{a}, \vb*{b}) = c(\vb*{a}, \vb*{b}) = c \cdot 0 = 0
    \end{equation*}
    となるので、$c\vb*{a} \in W^\perp$である。 $\qed$
  \end{subpattern}
\end{proof}

\sectionline
\section{直交補空間による直和分解}
\marginnote{\refbookC p137〜139}

「直交補空間」という名前は、「補集合」と同様に、何らかの集合を補う集合であることを想起させる。

実際、直交補空間$W^\perp$は、もとの集合$W$を補い、$V$全体を構成するような性質を持つ。

\br

\begin{theorem*}{直交補空間を用いた計量空間の分解}
  計量空間$V$の部分空間$W$に対して、
  \begin{equation*}
    V = W + W^\perp
  \end{equation*}
\end{theorem*}

\begin{proof}
  $W = \{ \vb*{o} \}$の場合は、任意の$\vb*{v} \in V$に対して$\vb*{o}$との内積は0になることから、$W^\perp$は$V$全体となる。
  \begin{equation*}
    W^\perp = \{ \vb*{v} \in V \mid  (\vb*{v}, \vb*{o}) = 0 \} = V
  \end{equation*}
  よって、
  \begin{equation*}
    V = W + W^\perp = \{ \vb*{o} \} + V = V
  \end{equation*}
  が成り立つ。

  \br

  以降、$W \neq \{ \vb*{o} \}$とする。

  $W$の基底$\{ \vb*{w}'_1, \ldots, \vb*{w}'_k \}$を1つとり、これに対してグラム・シュミットの直交化法を適用して、正規直交基底$\{ \vb*{w}_1, \ldots, \vb*{w}_k \}$を得る。

  \br

  任意の$\vb*{v} \in V$をとり、次のようにおく。
  \begin{equation*}
    \vb*{u} = \vb*{v} - \sum_{i=1}^k (\vb*{v}, \vb*{w}_i) \vb*{w}_i
  \end{equation*}
  $\vb*{u}$と$\vb*{w}_i$の内積を計算すると、
  \begin{align*}
    (\vb*{u}, \vb*{w}_i) & = \left( \vb*{v} - \sum_{j=1}^k (\vb*{v}, \vb*{w}_j) \vb*{w}_j, \vb*{w}_i \right) \\
                         & = (\vb*{v}, \vb*{w}_i) - \sum_{j=1}^k (\vb*{v}, \vb*{w}_j) (\vb*{w}_j, \vb*{w}_i) \\
                         & = (\vb*{v}, \vb*{w}_i) - \sum_{j=1}^k (\vb*{v}, \vb*{w}_j) \delta_{ij}            \\
                         & = (\vb*{v}, \vb*{w}_i) - (\vb*{v}, \vb*{w}_i)                                     \\
                         & = 0
  \end{align*}
  このように、任意の$i = 1, \ldots, k$に対して、$\vb*{u}$と$\vb*{w}_i$の内積が0になることから、$\vb*{u} \in W^\perp$である。

  \br

  一方、$\vb*{u}$の定義式を$\vb*{v}$を表す式として整理すると、
  \begin{equation*}
    \vb*{v} = \vb*{u} + \sum_{i=1}^k (\vb*{v}, \vb*{w}_i) \vb*{w}_i
  \end{equation*}
  となるが、$\vb*{w}_i$が$W$の正規直交基底であることから、
  \begin{equation*}
    \sum_{i=1}^k (\vb*{v}, \vb*{w}_i) \vb*{w}_i
  \end{equation*}
  の部分は、$W$の任意の元を表す。

  \br

  よって、$V$の任意の元$\vb*{v}$は、$W$の元と$W^\perp$の元$\vb*{u}$の和として表されるので、
  \begin{equation*}
    V = W + W^\perp
  \end{equation*}
  が成り立つ。 $\qed$
\end{proof}

\br

さらに、次の定理が成り立つことで、単なる空間の和ではなく、\hyperref[def:direct-sum]{直和}として分解できることがわかる。

\begin{theorem*}{直交補空間との交わり}
  計量空間$V$の部分空間$W$に対して、
  \begin{equation*}
    W \cap W^\perp = \{ \vb*{o} \}
  \end{equation*}
\end{theorem*}

\begin{proof}
  $\vb*{a} \in W \cap W^\perp$とすると、$\vb*{a} \in W$かつ$\vb*{a} \in W^\perp$である。

  $\vb*{a} \in W^\perp$より、$\vb*{a} \in W$に対しても内積が0になるので、
  \begin{equation*}
    (\vb*{a}, \vb*{a}) = 0
  \end{equation*}

  ここで、内積の性質より、
  \begin{equation*}
    (\vb*{a}, \vb*{a}) = \| \vb*{a} \|^2 \geq 0
  \end{equation*}
  であり、等号が成立するのは、$\vb*{a} = \vb*{o}$のときのみである。

  よって、$\vb*{a} = \vb*{o}$である。

  \br

  零ベクトルは任意のベクトルと直交し(内積が0になり)、また任意の部分空間に属するので、明らかに$\vb*{o} \in W \cap W^\perp$である。

  \br

  $\vb*{a}$は$W \cap W^\perp$の任意の元であり、$\vb*{a} = \vb*{o} \in W \cap W^\perp$であることがわかったので、
  \begin{equation*}
    W \cap W^\perp = \{ \vb*{o} \}
  \end{equation*}
  がいえる。 $\qed$
\end{proof}

\br

こうして、次の両方が成り立つことから、
\begin{enumerate}[label=\romanlabel]
  \item $V = W + W^\perp$
  \item $W \cap W^\perp = \{ \vb*{o} \}$
\end{enumerate}
\thmref{thm:direct-sum-equiv}より、計量空間$V$は部分空間$W$とその直交補空間$W^\perp$の直和として分解できる。

\br

よって、\thmref{thm:dim-direct-sum}より、次の定理が従う。

\begin{theorem*}{直交補空間と次元}
  計量空間$V$の部分空間$W$に対して、
  \begin{equation*}
    \dim V = \dim W + \dim W^\perp
  \end{equation*}
\end{theorem*}

\sectionline
\section{直交補空間の性質}
\marginnote{\refbookC p139}

\subsection{直交補空間の直交補空間はもとの空間}

\begin{theorem}{部分空間の双直交補と元空間の一致}{double-orthogonal-complement}
  計量空間$V$の部分空間$W$に対して、次が成り立つ。
  \begin{equation*}
    (W^\perp)^\perp = W
  \end{equation*}
\end{theorem}

\begin{proof}
  \todo{}
\end{proof}

\end{document}
