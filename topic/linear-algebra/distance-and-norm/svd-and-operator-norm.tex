\documentclass[../../../topic_linear-algebra]{subfiles}

\begin{document}

\sectionline
\section{最大特異値と作用素ノルム}
\marginnote{\refbookA p225}

$A$の最大特異値$\sigma_1$は、次の解釈をもつ。

\begin{theorem*}{最大特異値と作用素ノルムの一致}
  複素行列$A$に対して、
  \begin{equation*}
    \max_{\| \vb*{v} \| = 1}{\| A \vb*{v} \|} = \sigma_1
  \end{equation*}
\end{theorem*}

\begin{proof}
  \hyperref[thm:norm-expansion-svd]{特異値分解に基づくノルムの展開表示}と、$\sigma_i \leq \sigma_1$を用いて、
  \begin{equation*}
    \| A \vb*{v} \|^2 = \sum_{i=1}^r \sigma_i^2 |c_i|^2 \leq \sum_{i=1}^r \sigma_1^2 |c_i|^2 = \sigma_1^2 \sum_{i=1}^r |c_i|^2
  \end{equation*}
  
  ここで、$\| \vb*{v} \| = 1$ならば、$\vb*{v} \in \mathbb{C}^n$の$\mathcal{V}$に関する座標ベクトル$\vb*{c}$も$\| \vb*{c} \| = 1$となるので、
  \begin{equation*}
    \| A \vb*{v} \|^2 \leq \sigma_1^2 \sum_{i=1}^r |c_i|^2 = \sigma_1^2 \| \vb*{c} \|^2 = \sigma_1^2
  \end{equation*}
  
  よって、$\| A \vb*{v} \| \leq \sigma_1$が任意の単位ベクトル$\vb*{v}$に対して成り立つ。
  
  すなわち、
  \begin{equation*}
    \max_{\| \vb*{v} \| = 1} \| A \vb*{v} \| \leq \sigma_1
  \end{equation*}
  
  等号は、$c_1 = 1$で他の成分$c_2, \dots, c_r$が0のときに成り立つ。 $\qed$
\end{proof}

\br

\begin{definition}{作用素ノルム}
  複素行列$A$に対して、$A$の\keyword{作用素ノルム}を次のように定義する。
  \begin{equation*}
    \| A \| \coloneq \max_{\| \vb*{v} \| = 1} \| A \vb*{v} \|
  \end{equation*}
\end{definition}

\end{document}
