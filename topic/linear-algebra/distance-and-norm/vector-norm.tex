\documentclass[../../../topic_linear-algebra]{subfiles}

\usepackage{xr-hyper}
\externaldocument{../../../.tex_intermediates/topic_linear-algebra}

\begin{document}

\sectionline
\section{ノルムと距離の関係}
\marginnote{\refbookM p106 \\ \refbookO p24〜25}

\defref{def:norm-axioms}で定めたように、ノルムは「長さ」を拡張した概念であり、ベクトルに対して定義された。

\br

今、2つのベクトル$\vb*{a},\vb*{b}$を
\begin{equation*}
  \vb*{a} = \begin{pmatrix} a_1 \\ a_2 \end{pmatrix}, \quad
  \vb*{b} = \begin{pmatrix} b_1 \\ b_2 \end{pmatrix}
\end{equation*}
とおくと、先ほど示したユークリッド距離の式
\begin{equation*}
  d(A,B) = \sqrt{(b_1 - a_1)^2 + (b_2 - a_2)^2}
\end{equation*}
は内積を経由して次のように書き換えられる。
\begin{equation*}
  d(A,B) = \sqrt{(\vb*{b}-\vb*{a}, \vb*{b}-\vb*{a})} = \|\vb*{b}-\vb*{a}\|
\end{equation*}

\br

このように、距離からノルムを定めることができ、逆にノルムから距離を定めることもできる。

\br

上述の式は2次元平面上の点の間の距離を表しているが、ベクトルのノルムとしてユークリッド距離を考えれば、$n$次元空間への拡張も容易になる。
\begin{equation*}
  \vb*{a} = \begin{pmatrix} a_1 \\ \vdots \\ a_n \end{pmatrix}, \quad
  \vb*{b} = \begin{pmatrix} b_1 \\ \vdots \\ b_n \end{pmatrix}
\end{equation*}
とおいたとしても、
\begin{equation*}
  d(A,B) = \sqrt{\sum_{i=1}^n (b_i - a_i)^2} = \sqrt{(\vb*{b}-\vb*{a}, \vb*{b}-\vb*{a})} = \|\vb*{b}-\vb*{a}\|
\end{equation*}
が成り立つ。

\end{document}
