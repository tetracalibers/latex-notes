\documentclass[../../../topic_linear-algebra]{subfiles}

\usepackage{xr-hyper}
\externaldocument{../../../.tex_intermediates/topic_linear-algebra}

\begin{document}

\sectionline
\section{$L_p$ノルム}
\marginnote{\refbookM p106〜107 \\ \refbookN p3〜4}

ユークリッド距離とマンハッタン距離は、次の\keyword{$L_p$ノルム}によって一般化される。

\begin{definition*}{$L_p$ノルム}
  ベクトル$\vb*{a} = (a_i)_{i=1}^n$に対し、
  \begin{equation*}
    \|\vb*{a}\|_p = \left( \sum_{i=1}^n |a_i|^p \right)^{\frac{1}{p}}
  \end{equation*}
  は\defref{def:norm-axioms}を満たし、これを\keyword{$L_p$ノルム}という。
\end{definition*}

$L_p$ノルムが距離として意味を持つのは、$p \geq 1$のときである。

\br

$p=1$の場合、
\begin{equation*}
  \|\vb*{a}\|_1 = \sum_{i=1}^n |a_i|
\end{equation*}
となり、$\vb*{a}$を$\vb*{b}-\vb*{a}$に置き換えれば、\keyword{マンハッタン距離}の式となる。

\br

$p=2$の場合、
\begin{equation*}
  \|\vb*{a}\|_2 = \left( \sum_{i=1}^n |a_i|^2 \right)^{\frac{1}{2}} = \sqrt{\sum_{i=1}^n a_i^2}
\end{equation*}
となり、$\vb*{a}$を$\vb*{b}-\vb*{a}$に置き換えれば、\keyword{ユークリッド距離}の式となる。

\end{document}
