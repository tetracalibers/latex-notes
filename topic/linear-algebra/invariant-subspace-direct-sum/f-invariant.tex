\documentclass[../../../topic_linear-algebra]{subfiles}

\usepackage{xr-hyper}
\externaldocument{../../../.tex_intermediates/topic_linear-algebra}

\begin{document}

\sectionline
\section{不変部分空間}
\marginnote{\refbookA p114 \\ \refbookF p238〜239}

$V$上の線形変換$f$について、「変換$f$で写しても変わらない」という性質を考える

\begin{definition}{$f$不変}
  $f$を線形空間$V$の線形変換とする

  $V$の部分空間$W$に対して、
  \begin{equation*}
    f(W) \subset W
  \end{equation*}
  すなわち、
  \begin{equation*}
    \forall\vb*{w} \in W \Longrightarrow f(\vb*{w}) \in W
  \end{equation*}
  が成り立つとき、$W$は\keyword{$f$不変}な部分空間であるという

  \br

  また、$V=\mathbb{R}^n$で、$f$が正方行列$A$によって定まっているときは、$f$不変な部分空間$W$を\keyword{$A$不変}な部分空間ともいう
\end{definition}

\sectionline
\section{写像の制限と不変部分空間}
\marginnote{\refbookA p114 \\ \refbookF p240〜242、p363〜364}

\begin{theorem}{不変部分空間による線形変換のブロック型行列表現}{block-matrix-by-invariant-subspace}
  $V$を$n$次元線形空間とし、線形変換$f\colon V \to V$を考える

  このとき、$V$のある部分空間$W$が$f$不変ならば、$V$の適当な基底について、$f$は
  \begin{equation*}
    \begin{pmatrix}
      * & * \\
      O & *
    \end{pmatrix} \text{または}
    \begin{pmatrix}
      * & O \\
      * & *
    \end{pmatrix}
  \end{equation*}
  という形の行列で表すことができる
\end{theorem}

\begin{proof}
  $\dim(W) = r$とし、$W$の基底$\vb*{v}_1, \ldots, \vb*{v}_r$を延長して$V$の基底$\vb*{v}_1, \ldots, \vb*{v}_r, \vb*{v}_{r+1}, \ldots, \vb*{v}_n$をとる

  このとき、\hyperref[sec:construction-of-matrix-rep]{表現行列の構成法}より、
  \begin{equation*}
    f(\vb*{v}_j) = \sum_{i=1}^r a_{ij} \vb*{v}_i + \sum_{i=r+1}^n a_{ij} \vb*{v}_i \quad (1 \leq j \leq n)
  \end{equation*}
  とおける

  \br

  ここで、$W$は$f$不変であることは、$1 \leq j \leq r$の範囲では$f(\vb*{v}_j) \in W$であることを意味する

  $W$の元$f(\vb*{v}_j)$は、$W$の基底だけを用いて表現できるので、
  \begin{equation*}
    f(\vb*{v}_j) = \sum_{i=1}^r a_{ij} \vb*{v}_i \quad (1 \leq j \leq r)
  \end{equation*}
  すなわち、もともとの$f(\vb*{v}_j)$の式において、
  \begin{equation*}
    \sum_{i=r+1}^n a_{ij} \vb*{v}_i = \vb*{0} \quad (1 \leq j \leq r)
  \end{equation*}
  となっている

  $\vb*{v}_i$は基底なので線型独立であり、したがって、
  \begin{equation*}
    a_{ij} = 0 \quad (1 \leq j \leq r, \, r+1 \leq i \leq n)
  \end{equation*}
  が成り立つ

  \br

  この条件より、$f$の表現行列$(a_{ij})$は、
  \begin{equation*}
    \begin{pNiceArray}{cc|cc}[xdots={horizontal-labels,line-style = <->},first-row,last-col,margin,columns-width =1em]
      \Hdotsfor{2}^{r} & \Hdotsfor{2}^{n-r} \\
      \Block{2-2}<\large>{*} & & \Block{2-2}<\large>{*} && \Vdotsfor{2}^{r}  \\
      &&& \\
      \hline
      \Block{2-2}<\large>{O}& & \Block{2-2}<\large>{*} && \Vdotsfor{2}^{n-r} \\
      &&&
    \end{pNiceArray}
  \end{equation*}
  というような形になる

  \br

  また、$V$の基底として、順序を変えた$\vb*{v}_{r+1}, \ldots, \vb*{v}_n, \vb*{v}_1, \ldots, \vb*{v}_r$を取ることもできる

  この場合は、
  \begin{equation*}
    f(\vb*{v}_j) = \sum_{i=1}^r a_{ij} \vb*{v}_i + \sum_{i=r+1}^n a_{ij} \vb*{v}_i \quad (r+1 \leq j \leq n)
  \end{equation*}
  とおくと、$r+1 \leq j \leq n$の範囲($V$の基底の後半部分)で
  \begin{equation*}
    \sum_{i=1}^r a_{ij} \vb*{v}_i = \vb*{0}
  \end{equation*}
  となるので、すなわち、
  \begin{equation*}
    a_{ij} = 0 \quad (r+1 \leq j \leq n, \, 1 \leq i \leq r)
  \end{equation*}
  よって、$f$の表現行列$(a_{ij})$は、
  \begin{equation*}
    \begin{pNiceArray}{cc|cc}[xdots={horizontal-labels,line-style = <->},first-row,last-col,margin,columns-width =1em]
      \Hdotsfor{2}^{n-r} & \Hdotsfor{2}^{r} \\
      \Block{2-2}<\large>{*} & & \Block{2-2}<\large>{O} && \Vdotsfor{2}^{n-r}  \\
      &&& \\
      \hline
      \Block{2-2}<\large>{*}& & \Block{2-2}<\large>{*} && \Vdotsfor{2}^{r} \\
      &&&
    \end{pNiceArray}
  \end{equation*}
  という形になる

  \br

  以上より、2通りの$f$の表現行列の形が得られた $\qed$
\end{proof}

$V$の基底を$\vb*{v}_1, \ldots, \vb*{v}_r, \vb*{v}_{r+1}, \ldots, \vb*{v}_n$ととった場合、$f(\vb*{v}_j) \in W$は
\begin{equation*}
  f(\vb*{v}_j) = \sum_{i=1}^r a_{ij} \vb*{v}_i \quad (1 \leq j \leq r)
\end{equation*}
だけで表現できた

この$1 \leq i \leq r,\, 1 \leq j \leq r$の部分は、$f$の表現行列
\begin{equation*}
  A = (a_{ij}) = \begin{pNiceArray}{cc|cc}[xdots={horizontal-labels,line-style = <->},first-row,last-col,margin,columns-width =1em]
    \Hdotsfor{2}^{r} & \Hdotsfor{2}^{n-r} \\
    \Block{2-2}<\large>{A_{11}} & & \Block{2-2}<\large>{A_{12}} && \Vdotsfor{2}^{r}  \\
    &&& \\
    \hline
    \Block{2-2}<\large>{O}& & \Block{2-2}<\large>{A_{22}} && \Vdotsfor{2}^{n-r} \\
    &&&
  \end{pNiceArray}
\end{equation*}
の、$A_{11}$の部分に対応する

\br

つまり、この行列$A_{11}$は、$\vb*{v}_1, \ldots, \vb*{v}_r$で張られる$V$の部分空間$W$から$W$への線形写像$f'$を、基底$\vb*{v}_1, \ldots, \vb*{v}_r$について表現する行列になっている

\br

$f'$は、$f$の定義域を$W$に制限したものになっているが、$W$の元に限定して考える限り、実質的には$f$と区別がないものである

この意味で、写像$f'$を、写像$f$の$W$への\keyword{制限}と呼び、$f|_W$と表記する

\begin{definition}{写像の制限}
  写像$f\colon X \to Y$において、$X$のある部分集合$S$が与えられたとき、定義域を$S$に限定したものを$f$の$S$に対する\keyword{制限}といい、
  \begin{equation*}
    f|_S\colon S \to Y
  \end{equation*}
  と表す
\end{definition}

同様に、$V$の基底を$\vb*{v}_{r+1}, \ldots, \vb*{v}_n, \vb*{v}_1, \ldots, \vb*{v}_r$ととった場合、$f(\vb*{v}_j) \in W$は
\begin{equation*}
  \sum_{i=r+1}^n a_{ij} \vb*{v}_i \quad (r+1 \leq j \leq n)
\end{equation*}
だけで表現できた

この$r+1 \leq i \leq n,\, r+1 \leq j \leq n$の部分は、$f$の表現行列
\begin{equation*}
  A = (a_{ij}) = \begin{pNiceArray}{cc|cc}[xdots={horizontal-labels,line-style = <->},first-row,last-col,margin,columns-width =1em]
    \Hdotsfor{2}^{n-r} & \Hdotsfor{2}^{r} \\
    \Block{2-2}<\large>{A_{11}} & & \Block{2-2}<\large>{O} && \Vdotsfor{2}^{n-r}  \\
    &&& \\
    \hline
    \Block{2-2}<\large>{A_{21}}& & \Block{2-2}<\large>{A_{22}} && \Vdotsfor{2}^{r} \\
    &&&
  \end{pNiceArray}
\end{equation*}
の、$A_{22}$の部分に対応する

つまり、この場合は、$A_{22}$が変換$f$の$W$への制限$f|_W$を表現する行列になっている

\sectionline
\section{不変部分空間への直和分解}
\marginnote{\refbookA p114 \\ \refbookF p242〜245}

\thmref{thm:block-matrix-by-invariant-subspace}の証明では、$W$の基底$\vb*{v}_1, \ldots, \vb*{v}_r$を延長したものを$V$の基底$\vb*{v}_1, \ldots, \vb*{v}_r, \vb*{v}_{r+1}, \ldots, \vb*{v}_n$とした

このとき、
\begin{equation*}
  W' = \langle \vb*{v}_{r+1}, \ldots, \vb*{v}_n \rangle
\end{equation*}
とおくと、$V$の基底が$W, \, W'$の基底を合わせたものになっているため、\thmref{thm:direct-sum-and-basis}より、
\begin{equation*}
  V = W \oplus W'
\end{equation*}
となる

\br

ここで、もし$W'$も$f$不変であれば、右上の$A_{12}$も零行列になって、表現行列は
\begin{equation*}
  A = (a_{ij}) = \begin{pNiceArray}{cc|cc}[xdots={horizontal-labels,line-style = <->},first-row,last-col,margin,columns-width =1em]
    \Hdotsfor{2}^{r} & \Hdotsfor{2}^{n-r} \\
    \Block{2-2}<\large>{A_{11}} & & \Block{2-2}<\large>{O} && \Vdotsfor{2}^{r}  \\
    &&& \\
    \hline
    \Block{2-2}<\large>{O}& & \Block{2-2}<\large>{A_{22}} && \Vdotsfor{2}^{n-r} \\
    &&&
  \end{pNiceArray}
\end{equation*}
という\keyword{ブロック対角型}になる

\begin{theorem*}{不変部分空間への直和分解}
  線形空間$V$と、$V$上の線形変換$f$に対し、$V$が$f$不変な部分空間$W_1$と$W_2$の直和に分解することができれば、すなわち、
  \begin{enumerate}[label=\romanlabel]
    \item $V = W_1 \oplus W_2$
    \item $W_1, W_2$は$f$不変な$V$の部分空間
  \end{enumerate}
  となる$W_1,\,W_2$が存在すれば、適当な$V$の基底について、$f$は次のような形の行列で表せる

  \begin{equation*}
    \begin{pNiceArray}{ccccc}[xdots={horizontal-labels,line-style = <->},first-row,last-col,margin,columns-width =1em]
      \Hdotsfor{3}^{\dim(W_1)} & \Hdotsfor{2}^{\dim(W_2)} \\
      \Block[borders={bottom,top,right,left,tikz=dashed}]{3-3}<\large>{*} & & & \Block{3-2}<\large>{O} && \Vdotsfor{3}^{\dim(W_1)}  \\
      &&&& \\
      &&&& \\
      \Block{2-3}<\large>{O} && & \Block[borders={bottom,top,right,left,tikz=dashed}]{2-2}<\large>{*} && \Vdotsfor{2}^{\dim(W_2)} \\
      &&&&
    \end{pNiceArray}
  \end{equation*}
  または
  \begin{equation*}
    \begin{pNiceArray}{ccccc}[xdots={horizontal-labels,line-style = <->},first-row,last-col,margin,columns-width =1em]
      \Hdotsfor{2}^{\dim(W_2)} & \Hdotsfor{3}^{\dim(W_1)} \\
      \Block[borders={bottom,top,right,left,tikz=dashed}]{2-2}<\large>{*} & & \Block{2-3}<\large>{O} &&& \Vdotsfor{2}^{\dim(W_2)}  \\
      &&&& \\
      \Block{3-2}<\large>{O} && \Block[borders={bottom,top,right,left,tikz=dashed}]{3-3}<\large>{*} &&& \Vdotsfor{3}^{\dim(W_1)} \\
      &&&& \\
      &&&&
    \end{pNiceArray}
  \end{equation*}
\end{theorem*}

\begin{proof}
  $W_1$の基底、$W_2$の基底をこの順に並べるか、その反対の順に並べて、$V$の基底を構成することで、\thmref{thm:block-matrix-by-invariant-subspace}の証明と同様に示される $\qed$
\end{proof}

\sectionline

さらに、$V$をより細かい部分空間の直和に分解できる場合には、次のようになる

\begin{theorem*}{複数の不変部分空間への直和分解}
  線形空間$V$と、$V$上の線形変換$f$について、
  \begin{enumerate}[label=\romanlabel]
    \item $V = W_1 \oplus W_2 \oplus \cdots \oplus W_k$
    \item 各部分空間$W_i$は$f$不変な$V$の部分空間
  \end{enumerate}
  であるならば、適当な$V$の基底に対し、$f$は次のような形の行列で表せる
  \begin{equation*}
    \begin{pNiceArray}{cccc}[margin,columns-width=1em]
      \Block[borders={bottom,top,right,left,tikz=dashed}]{1-1}<\large>{*} & & & O \\
      & \Block[borders={bottom,top,right,left,tikz=dashed}]{1-1}<\large>{*} & & \\
      & & \ddots & \\
      O & & & \Block[borders={bottom,top,right,left,tikz=dashed}]{1-1}<\large>{*} \\
    \end{pNiceArray}
  \end{equation*}
\end{theorem*}

対角線上の各正方形の大きさは、各部分空間$W_i$の次元に対応する

\br

そこで、以上の議論を究極にまで押し進めると、次の定理になる

\begin{theorem}{一次元部分空間への直和分解}{diagonalizable-by-1d-invariant-sum}
  $n$次元部分空間$V$が、$n$個の$f$不変の1次元部分空間の直和に分解できるとき、すなわち、
  \begin{enumerate}[label=\romanlabel]
    \item $V = W_1 \oplus W_2 \oplus \cdots \oplus W_n$
    \item $W_i \, (i=1,2,\ldots,n)$は$f$不変な1次元部分空間
  \end{enumerate}
  となるときは、$f$は次のような\keyword{対角行列}で表せる
  \begin{equation*}
    \begin{pNiceArray}{cccc}[margin]
      * & & & O \\
      & * & & \\
      & & \ddots & \\
      O & & & * \\
    \end{pNiceArray}
  \end{equation*}
\end{theorem}

\sectionline
\section{一次元不変部分空間}\label{sec:1d-invariant-subspaces}
\marginnote{\refbookF p246〜247}

$W$を一次元部分空間とすると、これは基底$\vb*{w} \neq \vb*{0}$で張られる空間であるので、
\begin{equation*}
  W = \langle \vb*{w} \rangle = \{ \alpha \vb*{w} \mid \alpha \in K \}
\end{equation*}

この一次元部分空間$W$が$f$不変であるとは、定義より、
\begin{equation*}
  \forall \vb*{w} \in W \Longrightarrow f(\vb*{w}) \in W
\end{equation*}
であり、これで$W$の元は$\alpha \vb*{w}$とも$f(\vb*{w})$とも表せることになるので、
\begin{equation*}
  f(\vb*{w}) = \alpha \vb*{w} \quad (\alpha \in K)
\end{equation*}
がいえる

\br

以上をふまえて、\thmref{thm:diagonalizable-by-1d-invariant-sum}という定理をより具体的に整理してみる

\br

$\dim(V) = n$とすると、$V$上の線形変換$f$の表現行列$A$を\hyperref[sec:construction-of-matrix-rep]{構成}する式は、
\begin{equation*}
  (f(\vb*{w}_1), \ldots, f(\vb*{w}_n)) = (\vb*{w}_1, \ldots, \vb*{w}_n) A
\end{equation*}
となるが、ここで、
\begin{equation*}
  (f(\vb*{w}_1), \ldots, f(\vb*{w}_n)) = (\alpha_1 \vb*{w}_1, \ldots, \alpha_n \vb*{w}_n)
\end{equation*}
であるので、
\begin{equation*}
  (f(\vb*{w}_1), \ldots, f(\vb*{w}_n)) = (\vb*{w}_1, \ldots, \vb*{w}_n) \begin{pmatrix}
    \alpha_1 & 0        & \cdots & 0        \\
    0        & \alpha_2 & \cdots & 0        \\
    \vdots   & \vdots   & \ddots & \vdots   \\
    0        & 0        & \cdots & \alpha_n
  \end{pmatrix}
\end{equation*}
と書き換えられる

\br

したがって、$f$は基底$\vb*{w}_1, \ldots, \vb*{w}_n$について、次の\keyword{対角行列}で表される
\begin{equation*}
  \begin{pmatrix}
    \alpha_1 & 0        & \cdots & 0        \\
    0        & \alpha_2 & \cdots & 0        \\
    \vdots   & \vdots   & \ddots & \vdots   \\
    0        & 0        & \cdots & \alpha_n
  \end{pmatrix}
\end{equation*}

ここで現れたスカラー$\alpha_i$やベクトル$\vb*{w}_i$と、線形写像$f$との関係が、\keyword{固有値}・\keyword{固有ベクトル}と行列の\keyword{対角化}という話題に発展する

\end{document}
