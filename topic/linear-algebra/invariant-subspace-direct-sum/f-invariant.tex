\documentclass[../../../topic_linear-algebra]{subfiles}

\begin{document}

\sectionline
\section{不変部分空間}
\marginnote{\refbookA p114 \\ \refbookF p238〜239}

$V$上の線形変換$f$について、「変換$f$で写しても変わらない」という性質を考える

\begin{definition}{$f$不変}
  $f$を線形空間$V$の線形変換とする

  $V$の部分空間$W$に対して、
  \begin{equation*}
    f(W) \subset W
  \end{equation*}
  すなわち、
  \begin{equation*}
    \forall\vb*{w} \in W \Longrightarrow f(\vb*{w}) \in W
  \end{equation*}
  が成り立つとき、$W$は\keyword{$f$不変}な部分空間であるという

  \br

  また、$V=\mathbb{R}^n$で、$f$が正方行列$A$によって定まっているときは、$f$不変な部分空間$W$を\keyword{$A$不変}な部分空間ともいう
\end{definition}

\sectionline
\section{写像の制限と不変部分空間}
\marginnote{\refbookA p114 \\ \refbookF p240〜242、p363〜364}

\begin{definition}{写像の制限}
  写像$f\colon X \to Y$において、$X$のある部分集合$S$が与えられたとき、定義域を$S$に限定したものを$f$の$S$に対する\keyword{制限}といい、
  \begin{equation*}
    f|_S\colon S \to Y
  \end{equation*}
  と表す
\end{definition}

\sectionline

\begin{theorem}{不変部分空間による線形変換のブロック構造表現}
  $V$を$n$次元線形空間とし、線形変換$f\colon V \to V$を考える

  このとき、$V$のある部分空間$W$が$f$不変ならば、$V$の適当な基底について、$f$は
  \begin{equation*}
    \begin{pmatrix}
      * & * \\
      O & *
    \end{pmatrix} \text{または}
    \begin{pmatrix}
      * & O \\
      * & *
    \end{pmatrix}
  \end{equation*}
  という形の行列で表すことができる
\end{theorem}

\begin{proof}
  $\dim(W) = r$とし、$W$の基底$\vb*{v}_1, \ldots, \vb*{v}_r$を\hyperref[thm:basis-extension]{延長}して$V$の基底$\vb*{v}_1, \ldots, \vb*{v}_r, \vb*{v}_{r+1}, \ldots, \vb*{v}_n$をとる

  このとき、\hyperref[sec:construction-of-matrix-rep]{表現行列の構成法}より、
  \begin{equation*}
    f(\vb*{v}_j) = \sum_{i=1}^r a_{ij} \vb*{v}_i + \sum_{i=r+1}^n a_{ij} \vb*{v}_i \quad (1 \leq j \leq n)
  \end{equation*}
  とおける

  \br

  ここで、$W$は$f$不変であることは、$1 \leq j \leq r$の範囲では$f(\vb*{v}_j) \in W$であることを意味する

  $W$の元$f(\vb*{v}_j)$は、$W$の基底だけを用いて表現できるので、
  \begin{equation*}
    f(\vb*{v}_j) = \sum_{i=1}^r a_{ij} \vb*{v}_i \quad (1 \leq j \leq r)
  \end{equation*}
  すなわち、もともとの$f(\vb*{v}_j)$の式において、
  \begin{equation*}
    \sum_{i=r+1}^n a_{ij} \vb*{v}_i = \vb*{0} \quad (1 \leq j \leq r)
  \end{equation*}
  となっている

  $\vb*{v}_i$は基底なので線型独立であり、したがって、
  \begin{equation*}
    a_{ij} = 0 \quad (1 \leq j \leq r, \, r+1 \leq i \leq n)
  \end{equation*}
  が成り立つ

  \br

  この条件より、$f$の表現行列$(a_{ij})$は、
  \begin{equation*}
    \begin{pNiceArray}{cc|cc}[xdots={horizontal-labels,line-style = <->},first-row,last-col,margin,columns-width =1em]
      \Hdotsfor{2}^{r} & \Hdotsfor{2}^{n-r} \\
      \Block{2-2}<\large>{*} & & \Block{2-2}<\large>{*} && \Vdotsfor{2}^{r}  \\
      &&& \\
      \hline
      \Block{2-2}<\large>{O}& & \Block{2-2}<\large>{*} && \Vdotsfor{2}^{n-r} \\
      &&&
    \end{pNiceArray}
  \end{equation*}
  というような形になる

  \br

  また、$V$の基底として、順序を変えた$\vb*{v}_{r+1}, \ldots, \vb*{v}_n, \vb*{v}_1, \ldots, \vb*{v}_r$を取ることもできる

  この場合は、
  \begin{equation*}
    f(\vb*{v}_j) = \sum_{i=1}^r a_{ij} \vb*{v}_i + \sum_{i=r+1}^n a_{ij} \vb*{v}_i \quad (r+1 \leq j \leq n)
  \end{equation*}
  とおくと、$r+1 \leq j \leq n$の範囲($V$の基底の後半部分)で
  \begin{equation*}
    \sum_{i=1}^r a_{ij} \vb*{v}_i = \vb*{0}
  \end{equation*}
  となるので、すなわち、
  \begin{equation*}
    a_{ij} = 0 \quad (r+1 \leq j \leq n, \, 1 \leq i \leq r)
  \end{equation*}
  よって、$f$の表現行列$(a_{ij})$は、
  \begin{equation*}
    \begin{pNiceArray}{cc|cc}[xdots={horizontal-labels,line-style = <->},first-row,last-col,margin,columns-width =1em]
      \Hdotsfor{2}^{n-r} & \Hdotsfor{2}^{r} \\
      \Block{2-2}<\large>{*} & & \Block{2-2}<\large>{O} && \Vdotsfor{2}^{n-r}  \\
      &&& \\
      \hline
      \Block{2-2}<\large>{*}& & \Block{2-2}<\large>{*} && \Vdotsfor{2}^{r} \\
      &&&
    \end{pNiceArray}
  \end{equation*}
  という形になる

  \br

  以上より、2通りの$f$の表現行列の形が得られた $\qed$
\end{proof}

\end{document}
