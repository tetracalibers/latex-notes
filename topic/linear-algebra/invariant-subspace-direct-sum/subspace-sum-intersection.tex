\documentclass[../../../topic_linear-algebra]{subfiles}

\begin{document}

\sectionline
\section{部分空間の共通部分}
\marginnote{\refbookC p22}

与えられた部分空間から、新しく部分空間を作ることができる

\begin{theorem}{線形部分空間の共通部分は部分空間}
  $U,\,W$を体$K$上の$V$の部分空間とするとき、\keyword{共通部分}$U \cap W$は$V$の部分空間である
\end{theorem}

\begin{proof}
  \begin{subpattern}{\bfseries 和について}
    $\vb*{a}, \vb*{b} \in U \cap W$とすると、共通部分の定義より、$\vb*{a}$と$\vb*{b}$はどちらも$U$と$W$の両方に属していることになる

    つまり、$\vb*{a}, \vb*{b} \in U$かつ$\vb*{a}, \vb*{b} \in W$である

    \br

    $U$も$W$も部分空間なので、部分空間の定義より、
    \begin{align*}
      \vb*{a} + \vb*{b} & \in U \\
      \vb*{a} + \vb*{b} & \in W
    \end{align*}

    $\vb*{a} + \vb*{b}$が$U$と$W$の両方に属していることから、$\vb*{a} + \vb*{b}$は$U \cap W$に属する

    よって、$U \cap W$は和について閉じている $\qed$
  \end{subpattern}

  \begin{subpattern}{\bfseries スカラー倍について}
    $\vb*{a} \in U \cap W$と$c \in K$をとる

    共通部分の定義より、$\vb*{a}$は$U$と$W$の両方に属しているので、部分空間の定義より
    \begin{align*}
      c \vb*{a} & \in U \\
      c \vb*{a} & \in W
    \end{align*}

    よって、$c \vb*{a}$は$U \cap W$に属するため、$U \cap W$はスカラー倍について閉じている $\qed$
  \end{subpattern}
\end{proof}

\sectionline
\section{部分空間の和}
\marginnote{\refbookC p22〜23 \\ \refbookF p231〜232}

\begin{theorem}{線形部分空間の和は部分空間}
  $U,\,W$を体$K$上の$V$の部分空間とするとき、\keyword{和空間}
  \begin{equation*}
    U + W \coloneq \{ \vb*{u} + \vb*{w} \mid \vb*{u} \in U, \vb*{w} \in W \}
  \end{equation*}
  は$V$の部分空間である
\end{theorem}

\begin{proof}
  \begin{subpattern}{\bfseries 和について}
    $\vb*{a}_1, \vb*{a}_2 \in U, \, \vb*{b}_1, \vb*{b}_2 \in W$とする

    $U$と$W$は部分空間なので、部分空間の定義より
    \begin{equation*}
      \vb*{a}_1 + \vb*{a}_2 \in U, \quad \vb*{b}_1 + \vb*{b}_2 \in W
    \end{equation*}

    一方、和空間の定義より、$\vb*{a}_1 + \vb*{b}_1,\, \vb*{a}_2 + \vb*{b}_2$はそれぞれ$U+W$の元である

    これらの元の和をとったときに、その和も$U + W$に属していれば、和空間は和について閉じているといえる
    \begin{align*}
      (\vb*{a}_1 + \vb*{b}_1) + (\vb*{a}_2 + \vb*{b}_2) & = (\vb*{a}_1 + \vb*{a}_2) + (\vb*{b}_1 + \vb*{b}_2) \\
                                                        & \in U + W
    \end{align*}
    上式で、和空間は和について閉じていることが示された $\qed$
  \end{subpattern}

  \begin{subpattern}{\bfseries スカラー倍について}
    $\vb*{a} \in U, \, \vb*{b} \in W$と$c \in K$をとる

    $U$と$W$は部分空間なので、部分空間の定義より
    \begin{align*}
      c \vb*{a} & \in U \\
      c \vb*{b} & \in W
    \end{align*}

    一方、和空間の定義より、$\vb*{a} + \vb*{b}$は$U + W$の元である

    この元をスカラー倍したときに、そのスカラー倍も$U + W$に属していれば、和空間はスカラー倍について閉じているといえる
    \begin{align*}
      c(\vb*{a} + \vb*{b}) & = c \vb*{a} + c \vb*{b} \\
                           & \in U + W
    \end{align*}
    上式で、和空間はスカラー倍について閉じていることが示された $\qed$
  \end{subpattern}
\end{proof}

\sectionline

部分空間を生成するベクトルを用いて、部分空間の和を表せる

\begin{theorem}{部分空間の和と生成ベクトル}\label{thm:sum-of-subspaces-span}
  $K^n$の2つの部分空間$U = \langle \vb*{u}_1, \dots, \vb*{u}_m \rangle$と
  $W = \langle \vb*{w}_1, \dots, \vb*{w}_k \rangle$に対して、和空間$U + W$は
  \begin{equation*}
    U + W = \langle \vb*{u}_1, \vb*{u}_2, \dots, \vb*{u}_m, \vb*{w}_1, \vb*{w}_2, \dots, \vb*{w}_k \rangle
  \end{equation*}
  となる
\end{theorem}

\begin{proof}
  和空間 $U + W$ は
  \begin{equation*}
    U + W = \{ \vb*{x} \in K^n \mid \vb*{x} = \vb*{u} + \vb*{w},\ \vb*{u} \in U,\ \vb*{w} \in W \}
  \end{equation*}
  と定義される

  また、$\vb*{u}_1, \dots, \vb*{u}_m, \vb*{w}_1, \dots, \vb*{w}_k$ の張る部分空間は
  \begin{equation*}
    H = \langle \vb*{u}_1, \dots, \vb*{u}_m, \vb*{w}_1, \dots, \vb*{w}_k \rangle
  \end{equation*}
  である

  これらが等しいことを示せばよい

  \begin{subpattern}{$U+W \subseteq H$}
    任意の $\vb*{x} \in U + W$ に対し、$\vb*{x} = \vb*{u} + \vb*{w}$($\vb*{u} \in U$, $\vb*{w} \in W$)と書ける

    すなわち、
    \begin{align*}
      \vb*{u} & = a_1 \vb*{u}_1 + \dots + a_m \vb*{u}_m & (a_i \in K) \\
      \vb*{w} & = b_1 \vb*{w}_1 + \dots + b_k \vb*{w}_k & (b_j \in K)
    \end{align*}

    よって、
    \begin{equation*}
      \vb*{x} = \sum_{i=1}^m a_i \vb*{u}_i + \sum_{j=1}^k b_j \vb*{w}_j \in H
    \end{equation*}
  \end{subpattern}

  \begin{subpattern}{$H \subseteq U + W$}
    任意の $\vb*{x} \in H$ は
    \begin{equation*}
      \vb*{x} = \sum_{i=1}^m a_i \vb*{u}_i + \sum_{j=1}^k b_j \vb*{w}_j
    \end{equation*}
    と書ける

    ここで
    \begin{align*}
      \vb*{u} & = \sum_{i=1}^m a_i \vb*{u}_i \in U \\
      \vb*{w} & = \sum_{j=1}^k b_j \vb*{w}_j \in W
    \end{align*}
    とすれば、
    \begin{equation*}
      \vb*{x} = \vb*{u} + \vb*{w} \in U + W
    \end{equation*}
  \end{subpattern}

  \br

  以上より、$U+W \subseteq H$と$H \subseteq U + W$が成り立つので、$U + W = H$が示された $\qed$
\end{proof}

\sectionline
\section{部分空間の和の次元}
\marginnote{\refbookA p103 \\ \refbookC p39〜41}

\begin{theorem}{部分空間の和の次元}
  $K^n$の部分空間$V, \, W$に対して、次が成り立つ
  \begin{equation*}
    \dim(V + W) = \dim V + \dim W - \dim(V \cap W)
  \end{equation*}
\end{theorem}

\begin{proof}
  $\dim(V) = n, \, \dim(W) = m$とする

  \br

  $V \cap W$の基底$\mathcal{V} = \{ \vb*{u}_1,\dots, \vb*{u}_d \}$をとる

  これを\hyperref[thm:basis-extension]{基底の延長の定理}に基づいて、$V$の基底
  \begin{equation*}
    \mathcal{V} \cup \{ \vb*{v}_1, \dots, \vb*{v}_{n-d} \}
  \end{equation*}
  に延長する

  同様に、$\mathcal{V}$を$W$の基底
  \begin{equation*}
    \mathcal{V} \cup \{ \vb*{w}_1, \dots, \vb*{w}_{m-d} \}
  \end{equation*}
  に延長する

  \br

  このとき、$\vb*{u_1}, \dots, \vb*{u}_d, \vb*{v}_1, \dots, \vb*{v}_{n-d}, \vb*{w}_1, \dots, \vb*{w}_{m-d}$が$V + W$の基底になることを示す

  \begin{subpattern}{\bfseries $V+W$を生成すること}
    $\vb*{v} \in V, \, \vb*{w} \in W$とすると、それぞれ基底の線形結合で表すことができる
    \begin{align*}
      \vb*{v} & = \sum_{i=1}^d a_i \vb*{u}_i + \sum_{j=1}^{n-d} b_j \vb*{v}_j \\
      \vb*{w} & = \sum_{i=1}^d c_i \vb*{u}_i + \sum_{k=1}^{m-d} d_k \vb*{w}_k
    \end{align*}

    \br

    $V+W$の任意の元は、$\vb*{v} + \vb*{w}$と書けるので、
    \begin{equation*}
      \vb*{v} + \vb*{w} = \sum_{i=1}^d (a_i + c_i) \vb*{u}_i + \sum_{j=1}^{n-d} b_j \vb*{v}_j + \sum_{k=1}^{m-d} d_k \vb*{w}_k
    \end{equation*}
    となり、$\{ \vb*{u}_1, \dots, \vb*{u}_d, \vb*{v}_1, \dots, \vb*{v}_{n-d}, \vb*{w}_1, \dots, \vb*{w}_{m-d} \}$の線形結合で表せる $\qed$
  \end{subpattern}

  \begin{subpattern}{\bfseries 線型独立であること}
    $\vb*{u}_1, \dots, \vb*{u}_d, \vb*{v}_1, \dots, \vb*{v}_{n-d}, \vb*{w}_1, \dots, \vb*{w}_{m-d}$が線型独立であることを示すために、次のような線形関係式を考える
    \begin{equation*}
      \sum_{i=1}^d c_i \vb*{u}_i + \sum_{j=1}^{n-d} c_{d+j} \vb*{v}_j + \sum_{k=1}^{m-d} c_{d+n-d+k} \vb*{w}_k = \vb*{0}
    \end{equation*}

    ここで、$c_i \in K$はスカラーである

    \br

    この式を$V$と$W$の基底の線型結合として考えると、$V$の基底$\vb*{u}_i, \, \vb*{v}_j$に関する部分と$W$の基底$\vb*{u}_i, \, \vb*{w}_k$に関する部分がそれぞれ線形独立であるため、結局どの項においても$c_i = 0$である必要がある

    \br

    よって、$\vb*{u}_1, \dots, \vb*{u}_d, \vb*{v}_1, \dots, \vb*{v}_{n-d}, \vb*{w}_1, \dots, \vb*{w}_{m-d}$は線型独立である $\qed$
  \end{subpattern}

  \br

  以上より、$\vb*{u}_1, \dots, \vb*{u}_d, \vb*{v}_1, \dots, \vb*{v}_{n-d}, \vb*{w}_1, \dots, \vb*{w}_{m-d}$は$V + W$の基底であることが示された

  \br

  この基底をなすベクトルの個数(次元)について考えると、
  \begin{align*}
    \dim(V + W) & = d + (n - d) + (m - d) \\
                & = n + m - d
  \end{align*}
  となる

  ここで、$d = \dim(V \cap W)$なので、
  \begin{equation*}
    \dim(V + W) = \dim V + \dim W - \dim(V \cap W)
  \end{equation*}
  と書き換えられ、目的の式が得られた $\qed$
\end{proof}

\end{document}
