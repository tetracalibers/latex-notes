\documentclass[../../../topic_linear-algebra]{subfiles}

\begin{document}

\sectionline
\section{部分空間の和}
\marginnote{\refbookC p22〜23 \\ \refbookF p231〜232 \\ \refbookG p6}

\begin{mindflow}
  \placeholder{再編予定(\refbookS p20〜21)}
\end{mindflow}

\begin{theorem}{線形部分空間の和は部分空間}
  $U,\,W$を体$K$上の$V$の部分空間とするとき、\keyword{和空間}
  \begin{equation*}
    U + W \coloneq \{ \vb*{u} + \vb*{w} \mid \vb*{u} \in U, \vb*{w} \in W \}
  \end{equation*}
  は$V$の部分空間である
\end{theorem}

\begin{proof}
  \begin{subpattern}{\bfseries 和について}
    $\vb*{a}_1, \vb*{a}_2 \in U, \, \vb*{b}_1, \vb*{b}_2 \in W$とする

    $U$と$W$は部分空間なので、部分空間の定義より
    \begin{equation*}
      \vb*{a}_1 + \vb*{a}_2 \in U, \quad \vb*{b}_1 + \vb*{b}_2 \in W
    \end{equation*}

    一方、和空間の定義より、$\vb*{a}_1 + \vb*{b}_1,\, \vb*{a}_2 + \vb*{b}_2$はそれぞれ$U+W$の元である

    これらの元の和をとったときに、その和も$U + W$に属していれば、和空間は和について閉じているといえる
    \begin{align*}
      (\vb*{a}_1 + \vb*{b}_1) + (\vb*{a}_2 + \vb*{b}_2) & = (\vb*{a}_1 + \vb*{a}_2) + (\vb*{b}_1 + \vb*{b}_2) \\
                                                        & \in U + W
    \end{align*}
    上式で、和空間は和について閉じていることが示された $\qed$
  \end{subpattern}

  \begin{subpattern}{\bfseries スカラー倍について}
    $\vb*{a} \in U, \, \vb*{b} \in W$と$c \in K$をとる

    $U$と$W$は部分空間なので、部分空間の定義より
    \begin{align*}
      c \vb*{a} & \in U \\
      c \vb*{b} & \in W
    \end{align*}

    一方、和空間の定義より、$\vb*{a} + \vb*{b}$は$U + W$の元である

    この元をスカラー倍したときに、そのスカラー倍も$U + W$に属していれば、和空間はスカラー倍について閉じているといえる
    \begin{align*}
      c(\vb*{a} + \vb*{b}) & = c \vb*{a} + c \vb*{b} \\
                           & \in U + W
    \end{align*}
    上式で、和空間はスカラー倍について閉じていることが示された $\qed$
  \end{subpattern}
\end{proof}

\br

3つ以上の部分空間の和も同様に考えて、一般に\keyword{和空間}は次のように定義される

\begin{definition}{和空間}
  線形空間$V$と、その部分空間$V_1,\ldots,V_k$が与えられたときに、
  \begin{equation*}
    \vb*{v} = \vb*{v}_1 + \cdots + \vb*{v}_k \quad (\vb*{v}_i \in V_i, i = 1, \ldots, k)
  \end{equation*}
  と表されるベクトル$\vb*{v}$全体がなす集合を$V_1,\ldots,V_k$の\keyword{和空間}といい、
  \begin{equation*}
    \sum_{i=1}^k V_i
  \end{equation*}
  と書く
\end{definition}

\subsection{和空間を張るベクトル}

部分空間を生成するベクトルを用いて、部分空間の和を表せる

\begin{theorem}{部分空間の和と生成ベクトル}\label{thm:sum-of-subspaces-span}
  $K^n$の2つの部分空間$U = \langle \vb*{u}_1, \dots, \vb*{u}_m \rangle$と
  $W = \langle \vb*{w}_1, \dots, \vb*{w}_k \rangle$に対して、和空間$U + W$は
  \begin{equation*}
    U + W = \langle \vb*{u}_1, \vb*{u}_2, \dots, \vb*{u}_m, \vb*{w}_1, \vb*{w}_2, \dots, \vb*{w}_k \rangle
  \end{equation*}
  となる
\end{theorem}

\begin{proof}
  和空間 $U + W$ は
  \begin{equation*}
    U + W = \{ \vb*{x} \in K^n \mid \vb*{x} = \vb*{u} + \vb*{w},\ \vb*{u} \in U,\ \vb*{w} \in W \}
  \end{equation*}
  と定義される

  また、$\vb*{u}_1, \dots, \vb*{u}_m, \vb*{w}_1, \dots, \vb*{w}_k$ の張る部分空間は
  \begin{equation*}
    H = \langle \vb*{u}_1, \dots, \vb*{u}_m, \vb*{w}_1, \dots, \vb*{w}_k \rangle
  \end{equation*}
  である

  これらが等しいことを示せばよい

  \begin{subpattern}{$U+W \subseteq H$}
    任意の $\vb*{x} \in U + W$ に対し、$\vb*{x} = \vb*{u} + \vb*{w}$($\vb*{u} \in U$, $\vb*{w} \in W$)と書ける

    すなわち、
    \begin{align*}
      \vb*{u} & = a_1 \vb*{u}_1 + \dots + a_m \vb*{u}_m & (a_i \in K) \\
      \vb*{w} & = b_1 \vb*{w}_1 + \dots + b_k \vb*{w}_k & (b_j \in K)
    \end{align*}

    よって、
    \begin{equation*}
      \vb*{x} = \sum_{i=1}^m a_i \vb*{u}_i + \sum_{j=1}^k b_j \vb*{w}_j \in H
    \end{equation*}
  \end{subpattern}

  \begin{subpattern}{$H \subseteq U + W$}
    任意の $\vb*{x} \in H$ は
    \begin{equation*}
      \vb*{x} = \sum_{i=1}^m a_i \vb*{u}_i + \sum_{j=1}^k b_j \vb*{w}_j
    \end{equation*}
    と書ける

    ここで
    \begin{align*}
      \vb*{u} & = \sum_{i=1}^m a_i \vb*{u}_i \in U \\
      \vb*{w} & = \sum_{j=1}^k b_j \vb*{w}_j \in W
    \end{align*}
    とすれば、
    \begin{equation*}
      \vb*{x} = \vb*{u} + \vb*{w} \in U + W
    \end{equation*}
  \end{subpattern}

  \br

  以上より、$U+W \subseteq H$と$H \subseteq U + W$が成り立つので、$U + W = H$が示された $\qed$
\end{proof}

\sectionline
\section{和空間の包含関係}
\marginnote{\refbookF p233}

\begin{theorem}{和空間における部分空間の和集合の包含}
  和空間は、和集合を部分集合として包含する

  すなわち、$U,\,W$を$V$の部分空間とするとき、
  \begin{equation*}
    U + W \supset U \cup W
  \end{equation*}
  が成り立つ
\end{theorem}

\begin{proof}
  \hyperref[thm:subspace-contains-zero]{部分空間はいずれも零ベクトルを含む}ので、
  たとえば、$U = \{ \vb*{0} \}$の場合、
  \begin{equation*}
    U + W \supset W
  \end{equation*}
  同様に、
  \begin{equation*}
    U + W \supset U
  \end{equation*}

  よって、$U+W$は$U$または$W$を包含することがわかる

  すなわち、
  \begin{equation*}
    U + W \supset U \cup W
  \end{equation*}
  が成り立つ $\qed$
\end{proof}

\sectionline

\begin{theorem}{和空間の最小包含性}
  $U,\,W$を$V$の部分空間とする

  和空間$U+W$は、$U$と$W$を含む部分空間のうち、最小のものである
\end{theorem}

\begin{proof}
  $V$の任意の部分空間のうち、$U$と$W$の両方を包含するもの$V'$を考える

  このとき、部分空間は和に閉じているため、$V'$は$U+W$も包含する
  \begin{equation*}
    V' \supset U + W
  \end{equation*}

  よって、$V'$の任意性から、$U + W$は$U$と$W$を含む部分空間のうち、最小のものとなる $\qed$
\end{proof}

\sectionline

このように、和空間$U+W$は、$U$や$W$を部分空間として含むが、$U$や$W$より真に大きい($U,\,W$を真部分集合として含む)とは限らない

\br

別の角度からいうと、
\begin{equation*}
  V = W_1 + W_2
\end{equation*}
という関係があるだけで、「$V$が$W_1$と$W_2$の和に分解された」というのは適当ではない

\br

和空間が持つこの欠陥を補うために、和空間の概念をより精密化したものが、次に述べる\keyword{直和}である

\end{document}
