\documentclass[../../../topic_linear-algebra]{subfiles}

\begin{document}

\sectionline
\section{部分空間の共通部分}
\marginnote{\refbookC p22}

\begin{mindflow}
  \placeholder{再編予定(\refbookS p20〜21)}
\end{mindflow}

与えられた部分空間から、新しく部分空間を作ることができる

\begin{theorem}{線形部分空間の共通部分は部分空間}
  $U,\,W$を体$K$上の$V$の部分空間とするとき、\keyword{共通部分}$U \cap W$は$V$の部分空間である
\end{theorem}

\begin{proof}
  \begin{subpattern}{\bfseries 和について}
    $\vb*{a}, \vb*{b} \in U \cap W$とすると、共通部分の定義より、$\vb*{a}$と$\vb*{b}$はどちらも$U$と$W$の両方に属していることになる

    つまり、$\vb*{a}, \vb*{b} \in U$かつ$\vb*{a}, \vb*{b} \in W$である

    \br

    $U$も$W$も部分空間なので、部分空間の定義より、
    \begin{align*}
      \vb*{a} + \vb*{b} & \in U \\
      \vb*{a} + \vb*{b} & \in W
    \end{align*}

    $\vb*{a} + \vb*{b}$が$U$と$W$の両方に属していることから、$\vb*{a} + \vb*{b}$は$U \cap W$に属する

    よって、$U \cap W$は和について閉じている $\qed$
  \end{subpattern}

  \begin{subpattern}{\bfseries スカラー倍について}
    $\vb*{a} \in U \cap W$と$c \in K$をとる

    共通部分の定義より、$\vb*{a}$は$U$と$W$の両方に属しているので、部分空間の定義より
    \begin{align*}
      c \vb*{a} & \in U \\
      c \vb*{a} & \in W
    \end{align*}

    よって、$c \vb*{a}$は$U \cap W$に属するため、$U \cap W$はスカラー倍について閉じている $\qed$
  \end{subpattern}
\end{proof}

\end{document}
