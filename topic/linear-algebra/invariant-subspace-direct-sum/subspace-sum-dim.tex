\documentclass[../../../topic_linear-algebra]{subfiles}

\begin{document}

\sectionline
\section{部分空間の和の次元}
\marginnote{\refbookA p103 \\ \refbookC p39〜41}

\begin{theorem}{部分空間の和の次元}
  $K^n$の部分空間$V, \, W$に対して、次が成り立つ
  \begin{equation*}
    \dim(V + W) = \dim V + \dim W - \dim(V \cap W)
  \end{equation*}
\end{theorem}

\begin{proof}
  $\dim(V) = n, \, \dim(W) = m$とする

  \br

  $V \cap W$の基底$\mathcal{V} = \{ \vb*{u}_1,\dots, \vb*{u}_d \}$をとる

  これを\hyperref[thm:basis-extension]{基底の延長の定理}に基づいて、$V$の基底
  \begin{equation*}
    \mathcal{V} \cup \{ \vb*{v}_1, \dots, \vb*{v}_{n-d} \}
  \end{equation*}
  に延長する

  同様に、$\mathcal{V}$を$W$の基底
  \begin{equation*}
    \mathcal{V} \cup \{ \vb*{w}_1, \dots, \vb*{w}_{m-d} \}
  \end{equation*}
  に延長する

  \br

  このとき、$\vb*{u_1}, \dots, \vb*{u}_d, \vb*{v}_1, \dots, \vb*{v}_{n-d}, \vb*{w}_1, \dots, \vb*{w}_{m-d}$が$V + W$の基底になることを示す

  \begin{subpattern}{\bfseries $V+W$を生成すること}
    $\vb*{v} \in V, \, \vb*{w} \in W$とすると、それぞれ基底の線形結合で表すことができる
    \begin{align*}
      \vb*{v} & = \sum_{i=1}^d a_i \vb*{u}_i + \sum_{j=1}^{n-d} b_j \vb*{v}_j \\
      \vb*{w} & = \sum_{i=1}^d c_i \vb*{u}_i + \sum_{k=1}^{m-d} d_k \vb*{w}_k
    \end{align*}

    \br

    $V+W$の任意の元は、$\vb*{v} + \vb*{w}$と書けるので、
    \begin{equation*}
      \vb*{v} + \vb*{w} = \sum_{i=1}^d (a_i + c_i) \vb*{u}_i + \sum_{j=1}^{n-d} b_j \vb*{v}_j + \sum_{k=1}^{m-d} d_k \vb*{w}_k
    \end{equation*}
    となり、$\{ \vb*{u}_1, \dots, \vb*{u}_d, \vb*{v}_1, \dots, \vb*{v}_{n-d}, \vb*{w}_1, \dots, \vb*{w}_{m-d} \}$の線形結合で表せる $\qed$
  \end{subpattern}

  \begin{subpattern}{\bfseries 線型独立であること}
    $\vb*{u}_1, \dots, \vb*{u}_d, \vb*{v}_1, \dots, \vb*{v}_{n-d}, \vb*{w}_1, \dots, \vb*{w}_{m-d}$が線型独立であることを示すために、次のような線形関係式を考える
    \begin{equation*}
      \sum_{i=1}^d c_i \vb*{u}_i + \sum_{j=1}^{n-d} c_{d+j} \vb*{v}_j + \sum_{k=1}^{m-d} c_{d+n-d+k} \vb*{w}_k = \vb*{0}
    \end{equation*}

    ここで、$c_i \in K$はスカラーである

    \br

    この式を$V$と$W$の基底の線型結合として考えると、$V$の基底$\vb*{u}_i, \, \vb*{v}_j$に関する部分と$W$の基底$\vb*{u}_i, \, \vb*{w}_k$に関する部分がそれぞれ線形独立であるため、結局どの項においても$c_i = 0$である必要がある

    \br

    よって、$\vb*{u}_1, \dots, \vb*{u}_d, \vb*{v}_1, \dots, \vb*{v}_{n-d}, \vb*{w}_1, \dots, \vb*{w}_{m-d}$は線型独立である $\qed$
  \end{subpattern}

  \br

  以上より、$\vb*{u}_1, \dots, \vb*{u}_d, \vb*{v}_1, \dots, \vb*{v}_{n-d}, \vb*{w}_1, \dots, \vb*{w}_{m-d}$は$V + W$の基底であることが示された

  \br

  この基底をなすベクトルの個数(次元)について考えると、
  \begin{align*}
    \dim(V + W) & = d + (n - d) + (m - d) \\
                & = n + m - d
  \end{align*}
  となる

  ここで、$d = \dim(V \cap W)$なので、
  \begin{equation*}
    \dim(V + W) = \dim V + \dim W - \dim(V \cap W)
  \end{equation*}
  と書き換えられ、目的の式が得られた $\qed$
\end{proof}

\end{document}
