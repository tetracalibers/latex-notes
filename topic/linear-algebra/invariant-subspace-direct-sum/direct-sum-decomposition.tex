\documentclass[../../../topic_linear-algebra]{subfiles}

\begin{document}

\sectionline
\section{直和分解}
\marginnote{\refbookA p113〜114 \\ \refbookF p233〜235 \\ \refbookG p6〜7}

\begin{definition}{直和分解}\label{def:direct-sum}
  線形空間$V$の部分集合$W_1,\,W_2$に対して、任意の$\vb*{v} \in V$が$\vb*{w}_1 \in W_1, \, \vb*{w}_2 \in W_2$によって
  \begin{equation*}
    \vb*{v} = \vb*{w}_1 + \vb*{w}_2
  \end{equation*}
  と一意的に表されるとき、$V$は$W_1$と$W_2$の\keyword{直和}である(\keyword{直和}に分解される)といい、
  \begin{equation*}
    V = W_1 \oplus W_2
  \end{equation*}
  と書く
\end{definition}

この定義は、次のように言い換えることができる

\begin{theorem}{直和分解の同値条件}\label{thm:direct-sum-equiv}
  線形空間$V$の部分集合$W_1,\,W_2$に対して$V=W_1 \oplus W_2$が成り立つことと、
  \begin{enumerate}[label=\romanlabel]
    \item $V = W_1 + W_2$
    \item $W_1 \cap W_2 = \{ \vb*{0} \}$
  \end{enumerate}
  の両方が成り立つことは同値である
\end{theorem}

\begin{proof}
  \begin{subpattern}{(\romannum{i}), (\romannum{ii}) $\Longrightarrow$ $V = W_1 \oplus W_2$}
    $\vb*{w}_1,\,\vb*{w}_1' \in W_1, \, \vb*{w}_2,\,\vb*{w}_2' \in W_2$とする

    仮定(\romannum{i})と和空間の定義より、
    \begin{equation*}
      \vb*{v} = \vb*{w}_1 + \vb*{w}_2 = \vb*{w}_1' + \vb*{w}_2'
    \end{equation*}

    この等式は、移項によって次のように変形できる
    \begin{equation*}
      \vb*{w}_1 - \vb*{w}_1' = \vb*{w}_2' - \vb*{w}_2
    \end{equation*}
    部分空間は和に閉じているため、左辺は$W_1$に、右辺は$W_2$に属する

    よって、このベクトルは$W_1 \cap W_2$に属する

    仮定$(\romannum{ii})$より、$W_1 \cap W_2$の元は零ベクトルであるので、
    \begin{align*}
      \vb*{w}_1 - \vb*{w}_1' & = \vb*{0} \\
      \vb*{w}_2' - \vb*{w}_2 & = \vb*{0}
    \end{align*}

    したがって、
    \begin{equation*}
      \vb*{w}_1 = \vb*{w}_1', \quad \vb*{w}_2 = \vb*{w}_2'
    \end{equation*}
    となり、$\vb*{v}$の表現の一意性が示された $\qed$
  \end{subpattern}

  \begin{subpattern}{$V = W_1 \oplus W_2$ $\Longrightarrow$ (\romannum{i}), (\romannum{ii})}
    和空間の定義をふまえると、(\romannum{i})は直和分解の定義に含まれる

    \br

    (\romannum{ii})を示すため、$\vb*{v} \in W_1 \cap W_2$とする

    $\vb*{v}$は零ベクトルを用いて、
    \begin{equation*}
      \vb*{v} = \vb*{v} + \vb*{0} = \vb*{0} + \vb*{v}
    \end{equation*}
    と表せるが、直和分解の定義より、$\vb*{v}$の表現は一意的であるので、
    \begin{equation*}
      \vb*{v} = \vb*{0}
    \end{equation*}
    を得る

    よって、$W_1 \cap W_2 = \{ \vb*{0} \}$が成り立つ $\qed$
  \end{subpattern}
\end{proof}

\subsection{直和分解の一意性を表す条件}

3つ以上の部分空間による直和を考えるにあたって、直和分解の定義に含まれていた「一意性」を表す条件を定式化する

\begin{theorem}{部分空間の和における表現の一意性}
  和空間$\sum_{i=1}^k V_i$の元$\vb*{v}$を、部分空間$V_1, \ldots, V_k$の元$\vb*{v}_i \in V_i$の和として
  \begin{equation*}
    \vb*{v} = \sum_{i=1}^k \vb*{v}_i \quad (\vb*{v}_i \in V_i)
  \end{equation*}
  と書くとする

  このとき、次の条件が成り立てば、和に使われる$\vb*{v}_i$は$\vb*{v}$により一意的に定まる
  \begin{equation*}
    \vb*{v}_1 + \cdots + \vb*{v}_k = \vb*{0} \, \Longrightarrow \, \vb*{v}_1 = \cdots = \vb*{v}_k = \vb*{0}
  \end{equation*}
\end{theorem}

\begin{proof}
  仮に、$\vb*{v}$が2通りの和で表せるとする
  \begin{equation*}
    \vb*{v} = \sum_{i=1}^k \vb*{v}_i = \sum_{i=1}^k \vb*{v}_i' \quad (\vb*{v}_i, \vb*{v}_i' \in V_i)
  \end{equation*}

  このとき、
  \begin{equation*}
    \sum_{i=1}^k (\vb*{v}_i - \vb*{v}_i') = \vb*{0}
  \end{equation*}
  となるが、ここで$\vb*{v}_i - \vb*{v}_i'$は$V_i$に属する

  \br

  そこで、$\vb*{w}_i = \vb*{v}_i - \vb*{v}_i' \in V_i$とおき、
  \begin{equation*}
    \vb*{w}_1 + \cdots + \vb*{w}_k = \vb*{0} \, \Longrightarrow \, \vb*{w}_1 = \cdots = \vb*{w}_k = \vb*{0}
  \end{equation*}
  という条件を満たすとすると、$\vb*{w}_i = \vb*{0}$より、
  \begin{equation*}
    \vb*{v}_i = \vb*{v}_i' \quad (i=1,\ldots,k)
  \end{equation*}
  が導かれる

  したがって、$\vb*{v}$の和に使われる$\vb*{v}_i$は一意的に定まる $\qed$
\end{proof}

\br

この一意性の条件を用いて、複数の部分空間による直和を次のように定義する

\begin{definition}{直和}
  線形空間$V$と、その部分空間$V_1,\ldots,V_k$が与えられたとき、$\vb*{v}_i \in V_i,\,\vb*{v} \in \sum_{i=1}^k V_i$に対して、
  \begin{equation*}
    \vb*{v}_1 + \cdots + \vb*{v}_k = \vb*{0} \, \Longrightarrow \, \vb*{v}_1 = \cdots = \vb*{v}_k = \vb*{0}
  \end{equation*}
  が成り立つとき、$\displaystyle\sum_{i=1}^{k} V_i$は$V$の\keyword{直和}であるといい、
  \begin{equation*}
    \bigoplus_{i=1}^k V_i
  \end{equation*}
  と書く
\end{definition}

\end{document}
