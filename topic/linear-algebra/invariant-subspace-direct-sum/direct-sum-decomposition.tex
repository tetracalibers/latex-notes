\documentclass[../../../topic_linear-algebra]{subfiles}

\begin{document}

\sectionline
\section{直和分解}
\marginnote{\refbookA p113〜114 \\ \refbookF p233〜235}

\begin{definition}{直和分解}
  線形空間$V$の部分集合$W_1,\,W_2$に対して、任意の$\vb*{v} \in V$が$\vb*{w}_1 \in W_1, \, \vb*{w}_2 \in W_2$によって
  \begin{equation*}
    \vb*{v} = \vb*{w}_1 + \vb*{w}_2
  \end{equation*}
  と一意的に表されるとき、$V$は$W_1$と$W_2$の\keyword{直和}である(\keyword{直和}に分解される)といい、
  \begin{equation*}
    V = W_1 \oplus W_2
  \end{equation*}
  と書く
\end{definition}

この定義は、次のように言い換えることができる

\begin{theorem}{直和分解の同値条件}
  線形空間$V$の部分集合$W_1,\,W_2$に対して$V=W_1 \oplus W_2$が成り立つことと、
  \begin{enumerate}[label=\romanlabel]
    \item $V = W_1 + W_2$
    \item $W_1 \cap W_2 = \{ \vb*{0} \}$
  \end{enumerate}
  の両方が成り立つことは同値である
\end{theorem}

\begin{proof}
  \begin{subpattern}{(\romannum{i}), (\romannum{ii}) $\Longrightarrow$ $V = W_1 \oplus W_2$}
    $\vb*{w}_1,\,\vb*{w}_1' \in W_1, \, \vb*{w}_2,\,\vb*{w}_2' \in W_2$とする

    仮定(\romannum{i})と和空間の定義より、
    \begin{equation*}
      \vb*{v} = \vb*{w}_1 + \vb*{w}_2 = \vb*{w}_1' + \vb*{w}_2'
    \end{equation*}

    この等式は、移項によって次のように変形できる
    \begin{equation*}
      \vb*{w}_1 - \vb*{w}_1' = \vb*{w}_2' - \vb*{w}_2
    \end{equation*}
    部分空間は和に閉じているため、左辺は$W_1$に、右辺は$W_2$に属する

    よって、このベクトルは$W_1 \cap W_2$に属する

    仮定$(\romannum{ii})$より、$W_1 \cap W_2$の元は零ベクトルであるので、
    \begin{align*}
      \vb*{w}_1 - \vb*{w}_1' & = \vb*{0} \\
      \vb*{w}_2' - \vb*{w}_2 & = \vb*{0}
    \end{align*}

    したがって、
    \begin{equation*}
      \vb*{w}_1 = \vb*{w}_1', \quad \vb*{w}_2 = \vb*{w}_2'
    \end{equation*}
    となり、$\vb*{v}$の表現の一意性が示された $\qed$
  \end{subpattern}

  \begin{subpattern}{$V = W_1 \oplus W_2$ $\Longrightarrow$ (\romannum{i}), (\romannum{ii})}
    和空間の定義をふまえると、(\romannum{i})は直和分解の定義に含まれる

    \br

    (\romannum{ii})を示すため、$\vb*{v} \in W_1 \cap W_2$とする

    $\vb*{v}$は零ベクトルを用いて、
    \begin{equation*}
      \vb*{v} = \vb*{v} + \vb*{0} = \vb*{0} + \vb*{v}
    \end{equation*}
    と表せるが、直和分解の定義より、$\vb*{v}$の表現は一意的であるので、
    \begin{equation*}
      \vb*{v} = \vb*{0}
    \end{equation*}
    を得る

    よって、$W_1 \cap W_2 = \{ \vb*{0} \}$が成り立つ $\qed$
  \end{subpattern}
\end{proof}

3つ以上の部分空間の直和分解も、同様に定義される

\end{document}
