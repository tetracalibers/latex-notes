\documentclass[../../../topic_linear-algebra]{subfiles}

\begin{document}

\sectionline
\section{連立一次方程式の行列表記}
\marginnote{\refbookA p22〜25}

未知数$x_1, x_2, \dots, x_n$に関する連立方程式として
\begin{equation*}
  \left\{
  \begin{alignedat}{12}
    a_{11} & x_1    & {}+{} & a_{12} & x_2 & {}+{} & \cdots & {}+{} & a_{1n} & x_n & {}={} & b_1    \\
    a_{21} & x_1    & {}+{} & a_{22} & x_2 & {}+{} & \cdots & {}+{} & a_{2n} & x_n & {}={} & b_2    \\
           & \vdots &       & \vdots &     &       &        &       & \vdots &     &       & \vdots \\
    a_{m1} & x_1    & {}+{} & a_{m2} & x_2 & {}+{} & \cdots & {}+{} & a_{mn} & x_n & {}={} & b_m    \\
  \end{alignedat}
  \right.
\end{equation*}
を考える

$a_{ij}$などは与えられた定数であり、\keyword{係数}と呼ばれる

$i$番目の式の$x_j$の係数を$a_{ij}$と書いている

\br

ここで、係数だけを集めて行列を作る
\begin{equation*}
  A = \begin{pmatrix}
    a_{11} & a_{12} & \cdots & a_{1n} \\
    a_{21} & a_{22} & \cdots & a_{2n} \\
           & \vdots &        & \vdots \\
    a_{m1} & a_{m2} & \cdots & a_{mn}
  \end{pmatrix}
\end{equation*}

すると、先ほどの連立方程式は、ベクトル形で
\begin{equation*}
  x_1 \vb*{a}_1 + x_2 \vb*{a}_2 + \cdots + x_n \vb*{a}_n = \vb*{b}
\end{equation*}
と書ける

\br

また、$n$個の未知数$x_1, x_2, \dots, x_n$からベクトルを作る
\begin{equation*}
  \vb*{x} = \begin{pmatrix}
    x_1    \\
    x_2    \\
    \vdots \\
    x_n
  \end{pmatrix}
\end{equation*}

すると、ベクトル形の方程式の左辺のベクトルを、行列$A$とベクトル$\vb*{x}$の積と考えて、$A\vb*{x}$と表記できる

こうして、もとの連立一次方程式は、行列形の方程式
\begin{equation*}
  A\vb*{x} = \vb*{b}
\end{equation*}
に書き換えられる

\end{document}
