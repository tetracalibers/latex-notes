\documentclass[../../../topic_linear-algebra]{subfiles}

\begin{document}

\sectionline
\section{拡大係数行列}
\marginnote{\refbookB p99〜100}

次のような連立一次方程式を考える

\br

\begin{center}
  \systeme{
    x + y =3,
    -x + z= -1,
    -2x - y + 2z = -4
  }
\end{center}

\br

0や1の係数を省略せずに書くと、

\br

\begin{center}
  \systeme{
    1x + 1y + 0z = 3,
    -1x + 0y + 1z = -1,
    -2x - 1y + 2z = -4
  }
\end{center}

\br

となるので、これを行列で表すと、

\begin{equation*}
  \begin{pmatrix}
    1  & 1  & 0 \\
    -1 & 0  & 1 \\
    -2 & -1 & 2
  \end{pmatrix} \begin{pmatrix}
    x \\
    y \\
    z
  \end{pmatrix} = \begin{pmatrix}
    3  \\
    -1 \\
    -4
  \end{pmatrix}
\end{equation*}

\br

左辺の行列は、連立方程式の係数だけを取り出した行列になっているので、\keyword{係数行列}と呼ばれる
\begin{equation*}
  A = \begin{pmatrix}
    1  & 1  & 0 \\
    -1 & 0  & 1 \\
    -2 & -1 & 2
  \end{pmatrix}
\end{equation*}

\br

また、右辺のベクトルは定数項をまとめたものになっているので、\keyword{定数項ベクトル}と呼ばれる
\begin{equation*}
  \vb*{b} = \begin{pmatrix}
    3  \\
    -1 \\
    -4
  \end{pmatrix}
\end{equation*}

\br

連立方程式の解を求める過程の式変形(行基本変形)によって、係数行列$A$と定数項ベクトル$\vb*{b}$の成分が変化していく

\br

そこで、変形の過程で変化する数(操作の対象)を、次のように1つの行列$(A \mid \vb*{b})$としてまとめてしまおう

\begin{equation*}
  \begin{pNiceArray}{ccc|c}
    1 & 1 & 0 & 3 \\
    -1 & 0 & 1 & -1\\
    -2 & -1 & 2 & -4
  \end{pNiceArray}
\end{equation*}

\br

このように、係数行列と定数項ベクトルを1つの行列としてまとめたものを\keyword{拡大係数行列}という

\br

拡大係数行列に対して行基本変形を繰り返し行うことで、連立一次方程式の解を求めることができる

\end{document}
