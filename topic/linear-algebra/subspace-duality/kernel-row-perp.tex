\documentclass[../../../topic_linear-algebra]{subfiles}

\usepackage{xr-hyper}
\externaldocument{../../../.tex_intermediates/topic_linear-algebra}

\begin{document}

\sectionline
\section{行空間と核空間の直交}
\marginnote{\refbookC p137 \\ \refbookA p123}

$A = (a_{ij})$を$m \times n$行列とする。

このとき、$A$の第$i$行ベクトル$\begin{pmatrix} a_{i1} & \cdots & a_{in} \end{pmatrix} \in {}^t\mathbb{R}^n$は、\secref{sec:row-vector-functional}より、次の線形汎関数$\phi_i$と同一視できる。
\begin{equation*}
  \phi_i = a_{i1} x_1 + \cdots + a_{in} x_n
\end{equation*}
この行ベクトル$\phi_i \in {}^t\mathbb{R}^n$が張る空間$\langle \phi_1, \ldots, \phi_n \rangle \subset {}^t\mathbb{R}^n$を、$A$の\keyword{行空間}と呼び、$\Row A$と書く。

\br

ここで、$x_j$はベクトル$\vb*{v} \in \mathbb{R}^n$に対して、$j$番目の成分を返す\keyword{座標関数}である。
そこで、
\begin{align*}
  \phi_i(\vb*{v}) &= a_{i1} x_1(\vb*{v}) + \cdots + a_{in} x_n(\vb*{v}) \\
  &= a_{i1} v_1 + \cdots + a_{in} v_n
\end{align*}
とみると、$\phi_i$は$\vb*{v}$に作用して、$A$の第$i$行ベクトルと$\vb*{v}$の内積を返すことがわかる。

\br

すると、$\phi_i(\vb*{v})$を縦に並べたものは、$A\vb*{v}$に一致する。
\begin{equation*}
  A\vb*{v} = \begin{pmatrix}
    a_{11} v_1 + \cdots + a_{1n} v_n \\
    \vdots \\
    a_{m1} v_1 + \cdots + a_{mn} v_n
  \end{pmatrix} = \begin{pmatrix}
    \phi_1(\vb*{v}) \\
    \vdots \\
    \phi_m(\vb*{v})
  \end{pmatrix}
\end{equation*}

\br

このとき、$A\vb*{v} = \vb*{o}$となる場合は、
\begin{equation*}
  \phi_i(\vb*{v}) = 0 \quad (i = 1, \ldots, m)
\end{equation*}
が成り立つことになる。

すなわち、$A$のすべての行ベクトルに対して、$\vb*{v}$との内積が0になる。

\br

このことから、$A$の行空間に属するベクトルと、$A\vb*{v} = \vb*{o}$の解空間$\Ker A$に属するベクトル$\vb*{v}$は、互いに直交することがわかる。

よって、次の関係が成り立つ。
\begin{equation*}
  \Ker A = (\Row A)^\perp
\end{equation*}

また、\thmref{thm:double-orthogonal-complement}より、$(\Row A)^\perp$の直交補空間は$\Row A$に一致することから、両辺の直交補空間をとると、次も成り立つ。
\begin{equation*}
  (\Ker A)^\perp = \Row A
\end{equation*}

\begin{theorem}{核空間と行空間の直交関係}{kernel-row-orthogonality}
  $A$の核空間と、$A$の行空間(行ベクトルが張る空間)は、直交補空間の関係にある。
  \begin{align*}
    \Ker A &= (\Row A)^\perp \\
    (\Ker A)^\perp &= \Row A
  \end{align*}
\end{theorem}

この定理は、核空間と像空間との関係として言い換えることもできる。

\br

\secref{sec:image-and-column-space}より、$A$の像空間$\Im A$は、$A$の列ベクトルが張る空間(列空間)であった。

$A$を転置すると行と列が入れ替わるので、$A^\top$の行空間は$A$の列空間に対応する。

\br

よって、定理を次のように書き換えることができる。

\begin{theorem*}{核空間と転置行列の像空間の直交関係}
  $A$の核空間と、$A^\top$の像空間は、直交補空間の関係にある。
  \begin{align*}
    \Ker A &= (\Im A^\top)^\perp \\
    (\Ker A)^\perp &= \Im A^\top
  \end{align*}
\end{theorem*}

\subsection{直交補空間から零化空間へ}

さて、$\Ker A = (\Row A)^\perp$という\keyword{直交補空間}の関係を導くにあたって、ここでは次のような議論を行った。
\begin{enumerate}
  \item 横ベクトルと同一視できる線形汎関数を考える
  \item 線形汎関数に縦ベクトルを作用させたものを内積とみなす
  \item 内積が0になることから直交補空間の関係を導く
\end{enumerate}

つまり、ここでは内積が定められている空間(計量空間)で議論を行ったわけだが、内積を考えずに、線形汎関数の集合(双対空間)だけで議論を行うこともできる。

\br

直交補空間の概念を内積を使わずに拡張し、一般の線形空間上で定義したものが、次に述べる\keyword{零化空間}である。

\end{document}
