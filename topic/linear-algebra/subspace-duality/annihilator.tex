\documentclass[../../../topic_linear-algebra]{subfiles}

\begin{document}

\sectionline
\section{零化空間}
\marginnote{\refbookG p62 \\ \refbookR p240〜241}

$V$を線形空間とし、その部分空間$W \subset V$を考える。

$V$の双対空間$V^*$(線形汎関数の集合)の中で、「$W$の元に作用させると0になる」ような線形汎関数を集めた集合を\keywordJE{零化空間}{annihilator}という。

\begin{definition}{零化空間}
  $V$を$n$次元の線形空間とする。$W$を$V$の部分空間とするとき、
  \begin{equation*}
    W^\perp = \{\phi \in V^* \mid \forall \vb*{w} \in W, \langle \phi, \vb*{w} \rangle = 0 \}
  \end{equation*}
  を$W$の\keyword{零化空間}という。
\end{definition}

$\phi \in V^*$が$W$のすべてのベクトル$\vb*{w}$に対して$\phi(\vb*{w}) = 0$となるとき、その$\phi$は$W$を「全滅させてしまう(\en{annihilate})」という意味で、零化空間は\en{annihilator}と呼ばれる。

\subsection{零化空間は$V^*$の部分空間}

$V^*$の中から、$\phi(\vb*{w}) = 0$を満たす元$\phi \in V^*$を取り出した集合が$W^\perp$であるので、$W^\perp$は$V^*$の部分空間である。

\begin{theorem}{零化空間の双対空間への包含関係}
  $V$を$n$次元の線形空間とし、$W$を$V$の部分空間とする。
  
  このとき、$W$の零化空間$W^\perp$は、$V$の双対空間$V^*$の部分空間である。
\end{theorem}

\begin{proof}
  \todo{}
\end{proof}

\end{document}
