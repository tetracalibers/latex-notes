\documentclass[../../../topic_linear-algebra]{subfiles}

\begin{document}

\sectionline
\section{実二次形式の標準化}
\marginnote{\refbookA p210 \\ \refbookF p298〜299}

$A$が実対称行列であることから、$A$は適当な直交行列$P$を用いて対角化できる
\begin{equation*}
  B = P^{-1}AP = \begin{pmatrix}
    \alpha_1 &        &          \\
             & \ddots &          \\
             &        & \alpha_n
  \end{pmatrix}
\end{equation*}

\br

与えられた二次形式$Q(\vb*{x}) = {}^t \vb*{x} A \vb*{x}$に対して、$\vb*{y} = P^{-1}\vb*{x}$、すなわち$\vb*{x} = P\vb*{y}$という変数の変換を行うと、
\begin{equation*}
  \begin{WithArrows}
    {}^t \vb*{x} A \vb*{x} & = {}^t (P\vb*{y}) A (P\vb*{y}) \\
    & = {}^t \vb*{y} {}^t P A P \vb*{y} \Arrow{直交行列の定義${}^t P = P^{-1}$} \\
    & = {}^t \vb*{y} (P^{-1}AP) \vb*{y} \\
    & = {}^t \vb*{y} B \vb*{y}
  \end{WithArrows}
\end{equation*}
となるので、変数$\vb*{y}$に関する係数行列は$B = P^{-1}AP$である

\br

$B$の形から、実際に${}^t \vb*{y} B \vb*{y}$を計算してみると、
\begin{align*}
  {}^t \vb*{y} B \vb*{y} & = \begin{pmatrix}
                               y_1 & \cdots & y_n
                             \end{pmatrix} \begin{pmatrix}
                                             \alpha_1 &        &          \\
                                                      & \ddots &          \\
                                                      &        & \alpha_n
                                           \end{pmatrix} \begin{pmatrix}
                                                           y_1    \\
                                                           \vdots \\
                                                           y_n
                                                         \end{pmatrix} \\
                         & = \alpha_1 y_1^2 + \cdots + \alpha_n y_n^2
\end{align*}
となり、\keyword{交叉項}$y_iy_j \, (i \neq j)$が現れない形に書き換わったことがわかる

\br

このような交叉項のない形を実二次形式の\keyword{標準形}という

\begin{theorem*}{実二次形式の直交対角化と標準形}
  実二次形式$Q(\vb*{x}) = {}^t \vb*{x} A \vb*{x}$に対して、$A$を対角化する直交行列$P$による座標変換$\vb*{x} = P\vb*{y}$を行えば、
  \begin{equation*}
    Q(\vb*{x}) = \alpha_1 y_1^2 + \cdots + \alpha_n y_n^2
  \end{equation*}
  という、変数$\vb*{y}$に関する交叉項のない形(実二次形式の\keyword{標準形})にできる

  ここで、$\alpha_1, \ldots, \alpha_n$は重複を含めて$A$の固有値と一致する
\end{theorem*}

\end{document}
