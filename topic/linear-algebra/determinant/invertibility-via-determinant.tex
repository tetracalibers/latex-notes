\documentclass[../../../topic_linear-algebra]{subfiles}

\begin{document}

\sectionline
\section{行列式と正則性}
\marginnote{\refbookA p164 \\ \refbookF p132〜133}

行列式は、正則性の判定にも利用できる

\begin{theorem}{正則性と行列式の非零性}
  \begin{equation*}
    A\text{が正則行列} \Longleftrightarrow \det(A) \neq 0
  \end{equation*}
\end{theorem}

\begin{proof}
  \begin{subpattern}{$\Longrightarrow$}
    $A$が正則であることから、
    \begin{equation*}
      AA^{-1} = E
    \end{equation*}
    両辺の行列式をとって、
    \begin{equation*}
      \det(AA^{-1}) = \det(E)
    \end{equation*}
    左辺には行列式の乗法性を適用し、右辺は単位行列の行列式の値が1であることから、
    \begin{equation*}
      \det(A)\det(A^{-1}) = 1
    \end{equation*}
    もし$\det(A) = 0$だと仮定すると、$0 = 1$という矛盾した式になる

    よって、$\det(A) \neq 0$でなければならない $\qed$
  \end{subpattern}

  \begin{subpattern}{$\Longleftarrow$}
    \hyperref[thm:lin-indep-if-det-nonzero]{$\det(A) \neq 0$であることから、行列$A$の列ベクトルは線型独立}である

    そして、\hyperref[thm:invertible-iff-col-indep]{$A$の列ベクトルが線型独立であることと、$A$が正則であることは同値}である $\qed$
  \end{subpattern}
\end{proof}

この定理の派生として、行列式を次の形で使うことが多い

\begin{theorem}{消去法の原理}
  $A$を正方行列とするとき、
  \begin{equation*}
    A\vb*{x} = \vb*{0}\text{に非自明解が存在する} \Longleftrightarrow \det(A) = 0
  \end{equation*}
\end{theorem}

\end{document}
