\documentclass[../../../topic_linear-algebra]{subfiles}

\begin{document}

\sectionline
\section{行列式の基本性質}
\marginnote{\refbookA p161〜166 \\ \refbookF p113〜121}

次の性質により、以後議論する行列式の性質が列に対して成り立つなら、行に対しても成り立つといえるようになる

\begin{theorem}{行列式の対称性}\label{thm:determinant-transpose-invariance}
  \begin{equation*}
    \det({}^t A) = \det(A)
  \end{equation*}
\end{theorem}

\begin{proof}
  行列式の定義より、行列${}^tA$の行列式は、行列$A$の行列式に現れる$a_{i,\sigma(i)}$の添字を入れ替えたもの$a_{\sigma(i),i}$の積和になる
  \begin{equation*}
    \det({}^t A) = \sum_{\sigma \in S_n} \sgn(\sigma) \prod_{i=1}^n a_{\sigma(i),i}
  \end{equation*}

  一方、$j=\sigma(i)$とおくと、$i = \sigma^{-1}(j)$となるので、添字の変数を変換して
  \begin{align*}
    \prod_{i=1}^n a_{\sigma(i),i}
     & = \prod_{j=1}^n a_{j,\sigma^{-1}(j)}
  \end{align*}

  よって、$\det({}^t A)$の各項は、
  \begin{equation*}
    \sgn(\sigma^{-1}) \prod_{j=1}^n a_{j,\sigma^{-1}(j)}
  \end{equation*}
  となるが、これは$\det(A)$の定義式の$\sigma^{-1}$に対応する項と同じである

  \br

  ここで、$\rho = \sigma^{-1}$とおくと、$\sigma = \rho^{-1}$であり、\hyperref[thm:sign-of-inverse-permutation]{逆置換の符号}から$\sgn(\sigma) = \sgn(\rho^{-1}) = \sgn(\rho)$であるから、
  \begin{equation*}
    \det({}^t A) = \sum_{\rho \in S_n} \sgn(\rho) \prod_{j=1}^n a_{j,\rho(j)}  = \det(A)
  \end{equation*}

  よって、$\det({}^t A) = \det(A)$が示された $\qed$
\end{proof}

\sectionline

\begin{theorem}{行列式の列についての交代性}
  行列$A = (\vb*{a}_1, \ldots, \vb*{a}_n)$において、2つの列を入れ替えた行列を作ると、その行列の行列式の値は、元の行列$A$の行列式の値の$(-1)$倍になる
  \begin{multline*}
    \det(\vb*{a}_1, \ldots, \vb*{a}_i, \ldots, \vb*{a}_j, \ldots, \vb*{a}_n) \\= -\det(\vb*{a}_1, \ldots, \vb*{a}_j, \ldots, \vb*{a}_i, \ldots, \vb*{a}_n) \\ (1 \leq i < j \leq n)
  \end{multline*}
\end{theorem}

\begin{proof}
  元々の行列$A$の行列式の各項が、
  \begin{equation*}
    f(\sigma) =\sgn(\sigma) a_{\sigma(1),1} \cdots a_{\sigma(i),i} \cdots a_{\sigma(j),j} \cdots a_{\sigma(n),n}
  \end{equation*}
  であるのに対し、第$i$列と$j$列を入れ替えた行列の行列式の各項は、
  \begin{equation*}
    \sgn(\sigma) a_{\sigma(1),1} \cdots a_{\sigma(i),j} \cdots a_{\sigma(j),i} \cdots a_{\sigma(n),n}
  \end{equation*}
  となる

  ここで、$i$を$j$に、$j$を$i$に写す互換$\sigma_0 = (ij)$を考え、$\tau = \sigma \sigma_0$とおくと、$\sigma(j) = \tau(i), \, \sigma(i) = \tau(j)$となるので、
  \begin{equation*}
    f(\tau) = \sgn(\tau) a_{\tau(1),1} \cdots a_{\tau(i),i} \cdots a_{\tau(j),j} \cdots a_{\tau(n),n}
  \end{equation*}

  このとき、\hyperref[thm:sum-invariance-under-permutation-action]{置換群の左右作用に対する和の不変性}より、
  \begin{equation*}
    \sum_{\sigma \in S_n} f(\sigma) = \sum_{\sigma \in S_n} f(\sigma\sigma_0) = \sum_{\tau \in S_n} f(\tau)
  \end{equation*}
  すなわち、$\sigma$全体の総和は$\tau$全体の総和に一致する

  \br

  さらに、\hyperref[thm:sign-multiplicativity]{置換の符号の乗法性}より、
  \begin{equation*}
    \sgn(\tau) = \sgn(\sigma) \sgn(\sigma_0) = -\sgn(\sigma)
  \end{equation*}
  であるから、
  \begin{equation*}
    f(\sigma) = -f(\tau)
  \end{equation*}

  よって、列の交換後、行列式全体が$(-1)$倍される $\qed$
\end{proof}

\sectionline

\begin{theorem}{行列式の列についての多重線形性}
  行列式を列の関数とみたとき、この関数は、どの列についても線形である
  \begin{multline*}
    \det(\vb*{a}_1,\ldots,\alpha\vb*{u} + \beta\vb*{v}, \ldots, \vb*{a}_n) \\
    = \alpha\det(\vb*{a}_1,\ldots,\vb*{u},\ldots,\vb*{a}_n) \\
    + \beta\det(\vb*{a}_1,\ldots,\vb*{v},\ldots,\vb*{a}_n)
  \end{multline*}
\end{theorem}

\begin{proof}
  $\sigma \in S_n$に対応する各項について、
  \begin{equation*}
    a_{\sigma(1),1} \cdots (\alpha u_{\sigma(i)} + \beta v_{\sigma(i)}) \cdots a_{\sigma(n),n}
  \end{equation*}

  $C = a_{\sigma(1),1} \cdots a_{\sigma(n),n}$とし、$A = \alpha u_{\sigma(i)}, \, B = \beta v_{\sigma(i)}$とおくと、
  \begin{equation*}
    C(A + B) = CA + CB = \alpha C u_{\sigma(i)} + \beta C v_{\sigma(i)}
  \end{equation*}
  のように展開できる

  よって、
  \begin{multline*}
    \alpha (a_{\sigma(1),1} \cdots u_{\sigma(i)} \cdots a_{\sigma(n),n}) \\
    + \beta(a_{\sigma(1),1} \cdots v_{\sigma(i)} \cdots a_{\sigma(n),n})
  \end{multline*}
  を用いれば、行列式の定義に基づいて定理が成り立つことがわかる $\qed$
\end{proof}

\sectionline

\hyperref[thm:determinant-transpose-invariance]{行列式の対称性}より、次の定理も得られる

\begin{theorem}{行列式の行についての多重線形性と交代性}
  行列式は行に関しても多重線形性と交代性をもつ
\end{theorem}

以降、列に対して成り立つ性質は行に対しても成り立つとし、列の場合のみを記載する

\sectionline

\begin{theorem}{列の重複による行列式の零化}\label{thm:det-zero-if-columns-repeat}
  $A = (\vb*{a}_1, \ldots, \vb*{a}_n)$の$n$個の列の中に、まったく同じものがあれば、
  \begin{equation*}
    \det(A) = 0
  \end{equation*}
  となる
\end{theorem}

\begin{proof}
  行列$A$の列ベクトルに、共通のベクトル$\vb*{u}$が含まれているとする
  \begin{equation*}
    A = (\ldots, \vb*{u}, \ldots, \vb*{u}, \ldots)
  \end{equation*}
  この2つの$\vb*{u}$の列を入れ替えると、
  \begin{equation*}
    \det(\ldots, \vb*{u}, \ldots, \vb*{u}, \ldots) = -\det(\ldots, \vb*{u}, \ldots, \vb*{u}, \ldots)
  \end{equation*}
  ところが、入れ替えの前後で行列そのものは変化していない(まったく同じ列を入れ替えても行列は同じ)ので、行列式の値も変わらないはずである

  すなわち、
  \begin{equation*}
    \det A = -\det A
  \end{equation*}
  が成り立つ

  ここで、両辺に$\det(A)$を足すと、
  \begin{equation*}
    2\det A = 0
  \end{equation*}
  より、$\det A = 0$が成り立つ $\qed$
\end{proof}

\sectionline
\section{基本変形と行列式}
\marginnote{\refbookF p117〜118 \\ \refbookA p162}

行列式の性質から、行列の列や行に関する基本変形と行列式の関係が見えてくる

\begin{theorem}{基本変形と行列式の関係}
  \begin{enumerate}[label=\romanlabel]
    \item 列(行)を交換すると行列式の符号が交換される
    \item 列(行)を定数倍すると、行列式の値も定数倍される
    \item 列(行)に他の列(行)の定数倍を加えても行列式の値は変化しない
  \end{enumerate}
\end{theorem}

(\romannum{i})は行列式の交代性、(\romannum{ii})は多重線形性であり、(\romannum{iii})は次の定理によって示される

\begin{theorem}{列の掃き出しに関する不変性}
  $i \neq j$のとき、
  \begin{multline*}
    \det(\ldots, \vb*{a}_i + c \vb*{a}_j,\ldots, \vb*{a}_j \ldots) \\= \det(\ldots, \vb*{a}_i,\ldots, \vb*{a}_j \ldots)
  \end{multline*}
\end{theorem}

\begin{proof}
  行列式の多重線形性より、
  \begin{multline*}
    \det(\ldots, \vb*{a}_i + c \vb*{a}_j,\ldots, \vb*{a}_j \ldots) \\
    = \det(\ldots, \vb*{a}_i,\ldots, \vb*{a}_j \ldots) + c\det(\ldots, \vb*{a}_j,\ldots, \vb*{a}_j \ldots)
  \end{multline*}
  ここで、同じ列ベクトル$\vb*{a}_j$が2つ含まれている行列式の値は0になるので、
  \begin{equation*}
    \det(\ldots, \vb*{a}_i + c \vb*{a}_j,\ldots, \vb*{a}_j \ldots) = \det(\ldots, \vb*{a}_i,\ldots, \vb*{a}_j \ldots)
  \end{equation*}
  だけが残る $\qed$
\end{proof}

\end{document}
