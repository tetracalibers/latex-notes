\documentclass[../../../topic_linear-algebra]{subfiles}

\usepackage{xr-hyper}
\externaldocument{../../../.tex_intermediates/topic_linear-algebra}

\begin{document}

\sectionline
\section{置換と互換}
\marginnote{\refbookA p155〜158}

たとえば、$(1,2,3,4)$を並び替えた列$(i,j,k,l)$があるとして、
\begin{align*}
  1 & \longmapsto i \\
  2 & \longmapsto j \\
  3 & \longmapsto k \\
  4 & \longmapsto l
\end{align*}
というように、番号を並び替える操作そのものを写像とみなし、\keyword{置換}と呼ぶ

\begin{definition*}{置換}
  集合$\{ 1,2,\ldots,n \}$からそれ自身への写像$\sigma$が全単射であるとき、$\sigma$は$n$次の\keyword{置換}であるという
\end{definition*}

たとえば、
\begin{equation*}
  \sigma(1) = 2, \quad \sigma(2) = 3, \quad \sigma(3) = 1
\end{equation*}
によって3次の置換を定めることができる

この置換を、
\begin{equation*}
  \sigma = \begin{pmatrix}
    1 & 2 & 3 \\
    2 & 3 & 1
  \end{pmatrix}
\end{equation*}
と表記する

\subsection{置換の積}

写像とみる利点の1つは、積が定義できることである

\br

もう1つの置換
\begin{equation*}
  \tau = \begin{pmatrix}
    1 & 2 & 3 \\
    1 & 3 & 2
  \end{pmatrix}
\end{equation*}
が与えられたとき、合成写像$\sigma \circ \tau$は、
\begin{gather*}
  1 \overset{\tau}{\longmapsto} 1 \overset{\sigma}{\longmapsto} 2 \\
  2 \overset{\tau}{\longmapsto} 3 \overset{\sigma}{\longmapsto} 1 \\
  3 \overset{\tau}{\longmapsto} 2 \overset{\sigma}{\longmapsto} 3
\end{gather*}
なので、
\begin{equation*}
  \sigma \tau = \begin{pmatrix}
    1 & 2 & 3 \\
    2 & 1 & 3
  \end{pmatrix}
\end{equation*}
である

通常、合成の記号$\circ$を書かずに$\sigma \tau$と表記する

\br

なお、$\sigma\tau$と$\tau\sigma$は一般に異なる

\br

写像の合成の結合法則から、置換の積でも結合法則が成り立つ

\begin{theorem*}{置換の積の結合法則}
  \begin{equation*}
    (\sigma \tau) \rho = \sigma (\tau \rho)
  \end{equation*}
\end{theorem*}

\subsection{恒等置換}

恒等写像
\begin{gather*}
  \id\colon \{ 1,2,\ldots,n \} \longmapsto \{ 1,2,\ldots,n \} \\
  \id(i) = i \quad (1 \leq i \leq n)
\end{gather*}
は置換であるので、これを\keyword{恒等置換}と呼び、
\begin{equation*}
  e = \id
\end{equation*}
と書く

\br

任意の置換$\sigma$に対して、明らかに
\begin{equation*}
  \sigma e = e \sigma = \sigma
\end{equation*}
が成り立つ

\br

また、次の性質はのちに行列式の性質を議論する際に重要になる

\begin{theorem}{恒等置換の単調性による特徴づけ}{identity-permutation-characterized-by-monotonicity}
  $i \leq \sigma(i)$(あるいは$i \geq \sigma(i)$)を満たす置換$\sigma$は恒等置換しか存在しない
\end{theorem}

\begin{proof}
  $\sigma$が恒等置換でないと仮定する

  条件$i \leq \sigma(i)$より、「元の位置より後ろに移される」、すなわち「すべてが自分以上に移る」ことになる

  たとえば、1を2に、2を3に、$\ldots$、$n-1$を$n$に写す置換を考える

  しかし、集合$\{ 1,2,\ldots,n \}$の要素は$n$個しかないので、$n$を$n+1$に写すことはできない

  そこで、$n$を$n$に写すとすると、$n-1$も$n$も$n$に写ることになり、これは置換が全単射であるという定義に反する

  \br

  $i \geq \sigma(i)$の場合も、「元の位置より前に移される」、すなわち「すべてが自分以下に移る」ことになると考えると、同様の矛盾が生じる

  \br

  よって、$\sigma$は恒等置換でなければならない $\qed$
\end{proof}

\subsection{逆置換}

置換$\sigma$は、定義より全単射であるので、逆写像$\sigma^{-1}$が存在する

これを\keyword{逆置換}と呼ぶ

\subsection{置換の集合}

すべての$n$次の置換からなる集合は\keyword{群}と呼ばれる構造を持っている

これを\keyword{$n$次対称群}と呼び、記号$S_n$で表す

\subsection{互換}

置換の中で最も基本的なのは、2文字だけを交換する置換である

\begin{definition*}{互換}
  $1 \leq i \neq j \leq n$のとき、$\sigma(i) = j,\, \sigma(j) = i$であって、$k$が$i,\,j$以外のとき$\sigma(k)=k$とすることで得られる置換を
  \begin{equation*}
    \sigma = (ij)
  \end{equation*}
  と書き、このような置換を\keyword{互換}という
\end{definition*}

たとえば、
\begin{equation*}
  (24) = \begin{pmatrix}
    1 & 2 & 3 & 4 & 5 \\
    1 & 4 & 3 & 2 & 5
  \end{pmatrix}
\end{equation*}

\subsection{互換の逆置換}

互換は$(ij)$と書いても$(ji)$と書いても同じ操作を表す

$i$と$j$を交換してから$j$と$i$を交換すると元に戻るが、この$(ij)$と$(ji)$は互換としては同じなので、
\begin{shaded}
  互換の逆置換は自分自身
\end{shaded}
である

\subsection{置換の一行表示}

置換を表す2行の表示は、下の行だけで情報としては十分なので、たとえば
\begin{equation*}
  \sigma = \begin{pmatrix}
    1 & 2 & 3 & 4 & 5 \\
    1 & 4 & 3 & 2 & 5
  \end{pmatrix}
\end{equation*}
を$\sigma = 14325$などと書いてしまうと便利である

これを$\sigma$の\keyword{一行表示}と呼ぶ

\subsection{互換と置換の積}

一行表示を用いた場合、互換と置換の積はたとえば次のように書ける

$\sigma = 14325$とすると、
\begin{equation*}
  (12)\sigma = 24315,\quad \sigma(12) = 41325
\end{equation*}

$(12)\sigma$は、$\sigma = 14325$に互換$(12)$を作用させて、$24315$となる

\br

$\sigma(12)$は、$12345$に互換$(12)$を作用させて$21345$とし、さらに置換$\sigma$を作用させることを意味する

置換$\sigma$は、4と2を入れ替える置換なので、$21345$に対して$\sigma$を作用させると、$41325$となる

\br

この例の結果を一般的に述べると、次のようになる

\begin{theorem*}{互換と置換の積}
  $\sigma \in S_n$に対して、$\tau = (ij)$を左からかけた$\tau\sigma$の一行表示は、$\sigma$の数字$i$と$j$を交換したものである

  また、$\tau$を右からかけた$\sigma\tau$の一行表示は、$\sigma$の$i$番目の数字と$j$番目の数字を交換したものである
\end{theorem*}

\subsection{互換の積への分解}

たとえば、$\sigma = 2413$とすると、これは、
\begin{enumerate}
  \item $1234$の3と4を交換して$1243$
  \item $1243$の1と2を交換して$2143$
  \item $2143$の2と3を交換して$2413$
\end{enumerate}
というように、互換に分解して考えることができる

数式でまとめると、
\begin{equation*}
  \sigma = (34)(12)(23)
\end{equation*}

\begin{theorem*}{互換の積への置換の分解}
  任意の置換$\sigma$は、いくつかの互換の積として書ける
\end{theorem*}

\begin{proof}
  $n$に対する帰納法を用いる

  \br

  $n=1$のときは、互換の定義における$i,\,j$の条件を満たさず、$i,\,j$以外の$k$について$\sigma(k)=k$とすることで得られる置換に相当するので、1つの互換とみなせる

  \br

  $(n-1)$次以下の置換が互換の積で書けることを仮定する

  $\sigma$を$n$次の置換とし、$\sigma(n)$の値を$c$とする

  \br

  $c=n$すなわち$\sigma(c)=c$の場合、$\sigma$は$c$をまったく動かしていないため、実質的に$c-1$までの数字だけを並び替えていることになる

  そのため、$\sigma$は$c-1$すなわち$(n-1)$次の置換とみなせるため、帰納法の仮定より、互換の積として書ける

  \br

  $c \neq n$の場合、$\sigma(c)$を$d$とし、$d$と$c$を交換する互換$\tau = (cd)$を考える

  このとき、$\tau\sigma$は、$\sigma$の数字$c$と$d$を交換したものであるので、
  \begin{equation*}
    \tau\sigma = \begin{pmatrix}
      1 & 2 & \cdots & c-1 & c         & \cdots & n \\
      1 & 2 & \cdots & c-1 & \sigma(c) & \cdots & n
    \end{pmatrix}
  \end{equation*}
  $c$が$n$に一致しないという仮定をふまえると、
  \begin{equation*}
    \tau\sigma(n) = n
  \end{equation*}
  であることが読み取れる

  よって、$\tau\sigma$は実質的に$(n-1)$次の置換とみなせるので、帰納法の仮定より、互換の積として書ける
  \begin{equation*}
    \tau\sigma = \tau_1 \tau_2 \cdots \tau_m
  \end{equation*}
  ゆえに、
  \begin{equation*}
    \sigma = \tau^{-1}\tau_1 \tau_2 \cdots \tau_m
  \end{equation*}
  であるが、互換の逆置換は自分自身であるので、
  \begin{equation*}
    \sigma = \tau \tau_1 \tau_2 \cdots \tau_m
  \end{equation*}
  と書ける $\qed$
\end{proof}

\sectionline
\section{置換の符号と偶奇}
\marginnote{\refbookA p177〜179、p158〜159 \\ \refbookF p103}

すべての置換は互換の積に分解できるが、その方法は一通りではない

しかし、互換の積の個数の偶奇性は、置換が与えられれば定まる

\br

このことを証明するために、置換と多項式の関係を考察する

\subsection{置換の多項式への作用}

置換$\sigma \in S_n$と$n$変数多項式$f = f(x_1, x_2, \ldots, x_n)$が与えられたとき、変数$x_i$に$x_{\sigma(i)}$を代入することにより、式$\sigma f$を
\begin{equation*}
  (\sigma f)(x_1, \ldots, x_n) = f(x_{\sigma(1)}, \ldots, x_{\sigma(n)})
\end{equation*}
と定める

\begin{theorem}{置換作用の結合法則}{permutation-action-associativity}
  $f = f(x_1,\ldots,x_n)$を$n$変数の多項式とし、$\sigma, \tau \in S_n$とするとき、
  \begin{equation*}
    (\sigma \tau) f = \sigma (\tau f)
  \end{equation*}
\end{theorem}

\begin{proof}
  式$\tau f$は、
  \begin{equation*}
    (\tau f)(x_1, \ldots, x_n) = f(x_{\tau(1)}, \ldots, x_{\tau(n)})
  \end{equation*}
  である

  さらに$\sigma$を作用させると、$x_{\tau(i)}$は$x_{\sigma(\tau(i))} = x_{(\sigma\tau)(i)}$に置き換わるので、
  \begin{align*}
    (\sigma(\tau f)) & = f(x_{(\sigma\tau)(1)}, \ldots, x_{(\sigma\tau)(n)}) \\
                     & = ((\sigma\tau) f)(x_1, \ldots, x_n)
  \end{align*}
  が成り立つ $\qed$
\end{proof}

\subsection{互換の差積への作用}

次のような$n$変数の多項式を\keyword{差積}と呼ぶ

\begin{equation*}
  \begin{NiceArray}{llll}
    (x_1 - x_2) & (x_1 - x_3) & \cdots & (x_1 - x_n)     \\
                & (x_2 - x_3) & \cdots & (x_2 - x_n)     \\
                &             & \ddots &                 \\
                &             &        & (x_{n-1} - x_n) \\
  \end{NiceArray}
\end{equation*}

\begin{definition*}{差積}
  次のような$n$変数の多項式を\keyword{差積}と呼ぶ
  \begin{equation*}
    \Delta_n = \prod_{1 \leq i < j \leq n} (x_i - x_j)
  \end{equation*}
\end{definition*}

置換の符号を理解するために、差積を使うことができる

その第一歩となるのが、次の定理である

\begin{theorem}{互換による差積の符号変化}{transposition-negates-delta}
  $\tau$を互換とするとき、
  \begin{equation*}
    \tau \Delta_n = - \Delta_n
  \end{equation*}
\end{theorem}

\begin{proof}
  $i<j$として、$\tau = (ij)$とすると、各因子$x_s - x_t \, (1 \leq s < t \leq n)$の変化は次のようになる

  \begin{subpattern}{\bfseries $x_i - x_j$は$x_j - x_i$になる}
    $x_i$と$x_j$を入れ替えることで、その差が逆転して符号が反転する
    \begin{equation*}
      x_j - x_i = -(x_i - x_j)
    \end{equation*}
    よって、この項は$-1$倍の効果をもたらす
  \end{subpattern}

  \begin{subpattern}{\bfseries $s < i < j$のとき、$x_s - x_i$と$x_s - x_j$が入れ替わる}
    この場合、$s$は$i,\, j$より前の添字である

    \begin{itemize}
      \item 互換前: $(x_s - x_i)(x_s - x_j)$
      \item 互換後: $(x_s - x_j)(x_s - x_i)$
    \end{itemize}

    2つの項が交換されるだけなので、積の絶対値は変わらず、符号にも影響しない
  \end{subpattern}

  \begin{subpattern}{\bfseries $i < j < s$のとき、$x_i - x_s$と$x_j - x_s$が入れ替わる}
    この場合、$s$ は $i,\, j$ より後の添字である

    \begin{itemize}
      \item 互換前: $(x_i - x_s)(x_j - x_s)$
      \item 互換後: $(x_j - x_s)(x_i - x_s)$
    \end{itemize}

    この場合も、並び順だけが入れ替わり、符号には影響しない
  \end{subpattern}

  \begin{subpattern}{\bfseries $i < s < j$のとき、$x_i - x_s$と$x_s - x_j$は…}
    この場合、$s$は$i$と$j$の間にある添字である

    \begin{itemize}
      \item 互換前: $(x_i - x_s)(x_s - x_j)$
      \item 互換後: $(x_j - x_s)(x_s - x_i)$
    \end{itemize}

    互換前の積を変形してみると、
    \begin{align*}
      (x_i - x_s)(x_s - x_j) & = -(x_i - x_s)(x_j - x_s) \\
                             & = (x_s - x_i)(x_j - x_s)  \\
                             & = (x_j - x_s)(x_s - x_i)
    \end{align*}
    という形で、互換後の積が得られる

    よって、この場合も積の符号は変わらない
  \end{subpattern}

  以上をふまえると、符号が反転するのは$x_i - x_j$の項だけである

  よって、1回の互換$(ij)$によって、差積全体は$(-1)$倍される $\qed$
\end{proof}

\subsection{置換の符号}

\begin{theorem*}{置換による差積の符号変化}
  置換$\sigma \in S_n$が$s$個の互換の積として書けるならば、
  \begin{equation*}
    \sigma \Delta_n = (-1)^s \Delta_n
  \end{equation*}
  が成り立つ
\end{theorem*}

\begin{proof}
  置換$\sigma$を$s$個の互換の積$\sigma = \tau_1 \cdots \tau_s$と書いたとき、
  \begin{equation*}
    \sigma \Delta_n = (\tau_1 \cdots \tau_s) \Delta_n
  \end{equation*}
  \thmref{thm:permutation-action-associativity}を用いて、
  \begin{equation*}
    \sigma \Delta_n = (\tau_1 \cdots \tau_{s-1}) (\tau_s \Delta_n)
  \end{equation*}
  \thmref{thm:transposition-negates-delta}を繰り返し用いると、
  \begin{align*}
    \sigma \Delta_n & = (\tau_1 \cdots \tau_{s-1}) (-\Delta_n)  \\
                    & = (-1)(\tau_1 \cdots \tau_{s-1}) \Delta_n \\
                    & = (-1)^s \Delta_n
  \end{align*}
  が最終的に得られる $\qed$
\end{proof}

この定理における$\sigma\Delta_n$は、$\sigma$をどのような互換の積として表すかとは無関係に、$\sigma$が与えられれば決まる多項式である

\br

そして、$(-1)^s$という部分から、$\sigma$を互換の積で表したとき、その個数$s$が偶数であれば符号は$+$に、奇数であれば符号は$-$になることがわかる

\br

このようにして、次の定理が示されたことになる

\begin{theorem*}{置換の符号の存在}
  置換$\sigma$を互換の積として書くとき、用いられる互換の個数の偶奇は$\sigma$のみによって決まる
\end{theorem*}

そこで、置換の符号を次のように定義する

\begin{definition*}{置換の符号}
  置換$\sigma \in S_n$を互換の積$\sigma = \tau_1 \cdots \tau_i$として書いたとき、$\sigma$の\keyword{符号}を
  \begin{equation*}
    \sgn(\sigma) = (-1)^i
  \end{equation*}
  と定義する
\end{definition*}

そして、互換の個数の偶奇をそのまま、置換の偶奇として定める

\begin{definition*}{偶置換と奇置換}
  置換$\sigma \in S_n$の符号$\sgn(\sigma)$が$+1$であれば$\sigma$を\keyword{偶置換}と呼び、$-1$であれば\keyword{奇置換}と呼ぶ
\end{definition*}

\sectionline
\section{置換の性質}
\marginnote{\refbookA p157、159}

\begin{theorem}{逆置換の符号}{sign-of-inverse-permutation}
  \begin{equation*}
    \sgn(\sigma^{-1}) = \sgn(\sigma)
  \end{equation*}
\end{theorem}

\begin{proof}
  置換$\sigma$を互換の積として書くと、逆置換はその互換の順序を逆にしたものになる

  すなわち、$\sigma = \tau_1 \cdots \tau_s$とすると、
  \begin{equation*}
    \sigma^{-1} = \tau_s^{-1} \cdots \tau_1^{-1}
  \end{equation*}
  であるが、互換の逆置換は自分自身であるので、
  \begin{equation*}
    \sgn(\sigma^{-1}) = (-1)^s = \sgn(\sigma)
  \end{equation*}
  が成り立つ $\qed$
\end{proof}

\sectionline

\begin{theorem}{置換の符号の乗法性}{sign-multiplicativity}
  \begin{equation*}
    \sgn(\sigma \tau) = \sgn(\sigma) \sgn(\tau)
  \end{equation*}
\end{theorem}

\begin{proof}
  それぞれを互換の積$\sigma = \tau_1 \cdots \tau_i$、$\tau = \rho_1 \cdots \rho_j$と書くと、
  \begin{equation*}
    \sigma \tau = \tau_1 \cdots \tau_i \rho_1 \cdots \rho_j
  \end{equation*}
  である

  このとき、$\sgn(\sigma) = (-1)^i, \, \sgn(\tau) = (-1)^j$なので、
  \begin{equation*}
    \sgn(\sigma \tau) = (-1)^{i+j} = (-1)^i (-1)^j = \sgn(\sigma) \sgn(\tau)
  \end{equation*}
  が成り立つ $\qed$
\end{proof}

\sectionline

\begin{theorem}{置換群の左右作用に対する和の不変性}{sum-invariance-under-permutation-action}
  $f$を$S_n$上の関数とするとき、任意の$\tau \in S_n$に対して、次が成り立つ
  \begin{equation*}
    \sum_{\sigma \in S_n} f(\tau\sigma) = \sum_{\sigma \in S_n} f(\sigma) = \sum_{\sigma \in S_n} f(\sigma\tau)
  \end{equation*}
\end{theorem}

\begin{proof}
  $\tau$を固定して、$\sigma$をすべての置換($S_n$の元)全体にわたって動かすとき、$\tau\sigma$も$S_n$の全体を動く

  言い換えると、写像$S_n \to S_n$を$\sigma \longmapsto \tau\sigma$と定めると、これは全単射である

  したがって、
  \begin{equation*}
    \sum_{\sigma \in S_n} f(\sigma) = \sum_{\sigma \in S_n} f(\tau\sigma)
  \end{equation*}
  が成り立つ

  \br

  同様に、写像$S_n \to S_n$を$\sigma \longmapsto \sigma\tau$と定めると、これも全単射であるので、同様に、
  \begin{equation*}
    \sum_{\sigma \in S_n} f(\sigma) = \sum_{\sigma \in S_n} f(\sigma\tau)
  \end{equation*}
  が成り立つことがわかる $\qed$
\end{proof}

\end{document}
