\documentclass[../../../topic_linear-algebra]{subfiles}

\begin{document}

\sectionline
\section{行列式の特徴づけ}
\marginnote{\refbookA p162〜163 \\ \refbookF p123〜127}

$n$個の与えられた$n$次実ベクトル$\vb*{a}_1, \ldots, \vb*{a}_n$に対して、ある実数が定まるとき、これを$F(\vb*{a}_1, \ldots, \vb*{a}_n)$と表すことにする

\begin{theorem}{多重線形性と交代性による行列式の特徴づけ}{determinant-characterization-by-properties}
  写像$F\colon \mathbb{R}^n \times \cdots \times \mathbb{R}^n \to \mathbb{R}$が多重線形性と交代性を満たすならば、
  \begin{equation*}
    F(\vb*{a}_1, \ldots, \vb*{a}_n) = F(\vb*{e}_1, \ldots, \vb*{e}_n) \det(\vb*{a}_1, \ldots, \vb*{a}_n)
  \end{equation*}
\end{theorem}

\begin{proof}
  多重線形性により、
  \begin{align*}
    F(\vb*{a}_1, \ldots, \vb*{a}_n) & = F \left( \sum_{i=1}^n a_{i_1 1} \vb*{e}_{i_1}, \ldots, \sum_{i=1}^n a_{i_n n} \vb*{e}_{i_n} \right) \\
                                    & = \sum_{i_1, \ldots, i_n} a_{i_1 1} \cdots a_{i_n n} F(\vb*{e}_{i_1}, \ldots, \vb*{e}_{i_n})
  \end{align*}

  和において、各$i_k \, (1 \leq k \leq n)$は行番号なのでそれぞれ1から$n$まで動く

  \br

  ここで、\hyperref[thm:det-zero-if-columns-repeat]{交代性から導かれる定理}より、$(i_1, \ldots, i_n)$に同じ添字が2つ以上ある場合には$F(\vb*{e}_{i_1}, \ldots, \vb*{e}_{i_n}) = 0$である

  したがって、この和は$(i_1, \ldots, i_n)$がすべて異なる場合、すなわち$(i_1, \ldots, i_n)$が$(1, \ldots, n)$の置換である場合にのみ寄与する

  \br

  よって、$(i_1, \ldots, i_n)$にわたる和は、実際には$n$次の置換
  \begin{equation*}
    \sigma = \begin{pmatrix}
      1   & 2   & \cdots & n   \\
      i_1 & i_2 & \cdots & i_n
    \end{pmatrix} \in S_n
  \end{equation*}
  にわたる和であるとみなせる

  \br

  この対応により、$(i_1, \ldots, i_n)$と$\sigma \in S_n$を同一視すると、
  \begin{equation*}
    F(\vb*{e}_{i_1}, \ldots, \vb*{e}_{i_n}) = F(\vb*{e}_{\sigma(1)}, \ldots, \vb*{e}_{\sigma(n)})
  \end{equation*}

  さらに、$(\vb*{e}_{\sigma(1)}, \ldots, \vb*{e}_{\sigma(n)})$を$(\vb*{e}_1, \ldots, \vb*{e}_n)$に並び替えることを考える

  すなわち、$\sigma$の逆置換$\sigma^{-1}$を考えることになる

  交代性によって、1回の互換につき$(-1)$倍されるが、全体の符号は互換の回数によって定まるので、$\sgn(\sigma^{-1}) = \sgn(\sigma)$となる
  \begin{equation*}
    F(\vb*{e}_{\sigma(1)}, \ldots, \vb*{e}_{\sigma(n)}) = \sgn(\sigma) F(\vb*{e}_1, \ldots, \vb*{e}_n)
  \end{equation*}

  \br

  以上より、
  \begin{align*}
     & \phantom{=} F(\vb*{a}_1, \ldots, \vb*{a}_n)                                                                                \\
     & = \sum_{\sigma \in S_n} a_{\sigma(1) 1} \cdots a_{\sigma(n) n} F(\vb*{e}_{\sigma(1)}, \ldots, \vb*{e}_{\sigma(n)})         \\
     & = \sum_{\sigma \in S_n} a_{\sigma(1) 1} \cdots a_{\sigma(n) n} \sgn(\sigma) F(\vb*{e}_1, \ldots, \vb*{e}_n)                \\
     & = \left( \sum_{\sigma \in S_n} \sgn(\sigma) a_{\sigma(1) 1} \cdots a_{\sigma(n) n} \right) F(\vb*{e}_1, \ldots, \vb*{e}_n) \\
     & = \det(\vb*{a}_1, \ldots, \vb*{a}_n) F(\vb*{e}_1, \ldots, \vb*{e}_n)
  \end{align*}
  となり、目的の等式が示された $\qed$
\end{proof}

ここで、$F(\vb*{e}_1, \ldots, \vb*{e}_n) = 1$であれば、
\begin{equation*}
  F(\vb*{a}_1, \ldots, \vb*{a}_n) = \det(\vb*{a}_1, \ldots, \vb*{a}_n)
\end{equation*}
と表せることになる

\br

この$F(\vb*{e}_1, \ldots, \vb*{e}_n) = 1$を\keyword{正規化}の条件といい、行列式は
\begin{enumerate}[label=\romanlabel]
  \item 双線形性
  \item 交代性
  \item 正規化の条件
\end{enumerate}
によって特徴づけられる

\br

すなわち、行列式は、この3つの条件を満たすような
\begin{shaded}
  $n$個の列ベクトル$\vb*{a}_1, \ldots, \vb*{a}_n$で定まる関数
\end{shaded}
として定義することもできる

\end{document}
