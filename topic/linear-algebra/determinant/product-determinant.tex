\documentclass[../../../topic_linear-algebra]{subfiles}

\begin{document}

\sectionline
\section{行列の積と行列式}
\marginnote{\refbookA p164 \\ \refbookF p131〜132}

行列式の特徴づけから導ける性質として、次が重要である

\begin{theorem}{行列式の乗法性}{determinant-multiplicativity}
  $A,\,B$を同じ型の行列とするとき、
  \begin{equation*}
    \det(AB) = \det(A) \det(B)
  \end{equation*}
\end{theorem}

\begin{proof}
  $B$の列ベクトルを$\vb*{b}_1, \ldots, \vb*{b}_n$とし、次の関数
  \begin{equation*}
    F(\vb*{b}_1, \ldots, \vb*{b}_n) = \det(A\vb*{b}_1, \ldots, A\vb*{b}_n)
  \end{equation*}
  を考える

  ここで、$\det$は列ベクトルに対して交代性をもつため、この関数$F$も交代性をもつ

  また、$\det$の多重線形性に加え、$A$による作用は線形写像であるから、$F$も多重線形性を満たす

  \br

  よって、\hyperref[thm:determinant-characterization-by-properties]{多重線形性と交代性による行列式の特徴づけ}より、
  \begin{equation*}
    F(\vb*{b}_1, \ldots, \vb*{b}_n) = F(\vb*{e}_1, \ldots, \vb*{e}_n) \det(B)
  \end{equation*}

  \br

  一方、$F$の引数を単位ベクトル$\vb*{e}_1, \ldots, \vb*{e}_n$にしたもの
  \begin{equation*}
    F(\vb*{e}_1, \ldots, \vb*{e}_n) = \det(A\vb*{e}_1, \ldots, A\vb*{e}_n)
  \end{equation*}
  を考えると、
  \begin{align*}
    F(\vb*{e}_1, \ldots, \vb*{e}_n) & = \det(A\vb*{e}_1, \ldots, A\vb*{e}_n) \\
                                    & = \det(\vb*{a}_1, \ldots, \vb*{a}_n)   \\
                                    & = \det(A)
  \end{align*}

  \br

  よって、
  \begin{equation*}
    F(\vb*{b}_1, \ldots, \vb*{b}_n) = \det(A) \det(B)
  \end{equation*}

  \br

  ここで、$F(\vb*{b}_1, \ldots, \vb*{b}_n)$の定義を思い出すと、
  \begin{equation*}
    \det(A\vb*{b}_1, \ldots, A\vb*{b}_n) = \det(A) \det(B)
  \end{equation*}

  左辺の行列$(A\vb*{b}_1, \ldots, A\vb*{b}_n)$は、行列$B$の各列ベクトルに対して$A$を左から作用させたものであり、行列$AB$を意味している

  \br

  したがって、
  \begin{equation*}
    \det(AB) = \det(A) \det(B)
  \end{equation*}
  が成り立つ $\qed$
\end{proof}

\sectionline

行列式の乗法性を繰り返し適用することで、次の定理が得られる

\begin{theorem*}{累乗行列の行列式と行列式の累乗}
  $A,\,B$を正方行列とするとき、任意の自然数$n$について、
  \begin{equation*}
    \det(A^n) = \det(A)^n
  \end{equation*}
\end{theorem*}

\end{document}
