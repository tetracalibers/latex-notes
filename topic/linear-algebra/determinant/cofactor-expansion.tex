\documentclass[../../../topic_linear-algebra]{subfiles}

\usepackage{xr-hyper}
\externaldocument{../../../.tex_intermediates/topic_linear-algebra}

\begin{document}

\sectionline
\section{余因子展開}
\marginnote{\refbookA p142〜144、p166〜169 \\ \refbookF p133〜139}

3次正方行列において、第1列を次のようにとらえる
\begin{equation*}
  \begin{pmatrix}
    a_{11} \\
    a_{21} \\
    a_{31}
  \end{pmatrix} = a_{11} \begin{pmatrix}
    1 \\
    0 \\
    0
  \end{pmatrix} + a_{21} \begin{pmatrix}
    0 \\
    1 \\
    0
  \end{pmatrix} + a_{31} \begin{pmatrix}
    0 \\
    0 \\
    1
  \end{pmatrix}
\end{equation*}

これをふまえて、3次行列式を、第1列に関する線形性を用いて、次のような和に分解してみる
\begin{align*}
   & \phantom{=\,\,} \left| \begin{matrix}
                              a_{11} & a_{12} & a_{13} \\
                              a_{21} & a_{22} & a_{23} \\
                              a_{31} & a_{32} & a_{33}
                            \end{matrix} \right|                                                                \\
   & = \left| \begin{matrix}
                a_{11} & a_{12} & a_{13} \\
                0      & a_{22} & a_{23} \\
                0      & a_{32} & a_{33}
              \end{matrix} \right| + \left| \begin{matrix}
                                              0      & a_{12} & a_{13} \\
                                              a_{21} & a_{22} & a_{23} \\
                                              0      & a_{32} & a_{33}
                                            \end{matrix} \right| + \left| \begin{matrix}
                                                                            0      & a_{12} & a_{13} \\
                                                                            0      & a_{22} & a_{23} \\
                                                                            a_{31} & a_{32} & a_{33}
                                                                          \end{matrix} \right|                  \\
   & = a_{11} \left| \begin{matrix}
                       1 & a_{12} & a_{13} \\
                       0 & a_{22} & a_{23} \\
                       0 & a_{32} & a_{33}
                     \end{matrix} \right| + a_{21} \left| \begin{matrix}
                                                            0 & a_{12} & a_{13} \\
                                                            1 & a_{22} & a_{23} \\
                                                            0 & a_{32} & a_{33}
                                                          \end{matrix} \right| + a_{31} \left| \begin{matrix}
                                                                                                 0 & a_{12} & a_{13} \\
                                                                                                 0 & a_{22} & a_{23} \\
                                                                                                 1 & a_{32} & a_{33}
                                                                                               \end{matrix} \right|
\end{align*}

\br

ここで、たとえば、
\begin{equation*}
  \left| \begin{matrix}
    1 & a_{12} & a_{13} \\
    0 & a_{22} & a_{23} \\
    0 & a_{32} & a_{33}
  \end{matrix} \right|
\end{equation*}
をどのように表せるかを考える

\br

まず、$(1,1)$成分を要にして第1行の掃き出しを行えば、
\begin{equation*}
  \left| \begin{matrix}
    1 & a_{12} & a_{13} \\
    0 & a_{22} & a_{23} \\
    0 & a_{32} & a_{33}
  \end{matrix} \right| = \left| \begin{matrix}
    1 & 0      & 0      \\
    0 & a_{22} & a_{23} \\
    0 & a_{32} & a_{33}
  \end{matrix} \right|
\end{equation*}
が得られる

\br

そこで、
\begin{equation*}
  \vb*{u}_1 = \begin{pmatrix}
    a_{22} \\
    a_{32}
  \end{pmatrix},\quad
  \vb*{u}_2 = \begin{pmatrix}
    a_{23} \\
    a_{33}
  \end{pmatrix}
\end{equation*}
とおき、
\begin{equation*}
  F(\vb*{u}_1, \vb*{u}_2) = \left| \begin{matrix}
    1 & 0      & 0      \\
    0 & a_{22} & a_{23} \\
    0 & a_{32} & a_{33}
  \end{matrix} \right| = F(\vb*{e}_1, \vb*{e}_2) \cdot \left| \begin{matrix}
    a_{22} & a_{23} \\
    a_{32} & a_{33}
  \end{matrix} \right|
\end{equation*}
とみなす

\br

ここで、
\begin{equation*}
  F(\vb*{e}_1, \vb*{e}_2) = \left| \begin{matrix}
    1 & 0 & 0 \\
    0 & 1 & 0 \\
    0 & 0 & 1
  \end{matrix} \right| = 1
\end{equation*}
であるから、結局、
\begin{equation*}
  \left| \begin{matrix}
    1 & a_{12} & a_{13} \\
    0 & a_{22} & a_{23} \\
    0 & a_{32} & a_{33}
  \end{matrix} \right| = \left| \begin{matrix}
    a_{22} & a_{23} \\
    a_{32} & a_{33}
  \end{matrix} \right|
\end{equation*}
が得られる

\br

2項めの行列式も同様に、掃き出し法によって、
\begin{equation*}
  \left| \begin{matrix}
    0 & a_{12} & a_{13} \\
    1 & a_{22} & a_{23} \\
    0 & a_{32} & a_{33}
  \end{matrix} \right| = \left| \begin{matrix}
    0 & a_{12} & a_{13} \\
    1 & 0      & 0      \\
    0 & a_{32} & a_{33}
  \end{matrix} \right|
\end{equation*}
これを、
\begin{equation*}
  \vb*{u}_1 = \begin{pmatrix}
    a_{12} \\
    a_{32}
  \end{pmatrix},\quad
  \vb*{u}_2 = \begin{pmatrix}
    a_{13} \\
    a_{33}
  \end{pmatrix}
\end{equation*}
の関数$F(\vb*{u}_1, \vb*{u}_2)$とみなす

\br

交代性より、
\begin{multline*}
  F(\vb*{e}_1,\vb*{e}_2) = \left| \begin{matrix}
    0 & 1 & 0 \\
    1 & 0 & 0 \\
    0 & 0 & 1
  \end{matrix} \right| = \det(\vb*{e}_2, \vb*{e}_1, \vb*{e}_3) \\
  = -\det(\vb*{e}_1, \vb*{e}_2, \vb*{e}_3) = -1
\end{multline*}
なので、
\begin{equation*}
  \left| \begin{matrix}
    0 & a_{12} & a_{13} \\
    1 & a_{22} & a_{23} \\
    0 & a_{32} & a_{33}
  \end{matrix} \right| = -\left| \begin{matrix}
    a_{12} & a_{13} \\
    a_{32} & a_{33}
  \end{matrix} \right|
\end{equation*}

\br

最後の項の行列式も同様にして、
\begin{equation*}
  \left| \begin{matrix}
    0 & a_{12} & a_{13} \\
    0 & a_{22} & a_{23} \\
    1 & a_{32} & a_{33}
  \end{matrix} \right| = \left| \begin{matrix}
    0 & a_{12} & a_{13} \\
    0 & 0      & 0      \\
    1 & a_{32} & a_{33}
  \end{matrix} \right|= \left| \begin{matrix}
    a_{12} & a_{13} \\
    a_{32} & a_{33}
  \end{matrix} \right|
\end{equation*}
と表せる

\br

以上より、3次行列式は、次のような2次行列式の和に分解できる
\begin{multline*}
  \left| \begin{matrix}
    a_{11} & a_{12} & a_{13} \\
    a_{21} & a_{22} & a_{23} \\
    a_{31} & a_{32} & a_{33}
  \end{matrix} \right| \\ = a_{11} \left| \begin{matrix}
    a_{22} & a_{23} \\
    a_{32} & a_{33}
  \end{matrix} \right| - a_{21} \left| \begin{matrix}
    a_{12} & a_{13} \\
    a_{32} & a_{33}
  \end{matrix} \right| + a_{31} \left| \begin{matrix}
    a_{12} & a_{13} \\
    a_{32} & a_{33}
  \end{matrix} \right|
\end{multline*}

\sectionline

このような行列式の展開を一般化したものが、\keyword{余因子展開}である

\begin{definition}{余因子}
  $n$次正方行列$A = (a_{ij})$から、第$i$行と第$j$列を取り除いて$(n-1)$次の正方行列$\Delta_{ij}$を作り、その行列式に符号$(-1)^{i+j}$をかけたものを、$A$の$(i,j)$\keyword{余因子}と呼び、$\tilde{a}_{ij}$と書く
  \begin{equation*}
    \tilde{a}_{ij} = (-1)^{i+j} \det(\Delta_{ij})
  \end{equation*}
\end{definition}

\begin{theorem*}{余因子展開}
  $\det(A)$は次のように\keyword{余因子展開}できる
  \begin{description}
    \item[第$j$列に関する展開]~\\ $\det(A) = \tilde{a}_{1j} a_{1j} + \tilde{a}_{2j} a_{2j} + \cdots + \tilde{a}_{nj} a_{nj}$
    \item[第$i$行に関する展開]~\\ $\det(A) = \tilde{a}_{i1} a_{i1} + \tilde{a}_{i2} a_{i2} + \cdots + \tilde{a}_{in} a_{in}$
  \end{description}
\end{theorem*}

\begin{proof}
  列に関する展開だけを示せば、行の方は\thmref{thm:determinant-transpose-invariance}よりしたがう

  \br

  行列$A$を$A = (\vb*{a}_1, \ldots, \vb*{a}_n)$のように列ベクトル表示する

  すると、
  \begin{equation*}
    \vb*{a}_j = a_{1j} \vb*{e}_1 + \cdots + a_{nj} \vb*{e}_n
  \end{equation*}
  なので、行列式の多重線形性を用いて、
  \begin{align*}
    \det(A) & = | \vb*{a}_1, \ldots, \vb*{a}_j, \ldots, \vb*{a}_n |                       \\
            & = \sum_{i=1}^{n} | \vb*{a}_1, \ldots, a_{ij} \vb*{e}_i, \ldots, \vb*{a}_n | \\
            & = \sum_{i=1}^{n} a_{ij} | \vb*{a}_1, \ldots, \vb*{e}_i, \ldots, \vb*{a}_n |
  \end{align*}

  $| \vb*{a}_1, \ldots, \vb*{e}_i, \ldots, \vb*{a}_n |$に対して、$(i,j)$成分を要にして第$i$行を掃き出す操作を行うと、
  \begin{equation*}
    \left| \begin{matrix}
      a_{11} & \cdots & 0 & \cdots & a_{1n} \\
      \vdots & \ddots & 0 & \ddots & \vdots \\
      a_{i1} & \cdots & 1 & \cdots & a_{in} \\
      \vdots & \ddots & 0 & \ddots & \vdots \\
      a_{n1} & \cdots & 0 & \cdots & a_{nn}
    \end{matrix} \right| = \left| \begin{matrix}
      a_{11} & \cdots & 0 & \cdots & a_{1n} \\
      \vdots & \ddots & 0 & \ddots & \vdots \\
      0      & \cdots & 1 & \cdots & 0      \\
      \vdots & \ddots & 0 & \ddots & \vdots \\
      a_{n1} & \cdots & 0 & \cdots & a_{nn}
    \end{matrix} \right|
  \end{equation*}

  \br

  さらに、$i$行目を1つ上の行と順に交換して1行目まで移動し、次に$j$列目を1つ左の列と順に交換して1列目まで移動する

  \br

  行や列の交換から生じる符号の変化は、$(i-1)+(j-1)$の交換を行っているので、$(-1)^{i+j-2} = (-1)^2(-1)^{i+j} = (-1)^{i+j}$となる

  \br

  よって、次のような形が得られる
  \begin{align*}
    | \vb*{a}_1, \ldots, \vb*{e}_i, \ldots, \vb*{a}_n | & =  (-1)^{i+j} \left| \begin{matrix}
                                                                                 1      & 0      & \cdots & 0      & 0      \\
                                                                                 0      & a_{11} & \cdots & \cdots & a_{1n} \\
                                                                                 \vdots & \vdots & \ddots & \ddots & \vdots \\
                                                                                 \vdots & \vdots & \ddots & \ddots & \vdots \\
                                                                                 0      & a_{n1} & \cdots & \cdots & a_{nn}
                                                                               \end{matrix} \right| \\
                                                        & = (-1)^{i+j} \left| \begin{matrix}
                                                                                a_{11} & \cdots & \cdots & a_{1n} \\
                                                                                \vdots & \ddots & \ddots & \vdots \\
                                                                                \vdots & \ddots & \ddots & \vdots \\
                                                                                a_{n1} & \cdots & \cdots & a_{nn}
                                                                              \end{matrix} \right|
  \end{align*}

  ここで現れる行列式は、第1行・第1列に移動させた第$i$行・第$j$列を取り除いた$(n-1)$次正方行列の行列式である

  よって、符号の部分も合わせて、余因子の定義より、次のように書ける
  \begin{equation*}
    | \vb*{a}_1, \ldots, \vb*{e}_i, \ldots, \vb*{a}_n | = \tilde{a}_{ij}
  \end{equation*}

  したがって、行列$A$の行列式は、
  \begin{equation*}
    \det(A) = \sum_{i=1}^{n} a_{ij} \tilde{a}_{ij}
  \end{equation*}
  と書けることが示された $\qed$
\end{proof}

\end{document}
