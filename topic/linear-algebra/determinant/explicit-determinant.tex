\documentclass[../../../topic_linear-algebra]{subfiles}

\usepackage{xr-hyper}
\externaldocument{../../../.tex_intermediates/topic_linear-algebra}

\begin{document}

\sectionline
\section{行列式の定義}
\marginnote{\refbookA p159 \\ \refbookF p107〜108}

ある正方行列の\keyword{行列式}は、
\begin{enumerate}
  \item 各列から1つずつ、行に重複がないように成分を選ぶ
  \item それらをかけ合わせる
  \item 符号をつけて足す
\end{enumerate}
という手順で定まる値である

\begin{definition}{行列式}
  $n$次正方行列$A = (a_{ij})$に対して、
  \begin{equation*}
    \sum_{\sigma \in S_n} \sgn(\sigma) \prod_{i=1}^n a_{i,\sigma(i)}
  \end{equation*}
  で定められる値を$A$の\keyword{行列式}と呼び、$|A|$あるいは$\det(A)$と表記する
\end{definition}

\sectionline
\section{三角行列の行列式}
\marginnote{\refbookF p111〜112 \\ \refbookA p160}

\keyword{三角行列}の場合、各列から1つずつ、0でない成分を重複なく選び出す方法は、対角成分をすべて選ぶしかない

\begin{theorem}{三角行列の行列式}{det-of-triangular-matrix}
  三角行列の行列式は、対角成分の積である
  \begin{equation*}
    \left|\begin{matrix}
      a_{11}           & a_{12} & \dots  & a_{1n} \\
                       & a_{22} & \dots  & a_{2n} \\
                       &        & \ddots & \vdots \\
      \text{\Large{0}} &        &        & a_{nn}
    \end{matrix}\right| = a_{11} a_{22} \cdots a_{nn}
  \end{equation*}
  \begin{equation*}
    \left|\begin{matrix}
      a_{11} &        &        & \text{\Large{0}} \\
      a_{21} & a_{22}                             \\
      \vdots &        & \ddots                    \\
      a_{n1} & a_{n2} & \dots  & a_{nn}
    \end{matrix}\right| = a_{11} a_{22} \cdots a_{nn}
  \end{equation*}
\end{theorem}

\begin{proof}
  行列式において、
  \begin{equation*}
    a_{1,\sigma(1)} a_{2,\sigma(2)} \cdots a_{n,\sigma(n)} = 0
  \end{equation*}
  となる項は、和をとったときに消えてしまう

  したがって、
  \begin{equation*}
    a_{1,\sigma(1)} a_{2,\sigma(2)} \cdots a_{n,\sigma(n)} \neq 0
  \end{equation*}
  すなわち
  \begin{equation*}
    a_{1,\sigma(1)} \neq 0, \ldots, a_{n,\sigma(n)} \neq 0
  \end{equation*}
  となるような選び方を考える

  \begin{subpattern}{\bfseries 上三角行列の場合}
    上三角行列の定義より、$i > j$ならば$a_{ij} = 0$である

    $a_{ij} \neq 0$とするには、$i \leq j$でなければならないので、$a_{i,\sigma(i)}$においては、
    \begin{equation*}
      i \leq \sigma(i)
    \end{equation*}
    である必要がある

    そして、この条件を満たす置換は、\thmref{thm:identity-permutation-characterized-by-monotonicity}より、恒等置換しか存在しないので、
    \begin{equation*}
      \sigma(i) = i
    \end{equation*}
    より、$a_{ii}$の積によって行列式の値が構成される

    また、恒等置換は0(偶数)回の互換で構成されるので、各項の符号は正となる $\qed$
  \end{subpattern}

  \begin{subpattern}{\bfseries 下三角行列の場合}
    下三角行列の定義より、$i < j$ならば$a_{ij} = 0$である

    $a_{ij} \neq 0$とするには、$i \geq j$でなければならないので、$a_{i,\sigma(i)}$においては、
    \begin{equation*}
      i \geq \sigma(i)
    \end{equation*}
    である必要がある

    そして、この条件を満たす置換も、\thmref{thm:identity-permutation-characterized-by-monotonicity}より、恒等置換しか存在しないので、上三角行列の場合と同様の結果が得られる $\qed$
  \end{subpattern}
\end{proof}

\sectionline

\keyword{対角行列}は、上三角行列でもあり下三角行列でもあるので、上の定理の特別な場合として次が成り立つ

\begin{theorem*}{対角行列の行列式}
  対角行列の行列式は、対角成分の積である
  \begin{equation*}
    \left|\begin{matrix}
      a_{11}                                                   \\
       & a_{22}           &        & \text{\Large{0}}          \\
       &                  & \ddots                             \\
       & \text{\Large{0}} &        & \ddots                    \\
       &                  &        &                  & a_{nn}
    \end{matrix}\right| = a_{11} a_{22} \cdots a_{nn}
  \end{equation*}
\end{theorem*}

特に、対角成分がすべて1の場合が\keyword{単位行列}である

\begin{theorem*}{単位行列の行列式}
  単位行列の行列式は1である
  \begin{equation*}
    | E | = 1
  \end{equation*}
\end{theorem*}

\end{document}
