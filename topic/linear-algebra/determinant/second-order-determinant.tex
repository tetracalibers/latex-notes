\documentclass[../../../topic_linear-algebra]{subfiles}

\begin{document}

\sectionline
\section{二元連立一次方程式の解の判別式}

連立方程式を解く方法の一つとして、方程式の両辺を足したり引いたりして、文字を消去することで解を求める\keyword{加減法}という手法がある。

\br

次の$x, y$に関する連立一次方程式を、加減法で解く過程を追いかけてみよう。
\begin{equation*}
  \begin{cases}
    a_1 x + b_1 y = p_1 \\
    a_2 x + b_2 y = p_2
  \end{cases}
\end{equation*}

\subsection{$y$を消去して$x$を求める}

$y$を消去するには、次のようにすればよい。
\begin{equation*}
  \text{1つ目の式} \times b_2 - \text{2つ目の式} \times b_1
\end{equation*}

こうすることで、$x$のみに関する方程式が得られる。
\begin{equation*}
  (a_1 b_2 - a_2 b_1) x = p_1 b_2 - p_2 b_1
\end{equation*}

ここで、$x$の係数$a_1 b_2 - a_2 b_1 \neq 0$であれば、
\begin{equation*}
  x = \frac{p_1 b_2 - p_2 b_1}{a_1 b_2 - a_2 b_1}
\end{equation*}
として、解を求められる。

\br

逆に、$a_1 b_2 - a_2 b_1 = 0$であれば、解を一意に求めることはできない。

\br

この意味で、$a_1 b_2 - a_2 b_1$はこの連立方程式の解の\keyword{判別式}のような役割を持っているといえる。
この重要な量を一旦$\Delta_2$とおくことにしよう。
\begin{equation*}
  \Delta_2 = a_1 b_2 - a_2 b_1
\end{equation*}

\subsection{$x$を消去して$y$を求める}

$x$を消去するには、次のようにすればよい。
\begin{equation*}
  \text{1つ目の式} \times a_2 - \text{2つ目の式} \times a_1
\end{equation*}

こうすることで、$y$のみに関する方程式が得られる。
\begin{equation*}
  (a_2 b_1 - a_1 b_2) y = p_2 a_1 - p_1 a_2
\end{equation*}

ここでも、$y$の係数$a_2 b_1 - a_1 b_2$の値によって、解を求められるかどうかが変わってくる。

この$y$の係数は、先ほど導入した$\Delta_2$を用いると、次のように表せる。
\begin{equation*}
  a_2 b_1 - a_1 b_2 = - (a_1 b_2 - a_2 b_1) = - \Delta_2
\end{equation*}

\subsection{$x$を消去して$y$を求める(式を入れ替えた場合)}

$x$を消去する前に、あえて式の順番を入れ替えた場合を考えてみよう。
\begin{equation*}
  \begin{cases}
    a_2 x + b_2 y = p_2 \\
    a_1 x + b_1 y = p_1
  \end{cases}
\end{equation*}

すると、$x$を消去するには、
\begin{equation*}
  \text{1つ目の式} \times a_1 - \text{2つ目の式} \times a_2
\end{equation*}
とすればよいことになる。

\br

これより得られる$y$についての方程式は、次のようになる。
\begin{equation*}
  (a_1 b_2 - a_2 b_1) y = p_1 a_2 - p_2 a_1
\end{equation*}

この場合、$y$の係数は、
\begin{equation*}
  a_1 b_2 - a_2 b_1 = \Delta_2
\end{equation*}
となり、今度はマイナスの符号がつかない$\Delta_2$そのものになっている。

\br

どうやら、
\begin{emphabox}
  \begin{spacebox}
    \begin{center}
      式の順番を入れ替えたら、判別式$\Delta_2$の符号が反転する
    \end{center}
  \end{spacebox}
\end{emphabox}
ようだ。(このことは後の議論への伏線として、頭の片隅に置いておこう。)

\sectionline
\section{係数行列と二次行列式}

先ほどの連立一次方程式を、行列を使って表すと次のようになる。
\begin{equation*}
  \dlinelabelmath[BurntOrange]{\begin{pmatrix}
    a_1 & b_1 \\
    a_2 & b_2
  \end{pmatrix}}{係数行列$A$}
  \begin{pmatrix}
    x \\
    y
  \end{pmatrix}
  =
  \begin{pmatrix}
    p_1 \\
    p_2
  \end{pmatrix}
\end{equation*}

\br

すると、この連立一次方程式の解の判別式として導入した、
\begin{equation*}
  \Delta_2 = a_1 b_2 - a_2 b_1
\end{equation*}
という量は、\keyword[BurntOrange]{係数行列$A$}の成分だけで決まるものだとわかる。

\br

そこで、この$\Delta_2$を2次正方行列$A$の\keyword{行列式}と呼ぶことにし、次のように表す。
\begin{equation*}
  \det(A) = \begin{vmatrix}
    a_1 & b_1 \\
    a_2 & b_2
  \end{vmatrix} = a_1 b_2 - a_2 b_1
\end{equation*}

\br

\begin{definition*}{二次正方行列の行列式}
  2次正方行列$A = (a_{ij})$に対して、$A$の\keyword{行列式}$\det(A)$を次のように定義する。
  \begin{equation*}
    \det(A) = \begin{vmatrix}
      a_{11} & a_{12} \\
      a_{21} & a_{22}
    \end{vmatrix} = a_{11} a_{22} - a_{21} a_{12}
  \end{equation*}
\end{definition*}

\subsection{加減法の操作と二次行列式の覚え方}\label{sec:det2x2-add-subtract}

そもそも$\Delta_2$は、次の連立一次方程式
\begin{equation*}
  \begin{cases}
    a_1 x + b_1 y = p_1 \\
    a_2 x + b_2 y = p_2
  \end{cases}
\end{equation*}
において、
\begin{equation*}
  \text{1つ目の式} \times b_2 - \text{2つ目の式} \times b_1
\end{equation*}
という操作を行うことで現れたものだった。

\br

この加減法の操作をイメージして、2次正方行列の行列式は「係数を交差させるようにかけて引く」と覚えるとよい。
\begin{gather*}
  \begin{vNiceMatrix}[margin]
    \CodeBefore
      \cellcolor{carnationpink!20}{1-1}
      \cellcolor{carnationpink!20}{2-2}
      \cellcolor{SkyBlue!20}{1-2}
      \cellcolor{SkyBlue!20}{2-1}
    \Body
    a_1 & b_1 \\
    a_2 & b_2 \\
    \CodeAfter
      \tikz \draw[line width=2pt, opacity=0.8, Rhodamine, -Straight Barb] (1-1.north west) -- (2-2.south east);
      \tikz \draw[line width=2pt, opacity=0.8, Cerulean, -Straight Barb] (1-2.north east) -- (2-1.south west);
  \end{vNiceMatrix} = \textcolor{Rhodamine}{a_1b_2} - \textcolor{Cerulean}{a_2b_1} \\[1ex]
  \text{\bfseries 係数を交差させるようにかけて引く}
\end{gather*}

\end{document}
