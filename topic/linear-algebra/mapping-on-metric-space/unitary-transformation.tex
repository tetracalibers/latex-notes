\documentclass[../../../topic_linear-algebra]{subfiles}

\begin{document}

\sectionline
\section{ユニタリ変換}
\marginnote{\refbookA p77〜82 \\ \refbookC p126〜131}

体$\mathbb{C}$上の計量空間において、内積を保つ線形変換を\keyword{ユニタリ変換}という

\begin{definition}{ユニタリ変換}
  体$\mathbb{C}$上の計量空間$V$における線形変換$f$が\keyword{ユニタリ変換}であるとは、任意の$\vb*{u},\,\vb*{v} \in V$に対し、
  \begin{equation*}
    (f(\vb*{u}),f(\vb*{v})) = (\vb*{u},\vb*{v})
  \end{equation*}
  が成り立つことである
\end{definition}

体$\mathbb{R}$上のユニタリ変換は、\keyword{直交変換}と呼ばれる

\subsection{ユニタリ変換の表現行列}

ユニタリ行列の性質である\hyperref[thm:unitary-characterized-by-inner-product-invariance]{内積不変性}
\begin{equation*}
  (A\vb*{u}, A\vb*{v}) = (\vb*{u}, \vb*{v})
\end{equation*}
から、ユニタリ変換の表現行列は\keyword{ユニタリ行列}であることがわかる

\begin{theorem*}{ユニタリ変換とユニタリ行列表現}
  計量空間上の線形変換$f$がユニタリ変換であることと、$f$の表現行列$A$がユニタリ行列であることは同値である
\end{theorem*}

このことから、ユニタリ行列の性質は、ユニタリ変換の性質として言い換えることができる

\subsection{ユニタリ変換とノルム}

ユニタリ行列の\hyperref[thm:unitary-characterized-by-norm-invariance]{ノルム不変性}から、
\begin{shaded}
  ユニタリ変換はベクトルの長さを変えない変換
\end{shaded}
でもあることがわかる

\begin{theorem*}{ユニタリ変換とノルム保存性}
  計量空間$V$における線形変換を$f$がユニタリ変換であることと、任意の$\vb*{v} \in V$に対し
  \begin{equation*}
    \|f(\vb*{v})\| = \|\vb*{v}\|
  \end{equation*}
  が成り立つことは同値である
\end{theorem*}

\end{document}
