\documentclass[../../../topic_linear-algebra]{subfiles}

\begin{document}

\sectionline
\section{エルミート変換}
\marginnote{\refbookC p126〜131}

\begin{definition}{エルミート変換}
  体$\mathbb{C}$上の計量空間$V$における線形空間$f$が\keyword{エルミート変換}であるとは、任意の$\vb*{u},\,\vb*{v} \in V$に対し、
  \begin{equation*}
    (f(\vb*{u}), \vb*{v}) = (\vb*{u}, f(\vb*{v}))
  \end{equation*}
  が成り立つことである
\end{definition}

体$\mathbb{R}$上のエルミート変換は、\keyword{対称変換}と呼ばれる

\end{document}
