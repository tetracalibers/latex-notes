\documentclass[../../../topic_linear-algebra]{subfiles}

\begin{document}

\sectionline
\section{線形空間の公理}
\marginnote{\refbookH p158〜163 \\ \refbookC p143〜166 \\ \refbookG p25〜27}

線形代数の理論は線型独立性や線形写像を基礎にしている

これらは線形結合、すなわちベクトルの\keyword{和}と\keyword{スカラー倍}を用いて定義された

任意のベクトルは線形結合で表され、線形写像は線形結合を保つ写像として定義される

\br

そこで、和とスカラー倍が定義された一般の集合に対しても、線型空間の理論を適用できないか?と考える

\br

和とスカラー倍が定義された一般の集合を、改めて\keyword{線形空間}として定義する

そして、その集合の元を\keyword{ベクトル}と呼ぶことにする

和とスカラー倍が定義されていれば、線形結合によりその元を表すことができるからだ

\begin{definition}{線形空間}
  集合$V$の任意の元$\vb*{a},\,\vb*{b}$と体$K$の任意の元$k$に対して、$V$の元$\vb*{a} + \vb*{b}$(\keyword{和})が定まり、$V$の元$k\vb*{a}$(\keyword{スカラー倍})が定まるとする

  これらの演算が次の条件を満たすとき、$V$を$K$上の\keyword{線形空間}、あるいは$K$線型空間と呼び、線型空間の元を\keyword{ベクトル}と呼ぶ

  \begin{enumerate}[label=\romanlabel]
    \item 交換法則:$\vb*{a} + \vb*{b} = \vb*{b} + \vb*{a}$
    \item 結合法則:$(\vb*{a} + \vb*{b}) + \vb*{c} = \vb*{a} + (\vb*{b} + \vb*{c})$、$k(l\vb*{a}) = (kl)\vb*{a}$
    \item 分配法則:$k(\vb*{a} + \vb*{b}) = k\vb*{a} + k\vb*{b}$、$(k + l)\vb*{a} = k\vb*{a} + l\vb*{a}$
    \item $1 \vb*{a} = \vb*{a}$($1$は体$K$の乗法に関する単位元)
    \item 零元の存在:$\vb*{0}$と書かれる特別な元が存在し、任意の$\vb*{a} \in V$に対して$\vb*{a} + \vb*{0} = \vb*{a}$
    \item 和に関する逆元の存在:任意の$\vb*{a} \in V$に対して$-\vb*{a}$と書かれる特別な元が存在し、$\vb*{a} + (-\vb*{a}) = (-\vb*{a}) + \vb*{a} = \vb*{0}$
  \end{enumerate}
\end{definition}

\end{document}
