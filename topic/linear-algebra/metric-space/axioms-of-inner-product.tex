\documentclass[../../../topic_linear-algebra]{subfiles}

\begin{document}

\sectionline
\section{内積:ベクトルの「近さ」を返す関数}

2つのベクトルがどれくらい似ているかを議論するために、\keyword{内積}という尺度を導入しよう。

\br

\keyword{内積}は、2つのベクトルを引数にとり、その「近さ」を表すスカラー値を返す関数として定義する。

\br

具体的な定義式を知る前に、「近さ」を測る道具として、どのような性質を持っていてほしいかを整理しておこう。

具体的な定義式は、その性質を満たすように「作る」ことにする。

\begin{definition}{内積の公理($\mathbb{R}$上の線形空間)}
  $\mathbb{R}$上の線形空間$V$を考え、$\vb*{u}, \vb*{v}, \vb*{w} \in V,\, c \in \mathbb{R} $とする。\\
  2つのベクトルを引数にとり、実数を返す関数$(\cdot, \cdot): V \times V \to \mathbb{R}$として、次の性質を満たすものを\keyword{内積}という。
  \begin{description}
    \item[対称性]~\\
          $(\vb*{u}, \vb*{v}) = (\vb*{v}, \vb*{u})$
    \item[双線形性1.スカラー倍]~\\
          $(c\vb*{u}, \vb*{v}) = (\vb*{u}, c\vb*{v}) = c(\vb*{u}, \vb*{v})$
    \item[双線形性2.和]~\\
          $(\vb*{u} + \vb*{w}, \vb*{v}) = (\vb*{u}, \vb*{v}) + (\vb*{w}, \vb*{v})$\\
          $(\vb*{u}, \vb*{v} + \vb*{w}) = (\vb*{u}, \vb*{v}) + (\vb*{u}, \vb*{w})$
    \item[正定値性]~\\
          $(\vb*{u}, \vb*{u}) \geq 0, (\vb*{u}, \vb*{u}) = 0 \iff \vb*{u} = \vb*{0}$
  \end{description}
\end{definition}

内積が定められた線形空間を、\keyword{計量線形空間}、または単に\keyword{計量空間}という。(計量空間の厳密な定義は後に述べる。)

\br

まずは、ここで示した内積の持つべき性質のそれぞれの意図を考えてみよう。

\subsection{対称性}

$\vb*{u}$が$\vb*{v}$にどれくらい近いか?という視点で測っても、$\vb*{v}$が$\vb*{u}$にどれくらい近いか?という視点で測っても、得られる「近さ」は同じであってほしい、という性質。

\subsection{双線形性}

どちらかのベクトルをスカラー倍してから「近さ」を測りたいとき、元のベクトルとの近さを測っておいて、それを定数倍することでも目的の「近さ」を求められる、という性質。

\br

また、ほかのベクトルを足してから「近さ」を測りたいとき、足し合わせたいベクトルそれぞれについて近さを測っておいて、それを合計することでも目的の「近さ」を求められる、という性質。

\br

これらは、近さを測るという「操作」と「演算」が入れ替え可能であるという、\keyword{線形性}と呼ばれる性質である。

2つの引数$\vb*{u}, \vb*{v}$のどちらに関しても線形性があるということで、「双」がついている。

\subsection{正定値性}

ベクトルの「近さ」とは、向きがどれくらい近いか、という尺度でもある。

同じ方向なら正の数、逆の方向なら負の数をとるのが自然だと考えられる。

\br

自分自身との「近さ」を測るとき、自分と自分は完全に同じ向きであるから、その「近さ」は正の数であるはずだ。

自分自身との「近さ」が$0$になるようなベクトルは、零ベクトル$\vb*{o}$だけである。

\end{document}
