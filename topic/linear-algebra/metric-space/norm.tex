\documentclass[../../../topic_linear-algebra]{subfiles}

\begin{document}

\sectionline
\section{ノルム:ベクトルの「大きさ」}

内積の正定値性より、自分自身との内積は次の性質を持つ。
\begin{itemize}
  \item 常に正の数である
  \item 零ベクトルのときだけ0になる
\end{itemize}

零ベクトルのときだけ0になることから、自分自身との内積は、そのベクトル自身の大きさ(長さ)に応じてスケーリングする量なのではないか?と予想される。
また、大きさを表す量は、当然正の数である必要がある。

\br

このように、内積の正定値性は、自分自身との内積を使ってそのベクトルの「大きさ」を測ることができることを示唆する性質と考えることができる。

\br

ベクトルの「大きさ」を表現する量を\keyword{ノルム}という。

\br

ベクトルの「大きさ」の測り方(ノルムの具体的な定義)はさまざま考えられ、用途によって使い分けられるが、ノルムを名乗るものはどれも次の性質を満たすように作る必要がある。

\begin{definition}{ノルムの公理($\mathbb{R}$上の線形空間)}\label{def:norm-axioms}
  $\mathbb{R}$上の線形空間$V$を考え、$\vb*{u}, \vb*{v} \in V,\, c \in \mathbb{R} $とする。\\
  ベクトルを引数にとり、非負の実数を返す関数$\|\cdot\|: V \to \mathbb{R}$として、次の性質を満たすものを\keyword{ノルム}という。
  \begin{description}
    \item[非負性]~\\
          $\|\vb*{u}\| \geq 0, \|\vb*{u}\| = 0 \iff \vb*{u} = \vb*{0}$
    \item[斉次性]~\\
          $\|c\vb*{u}\| = |c|\|\vb*{u}\|$
    \item[三角不等式]~\\
          $\|\vb*{u} + \vb*{v}\| \leq \|\vb*{u}\| + \|\vb*{v}\|$
  \end{description}
\end{definition}

ノルムの非負性は、内積の正定値性そのものである。

\br

斉次性についてはどうだろうか?

スカラー倍したベクトルの自身との内積を考えてみると、スカラー倍に対する内積の双線形性より、次のようになる。
\begin{equation*}
  (c\vb*{u}, c\vb*{u}) = c^2(\vb*{u}, \vb*{u})
\end{equation*}
これはノルムの斉次性を満たしていない。

\br

自分自身との内積をそのままノルムとして使おうとすると、ベクトルを$c$倍したらその長さは$c^2$倍されるという、不自然な定義になってしまう。

\br

そこで、二乗を消すために平方根をとることで、斉次性も満たす量を得ることができる。
\begin{equation*}
  \sqrt{(c\vb*{u}, c\vb*{u})} = \sqrt{c^2(\vb*{u}, \vb*{u})} = |c|\sqrt{(\vb*{u}, \vb*{u})}
\end{equation*}
このように自身との内積の平方根をとった形を、ベクトル$\vb*{u}$の\keyword{ノルム}と定義することにする。

\begin{definition}{ベクトルのノルム}
  計量空間$V$上のベクトル$\vb*{a}$の\keyword{ノルム}(\keyword{長さ})を次のように定義する。
  \begin{align*}
    \|\vb*{a}\| = \sqrt{(\vb*{a}, \vb*{a})}
  \end{align*}
\end{definition}

\end{document}
