\documentclass[../../../topic_linear-algebra]{subfiles}

\begin{document}

\sectionline
\section{$\mathbb{C}^n$上の内積}

複素数$z=a+bi$に対して、
\begin{equation*}
  (a+bi)(a-bi) = a^2 + b^2 \geq 0
\end{equation*}
という式が成り立つ

このとき、$a-bi$を$z$の\keyword{共役複素数}といい、$\overline{z}$と表記する

また、$\sqrt{a^2 + b^2}$は$z$の\keyword{絶対値}と呼ばれ、$|z|$と表記する

\br

すなわち、冒頭の不等式は、
\begin{equation*}
  |z|^2 = z \overline{z} \geq 0
\end{equation*}
と書き換えられる

\br

このことを利用して、$\mathbb{C}^n$上の内積は、次のように定義すると$\mathbb{R}^n$の場合の自然な拡張になる

\begin{definition}{$\mathbb{C}^n$上の内積(標準内積)}{standard-inner-product-Cn}
  $\vb*{a} = (a_i)_{i=1}^n,\, \vb*{b} = (b_i)_{i=1}^n \in \mathbb{C}^n$に対して、
  \begin{equation*}
    (\vb*{a}, \vb*{b}) = \sum_{i=1}^n a_i \overline{b_i}
  \end{equation*}
  を$\mathbb{C}^n$上の\keyword{内積}と定義する

  この内積は\keyword{標準内積}、あるいは\keyword{標準エルミート内積}とも呼ばれる
\end{definition}

このように定めることで、特に、
\begin{equation*}
  (\vb*{a}, \vb*{a}) = \sum_{i=1}^n a_i \overline{a_i} = \sum_{i=1}^n |a_i|^2 \geq 0
\end{equation*}
であるので、$\mathbb{R}^n$の場合と同様に、ベクトルの\keyword{ノルム}を定義できる

\sectionline

$\mathbb{R}^n$上の内積で成り立つ性質の多くは、$\mathbb{C}^n$上の内積でも成り立つが、対称性に関しては注意が必要である

\begin{theorem}{標準内積の対称性}{standard-inner-product-symmetry}
  $\vb*{u}, \vb*{v} \in \mathbb{C}^n$に対して、次が成り立つ
  \begin{equation*}
    (\vb*{u}, \vb*{v}) = \overline{(\vb*{v}, \vb*{u})}
  \end{equation*}
\end{theorem}

\begin{proof}
  $\overline{\overline{z}} = z$をふまえると、
  \begin{align*}
    \overline{(\vb*{v}, \vb*{u})} & = \overline{\sum_{i=1}^n v_i \overline{u_i}} \\
                                  & = \sum_{i=1}^n \overline{v_i \overline{u_i}} \\
                                  & = \sum_{i=1}^n \overline{v_i} u_i            \\
                                  & = \sum_{i=1}^n u_i \overline{v_i}            \\
                                  & = (\vb*{u}, \vb*{v})
  \end{align*}
  となり、目的の式が示された $\qed$
\end{proof}

複素数$z=a+bi$において、$b=0$の場合、$z$は実数である

このとき、$a+0i = a-0i = a$であるから、$z$が実数の場合、
\begin{equation*}
  \overline{z} = z
\end{equation*}
が成り立つ

\br

よって、$\vb*{u},\,\vb*{v} \in \mathbb{R}^n$であるなら、$\mathbb{C}^n$上の内積の対称性の式は
\begin{equation*}
  (\vb*{u}, \vb*{v}) = \overline{(\vb*{v}, \vb*{u})} = (\vb*{v}, \vb*{u})
\end{equation*}
と書き換えられ、これは$\mathbb{R}^n$上の内積の対称性そのものである

\br

つまり、$\mathbb{C}^n$上の内積の対称性は、$\mathbb{R}^n$上の内積の対称性も含んだ表現になっている

\sectionline
\section{転置による内積の表現}

内積は、転置を用いて表現することもできる

\begin{theorem}{転置による内積の表現}{inner-product-as-transpose-product}
  \begin{equation*}
    (\vb*{a},\vb*{b}) = {}^t \vb*{a} \cdot \overline{\vb*{b}} = \left(a_1,\,a_2,\,\ldots,\,a_n\right) \begin{pmatrix}
      \overline{b_1} \\
      \overline{b_2} \\
      \vdots         \\
      \overline{b_n}
    \end{pmatrix}
  \end{equation*}
\end{theorem}

\end{document}
