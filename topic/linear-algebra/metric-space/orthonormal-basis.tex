\documentclass[../../../topic_linear-algebra]{subfiles}

\begin{document}

\sectionline
\section{直交系と直交基底}
\marginnote{\refbookC p117〜118 \\ \refbookF p181〜182}

\begin{definition}{直交系と直交基底}
  計量空間$V$の$\vb*{0}$でないベクトル$\vb*{a}_1, \vb*{a}_2, \ldots, \vb*{a}_n$がどの2つも互いに直交する、すなわち、
  \begin{equation*}
    (\vb*{a}_i, \vb*{a}_j) = 0 \quad (i \neq j)
  \end{equation*}
  が成り立つとき、$\vb*{a}_1, \vb*{a}_2, \ldots, \vb*{a}_n$を\keyword{直交系}という

  直交系が$V$の基底であるとき、\keyword{直交基底}と呼ばれる
\end{definition}

\subsection{直交系の線型独立性}

\begin{theorem}{直交系の線型独立性}\label{thm:orthogonal-set-is-independent}
  計量空間の直交系$\vb*{a}_1, \vb*{a}_2, \ldots, \vb*{a}_n$は線型独立である
\end{theorem}

\begin{proof}
  係数$c_1, c_2, \ldots, c_n \in K$を用いた線形関係式
  \begin{equation*}
    c_1 \vb*{a}_1 + c_2 \vb*{a}_2 + \cdots + c_n \vb*{a}_n = \vb*{0}
  \end{equation*}
  を考える

  このとき、$\vb*{a}_j \, (j=1,2,\ldots,n)$との内積をとると、
  \begin{equation*}
    (c_1 \vb*{a}_1 + c_2 \vb*{a}_2 + \cdots + c_n \vb*{a}_n, \vb*{a}_j) = 0
  \end{equation*}
  内積の双線形性より、
  \begin{gather*}
    c_1 (\vb*{a}_1, \vb*{a}_j) + c_2 (\vb*{a}_2, \vb*{a}_j) + \cdots + c_n (\vb*{a}_n, \vb*{a}_j) = 0 \\
    \sum_{i=1}^n c_i (\vb*{a}_i, \vb*{a}_j) = 0
  \end{gather*}

  ここで、$\vb*{a}_i$は直交系であることから、$i \neq j$の場合、
  \begin{equation*}
    (\vb*{a}_i, \vb*{a}_j) = 0
  \end{equation*}
  よって、$i \neq j$の項はすべて0になり、残るのは
  \begin{equation*}
    c_j (\vb*{a}_j, \vb*{a}_j) = 0
  \end{equation*}

  ここで、直交系の定義より、$\vb*{a}_j \neq \vb*{0}$なので、
  \begin{equation*}
    (\vb*{a}_j, \vb*{a}_j) \neq 0
  \end{equation*}
  よって、$c_j = 0$でなければならず、これは$\vb*{a}_1, \vb*{a}_2, \ldots, \vb*{a}_n$が線型独立であることを意味する $\qed$
\end{proof}

\sectionline
\section{正規直交系と正規直交基底}
\marginnote{\refbookC p117〜119 \\ \refbookF p181〜182}

\begin{definition}{正規直交系と正規直交基底}
  計量空間$V$の$\vb*{0}$でないベクトル$\vb*{a}_1, \vb*{a}_2, \ldots, \vb*{a}_n$が直交系であり、さらに、どのベクトルもそのノルムが1に等しいとき、$\vb*{a}_1, \vb*{a}_2, \ldots, \vb*{a}_n$を\keyword{正規直交系}という

  正規直交系が$V$の基底であるとき、\keyword{正規直交基底}と呼ばれる
\end{definition}

\subsection{正規直交基底と内積}

正規直交基底どうしの内積は、\keyword{クロネッカーのデルタ}記号を用いて、簡潔に表現できる

\begin{definition}{クロネッカーのデルタ}
  次のように定義される$\delta_{ij}$を\keyword{クロネッカーのデルタ}という
  \begin{equation*}
    \delta_{ij} =
    \begin{cases}
      1 & (i=j)      \\
      0 & (i \neq j)
    \end{cases}
  \end{equation*}
\end{definition}

\begin{theorem}{正規直交基底同士の内積}
  計量空間$V$の正規直交基底$\vb*{e}_1, \vb*{e}_2, \ldots, \vb*{e}_n$の内積に関して、次が成り立つ
  \begin{equation*}
    (\vb*{e}_i, \vb*{e}_j) = \delta_{ij} \quad (i,j=1,2,\ldots,n)
  \end{equation*}
\end{theorem}

\begin{proof}
  $\vb*{e}_1, \vb*{e}_2, \ldots, \vb*{e}_n$の直交性より、$i \neq j$のときは、
  \begin{equation*}
    (\vb*{e}_i, \vb*{e}_j) = 0
  \end{equation*}

  また、$\vb*{e}_1, \vb*{e}_2, \ldots, \vb*{e}_n$はすべてノルムが1であることから、$i=j$のときは、
  \begin{equation*}
    (\vb*{e}_i, \vb*{e}_i) = \|\vb*{e}_i\|^2 = 1
  \end{equation*}

  この場合分けとそれぞれの結果は、クロネッカーのデルタ記号の定義と一致する $\qed$
\end{proof}

\end{document}
