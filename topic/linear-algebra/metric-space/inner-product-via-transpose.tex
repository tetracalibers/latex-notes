\documentclass[../../../topic_linear-algebra]{subfiles}

\begin{document}

\sectionline
\section{転置行列と内積}
\marginnote{\refbookA p78〜79}

内積は、転置を用いて表現することもできる

\begin{theorem}{転置による内積の表現}
  \begin{equation*}
    (\vb*{a},\vb*{b}) = {}^t \vb*{a} \cdot \vb*{b} = \left(a_1,\,a_2,\,\ldots,\,a_n\right) \begin{pmatrix}
      b_1    \\
      b_2    \\
      \vdots \\
      b_n
    \end{pmatrix}
  \end{equation*}
\end{theorem}

\sectionline

転置行列と内積は、次の公式によってうまく関係している

\begin{theorem}{随伴公式}
  $A$を$n$次正方行列とするとき、
  \begin{equation*}
    (A\vb*{u},\vb*{v}) = (\vb*{u},\transpose{A}\vb*{v})
  \end{equation*}
\end{theorem}

\begin{proof}
  転置を用いて内積を書くと、
  \begin{equation*}
    (A\vb*{u},\vb*{v}) = {}^t(A\vb*{u}) \vb*{v}
  \end{equation*}
  \hyperref[thm:transpose-of-product]{転置と行列積の順序反転性}より、${}^t(A\vb*{u}) = {}^t\vb*{u} \transpose{A}$なので、
  \begin{equation*}
    (A\vb*{u},\vb*{v}) = ({}^t\vb*{u} \transpose{A}) \vb*{v}
  \end{equation*}
  行列の積の結合法則を用いて、
  \begin{equation*}
    (A\vb*{u},\vb*{v}) = {}^t\vb*{u} (\transpose{A}\vb*{v})
  \end{equation*}
  右辺を内積として書き直すと、
  \begin{equation*}
    (A\vb*{u},\vb*{v}) = (\vb*{u},\transpose{A}\vb*{v})
  \end{equation*}
  となり、目的の等式が得られる $\qed$
\end{proof}

\end{document}
