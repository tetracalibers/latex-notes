\documentclass[../../../topic_linear-algebra]{subfiles}

\begin{document}

\sectionline
\section{$\mathbb{R}^n$上の内積とノルム}
\marginnote{\refbookA p76}

$\mathbb{R}^n$にはベクトル演算という構造があるわけだが、\keyword{内積}という付加的な構造を定める

\begin{definition}{$\mathbb{R}^n$上の内積}
  $\vb*{a} = (a_i)_{i=1}^n,\, \vb*{b} = (b_i)_{i=1}^n \in \mathbb{R}^n$に対して、
  \begin{equation*}
    (\vb*{a}, \vb*{b}) = \sum_{i=1}^n a_i b_i
  \end{equation*}
  を$\mathbb{R}^n$上の\keyword{内積}と呼ぶ
\end{definition}

\sectionline

特に、
\begin{equation*}
  (\vb*{a}, \vb*{a}) = a_1^2 + a_2^2 + \cdots + a_n^2 \geq 0
\end{equation*}
なので、
\begin{equation*}
  \|\vb*{a}\| \coloneq \sqrt{(\vb*{a}, \vb*{a})} \geq 0
\end{equation*}
が定義できる

\begin{definition}{$\mathbb{R}^n$上のノルム}
  $\mathbb{R}^n$上のベクトル$\vb*{a}$の\keyword{長さ}(\keyword{ノルム})を次のように定義する
  \begin{align*}
    \|\vb*{a}\| = \sqrt{(\vb*{a}, \vb*{a})}
  \end{align*}
\end{definition}

\sectionline
\section{$\mathbb{R}^n$上の内積の性質}
\marginnote{\refbookA p76}

\begin{theorem}{$\mathbb{R}^n$上の内積の双線形性}
  $\vb*{u},\,\vb*{v},\,\vb*{u}_1,\,\vb*{u}_2,\,\vb*{v}_1,\,\vb*{v}_2 \in \mathbb{R}^n,\,c\in \mathbb{R}$に対して、以下が成立する
  \begin{enumerate}[label=\romanlabel]
    \item $(\vb*{u}_1 + \vb*{u}_2, \vb*{v}) = (\vb*{u}_1, \vb*{v}) + (\vb*{u}_2, \vb*{v})$
    \item $(c\vb*{u}, \vb*{v}) = c(\vb*{u}, \vb*{v})$
    \item $(\vb*{u}, \vb*{v}_1 + \vb*{v}_2) = (\vb*{u}, \vb*{v}_1) + (\vb*{u}, \vb*{v}_2)$
    \item $(\vb*{u}, c\vb*{v}) = c(\vb*{u}, \vb*{v})$
  \end{enumerate}
\end{theorem}

\begin{proof}
  行列のかけ算と和に関する分配法則、行列のスカラー倍についての性質から従う $\qed$
\end{proof}

\br

\begin{theorem}{$\mathbb{R}^n$上の内積の対称性}
  $\vb*{u}, \vb*{v} \in \mathbb{R}^n$に対して、次が成り立つ
  \begin{equation*}
    (\vb*{u}, \vb*{v}) = (\vb*{v}, \vb*{u})
  \end{equation*}
\end{theorem}

\begin{proof}
  実数同士の乗算は可換であることから、$\mathbb{R}^n$上の内積の定義により
  \begin{equation*}
    (\vb*{u}, \vb*{v}) = \sum_{i=1}^n u_i v_i = \sum_{i=1}^n v_i u_i = (\vb*{v}, \vb*{u})
  \end{equation*}
  となり、明らかに成り立つ $\qed$
\end{proof}

\br

\begin{theorem}{$\mathbb{R}^n$上の内積の正値性}
  $\vb*{u} \in \mathbb{R}^n$に対して、
  \begin{equation*}
    (\vb*{u}, \vb*{u}) \geq 0
  \end{equation*}
  であり、$\vb*{u} = \vb*{0}$のときに限り、等号が成立する
\end{theorem}

\begin{proof}
  内積の定義より、
  \begin{equation*}
    (\vb*{u}, \vb*{u}) = \sum_{i=1}^n u_i^2 \geq 0
  \end{equation*}
  である

  ここで現れた$u_i^2$は、$u_i$が0のときに限り0になるので、$\vb*{u} = \vb*{0}$のときに限り、等号が成立する $\qed$
\end{proof}

\sectionline
\section{$\mathbb{R}^n$上の内積と直交}
\marginnote{\refbookA p77}

\begin{theorem}{$\mathbb{R}^n$上の内積に対するコーシー・シュワルツの不等式}
  $\vb*{u}, \vb*{v} \in \mathbb{R}^n$に対して、次が成り立つ
  \begin{equation*}
    |(\vb*{u}, \vb*{v})| \leq \|\vb*{u}\| \|\vb*{v}\|
  \end{equation*}
\end{theorem}

\begin{proof}
  任意の$t\in \mathbb{R}$に対して、
  \begin{equation*}
    \|\vb*{u} - t\vb*{v}\|^2 = (\vb*{u} - t\vb*{v}, \vb*{u} - t\vb*{v}) \geq 0
  \end{equation*}
  が成り立つ

  ここで、内積の双線形性を用いて左辺を展開すると、
  \begin{align*}
    (\vb*{u}, \vb*{u}) - 2t(\vb*{u}, \vb*{v}) + t^2(\vb*{v}, \vb*{v}) & \geq 0 \\
    \|\vb*{u}\|^2 - 2t(\vb*{u}, \vb*{v}) + t^2\|\vb*{v}\|^2           & \geq 0
  \end{align*}
  これは$t$についての2次式であり、判別式が0以下であることから、次の不等式が成り立つ
  \begin{align*}
    (-2(\vb*{u}, \vb*{v}))^2 - 4\|\vb*{u}\|^2\|\vb*{v}\|^2 & \leq 0                           \\
    4(\vb*{u}, \vb*{v})^2                                  & \leq 4\|\vb*{u}\|^2\|\vb*{v}\|^2
  \end{align*}
  よって、両辺を4で割ると
  \begin{equation*}
    |(\vb*{u}, \vb*{v})| \leq \|\vb*{u}\| \|\vb*{v}\|
  \end{equation*}
  が得られる $\qed$
\end{proof}

これより、$\vb*{u},\,\vb*{v}$が$0$でないとき、
\begin{equation*}
  -1 \leq \frac{(\vb*{u}, \vb*{v})}{\|\vb*{u}\| \|\vb*{v}\|} \leq 1
\end{equation*}
なので、
\begin{equation*}
  \cos \theta = \frac{(\vb*{u}, \vb*{v})}{\|\vb*{u}\| \|\vb*{v}\|} \quad (0 \leq \theta \leq \pi)
\end{equation*}
を介して$\vb*{u},\,\vb*{v}$の\keyword{なす角}を定義できる

\begin{definition}{$\mathbb{R}^n$上のベクトルのなす角}
  $\vb*{u},\,\vb*{v} \in \mathbb{R}^n$に対して、
  \begin{equation*}
    \cos \theta = \frac{(\vb*{u}, \vb*{v})}{\|\vb*{u}\| \|\vb*{v}\|} \quad (0 \leq \theta \leq \pi)
  \end{equation*}
  により定まる$\theta$を$\vb*{u},\,\vb*{v}$の\keyword{なす角}という
\end{definition}

$\cos \theta = 0$は、幾何学的には$\vb*{u}$と$\vb*{v}$のなす角が直角であることを意味する

\begin{definition}{$\mathbb{R}^n$上のベクトルの直交}
  $\vb*{u},\,\vb*{v} \in \mathbb{R}^n$に対して、
  \begin{equation*}
    (\vb*{u}, \vb*{v}) = 0
  \end{equation*}
  が成り立つとき、$\vb*{u}$と$\vb*{v}$は\keyword{直交}するといい、
  \begin{equation*}
    \vb*{u} \perp \vb*{v}
  \end{equation*}
  と表記する
\end{definition}

\sectionline
\section{$\mathbb{C}^n$上の内積}

複素数$z=a+bi$に対して、
\begin{equation*}
  (a+bi)(a-bi) = a^2 + b^2 \geq 0
\end{equation*}
という式が成り立つ

このとき、$a-bi$を$z$の\keyword{共役複素数}といい、$\overline{z}$と表記する

また、$\sqrt{a^2 + b^2}$は$z$の\keyword{絶対値}と呼ばれ、$|z|$と表記する

\br

すなわち、冒頭の不等式は、
\begin{equation*}
  |z|^2 = z \overline{z} \geq 0
\end{equation*}
と書き換えられる

\br

このことを利用して、$\mathbb{C}^n$上の内積は、次のように定義すると$\mathbb{R}^n$の場合の自然な拡張になる

\begin{definition}{$\mathbb{C}^n$上の内積(標準内積)}\label{def:standard-inner-product-Cn}
  $\vb*{a} = (a_i)_{i=1}^n,\, \vb*{b} = (b_i)_{i=1}^n \in \mathbb{C}^n$に対して、
  \begin{equation*}
    (\vb*{a}, \vb*{b}) = \sum_{i=1}^n a_i \overline{b_i}
  \end{equation*}
  を$\mathbb{C}^n$上の\keyword{内積}と定義する

  この内積は\keyword{標準内積}、あるいは\keyword{標準エルミート内積}とも呼ばれる
\end{definition}

このように定めることで、特に、
\begin{equation*}
  (\vb*{a}, \vb*{a}) = \sum_{i=1}^n a_i \overline{a_i} = \sum_{i=1}^n |a_i|^2 \geq 0
\end{equation*}
であるので、$\mathbb{R}^n$の場合と同様に、ベクトルの\keyword{ノルム}を定義できる

\sectionline

$\mathbb{R}^n$上の内積で成り立つ性質の多くは、$\mathbb{C}^n$上の内積でも成り立つが、対称性に関しては注意が必要である

\begin{theorem}{標準内積の対称性}\label{thm:standard-inner-product-symmetry}
  $\vb*{u}, \vb*{v} \in \mathbb{C}^n$に対して、次が成り立つ
  \begin{equation*}
    (\vb*{u}, \vb*{v}) = \overline{(\vb*{v}, \vb*{u})}
  \end{equation*}
\end{theorem}

\begin{proof}
  $\overline{\overline{z}} = z$をふまえると、
  \begin{align*}
    \overline{(\vb*{v}, \vb*{u})} & = \overline{\sum_{i=1}^n v_i \overline{u_i}} \\
                                  & = \sum_{i=1}^n \overline{v_i \overline{u_i}} \\
                                  & = \sum_{i=1}^n \overline{v_i} u_i            \\
                                  & = \sum_{i=1}^n u_i \overline{v_i}            \\
                                  & = (\vb*{u}, \vb*{v})
  \end{align*}
  となり、目的の式が示された $\qed$
\end{proof}

複素数$z=a+bi$において、$b=0$の場合、$z$は実数である

このとき、$a+0i = a-0i = a$であるから、$z$が実数の場合、
\begin{equation*}
  \overline{z} = z
\end{equation*}
が成り立つ

\br

よって、$\vb*{u},\,\vb*{v} \in \mathbb{R}^n$であるなら、$\mathbb{C}^n$上の内積の対称性の式は
\begin{equation*}
  (\vb*{u}, \vb*{v}) = \overline{(\vb*{v}, \vb*{u})} = (\vb*{v}, \vb*{u})
\end{equation*}
と書き換えられ、これは$\mathbb{R}^n$上の内積の対称性そのものである

\br

つまり、$\mathbb{C}^n$上の内積の対称性は、$\mathbb{R}^n$上の内積の対称性も含んだ表現になっている

\sectionline
\section{転置による内積の表現}

内積は、転置を用いて表現することもできる

\begin{theorem}{転置による内積の表現}\label{thm:inner-product-as-transpose-product}
  \begin{equation*}
    (\vb*{a},\vb*{b}) = {}^t \vb*{a} \cdot \overline{\vb*{b}} = \left(a_1,\,a_2,\,\ldots,\,a_n\right) \begin{pmatrix}
      \overline{b_1} \\
      \overline{b_2} \\
      \vdots         \\
      \overline{b_n}
    \end{pmatrix}
  \end{equation*}
\end{theorem}

\end{document}
