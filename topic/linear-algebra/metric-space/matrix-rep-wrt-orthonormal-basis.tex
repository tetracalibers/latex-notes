\documentclass[../../../topic_linear-algebra]{subfiles}

\begin{document}

\sectionline
\section{正規直交基底による表現行列の展開}
\marginnote{\refbookI p1〜3}

$\mathbb{R}^n$から$\mathbb{R}^m$への線形写像は、ある$m \times n$型行列$A$によって表現される

\br

これを定める基本的な方法は、
\begin{enumerate}
  \item 定義域$\mathbb{R}^n$に一つの\keyword{正規直交基底}(互いに直交する単位ベクトル)$\{\vb*{u}_1,\ldots,\vb*{u}_n\}$を定める
  \item それぞれが写像されるべき$m$次元ベクトル(\keyword{像})$\vb*{a}_1,\ldots,\vb*{a}_n$を指定する
\end{enumerate}
という手順であり、このとき、行列$A$は
\begin{equation*}
  A = \begin{pmatrix}
    \vb*{a}_1 \\
    \vdots    \\
    \vb*{a}_n
  \end{pmatrix} \begin{pmatrix}
    \vb*{u}_1 & \cdots & \vb*{u}_n
  \end{pmatrix} = \vb*{a}_1 \vb*{u}_1^\top + \cdots + \vb*{a}_n \vb*{u}_n^\top
\end{equation*}
と書くことができる($\top$は転置を表す)

\begin{theorem}{正規直交基底による表現行列の展開}\label{thm:orthobasis-formula-for-rep-matrix}
  $\mathbb{R}^n$から$\mathbb{R}^m$への線形写像$f$の表現行列$A$は、$\mathbb{R}^n$の正規直交基底$\{\vb*{u}_1,\ldots,\vb*{u}_n\}$を用いて、次のように表すことができる
  \begin{equation*}
    A = \sum_{i=1}^n f(\vb*{u}_i) \vb*{u}_i^\top
  \end{equation*}
\end{theorem}

実際、両辺に$\vb*{u}_i$をかけると、
\begin{equation*}
  A \vb*{u}_i = \sum_{j=1}^n \vb*{a}_j \vb*{u}_j^\top \vb*{u}_i
  = \sum_{j=1}^n \vb*{a}_j \delta_{ij}
  = \vb*{a}_i
\end{equation*}
より、
\begin{equation*}
  A \vb*{u}_i = \vb*{a}_i \quad (i = 1, \ldots, n)
\end{equation*}
が成り立つことがわかる

\sectionline

特に、$\mathbb{R}^n$の正規直交基底として\keyword{標準基底}$\{\vb*{e}_1,\ldots,\vb*{e}_n\}$を選ぶと、行列$A$は次のように表せる
\begin{align*}
  A & = \sum_{i=1}^n \vb*{a}_i \vb*{e}_i^\top                               \\
    & = \begin{pmatrix}
          \vb*{a}_{11} \\
          \vdots       \\
          \vb*{a}_{m1}
        \end{pmatrix} \begin{pmatrix}
                        1 & \cdots & 0 \\
                      \end{pmatrix} + \cdots + \begin{pmatrix}
                                                 \vb*{a}_{1n} \\
                                                 \vdots       \\
                                                 \vb*{a}_{mn}
                                               \end{pmatrix} \begin{pmatrix}
                                                               0 & \cdots & 1
                                                             \end{pmatrix} \\
    & = \begin{pmatrix}
          \vb*{a}_{11} & \cdots & \vb*{a}_{1n} \\
          \vdots       & \ddots & \vdots       \\
          \vb*{a}_{m1} & \cdots & \vb*{a}_{mn}
        \end{pmatrix}
\end{align*}

\br

すなわち、表現行列$A$は、
\begin{shaded}
  \keyword{像}$\vb*{a}_1,\ldots,\vb*{a}_n$を列として順に並べた行列$\begin{pmatrix}  \vb*{a}_1 & \cdots & \vb*{a}_n\end{pmatrix}$
\end{shaded}
となる

\end{document}
