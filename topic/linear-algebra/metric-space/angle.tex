\documentclass[../../../topic_linear-algebra]{subfiles}

\begin{document}

\sectionline
\section{ベクトルのなす角}

コーシー・シュワルツの不等式は、絶対値の性質から、次のように書き換えられる。
\begin{equation*}
  - \| \vb*{u} \| \| \vb*{v} \| \leq (\vb*{u}, \vb*{v}) \leq \| \vb*{u} \| \| \vb*{v} \|
\end{equation*}

$\vb*{u},\,\vb*{v}$が$\vb*{o}$でないときには、
\begin{equation*}
  -1 \leq \frac{(\vb*{u}, \vb*{v})}{\|\vb*{u}\| \|\vb*{v}\|} \leq 1
\end{equation*}
となるので、
\begin{equation*}
  \cos \theta = \frac{(\vb*{u}, \vb*{v})}{\|\vb*{u}\| \|\vb*{v}\|} \quad (0 \leq \theta \leq \pi)
\end{equation*}
を介して$\vb*{u},\,\vb*{v}$の\keyword{なす角}を定義できる。

\begin{definition}{ベクトルのなす角}{angle-between-vectors}
  計量空間$V$上のベクトル$\vb*{u},\vb*{v}$に対して、
  \begin{equation*}
    \cos \theta = \frac{(\vb*{u}, \vb*{v})}{\|\vb*{u}\| \|\vb*{v}\|} \quad (0 \leq \theta \leq \pi)
  \end{equation*}
  により定まる$\theta$を$\vb*{u},\,\vb*{v}$の\keyword{なす角}という。
\end{definition}

\sectionline
\section{内積が表す「関係の強さ」}\label{sec:inner-product-similarity}

ベクトルのなす角の式から、内積の幾何的な解釈を捉えることができる。
\begin{equation*}
  (\vb*{u}, \vb*{v}) = \| \vb*{u} \| \| \vb*{v} \| \cos \theta
\end{equation*}
この式を、次のような図でイメージしてみよう。

\begin{center}
  \begin{tikzpicture}[scale=2.5]
    % Coordinates
    \coordinate (O) at (0,0);
    \coordinate (V) at (1.7,0);          % Vector v
    \coordinate (U) at (1.2,0.7);        % Vector u

    % Projection of u onto v
    \coordinate (P) at ($(O)!(U)!(V)$);          % foot of perpendicular

    % Right angle mark
    \draw pic[draw=lightslategray, angle radius=1ex]
      {right angle=U--P--O};

    % Draw angle between u and v
    \draw pic[
        draw=black,
        ->,
        angle radius=7.5mm,
        angle eccentricity=1.5,
        "$\theta$"
      ] {angle=V--O--U};

    % Dashed line for projection
    \draw[densely dashed, thick, lightslategray] (U) -- (P);

    % Draw vectors
    \draw[vector, Periwinkle!75] (O) -- (U) node[above] {$\vb*{u}$};
    \draw[vector, Periwinkle!75] (O) -- (V) node[right] {$\vb*{v}$};

    \draw[<->, thick, Straight Barb-Straight Barb, Rhodamine] ([yshift=-0.5ex]O) -- ([yshift=-0.5ex]V) node[below, pos=0.85] {$\|\vb*{v}\|$};
    \draw[<->, thick, Straight Barb-Straight Barb, Cerulean] ([yshift=-1ex]O) -- ([yshift=-1ex]P) node[below, midway] {$\|\vb*{u}\|\cos\theta$};
  \end{tikzpicture}
\end{center}

この図から、内積は次のようにも解釈できる。

\begin{emphabox}
  \begin{spacebox}
    \begin{center}
      $\vb*{u}$の$\vb*{v}$との内積とは、\\
      $\vb*{v}$自身の長さと、$\vb*{u}$の$\vb*{v}$方向の長さの積である
    \end{center}
  \end{spacebox}
\end{emphabox}

ここで、「$\vb*{u}$の$\vb*{v}$方向の長さ」は、後に\keyword{正射影}という量として定義する。

\subsection{自分自身との内積の再解釈}

$\vb*{v}$と$\vb*{u}$が平行でまったく同じ方向を向いている場合、「$\vb*{u}$の$\vb*{v}$方向の長さ」は、$\vb*{u}$の長さそのものである。
\begin{equation*}
  \| \vb*{u} \| \cos \theta = \| \vb*{u} \|
\end{equation*}

また、このとき、$\vb*{v}$と$\vb*{u}$は互いに正の数のスカラー倍で表すことができるので、
\begin{equation*}
  \| \vb*{v} \| = k \| \vb*{u} \| \quad (k > 0)
\end{equation*}

すると、内積は、
\begin{equation*}
  (\vb*{u}, \vb*{v}) = k \| \vb*{u} \|^2
\end{equation*}

\br

ここで、$\vb*{v} = \vb*{u}$の場合は、$\vb*{u}$を特にスケーリングしなくても$\vb*{v}$に一致するので、$k=1$である。
\begin{equation*}
  (\vb*{u}, \vb*{u}) = \| \vb*{u} \|^2
\end{equation*}

このように$\vb*{u}$の$\vb*{u}$方向の長さは$\vb*{u}$自身の長さであることから、自分自身との内積は$\text{\bfseries 長さ}^2$となる。

その平方根をとれば長さが得られるということで、ベクトルのノルムの定義
\begin{equation*}
  \| \vb*{u} \| = \sqrt{(\vb*{u}, \vb*{u})}
\end{equation*}
を自然に解釈することができる。

\subsection{平行の度合いと内積}

同じ方向を向いているベクトルどうしは、平行に近ければ近いほど、これらは互いに似ていて「関係性の強い」ベクトルだといえる。

\br

2つのベクトルが同方向で完全に平行なとき、なす角$\theta$は0であるので、$\cos \theta$の値は$1$($\cos\theta$の最大値)となる。

つまり、同方向で平行に近い「似た」ベクトルほど、内積の値は最大値に近くなる。

\subsection{逆方向と内積の符号}

一方、2つのベクトルが完全に平行で、逆の方向を向いているなら、片方のベクトルはもう片方のベクトルを負の数を使ってスカラー倍したものになる。

\br

逆向きのベクトルどうしは、近い方向どころかむしろ「かけ離れた方向を向いている」といえる。

内積が「向きの似ている度合い」なら、「近い方向を向いている」度合いを正の数で、「かけ離れた方向を向いている」度合いを負の数で表すのが自然である。

\br

実際、2つのベクトルが逆向きで完全に平行なとき、$\cos \theta$の値は$-1$($\cos\theta$の最小値)となる。

つまり、逆方向に近い「かけ離れた」ベクトルほど、内積の値は最小値に近くなる。

\sectionline
\section{ベクトルの直交}

「同じ向きに近い」場合と「逆向きに近い」場合が切り替わるのは、2つのベクトルどうしが垂直なときである。

ならば、内積の正と負が切り替わる境界、すなわち内積が$0$になる場合とは、2つのベクトルが直交する場合であるのが自然ではないだろうか。

\br

別な考え方として、完全に垂直な2つのベクトルは、互いに全く影響を与えない方向を向いている。

2つのベクトルが直交している場合、2つのベクトルは互いに全く関係がないものとして、関係の強さを表す内積の値は$0$にしたい。

\br

実際、内積の定義はこの解釈に沿うものになっている。

$\vb*{u}$と$\vb*{v}$のなす角が直角であるとき、$\cos \theta = 0$となるので、内積も0になる。

\br

幾何学的なイメージができない高次元の場合についても、内積が0になること、すなわち2つのベクトルが無関係であることを\keyword{直交}の定義としてしまおう。

\begin{definition*}{ベクトルの直交}
  計量空間$V$上のベクトル$\vb*{u},\vb*{v}$に対して、
  \begin{equation*}
    (\vb*{u}, \vb*{v}) = 0
  \end{equation*}
  が成り立つとき、$\vb*{u}$と$\vb*{v}$は\keyword{直交}するといい、
  \begin{equation*}
    \vb*{u} \perp \vb*{v}
  \end{equation*}
  と表記する。
\end{definition*}

\end{document}
