\documentclass[../../../topic_linear-algebra]{subfiles}

\begin{document}

\sectionline
\section{写像の制限}
\marginnote{\refbookS p10 \\ \refbookC p57 \\ \refbookP p198}

定義域を一部分に限定する、すなわち写像の定義域をその部分集合に取り替えることを、写像の\keywordJE{制限}{restriction}という。

\begin{definition}{写像の制限}
  写像$f\colon X \to Y$に対して、部分集合$U \subset X$と$V \subset Y$が$f(U) \subset V$を満たすとき、$x \in U$に対して$f(x) \in V$を対応させる写像$U \to V$を写像$f$の$U$への\keyword{制限}といい、$f|_U$と表す。
  \begin{equation*}
  \begin{array}{lclc}
    f|_U \colon & U         & \longrightarrow & V          \\
            & \rotatebox{90}{$\in$} &                 & \rotatebox{90}{$\in$} \\
            & x              & \longmapsto     & f(x)
  \end{array}
\end{equation*}
\end{definition}

\begin{figure}[h]
\centering
\caption*{\bfseries $f\colon X \to Y$(定義域が$X$)}
  \scalebox{0.75}{
    \begin{tikzpicture}
      \def\rB{2cm}
      \def\rC{2.5cm}       % 円Cの半径(Bより大きい)
      \def\rCopyA{1cm}
      \def\rCopyCopyA{1.4cm} % C内の小円の半径(Aのコピー)
      \def\rSubA{0.9cm}
      \def\dotSize{2pt}
      \def\spaceX{6}

      % 各円の中心X座標
      \def\xB{0}
      \def\xC{\spaceX+1.5*0.5}

      % 円A, B
      \begin{scope}[local bounding box=circleB]
        \draw[fill=SkyBlue!40, draw=lightslategray, thick] (\xB,0) circle (\rB);
      \end{scope}
      \begin{scope}[local bounding box=circleC]
        \draw[lightslategray, thick] (\xC,0) circle (\rC);
      \end{scope}

      % ラベル
      \node[yshift=\rB] at (\xB,0.5) {\large $X$};
      \node[yshift=\rC] at (\xC,0.5) {\large $Y$}; % y位置も少し上へ

      % C内の小円
      \begin{scope}[local bounding box=circleSubC]
        \draw[draw=lightslategray, fill=Rhodamine!40] (\xC,0) circle (\rCopyCopyA);
      \end{scope}

      % 点(B内)
      \fill (\xB+0.2,0.25) circle (\dotSize);
      % 点(C内)
      \fill (\xC+0.2,0.25) circle (\dotSize);

      % B → C を結ぶ矢印
      \draw[->, thick, shorten >= 1em, shorten <= 1em] (circleB.east) -- (circleC.west) node[midway, above] {$f$};

      % B → C の対応矢印
      \draw[RoyalBlue, |->, very thick, shorten >= 4pt, shorten <= 4pt]
      (\xB+0.2,0.25) to[out=35,in=145] (\xC+0.2,0.25);

      % ラベル(コピー情報)
      \node[yshift=2ex] at (\xC,\rCopyCopyA) {$f(X)$};

      \draw[lightslategray] (circleB.north) -- (circleSubC.north);
      \draw[lightslategray] (circleB.south) -- (circleSubC.south);
    \end{tikzpicture}
  }
\end{figure}

\begin{figure}[h]
\centering
\caption*{\bfseries $f|_U\colon U \to V$(定義域を$U$に制限)}
  \scalebox{0.75}{
    \begin{tikzpicture}
      \def\rB{2cm}
      \def\rC{2.5cm}       % 円Cの半径(Bより大きい)
      \def\rCopyA{1cm}
      \def\rCopyCopyA{1.6cm} % C内の小円の半径(Aのコピー)
      \def\rCopyCopyCopyA{0.9cm}
      \def\rSubA{0.9cm}
      \def\dotSize{2pt}
      \def\spaceX{6}

      % 各円の中心X座標
      \def\xB{0}
      \def\xC{\spaceX+1.5*0.5}

      % 円A, B
      \begin{scope}[local bounding box=circleB]
        \draw[lightslategray, thick] (\xB,0) circle (\rB);
      \end{scope}
      \begin{scope}[local bounding box=circleC]
        \draw[lightslategray, thick] (\xC,0) circle (\rC);
      \end{scope}

      % ラベル
      \node[yshift=\rB] at (\xB,0.5) {\large $X$};
      \node[yshift=\rC] at (\xC,0.5) {\large $Y$}; % y位置も少し上へ

      % B内の小円
      \begin{scope}[local bounding box=circleSubB]
        \draw[fill=SkyBlue!40, draw=lightslategray] (\xB,0) circle (\rCopyA);
      \end{scope}

      % C内の小円
      \begin{scope}[local bounding box=circleSubC]
        \draw[draw=lightslategray] (\xC,0) circle (\rCopyCopyA);
      \end{scope}
      
      % 小円の中の小円
      \begin{scope}[local bounding box=circleSubSubC]
        \draw[fill=Rhodamine!40, draw=lightslategray] (\xC,0) circle (\rCopyCopyCopyA);
      \end{scope}

      % 点(B内)
      \fill (\xB+0.2,0.25) circle (\dotSize);
      % 点(C内)
      \fill (\xC+0.2,0.25) circle (\dotSize);

      % B → C を結ぶ矢印
      \draw[->, thick, shorten >= 1em, shorten <= 1em] (circleB.east) -- (circleC.west) node[midway, above] {$f|_U$};

      % B → C の対応矢印
      \draw[RoyalBlue, |->, very thick, shorten >= 4pt, shorten <= 4pt]
      (\xB+0.2,0.25) to[out=35,in=145] (\xC+0.2,0.25);

      % ラベル(コピー情報)
      \node[yshift=2ex] at (\xB,\rCopyA) {$U$};
      \node[yshift=2ex] at (\xC,\rCopyCopyA) {$V$};
      \node[yshift=1.5ex] at (\xC,\rCopyCopyCopyA) {$f(U)$};

      \draw[lightslategray] (circleSubB.north) -- (circleSubSubC.north);
      \draw[lightslategray] (circleSubB.south) -- (circleSubSubC.south);
    \end{tikzpicture}
  }
\end{figure}

\end{document}
