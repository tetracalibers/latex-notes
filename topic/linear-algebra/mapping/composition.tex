\documentclass[../../../topic_linear-algebra]{subfiles}

\begin{document}

\sectionline
\section{合成写像}
\marginnote{
  \refweb{合成関数(合成写像)の定義と性質~注意点を添えて~}{https://mathlandscape.com/composite-function/}
}

対応づける操作を続けて行うことは、写像の\keyword{合成}として定義される。

\begin{center}
  \scalebox{0.8}{
    \begin{tikzpicture}
      \def\rA{1.5cm}
      \def\rB{1.5cm}
      \def\rC{1.5cm}
      \def\dotSize{2pt}
      \def\spaceX{5}

      % 各円の中心X座標
      \def\xA{0}
      \def\xB{\spaceX}
      \def\xC{\spaceX*2}

      % 円A, B, C
      \begin{scope}[local bounding box=circleA]
        \draw[lightslategray, thick] (\xA,0) circle (\rA);
      \end{scope}
      \begin{scope}[local bounding box=circleB]
        \draw[draw=lightslategray, thick] (\xB,0) circle (\rB);
      \end{scope}
      \begin{scope}[local bounding box=circleC]
        \draw[draw=lightslategray, thick] (\xC,0) circle (\rC);
      \end{scope}

      % ラベル
      \node[yshift=\rA] at (\xA,0.5) {\large $X$};
      \node[yshift=\rB] at (\xB,0.5) {\large $Y$};
      \node[yshift=\rC] at (\xC,0.5) {\large $Z$}; % y位置も少し上へ

      % 点(A内)
      \fill (\xA+0.25,0.5) circle (\dotSize) node[below] {$x$};
      % 点(B内)
      \fill (\xB+0.2,0.25) circle (\dotSize) node[below] {$y$};
      % 点(C内の小円)
      \fill (\xC+0.2,0.25) circle (\dotSize) node[below] {$z$};

      % A → B の写像矢印
      \draw[RoyalBlue, |->, very thick, shorten >= 4pt, shorten <= 4pt]
      (\xA+0.25,0.5) to[out=45,in=135] (\xB+0.2,0.25);
      % B → C の対応矢印
      \draw[RoyalBlue, |->, very thick, shorten >= 4pt, shorten <= 4pt]
      (\xB+0.2,0.25) to[out=45,in=135] (\xC+0.2,0.25);

      % A → B を結ぶ矢印
      \draw[->, thick, shorten >= 1em, shorten <= 1em] (circleA.east) -- (circleB.west) node[midway, above] {$f$};
      % B → C を結ぶ矢印
      \draw[->, thick, shorten >= 1em, shorten <= 1em] (circleB.east) -- (circleC.west) node[midway, above] {$g$};
      
      % A → C の合成矢印
      \draw[->, thick, shorten >= 1em, shorten <= 1em]
      (circleA.south) to[out=-35,in=-145] (circleC.south);
      \node[yshift=-3em] at (circleB.south) {$g \circ f$};
    \end{tikzpicture}
  }
\end{center}

\begin{definition}{合成写像}{composition-of-maps}
  2つの写像$f\colon X \to Y, g\colon Y \to Z$が与えられたとき、$X$の元に対して$Z$の元を対応させる$X$から$Z$への写像を$f$と$g$の\keyword{合成写像}といい、$g \circ f$と表す。
  \begin{equation*}
    (g \circ f)(x) = g(f(x))
  \end{equation*}
\end{definition}

\subsection{写像の合成の性質}

\begin{theorem}{写像の合成の結合法則}{composition-associativity}
  \begin{equation*}
    h \circ (g \circ f) = (h \circ g) \circ f
  \end{equation*}
\end{theorem}

\begin{proof}
  $f\colon X\to Y, \, g\colon Y\to Z, \, h\colon Z\to W$とする。
  
  任意の$x\in X$に対して、
  \begin{equation*}
    \bigl(h\circ (g\circ f)\bigr)(x)
    = h\bigl((g\circ f)(x)\bigr)
    = h\bigl(g(f(x))\bigr)
  \end{equation*}
  一方で、
  \begin{equation*}
    \bigl((h\circ g)\circ f\bigr)(x)
    = (h\circ g)\bigl(f(x)\bigr)
    = h\bigl(g(f(x))\bigr)
  \end{equation*}
  したがって、任意の$x\in X$について両辺が一致するので、
  \begin{equation*}
    h\circ (g\circ f) = (h\circ g)\circ f
  \end{equation*}
  が成り立つ。 $\qed$
\end{proof}

\subsection{単射な写像の合成}

\begin{mindflow}
  \todo{単射な写像の合成は単射である}
\end{mindflow}

\subsection{全射な写像の合成}

\begin{mindflow}
  \todo{全射な写像の合成は全射である}
\end{mindflow}

\end{document}
