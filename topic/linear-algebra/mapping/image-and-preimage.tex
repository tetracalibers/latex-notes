\documentclass[../../../topic_linear-algebra]{subfiles}

\begin{document}

\sectionline
\section{像と逆像}
\marginnote{\refbookS p10}

\begin{definition}{部分集合の像}
  写像$f\colon X \to Y$が与えられたとき、$X$の部分集合$U$に対し、$x \in U$を$f$で写したものの集合を$U$の\keyword{像}という。
  \begin{equation*}
    f(U) = \{ f(x) \mid x \in U \} \subset Y
  \end{equation*}
\end{definition}

\begin{center}
  \scalebox{0.8}{
    \begin{tikzpicture}
      \def\rB{2cm}
      \def\rC{2.5cm}       % 円Cの半径(Bより大きい)
      \def\rCopyA{1cm}
      \def\rCopyCopyA{1.4cm} % C内の小円の半径(Aのコピー)
      \def\rSubA{0.9cm}
      \def\dotSize{2pt}
      \def\spaceX{6}

      % 各円の中心X座標
      \def\xB{0}
      \def\xC{\spaceX+1.5*0.5}

      % 円A, B
      \begin{scope}[local bounding box=circleB]
        \draw[lightslategray, thick] (\xB,0) circle (\rB);
      \end{scope}
      \begin{scope}[local bounding box=circleC]
        \draw[lightslategray, thick] (\xC,0) circle (\rC);
      \end{scope}

      % ラベル
      \node[yshift=\rB] at (\xB,0.5) {\large $X$};
      \node[yshift=\rC] at (\xC,0.5) {\large $Y$}; % y位置も少し上へ

      % B内の小円
      \begin{scope}[local bounding box=circleSubB]
        \draw[fill=SkyBlue!40, draw=lightslategray] (\xB,0) circle (\rCopyA);
      \end{scope}

      % C内の小円
      \begin{scope}[local bounding box=circleSubC]
        \draw[fill=Rhodamine!40, draw=lightslategray] (\xC,0) circle (\rCopyCopyA);
      \end{scope}

      % 点(B内)
      \fill (\xB+0.2,0.25) circle (\dotSize);
      % 点(C内)
      \fill (\xC+0.2,0.25) circle (\dotSize);

      % B → C を結ぶ矢印
      \draw[->, thick, shorten >= 1em, shorten <= 1em] (circleB.east) -- (circleC.west) node[midway, above] {$f$};

      % B → C の対応矢印
      \draw[RoyalBlue, |->, very thick, shorten >= 4pt, shorten <= 4pt]
      (\xB+0.2,0.25) to[out=35,in=145] (\xC+0.2,0.25);

      % ラベル(コピー情報)
      \node[yshift=2ex] at (\xB,\rCopyA) {$U$};
      \node[yshift=2ex] at (\xC,\rCopyCopyA) {$f(U)$};

      \draw[lightslategray] (circleSubB.north) -- (circleSubC.north);
      \draw[lightslategray] (circleSubB.south) -- (circleSubC.south);
    \end{tikzpicture}
  }
\end{center}

$X = U$の場合、$X$の像は$f$の像と呼ばれる。

\begin{definition}{写像の像}
  写像$f\colon X \to Y$が与えられたとき、$x \in X$を$f$で写したものの集合を$f$の\keyword{像}という。
  \begin{equation*}
    f(X) = \{ f(x) \mid x \in X \} \subset Y
  \end{equation*}
\end{definition}

\begin{center}
  \scalebox{0.8}{
    \begin{tikzpicture}
      \def\rB{2cm}
      \def\rC{2.5cm}       % 円Cの半径(Bより大きい)
      \def\rCopyCopyA{1.4cm} % C内の小円の半径(Aのコピー)
      \def\rSubA{0.9cm}
      \def\dotSize{2pt}
      \def\spaceX{6}

      % 各円の中心X座標
      \def\xB{0}
      \def\xC{\spaceX+1.5*0.5}

      % 円A, B
      \begin{scope}[local bounding box=circleB]
        \draw[lightslategray, thick, fill=SkyBlue!40] (\xB,0) circle (\rB);
      \end{scope}
      \begin{scope}[local bounding box=circleC]
        \draw[lightslategray, thick] (\xC,0) circle (\rC);
      \end{scope}

      % ラベル
      \node[yshift=\rB] at (\xB,0.5) {\large $X$};
      \node[yshift=\rC] at (\xC,0.5) {\large $Y$}; % y位置も少し上へ

      % C内の小円
      \begin{scope}[local bounding box=circleSubC]
        \draw[fill=Rhodamine!40, draw=lightslategray] (\xC,0) circle (\rCopyCopyA);
      \end{scope}

      % 点(B内)
      \fill (\xB+0.2,0.25) circle (\dotSize);
      % 点(C内)
      \fill (\xC+0.2,0.25) circle (\dotSize);

      % B → C を結ぶ矢印
      \draw[->, thick, shorten >= 1em, shorten <= 1em] (circleB.east) -- (circleC.west) node[midway, above] {$f$};

      % B → C の対応矢印
      \draw[RoyalBlue, |->, very thick, shorten >= 4pt, shorten <= 4pt]
      (\xB+0.2,0.25) to[out=35,in=145] (\xC+0.2,0.25);

      % ラベル(コピー情報)
      \node[yshift=2ex] at (\xC,\rCopyCopyA) {$f(X)$};

      \draw[lightslategray] (circleB.north) -- (circleSubC.north);
      \draw[lightslategray] (circleB.south) -- (circleSubC.south);
    \end{tikzpicture}
  }
\end{center}

\begin{definition}{部分集合の逆像}
  写像$f\colon X \to Y$が与えられたとき、$Y$の部分集合$V$に対し、$f$によって$V$の元に写るような$x \in X$の集合を$V$の\keyword{逆像}という。
  \begin{equation*}
    f^{-1}(V) = \{ x \in X \mid f(x) \in V \} \subset X
  \end{equation*}
\end{definition}

\begin{center}
  \scalebox{0.8}{
    \begin{tikzpicture}
      \def\rB{2cm}
      \def\rC{2.5cm}       % 円Cの半径(Bより大きい)
      \def\rCopyA{1cm}
      \def\rCopyCopyA{1.4cm} % C内の小円の半径(Aのコピー)
      \def\rSubA{0.9cm}
      \def\dotSize{2pt}
      \def\spaceX{6}

      % 各円の中心X座標
      \def\xB{0}
      \def\xC{\spaceX+1.5*0.5}

      % 円A, B
      \begin{scope}[local bounding box=circleB]
        \draw[lightslategray, thick] (\xB,0) circle (\rB);
      \end{scope}
      \begin{scope}[local bounding box=circleC]
        \draw[lightslategray, thick] (\xC,0) circle (\rC);
      \end{scope}

      % ラベル
      \node[yshift=\rB] at (\xB,0.5) {\large $X$};
      \node[yshift=\rC] at (\xC,0.5) {\large $Y$}; % y位置も少し上へ

      % B内の小円
      \begin{scope}[local bounding box=circleSubB]
        \draw[fill=SkyBlue!40, draw=lightslategray] (\xB,0) circle (\rCopyA);
      \end{scope}

      % C内の小円
      \begin{scope}[local bounding box=circleSubC]
        \draw[fill=Rhodamine!40, draw=lightslategray] (\xC,0) circle (\rCopyCopyA);
      \end{scope}

      % 点(B内)
      \fill (\xB+0.2,0.25) circle (\dotSize);
      % 点(C内)
      \fill (\xC+0.2,0.25) circle (\dotSize);

      % B → C を結ぶ矢印
      \draw[->, thick, shorten >= 1em, shorten <= 1em] (circleB.east) -- (circleC.west) node[midway, above] {$f$};

      % B → C の対応矢印
      \draw[RoyalBlue, |->, very thick, shorten >= 4pt, shorten <= 4pt]
      (\xB+0.2,0.25) to[out=35,in=145] (\xC+0.2,0.25);

      % ラベル
      \node[yshift=2ex] at (\xB,\rCopyA) {$f^{-1}(V)$};
      \node[yshift=2ex] at (\xC,\rCopyCopyA) {$V$};

      \draw[lightslategray] (circleSubB.north) -- (circleSubC.north);
      \draw[lightslategray] (circleSubB.south) -- (circleSubC.south);
    \end{tikzpicture}
  }
\end{center}

\begin{definition}{終域の元の逆像}\label{def:preimage-of-element}
  写像$f\colon X \to Y$が与えられたとき、$Y$の元$y$に対し、$f$によって$y$に写るような$x \in X$の集合を$y$の\keyword{逆像}という。
  \begin{equation*}
    f^{-1}(y) = \{ x \in X \mid f(x) = y \} \subset X
  \end{equation*}
\end{definition}

\begin{center}
  \scalebox{0.8}{
    \begin{tikzpicture}
      \def\rB{2cm}
      \def\rC{2cm}       % 円Cの半径(Bより大きい)
      \def\rCopyA{1.1cm}
      \def\rSubA{0.9cm}
      \def\dotSize{2pt}
      \def\spaceX{6}

      % 各円の中心X座標
      \def\xB{0}
      \def\xC{\spaceX+1.5*0.5}

      % 円A, B
      \begin{scope}[local bounding box=circleB]
        \draw[lightslategray, thick] (\xB,0) circle (\rB);
      \end{scope}
      \begin{scope}[local bounding box=circleC]
        \draw[lightslategray, thick] (\xC,0) circle (\rC);
      \end{scope}

      % ラベル
      \node[yshift=\rB] at (\xB,0.5) {\large $X$};
      \node[yshift=\rC] at (\xC,0.5) {\large $Y$}; % y位置も少し上へ

      % B内の小円
      \begin{scope}[local bounding box=circleSubB]
        \draw[fill=SkyBlue!40, draw=lightslategray] (\xB,0) circle (\rCopyA);
      \end{scope}

      % 点(B内)
      \fill (\xB+0.2,0.25) circle (\dotSize) node[left, font=\small] {$x_1$};
      \fill (\xB,-0.5) circle (\dotSize) node[left, font=\small] {$x_2$};
      % 点(C内)
      \fill (\xC+0.2,0.25) circle (\dotSize) node[right, font=\small] {$y$};

      % B → C を結ぶ矢印
      \draw[->, thick, shorten >= 1em, shorten <= 1em] (circleB.east) -- (circleC.west) node[midway, above] {$f$};

      % B → C の対応矢印
      \draw[RoyalBlue, |->, very thick, shorten >= 4pt, shorten <= 4pt]
      (\xB+0.2,0.25) to[out=35,in=145] (\xC+0.2,0.25);
      \draw[RoyalBlue, |->, very thick, shorten >= 4pt, shorten <= 4pt]
      (\xB,-0.5) to[out=-35,in=-145] (\xC+0.2,0.25);

      % ラベル
      \node[yshift=2ex] at (\xB,\rCopyA) {$f^{-1}(y)$};

      \draw[lightslategray] (circleSubB.north) -- (\xC+0.2,0.25);
      \draw[lightslategray] (circleSubB.south) -- (\xC+0.2,0.25);
    \end{tikzpicture}
  }
\end{center}

\end{document}
