\documentclass[../../../topic_linear-algebra]{subfiles}

\begin{document}

\sectionline
\section{恒等写像と包含写像}
\marginnote{
  \refbookS p9 \\
  \refweb{恒等写像(id),包含写像とは何か}{https://mathlandscape.com/identity-map/}
}

自分自身に対応させる写像は、次のように書ける。
\begin{equation*}
  f(x) = x
\end{equation*}
この場合、$f$によって$x$が$y$に変わるようなことはなく、$f$は$x$をそのまま写す(何も変えない)。

\br

任意の$x \in X$に対して、$f(x) = x$が成り立ち、かつ$f(x) \in Y$を満たすような状況は、次の2通りが考えられる。
\begin{itemize}
  \item $X = Y$の場合(左図)を\keywordJE{恒等写像}{identity map}という。
  \item $X \subset Y$の場合(右図)を\keywordJE{包含写像}{inclusion map}という。
\end{itemize}

\begin{figure}[h]
  \centering
  \begin{minipage}{0.49\columnwidth}
    \centering
  \begin{tikzpicture}
    \def\rA{1.5cm}       % 円の半径
    \def\dotSize{2pt}    % 点のサイズ
    \def\spaceX{3}       % 中心座標のX距離

    \def\leftX{-\spaceX}
    \def\rightX{\spaceX}

    % 半径2cmの円を配置
    \draw[lightslategray,thick, fill=SkyBlue!40] (\leftX,0) circle (\rA);

    % 左の円の中に点を不規則に配置
    \fill (\leftX+0.25,0.25) circle (\dotSize) node[above, font=\small] {$x$};

    % 自分自身に戻ってくる矢印
    \draw [RoyalBlue, |->, very thick] (\leftX+0.25,0) arc(-160:160:1);

    % ラベル
    \node[yshift=2ex] at (\leftX,\rA) {\bfseries $X = Y$};
  \end{tikzpicture}
  \end{minipage}
  \begin{minipage}{0.49\columnwidth}
    \centering
  \begin{tikzpicture}
    \def\rB{2.25cm}         % 円の半径
    \def\rCopy{1cm}      % 円の中の小円の半径
    \def\dotSize{2pt}    % 点のサイズ
    \def\spaceX{3}       % 中心座標のX距離

    \def\rightX{\spaceX}

    % 半径2cmの円を配置
    \draw[lightslategray,thick] (\rightX,0) circle (\rB);

    % 円の中にさらに小さな円を配置
    \draw[fill=SkyBlue!40, draw=lightslategray] (\rightX,-0.25) circle (\rCopy);
    
    % 円の中の小さな円の中に点を不規則に配置
    \fill (\rightX,-0.5) circle (\dotSize) node[above, font=\small] {$x$};
    
    % 自分自身に戻ってくる矢印
    \draw [RoyalBlue, |->, very thick] (\rightX,-0.75) arc(-160:160:1);

    % ラベル
    \node[yshift=2ex] at (\rightX,\rB) {$Y$};
    \node[yshift=-2ex] at (\rightX,1.5) {$X$};
  \end{tikzpicture}
  \end{minipage}
\end{figure}

\begin{definition*}{恒等写像}
  集合$X$に対して、$X$の任意の元$x$を$x \in X$に対応させる$X$から$X$への写像を$X$上の\keyword{恒等写像}といい、$\id_X$あるいは単に$\id$と表す。
  \begin{equation*}
  \begin{array}{lclc}
    \id_X \colon & X         & \longrightarrow & X          \\
            & \rotatebox{90}{$\in$} &                 & \rotatebox{90}{$\in$} \\
            & x              & \longmapsto     & x
  \end{array}
\end{equation*}
\end{definition*}

\begin{definition*}{包含写像}
  定義域$X$が終域$Y$の部分集合のとき、$X$の任意の元$x$を$x \in Y$に対応させる写像を\keyword{包含写像}といい、$\iota$と表す。
  \begin{equation*}
  \begin{array}{lclc}
    \iota \colon & X         & \longrightarrow & Y          \\
            & \rotatebox{90}{$\in$} &                 & \rotatebox{90}{$\in$} \\
            & x              & \longmapsto     & x
  \end{array}
\end{equation*}
\end{definition*}

\end{document}
