\documentclass[../../../topic_linear-algebra]{subfiles}

\begin{document}

\sectionline
\section{逆写像と可逆な写像}
\marginnote{\refbookC p61〜62 \\ \refbookS p10}

\keyword{全単射}な写像では、終域のどんな元も、定義域のただ1つの元の像となっている。

そのため、終域の元からその像になるような定義域の元をただ1つ決めることができる。

\br

この対応は、元々の写像の終域から定義域への写像となり、\keyword{逆写像}と呼ばれる。

\begin{definition}{逆写像}
  写像$f\colon X \to Y$が全単射であるとき、任意の$y \in Y$に対して、$f(x) = y$となる$x \in X$がただ一つ存在する。
  
  このとき、$Y$の元$y$に対して、$f(x) = y$となる$x$を対応させる写像を$f$の\keywordJE{逆写像}{inverse mapping}といい、次のように表す。
  \begin{equation*}
    f^{-1}\colon Y \to X
  \end{equation*}
\end{definition}

\begin{center}
  \begin{tikzpicture}
    \def\rA{1.5cm}       % 左の円の半径
    \def\rB{2cm}         % 右の円の半径
    \def\dotSize{2pt}    % 点のサイズ
    \def\spaceX{3}       % 中心座標のX距離

    \def\leftX{-\spaceX}
    \def\rightX{\spaceX}

    % 半径2cmの円を2つ横並びに配置
    \draw[lightslategray,thick] (\leftX,0) circle (\rA);
    \draw[lightslategray,thick] (\rightX,0) circle (\rB);

    % 2つの円の間に矢印
    \draw[<-, thick] (-1,0) -- (0.5,0) node[midway, above] {$f^{-1}$};
    
    % 左の円の中に2つの点を不規則に配置
    \fill (\leftX+0.25,0.5) circle (\dotSize) node[left, font=\small] {$x_1$};
    \fill (\leftX,-0.5) circle (\dotSize) node[left, font=\small] {$x_2$};

    % 右の円の中の小さな円の中に2つの点を不規則に配置
    \fill (\rightX+0.2,0.25) circle (\dotSize) node[below right, font=\small] {$y_1$};
    \fill (\rightX,-0.5) circle (\dotSize) node[below right, font=\small] {$y_2$};

    % 曲がった矢印を2つの点に向けて引く
    \draw[RoyalBlue, <-|, very thick, shorten >= 4pt, shorten <= 4pt]
    (\leftX+0.25,0.5) to[out=45,in=135] (\rightX+0.2,0.25);
    \draw[RoyalBlue, <-|, very thick, shorten >= 4pt, shorten <= 4pt]
    (\leftX,-0.5) to[out=-45,in=-135] (\rightX,-0.5);

    % ラベル
    \node[yshift=2ex] at (\leftX,\rA) {$X$};
    \node[yshift=2ex] at (\rightX,\rB) {$Y$};
  \end{tikzpicture}
\end{center}

\begin{definition}{可逆}
  写像$f \colon X \to Y$の逆写像が存在するとき、$f$は\keywordJE{可逆}{invertible}であるという。
\end{definition}

逆写像を定義する上での前提は、写像が全単射であることだったため、
\begin{emphabox}
  \begin{spacebox}
    \begin{center}
      \keyword{可逆}な写像とは、\keyword{全単射}な写像
    \end{center}
  \end{spacebox}
\end{emphabox}
のことをいう。

\br

また、逆写像は\keyword{恒等写像}によって定義することもできる。

そのための議論を次に行う。

\subsection{左逆写像と右逆写像}

$f$で写した後に$g$を適用すると元に戻るとき、$g$は$f$の\keyword{左逆写像}と呼ばれる。

\begin{definition}{左逆写像}
  写像$f\colon X \to Y$に対して、写像$g\colon Y \to X$が存在して、
  \begin{equation*}
    g \circ f = \id_X
  \end{equation*}
  を満たすとき、$g$は$f$の\keyword{左逆写像}であるという。
\end{definition}

\begin{figure}[h]
  \centering
  % --- 左逆写像:g ∘ f = id_X(集合レベル) ---
  \begin{minipage}[t]{0.45\textwidth}
    \caption*{\bfseries 集合レベルでの左逆写像}
    \begin{equation*}
      \begin{tikzcd}[every label/.append style = {font = \small}]
        X \arrow[r, "f"] \arrow[dr, swap, "\id_X"] & Y \arrow[d, "g"] \\
        & X
      \end{tikzcd}
    \end{equation*}
  \end{minipage}
  % --- 左逆写像:要素レベル(x \in X に対して g(f(x)) = x) ---
  \begin{minipage}[t]{0.45\textwidth}
    \caption*{\bfseries 要素レベルでの左逆写像}
    \begin{equation*}
      \begin{tikzcd}[every label/.append style = {font = \small}]
        x \arrow[r,mapsto, "f"] & f(x) \arrow[r,mapsto, "g"] & x
      \end{tikzcd}
    \end{equation*}
  \end{minipage}
\end{figure}

\br

$g$で戻した後に$f$を適用すると元に戻るとき、$g$は$f$の\keyword{右逆写像}と呼ばれる。

\begin{definition}{右逆写像}
  写像$f\colon X \to Y$に対して、写像$g\colon Y \to X$が存在して、
  \begin{equation*}
    f \circ g = \id_Y
  \end{equation*}
  を満たすとき、$g$は$f$の\keyword{右逆写像}であるという。
\end{definition}

\begin{figure}[h]
  \centering
  % --- 右逆写像:f ∘ g = id_Y(集合レベル) ---
  \begin{minipage}[t]{0.45\textwidth}
    \caption*{\bfseries 集合レベルでの右逆写像}
    \begin{equation*}
\begin{tikzcd}[every label/.append style = {font = \small}]
Y \arrow[r, "g"] \arrow[dr, swap, "\id_Y"] & X \arrow[d, "f"] \\
& Y
\end{tikzcd}
\end{equation*}
  \end{minipage}
  % --- 右逆写像:要素レベル(y \in Y に対して f(g(y)) = y) ---
  \begin{minipage}[t]{0.45\textwidth}
    \caption*{\bfseries 要素レベルでの右逆写像}
    \begin{equation*}
\begin{tikzcd}[every label/.append style = {font = \small}]
y \arrow[r,mapsto, "g"] & g(y) \arrow[r,mapsto, "f"] & y
\end{tikzcd}
\end{equation*}
  \end{minipage}
\end{figure}

\subsection{全単射の特徴づけ}

\begin{theorem}{全単射と逆写像の存在}
  写像$f\colon X \to Y$に対して、次の2つは同値である。
  \begin{enumerate}[label=\romanlabel]
    \item $f$は全単射である
    \item $f$の左逆写像であり、右逆写像でもある写像$g\colon Y \to X$が存在する
  \end{enumerate}
\end{theorem}

\begin{proof}
  \todo{\refbookC p62}
\end{proof}

逆写像は、次のように定義することもできる。

\begin{definition}{逆写像(恒等写像による定義)}
  写像$g \colon Y \to X$が写像$f \colon X \to Y$の\keyword{逆写像}であるとは、次が成り立つことをいう。
  \begin{equation*}
    g \circ f = \id_X \quad \text{かつ} \quad f \circ g = \id_Y
  \end{equation*}
\end{definition}

\end{document}
