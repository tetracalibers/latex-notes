\documentclass[../../../topic_linear-algebra]{subfiles}

\usepackage{xr-hyper}
\externaldocument{../../../.tex_intermediates/topic_linear-algebra}

\begin{document}

\sectionline
\section{単射と全射}
\marginnote{
  \refbookS p10、p23 \\
  \refweb{全射・単射・全単射の定義をわかりやすく~具体例を添えて~}{https://mathlandscape.com/bijection/}
}

写像の性質として、次の2点が重要になる。
\begin{enumerate}[label=\romanlabel]
  \item 異なる2元を写して同じ元になるか?
  \item 終域のどんな元も、定義域の元になるか?($y$に写るような$x$があるか?)
\end{enumerate}

これら2つの視点から、\keyword{単射}と\keyword{全射}が定義される。
\begin{enumerate}[label=\romanlabel]
  \item 「異なる元は異なる元に写る」という性質を満たすとき、写像は\keyword{単射}であるという。
  \item 「どんな$y$も$X$の元の像となる」という性質を満たすとき、写像は\keyword{全射}であるという。
\end{enumerate}

\subsection{単射の定義とイメージ}

\begin{definition}{単射}{injective-map}
  写像$f\colon X \to Y$が\keyword{単射}であるとは、$X$の任意の要素$x_1, x_2$に対して、次が成り立つことをいう。
  \begin{equation*}
    f(x_1) = f(x_2) \implies x_1 = x_2
  \end{equation*}
\end{definition}

この条件の対偶をとると、
\begin{equation*}
  x_1 \ne x_2 \implies f(x_1) \ne f(x_2)
\end{equation*}
となる。つまり、\keyword{単射}であるということは、
\begin{emphabox}
  \begin{spacebox}
    \begin{center}
      異なる元が$f$によって同じ元に対応することはない
    \end{center}
  \end{spacebox}
\end{emphabox}
ということにほかならない。

\br

これは、元々の集合の元が、$f$で写しても他の元と重なってしまうようなことはない、ということを意味する。
情報が失われることなく、すべての元がそのまま写るので、
\begin{emphabox}
  \begin{spacebox}
    \begin{center}
      \keyword{単射}な写像は、定義域を終域の中にそっくり「コピーする」
    \end{center}
  \end{spacebox}
\end{emphabox}
と考えることができる。

\begin{center}
  \begin{tikzpicture}
    \def\rA{1.5cm}       % 左の円の半径
    \def\rB{2cm}         % 右の円の半径
    \def\rCopy{1cm}      % 右の円の中の小円の半径
    \def\dotSize{2pt}    % 点のサイズ
    \def\spaceX{3}       % 中心座標のX距離

    \def\leftX{-\spaceX}
    \def\rightX{\spaceX}

    % 半径2cmの円を2つ横並びに配置
    \begin{scope}[local bounding box=circleX]
      \draw[lightslategray,thick,fill=SkyBlue!40] (\leftX,0) circle (\rA);
    \end{scope}
    \draw[lightslategray,thick] (\rightX,0) circle (\rB);

    % 2つの円の間に矢印
    \draw[->, thick] (-1,0) -- (0.5,0) node[midway, above] {\textbf{単射}};

    % 右の円の中にさらに小さな円を配置
    \begin{scope}[local bounding box=circleXcopy]
      \draw[fill=SkyBlue!40, draw=lightslategray] (\rightX,0) circle (\rCopy);
    \end{scope}

    % 左の円の中に3つの点を不規則に配置
    \fill (\leftX+0.25,0.5) circle (\dotSize);
    \fill (\leftX,-0.5) circle (\dotSize);
    \fill (\leftX+0.75,-0.2) circle (\dotSize);

    % 右の円の中の小さな円の中に3つの点を不規則に配置
    \fill (\rightX+0.2,0.25) circle (\dotSize);
    \fill (\rightX,-0.5) circle (\dotSize);
    \fill (\rightX-0.5,-0.2) circle (\dotSize);
    
    % 小さな円からはみ出している点
    %\fill (\rightX+1.2,-0.9) circle (\dotSize);

    % 曲がった矢印を2つの点に向けて引く
    \draw[RoyalBlue, |->, very thick, shorten >= 4pt, shorten <= 4pt]
    (\leftX+0.25,0.5) to[out=45,in=135] (\rightX+0.2,0.25);
    \draw[RoyalBlue, |->, very thick, shorten >= 4pt, shorten <= 4pt]
    (\leftX,-0.5) to[out=-45,in=-135] (\rightX,-0.5);
    \draw[RoyalBlue, |->, very thick, shorten >= 4pt, shorten <= 4pt]
    (\leftX+0.75,-0.2) to[out=-35,in=-145] (\rightX-0.5,-0.2);

    % ラベル
    \node[yshift=2ex] at (\leftX,\rA) {\large $X$};
    \node[yshift=2ex] at (\rightX,\rB) {\large $Y$};
    \node[Cerulean] at (\rightX,1.25) {\small\bfseries $X$のコピー};
    
    \draw[lightslategray] (circleX.north) -- (circleXcopy.north);
    \draw[lightslategray] (circleX.south) -- (circleXcopy.south);
  \end{tikzpicture}
\end{center}

また、単射は\defref{def:preimage-of-element}を用いて定義することもできる。

\begin{definition*}{単射(逆像による定義)}
  写像$f\colon X \to Y$が\keyword{単射}であるとは、任意の$y \in Y$に対し、逆像$f^{-1}(y)$の元の個数が$1$以下であることをいう。
\end{definition*}

$y$の逆像は、$f$によって$y$に写るような$x \in X$の集合である。

この定義でも、同じ$y$に写るような$x$がただ一つしかない(もしくは存在しない)ことが\keyword{単射}だと述べている。

\subsection{全射の定義とイメージ}

\begin{definition}{全射}{surjective-map}
  写像$f\colon X \to Y$が\keyword{全射}であるとは、$f$の像$f(X)$が$Y$に一致することをいう。
  \begin{equation*}
    f(X) = Y
  \end{equation*}
\end{definition}

$X$から$Y$への写像$f$が\keyword{全射}であるとは、
\begin{emphabox}
  \begin{spacebox}
    \begin{center}
      $Y$のすべての元が、$X$のある元の像となる
    \end{center}
  \end{spacebox}
\end{emphabox}
ということを意味する。

\br

つまり、
\begin{emphabox}
  \begin{spacebox}
    \begin{center}
      \keyword{全射}な写像は、定義域の元の像で終域を「埋め尽くす」
    \end{center}
  \end{spacebox}
\end{emphabox}
と考えることができる。

\begin{center}
  \begin{tikzpicture}
    \def\rA{1.5cm}       % 左の円の半径
    \def\rB{2cm}         % 右の円の半径
    \def\dotSize{2pt}    % 点のサイズ
    \def\spaceX{3}       % 中心座標のX距離

    \def\leftX{-\spaceX}
    \def\rightX{\spaceX}

    % 半径2cmの円を2つ横並びに配置
    \begin{scope}[local bounding box=circleX]
      \draw[lightslategray,thick] (\leftX,0) circle (\rA);
    \end{scope}
    \begin{scope}[local bounding box=circleY]
      \draw[lightslategray,thick,fill=Rhodamine!40] (\rightX,0) circle (\rB);
    \end{scope}

    % 2つの円の間に矢印
    \draw[->, thick] (-1,0) -- (0.5,0) node[midway, above] {\textbf{全射}};

    % 左の円の中に3つの点を不規則に配置
    \fill (\leftX+0.25,0.5) circle (\dotSize);
    \fill (\leftX,-0.5) circle (\dotSize);
    \fill (\leftX+0.75,-0.2) circle (\dotSize);

    % 右の円の中の小さな円の中に2つの点を不規則に配置
    \fill (\rightX+0.2,0.25) circle (\dotSize);
    \fill (\rightX,-0.5) circle (\dotSize);
    
    % 曲がった矢印を2つの点に向けて引く
    \draw[RoyalBlue, |->, very thick, shorten >= 4pt, shorten <= 4pt]
    (\leftX+0.25,0.5) to[out=35,in=145] (\rightX+0.2,0.25);
    \draw[RoyalBlue, |->, very thick, shorten >= 4pt, shorten <= 4pt]
    (\leftX,-0.5) to[out=-35,in=-145] (\rightX,-0.5);
    \draw[RoyalBlue, |->, very thick, shorten >= 4pt, shorten <= 4pt]
    (\leftX+0.75,-0.2) to[out=-35,in=-145] (\rightX,-0.5);
    
    % ラベル
    \node[yshift=2ex] at (\leftX,\rA) {\large $X$};
    \node[yshift=2ex] at (\rightX,\rB) {\large $Y = f(X)$};
    
    \draw[lightslategray] (circleX.north) -- (circleY.north);
    \draw[lightslategray] (circleX.south) -- (circleY.south);
  \end{tikzpicture}
\end{center}

\subsection{全単射の定義とイメージ}

\begin{definition*}{全単射}
  写像$f$が\keyword{全単射}であるとは、$f$が単射かつ全射であることをいう。
\end{definition*}

単射と全射は、それぞれ次の条件を満たす性質だった。
\begin{itemize}
  \item \keyword{単射}:$y$に写るような$x$はただ一つしかない(異なる元は異なる元に写る)
  \item \keyword{全射}:どんな$y$も$X$の元の像$x$になる(すべての元に漏れなく写る)
\end{itemize}

これらの両方を満たす性質である\keyword{全単射}は、次のような性質だといえる。
\begin{emphabox}
  \begin{spacebox}
    \begin{center}
      どんな$y$も、$x$に対するただ一つの像となる
    \end{center}
  \end{spacebox}
\end{emphabox}

\begin{center}
  \begin{tikzpicture}
    \def\rA{1.5cm}       % 左の円の半径
    \def\rB{2cm}         % 右の円の半径
    \def\rCopy{1cm}      % 右の円の中の小円の半径
    \def\dotSize{2pt}    % 点のサイズ
    \def\spaceX{3}       % 中心座標のX距離

    \def\leftX{-\spaceX}
    \def\rightX{\spaceX}

    % 半径2cmの円を2つ横並びに配置
    \begin{scope}[local bounding box=circleX]
      \draw[lightslategray,thick,fill=SkyBlue!40] (\leftX,0) circle (\rA);
    \end{scope}
    \begin{scope}[local bounding box=circleY]
      \draw[lightslategray,thick, fill=Rhodamine!40] (\rightX,0) circle (\rB);
    \end{scope}

    % 2つの円の間に矢印
    \draw[->, thick] (-1,0) -- (0.5,0) node[midway, above] {\textbf{全単射}};
    
    % 左の円の中に3つの点を不規則に配置
    \fill (\leftX+0.25,0.5) circle (\dotSize);
    \fill (\leftX,-0.5) circle (\dotSize);
    \fill (\leftX+0.75,-0.2) circle (\dotSize);

    % 右の円の中の小さな円の中に3つの点を不規則に配置
    \fill (\rightX+0.2,0.25) circle (\dotSize);
    \fill (\rightX,-0.5) circle (\dotSize);
    \fill (\rightX-0.5,-0.2) circle (\dotSize);
    
    % 曲がった矢印を2つの点に向けて引く
    \draw[RoyalBlue, |->, very thick, shorten >= 4pt, shorten <= 4pt]
    (\leftX+0.25,0.5) to[out=35,in=145] (\rightX+0.2,0.25);
    \draw[RoyalBlue, |->, very thick, shorten >= 4pt, shorten <= 4pt]
    (\leftX,-0.5) to[out=-35,in=-145] (\rightX,-0.5);
    \draw[RoyalBlue, |->, very thick, shorten >= 4pt, shorten <= 4pt]
    (\leftX+0.75,-0.2) to[out=-35,in=-145] (\rightX-0.5,-0.2);

    % ラベル
    \node[yshift=2ex] at (\leftX,\rA) {\large $X$};
    \node[yshift=2ex] at (\rightX,\rB) {\large $Y = f(X)$};
    
    \draw[lightslategray] (circleX.north) -- (circleY.north);
    \draw[lightslategray] (circleX.south) -- (circleY.south);
  \end{tikzpicture}
\end{center}

\subsection{全単射の一対一対応による同一視}

単射は「$X$のコピーを$Y$の中に作る」ことだった。

さらに全射でもあれば、「$X$のコピーは$Y$に一致する」ことになる。

\br

そのため、$X$と$Y$の間の写像$f$が\keyword{全単射}であれば、$X$と$Y$を\keywordJE{同一視}{identify}する(同じ集合とみなす)という考え方もできる。

\begin{emphabox}
  \begin{spacebox}
    \begin{center}
      \keyword{全単射}な写像$f$により、$X$の元と$Y$の元は\keyword{一対一に対応}する
    \end{center}
  \end{spacebox}
\end{emphabox}

$X$と$Y$を同一視するときには、どの対応により$X$と$Y$が同じものであると考えているかを明らかにしておく必要がある。

\end{document}
