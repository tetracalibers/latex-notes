\documentclass[../../../topic_linear-algebra]{subfiles}

\begin{document}

\sectionline
\section{ベクトルの集合が張る空間}
\marginnote{\refbookA p6〜8}

\begin{definition}{ベクトルの集合が張る空間}\label{def:span-of-vectors}
  $k$個のベクトル$\vb*{a}_1, \vb*{a}_2, \dots, \vb*{a}_k \in \mathbb{R}^n$を与えたとき、$\vb*{a}_1, \vb*{a}_2, \dots, \vb*{a}_k$の線形結合全体の集合を
  \begin{equation*}
    \langle \vb*{a}_1, \vb*{a}_2, \dots, \vb*{a}_k \rangle
  \end{equation*}
  によって表し、これを$\vb*{a}_1, \vb*{a}_2, \dots, \vb*{a}_k$が\keyword{張る空間}という
\end{definition}

\end{document}
