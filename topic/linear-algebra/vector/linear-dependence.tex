\documentclass[../../../topic_linear-algebra]{subfiles}

\begin{document}

\sectionline
\section{線形従属}

ベクトルの組を考え、どれか1つのベクトルがほかのベクトルの線形結合で表せるとき、それらのベクトルの組は\keyword{線形従属}であるという。

\begin{definition}{線形従属}
  $k$個のベクトル$\{ \vb*{a}_1, \ldots, \vb*{a}_k \}$が\keyword{線形従属}であるとは、少なくとも1つは$0$でない$k$個の係数$\{c_1, \ldots, c_k\}$を用意すれば、それらを使った線形結合を零ベクトル$\vb*{o}$にできることをいう。
  \begin{equation*}
    c_1 \vb*{a}_1 + c_2 \vb*{a}_2 + \cdots + c_k\vb*{a}_{k} = \vb*{o}
  \end{equation*}
\end{definition}

たとえば、$c_1$が$0$でないとき、線形結合を零ベクトルにできるということは、次のような式変形ができることになる。
\begin{equation*}
  \vb*{a}_1 = -\frac{c_2}{c_1} \vb*{a}_2 - \frac{c_3}{c_1} \vb*{a}_3 - \cdots - \frac{c_k}{c_1} \vb*{a}_{k}
\end{equation*}
つまり、ベクトル$\vb*{a}_1$をほかのベクトルの線形結合で表せている。

\subsection{「従属」という言葉を味わう}

自分自身をほかのベクトルを使って表現できるということは、ほかのベクトルに依存している(従っている)ということになる。

\br

たとえば、$\vb*{a}_1$と$\vb*{a}_2$の線形結合で表せるベクトル$\vb*{a}_3$は、 この2つのベクトル$\vb*{a}_1 ,\, \vb*{a}_2$に従っているといえる。
\begin{equation*}
  \vb*{a}_3 = 2 \vb*{a}_1 + \vb*{a}_2
\end{equation*}

しかし、「$\vb*{a}_3$が$\vb*{a}_1 ,\, \vb*{a}_2$に従っている」という一方的な主従関係になっているわけではない。その逆もまた然りである。

なぜなら、次のような式変形もできるからだ。
\begin{equation*}
  \vb*{a}_2 = \vb*{a}_3 - 2 \vb*{a}_1
\end{equation*}
この式で見れば、今度は$\vb*{a}_2$が$\vb*{a}_1 ,\, \vb*{a}_3$に従っていることになる。

\br

このように、線形従属とは、「どちらがどちらに従う」という主従関係ではなく、ベクトルの組の中での相互の依存関係を表すものである。

\end{document}
