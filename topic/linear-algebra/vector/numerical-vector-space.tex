\documentclass[../../../topic_linear-algebra]{subfiles}

\usepackage{xr-hyper}
\externaldocument{../../../.tex_intermediates/topic_linear-algebra}

\begin{document}

\sectionline
\section{体と数ベクトル空間}
\marginnote{\refbookS p1〜2}

$n$個の数を縦に並べたベクトル($n$次元ベクトル)をすべて集めた集合を、\keyword{数ベクトル空間}といい、$K^n$と表す。
ここで、$K$は\keyword{体}を表している。

\br

\begin{handout}[補足:体とはなにか?]
  足し算・引き算・掛け算・(0でない数での)割り算が「普通の感覚で」できて、計算の基本法則(結合法則、交換法則、分配法則など)がすべて成り立つ数の集まりのことを\keywordJE{体}{field}という。
\end{handout}

具体的には、$K = \mathbb{R}$(実数全体の集合)や$K = \mathbb{C}$(複素数全体の集合)の場合を考えることが多い。
\begin{itemize}
  \item $n$個の\keyword{実数}を縦に並べたベクトルの集合(\keyword{実ベクトル空間})は、$\mathbb{R}^n$と表す。
  \item $n$個の\keyword{複素数}を縦に並べたベクトルの集合(\keyword{複素ベクトル空間})は、$\mathbb{C}^n$と表す。
\end{itemize}

\end{document}
