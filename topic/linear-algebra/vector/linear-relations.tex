\documentclass[../../../topic_linear-algebra]{subfiles}

\begin{document}

\sectionline
\section{線形関係式}

線形従属や線型独立の定義では、
\begin{equation*}
  \text{線形結合} = \vb*{o}
\end{equation*}
という関係式を考えた。
以降、この関係式を\keyword{線形関係式}と呼ぶことにする。

\begin{definition*}{線形関係式}
  ベクトル$\vb*{a}_1, \dots, \vb*{a}_k$に対する等式
  \begin{equation*}
    c_1 \vb*{a}_1 + \cdots + c_k \vb*{a}_k = \vb*{o}
  \end{equation*}
  を、$\vb*{a}_1, \dots, \vb*{a}_k$の\keyword{線形関係式}という。
\end{definition*}

特に、$c_1 = \cdots = c_k = 0$として得られる線形関係式を\keyword{自明な線形関係式}という。

これ以外の場合、つまり$c_i \neq 0$となるような$i$が少なくとも1つあるならば、これは\keyword{非自明な線形関係式}である。

\end{document}
