\documentclass[../../../topic_linear-algebra]{subfiles}

\begin{document}

\sectionline
\section{高次元への対応:数ベクトル}

2次元以上の空間内の「移動」を表すには、「縦」と「横」などといった2方向だけでなく、もっと多くの方向への移動量を組み合わせて考える必要がある。

また、4次元を超えてしまうと、矢印の描き方すら想像がつかなくなってしまう。
それは、方向となる軸が多すぎて、どの方向に進むかを表すのが難しくなるためだ。

\br

そこで、一旦「向き」の情報を取り除くことで、高次元に立ち向かえないかと考える。

移動を表す矢印は「どの方向に進むか」と「どれくらい進むか」という向きと大きさの情報を持っているが、その「どれくらい進むか」だけを取り出して並べよう。

\begin{figure}[h]
  \centering
  \begin{minipage}{0.4\columnwidth}
    \centering
    \scalebox{1.2}{
      \begin{tikzpicture}
        \def\vx{2}
        \def\vy{1.5}

        \coordinate (A) at (0,0);
        \coordinate (B) at (\vx, \vy);

        % x軸方向のベクトル
        \draw[vector, dashed, very thick, Rhodamine] (0,0) -- (\vx, 0) node[below, midway] {$a_1$};
        % y軸方向のベクトル
        \draw[vector, dashed, very thick, Cerulean] (\vx, 0) -- (\vx, \vy) node[right, midway] {$a_2$};

        % ベクトル
        \draw[vector, very thick, BurntOrange] (A) -- (B) node[midway, above left] {$\vb*{a}$};
      \end{tikzpicture}
    }
  \end{minipage}
  \begin{minipage}{0.4\columnwidth}
    \centering
    \large
    \color{BurntOrange}
    \begin{equation*}
      \vb*{a} = \begin{pmatrix} \textcolor{Rhodamine}{a_1} \\ \textcolor{Cerulean}{a_2} \end{pmatrix}
    \end{equation*}
  \end{minipage}
\end{figure}

こうして単に「数を並べたもの」もベクトルと呼ぶことにし、このように定義したベクトルを\keyword{数ベクトル}という。

\br

数を並べるとき、縦と横の2通りがある。それぞれ\keyword{列ベクトル}、\keyword{行ベクトル}として定義する。

\begin{definition}{列ベクトル}
  数を縦に並べたものを\keyword{列ベクトル}という。
  \begin{equation*}
    \vb*{a} = (a_i) = \begin{pmatrix} a_1 \\ a_2 \\ \vdots \\ a_n \end{pmatrix}
  \end{equation*}
\end{definition}

\begin{definition}{行ベクトル}
  数を横に並べたものを\keyword{行ベクトル}という。
  \begin{equation*}
    \vb*{a} = (a_i) = \begin{pmatrix} a_1 & a_2 & \cdots & a_n \end{pmatrix}
  \end{equation*}
\end{definition}

単に「ベクトル」と言った場合は、列ベクトルを指すことが多い。

\br

行ベクトルは、列ベクトルを横倒しにしたもの(列ベクトルの\keyword{転置})と捉えることもできる。

\begin{theorem}{転置による行ベクトルの表現}
  行ベクトルは、列ベクトル$\vb*{a}$を\keyword{転置}したものとして表現できる。
  \begin{equation*}
    \vb*{a}^\top = \begin{pmatrix} a_1 & a_2 & \cdots & a_n \end{pmatrix}
  \end{equation*}
\end{theorem}

\sectionline
\section{ベクトルの和}

ベクトルによって数をまとめて扱えるようにするために、ベクトルどうしの演算を定義したい。

\br

ベクトルどうしの足し算は、同じ位置にある数どうしの足し算として定義する。

\begin{definition}{ベクトルの和}
  2つの$n$次元ベクトル$\vb*{a}$と$\vb*{b}$の和を次のように定義する。
  \begin{equation*}
    \vb*{a} + \vb*{b} = (a_i) + (b_i) = \begin{pmatrix} a_1 + b_1 \\ a_2 + b_2 \\ \vdots \\ a_n + b_n \end{pmatrix}
  \end{equation*}
\end{definition}

$i$番目の数が$\vb*{a}$と$\vb*{b}$の両方に存在していなければ、その位置の数どうしの足し算を考えることはできない。

そのため、ベクトルの和が定義できるのは、同じ次元を持つ(並べた数の個数が同じ)ベクトルどうしに限られる。

\subsection{移動の合成としてのイメージ}

数ベクトルを「どれくらい進むか」を並べたものと捉えると、同じ位置にある数どうしを足し合わせるということは、同じ向きに進む量を足し合わせるということになる。

たとえば、$x$軸方向に$a_1$、$y$軸方向に$a_2$進んだ場所から、さらに$x$軸方向に$b_1$、$y$軸方向に$b_2$進む…というような「移動の合成」を表すのが、ベクトルの和である。

\begin{figure}[h]
  \centering
  \begin{minipage}[c]{0.49\columnwidth}
    \centering
    \scalebox{0.9}{
      \begin{tikzpicture}
        \def\xmin{0}
        \def\xmax{5}
        \def\ymin{0}
        \def\ymax{4.5}
        \def\ax{2}
        \def\ay{1}
        \def\bx{0.5}
        \def\by{2}

        \coordinate (S) at (1,1);

        % 0.5刻みのグリッド
        \draw[dotted, lightslategray] (\xmin, \ymin) grid[step=0.5] (\xmax, \ymax);

        % 座標軸
        \draw[axis] (\xmin, 0) -- (1,0) node[right] {$x$};
        \draw[axis] (0, \ymin) -- (0, 1) node[above] {$y$};

        % ベクトルaのx成分
        \draw[thick, densely dashed, Rhodamine] (S) -- ++(\ax, 0) node[below, midway] {\small$a_1$};
        % ベクトルaのy成分
        \draw[thick, densely dashed, Cerulean] ($(S)+(\ax,0)$) -- ++(0, \ay) node[left, midway] {\small$a_2$};

        % ベクトルbのx成分
        \draw[thick, densely dashed, Rhodamine] ($(S)+(\ax,\ay)$) -- ++(\bx, 0) node[below, midway] {\small$b_1$};
        % ベクトルbのy成分
        \draw[thick, densely dashed, Cerulean] ($(S)+(\ax,\ay)+(\bx, 0)$) -- ++(0, \by) node[right, midway] {\small$b_2$};

        \begin{scope}[transparency group, opacity=0.9]
          % ベクトルa
          \draw[vector, very thick, Orchid] (S) -- ($(S)+(\ax,\ay)$) node[above, midway] {$\vb*{a}$};
          % ベクトルb
          \draw[vector, very thick, Orchid] ($(S)+(\ax,\ay)$) -- ++(\bx,\by) node[left, midway] {$\vb*{b}$};
        \end{scope}

        % ベクトルa+b
        \draw[vector, very thick, BurntOrange] (S) -- ++(\ax+\bx, \ay+\by) node[above, sloped, midway] {$\vb*{a}+\vb*{b}$};
      \end{tikzpicture}
    }
  \end{minipage}
  \begin{minipage}[c]{0.49\columnwidth}
    \centering
    \scalebox{0.9}{
      \begin{tikzpicture}
        \def\xmin{0}
        \def\xmax{5}
        \def\ymin{0}
        \def\ymax{4.5}
        \def\ax{2}
        \def\ay{1}
        \def\bx{0.5}
        \def\by{2}

        \coordinate (S) at (1,1);

        % 0.5刻みのグリッド
        \draw[dotted, lightslategray] (\xmin, \ymin) grid[step=0.5] (\xmax, \ymax);

        % 座標軸
        \draw[axis] (\xmin, 0) -- (1,0) node[right] {$x$};
        \draw[axis] (0, \ymin) -- (0, 1) node[above] {$y$};

        % ベクトルa+bのx成分
        \draw[thick, densely dashed, Rhodamine] (S) -- ++($(\ax,0)+(\bx,0)$) node[below, midway] {\small$a_1+b_1$};
        % ベクトルa+bのy成分
        \draw[thick, densely dashed, Cerulean] ($(S)+(\ax,0)+(\bx,0)$) -- ++($(0,\ay)+(0,\by)$) node[right, midway] {\small$a_2+b_2$};

        \begin{scope}[transparency group, opacity=0.45]
          % ベクトルa
          \draw[vector, very thick, lightslategray] (S) -- ($(S)+(\ax,\ay)$) node[above, midway] {$\vb*{a}$};
          % ベクトルb
          \draw[vector, very thick, lightslategray] ($(S)+(\ax,\ay)$) -- ++(\bx,\by) node[left, midway] {$\vb*{b}$};
        \end{scope}

        % ベクトルa+b
        \draw[vector, very thick, BurntOrange] (S) -- ++(\ax+\bx, \ay+\by) node[above, sloped, midway] {$\vb*{a}+\vb*{b}$};
      \end{tikzpicture}
    }
  \end{minipage}
\end{figure}

\subsection{平行四辺形の法則}

\todo{平行移動しても同じベクトルなので…}

\subsection{ベクトルの差:逆向きにしてから足す}

\todo{irobutsu-linear-algebra 2.1.2 ベクトルの差}

\subsection{矢に沿った移動で考える}

\todo{手持ちの画像を参考に、和と差の両方について書く}

\sectionline
\section{ベクトルのスカラー倍}

「どれくらい進むか」を表す数たち全員に同じ数をかけることで、向きを変えずにベクトルを「引き伸ばす」ことができる。

\begin{center}
  \begin{tikzpicture}
    \coordinate (a) at (0,0);
    \coordinate (t) at (4,2);
    \coordinate (b) at ($(a)!1.5cm!(t)$);
    \draw [vector, thick, TealBlue] (a) -- (t) node[midway, below=0.5em] {$k\vb*{a}$};

    \coordinate (c) at ($(a)!0.2cm!90:(b)$);
    \coordinate (d) at ($(b)+(c)-(a)$);
    \draw [vector, thick, BurntOrange] (c) -- node[midway,auto] {$\vb*{a}$} (d);
  \end{tikzpicture}
\end{center}

ここで向きごとにかける数を変えてしまうと、いずれかの方向に多く進むことになり、ベクトルの向きが変わってしまう。そのため、「同じ」数をかけることに意味がある。

\begin{figure}[H]
  \centering
  \begin{minipage}{0.49\columnwidth}
    \centering
    \scalebox{0.9}{
      \begin{tikzpicture}
        \def\xmin{0}
        \def\xmax{5}
        \def\ymin{0}
        \def\ymax{5.5}
        \def\ax{1}
        \def\ay{0.5}
        \def\k{3}

        \coordinate (S) at (1,1);

        % 0.5刻みのグリッド
        \draw[dotted, lightslategray] (\xmin, \ymin) grid[step=0.5] (\xmax, \ymax);

        % ベクトルaのx成分
        \draw[thick, densely dashed, Rhodamine] (S) -- ++(\ax, 0) node[below, midway] {\small$a_1$};
        % ベクトルaのy成分
        \draw[thick, densely dashed, Cerulean] ($(S)+(\ax,0)$) -- ++(0, \ay) node[right, midway] {\small$a_2$};

        % ベクトルkaのx成分
        \draw[thick, densely dashed, Rhodamine] ([yshift=1.5cm]S) -- ++(\k*\ax, 0) node[below, midway] {\small$3a_1$};
        % ベクトルkaのy成分
        \draw[thick, densely dashed, Cerulean] ([yshift=1.5cm]$(S)+(\k*\ax, 0)$) -- ++(0, \k*\ay) node[right, midway] {\small$3a_2$};

        % ベクトルa
        \draw[vector, very thick, BurntOrange] (S) -- ($(S)+(\ax,\ay)$) node[above=0.1em, pos=0.4] {$\vb*{a}$};

        % ベクトルkaを上に平行移動したベクトル
        \draw[vector, very thick, TealBlue] ([yshift=1.5cm]S) -- ([yshift=1.5cm]$(S)+\k*(\ax,\ay)$) node[midway, auto] {$3\vb*{a}$};
      \end{tikzpicture}
    }
  \end{minipage}
  \begin{minipage}{0.49\columnwidth}
    \centering
    \scalebox{0.9}{
      \begin{tikzpicture}
        \def\xmin{0}
        \def\xmax{5}
        \def\ymin{0}
        \def\ymax{5.5}
        \def\ax{1}
        \def\ay{0.5}
        \def\kx{3}
        \def\ky{5}

        \coordinate (S) at (1,1);

        % 0.5刻みのグリッド
        \draw[dotted, lightslategray] (\xmin, \ymin) grid[step=0.5] (\xmax, \ymax);

        % ベクトルaのx成分
        \draw[thick, densely dashed, Rhodamine] (S) -- ++(\ax, 0) node[below, midway] {\small$a_1$};
        % ベクトルaのy成分
        \draw[thick, densely dashed, Cerulean] ($(S)+(\ax,0)$) -- ++(0, \ay) node[right, midway] {\small$a_2$};

        % ベクトルkaのx成分
        \draw[thick, densely dashed, Rhodamine] ([yshift=1.5cm]S) -- ++(\kx*\ax, 0) node[below, midway] {\small$3a_1$};
        % ベクトルkaのy成分
        \draw[thick, densely dashed, Cerulean] ([yshift=1.5cm]$(S)+(\kx*\ax, 0)$) -- ++(0, \ky*\ay) node[right, midway] {\small$5a_2$};

        % ベクトルa
        \draw[vector, very thick, BurntOrange] (S) -- ($(S)+(\ax,\ay)$) node[above=0.1em, pos=0.4] {$\vb*{a}$};

        % ベクトルkaを上に平行移動したベクトル
        \draw[vector, very thick, TealBlue] ([yshift=1.5cm]S) -- ([yshift=1.5cm]$(S)+(\kx*\ax,\ky*\ay)$) node[midway, auto] {$\vb*{?}$};
      \end{tikzpicture}
    }
  \end{minipage}
\end{figure}

そこで、ベクトルの定数倍(スカラー倍)を次のように定義する。

\begin{definition}{ベクトルのスカラー倍}
  $n$次元ベクトル$\vb*{a}$の$k$倍を次のように定義する。
  \begin{equation*}
    k\vb*{a} = k(a_i) = \begin{pmatrix} ka_1 \\ ka_2 \\ \vdots \\ ka_n \end{pmatrix}
  \end{equation*}
\end{definition}

\end{document}
