\documentclass[../../../topic_linear-algebra]{subfiles}

\begin{document}

\sectionline
\section{基底にできるベクトルの条件}
\marginnote{\refbookL p17〜18}

座標系では、空間内(たとえば2次元空間であれば平面上)のあらゆる点を表すことができ、それらの点はベクトルで指し示す形でも表現できる。

基底が「座標系を設置するための土台」となるなら、基底とは、あらゆるベクトルを表すための材料とみなすことができる。

\br

では、基底として使えるベクトルとは、どのようなベクトルだろうか?

\subsection{基底とは過不足ない組み合わせ}

まず、座標系とは、次の条件を当たり前に満たすものである。

\begin{enumerate}[label=\romanlabel]
  \item あらゆる点を表すことができる
  \item 1つの点の表し方は一通りである
\end{enumerate}

数ベクトルは基底が決まれば座標として使えるので、座標系の条件をベクトルの言葉に言い換えると、
\begin{enumerate}[label=\romanlabel]
  \item 基底が決まれば、その線形結合であらゆるベクトルを表せる
  \item 基底が決まれば、1つのベクトルの線形結合は一通りに定まる
\end{enumerate}

つまり、空間内のすべてのベクトルを表現する上で、基底は不十分であってはいけないし、無駄があってもいけない。

\subsection{不十分を考える(2次元平面の例)}

たとえば、2次元座標系を表現するにあたって、必ずしも基底ベクトルが直交している必要はない。

しかし、平行なベクトルは明らかに基底(座標軸の土台)として使うことはできない。

\begin{center}
  \begin{tikzpicture}
    \def\a{1.5}
    \def\b{2.5}

    % ベクトルa_1を引き伸ばしたx軸
    \draw[axis] (0, 0) -- (3*\a, 0) node[right] {$x$};
    % ベクトルa_1
    \draw[vector, Rhodamine] (0, 0) -- (\a, 0) node[midway, above] {$\vb*{a}_1$};

    \begin{scope}[yshift=-1em]
      % ベクトルa_2を引き伸ばしたy軸
      \draw[axis] (0, 0) -- (2.5*\b, 0) node[right] {$y$};
      % ベクトルa_2
      \draw[vector, Cerulean] (0, 0) -- (\b, 0) node[midway, below] {$\vb*{a}_2$};
    \end{scope}
  \end{tikzpicture}
\end{center}

$x$軸と$y$軸が平行だと、$(x,y)$の組で平面上の点を表すことはできない。

2次元平面$\mathbb{R}^2$上の点やベクトルは、2つの方向を用意しないと表せないのだから、基底となるベクトルは互いに平行でない必要がある。

\subsection{無駄を考える(2次元平面の例)}

平行な2つのベクトルは、互いに互いをスカラー倍で表現できてしまう。このようなベクトルの組は基底にはできない。

\begin{equation*}
  \vb*{a}_2 = k \vb*{a}_1
\end{equation*}

この平行な2つのベクトル$\{\vb*{a}_1,\vb*{a}_2\}$に加えて、これらに平行でないもう1つのベクトル$\vb*{a}_3$を用意すれば、$\vb*{a}_1$と$\vb*{a}_3$の線型結合か、$\vb*{a}_2$と$\vb*{a}_3$の線型結合かのどちらかで、平面上の他のベクトルを表現できるようになる。

\br

しかし、$\vb*{a}_2$は結局$\vb*{a}_1$のスカラー倍($\vb*{a}_1$と$\vb*{a}_3$の線型結合の特別な場合)で表現できてしまうのだから、「他のベクトルを表す材料」となるベクトルの組を考える上で、$\vb*{a}_2$は無駄なベクトルだといえる。

\br

2次元平面を表現するには2本の座標軸があれば十分なように、基底とは、「これさえあれば他のベクトルを表現できる」という、必要最低限のベクトルの組み合わせにしたい。

基底の候補の中に、互いに互いを表現できる複数のベクトルが含まれているなら、その中の1つを残せば十分である。

\sectionline

ここまでの考察から、あるベクトルの組を基底として使えるかどうかを考える上で、「互いに互いを表現できるか」という視点が重要になることがわかる。

\begin{itemize}
  \item 互いに線型結合で表現できるベクトルだけでは不十分
  \item 互いに線型結合で表現できるベクトルが含まれていると無駄がある
\end{itemize}

ベクトルの組の「互いに互いを表現できるか」に着目した性質を表現する概念として、\keyword{線型従属}と\keyword{線型独立}がある。

\begin{itemize}
  \item \keyword{線型従属}:互いに互いを表現できるベクトルが含まれていること
  \item \keyword{線型独立}:互いに互いを表現できない、独立したベクトルだけで構成されていること
\end{itemize}

\end{document}
