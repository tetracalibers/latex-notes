\documentclass[../../../topic_linear-algebra]{subfiles}

\begin{document}

\sectionline
\section{線形結合}

ベクトルを「引き伸ばす」スカラー倍と、「つなぎ合わせる」足し算を組み合わせることで、あるベクトルを他のベクトルを使って表すことができる。

\begin{figure}[h]
  \centering
  \begin{minipage}{0.45\columnwidth}
    \centering
    \scalebox{1.2}{
      \begin{tikzpicture}
        \def\xmin{-0.5}
        \def\xmax{3.5}
        \def\ymin{0}
        \def\ymax{3}
        \def\ax{2.5}
        \def\ay{2}

        \coordinate (S) at (0.5,0.5);

        % 透明なグリッド(bounding boxの調整用)
        \draw[dotted, opacity=0] (\xmin, \ymin) grid[step=0.5] (\xmax, \ymax);

        % ベクトル
        \draw[vector, very thick, BurntOrange] (S) -- ++(\ax,\ay) node[midway, auto] {$\vb*{a}_3$};

        % x成分とy成分
        \draw[vector, dashed, very thick, CarnationPink] (S) -- ++(\ax,0) node[below, pos=0.7] {$\lambda_1 \vb*{a}_1$};
        \draw[vector, dashed, very thick, SkyBlue] (S) -- ++(0,\ay) node[left, pos=0.6] {$\lambda_2 \vb*{a}_2$};

        % 基底ベクトル
        \draw[vector, very thick, Rhodamine] ([yshift=0cm]S) -- ++(1,0) node[below, pos=0.4] {$\vb*{a}_1$};
        \draw[vector, very thick, Cerulean] ([xshift=0cm]S) -- ++(0,0.5) node[left, near start] {$\vb*{a}_2$};
      \end{tikzpicture}
    }
  \end{minipage}
  \begin{minipage}{0.45\columnwidth}
    \large
    \begin{equation*}
      \textcolor{BurntOrange}{\vb*{a}_3} = \textcolor{CarnationPink}{\lambda_1} \textcolor{Rhodamine}{\vb*{a}_1} + \textcolor{SkyBlue}{\lambda_2} \textcolor{Cerulean}{\vb*{a}_2}
    \end{equation*}
  \end{minipage}
\end{figure}

このように、スカラー倍と和のみを使った形を\keyword{一次結合}もしくは\keyword{線形結合}という。

\end{document}
