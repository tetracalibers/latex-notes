\documentclass[../../../topic_linear-algebra]{subfiles}

\begin{document}

\sectionline
\section{移動の表現としてのベクトル}

平面上のある点の位置を表すのに、よく使われるのが\keyword{直交座標系}である。

直交座標系では、$x$軸と$y$軸を垂直に張り、
\begin{itemize}
  \item 原点$O$からの$x$軸方向の移動量($x$座標)
  \item 原点$O$からの$y$軸方向の移動量($y$座標)
\end{itemize}
という2つの数の組で点の位置を表す。

\begin{figure}[h]
  \centering
  \begin{minipage}[b]{0.49\columnwidth}
    \centering
    \scalebox{0.9}{
      \begin{tikzpicture}
        \def\xmin{-1}
        \def\xmax{4}
        \def\ymin{-1}
        \def\ymax{4}
        \def\vx{2}
        \def\vy{1.5}

        % 0.5刻みのグリッド
        \draw[dotted, lightslategray] (\xmin, \ymin) grid[step=0.5] (\xmax, \ymax);

        \draw[axis] (\xmin, 0) -- (\xmax, 0) node[right] {$x$};
        \draw[axis] (0, \ymin) -- (0, \ymax) node[above] {$y$};

        % 原点
        \node at (0, 0) [below left] {$O$};

        % x軸方向の移動量
        \draw[dashed, very thick, Rhodamine] (0, 0) -- (\vx, 0) node[below, midway] {$a$};
        % y軸方向の移動量
        \draw[dashed, very thick, Cerulean] (\vx,0) -- (\vx, \vy) node[right, midway] {$b$};

        % 点
        \draw (\vx, \vy) node[circle, fill, inner sep=1.5pt] {};
        \node at (\vx, \vy) [above right] {$(a,b)$};
      \end{tikzpicture}
    }
    \caption*{\bfseries「位置の特定」という視点}
  \end{minipage}
  \begin{minipage}[b]{0.49\columnwidth}
    \centering
    \scalebox{0.9}{
      \begin{tikzpicture}
        \def\xmin{-1}
        \def\xmax{4}
        \def\ymin{-1}
        \def\ymax{4}
        \def\vx{2}
        \def\vy{1.5}

        % 平面
        \draw[dotted, lightslategray] (\xmin, \ymin) -- (\xmax, \ymin) -- (\xmax, \ymax) -- (\xmin, \ymax) -- cycle;

        \coordinate (A) at (0,0);
        \coordinate (B) at (\vx, \vy);

        % ベクトル
        \draw[vector, very thick, BurntOrange, shorten >=0.25em, shorten <=0.25em] (A) -- (B);

        % 点A
        \draw (A) node[circle, fill, inner sep=1.5pt] {};
        \node at (A) [below left] {$A$};

        % 点B
        \draw (B) node[circle, fill, inner sep=1.5pt] {};
        \node at (B) [above right] {$B$};
      \end{tikzpicture}
    }
    \caption*{\bfseries「移動」という視点}
  \end{minipage}
\end{figure}

\br

座標とは、「$x$軸方向の移動」と「$y$軸方向の移動」という2回の移動を行った結果である。

右にどれくらい、上にどれくらい、という考え方で平面上の「位置」を特定しているわけだが、単に「移動」を表したいだけなら、点から点へ向かう矢印で一気に表すこともできる。

\br

ある地点から別のある地点への「移動」を表す矢印を\keyword{ベクトル}という。

\br

ベクトルが示す、ある地点からこのように移動すれば、この地点にたどり着く…といった「移動」の情報は、相対的な「位置関係」を表す上で役に立つ。

\subsection{平行移動してもベクトルは同じ}

座標は「位置」を表すものだが、ベクトルは「移動」を表すものにすぎない。

座標は「原点からの」移動量によって位置を表すが、ベクトルは始点の位置にはこだわらない。

\br

たとえば、次の2つのベクトルは始点の位置は異なるが、同じ向きに同じだけ移動している矢印なので、同じベクトルとみなせる。

\begin{center}
  \scalebox{1.2}{
    \begin{tikzpicture}
      \def\xmin{-1}
      \def\xmax{4}
      \def\ymin{-1}
      \def\ymax{4}
      \def\vx{1.5}
      \def\vy{2}
      \def\sx{2.5}
      \def\sy{1}

      \draw[axis] (\xmin, 0) -- (\xmax, 0) node[right] {$x$};
      \draw[axis] (0, \ymin) -- (0, \ymax) node[above] {$y$};

      % 原点
      \node at (0, 0) [below left] {$O$};

      % ベクトル
      \draw[vector, very thick, magenta] (0, 0) -- (\vx, \vy);

      % 平行移動したベクトル
      \draw[vector, very thick, magenta] (\sx, \sy) -- ($(\sx,\sy)+(\vx,\vy)$);

      % 平行移動を表す破線
      \draw[dashed, lightslategray] (0, 0) -- (\sx, \sy);
      \draw[dashed, lightslategray] (\vx, \vy) -- ($(\vx,\vy)+(\sx,\sy)$);
    \end{tikzpicture}
  }
\end{center}

このような「同じ向きに同じだけ移動している矢印」は、平面内では平行な関係にある。

つまり、平行移動して重なる矢印は、同じベクトルとみなすことができる。

\subsection{移動の合成とベクトルの分解}

ベクトルは、各方向への移動の合成として考えることもできる。

純粋に「縦」と「横」に分解した場合は直交座標の考え方によく似ているが、必ずしも直交する方向のベクトルに分解する必要はない。

\begin{figure}[h]
  \centering
  \begin{minipage}{0.49\columnwidth}
    \centering
    \scalebox{1.2}{
      \begin{tikzpicture}
        \def\xmin{-1}
        \def\xmax{4}
        \def\ymin{-1}
        \def\ymax{4}
        \def\vx{1.5}
        \def\vy{2}

        \coordinate (A) at (0,0);
        \coordinate (B) at (\vx, \vy);

        % ベクトル
        \draw[vector, very thick, BurntOrange, shorten >=0.25em, shorten <=0.25em] (A) -- (B);

        % x軸方向のベクトル
        \draw[vector, dashed, very thick, Rhodamine] (0,0) -- (\vx, 0);
        % y軸方向のベクトル
        \draw[vector, dashed, very thick, Cerulean, shorten >=0.25em] (\vx, 0) -- (\vx, \vy);

        % 点A
        \draw (A) node[circle, fill, inner sep=1.5pt] {};
        \node at (A) [below left] {$A$};

        % 点B
        \draw (B) node[circle, fill, inner sep=1.5pt] {};
        \node at (B) [above right] {$B$};
      \end{tikzpicture}
    }
    \caption*{\bfseries 「縦」と「横」に分解}
  \end{minipage}
  \begin{minipage}{0.49\columnwidth}
    \centering
    \scalebox{1.2}{
      \begin{tikzpicture}
        \def\xmin{-1}
        \def\xmax{4}
        \def\ymin{-1}
        \def\ymax{4}
        \def\vx{1.5}
        \def\vy{2}

        \coordinate (A) at (0,0);
        \coordinate (B) at (\vx, \vy);

        % ベクトル
        \draw[vector, very thick, BurntOrange, shorten >=0.25em, shorten <=0.25em] (A) -- (B);

        % x軸方向のベクトル
        \draw[vector, dashed, very thick, Rhodamine] (0,0) -- (\vx, 0.75);
        % y軸方向のベクトル
        \draw[vector, dashed, very thick, Cerulean, shorten >=0.25em] (\vx, 0.75) -- (\vx, \vy);

        % 点A
        \draw (A) node[circle, fill, inner sep=1.5pt] {};
        \node at (A) [below left] {$A$};

        % 点B
        \draw (B) node[circle, fill, inner sep=1.5pt] {};
        \node at (B) [above right] {$B$};
      \end{tikzpicture}
    }
    \caption*{\bfseries 他の分解も考えられる}
  \end{minipage}
\end{figure}

\end{document}
