\documentclass[../../../topic_linear-algebra]{subfiles}

\begin{document}

\sectionline
\section{ベクトルの次元}
\marginnote{\refbookL p18}

$n$個の成分からなるベクトルは、$n$次元ベクトルと呼ばれる。

ここで、「次元」とは何だろうか?

\br

数ベクトルは、進む方向の数だけ「どれくらい進むか」を表す数値を並べたものとして導入した。

ここで、「進む方向」の情報は基底によって付け加えることができ、基底はいわば座標軸に対応する。

\br

2次元平面座標系は$x$軸と$y$軸という2つの座標軸で表されるように、次元とは座標軸の数、すなわち基底ベクトルの本数(基準となる方向がいくつあるか)に対応する。

\begin{emphabox}
  \begin{spacebox}
    \begin{center}
      \keyword{次元}とは、1つの基底を構成するベクトルの本数
    \end{center}
  \end{spacebox}
\end{emphabox}

方向の数だけ移動量を並べた数ベクトルは、次元の数だけ成分を持つことになる。
これが、$n$個の成分からなるベクトルが$n$次元ベクトルと呼ばれることに対する、ひとつの解釈である。

\end{document}
