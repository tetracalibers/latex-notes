\documentclass[../../../topic_linear-algebra]{subfiles}

\begin{document}

\sectionline
\section{二つのベクトルに垂直なベクトル}
\marginnote{
  \refweb{ベクトルの外積(クロス積)}{https://wiis.info/math/linear-algebra/vector/cross-product/}
}

3次元空間における2つのベクトル$\vb*{a}, \vb*{b} \in \mathbb{R}^3$に対して、どちらとも垂直になるベクトル$\vb*{v}$を求める問題を考える。

\br

もし$\vb*{a}, \vb*{b}$がどちらも零ベクトルであるなら、零ベクトルは任意のベクトルと垂直であるため、任意のベクトル$\vb*{v}$が$\vb*{a}, \vb*{b}$の双方と垂直になる。

\br

そこで、$\vb*{a}, \vb*{b}$の少なくとも一方が零ベクトルでない場合を考えることにする。

\br

どちらも$\vb*{v}$と直交するという条件は、内積を用いて次のように書ける。
\begin{align*}
  \vb*{a} \cdot \vb*{v} &= \vb*{o} \\
  \vb*{b} \cdot \vb*{v} &= \vb*{o}
\end{align*}

成分を用いて書くと、
\begin{align*}
  a_1 v_1 + a_2 v_2 + a_3 v_3 &= 0 \\
  b_1 v_1 + b_2 v_2 + b_3 v_3 &= 0
\end{align*}

ここで、$\vb*{a} \cdot \vb*{v} = \vb*{o}$の両辺に$b_1$を、$\vb*{b} \cdot \vb*{v} = \vb*{o}$の両辺に$a_1$をかける。
\begin{align*}
  b_1 (a_1 v_1 + a_2 v_2 + a_3 v_3) &= 0 \\
  a_1 (b_1 v_1 + b_2 v_2 + b_3 v_3) &= 0
\end{align*}
それぞれ展開して、
\begin{align*}
  a_1 b_1 v_1 + a_2 b_1 v_2 + a_3 b_1 v_3 &= 0 \\
  a_1 b_1 v_1 + a_1 b_2 v_2 + a_1 b_3 v_3 &= 0
\end{align*}
辺々を引いた上で整理すると、
\begin{equation*}
  (a_2 b_1 - a_1 b_2) v_2 + (a_3 b_1 - a_1 b_3) v_3 = 0
\end{equation*}

この式は、次のような内積が0となる式として解釈することができる。
\begin{equation*}
  \begin{pmatrix}
    a_2 b_1 - a_1 b_2 \\
    a_3 b_1 - a_1 b_3
  \end{pmatrix} \cdot \begin{pmatrix}
    v_2 \\
    v_3
  \end{pmatrix} = 0
\end{equation*}

ここで、2次元ベクトルについて、次が成り立つことを利用する。
\begin{equation*}
  \begin{pmatrix}
    x \\
    y
  \end{pmatrix} \cdot \begin{pmatrix}
    y \\
    -x
  \end{pmatrix} = xy - xy = 0
\end{equation*}
これより、
\begin{equation*}
  \begin{pmatrix}
    a_2 b_1 - a_1 b_2 \\
    a_3 b_1 - a_1 b_3
  \end{pmatrix} \cdot \begin{pmatrix}
    a_3 b_1 - a_1 b_3 \\
    -(a_2 b_1 - a_1 b_2)
  \end{pmatrix} = 0
\end{equation*}
となるので、
\begin{equation*}
  \begin{pmatrix}
    v_2 \\
    v_3
  \end{pmatrix} = \begin{pmatrix}
    a_3 b_1 - a_1 b_3 \\
    -(a_2 b_1 - a_1 b_2)
  \end{pmatrix} = \begin{pmatrix}
    a_3 b_1 - a_1 b_3 \\
    a_1 b_2 - a_2 b_1
  \end{pmatrix}
\end{equation*}
とおけばよい。

\br

これで$v_2, v_3$が求まったので、あとは$v_1$を求めればよい。

$\vb*{a} \cdot \vb*{v} = \vb*{o}$という式
\begin{equation*}
  a_1 v_1 + a_2 v_2 + a_3 v_3 = 0
\end{equation*}
において、$a_1 \neq 0$と仮定すると、
\begin{align*}
  v_1 &= -\frac{a_2 v_2 + a_3 v_3}{a_1} \\
  &= - \frac{a_2v_2}{a_1} - \frac{a_3v_3}{a_1} \\
  &= - \frac{a_2(a_3b_1 - a_1b_3)}{a_1} - \frac{a_3(a_1b_2 - a_2b_1)}{a_1} \\
  &= - \frac{a_2a_3b_1}{a_1} + \frac{a_2a_1b_3}{a_1} - \frac{a_3a_1b_2}{a_1} + \frac{a_3a_2b_1}{a_1} \\
  &= - \frac{a_2a_3b_1}{a_1} + a_2b_3 - a_3b_2 + \frac{a_2a_3b_1}{a_1} \\
  &= a_2b_3 - a_3b_2
\end{align*}

以上より、
\begin{equation*}
  \begin{pmatrix}
    v_1 \\
    v_2 \\
    v_3
  \end{pmatrix} = \begin{pmatrix}
    a_2b_3 - a_3b_2 \\
    a_3b_1 - a_1b_3 \\
    a_1b_2 - a_2b_1
  \end{pmatrix}
\end{equation*}

\begin{definition*}{$\mathbb{R}^3$における二つのベクトルの外積}
  3次元空間における2つのベクトル$\vb*{a}, \vb*{b} \in \mathbb{R}^3$に対して、次のように定義されるベクトルを、$\vb*{a}$と$\vb*{b}$の\keyword{外積}あるいは\keyword{クロス積}という。
  \begin{equation*}
    \vb*{a} \times \vb*{b} = \begin{pmatrix}
      a_2b_3 - a_3b_2 \\
      a_3b_1 - a_1b_3 \\
      a_1b_2 - a_2b_1
    \end{pmatrix}
  \end{equation*}
\end{definition*}

\end{document}
