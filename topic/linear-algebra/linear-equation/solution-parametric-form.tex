\documentclass[../../../topic_linear-algebra]{subfiles}

\begin{document}

\sectionline
\section{一般解のパラメータ表示}
\marginnote{\refbookA p33〜36}

右端の列に主成分がない場合は、一般には無数個の解が存在する

解の集合が直線を成していたり、もっと高い次元の図形になっていることがある

\br

解が1つに定まらない場合は、解の全体像を知ることが方程式を「解く」ことになる

\sectionline

係数行列$A$の$n$個の列が、$n$個の変数に対応していることを思い出そう

\begin{definition}{主変数と自由変数}
  行列$A$を行基本変形により行階段形にしたとき、主成分がある列に対応する変数を\keyword{主変数}と呼び、それ以外の変数を\keyword{自由変数}と呼ぶ
\end{definition}

\sectionline

解が存在する場合には、
\begin{equation*}
  \vb*{x} = \vb*{x}_0 + t_1\vb*{u}_1 + t_2\vb*{u}_2 + \cdots + t_{n-r}\vb*{u}_{n-r}
\end{equation*}
という形の一般解の表示(問題Dの答え)が得られる

ここで、$r$は行列$A$の階数である

\sectionline

自由変数、すなわちパラメータの個数を\keyword{解の自由度}と呼ぶ

\begin{align*}
  \text{解の自由度} & = \text{(変数の個数)} - \rank(A) \\
               & = n - r
\end{align*}

これは、解全体の集合が何次元の空間なのかを表している(問題Cの答え)

\end{document}
