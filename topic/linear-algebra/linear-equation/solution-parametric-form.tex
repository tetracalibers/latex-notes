\documentclass[../../../topic_linear-algebra]{subfiles}

\begin{document}

\sectionline
\section{一般解のパラメータ表示}
\marginnote{\refbookA p33〜36}

右端の列に主成分がない場合は、一般には無数個の解が存在する

解の集合が直線を成していたり、もっと高い次元の図形になっていることがある

\br

解が1つに定まらない場合は、解の全体像を知ることが方程式を「解く」ことになる

\sectionline

係数行列$A$の$n$個の列が、$n$個の変数に対応していることを思い出そう

\begin{definition}{主変数と自由変数}
  行列$A$を行基本変形により行階段形にしたとき、主成分がある列に対応する変数を\keyword{主変数}と呼び、それ以外の変数を\keyword{自由変数}と呼ぶ
\end{definition}

\br

たとえば、次のような既約行階段形に変形した拡大係数行列を考える

\begin{equation*}
  \tilde{A_\circ} =\begin{pNiceArray}{ccccc|c}[first-row,code-for-first-row=\color{Rhodamine},last-row,code-for-last-row=\color{Cerulean},xdots={horizontal-labels,line-style ={ <->,thick}},margin]
    1 & & 3 & 4 & &                           \\
    1 & 2 & 0 & 0 & -1 & -3 \\
    0 & 0 & 1 & 0 & 2 & 1 \\
    0 & 0 & 0 & 1 & 1 & 2 \\
    0 & 0 & 0 & 0 & 0 & 0\\
    \Hdotsfor{5}_{\displaystyle 5}
    \CodeAfter
    \tikz \draw[thick,Rhodamine] (1-1) circle (0.5em);
    \tikz \draw[thick,Rhodamine] (2-3) circle (0.5em);
    \tikz \draw[thick,Rhodamine] (3-4) circle (0.5em);
  \end{pNiceArray}
\end{equation*}

変数を使って方程式の形に直すと、
\begin{equation*}
  \left\{
  \begin{NiceArray}{rrrrrrrrrrc}[margin]
    x_1                     & +                    & 2x_2                    & \color{lightgray}{+} & \color{lightgray}{0x_3} & \color{lightgray}{+} & \color{lightgray}{0x_4} & - & x_5  & = & -3 \\
    \color{lightgray}{0x_1} & \color{lightgray}{+} & \color{lightgray}{0x_2} & +                    & x_3                     & \color{lightgray}{+} & \color{lightgray}{0x_4} & + & 2x_5 & = & 1  \\
    \color{lightgray}{0x_1} & \color{lightgray}{+} & \color{lightgray}{0x_2} & \color{lightgray}{+} & \color{lightgray}{0x_3} & \color{lightgray}{+} & x_4                     & + & x_5  & = & 2
    \CodeAfter
    \tikz \draw[thick,Rhodamine] (1-1) circle (0.75em);
    \tikz \draw[thick,Rhodamine] (2-5) circle (0.75em);
    \tikz \draw[thick,Rhodamine] (3-7) circle (0.75em);
  \end{NiceArray}
  \right.
\end{equation*}

\br

主成分がある列は$1,3,4$列なので、主変数は$x_1, x_3, x_4$である

それ以外の$x_2, x_5$は自由変数となる

\br

\begin{center}
  \systeme{
    x_1 + 2x_2 - x_5 = -3,
    x_3 + 2x_5 = 1,
    x_4 + x_5 = 2
  }
\end{center}

において、自由変数を含む項を左辺に移行すれば、

\begin{center}
  \systeme{
    x_1 = -3 - 2x_2 + x_5,
    x_3 = 1 - 2x_5,
    x_4 = 2 - x_5
  }
\end{center}
となる

\br

自由変数の値を自由に選んで、主変数の値をこの等式によって定めれば、方程式の解になる

\br

そこで、
\begin{equation*}
  x_2 = t_1, \quad x_5 =  t_2
\end{equation*}
とおけば、

\begin{center}
  \systeme{
    x_1 = -3 - 2t_1 + t_2,
    x_3 = 1 - 2t_2,
    x_4 = 2 - t_2
  }
\end{center}

すなわち、

\begin{center}
  \systeme{
    x_1 = -3 - 2t_1 + t_2,
    x_2 = t_1,
    x_3 = 1 - 2t_2,
    x_4 = 2 - t_2,
    x_5 = t_2
  }
\end{center}

と書ける

\br

これをベクトル形に直すことで、一般的な解のパラメータ表示を得られる

\begin{equation*}
  \vb*{x} = \begin{pmatrix}
    -3 \\
    0  \\
    1  \\
    2  \\
    0
  \end{pmatrix} + t_1 \begin{pmatrix}
    -2 \\
    1  \\
    0  \\
    0  \\
    0
  \end{pmatrix} + t_2 \begin{pmatrix}
    1  \\
    0  \\
    -2 \\
    -1 \\
    1
  \end{pmatrix}
\end{equation*}

\sectionline
\marginnote{\refbookC p300〜301}

一般化するために、$P\vb*{x} = \vb*{q}$を次のように表して考える
\begin{equation*}
  (P \mid \vb*{q}) = \begin{pmatrix}
    \vb*{p}_1 & q_1     \\
    \vdots    & \vdots  \\
    \vb*{p}_r & q_r     \\
    \vb*{0}   & q_{r+1} \\
    \vdots    & \vdots  \\
    \vb*{0}   & q_m
  \end{pmatrix}
\end{equation*}
ここで、$\vb*{p}_1 \neq \vb*{0}, \dots, \vb*{p}_r \neq \vb*{0}$であるとする

\br

このとき、解を持つための条件は、
\begin{equation*}
  q_{r+1} = q_{r+2} = \cdots = q_m = 0
\end{equation*}
であった

\br

さて、$P$において、主成分を含む列を$j_1, j_2, \dots, j_r \quad (r = \rank(P))$とする

\begin{equation*}
  (P \mid \vb*{q}) =\begin{pNiceArray}{ccccccc|c}[first-row,code-for-first-row=\color{Rhodamine},last-row,code-for-last-row=\color{Cerulean},xdots={horizontal-labels,line-style ={ <->,thick}},margin]
    %\CodeBefore
    %\rectanglecolor{carnationpink!40}{1-1}{4-5}
    %\Body
    j_1    &        & j_2    & \cdots & j_r    &        &        &                    \\
    1      & \star  & 0      & \cdots & 0      & \star     & \star& q_1 \\
    0      & 0      & 1      & \cdots & 0      & \star     & \star& q_2 \\
    \vdots & \vdots & \vdots & \ddots & \vdots & \vdots      & \vdots     & \vdots \\
    0      & 0      & 0      & \cdots & 1      & \star      & \star  & q_r \\
    0      & 0      & 0      & \cdots & 0      & 0      & 0   & 0 \\
    \vdots & \vdots & \vdots & \ddots & \vdots & \vdots& \vdots & \vdots                  \\
    0      & 0      & 0      & \cdots & 0      & 0      & 0& 0\\
    \Hdotsfor{7}_{$n$}
  \end{pNiceArray}
\end{equation*}

すると、主変数$x_{j_i} \, (i = 1, 2, \dots, r)$は、次のように表される

\begin{gather*}
  x_{j_i} + \sum_k \star x_k = q_i \quad (k > j_i \text{かつ} k \notin \{j_1, j_2, \dots, j_r\})  \\
  \therefore  x_{j_i} = q_i - \sum_k \star x_k
\end{gather*}

ここで、$x_k$は$j_i$よりも右にある$\star$に対応する変数である

既約行階段行列では、$j_i$列の主成分以外の要素はすべて0であるため、$\star$に対応する自由変数のみが残る(これが$k \notin \{j_1, j_2, \dots, j_r\}$とした意味である)

\br

つまり、$x_{j_1}, x_{j_2}, \dots, x_{j_r}$以外の自由変数$x_k$に勝手な数を与えるごとに、主変数$x_{j_1}, x_{j_2}, \dots, x_{j_r}$は定まる

\br

このような自由変数は$n-r$個あるので、$P\vb*{x} = \vb*{q}$の解は、$n-r$個のパラメータを用いて表せる

\sectionline

まとめると、解が存在する場合には、$r$を行列$A$の階数として
\begin{equation*}
  \vb*{x} = \vb*{q} + \sum_{i=1}^{n-r} t_i \vb*{u}_i
\end{equation*}
という形の一般解の表示(問題Dの答え)が得られる

\br
\marginnote{\refbookB p103}

ここで、パラメータ$t_i$をかけた列ベクトル$\vb*{u}_i$を連立方程式の\keyword{基本解}と呼ぶ

また、パラメータをかけていない列ベクトル$\vb*{q}$は、連立方程式の定数項から決まる解であり、これを\keyword{特殊解}と呼ぶ

\sectionline
\section{解の自由度}\label{sec:degrees-of-freedom}
\marginnote{\refbookB p113〜114}

連立一次方程式の一般解は、基本解の線形結合と特殊解の和で表された

そして、基本解の線形結合は、基本解の個数の分だけパラメータを用いて表された

\br

パラメータの個数は、自由変数の個数でもあり、基本解の個数でもある

\br

このとき、パラメータの個数は、解を表す自由度と考えられる

そこで、解を表すパラメータの個数を\keyword{解の自由度}と呼ぶ

\begin{align*}
  \text{解の自由度} & = \text{(変数の個数)} - \rank(A) \\
               & = n - r
\end{align*}

解の自由度は、解全体のなす集合の大きさ、すなわち何次元の空間なのかを表している(問題Cの答え)

\end{document}
