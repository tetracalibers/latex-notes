\documentclass[../../../topic_linear-algebra]{subfiles}

\begin{document}

\sectionline
\section{非自明解の存在と有限従属性定理}
\marginnote{\refbookA p40〜41}

斉次形方程式$A\vb*{x} = \vb*{0}$の非自明解の存在に対して、次の解釈もできる

\begin{theorem}{斉次形方程式の非自明解の存在と線形従属}
  $m \times n$型行列$A$の列ベクトルを$\vb*{a}_1, \vb*{a}_2, \dots, \vb*{a}_n$とするとき、
  \begin{align*}
                              & A\vb*{x} = \vb*{0}\text{に自明でない解がある}                \\
    \Longleftrightarrow \quad & \vb*{a}_1, \vb*{a}_2, \dots, \vb*{a}_n\text{が線形従属}
  \end{align*}
\end{theorem}

\begin{proof}
  $A\vb*{x} = \vb*{0}$は、ベクトルの等式
  \begin{equation*}
    x_1 \vb*{a}_1 + x_2 \vb*{a}_2 + \cdots + x_n \vb*{a}_n = \vb*{0}
  \end{equation*}
  と同じものである

  \begin{subpattern}{$\Longrightarrow$}
    もし自明でない解があるならば、$x_1, x_2, \dots, x_n$のうち少なくとも1つは0ではない

    $x_1 \vb*{a}_1 + x_2 \vb*{a}_2 + \cdots + x_n \vb*{a}_n = \vb*{0}$が成り立つもとで、0でない係数が存在するということは、$\vb*{a}_1, \vb*{a}_2, \dots, \vb*{a}_n$が線形従属であることを意味する $\qed$
  \end{subpattern}

  \begin{subpattern}{$\Longleftarrow$}
    対偶を示す

    $\vb*{a}_1, \vb*{a}_2, \dots, \vb*{a}_n$が線形独立であれば、
    \begin{equation*}
      x_1 \vb*{a}_1 + x_2 \vb*{a}_2 + \cdots + x_n \vb*{a}_n = \vb*{0}
    \end{equation*}
    において、すべての係数$x_1, x_2, \dots, x_n$は0でなければならない

    よって、0以外の解(非自明解)は存在しないことになる $\qed$
  \end{subpattern}
\end{proof}

\sectionline

斉次形方程式に自明でない解が存在することは、$\rank(A) \neq n$、すなわち解の自由度が0ではないことと同値であった

\br

一般に、斉次形の線型方程式$A\vb*{x} = \vb*{0}$の解の自由度は、$n$を変数の個数とするとき$n - \rank(A)$なので、次が成り立つ

\begin{theorem}{列ベクトルの線型独立性と階数}\label{thm:lin-indep-iff-rank-n}
  $\vb*{a}_1, \vb*{a}_2, \dots, \vb*{a}_n \in \mathbb{R}^m$に対して、$A=(\vb*{a}_1, \vb*{a}_2, \dots, \vb*{a}_n)$とおくと、
  \begin{equation*}
    \vb*{a}_1, \vb*{a}_2, \dots, \vb*{a}_n\text{が線型独立} \Longleftrightarrow \rank(A) = n
  \end{equation*}
\end{theorem}

このことから、次の重要な結論が導かれる

\begin{theorem}{有限従属性定理}\label{thm:finite-dependency}
  $\mathbb{R}^m$内の$m$個よりも多いベクトルからなる集合は線形従属である
\end{theorem}

\begin{proof}
  \todo{\refbookA p41 (系1.6.6)}
\end{proof}

この結論は、幾何的な直観からは自然だといえる

平面$\mathbb{R}^2$内の3つ以上のベクトルがあれば、自動的に線形従属になる

\br

この事実は、\keyword{次元}の概念を議論する際の基礎になる

\br

同じことを線型方程式の文脈に言い換えると、次のようになる

\begin{theorem}{有限従属性定理の線型方程式版}
  斉次線型方程式$A\vb*{x} = \vb*{0}$において、変数の個数が方程式の個数よりも多いときには、非自明な解が存在する
\end{theorem}

\br

また、次のようにも言い換えられる

\begin{theorem}{有限従属性定理の抽象版}\label{thm:abstract-finite-dependency}
  $\vb*{v}_1, \vb*{v}_2, \dots, \vb*{v}_k \in \mathbb{R}^n$とする

  $\langle \vb*{v}_1, \vb*{v}_2, \dots, \vb*{v}_k \rangle$に含まれる$k$個よりも多い個数のベクトルの集合は線形従属である
\end{theorem}

\begin{proof}
  \todo{\refbookA p41 (問 1.14)}
\end{proof}

\sectionline
\section{行列の階数と線型独立性}
\marginnote{\refbookA p42〜44}

次の事実は、行変形のもっとも重要な性質である

\begin{theorem}{行基本変形による線型独立性の不変性}\label{thm:row-operation-preserves-dependence}
  行変形はベクトルの線形関係を保つ

  すなわち、行列$A = (\vb*{a}_1, \dots, \vb*{a}_n)$に行の変形を施して$B = (\vb*{b}_1, \dots, \vb*{b}_n)$が得られたとするとき、
  \begin{equation*}
    \sum_{i=1}^n c_i \vb*{a}_i = \vb*{0} \Longleftrightarrow \sum_{i=1}^n c_i \vb*{b}_i = \vb*{0}
  \end{equation*}

  特に、
  \begin{equation*}
    \{ \vb*{a}_1, \dots, \vb*{a}_n \} \text{が線型独立} \Longleftrightarrow \{ \vb*{b}_1, \dots, \vb*{b}_n \} \text{が線型独立}
  \end{equation*}
\end{theorem}

\begin{proof}
  \todo{\refbookA p42 (命題 1.6.8)}
\end{proof}

\sectionline

\begin{definition}{主列ベクトル}\label{def:pivot-columns}
  行列$A = (\vb*{a}_1, \vb*{a}_2, \dots, \vb*{a}_n)$を行階段形にしたときに、主成分のある列番号を$i_1, i_2, \dots, i_r$とする

  ここで、$r$は$A$の階数である

  このとき、$\vb*{a}_{i_1}, \vb*{a}_{i_2}, \dots, \vb*{a}_{i_r}$を\keyword{主列ベクトル}という
\end{definition}

\begin{theorem}{主列ベクトルと線型独立性}
  行列の主列ベクトルの集合は線型独立である

  また、主列ベクトル以外の列ベクトルは、主列ベクトルの線形結合である
\end{theorem}

\begin{proof}
  \todo{\refbookA p43 (命題 1.6.11)}
\end{proof}

\br

掃き出し法は、行列の列ベクトルの中から、$\rank(A)$個の線型独立な列ベクトルを選び出す方法を与えていることになる

\begin{theorem}{列ベクトルの線形従属性と階数}
  行列$A$の列ベクトルから$\rank(A)$個よりも多いベクトルを選ぶと、線形従属になる
\end{theorem}

\begin{proof}
  \todo{\refbookA p43 (命題 1.6.12)}
\end{proof}

\br

以上によって、行列の階数に関する次の理解が得られたことになる

\begin{theorem}{階数と線型独立な列ベクトルの最大個数}\label{thm:rank-equals-max-indep-cols}
  行列$A$の階数$\rank(A)$は、$A$の列ベクトルに含まれる線型独立なベクトルの最大個数と一致する
\end{theorem}

\begin{proof}
  \todo{\refbookA p43 (定理 1.6.13)}
\end{proof}

「行変形を繰り返して行階段形にしたときの0でない段の数」として導入した階数という量の、より本質的な意味がわかったことになる

\br

特に、
\begin{shaded}
  行変形によって定めた階数が行変形の仕方によらない
\end{shaded}
という事実がこの定理からしたがう

\sectionline

\begin{theorem}{2つの行列の階数の和}
  $A,\,B$を同じ型の行列とするとき、
  \begin{equation*}
    \rank(A+B) \leq \rank(A) + \rank(B)
  \end{equation*}
\end{theorem}

\begin{proof}
  \todo{\refbookA p44 問 1.15}
\end{proof}

\end{document}
