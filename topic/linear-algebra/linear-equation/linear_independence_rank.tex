\documentclass[../../../topic_linear-equation]{subfiles}

\begin{document}

\sectionline
\section{線型独立性}
\marginnote{\refbookA p38〜40}

線型独立性の定義式を移項することで、次の事実が得られる

\begin{theorem}{線型結合の一意性}
  線型独立性は、線形結合の一意性
  \begin{gather*}
    c_1 \vb*{a}_1 + \cdots + c_k \vb*{a}_k = c'_1 \vb*{a}_1 + \cdots + c'_k \vb*{a}_k \\ \Longrightarrow c_1 = c'_1, \dots, c_k = c'_k
  \end{gather*}
  と同値である
\end{theorem}

つまり、
\begin{shaded}
  線型独立性は、両辺の係数比較ができるという性質
\end{shaded}
であるとも理解できる

\sectionline

\begin{definition}{線形関係式}
  ベクトル$\vb*{a}_1, \vb*{a}_2, \dots, \vb*{a}_k$に対する等式
  \begin{equation*}
    c_1 \vb*{a}_1 + c_2 \vb*{a}_2 + \cdots + c_k \vb*{a}_k = \vb*{0}
  \end{equation*}
  を、$\vb*{a}_1, \vb*{a}_2, \dots, \vb*{a}_k$の\keyword{線形関係式}という
\end{definition}

特に、$c_1 = c_2 = \cdots = c_k = 0$として得られる線形関係式を\keyword{自明な線形関係式}という

これ以外の場合、つまり$c_i \neq 0$となるような$i$が少なくとも1つあるならば、これは\keyword{非自明な線形関係式}である

\begin{theorem}{非自明な線形関係式の存在と線形従属}
  ベクトルの集まりは、それらに対する非自明な線形関係式が存在するとき、そのときに限り線形従属である
\end{theorem}

\begin{proof}
  ベクトルの集まりが線型独立であることは、それらに対する線形関係式はすべて自明であるというのが定義である

  それを否定すると、「自明でない線形関係式が存在する」となる $\qed$
\end{proof}

\sectionline
\section{非自明解の存在と有限従属性定理}
\marginnote{\refbookA p40〜41}

斉次形方程式$A\vb*{x} = \vb*{0}$の非自明解の存在に対して、次の解釈もできる

\begin{theorem}{斉次形方程式の非自明解の存在と線形従属}
  $m \times n$型行列$A$の列ベクトルを$\vb*{a}_1, \vb*{a}_2, \dots, \vb*{a}_n$とするとき、
  \begin{align*}
                              & A\vb*{x} = \vb*{0}\text{に自明でない解がある}                \\
    \Longleftrightarrow \quad & \vb*{a}_1, \vb*{a}_2, \dots, \vb*{a}_n\text{が線形従属}
  \end{align*}
\end{theorem}

\begin{proof}
  \todo{\refbookA p40 (命題1.6.4)}
\end{proof}

\sectionline

斉次形方程式に自明でない解が存在することは、$\rank(A) \neq n$、すなわち解の自由度が0ではないことと同値であった

一般に、斉次形の線型方程式$A\vb*{x} = \vb*{0}$の解の自由度は、$n$を変数の個数とするとき$n - \rank(A)$なので、次が成り立つ

\begin{theorem}{列ベクトルの線型独立性と階数}
  $\vb*{a}_1, \vb*{a}_2, \dots, \vb*{a}_n \in \mathbb{R}^m$に対して、$A=(\vb*{a}_1, \vb*{a}_2, \dots, \vb*{a}_n)$とおくと、
  \begin{equation*}
    \vb*{a}_1, \vb*{a}_2, \dots, \vb*{a}_n\text{が線型独立} \Longleftrightarrow \rank(A) = n
  \end{equation*}
\end{theorem}

このことから、次の重要な結論が導かれる

\begin{theorem}{有限従属性定理}
  $\mathbb{R}^m$内の$m$個よりも多いベクトルからなる集合は線形従属である
\end{theorem}

\begin{proof}
  \todo{\refbookA p41 (系1.6.6)}
\end{proof}

この結論は、幾何的な直観からは自然だといえる

平面$\mathbb{R}^2$内の3つ以上のベクトルがあれば、自動的に線形従属になる

\br

この事実は、\keyword{次元}の概念を議論する際の基礎になる

\br

同じことを線型方程式の文脈に言い換えると、次のようになる

\begin{theorem}{有限従属性定理の線型方程式版}
  斉次線型方程式$A\vb*{x} = \vb*{0}$において、変数の個数が方程式の個数よりも多いときには、非自明な解が存在する
\end{theorem}

\br

また、次のようにも言い換えられる

\begin{theorem}{有限従属性定理の抽象版}
  $\vb*{v}_1, \vb*{v}_2, \dots, \vb*{v}_k \in \mathbb{R}^n$とする

  $\langle \vb*{v}_1, \vb*{v}_2, \dots, \vb*{v}_k \rangle$に含まれる$k$個よりも多い個数のベクトルの集合は線形従属である
\end{theorem}

\begin{proof}
  \todo{\refbookA p41 (問 1.14)}
\end{proof}

\sectionline
\section{行列の階数と線型独立性}
\marginnote{\refbookA p42〜}

次の事実は、行変形のもっとも重要な性質である

\begin{theorem}{行変形はベクトルの線形関係を保つ}
  行列$A = (\vb*{a}_1, \dots, \vb*{a}_n)$に行の変形を施して$B = (\vb*{b}_1, \dots, \vb*{b}_n)$が得られたとする

  このとき、
  \begin{equation*}
    \sum_{i=1}^n c_i \vb*{a}_i = \vb*{0} \Longleftrightarrow \sum_{i=1}^n c_i \vb*{b}_i = \vb*{0}
  \end{equation*}

  特に、
  \begin{equation*}
    \{ \vb*{a}_1, \dots, \vb*{a}_n \} \text{が線型独立} \Longleftrightarrow \{ \vb*{b}_1, \dots, \vb*{b}_n \} \text{が線型独立}
  \end{equation*}
\end{theorem}

\begin{proof}
  \todo{\refbookA p42 (命題 1.6.8)}
\end{proof}

\end{document}
