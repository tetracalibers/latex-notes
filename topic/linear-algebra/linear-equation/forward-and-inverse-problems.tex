\documentclass[../../../topic_linear-algebra]{subfiles}

\begin{document}

\sectionline
\section{順問題と逆問題}
\marginnote{\refbookL p93〜94}

世の中には、「ベクトル$\vb*{x}$を入力するとベクトル$\vb*{y} = A\vb*{x}$が出力される」という形で表せる対象がたくさんある

\br

この$\vb*{y} = A\vb*{x}$という式は、
\begin{shaded}
  原因$\vb*{x}$を知って結果$\vb*{y}$を予測する
\end{shaded}
という場面(\keyword{順問題})でそのまま使うことができる

\br

一方、次のような
\begin{shaded}
  結果$\vb*{y}$を知って原因$\vb*{x}$を推定する
\end{shaded}
という問題(\keyword{逆問題})を考えなければいけない場合もある

\br

$\vb*{y} = A\vb*{x}$という式(結果)から$\vb*{x}$(原因)を求めるという問題は、\keyword{連立一次方程式}を解くということに他ならない

\end{document}
