\documentclass[../../../topic_linear-algebra]{subfiles}

\begin{document}

\sectionline
\section{行基本変形}
\marginnote{\refbookA p25}

連立一次方程式を行列によってとり扱うとき、1つ1つの方程式は行列の行によって表されている

そこで、連立方程式の式変形に対応する操作として、行列の行に関する次のような操作(変形)を考える

\begin{definition}{行基本変形}
  行列への次の3種類の操作を\keyword{行基本変形}という
  \begin{enumerate}[label=\romanlabel]
    \item ある行の定数倍を他の行に加える
    \item ある行に0でない数をかける
    \item 2つの行を交換する
  \end{enumerate}
\end{definition}

\end{document}
