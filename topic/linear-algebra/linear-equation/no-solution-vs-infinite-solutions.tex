\documentclass[../../../topic_linear-algebra]{subfiles}

\begin{document}

\sectionline
\section{解が無限個ある場合}
\marginnote{\refbookB p101}

掃き出し法によって、係数行列を単位行列に変形できない場合もある

\begin{itemize}
  \item 解が一意に定まらない場合(解が無限個ある場合)
  \item 解が存在しない場合
\end{itemize}

まずは、解が一意に定まらない場合を見てみよう

\br

\begin{tcbraster}[raster columns=2, raster equal height=rows,size=small, empty, raster column skip=1cm, raster row skip=0.5cm]
  \begin{tcolorbox}
    \begin{equation*}
      \begin{pNiceMatrix}[margin,vlines = 4,last-col, code-for-last-col = \color{BurntOrange}]
        1 & -1 & -3 & 1 & R_1 \\
        2 & 1  & 0  & 2 & R_2 \\
        1 & 0  & -1 & 1 & R_3
      \end{pNiceMatrix}
    \end{equation*}
  \end{tcolorbox}
  \begin{tcolorbox}
    \systeme{
      x - y - 3z = 1,
      2x + y = 2,
      x - z = 1
    }
  \end{tcolorbox}
\end{tcbraster}

$(1,1)$成分を要にして第1列を掃き出す:

\begin{tcbraster}[raster columns=2, raster equal height=rows,size=small, empty, raster column skip=1cm, raster row skip=0.5cm]
  \begin{tcolorbox}
    \begin{equation*}
      \begin{pNiceMatrix}[margin,vlines = 4,last-col, code-for-last-col = \color{BurntOrange}]
        1 & -1 & -3 & 1 &            \\
        0 & 3  & 6  & 0 & R_2 - 2R_1 \\
        0 & 1  & 2  & 0 & R_3 - R_1
        \CodeAfter
        \tikz \draw[thick,magenta] (1-1) circle (0.5em);
      \end{pNiceMatrix}
    \end{equation*}
  \end{tcolorbox}
  \begin{tcolorbox}
    \systeme{
      x - y - 3z = 1,
      3y + 6z = 0,
      y + 2z = 0
    }
  \end{tcolorbox}
\end{tcbraster}

$(2,2)$成分を1にする:

\begin{tcbraster}[raster columns=2, raster equal height=rows,size=small, empty, raster column skip=1cm, raster row skip=0.5cm]
  \begin{tcolorbox}
    \begin{equation*}
      \begin{pNiceMatrix}[margin,vlines = 4,last-col, code-for-last-col = \color{BurntOrange}]
        1 & -1 & -3 & 1 &                 \\
        0 & 1  & 2  & 0 & \frac{1}{3} R_2 \\
        0 & 1  & 2  & 0 &
        \CodeAfter
        \tikz \draw[thick,magenta] (2-2) circle (0.5em);
      \end{pNiceMatrix}
    \end{equation*}
  \end{tcolorbox}
  \begin{tcolorbox}
    \systeme{
      x - y - 3z = 1,
      y + 2z = 0,
      y + 2z = 0
    }
  \end{tcolorbox}
\end{tcbraster}

$(2,2)$成分を要にして第2列を掃き出す:

\begin{tcbraster}[raster columns=2, raster equal height=rows,size=small, empty, raster column skip=1cm, raster row skip=0.5cm]
  \begin{tcolorbox}
    \begin{equation*}
      \begin{pNiceMatrix}[margin,vlines = 4,last-col, code-for-last-col = \color{BurntOrange}]
        1 & -1 & -3 & 1 &           \\
        0 & 1  & 2  & 0 &           \\
        0 & 0  & 0  & 0 & R_3 - R_2
        \CodeAfter
        \tikz \draw[thick,magenta] (2-2) circle (0.5em);
      \end{pNiceMatrix}
    \end{equation*}
  \end{tcolorbox}
  \begin{tcolorbox}
    \systeme{
      x - y - 3z = 1,
      y + 2z = 0,
      0 = 0
    }
  \end{tcolorbox}
\end{tcbraster}

1を対角成分として持つ列の対角成分以外を0にする:

\begin{tcbraster}[raster columns=2, raster equal height=rows,size=small, empty, raster column skip=1cm, raster row skip=0.5cm]
  \begin{tcolorbox}
    \begin{equation*}
      \begin{pNiceMatrix}[margin,vlines = 4,last-col, code-for-last-col = \color{BurntOrange}]
        \CodeBefore
        \cellcolor{carnationpink!30}{1-1}
        \cellcolor{carnationpink!30}{2-2}
        \cellcolor{carnationpink!30}{1-2}
        \cellcolor{carnationpink!30}{1-3}
        \cellcolor{carnationpink!30}{2-3}
        \Body
        1 & 0 & -1 & 1 & R_1 - R_2 \\
        0 & 1 & 2  & 0 &           \\
        0 & 0 & 0  & 0 &
        \CodeAfter
        \tikz \draw[thick,magenta] (1-1) circle (0.5em);
        \tikz \draw[thick,magenta] (2-2) circle (0.5em);
      \end{pNiceMatrix}
    \end{equation*}
  \end{tcolorbox}
  \begin{tcolorbox}
    \systeme{
      x -z = 1,
      y + 2z = 0,
      0 = 0
    }
  \end{tcolorbox}
\end{tcbraster}

\br

この連立方程式は、実質的に2本の方程式しか持たないことがわかる

\begin{center}
  \systeme{
    x - z = 1,
    y + 2z = 0
  }
\end{center}

$x,y$について解くと、

\begin{center}
  \systeme{
    x = z + 1,
    y = -2z
  }
\end{center}

となるので、$z$に任意の数$z=\alpha$を与えて解が得られる

\begin{equation*}
  \begin{pmatrix}
    x \\
    y \\
    z
  \end{pmatrix} = \begin{pmatrix}
    \alpha + 1 \\
    -2\alpha   \\
    \alpha
  \end{pmatrix} = \alpha \begin{pmatrix}
    1  \\
    -2 \\
    1
  \end{pmatrix} + \begin{pmatrix}
    1 \\
    0 \\
    0
  \end{pmatrix}
\end{equation*}

\sectionline
\section{解が存在しない場合}
\marginnote{\refbookB p104}

次のような連立一次方程式を考える

\begin{tcbraster}[raster columns=2, raster equal height=rows,size=small, empty, raster column skip=1cm, raster row skip=0.5cm]
  \begin{tcolorbox}
    \begin{equation*}
      \begin{pNiceMatrix}[margin,vlines = 4,last-col, code-for-last-col = \color{BurntOrange}]
        1 & -1 & -3 & 1 & R_1 \\
        2 & 1  & 0  & 1 & R_2 \\
        1 & 0  & -1 & 1 & R_3
      \end{pNiceMatrix}
    \end{equation*}
  \end{tcolorbox}
  \begin{tcolorbox}
    \systeme{
      x - y - 3z = 1,
      2x + y = 1,
      x - z = 1
    }
  \end{tcolorbox}
\end{tcbraster}

$(1,1)$成分を要にして第1列を掃き出す:

\begin{tcbraster}[raster columns=2, raster equal height=rows,size=small, empty, raster column skip=1cm, raster row skip=0.5cm]
  \begin{tcolorbox}
    \begin{equation*}
      \begin{pNiceMatrix}[margin,vlines = 4,last-col, code-for-last-col = \color{BurntOrange}]
        1 & -1 & -3 & 1  &            \\
        0 & 3  & 6  & -1 & R_2 - 2R_1 \\
        0 & 1  & 2  & 0  & R_3 - R_1
        \CodeAfter
        \tikz \draw[thick,magenta] (1-1) circle (0.5em);
      \end{pNiceMatrix}
    \end{equation*}
  \end{tcolorbox}
  \begin{tcolorbox}
    \systeme{
      x - y - 3z = 1,
      3y + 6z = -1,
      y + 2z = 0
    }
  \end{tcolorbox}
\end{tcbraster}

$(2,2)$成分を1にするため、第2行と第3行を入れ替える:

\begin{tcbraster}[raster columns=2, raster equal height=rows,size=small, empty, raster column skip=1cm, raster row skip=0.5cm]
  \begin{tcolorbox}
    \begin{equation*}
      \begin{pNiceMatrix}[margin,vlines = 4,last-col, code-for-last-col = \color{BurntOrange}]
        1 & -1 & -3 & 1  &     \\
        0 & 1  & 2  & 0  & R_3 \\
        0 & 3  & 6  & -1 & R_2
        \CodeAfter
        \tikz \draw[thick,magenta] (2-2) circle (0.5em);
      \end{pNiceMatrix}
    \end{equation*}
  \end{tcolorbox}
  \begin{tcolorbox}
    \systeme{
      x - y - 3z = 1,
      y + 2z = 0,
      3y + 6z = -1
    }
  \end{tcolorbox}
\end{tcbraster}

$(2,2)$成分を要にして第2列を掃き出す:

\begin{tcbraster}[raster columns=2, raster equal height=rows,size=small, empty, raster column skip=1cm, raster row skip=0.5cm]
  \begin{tcolorbox}
    \begin{equation*}
      \begin{pNiceMatrix}[margin,vlines = 4,last-col, code-for-last-col = \color{BurntOrange}]
        \CodeBefore
        \cellcolor{carnationpink!30}{1-1}
        \cellcolor{carnationpink!30}{2-2}
        \cellcolor{carnationpink!30}{1-2}
        \cellcolor{carnationpink!30}{1-3}
        \cellcolor{carnationpink!30}{2-3}
        \Body
        1 & -1 & -3 & 1  &            \\
        0 & 1  & 2  & 0  &            \\
        0 & 0  & 0  & -1 & R_3 - 3R_2
        \CodeAfter
        \tikz \draw[thick,magenta] (1-1) circle (0.5em);
        \tikz \draw[thick,magenta] (2-2) circle (0.5em);
      \end{pNiceMatrix}
    \end{equation*}
  \end{tcolorbox}
  \begin{tcolorbox}
    \systeme{
      x - y - 3z = 1,
      y + 2z = 0,
      0 = -1
    }
  \end{tcolorbox}
\end{tcbraster}

\br

$0 = -1$という式が現れたので、この連立方程式には解が存在しない

\end{document}
