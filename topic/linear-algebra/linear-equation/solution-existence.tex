\documentclass[../../../topic_linear-algebra]{subfiles}

\begin{document}

\sectionline
\section{解の存在条件}
\marginnote{\refbookB p110〜111}

まず、一般の$\vb*{b}$の場合の解の存在(問題A)について考える

\br

拡大係数行列$\tilde{A}$は$A$の右端に1列追加して得られるので、掃き出しの過程を考えると、$\rank(\tilde{A})$は$\rank(A)$と等しいか、1だけ増えるかのどちらかであることがわかる

\br

また、方程式の拡大係数行列の行に関する基本変形は、元の連立方程式と同値な式への変形であるため、
\begin{shaded}
  基本変形によって得られる方程式の解は、元の方程式の解と同じ
\end{shaded}
となる

\br

そこで、$\tilde{A} = (A \mid \vb*{b})$の既約行階段形を$(P \mid \vb*{q})$とし、$A\vb*{x} = \vb*{b}$の代わりに
\begin{equation*}
  P\vb*{x} = \vb*{q}
\end{equation*}
を解くことを考える

\br

まず、
\begin{equation*}
  P = \begin{pmatrix}
    P_1 \\
    O
  \end{pmatrix}, \quad
  \vb*{q} = \begin{pmatrix}
    \vb*{q}_1 \\
    \vb*{q}_2
  \end{pmatrix}
\end{equation*}
とおく

ここで、$P_1$は$r \times n$行列($r = \rank(P)$)とし、$\vb*{q}_1$は$r$次元列ベクトル、$\vb*{q}_2$は$m - r$次元列ベクトルとする

\br

すると、$P\vb*{x} = \vb*{q}$は
\begin{equation*}
  \begin{pmatrix}
    P_1 \\
    O
  \end{pmatrix} \vb*{x}
  = \begin{pmatrix}
    P_1 \vb*{x} \\
    \vb*{o}
  \end{pmatrix}
  = \begin{pmatrix}
    \vb*{q}_1 \\
    \vb*{q}_2
  \end{pmatrix}
\end{equation*}
と表せる

\br

このとき、この方程式が解を持つには、$\vb*{q}_2 = \vb*{o}$でなければならない

たとえば、
\begin{equation*}\vb*{q}_2 = \begin{pmatrix}
    0 \\
    -1
  \end{pmatrix}
\end{equation*}
だとしたら、
\begin{equation*}
  \begin{pmatrix}
    P_1 \vb*{x} \\
    0           \\
    0
  \end{pmatrix} = \begin{pmatrix}
    \vb*{q}_1 \\
    0         \\
    -1
  \end{pmatrix}
\end{equation*}
となり、$0 = -1$という矛盾が生じる時点で、この方程式は不能になる

\br

このような$\vb*{q}_2 \neq \vb*{0}$の場合、拡大係数行列の階数は、係数行列の階数$+1$となっている

\begin{align*}
  P                & = \begin{pNiceArray}{>{\strut}cccccc}% <-- % mandatory
                         [margin, extra-margin=2pt,no-cell-nodes]
                         \rowcolor{carnationpink!40} 1 & 0 & 0 & * & * & 0 \\
                         0 & \rowcolor{carnationpink!40} 1 & 0 & * & * & 0 \\
                         0 & 0 & \rowcolor{carnationpink!40} 1 & * & * & 0 \\
                         0 & 0 & 0 & 0 & 0 & \rowcolor{carnationpink!40} 1 \\
                         0 & 0 & 0 & 0 & 0 & 0
                       \end{pNiceArray}   \\
  (P \mid \vb*{q}) & = \begin{pNiceArray}{>{\strut}cccccc|c}% <-- % mandatory
                         [margin, extra-margin=2pt,no-cell-nodes]
                         \rowcolor{carnationpink!40} 1 & 0 & 0 & * & * & 0 & 0 \\
                         0 & \rowcolor{carnationpink!40} 1 & 0 & * & * & 0 & 0 \\
                         0 & 0 & \rowcolor{carnationpink!40} 1 & * & * & 0 & 0 \\
                         0 & 0 & 0 & 0 & 0 & \rowcolor{carnationpink!40} 1 & 0 \\
                         0 & 0 & 0 & 0 & 0 & 0 & \rowcolor{carnationpink!40} 1
                       \end{pNiceArray}
\end{align*}

\br

一方、$\vb*{q}_2 = \vb*{o}$であれば、方程式は
\begin{equation*}
  P_1 \vb*{x} = \vb*{q}_1
\end{equation*}
となる

ここで、$P_1$は$r = \rank(P)$個の行をもち、行数と階数が一致しているということは、すべての行に主成分が現れていることを意味する

\br

主成分は最も左側にある0でない成分なので、係数拡大行列にするために右に1列追加したとしても、主成分の数は増えることがない

すなわち、$\vb*{q}_2 = \vb*{o}$の場合は係数行列と拡大係数行列の階数が一致する

\sectionline

以上の考察から、連立方程式$A\vb*{x} = \vb*{b}$の解が存在する条件は、
\begin{shaded}
  係数行列と係数拡大行列の階数が等しい
\end{shaded}
ことだとわかる

\br

そして、その階数$r$は、係数行列の行数とも一致していたため、次の2つの定理が得られる

\begin{theorem}{解の存在条件}
  $A$を$m \times n$型行列、$\vb*{b} \in \mathbb{R}^m$とする

  $\tilde{A} = (A \mid \vb*{b})$とおくとき、
  \begin{equation*}
    \rank(\tilde{A}) = \rank(A) \Longleftrightarrow A\vb*{x} = \vb*{b} \text{に解が存在する}
  \end{equation*}
\end{theorem}

\begin{proof}
  \todo{\refbookA p31 (定理1.5.1)}
\end{proof}

\begin{theorem}{解の存在条件の系}
  $A$を$m \times n$型行列とするとき、
  \begin{equation*}
    ^{\forall}\vb*{b} \in \mathbb{R}^m, A\vb*{x} = \vb*{b} \text{の解が存在する} \Longleftrightarrow \rank(A) = m
  \end{equation*}
\end{theorem}

\begin{proof}
  \todo{\refbookA p32 (定理1.5.2, 1.5.3)}
\end{proof}

\end{document}
