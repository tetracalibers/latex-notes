\documentclass[../../../topic_linear-algebra]{subfiles}

\begin{document}

\sectionline
\section{掃き出し法の段階ごとに得られる形}\label{sec:forms-in-gaussian-elimination}

ここまで見てきた、掃き出し法による連立方程式の解法をまとめると、大まかには次のような手順を踏むことになる
\begin{enumerate}
  \item 左の列から順に、対角成分を1にする
  \item 対角成分が1となっている列の対角成分以外を0にする
\end{enumerate}

手順1で得られる形を\keyword{行階段行列}と呼び、手順2で得られる形を\keyword{既約行階段行列}と呼ぶ

\br

ただし、$0 = -1$が現れたときのように、手順1(行階段行列への変形)だけで解が存在するかはわかってしまう

解の存在以外にも、行階段行列に変形した時点で読み取れる情報はさまざまある

\sectionline
\section{行階段行列}
\marginnote{\refbookA p26〜28 \\ \refbookB p81〜84}

掃き出し法では、あるステップで下の成分がすべて0になって、
\begin{equation*}
  \begin{pNiceArray}{>{\strut}cccccccc}% <-- % mandatory
    [margin, extra-margin=2pt,no-cell-nodes]
    \rowcolor{carnationpink!40} \spadesuit & * & * & * & * & * & * & *\\
    0 & \rowcolor{carnationpink!40} \spadesuit & * & * & * & * & * & * \\
    0 & 0 & \rowcolor{carnationpink!40} \spadesuit & * & * & * & * & * \\
    0 & 0 & 0 & \rowcolor{carnationpink!40} \spadesuit & * & * & * & * \\
    0 & 0 & 0 & 0 & 0 & 0 & 0 & 0
  \end{pNiceArray}
\end{equation*}
のような形になるのが典型例である。

ここで、0でない成分を$\spadesuit$で、任意の値をもつ成分を$*$で表している。

\subsection{零行}

一般には、成分が0ばかりの行が下にくる。
そのような行を\keyword{零行}という。

零行が現れない場合もあるし、複数現れる場合もある。

\subsection{主成分}\label{sec:def-pivot}

零行でない行に対して、一番左の0でない成分$\spadesuit$を\keyword{主成分}あるいは行に関する\keyword{要}と呼ぶ。

\subsection{行階段行列の一般形}

先ほど示した形では、行の主成分$\spadesuit$は左上から斜め右下$45^\circ$方向にまっすぐ並んでいるが、一般にはそうできるとは限らない。

\br

しかし、次のような形には必ずできる。
\begin{equation*}
  \begin{pNiceArray}{>{\strut}cccccccc}% <-- % mandatory
    [margin, extra-margin=2pt,no-cell-nodes]
    0 & \rowcolor{carnationpink!40} \spadesuit & * & * & * & * & * & * \\
    0 & 0 & 0 & \rowcolor{carnationpink!40} \spadesuit & * & * & * & * \\
    0 & 0 & 0 & 0 & \rowcolor{carnationpink!40} \spadesuit & * & * & * \\
    0 & 0 & 0 & 0 & 0 & 0 & 0 & \rowcolor{carnationpink!40} \spadesuit \\
    0 & 0 & 0 & 0 & 0 & 0 & 0 & 0
  \end{pNiceArray}
\end{equation*}

\begin{definition*}{行階段行列}
  次の条件を満たす行列を\keyword{行階段行列}という。
  \begin{itemize}
    \item 零行でない行の主成分が、下の行ほど1つ以上右にある
    \item 零行がある場合は、まとめてすべて下にある
  \end{itemize}
\end{definition*}

どんな行列も、行基本変形の繰り返しで行階段行列にできる。

\sectionline
\section{既約行階段行列}\label{sec:reduced-row-echelon-form}
\marginnote{\refbookA p29〜30 \\ \refbookB p82}

必要に応じて、行階段行列をさらに変形して次のような形にする

\begin{equation*}
  \begin{pNiceArray}{>{\strut}cccccccc}% <-- % mandatory
    [margin, extra-margin=2pt,no-cell-nodes]
    0 & \rowcolor{carnationpink!40} 1 & * & 0 & 0 & * & * & 0 \\
    0 & 0 & 0 & \rowcolor{carnationpink!40} 1 & 0 & * & * & 0 \\
    0 & 0 & 0 & 0 & \rowcolor{carnationpink!40} 1 & * & * & 0 \\
    0 & 0 & 0 & 0 & 0 & 0 & 0 & \rowcolor{carnationpink!40} 1 \\
    0 & 0 & 0 & 0 & 0 & 0 & 0 & 0
  \end{pNiceArray}
\end{equation*}

行の主成分はすべて1で、主成分のある列の主成分以外の成分はすべて0である

この形を\keyword{簡約化された行階段行列}あるいは\keyword{既約行階段行列}と呼ぶ

\br

与えられた行列$A$に対して、行基本変形の繰り返しで得られる行階段行列は一意的ではないが、既約行階段行列は一意的であることを後に議論する

そこで、既約行階段行列を$A_\circ$と書くことにする

\sectionline

変形の過程を
\begin{equation*}
  \text{行列}A \rightarrow \text{行階段行列} \rightarrow \text{簡約化された行階段行列}A_\circ
\end{equation*}
と2段階にわけるのは、計算の効率以上の意味がある

行階段行列にするところまでで解決する問題(解の存在と一意性など)もあるからである

\end{document}
