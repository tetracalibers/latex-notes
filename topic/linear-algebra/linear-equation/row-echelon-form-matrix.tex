\documentclass[../../../topic_linear-equation]{subfiles}

\begin{document}

\sectionline
\section{連立一次方程式の行列表記}
\marginnote{\refbookA p22〜25}

未知数$x_1, x_2, \dots, x_n$に関する連立方程式として
\begin{equation*}
  \left\{
  \begin{alignedat}{12}
    a_{11} & x_1    & {}+{} & a_{12} & x_2 & {}+{} & \cdots & {}+{} & a_{1n} & x_n & {}={} & b_1    \\
    a_{21} & x_1    & {}+{} & a_{22} & x_2 & {}+{} & \cdots & {}+{} & a_{2n} & x_n & {}={} & b_2    \\
           & \vdots &       & \vdots &     &       &        &       & \vdots &     &       & \vdots \\
    a_{m1} & x_1    & {}+{} & a_{m2} & x_2 & {}+{} & \cdots & {}+{} & a_{mn} & x_n & {}={} & b_m    \\
  \end{alignedat}
  \right.
\end{equation*}
を考える

$a_{ij}$などは与えられた定数であり、\keyword{係数}と呼ばれる

$i$番目の式の$x_j$の係数を$a_{ij}$と書いている

\br

ここで、係数だけを集めて行列を作る
\begin{equation*}
  A = \begin{pmatrix}
    a_{11} & a_{12} & \cdots & a_{1n} \\
    a_{21} & a_{22} & \cdots & a_{2n} \\
           & \vdots &        & \vdots \\
    a_{m1} & a_{m2} & \cdots & a_{mn}
  \end{pmatrix}
\end{equation*}

すると、先ほどの連立方程式は、ベクトル形で
\begin{equation*}
  x_1 \vb*{a}_1 + x_2 \vb*{a}_2 + \cdots + x_n \vb*{a}_n = \vb*{b}
\end{equation*}
と書ける

\br

また、$n$個の未知数$x_1, x_2, \dots, x_n$からベクトルを作る
\begin{equation*}
  \vb*{x} = \begin{pmatrix}
    x_1    \\
    x_2    \\
    \vdots \\
    x_n
  \end{pmatrix}
\end{equation*}

すると、ベクトル形の方程式の左辺のベクトルを、行列$A$とベクトル$\vb*{x}$の積と考えて、$A\vb*{x}$と表記できる

こうして、もとの連立一次方程式は、行列形の方程式
\begin{equation*}
  A\vb*{x} = \vb*{b}
\end{equation*}
に書き換えられる

\sectionline
\section{行基本変形}
\marginnote{\refbookA p25}

連立一次方程式を行列によってとり扱うとき、1つ1つの方程式は行列の行によって表されている

よって、行列の行に関する次のような操作(変形)を考えることは自然である

\begin{definition}{行基本変形}
  行列への次の3種類の操作を\keyword{行基本変形}という
  \begin{enumerate}[label=\romanlabel]
    \item ある行の定数倍を他の行に加える
    \item ある行に0でない数をかける
    \item 2つの行を交換する
  \end{enumerate}
\end{definition}

原則として上三角型を目指してこのような変形を繰り返すが、いつでも上三角型にできるわけではなく、\keyword{行階段行列}と呼ばれる形を作っていくのが\keyword{掃き出し法}と呼ばれる手法である

\sectionline
\section{行階段行列}
\marginnote{\refbookA p26〜28}

掃き出し法では、あるステップで下の成分がすべて0になって、
\begin{equation*}
  \begin{pNiceArray}{>{\strut}cccccccc}% <-- % mandatory
    [margin, extra-margin=2pt,no-cell-nodes]
    \rowcolor{red!15} \spadesuit & * & * & * & * & * & * & *\\
    0 & \rowcolor{red!15} \spadesuit & * & * & * & * & * & * \\
    0 & 0 & \rowcolor{red!15} \spadesuit & * & * & * & * & * \\
    0 & 0 & 0 & \rowcolor{red!15} \spadesuit & * & * & * & * \\
    0 & 0 & 0 & 0 & 0 & 0 & 0 & 0
  \end{pNiceArray}
\end{equation*}
のような形になるのが典型例である

0でない成分を$\spadesuit$で、任意の値をもつ成分を$*$で表した

\br

一般には、成分が0ばかりの行が下にくる

そのような行を\keyword{零行}という

零行が現れない場合もあるし、複数現れる場合もある

\br

零行でない行に対して、一番左の0でない成分$\spadesuit$を\keyword{主成分}と呼ぶ

\br

先ほど示した形では、行の主成分は左上から斜め右下$45^\circ$方向にまっすぐ並んでいるが、一般にはそうできるとは限らない

しかし、次のような形には必ずできる
\begin{equation*}
  \begin{pNiceArray}{>{\strut}cccccccc}% <-- % mandatory
    [margin, extra-margin=2pt,no-cell-nodes]
    0 & \rowcolor{red!15} \spadesuit & * & * & * & * & * & * \\
    0 & 0 & 0 & \rowcolor{red!15} \spadesuit & * & * & * & * \\
    0 & 0 & 0 & 0 & \rowcolor{red!15} \spadesuit & * & * & * \\
    0 & 0 & 0 & 0 & 0 & 0 & 0 & \rowcolor{red!15} \spadesuit \\
    0 & 0 & 0 & 0 & 0 & 0 & 0 & 0
  \end{pNiceArray}
\end{equation*}

\begin{definition}{行階段行列}
  次の条件を満たす行列を\keyword{行階段行列}という
  \begin{itemize}
    \item 零行でない行の主成分が、下の行ほど1つ以上右にある
    \item 零行がある場合は、まとめてすべて下にある
  \end{itemize}
\end{definition}

どんな行列も、行基本変形の繰り返しで行階段行列にできる

\sectionline
\section{行列の階数}
\marginnote{\refbookA p28〜29}

行階段行列に変形することで、重要な量が読み取れる

\begin{definition}{行列の階数}
  行列$A$を行階段行列に変形したとき、零行でない行の個数を$A$の\keyword{階数}(rank)と呼び、$\rank(A)$と書く
\end{definition}

変形の結果として得られる行階段行列は1通りとは限らないし、変形の途中の掃き出しの手順も1通りとは限らないが、階数は$A$のみによって定まる値であることが後に証明できる

\sectionline

$A$が$m \times n$型ならば、行は$m$個なので、$\rank(A)$は0以上$m$以下の整数である

\br

行階段行列において、零行でない行の個数は主成分の個数と一致するので、階数は行階段行列に変形したときの主成分の個数でもある

\br

行基本行列の主成分は各列に高々1つなので、主成分の個数は列の個数$n$を超えない

\br

したがって、次の重要な評価が成り立つ
\begin{equation*}
  0 \leq \rank(A) \leq \min(m,n)
\end{equation*}

\sectionline
\section{簡約化された行階段行列}
\marginnote{\refbookA p29〜30}

必要に応じて、行階段行列をさらに変形して次のような形にする

\begin{equation*}
  \begin{pNiceArray}{>{\strut}cccccccc}% <-- % mandatory
    [margin, extra-margin=2pt,no-cell-nodes]
    0 & \rowcolor{red!15} 1 & * & 0 & 0 & * & * & 0 \\
    0 & 0 & 0 & \rowcolor{red!15} 1 & 0 & * & * & 0 \\
    0 & 0 & 0 & 0 & \rowcolor{red!15} 1 & * & * & 0 \\
    0 & 0 & 0 & 0 & 0 & 0 & 0 & \rowcolor{red!15} 1 \\
    0 & 0 & 0 & 0 & 0 & 0 & 0 & 0
  \end{pNiceArray}
\end{equation*}

行の主成分はすべて1で、主成分のある列の主成分以外の成分はすべて0である

この形を\keyword{簡約化された行階段行列}と呼ぶ

\br

与えられた行列$A$に対して、行基本変形の繰り返しで得られる行階段行列は一意的ではないが、簡約化された行階段行列は一意的であることを後に議論する

そこで、簡約化された行階段行列を$A_\circ$と書くことにする

\sectionline

変形の過程を
\begin{equation*}
  \text{行列}A \rightarrow \text{行階段行列} \rightarrow \text{簡約化された行階段行列}A_\circ
\end{equation*}
と2段階にわけるのは、計算の効率以上の意味がある

行階段行列にするところまでで解決する問題(解の存在と一意性など)もあるからである

\end{document}
