\documentclass[../../../topic_linear-equation]{subfiles}

\begin{document}

\sectionline
\section{掃き出し法}
\marginnote{\refbookA p18〜21}

連立一次方程式において、文字の個数や方程式の本数が増えた場合にも見通しよく計算を進めるためには、\keyword{掃き出し法}と呼ばれる方法がある

\br

掃き出し法の基本方針は、次の形を目指すことである

\begin{center}
  \systeme{
    \star x_1 + * x_2 + * x_3 = *,
    \star x_2 + * x_3 = *,
    \star x_3 = *
  }
\end{center}

\begin{itemize}
  \item $*$はどんな数であってもよい(同じ数でなくてもよい)
  \item $\star$は0でない数を意味する
\end{itemize}

この形の方程式は\keyword{上三角形}と呼ばれ、いつでもこの形に変形できるわけではないが、1つの理想形である

\end{document}
