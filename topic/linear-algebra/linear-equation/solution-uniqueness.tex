\documentclass[../../../topic_linear-algebra]{subfiles}

\begin{document}

\sectionline
\section{解の一意性}
\marginnote{\refbookA p37〜38}

ここまでの議論で、問題Bが解決している

\begin{theorem}{解の一意性}
  $A\vb*{x} = \vb*{b}$の解が存在するとき、
  \begin{equation*}
    \text{解が一意的である} \Longleftrightarrow \rank(A) = n
  \end{equation*}
  ここで、$n$は変数の個数である
\end{theorem}

\begin{proof}
  \todo{\refbookA p37 (定理1.5.8)}
\end{proof}

斉次形の場合の非自明解の存在問題も解決している

\begin{theorem}{斉次形の非自明解の存在条件}
  斉次形の方程式$A\vb*{x} = \vb*{0}$において、
  \begin{equation*}
    \text{自明解しか存在しない} \Longleftrightarrow \rank(A) = n
  \end{equation*}
  ここで、$n$は変数の個数である
\end{theorem}

\begin{proof}
  斉次形の場合は自明解が常に存在するので、解の一意性は、それ以外の解がないということである $\qed$
\end{proof}

\sectionline

自由変数を$x_{j_1}, \dots, x_{j_{n-r}}$とするとき、一般解の表示
\begin{equation*}
  \vb*{x} = \vb*{x}_0 + t_1\vb*{u}_1 + t_2\vb*{u}_2 + \cdots + t_{n-r}\vb*{u}_{n-r}
\end{equation*}
の$j_k$番目の成分は等式
\begin{equation*}
  x_{j_k} = t_k
\end{equation*}
を意味するので、解が与えられたとき、パラメータの値は直接に読み取れる

\br

このことから、
\begin{equation*}
  \vb*{x} = \vb*{x}_0 + t_1\vb*{u}_1 + t_2\vb*{u}_2 + \cdots + t_{n-r}\vb*{u}_{n-r}
\end{equation*}
によって解を表示する際の$n-r$個のパラメータの値は一意的に定まることがわかる

この事実は、$\vb*{u}_1, \vb*{u}_2, \dots, \vb*{u}_{n-r} \in \mathbb{R}^m$が\keyword{線形独立}であると表現される

\end{document}
