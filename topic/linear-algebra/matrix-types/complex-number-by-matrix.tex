\documentclass[../../../topic_linear-algebra]{subfiles}

\begin{document}

\sectionline
\section{行列と複素数}

\begin{equation*}
  I = \begin{pmatrix}
    0 & -1 \\
    1 & 0
  \end{pmatrix}
\end{equation*}
とおき、
\begin{equation*}
  aE + bI = \begin{pmatrix}
    a & -b \\
    b & a
  \end{pmatrix} \quad (a, b \in \mathbb{R})
\end{equation*}
という形の行列を\keyword{複素数}と呼ぶことにより、複素数の定義ができる

この定義では、通常は$a + bi$と書かれるものを行列として実現している

\br

\todo{\refbookD p43〜49}

\end{document}
