\documentclass[../../../topic_linear-algebra]{subfiles}

\begin{document}

\sectionline
\section{エルミート行列と対称行列}
\marginnote{\refbookF p275〜276}

\begin{definition}{エルミート行列}
  複素正方行列$A$が次を満たすとき、$A$を\keyword{エルミート行列}という
  \begin{equation*}
    A^* = A
  \end{equation*}
\end{definition}

$A$が実正方行列のときは、
\begin{equation*}
  A\text{がエルミート行列} \Longleftrightarrow {}^tA = A
\end{equation*}
となり、このような$A$は\hyperref[def:symmetric-matrix]{対称行列}、あるいは\keyword{実対称行列}と呼ばれる

\sectionline
\section{エルミート行列の固有値}
\marginnote{\refbookF p282〜283 \\ \refbookA p201、p203}

行列の成分が実数であっても、特性方程式の根は一般には実数とは限らない

つまり、固有値は一般には複素数であるが、エルミート行列については次が成り立つ

\begin{theorem}{エルミート行列の固有値の実数性}\label{thm:eigenvalues-of-hermitian-are-real}
  エルミート行列の固有値はすべて実数である
\end{theorem}

\begin{proof}
  エルミート行列$A$の固有ベクトルを$\vb*{v}$とし、その固有値を$\alpha \in \mathbb{C}^n$とすると、
  \begin{equation*}
    A\vb*{v} = \alpha \vb*{v}
  \end{equation*}
  より、次が成り立つ
  \begin{align*}
    (A\vb*{v}, \vb*{v}) & = (\alpha \vb*{v}, \vb*{v}) \\
                        & = \alpha (\vb*{v}, \vb*{v})
  \end{align*}
  \br

  一方、\hyperref[thm:adjoint-identity]{随伴公式}から、次のようにも書ける
  \begin{equation*}
    (A\vb*{v}, \vb*{v}) = (\vb*{v}, A^*\vb*{v})
  \end{equation*}
  $A$がエルミート行列であることから、$A^* = A$なので、
  \begin{align*}
    (A\vb*{v}, \vb*{v}) & = (\vb*{v}, A\vb*{v})      \\
                        & = (\vb*{v}, \alpha\vb*{v})
  \end{align*}
  \hyperref[thm:conjugate-linearity-of-inner-product]{内積の共役線形性}に注意して、
  \begin{equation*}
    (A\vb*{v}, \vb*{v}) = \overline{\alpha}(\vb*{v}, \vb*{v})
  \end{equation*}

  \br

  ここまでで得られた$(A\vb*{v}, \vb*{v})$の2通りの表現をまとめると、
  \begin{equation*}
    \alpha (\vb*{v}, \vb*{v}) = \overline{\alpha}(\vb*{v}, \vb*{v})
  \end{equation*}
  移項して、
  \begin{equation*}
    (\alpha - \overline{\alpha})(\vb*{v}, \vb*{v}) = 0
  \end{equation*}
  ここで、$\vb*{v}$は固有ベクトルなので、$\vb*{v} \neq \vb*{0}$である

  よって、$(\vb*{v}, \vb*{v}) \neq 0$で両辺を割ることができ、次を得る
  \begin{equation*}
    \alpha = \overline{\alpha}
  \end{equation*}
  すなわち、$\alpha$は実数である $\qed$
\end{proof}

\sectionline

エルミート行列では、固有値が実数であることがうまく活きて、次の性質も成り立つ

\begin{theorem}{エルミート行列の固有値の直交性}\label{thm:hermitian-eigenvectors-orthogonality}
  エルミート行列の相異なる固有値を持つ固有ベクトルは直交する

  すなわち、エルミート行列$A$の固有ベクトル$\vb*{u},\,\vb*{v}$がそれぞれ固有値$\alpha,\,\beta \in \mathbb{R}^n$を持つとし、$\alpha \neq \beta$ならば、
  \begin{equation*}
    (\vb*{u}, \vb*{v}) = 0
  \end{equation*}
  が成り立つ
\end{theorem}

\begin{proof}
  固有値と固有ベクトルの定義より、
  \begin{align*}
    A\vb*{u} & = \alpha \vb*{u} \\
    A\vb*{v} & = \beta \vb*{v}
  \end{align*}
  が成り立つ

  \br

  一方、\hyperref[thm:adjoint-identity]{随伴公式}より、
  \begin{equation*}
    (A\vb*{u}, \vb*{v}) = (\vb*{u}, A^*\vb*{v})
  \end{equation*}
  であるが、$A$はエルミート行列なので、$A^* = A$が成り立つ
  \begin{equation*}
    (A\vb*{u}, \vb*{v}) = (\vb*{u}, A\vb*{v})
  \end{equation*}
  先ほどの固有値と固有ベクトルの関係を代入して、
  \begin{equation*}
    (\alpha \vb*{u}, \vb*{v}) = (\vb*{u}, \beta \vb*{v})
  \end{equation*}
  ここで、$\alpha,\,\beta$は実数なので、\hyperref[thm:conjugate-linearity-of-inner-product]{内積の共役線形性}を考慮しても、
  \begin{equation*}
    \alpha (\vb*{u}, \vb*{v}) = \beta (\vb*{u}, \vb*{v})
  \end{equation*}
  として、スカラーをそのまま外に出すことができる

  \br

  よって、
  \begin{equation*}
    (\alpha - \beta)(\vb*{u}, \vb*{v}) = 0
  \end{equation*}
  であるが、$\alpha \neq \beta$なので、$(\alpha - \beta) \neq 0$で両辺を割ることができ、
  \begin{equation*}
    (\vb*{u}, \vb*{v}) = 0
  \end{equation*}
  を得る $\qed$
\end{proof}

\subsection{エルミート行列の対角化に向けた考察}

$H$を$n$次エルミート行列とすると、その固有値は$n$個の実数として$\alpha_1,\ldots,\alpha_n$とおける

そして、$\alpha_i$に属する固有ベクトル$\vb*{v}_i$をとると、$\vb*{v}_1, \ldots, \vb*{v}_n$はどの2つも互いに直交する

\br

そこで、それぞれを次のように正規化する
\begin{equation*}
  \vb*{u}_i = \frac{\vb*{v}_i}{\|\vb*{v}_i\|} \quad (i = 1, \ldots, n)
\end{equation*}
すると、$\vb*{u}_1, \ldots, \vb*{u}_n$は互いに直交する単位ベクトルであるので、
\begin{equation*}
  U = (\vb*{u}_1, \ldots, \vb*{u}_n)
\end{equation*}
とおけば、\hyperref[thm:unitary-iff-columns-orthonormal]{$U$はユニタリ行列となる}

\br

$\vb*{u}_i$は$H$の各固有ベクトル$\vb*{v}_i$をスカラー倍したものなので、
\begin{equation*}
  H\vb*{u}_i = \alpha_i \vb*{u}_i
\end{equation*}
という関係が成り立つ

つまり、$U$の列ベクトル$\vb*{u}_1, \ldots, \vb*{u}_n$はそれぞれ$H$の固有値$\alpha_1, \ldots, \alpha_n$に属する固有ベクトルである

\br

さらに、ユニタリ行列はその\hyperref[def:unitary-matrix]{定義}から明らかに正則行列であるので、\hyperref[thm:diagonalization-columns-are-eigenvectors]{対角化行列の列ベクトルと固有ベクトルの対応}を振り返ると、
\begin{shaded}
  エルミート行列はユニタリ行列を用いて対角化できる
\end{shaded}
という「予感」がしてくる

\br

まだ「予感」としかいえないのは、エルミート行列の固有値$\alpha_1, \ldots, \alpha_n$が重複している可能性があるからである

\end{document}
