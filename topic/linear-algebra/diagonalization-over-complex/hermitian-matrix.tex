\documentclass[../../../topic_linear-algebra]{subfiles}

\begin{document}

\sectionline
\section{エルミート行列と対称行列}
\marginnote{\refbookF p275〜276}

\begin{definition}{エルミート行列}
  複素正方行列$A$が次を満たすとき、$A$を\keyword{エルミート行列}という
  \begin{equation*}
    A^* = A
  \end{equation*}
\end{definition}

$A$が実正方行列のときは、
\begin{equation*}
  A\text{がエルミート行列} \Longleftrightarrow {}^tA = A
\end{equation*}
となり、このような$A$は\hyperref[def:symmetric-matrix]{対称行列}、あるいは\keyword{実対称行列}と呼ばれる

\end{document}
