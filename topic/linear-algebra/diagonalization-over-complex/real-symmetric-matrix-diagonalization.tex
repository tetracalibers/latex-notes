\documentclass[../../../topic_linear-algebra]{subfiles}

\begin{document}

\sectionline
\section{実対称行列の対角化}
\marginnote{\refbookF p282〜284 \\ \refbookC p200〜201 \\ \refbookA p201}

エルミート行列は正規行列なので、次のことがいえる

\begin{theorem*}{エルミート行列のユニタリ対角化}
  エルミート行列はユニタリ行列を用いて対角化できる
\end{theorem*}

この定理を実行列の世界にもってくると、次のようになる

\begin{theorem*}{実対称行列の直交対角化}
  実対称行列は直交行列を用いて対角化できる
\end{theorem*}

\end{document}
