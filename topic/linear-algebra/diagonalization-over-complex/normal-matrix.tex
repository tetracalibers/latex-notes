\documentclass[../../../topic_linear-algebra]{subfiles}

\begin{document}

\sectionline
\section{正規行列}
\marginnote{\refbookC p197〜200 \\ \refbookF p287〜292 \\ \refbookA p209}

エルミート行列の対角化について議論するために、エルミート行列・ユニタリ行列を含むより包括的な概念として\keyword{正規行列}を導入する

\begin{definition}{正規行列}
  複素正方行列$A$が次を満たすとき、$A$を\keyword{正規行列}という
  \begin{equation*}
    A A^* = A^* A
  \end{equation*}
\end{definition}

\subsection{正規行列の例}

$A$をエルミート行列とすると、$A^* = A$なので、
\begin{align*}
  A A^* & = A^2 \\
  A^* A & = A^2
\end{align*}
となり、正規行列の定義を満たす

\begin{theorem}{エルミート行列の正規行列性}
  エルミート行列は正規行列である
\end{theorem}

\br

また、$A$をユニタリ行列とすると、$A^* = A^{-1}$なので、
\begin{align*}
  A A^* & = A A^{-1} = E \\
  A^* A & = A^{-1} A = E
\end{align*}
となり、こちらも正規行列の定義を満たす

\begin{theorem}{ユニタリ行列の正規行列性}
  ユニタリ行列は正規行列である
\end{theorem}

\subsection{正規行列の性質}

\begin{theorem}{todo}
  複素正方行列$A$が正規行列であることは、任意の$\vb*{v} \in \mathbb{C}^n$に対し、
  \begin{equation*}
    \|A\vb*{v}\| = \|A^*\vb*{v}\|
  \end{equation*}
  が成り立つことと同値である
\end{theorem}

\begin{proof}
  \todo{\refbookA p262 問6.9 (1)}
\end{proof}

\br

\begin{theorem}{todo}
  $A$を正規行列とするとき、$\vb*{v}$が$A$の固有値$\alpha$の固有ベクトルならば、$\vb*{v}$は$A^*$の固有値$\overline{\alpha}$の固有ベクトルである

  すなわち、
  \begin{equation*}
    A\vb*{v} = \alpha \vb*{v} \Longrightarrow A^*\vb*{v} = \overline{\alpha} \vb*{v}
  \end{equation*}
\end{theorem}

\begin{proof}
  \todo{\refbookA p262 問6.9 (2)}
\end{proof}

\end{document}
