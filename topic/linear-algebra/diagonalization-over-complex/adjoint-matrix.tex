\documentclass[../../../topic_linear-algebra]{subfiles}

\begin{document}

\sectionline
\section{転置行列と随伴行列}
\marginnote{\refbookF p275}

複素正方行列$A$の転置行列において、各成分をその共役複素数に置き換えた行列を\keyword{随伴行列}という

\begin{definition}{随伴行列}
  複素正方行列$A = (a_{ij})$に対し、$\overline{a_{ji}}$を$(i,j)$成分にもつ行列${}^t\overline{A}$を$A$の\keyword{随伴行列}といい、$A^*$と表す
\end{definition}

実数$x$の複素共役は$\overline{x} = x$であるので、$A$が実行列のときは、
\begin{equation*}
  A^* = {}^t A
\end{equation*}
すなわち、
\begin{shaded}
  実行列の世界では、随伴行列は転置行列
\end{shaded}
にすぎない

\sectionline

\hyperref[thm:transpose-involution]{転置を二回行うと元に戻る}ことと同様に、次が成り立つ

\begin{theorem}{随伴行列の自己反転性}
  複素正方行列$A$に対し、随伴行列を二回とると元に戻る
  \begin{equation*}
    (A^*)^* = A
  \end{equation*}
\end{theorem}

\begin{proof}
  随伴行列の定義より、
  \begin{equation*}
    (A^*)^* = {}^t\overline{A^*} = {}^t\overline{{}^t\overline{A}}
  \end{equation*}

  $A = (a_{ij})$とすると、$A$の各成分を共役複素数にした行列は、
  \begin{equation*}
    \overline{A} = (\overline{a_{ij}})
  \end{equation*}
  これを転置すると、
  \begin{equation*}
    {}^t\overline{A} = (\overline{a_{ji}})
  \end{equation*}
  さらに、もう一度各成分の複素共役をとると、
  \begin{equation*}
    {}^t\overline{{}^t\overline{A}} = (\overline{\overline{a_{ji}}}) = (a_{ji})
  \end{equation*}
  したがって、
  \begin{equation*}
    (A^*)^* = {}^t\overline{{}^t\overline{A}} = (a_{ij}) = A
  \end{equation*}
  が成り立つ $\qed$
\end{proof}

\sectionline

転置行列と複素共役の性質から、次の性質が成り立つ

\begin{theorem}{積に対するエルミート共役の順序反転性}\label{thm:adjoint-of-product}
  複素行列$A\,B$の積$AB$が定義できるとき、
  \begin{equation*}
    (AB)^* = B^* A^*
  \end{equation*}
\end{theorem}

\begin{proof}
  \todo{}
\end{proof}

\sectionline

随伴行列と\hyperref[def:standard-inner-product-Cn]{標準内積}は、次のような関係で結ばれる

\begin{theorem}{随伴公式}\label{thm:adjoint-identity}
  複素行列$A$と計量空間上のベクトル$\vb*{u},\,\vb*{v}$に対し、
  \begin{equation*}
    (A\vb*{u}, \vb*{v}) = (\vb*{u}, A^*\vb*{v})
  \end{equation*}
\end{theorem}

\begin{proof}
  \hyperref[thm:inner-product-as-transpose-product]{転置を用いて内積を表す}と、
  \begin{equation*}
    (A\vb*{u},\vb*{v}) = {}^t(A\vb*{u}) \overline{\vb*{v}}
  \end{equation*}
  \hyperref[thm:transpose-of-product]{転置と行列積の順序反転性}より、${}^t(A\vb*{u}) = {}^t\vb*{u} \transpose{A}$なので、
  \begin{equation*}
    (A\vb*{u},\vb*{v}) = ({}^t\vb*{u} \transpose{A}) \overline{\vb*{v}}
  \end{equation*}
  行列の積の結合法則を用いて、
  \begin{equation*}
    (A\vb*{u},\vb*{v}) = {}^t\vb*{u} (\transpose{A}\overline{\vb*{v}})
  \end{equation*}

  ここで、$\overline{{}^t \overline{A}}$は、$A=(a_{ij})$とすると、
  \begin{enumerate}
    \item $\overline{A} = (\overline{a_{ij}})$
    \item ${}^t\overline{A} = (\overline{a_{ji}})$
    \item $\overline{{}^t\overline{A}} = (\overline{\overline{a_{ji}}}) = (a_{ji}) = {}^t A$
  \end{enumerate}
  となり、${}^t A$と一致する

  これを用いて書き換えると、
  \begin{equation*}
    (A\vb*{u},\vb*{v}) = {}^t\vb*{u} (\overline{{}^t \overline{A}} \overline{\vb*{v}})
  \end{equation*}
  複素共役の積の性質$\overline{z_1} \cdot \overline{z_2} = \overline{z_1 z_2}$を用いて、
  \begin{equation*}
    (A\vb*{u},\vb*{v}) = {}^t\vb*{u} \overline{{}^t \overline{A} \vb*{v}}
  \end{equation*}

  この時点で、右辺を内積として書き直すと、$A\vb*{v}$の複素共役がなくなることに注意して、
  \begin{equation*}
    (A\vb*{u},\vb*{v}) = (\vb*{u}, {}^t \overline{A} \vb*{v})
  \end{equation*}

  随伴行列の定義$A^* = {}^t \overline{A}$より、
  \begin{equation*}
    (A\vb*{u},\vb*{v}) = (\vb*{u}, A^* \vb*{v})
  \end{equation*}
  となり、目的の等式が得られた $\qed$
\end{proof}

\end{document}
