\documentclass[../../../topic_linear-algebra]{subfiles}

\usepackage{xr-hyper}
\externaldocument{../../../.tex_intermediates/topic_linear-algebra}

\begin{document}

\sectionline
\section{基底によるベクトルの展開}
\marginnote{\refbookO p31〜32}

縦ベクトルをケット、横ベクトルをブラで表すことにする。

\subsection{測定値からベクトルを特定する}

基底$\ket{\vb*{a}_1}, \ket{\vb*{a}_2}$を用いて、ベクトル$\ket{\vb*{x}}$を次のような線形結合で表そう。
\begin{equation*}
  \ket{\vb*{x}} = x_1 \ket{\vb*{a}_1} + x_2 \ket{\vb*{a}_2}
\end{equation*}
ここで、係数$x_1, x_2$を求めるために、左からブラ$\bra{\vb*{a}_1}, \bra{\vb*{a}_2}$を作用させる。
\begin{align*}
  \braket{\vb*{a}_1|\vb*{x}} & = x_1 \braket{\vb*{a}_1|\vb*{a}_1} + x_2 \braket{\vb*{a}_1|\vb*{a}_2} \\
  \braket{\vb*{a}_2|\vb*{x}} & = x_1 \braket{\vb*{a}_2|\vb*{a}_1} + x_2 \braket{\vb*{a}_2|\vb*{a}_2} 
\end{align*}

ブラとケットが組み合わされたブラケット$\braket{\cdot|\cdot}$はスカラー値を表しているので、これは未知数$x_1, x_2$に関する連立方程式である。

この連立方程式を解けば、係数$x_1, x_2$が求まり、ベクトル$\ket{\vb*{x}}$を特定できる。

\subsection{直交基底による展開}

直交基底を用いると、線形結合の係数は連立方程式を解くことなく、内積を用いて直接計算できる。

\br

直交基底$\ket{\vb*{u}_1}, \ket{\vb*{u}_2}$に対して、ベクトル$\ket{\vb*{x}}$が次のように表されるとする。
\begin{equation*}
  \ket{\vb*{x}} = x_1 \ket{\vb*{u}_1} + x_2 \ket{\vb*{u}_2}
\end{equation*}

左からブラ$\bra{\vb*{u}_1}, \bra{\vb*{u}_2}$を作用させると、
\begin{align*}
  \braket{\vb*{u}_1|\vb*{x}} & = x_1 \braket{\vb*{u}_1|\vb*{u}_1} + x_2 \braket{\vb*{u}_1|\vb*{u}_2} \\
  \braket{\vb*{u}_2|\vb*{x}} & = x_1 \braket{\vb*{u}_2|\vb*{u}_1} + x_2 \braket{\vb*{u}_2|\vb*{u}_2}
\end{align*}

ここで、直交していれば内積は0になるので、$\braket{\vb*{u}_1|\vb*{u}_2}$や$\braket{\vb*{u}_2|\vb*{u}_1}$は0となる。
\begin{align*}
  \braket{\vb*{u}_1|\vb*{x}} & = x_1 \braket{\vb*{u}_1|\vb*{u}_1} \\
  \braket{\vb*{u}_2|\vb*{x}} & = x_2 \braket{\vb*{u}_2|\vb*{u}_2}
\end{align*}

この式から、係数$x_1, x_2$は次のように求まる。
\begin{align*}
  x_1 & = \frac{\braket{\vb*{u}_1|\vb*{x}}}{\braket{\vb*{u}_1|\vb*{u}_1}} \\
  x_2 & = \frac{\braket{\vb*{u}_2|\vb*{x}}}{\braket{\vb*{u}_2|\vb*{u}_2}}
\end{align*}

\begin{theorem}{直交基底によるベクトルの展開}{vector-expansion-by-orthogonal-basis}
  計量空間$V$の直交基底$\vb*{u}_1, \ldots, \vb*{u}_n$に対して、任意のベクトル$\vb*{v} \in V$は
  \begin{equation*}
    \vb*{v} = \sum_{i=1}^n \frac{(\vb*{v}, \vb*{u}_i)}{(\vb*{u}_i, \vb*{u}_i)} \vb*{u}_i
  \end{equation*}
  と表すことができる。
\end{theorem}

\begin{proof}
  ベクトル$\vb*{v}$が次のような線形結合
  \begin{equation*}
    \vb*{v} = c_1 \vb*{u}_1 + \cdots + c_n \vb*{u}_n
  \end{equation*}
  で表されるとし、係数を求めることを目指す。

  このとき、$\vb*{u}_j\, (j=1,2,\ldots,n)$との内積をとると、
  \begin{align*}
    (\vb*{v}, \vb*{u}_j) & = (c_1 \vb*{u}_1 + \cdots + c_n \vb*{u}_n, \vb*{u}_j)                           \\
                         & = c_1 (\vb*{u}_1, \vb*{u}_j) + \cdots + c_n (\vb*{u}_n, \vb*{u}_j) \\
                         & = \sum_{i=1}^n c_i (\vb*{u}_i, \vb*{u}_j)
  \end{align*}
  となるが、$\vb*{u}_i$は直交系であるため、$i \neq j$のとき$(\vb*{u}_i, \vb*{u}_j) = 0$である。

  よって、上の式において残るのは、$i=j$の項だけとなり、
  \begin{equation*}
    (\vb*{v}, \vb*{u}_j) = c_j (\vb*{u}_j, \vb*{u}_j)
  \end{equation*}

  ここで、直交系の定義より$\vb*{u}_j \neq \vb*{o}$なので、$(\vb*{u}_j, \vb*{u}_j) \neq 0$である。

  そこで、両辺を$(\vb*{u}_j, \vb*{u}_j)$で割ることができ、
  \begin{equation*}
    c_j = \frac{(\vb*{v}, \vb*{u}_j)}{(\vb*{u}_j, \vb*{u}_j)}
  \end{equation*}
  が得られる。 $\qed$
\end{proof}

\subsection{正規直交基底による展開}

さらに、$\ket{\vb*{u}_1}, \ket{\vb*{u}_2}$が正規直交基底であるなら、これらのノルムは1であるので、
\begin{equation*}
  \braket{\vb*{u}_1|\vb*{u}_1} = \|\vb*{u}_1\|^2 = 1, \quad \braket{\vb*{u}_2|\vb*{u}_2} = \|\vb*{u}_2\|^2 = 1
\end{equation*}

よって、係数$x_1, x_2$はさらに簡単な形で表すことができる。
\begin{align*}
  x_1 & = \braket{\vb*{u}_1|\vb*{x}} \\
  x_2 & = \braket{\vb*{u}_2|\vb*{x}}
\end{align*}

\begin{theorem}{正規直交基底によるベクトルの展開}{expansion-in-orthonormal-basis}
  計量空間$V$の正規直交基底$\vb*{u}_1, \ldots, \vb*{u}_n$に対して、任意のベクトル$\vb*{v} \in V$は
  \begin{equation*}
    \vb*{v} = \sum_{i=1}^n (\vb*{v}, \vb*{u}_i) \vb*{u}_i
  \end{equation*}
  と表すことができる。
\end{theorem}

\begin{proof}
  正規直交基底の場合、
  \begin{equation*}
    (\vb*{u}_i,\vb*{u}_i) = \|\vb*{u}_i\|^2 = 1
  \end{equation*}
  であることを用いると、\thmref{thm:vector-expansion-by-orthogonal-basis}において分母が1となり、この形が得られる。 $\qed$
\end{proof}

\end{document}
