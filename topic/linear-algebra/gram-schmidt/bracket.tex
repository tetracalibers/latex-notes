\documentclass[../../../topic_linear-algebra]{subfiles}

\begin{document}

\sectionline
\section{観測装置としての内積}
\marginnote{
  \refbookO p26〜27 \\
  \refweb{線形代数の基礎のキソ}{https://www1.econ.hit-u.ac.jp/kawahira/courses/kiso.html}
}

ここでは、内積$(\vb*{a}_1, \vb*{a}_2)$において、$\vb*{a}_1$と$\vb*{a}_2$の機能を分離して、
\begin{emphabox}
  \begin{spacebox}
    \begin{center}
      内積$(\vb*{a}_1, \vb*{a}_2)$とは、$\vb*{a}_1$で$\vb*{a}_2$を測って得られる値
    \end{center}
  \end{spacebox}
\end{emphabox}
という再解釈を行う。

\subsection{縦ベクトルが基本}

内積$(\vb*{a}_1, \vb*{a}_2)$は、次のように表すこともできた。
\begin{equation*}
  \vb*{a}_1^\top \vb*{a}_2
\end{equation*}
つまり、縦ベクトル$\vb*{a}_2$に、横ベクトル$\vb*{a}_1^\top$を作用させることで、内積が得られると読むことができる。

\br

このように、$\vb*{a}_2$が測りたい対象で、$\vb*{a}_1$がその測定器であるという視点を持つことができる。

\subsection{ブラとケット}

この視点を表す上で有用なのが、\keyword{ブラケット記法}である。

\begin{itemize}
  \item 横ベクトルに対応する記号として$\bra{\vb*{a}_1}$を定義し、これを\keyword{ブラ}ベクトルと呼ぶ
  \item 縦ベクトルに対応する記号として$\ket{\vb*{a}_2}$を定義し、これを\keyword{ケット}ベクトルと呼ぶ
\end{itemize}

これらの記号を用いると、内積は次のように表せる。
\begin{equation*}
  \braket{\vb*{a}_1|\vb*{a}_2}
\end{equation*}
このように、ブラとケットが組み合わされると、スカラー値$\braket{\vb*{a}_1|\vb*{a}_2}$が得られる。

\br

ケットという「観測対象」に対して、結果としてスカラー値を返すような関数(内積)を考えたとき、そのための「観測装置」となるのがブラである。

\br

観測装置であるブラ(横ベクトル)は、縦ベクトルからスカラー値を得るための写像$\mathbb{R}^n \to \mathbb{R}$とみることもできる。

この捉え方を一般化すると、\keyword{線形汎関数}という概念に結びつく。

\end{document}
