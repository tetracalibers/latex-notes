\documentclass[../../../topic_linear-algebra]{subfiles}

\begin{document}

\sectionline
\section{ベクトルの正射影}
\marginnote{\refbookF p184〜185}

正規直交基底をつくるにあたって、次の概念が重要になる

\begin{definition}{正射影}
  $\vb*{0}$でないベクトル$\vb*{a}$が与えられているとき、ベクトル$\vb*{x}$に対し、
  \begin{enumerate}[label=\romanlabel]
    \item $\vb*{p}$が$\vb*{a}$と平行
    \item $\vb*{x - p}$が$\vb*{a}$と直交
  \end{enumerate}
  という条件を満たすベクトル$\vb*{p}$を、$\vb*{x}$の$\vb*{a}$への\keyword{正射影}という
\end{definition}

\begin{theorem}{正射影の公式}
  ベクトル$\vb*{x}$のベクトル$\vb*{a}$への正射影$\vb*{p}$は、次のように表される
  \begin{equation*}
    \vb*{p} = \frac{(\vb*{x}, \vb*{a})}{(\vb*{a}, \vb*{a})} \vb*{a} = \frac{(\vb*{x}, \vb*{a})}{\|\vb*{a}\|^2} \vb*{a}
  \end{equation*}
\end{theorem}

\begin{proof}
  $\vb*{p}$が$\vb*{a}$と平行であることから、
  \begin{equation*}
    \vb*{p} = k \vb*{a} \quad (k \in K)
  \end{equation*}
  また、$\vb*{x - p}$が$\vb*{a}$と直交することから、
  \begin{equation*}
    (\vb*{x-p},\vb*{a}) = 0
  \end{equation*}
  よって、
  \begin{equation*}
    (\vb*{x} - k \vb*{a}, \vb*{a}) = 0
  \end{equation*}
  内積の双線形性より、
  \begin{equation*}
    (\vb*{x}, \vb*{a}) - k (\vb*{a}, \vb*{a}) = 0
  \end{equation*}
  ここで、正射影の定義より$\vb*{a} \neq \vb*{0}$なので、$(\vb*{a}, \vb*{a}) \neq 0$である

  よって、$(\vb*{a}, \vb*{a})$で割ることができ、
  \begin{equation*}
    k = \frac{(\vb*{x}, \vb*{a})}{(\vb*{a}, \vb*{a})}
  \end{equation*}
  と$k$が定まる

  最初の式に代入すると、
  \begin{equation*}
    \vb*{p} = \frac{(\vb*{x}, \vb*{a})}{(\vb*{a}, \vb*{a})} \vb*{a}
  \end{equation*}
  が得られる $\qed$
\end{proof}

\sectionline
\section{グラム・シュミットの直交化法}
\marginnote{\refbookC p119〜120 \\ \refbookF p182〜184 \\ \refbookA p82〜83}

計量空間$V$の線型独立なベクトル$\vb*{a}_1, \vb*{a}_2, \ldots, \vb*{a}_n$から、正規直交系をつくる方法を考える

\subsection{正規化}

まずは、$\vb*{a}_1$から、ノルムが1であるベクトルをつくる(\keyword{正規化})

そのためには、
\begin{equation*}
  \vb*{e}_1 = \frac{\vb*{a}_1}{\|\vb*{a}_1\|}
\end{equation*}
とすればよい

ここで、$\vb*{e}_1$は$\vb*{a}_1$をスカラー倍しただけなので、$\vb*{e}_1$と$\vb*{a}_1$は平行である

\subsection{直交化}

次に、$\vb*{e}_1$と直交するような$\vb*{e}_2$をつくる

そのために、$\vb*{a}_2$から、$\vb*{a}_2$の$\vb*{e}_1$への正射影を引いたものは、$\vb*{e}_1$と直交することを利用する

\br

$\vb*{a}_2$の$\vb*{e}_1$への正射影は、次のように計算できる
\begin{equation*}
  \frac{(\vb*{a}_2, \vb*{e}_1)}{\|\vb*{e}_1|^2} \vb*{e}_1 = (\vb*{a}_2, \vb*{e}_1) \vb*{e}_1
\end{equation*}

そこで、
\begin{equation*}
  \vb*{u}_2 = \vb*{a}_2 - (\vb*{a}_2, \vb*{e}_1) \vb*{e}_1
\end{equation*}
とおくと、$\vb*{u}_2$は$\vb*{e}_1$と直交する

\br

$\vb*{a}_2$と$\vb*{a}_1$が、したがって$\vb*{a}_2$と$\vb*{e}_1$が線型独立であることから、$\vb*{u}_2 \neq \vb*{0}$である

なぜなら、もし$\vb*{u}_2 = \vb*{0}$ならば、$\vb*{a}_2$は$\vb*{e}_1$の線形結合で表されることになり、$\vb*{a}_1, \vb*{a}_2$は線型従属になるからである

\br

そこで、$\vb*{u}_2$を次のように正規化することができ、
\begin{equation*}
  \vb*{e}_2 = \frac{\vb*{u}_2}{\|\vb*{u}_2\|}
\end{equation*}
とすれば、$\vb*{e}_2$は$\vb*{e}_1$と直交するノルムが1のベクトルになる

\sectionline

以上の手順を繰り返すことで、線型独立なベクトル$\vb*{a}_1, \vb*{a}_2, \ldots, \vb*{a}_n$から、正規直交系$\vb*{e}_1, \vb*{e}_2, \ldots, \vb*{e}_n$を得ることができる

\br

このような方法を\keyword{グラム・シュミットの直交化法}という

\begin{theorem}{グラム・シュミットの直交化法}
  計量空間$V$の線型独立なベクトル$\vb*{a}_1, \ldots, \vb*{a}_n$から、正規直交系$\vb*{e}_1, \ldots, \vb*{e}_n$を次のように構成できる
  \begin{gather*}
    \vb*{u}_k = \vb*{a}_k - \sum_{j=1}^{k-1} (\vb*{a}_k, \vb*{e}_j) \vb*{e}_j \\
    \vb*{e}_k = \frac{\vb*{u}_k}{\|\vb*{u}_k\|}
  \end{gather*}
  ここで、$k = 1, 2, \ldots, n$である
\end{theorem}

\subsection{正規直交基底の存在}

さらに、次の定理により、グラム・シュミットの直交化法は、線型独立なベクトルから正規直交系を得るだけでなく、任意の基底から正規直交基底を得る手法としても利用できる

\begin{theorem}{グラム・シュミットの直交化と生成空間の不変性}
  計量空間$V$の線型独立なベクトル$\vb*{a}_1, \ldots, \vb*{a}_n$から、グラム・シュミットの直交化法を用いて得られた正規直交系を$\vb*{e}_1, \ldots, \vb*{e}_n$とすると、$\vb*{a}_1,\ldots,\vb*{a}_n$が張る空間と$\vb*{e}_1,\ldots,\vb*{e}_n$が張る空間は一致する
  \begin{equation*}
    \langle \vb*{a}_1, \ldots, \vb*{a}_n \rangle = \langle \vb*{e}_1, \ldots, \vb*{e}_n \rangle
  \end{equation*}
\end{theorem}

\begin{proof}
  グラム・シュミットの直交化法では、各ステップ $k$ において、まず $\vb*{a}_k$ からその前に得られた直交ベクトル $\vb*{e}_1, \ldots, \vb*{e}_{k-1}$ への射影を引くことで、$\vb*{a}_k$ に直交するベクトルを構成する

  すなわち、
  \begin{equation*}
    \vb*{u}_k = \vb*{a}_k - \sum_{j=1}^{k-1} (\vb*{a}_k, \vb*{e}_j) \vb*{e}_j
  \end{equation*}
  と定義し、その後これを正規化して $\vb*{e}_k$ とする

  \br

  ここで、$\vb*{u}_k$は右辺の形から明らかなように、$\vb*{a}_k$ と $\vb*{e}_1, \ldots, \vb*{e}_{k-1}$ の線型結合である

  そしてさらに各 $\vb*{e}_j$($j < k$)は、それ以前の $\vb*{a}_1, \ldots, \vb*{a}_j$ の線型結合であることから、$\vb*{u}_k$ は結局 $\vb*{a}_1, \ldots, \vb*{a}_k$ の線型結合として書ける

  したがって、$\vb*{e}_k$ も $\vb*{a}_1, \ldots, \vb*{a}_k$ の線型結合となり、$\vb*{e}_1, \ldots, \vb*{e}_n$ はすべて $\vb*{a}_1, \ldots, \vb*{a}_n$ の線型結合である

  \br

  よって、すべての $\vb*{e}_k$ は $\langle \vb*{a}_1, \ldots, \vb*{a}_n \rangle$ に属することになり、
  \begin{equation*}
    \langle \vb*{e}_1, \ldots, \vb*{e}_n \rangle \subset \langle \vb*{a}_1, \ldots, \vb*{a}_n \rangle
  \end{equation*}
  が成り立つ

  \br

  両辺の部分空間の次元を考えると、$\vb*{a}_1, \ldots, \vb*{a}_n$ が線型独立であるため、$\langle \vb*{a}_1, \ldots, \vb*{a}_n \rangle$ の次元は $n$ である

  一方、$\vb*{e}_1, \ldots, \vb*{e}_n$ も\hyperref[thm:orthogonal-set-is-independent]{直交系であることから線型独立}であるため、$\langle \vb*{e}_1, \ldots, \vb*{e}_n \rangle$ の次元も $n$ である

  よって、\hyperref[thm:equal-dim-implies-equal-subspace]{部分空間の次元が等しいことから、両者は一致}する $\qed$
\end{proof}

\br

このように、グラム・シュミットの直交化法は、内積が定められている空間(計量空間)には正規直交基底が存在することを示している

\begin{theorem}{正規直交基底の存在}
  $\{\vb*{0}\}$でない任意の計量空間は正規直交基底を持つ
\end{theorem}

\subsection{線形従属なベクトルに適用した場合}\label{sec:gram-schmidt-with-dependent-vectors}

与えられたベクトルが線型独立でない場合にグラム・シュミットの直交化法を適用すると、いずれ射影を引いたベクトルが$\vb*{0}$になる

\br

ある$\vb*{a}_k$が、前のベクトルたち$\vb*{a}_1,\ldots,\vb*{a}_{k-1}$の線形結合として表される、すなわち線形従属であるとする

このとき、$\vb*{a}_k$は、すでに得られた正規直交系$\vb*{u}_1,\ldots,\vb*{u}_{k-1}$の線形結合として表すことができる

\begin{equation*}
  \vb*{a}_k = \sum_{i = 1}^{k-1}(\vb*{a}_k, \vb*{u}_i) \vb*{u}_i
\end{equation*}

つまり、$\vb*{a}_k$は、すでに得られた正規直交系$\vb*{u}_1,\ldots,\vb*{u}_{k-1}$の線形結合に完全に含まれている

ここで、射影をすべて引くと、次のように残りが$\vb*{0}$になる

\begin{equation*}
  \vb*{u}_k = \vb*{a}_k - \sum_{i=1}^{k-1}(\vb*{a}_k, \vb*{u}_i) \vb*{u}_i = \vb*{0}
\end{equation*}

\br

このように、グラム・シュミットの直交化法における射影を引く操作は、すでにある正規直交基底に重なっている成分(従属部分)を消し去ってしまう

この性質により、グラム・シュミット法は線形従属な場合でも破綻せずに使える

\br

しかし、結果として新しい成分がゼロになる(つまり新しい情報がない)ため、得られる直交系(正規直交基底)は完全な基底にはならない

\end{document}
