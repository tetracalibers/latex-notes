\documentclass[../../../topic_linear-algebra]{subfiles}

\begin{document}

\sectionline
\section{単射と全射}
\marginnote{\refbookL p118〜119}

ここまで、$\vb*{y} = A\vb*{x}$という関係について、次の2つの観点で議論してきた

\begin{enumerate}[label=\romanlabel]
  \item 同じ結果$\vb*{y}$が出るような原因$\vb*{x}$は唯一か
  \item どんな結果$\vb*{y}$にも、それが出るような原因$\vb*{x}$が存在するか
\end{enumerate}

$\vb*{y} = A\vb*{x}$を方程式と捉えると、(\romannum{i})は解の一意性、(\romannum{ii})は解の存在に対応する

\subsection{単射}

(\romannum{i})は、次のようにも言い換えられる

\begin{enumerate}[label=\romanlabel]
  \item 異なる原因$\vb*{x},\vb*{x}'$が、$A$で同じ結果に写ることがないか
\end{enumerate}

(\romannum{i})の条件が成り立つとき、「線形写像$\vb*{y} = A\vb*{x}$は\keyword{単射}である」という

\subsection{全射}

(\romannum{ii})は、次のようにも言い換えられる

\begin{enumerate}[label=\romanlabel,start=2]
  \item 元の空間全体(\keyword{定義域})を$A$で写した領域$\Im A$が、行き先の空間全体(\keyword{値域})に一致するか
\end{enumerate}

(\romannum{ii})の条件が成り立つとき、「線形写像$\vb*{y} = A\vb*{x}$は\keyword{全射}である」という

\subsection{全単射}

(\romannum{i})と(\romannum{ii})の両方が成り立つときは、「線形写像$\vb*{y} = A\vb*{x}$は\keyword{全単射}である」という

\sectionline
\section{零ベクトルと単射性}\label{sec:zero-vector-and-injectivity}
\marginnote{\refbookA p65〜66 \\ \refbookC p71〜73}

零写像と射影を除けば、$f$によってベクトルが「つぶれない」という性質は、次のように表せる
\begin{equation*}
  \vb*{v} \neq 0 \Longrightarrow f(\vb*{v}) \neq \vb*{o}
\end{equation*}

\todo{\refbookA p55 例2.1.15}

この条件は、実は線形写像が単射であることを意味している

対偶をとって、次のように表現できる

\begin{theorem}{零ベクトルへの写像による単射性の判定}\label{thm:injective-zero-test}
  線形写像$f$が単射であることと次は同値である
  \begin{equation*}
    f(\vb*{v}) = \vb*{o} \Longrightarrow \vb*{v} = \vb*{o}
  \end{equation*}
\end{theorem}

\begin{proof}
  \begin{enumerate}[label=\romanlabel]
    \item $f$が単射
    \item $f(\vb*{v}) = \vb*{o} \Longrightarrow \vb*{v} = \vb*{o}$
  \end{enumerate}

  \begin{subpattern}{(\romannum{i}) $\Longrightarrow$ (\romannum{ii})}
    零ベクトルの像は零ベクトルであることから、$f(\vb*{v}) = \vb*{o}$は、
    \begin{equation*}
      f(\vb*{v}) = f(\vb*{o})
    \end{equation*}
    と書き換えられる

    $f$の単射性により、この式から、
    \begin{equation*}
      \vb*{v} = \vb*{o}
    \end{equation*}
    がしたがう $\qed$
  \end{subpattern}

  \begin{subpattern}{(\romannum{ii}) $\Longrightarrow$ (\romannum{i})}
    $f(\vb*{v}_1) = f(\vb*{v}_2)$を満たす$\vb*{v}_1, \vb*{v}_2 \in \mathbb{R}^n$を考える

    このとき、$f$の線形性から、
    \begin{equation*}
      f(\vb*{v}_1 - \vb*{v}_2) = f(\vb*{v}_1) - f(\vb*{v}_2)
    \end{equation*}
    となる

    仮定(\romannum{ii})より、
    \begin{equation*}
      f(\vb*{v}_1 - \vb*{v}_2) = \vb*{o} \Longrightarrow \vb*{v}_1 - \vb*{v}_2 = \vb*{o}
    \end{equation*}
    がいえるので、$\vb*{v}_1 = \vb*{v}_2$が成り立つ

    $f(\vb*{v}_1) = f(\vb*{v}_2)$から$\vb*{v}_1 = \vb*{v}_2$が導かれたことで、$f$は単射であることが示された $\qed$
  \end{subpattern}
\end{proof}

\sectionline
\section{核・像と単射・全射}
\marginnote{\refbookL p119}

先ほどの定理で、線形写像$f$によって「潰れない」という条件が、単射性と同値であることが示された

つまり、線形写像$f$の核$\Ker f$が、$f$の単射性と関係しそうである

\br

また、線形写像$f$の像$\Im f$が値域と一致するかどうかが、$f$の全射性と関係する

\br

\begin{center}
  \begin{tikzpicture}[scale=0.7]
    % for main regions
    \def\OuterWidth{3.5}
    \def\OuterHeight{6}
    \def\OuterDistance{3}

    \def\InnerOffsetX{0.75}
    \def\InnerOffsetTop{1.2}
    \def\InnerOffsetBottom{0.75}

    % Define main regions
    \coordinate (V-nw) at (0,\OuterHeight);
    \coordinate (V-se) at (\OuterWidth,0);
    \coordinate (V-ne) at (\OuterWidth,\OuterHeight);
    \coordinate (W-nw) at ($(\OuterWidth + \OuterDistance,\OuterHeight)$);
    \coordinate (W-se) at ($(\OuterWidth*2 + \OuterDistance,0)$);
    \coordinate (W-sw) at (\OuterWidth + \OuterDistance,0);

    % Draw outer boxes for V and W
    \filldraw[thick, lightslategray!10, draw=lightslategray] (V-nw) rectangle (V-se);
    \filldraw[thick, lightslategray!10, draw=lightslategray] (W-nw) rectangle (W-se);

    % Define inner shaded boxes
    \coordinate (Ker-nw) at ($(V-nw) + (\InnerOffsetX,-\InnerOffsetTop)$);
    \coordinate (Ker-se) at ($(V-se) + (-\InnerOffsetX,\InnerOffsetBottom)$);
    \coordinate (Ker-ne) at ($(V-ne) + (-\InnerOffsetX,-\InnerOffsetTop)$);
    \coordinate (Im-nw) at ($(W-nw) + (\InnerOffsetX,-\InnerOffsetTop)$);
    \coordinate (Im-se) at ($(W-se) + (-\InnerOffsetX,\InnerOffsetBottom)$);
    \coordinate (Im-sw) at ($(W-sw) + (\InnerOffsetX,\InnerOffsetBottom)$);

    % Shaded regions: Ker f and Im f
    \filldraw[SkyBlue!40,draw=Cerulean] (Ker-nw) rectangle (Ker-se);
    \filldraw[carnationpink!40,draw=Rhodamine] (Im-nw) rectangle (Im-se);

    % Labels
    \node at ($(Ker-nw)!0.5!(Ker-nw -| Ker-se)$) {\large$\Ker f$};
    \node at ($(Im-nw)!0.5!(Im-nw -| Im-se)$) {\large$\Im f$};
    \node (V) at ($(V-nw)!0.5!(V-nw -| V-se)+(0,0.5)$) {\Large$V$};
    \node (W) at ($(W-nw)!0.5!(W-nw -| W-se)+(0,0.5)$) {\Large$W$};

    % Define identity element points
    \coordinate (x) at ([yshift=1ex]$(Ker-nw)!0.5!(Ker-se)$);
    \coordinate (fx) at ([yshift=1ex]$(Im-nw)!0.5!(Im-se)$);

    % Draw points
    \fill (x) circle (3pt);
    \fill[Cerulean] (fx) circle (3pt);
    \node[below] at (x) {$\vb*{x}$};
    \node[below, Cerulean] at (fx) {$\vb*{o}$};

    % Arrow for function f
    \draw[->, thick] ($(V) + (0.5*\OuterWidth+0.5, 0)$) -- ($(W) + (-0.5*\OuterWidth-0.5,0)$) node[midway, above] {\large$f$};

    % Dotted arrows connecting box corners and points
    \draw[densely dashed, thick, Rhodamine] (V-ne) -- (Im-nw);
    \draw[densely dashed, thick, Rhodamine] (V-se) -- (Im-sw);
    \draw[densely dashed, thick, Cerulean] (Ker-ne) -- (fx);
    \draw[densely dashed, thick, Cerulean] (Ker-se) -- (fx);
  \end{tikzpicture}
\end{center}

\subsection{単射となるときの核}\label{sec:injectivity-condition-kernel}

線形写像$f$が単射であるとは、「潰れない」ということなので、次のような状況である

\begin{center}
  \begin{tikzpicture}[scale=0.7]
    % for main regions
    \def\OuterWidth{3.5}
    \def\OuterHeight{6}
    \def\OuterDistance{3}

    \def\InnerOffsetX{0.75}
    \def\InnerOffsetTop{1.2}
    \def\InnerOffsetBottom{0.75}

    % Define main regions
    \coordinate (V-nw) at (0,\OuterHeight);
    \coordinate (V-se) at (\OuterWidth,0);
    \coordinate (V-ne) at (\OuterWidth,\OuterHeight);
    \coordinate (W-nw) at ($(\OuterWidth + \OuterDistance,\OuterHeight)$);
    \coordinate (W-se) at ($(\OuterWidth*2 + \OuterDistance,0)$);
    \coordinate (W-sw) at (\OuterWidth + \OuterDistance,0);

    % Draw outer boxes for V and W
    \filldraw[thick, lightslategray!10, draw=lightslategray] (V-nw) rectangle (V-se);
    \filldraw[thick, lightslategray!10, draw=lightslategray] (W-nw) rectangle (W-se);

    % Define inner shaded boxes
    \coordinate (Ker-nw) at ($(V-nw) + (\InnerOffsetX,-\InnerOffsetTop)$);
    \coordinate (Ker-se) at ($(V-se) + (-\InnerOffsetX,\InnerOffsetBottom)$);
    \coordinate (Ker-ne) at ($(V-ne) + (-\InnerOffsetX,-\InnerOffsetTop)$);
    \coordinate (Im-nw) at ($(W-nw) + (\InnerOffsetX,-\InnerOffsetTop)$);
    \coordinate (Im-se) at ($(W-se) + (-\InnerOffsetX,\InnerOffsetBottom)$);
    \coordinate (Im-sw) at ($(W-sw) + (\InnerOffsetX,\InnerOffsetBottom)$);

    % Shaded regions: Ker f and Im f
    \filldraw[carnationpink!40,draw=Rhodamine] (Im-nw) rectangle (Im-se);

    % Labels
    \node at ($(Im-nw)!0.5!(Im-nw -| Im-se)$) {\large$\Im f$};
    \node (V) at ($(V-nw)!0.5!(V-nw -| V-se)+(0,0.5)$) {\Large$V$};
    \node (W) at ($(W-nw)!0.5!(W-nw -| W-se)+(0,0.5)$) {\Large$W$};

    % Define identity element points
    \coordinate (x) at ([yshift=1ex]$(Ker-nw)!0.5!(Ker-se)$);
    \coordinate (fx) at ([yshift=1ex]$(Im-nw)!0.5!(Im-se)$);

    % Draw points
    \fill[Cerulean] (x) circle (3pt);
    \fill[Cerulean] (fx) circle (3pt);
    \node[above] at (x) {\large$\Ker f$};
    \node[below, Cerulean] at (fx) {$\vb*{o}$};

    % Arrow for function f
    \draw[->, thick] ($(V) + (0.5*\OuterWidth+0.5, 0)$) -- ($(W) + (-0.5*\OuterWidth-0.5,0)$) node[midway, above] {\large$f$};

    % Dotted arrows connecting box corners and points
    \draw[densely dashed, thick, Rhodamine] (V-ne) -- (Im-nw);
    \draw[densely dashed, thick, Rhodamine] (V-se) -- (Im-sw);
    \draw[densely dashed, thick, Cerulean] (x) -- (fx);
  \end{tikzpicture}
\end{center}

\br

つまり、$\Ker f$が零ベクトル$\vb*{o}$のみを含む状態であればよい

\begin{theorem}{線形写像の単射性と核の関係}\label{thm:injective-iff-trivial-kernel}
  $f$を線形写像とするとき、
  \begin{equation*}
    f\text{が単射} \Longleftrightarrow \Ker f = \{ \vb*{o} \}
  \end{equation*}
\end{theorem}

\begin{proof}
  $\Ker f$の定義は
  \begin{equation*}
    \Ker f = \{ \vb*{v} \in V \mid f(\vb*{v}) = \vb*{o} \}
  \end{equation*}

  これを踏まえて、次の2つが同値であることを示す
  \begin{enumerate}[label=\romanlabel]
    \item $f(\vb*{v}) = \vb*{o} \Longrightarrow \vb*{v} = \vb*{o}$
    \item $\Ker f = \{ \vb*{o} \}$
  \end{enumerate}

  \begin{subpattern}{(\romannum{i}) $\Longrightarrow$ (\romannum{ii})}
    このとき、$f(\vb*{v}) = \vb*{o}$が$\vb*{v} = \vb*{o}$を意味するので、$\Ker f$の元は零ベクトルのみになる

    よって、$\Ker f = \{ \vb*{o} \}$が成り立つ $\qed$
  \end{subpattern}

  \begin{subpattern}{(\romannum{ii}) $\Longrightarrow$ (\romannum{i})}
    $\Ker f = \{ \vb*{o} \}$であれば、$\Ker f$の元は零ベクトルのみである

    よって、$f(\vb*{v}) = \vb*{o}$が成り立つとき、$\vb*{v} = \vb*{o}$が成り立つことになる

    すなわち、$f(\vb*{v}) = \vb*{o} \Longrightarrow \vb*{v} = \vb*{o}$が成り立つ $\qed$
  \end{subpattern}
\end{proof}

\subsection{全射となるときの像}\label{sec:image-and-surjectivity}

線形写像$f$が全射であるとは、$\Im f$が行き先の空間全体を埋め尽くす状態である

このような状態であれば、たしかに$f(\vb*{x})$が$\Im f$からはみ出してしまうことはない

\begin{center}
  \begin{tikzpicture}[scale=0.7]
    % for main regions
    \def\OuterWidth{3.5}
    \def\OuterHeight{6}
    \def\OuterDistance{3}

    \def\InnerOffsetX{0.75}
    \def\InnerOffsetTop{1.2}
    \def\InnerOffsetBottom{0.75}

    % Define main regions
    \coordinate (V-nw) at (0,\OuterHeight);
    \coordinate (V-se) at (\OuterWidth,0);
    \coordinate (V-ne) at (\OuterWidth,\OuterHeight);
    \coordinate (W-nw) at ($(\OuterWidth + \OuterDistance,\OuterHeight)$);
    \coordinate (W-se) at ($(\OuterWidth*2 + \OuterDistance,0)$);
    \coordinate (W-sw) at (\OuterWidth + \OuterDistance,0);

    % Draw outer boxes for V and W
    \filldraw[thick, lightslategray!10, draw=lightslategray] (V-nw) rectangle (V-se);

    % Define inner shaded boxes
    \coordinate (Ker-nw) at ($(V-nw) + (\InnerOffsetX,-\InnerOffsetTop)$);
    \coordinate (Ker-se) at ($(V-se) + (-\InnerOffsetX,\InnerOffsetBottom)$);
    \coordinate (Ker-ne) at ($(V-ne) + (-\InnerOffsetX,-\InnerOffsetTop)$);
    \coordinate (Im-nw) at ($(W-nw) + (\InnerOffsetX,-\InnerOffsetTop)$);
    \coordinate (Im-se) at ($(W-se) + (-\InnerOffsetX,\InnerOffsetBottom)$);
    \coordinate (Im-sw) at ($(W-sw) + (\InnerOffsetX,\InnerOffsetBottom)$);

    % Shaded regions: Ker f and Im f
    \filldraw[SkyBlue!40,draw=Cerulean] (Ker-nw) rectangle (Ker-se);
    \filldraw[carnationpink!40,draw=Rhodamine] (W-nw) rectangle (W-se);

    % Labels
    \node at ($(Ker-nw)!0.5!(Ker-nw -| Ker-se)$) {\large$\Ker f$};
    \node (V) at ($(V-nw)!0.5!(V-nw -| V-se)+(0,0.5)$) {\Large$V$};
    \node (W) at ($(W-nw)!0.5!(W-nw -| W-se)+(0,0.5)$) {\large$\Im f$};

    % Define identity element points
    \coordinate (x) at ([yshift=1ex]$(Ker-nw)!0.5!(Ker-se)$);
    \coordinate (fx) at ([yshift=1ex]$(Im-nw)!0.5!(Im-se)$);

    % Draw points
    \fill (x) circle (3pt);
    \fill[Cerulean] (fx) circle (3pt);
    \node[below] at (x) {$\vb*{x}$};
    \node[below, Cerulean] at (fx) {$\vb*{o}$};

    % Arrow for function f
    \draw[->, thick] ($(V) + (0.5*\OuterWidth+0.5, 0)$) -- ($(W) + (-0.5*\OuterWidth-0.5,0)$) node[midway, above] {\large$f$};

    % Dotted arrows connecting box corners and points
    \draw[densely dashed, thick, Rhodamine] (V-ne) -- (W-nw);
    \draw[densely dashed, thick, Rhodamine] (V-se) -- (W-sw);
    \draw[densely dashed, thick, Cerulean] (Ker-ne) -- (fx);
    \draw[densely dashed, thick, Cerulean] (Ker-se) -- (fx);
  \end{tikzpicture}
\end{center}

\br

この状況を式で表すと、線形写像$f\colon V \to W$が全射であるとは、
\begin{equation*}
  \Im f = W
\end{equation*}
という条件と同値である

\subsection{単射・全射との離れ具合}

$\Ker f$が零ベクトルの集合に一致するなら$f$は単射であり、$\Im f$が写り先全体に一致するなら$f$は全射である

\br

このことから、$\Ker f$と$\Im f$は、それぞれ単射・全射と「どれくらいかけ離れているか」を測る尺度とも捉えられる

\br

\begin{center}
  \begin{tikzpicture}[scale=1.4]
    \coordinate (Vtop) at (0,4);
    \coordinate (Vbottom) at (0,0);
    \coordinate (Wtop) at (3,4);
    \coordinate (Wbottom) at (3,0);
    \coordinate (Ker-top) at (0,1.5);
    \coordinate (Im-top) at (3,2.5);

    % Nodes
    \node[above] at (Vtop) {\large $V$};
    \node[above] at (Wtop) {\large $W$};

    % Dotted arrows
    \draw[densely dashed, Rhodamine] (Vtop) -- (Im-top);
    \draw[densely dashed, Rhodamine] (Ker-top) -- (Wbottom);
    \draw[densely dashed, Cerulean] (Vbottom) -- (Wbottom);

    \draw[line width=2pt, SkyBlue] (Ker-top) -- (Vbottom);
    \draw[line width=2pt, lightslategray] (Wtop) -- (Im-top);

    \draw[line width=4pt, line cap=round, carnationpink] (Vtop) -- (Ker-top);
    \draw[line width=4pt, line cap=round, carnationpink] (Im-top) -- (Wbottom);

    % points
    \fill[SkyBlue] (Wbottom) circle (2pt);

    % Ker f = 単射とのずれ
    \draw[Cerulean, <->, Straight Barb-Straight Barb, very thick] ([xshift=-1.5ex]Ker-top) -- ([xshift=-1.5ex]Vbottom) node[midway,left, align=center] {$\Ker f$\\[-1ex] \bfseries\small "単射とのずれ"};
    % Im f
    \draw[Rhodamine, <->, Straight Barb-Straight Barb, very thick, shorten <= 0.25ex] ([xshift=1.5ex]Im-top) -- ([xshift=1.5ex]Wbottom) node[midway,right=0.5ex, align=center] {$\Im f$};
    % 全射とのずれ
    \draw[lightslategray, <->, Straight Barb-Straight Barb, very thick] ([xshift=1.5ex]Wtop) -- ([xshift=1.5ex]Im-top) node[midway,right, align=center] {\bfseries\small "全射とのずれ"};
  \end{tikzpicture}
\end{center}

\end{document}
