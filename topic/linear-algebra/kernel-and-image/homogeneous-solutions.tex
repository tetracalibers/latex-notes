\documentclass[../../../topic_linear-algebra]{subfiles}

\begin{document}

\sectionline
\section{核と斉次系方程式の解空間}

\note{「一般解のパラメータ表示」の後に移動予定}
\br

線形写像$f\colon \mathbb{R}^n \to \mathbb{R}^m$の表現行列を$A$とするとき、
\begin{equation*}
  \Ker f = \{ \vb*{v} \in \mathbb{R}^n \mid A\vb*{v} = \vb*{o} \}
\end{equation*}
と定めると、$f(\vb*{v}) = A\vb*{v}$という関係から、$\Ker f$と$\Ker A$は同じものを指す

\br

これは、斉次形の連立線形方程式$A\vb*{x} = \vb*{o}$の\keyword{解空間}そのものである

$\Ker A$の元は、$A\vb*{x} = \vb*{o}$の基本解を使ってパラメータ表示できる

\end{document}
