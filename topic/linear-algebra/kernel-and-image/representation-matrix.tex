\documentclass[../../../topic_linear-algebra]{subfiles}

\begin{document}

\sectionline
\section{線形写像の単射性と表現行列}
\marginnote{\refbookA p67〜68}

\note{「解の一意性」の後に移動予定}
\br

線形写像$f$が単射であることを、表現行列$A$の性質として述べると、次のような言い換えができる

\begin{theorem}{線形写像の単射性と表現行列}
  線形写像$f\colon \mathbb{R}^n \to \mathbb{R}^m$の表現行列を$A$とするとき、次はすべて同値である
  \begin{enumerate}[label=\romanlabel]
    \item $f$は単射
    \item $A\vb*{x} = \vb*{o}$は自明な解しか持たない
    \item $\rank(A) = n$
  \end{enumerate}
\end{theorem}

\begin{proof}
  \begin{subpattern}{(\romannum{i}) $\Longleftrightarrow$ (\romannum{ii})}
    線形写像$f$は、表現行列$A$を用いて次のように表せる
    \begin{equation*}
      f(\vb*{x}) = A\vb*{x}
    \end{equation*}

    $f$が単射であることの言い換えは、
    \begin{equation*}
      f(\vb*{x}) = \vb*{o} \Longrightarrow \vb*{x} = \vb*{o}
    \end{equation*}
    であり、$A\vb*{x} = \vb*{o}$が自明解しか持たないことは、
    \begin{equation*}
      A\vb*{x} = \vb*{o} \Longrightarrow \vb*{x} = \vb*{o}
    \end{equation*}
    が成り立つということである

    $f(\vb*{x}) = A\vb*{x}$であるから、これらの2つの条件は同値である $\qed$
  \end{subpattern}

  \begin{subpattern}{(\romannum{ii}) $\Longleftrightarrow$ (\romannum{iii})}
    斉次形の方程式$A\vb*{x} = \vb*{o}$に自明解しか存在しないことと
    \begin{equation*}
      \rank(A) = n
    \end{equation*}
    と同値であることは\hyperref[thm:homogeneous-trivial-iff-full-col-rank]{以前}証明済み $\qed$
  \end{subpattern}
\end{proof}

\sectionline
\section{線形写像の全射性と表現行列}
\marginnote{\refbookA p67〜68}

\note{「解の一意性」の後に移動}
\br

単射性と対比して、全射性についても表現行列の言葉で整理する

\begin{theorem}{線形写像の全射性と表現行列}
  線形写像$f\colon \mathbb{R}^n \to \mathbb{R}^m$の表現行列を$A$とするとき、次はすべて同値である
  \begin{enumerate}[label=\romanlabel]
    \item $f$は全射
    \item 任意の$\vb*{b} \in \mathbb{R}^m$に対して、$A\vb*{x} = \vb*{b}$には解が存在する
    \item $\rank(A) = m$
  \end{enumerate}
\end{theorem}

\begin{proof}
  \begin{subpattern}{(\romannum{i}) $\Longleftrightarrow$ (\romannum{ii})}
    線形写像$f$は、表現行列$A$を用いて次のように表せる
    \begin{equation*}
      f(\vb*{x}) = A\vb*{x}
    \end{equation*}

    $f$が全射であることの言い換えは、
    \begin{equation*}
      \forall \vb*{b} \in \mathbb{R}^m, \exists \vb*{x} \in \mathbb{R}^n, f(\vb*{x}) = \vb*{b}
    \end{equation*}
    であり、これは
    \begin{equation*}
      \forall \vb*{b} \in \mathbb{R}^m, A\vb*{x} = \vb*{b} \text{に解が存在する}
    \end{equation*}
    と同値である

    よって、これらの2つの条件は同値である $\qed$

  \end{subpattern}

  \begin{subpattern}{(\romannum{ii}) $\Longleftrightarrow$ (\romannum{iii})}
    $\rank(A) = m$が、次の条件
    \begin{equation*}
      ^{\forall}\vb*{b} \in \mathbb{R}^m, A\vb*{x} = \vb*{b} \text{の解が存在する}
    \end{equation*}
    ことと同値であることは、\hyperref[thm:full-row-rank-solvable]{以前}証明済み $\qed$
  \end{subpattern}
\end{proof}

\end{document}
