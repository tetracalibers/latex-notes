\documentclass[../../../topic_linear-algebra]{subfiles}

\begin{document}

\sectionline
\section{手がかりが足りない場合}
\marginnote{\refbookL p112〜113}

手がかりとなる情報を$\vb*{y}$とし、知りたい情報を$\vb*{x}$とする

\br

まずは、$\vb*{y}$の方が$\vb*{x}$より次元が小さい、すなわち$m < n$の場合を考える

このとき、表現行列$A$は横長の行列となる
\begin{equation*}
  \begin{pmatrix}
    y_1    \\
    \vdots \\
    y_m
  \end{pmatrix} = \begin{pmatrix}
    a_{11} & \cdots & \cdots & a_{1n} \\
    \vdots & \ddots & \vdots & \vdots \\
    a_{m1} & \cdots & \cdots & a_{mn}
  \end{pmatrix} \begin{pmatrix}
    x_1    \\
    \vdots \\
    \vdots \\
    x_n
  \end{pmatrix}
\end{equation*}

$m < n$の場合は、「知りたい量が$n$個もあるのに、手がかりはたった$m$個しかない」という状況になっている

\br

見方を変えると、$A$を作用させたことによって「情報の一部が欠落してしまった」ともいえる

\subsection{$m < n$の場合の線形写像の写し方}

$m < n$のとき、$A$は、元より次元の低い空間に写す線形写像を表す

そのため、$\vb*{x}$はこの線形写像によって「潰される」ことになる

\br

「潰される」とはどのようなことかというと、空間を張る$\vb*{x}$それぞれの居場所を、写す先では全員分用意することができないので、
\begin{shaded}
  複数の$\vb*{x}$を同じ$\vb*{y}$に写すしかない
\end{shaded}
ということである

\br

複数の$\vb*{x}$が同じ$\vb*{y}$に写ってしまうと、結果$\vb*{y}$から元の$\vb*{x}$を特定することはできなくなる(つまり、情報が失われている)

\sectionline
\section{線形写像の核}
\marginnote{\refbookL p112〜114 \\ \refbookC p79〜84}

次の図は、$1\times 3$行列$A$による線形写像を表している

同じ平面上の点がすべて同じ点に写されることで、平面の集まりである立体(3次元)が、点の集まりである直線(1次元)へと「潰されている」ことがわかる

\begin{center}
  \tdplotsetmaincoords{70}{120}
  \begin{tikzpicture}

    % グローバル座標定義
    \coordinate (A); % x
    \coordinate (B); % x'
    \coordinate (C); % T(v)

    %----------------------------
    % 左側: R^3 における T の前像
    %----------------------------
    \begin{scope}[tdplot_main_coords,scale=1.2]
      \foreach \z/\col [count=\i] in {-1/lightslategray!40, -0.5/SkyBlue!60, 0/lightslategray!40, 0.5/carnationpink!60} {
          \filldraw[\col,opacity=0.6]
          (-1.2,-1.2,\z) -- (1.2,-1.2,\z) -- (1.2,1.2,\z) -- (-1.2,1.2,\z) -- cycle;

          \ifnum\i=2
            \coordinate (Os) at (-1.2,1.2,\z);
            \coordinate (Oe) at (1.2,1.2,\z);

            % 平面上にラベル
            \node[SkyBlue,left] at (1.0,-1.25,\z) {$\Ker{A}$};
          \fi

          \ifnum\i=4
            \coordinate (O) at (0,0,\z);
            \coordinate (A) at (-0.7,0.8,\z);
            \coordinate (B) at (0.5,0.8,\z);

            \draw[->, -Straight Barb, carnationpink, thick, shorten >=0.5ex] (O) to (A);
            \draw[->, -Straight Barb, carnationpink, thick, shorten >=0.5ex] (O) to (B);

            \fill[lightgray] (O) circle (1.5pt) node[left] {$O$};
            \fill[Rhodamine] (A) circle (2pt) node[below] {$\vb*{x}$};
            \fill[Rhodamine] (B) circle (2pt) node[below] {$\vb*{x}'$};
          \fi
        }
    \end{scope}

    % 傾きを大きくする(例: y = 1.2x)
    \def\slope{1.2}

    %----------------------------
    % 右側: R^2 への写像後(傾いた直線 + 点)
    %----------------------------
    \begin{scope}[xshift=5cm]

      % 点を通る直線(範囲広め)
      \draw[very thick, lightslategray!40] (-1.5,-1.5*\slope) -- (1.5,1.5*\slope);

      % y
      \coordinate (C) at (0.6,{0.6*\slope});
      \fill[Rhodamine] (C) circle (3pt) node[below right] {$\vb*{y}$};

      % o
      \coordinate (Oy) at (-0.6,-0.6*\slope);
      \fill[Cerulean] (Oy) circle (3pt) node[below right] {$\vb*{o}$};

      % 他の点を並べる(y = slope * x)
      \foreach \x in {-1.2, 0, 1.2} {
          \pgfmathsetmacro{\y}{\slope*\x}
          \fill[lightslategray!40] (\x,\y) circle (3pt);
        }
    \end{scope}

    %----------------------------
    % 矢印: x, x' → T(v)
    %----------------------------
    \draw[vector, thick, shorten <= 3pt, shorten >=0.75ex, bend left=10,densely dashed, Rhodamine!80] (A) to (C);
    \draw[vector, thick, shorten <= 3pt, shorten >=1ex, bend right=10,densely dashed, Rhodamine!80] (B) to (C);

    \draw[vector, thick, shorten >=0.5ex, densely dashed,SkyBlue] (Os) to (Oy);
    \draw[vector, thick, shorten >=0.75ex, densely dashed, SkyBlue] (Oe) to (Oy);

    \draw[->, -Straight Barb, thick, shorten >=1ex, carnationpink] ([xshift=1ex]Oy) to ([xshift=1ex]C);

  \end{tikzpicture}
\end{center}

このとき、$A\vb*{x} = \vb*{o}$に写ってくるような$\vb*{x}$の集合を、$A$の\keyword{核}あるいは\keyword{カーネル}といい、$\Ker A$と表す

\subsection{$\Ker A$の次元}

上の図では、零ベクトル$\vb*{o}$(写り先の1次元空間の原点)に潰されている青い平面が$\Ker A$に相当する

平面なので、この$\Ker A$は2次元である

\br

もしも$m < n$でない場合、つまり潰れない写像の場合は、$A\vb*{x} = \vb*{o}$に写る$\vb*{x}$は零ベクトル$\vb*{o}$だけなので、$\Ker A$は0次元となる

\subsection{$\Ker A$に平行な方向の成分}

\todo{\refbookL p114}

\subsection{$\Ker f$の定義}

$A$が線形写像$f$の表現行列であるとすると、$\Ker f$を次のように定義できる

\begin{definition*}{線形写像の核}
  線形写像$f\colon V \to W$に対して、$f$による$\{\vb*{o}\}$の逆像$f^{-1}(\{\vb*{o}\})$を、線形写像$f$の\keyword{核}や\keyword{核空間}、あるいは\keyword{カーネル}といい、$\Ker(f)$と表記する
  \begin{equation*}
    \Ker(f) = f^{-1}(\{\vb*{o}\}) = \{ \vb*{v} \in V \mid f(\vb*{v}) = \vb*{o} \} \subset V
  \end{equation*}
\end{definition*}

\end{document}
