\documentclass[../../../topic_linear-algebra]{subfiles}

\begin{document}

\sectionline
\section{手がかりが多すぎる場合}\label{sec:linear-map-m-greater-n}
\marginnote{\refbookL p114〜115}

今度は、$\vb*{y}$の方が$\vb*{x}$より次元が大きい、すなわち$m > n$の場合を考える

このとき、表現行列$A$は縦長の行列となる
\begin{equation*}
  \begin{pmatrix}
    y_1    \\
    \vdots \\
    \vdots \\
    y_m
  \end{pmatrix} = \begin{pmatrix}
    a_{11} & \cdots & a_{1n} \\
    \vdots & \ddots & \vdots \\
    \vdots & \ddots & \vdots \\
    a_{m1} & \cdots & a_{mn}
  \end{pmatrix} \begin{pmatrix}
    x_1    \\
    \vdots \\
    x_n
  \end{pmatrix}
\end{equation*}

$m > n$の場合は、「知りたい量はたった$n$個しかないのに、手がかりが$m$個もある」という状況になっている

この場合、手がかりどうしが矛盾することもある

\subsection{$m > n$の場合の線形写像の写し方}

$m > n$のとき、$A$は、元より次元の高い空間に写す線形写像を表す

そのため、写り先の空間すべてをカバーすることはできない

\br

はみ出した$\vb*{y}$については、
\begin{shaded}
  そこに写ってきてくれる$\vb*{x}$が存在しない
\end{shaded}
ことになる

\br

現実の応用では、ノイズがのることで、はみ出した$\vb*{y}$が観測されることがある

そうなると、「手がかり$y_1,\ldots,y_m$すべてに符号する$\vb*{x}$は存在しない」ということになってしまう

\sectionline
\section{線形写像の像}
\marginnote{\refbookL p115 \\ \refbookC p79〜84}

与えられた$A$に対して、$\vb*{x}$をいろいろ動かしたときに$A$で写り得る$\vb*{y} = A\vb*{x}$の集合を$A$の\keyword{像}といい、$\Im A$で表す

\br

別の言い方をすると、$\Im A$は、元の空間全体を$A$で写した領域である

$\Im A$上にない$\vb*{y}$については、$\vb*{y} = A\vb*{x}$となるような$\vb*{x}$は存在しない

\subsection{$\Im f$の定義}

$A$が線形写像$f$の表現行列であるとすると、$\Im f$を次のように定義できる

\begin{definition*}{線形写像の像}
  線形写像$f\colon V \to W$に対して、$f$による$V$の像$f(V)$を、線形写像$f$の\keyword{像}や\keyword{像空間}といい、$\Im(f)$と表記する
  \begin{equation*}
    \Im(f) = f(V) = \{ f(\vb*{v}) \in W \mid \vb*{v} \in V \} \subset W
  \end{equation*}
\end{definition*}

\end{document}
