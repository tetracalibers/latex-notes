\documentclass[../../../topic_linear-algebra]{subfiles}

\usepackage{xr-hyper}
\externaldocument{../../../.tex_intermediates/topic_linear-algebra}

\begin{document}

\sectionline
\section{双線形写像}
\marginnote{\refbookR p254〜255}

\defref{def:bilinear-form}の一般化として、\keywordJE{双線形写像}{bilinear map}を考える。

\br

双線形写像とは、2つの線形空間の直積から線形空間への写像で、各成分に対して線形となるもののことである。

\begin{definition}{双線形写像}{bilinear-map}
  $V,W,U$を線型空間とする。写像$\Phi \colon V \times W \to U$が次の条件を満たすとき、$\Phi$を\keyword{双線形写像}という。
  \begin{enumerate}[label=\romanlabel]
    \item $\Phi(\vb*{v}_1 + \vb*{v}_2, \vb*{w}) = \Phi(\vb*{v}_1, \vb*{w}) + \Phi(\vb*{v}_2, \vb*{w})$
    \item $\Phi(\vb*{v}, \vb*{w}_1 + \vb*{w}_2) = \Phi(\vb*{v}, \vb*{w}_1) + \Phi(\vb*{v}, \vb*{w}_2)$
    \item $\Phi(c\vb*{v}, \vb*{w}) = \Phi(\vb*{v}, c\vb*{w}) = c\Phi(\vb*{v}, \vb*{w})$
  \end{enumerate}
  ここで、$\vb*{v}, \vb*{v}_1, \vb*{v}_2 \in V$, $\vb*{w}, \vb*{w}_1, \vb*{w}_2 \in W$, $c \in \mathbb{R}$である。
\end{definition}

\subsection{双線形写像は双線形形式の一般化}

$U = \mathbb{R}$とすると、$\Phi$は\defref{def:bilinear-form}となる。

すなわち、双線形写像は双線形形式の一般化である。

\end{document}
