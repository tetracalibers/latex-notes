\documentclass[../../../topic_linear-algebra]{subfiles}

\usepackage{xr-hyper}
\externaldocument{../../../.tex_intermediates/topic_linear-algebra}

\begin{document}

\sectionline
\section{一般の基底に関するベクトルの成分表示}
\marginnote{\refbookA p10 \\ \refbookC p95}

\secref{sec:basis-induced-isomorphism}でも簡単に述べたように、(有限次元)線形空間の元は、基底を使えば数ベクトルで表すことができる。

\br

$V$を線形空間とし、$\mathcal{V} = \{\vb*{v}_1, \ldots, \vb*{v}_n\}$をその基底とすると、任意の$\vb*{v} \in V$は、
\begin{equation*}
  \vb*{v} = \sum_{i=1}^n x_i \vb*{v}_i
\end{equation*}
と一意的に書ける。

\br

このとき、$x_i$を縦に並べた数ベクトルを、基底$\mathcal{V}$に関する$\vb*{v}$の\keyword{座標ベクトル}という。
\begin{equation*}
  \vb*{x} = \begin{pmatrix}
    x_1 \\
    \vdots \\
    x_n
  \end{pmatrix} \in K^n
\end{equation*}

すると$\vb*{v}$は、基底$\mathcal{V}$と、この座標ベクトル$\vb*{x}$の各成分との線形結合で表されるといえる。

\br

数ベクトルの表記に倣って$\vb*{v}$を\keyword{成分表示}する際は、基底$\mathcal{V}$に関する成分(座標)であることを明示するために、次のように右下に$\mathcal{V}$を添えて書き表すことにする。
\begin{equation*}
  \vb*{v} = \begin{pmatrix}
    x_1    \\
    \vdots \\
    x_n
  \end{pmatrix}_{\mathcal{V}} \in V
\end{equation*}

\subsection{座標写像の逆写像}

ある基底$\mathcal{V}$に関する座標ベクトルは、\defref*{def:coordinate-mapping}の\defref*{def:inverse-mapping}を用いて表すことができる。

\br

座標写像$\Phi_{\mathcal{V}}$は、$K^n$の座標$\vb*{x}$から$V$の元$\vb*{v}$を得る線形写像として定義された。
\begin{equation*}
  \Phi_{\mathcal{V}}(\vb*{x}) = \vb*{v}
\end{equation*}

\thmref{thm:coordinate-map-isomorphism}より、座標写像は全単射(可逆)であるので、その逆写像を定義できる。

座標写像の逆写像$\Phi_{\mathcal{V}}^{-1}$は、$\vb*{v} \in V$から、その基底$\mathcal{V}$に関する座標ベクトル$\vb*{x}$を得る線形写像となる。
\begin{equation*}
  \Phi^{-1}_\mathcal{V}(\vb*{v}) = \vb*{x}
\end{equation*}

すなわち、
\begin{equation*}
  \Phi^{-1}_\mathcal{V}(\vb*{v}) = \begin{pmatrix}
    x_1    \\
    \vdots \\
    x_n
  \end{pmatrix}
  \in K^n
\end{equation*}
として、座標ベクトルを表すことができる。

\end{document}
