\documentclass[../../../topic_linear-algebra]{subfiles}

\usepackage{xr-hyper}
\externaldocument{../../../.tex_intermediates/topic_linear-algebra}

\begin{document}

\sectionline
\section{表現行列の構成}\label{sec:construction-of-matrix-rep}
\marginnote{\refbookA p105 \\ \refbookC p96〜97}

\secref{sec:matrix-description-linear-map}で述べたように、数ベクトル空間の間の線形写像を定める行列は、各基本ベクトル$\vb*{e}_j$の$f$による像
\begin{equation*}
  f(\vb*{e}_j) = \vb*{a}_j = \begin{pmatrix}
    a_{1j} \\
    a_{2j} \\
    \vdots \\
    a_{mj}
  \end{pmatrix} \quad (1 \leq j \leq n)
\end{equation*}
を用いて、
\begin{equation*}
  \left( f(\vb*{e}_1), \ldots, f(\vb*{e}_n) \right) = \left( \vb*{a}_1, \ldots, \vb*{a}_n \right) = A
\end{equation*}
のように構成された

\br

この表現行列の構成を、部分空間$V,\,W$の基底をそれぞれ$\mathcal{V} = \{ \vb*{v}_1,\ldots, \vb*{v}_n\}, \, \mathcal{W} = \{\vb*{w}_1, \ldots, \vb*{w}_m\}$として一般化する

\br

このとき、$\vb*{a}_j$は座標写像$\Phi_{\mathcal{W}}$によって、
\begin{equation*}
  \Phi_{\mathcal{W}}(\vb*{a}_j) = \sum_{i=1}^m a_{ij} \vb*{w}_i \quad (1 \leq j \leq n)
\end{equation*}
のように$W$に写される

\br

また、$\vb*{e}_j$は座標写像$\Phi_{\mathcal{V}}$によって、
\begin{equation*}
  \Phi_{\mathcal{V}}(\vb*{e}_j) = \sum_{i=1}^n e_{ij} \vb*{v}_i \quad (1 \leq j \leq n)
\end{equation*}
のように$V$に写されるが、これは$\vb*{v}_j$そのものである

たとえば、$j=1$のときは、
\begin{equation*}
  \Phi_{\mathcal{V}}(\vb*{e}_1) = \sum_{i=1}^n e_{i1} \vb*{v}_i = \vb*{v}_1
\end{equation*}
となる

\br

よって、$\vb*{e}_j \mapsto \vb*{a}_j$という写像は、
\begin{equation*}
  \vb*{v}_j \mapsto \Phi_{\mathcal{W}}(\vb*{a}_j)
\end{equation*}
という$V$から$W$への写像$f$に対応する

(この対応は、可換図式からも明らか)

\br

記号を書き換えると、
\begin{equation*}
  f(\vb*{v}_j) = \Phi_{\mathcal{W}}(\vb*{a}_j) = \sum_{i=1}^m a_{ij} \vb*{w}_i
\end{equation*}
となり、右辺はさらに、
\begin{equation*}
  \sum_{i=1}^m a_{ij} \vb*{w}_i = (\vb*{w}_1, \ldots, \vb*{w}_m) \begin{pmatrix}
    a_{1j} \\
    \vdots \\
    a_{mj}
  \end{pmatrix}
\end{equation*}
と変形できるので、まとめると、
\begin{equation*}
  \left( f(\vb*{v}_1),\ldots, f(\vb*{v}_n) \right) = \left( \vb*{w}_1, \ldots, \vb*{w}_m \right) A
\end{equation*}
と表せる

\end{document}
