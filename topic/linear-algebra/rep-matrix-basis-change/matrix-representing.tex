\documentclass[../../../topic_linear-algebra]{subfiles}

\usepackage{xr-hyper}
\externaldocument{../../../.tex_intermediates/topic_linear-algebra}

\begin{document}

\sectionline
\section{基底に関する座標ベクトル}
\marginnote{\refbookA p10 \\ \refbookC p95}

$V$を線形空間とし、$\mathcal{V} = \{\vb*{v}_1, \vb*{v}_2, \ldots, \vb*{v}_n\}$をその基底とする

$V$の任意のベクトル$\vb*{v}$は、
\begin{equation*}
  \vb*{v} = \sum_{i=1}^n x_i \vb*{v}_i
\end{equation*}
と一意的に書ける

\br

ここで、$\Phi_\mathcal{V}$を座標写像とすると、その定義から、
\begin{equation*}
  \Phi^{-1}_\mathcal{V}(\vb*{v}) = \begin{pmatrix}
    x_1    \\
    x_2    \\
    \vdots \\
    x_n
  \end{pmatrix}
  \in \mathbb{R}^n
\end{equation*}

このベクトルを$\mathcal{V}$に関する$\vb*{v}$の\keyword{座標ベクトル}あるいは\keyword{成分表示}と呼び、
\begin{equation*}
  \vb*{v} = \begin{pmatrix}
    x_1    \\
    x_2    \\
    \vdots \\
    x_n
  \end{pmatrix}_{\mathcal{V}}
\end{equation*}
と書くことにする

\sectionline
\section{一般の基底に関する表現行列}
\marginnote{\refbookA p104〜106 \\ \refbookC p95〜96}

$V,\,W$をそれぞれ次元が$n,\,m$の線形空間とし、$f$を$V$から$W$への線形写像とする

また、$\mathcal{V}, \, \mathcal{W}$をそれぞれ$V,\,W$の基底とする

\br

\thmref{thm:subspace-isomorphic-to-Kn}より、座標写像が線形同型写像であることは、任意の部分空間が数ベクトル空間と同型であることを意味していた

よって、$V$から$W$への線形写像$f$は、数ベクトル空間との線形同型写像(座標写像)$\Phi_\mathcal{V}, \, \Phi_\mathcal{W}$を合成すれば、
\begin{equation*}
  f' = \Phi_\mathcal{W}^{-1} \circ f \circ \Phi_\mathcal{V} : \mathbb{R}^n \to \mathbb{R}^m
\end{equation*}
として、数ベクトル空間の間の写像と考えることができる

\br

この合成を図で整理して、次のように表す

\begin{center}
  \begin{tikzcd}
    V \arrow[r,"f"] & W \\
    \mathbb{R}^n \arrow[u,"{\Phi_{\mathcal{V}}}"] \arrow[r,"f'"] & \mathbb{R}^m \arrow[u,"{\Phi_{\mathcal{W}}}"]
  \end{tikzcd}
\end{center}

このような図を\keyword{図式}という

\br

下辺の矢印は、合成写像
\begin{equation*}
  \Phi^{-1}_{\mathcal{W}} \circ f \circ \Phi_{\mathcal{V}} : \mathbb{R}^n \to \mathbb{R}^m
\end{equation*}
を表していて、この写像は$\mathbb{R}^n$から$\mathbb{R}^m$への線形写像である

\br

左下の$\mathbb{R}^n$から右上の$W$への2通りの合成写像が一致するという意味で、この図式は\keyword{可換}であるという

\sectionline

数ベクトル空間の間の写像は、行列が定める線形写像であることから、この写像$f$は$m \times n$型行列$A$により表現される

\begin{center}
  \begin{tikzcd}
    V \arrow[r,"f"] & W \\
    \mathbb{R}^n \arrow[u,"{\Phi_{\mathcal{V}}}"] \arrow[r,"A"] & \mathbb{R}^m \arrow[u,"{\Phi_{\mathcal{W}}}"]
  \end{tikzcd}
\end{center}

\br

座標ベクトルの記法を用いると、写像$f$は次で与えられる
\begin{equation*}
  f\colon \begin{pmatrix}
    x_1    \\
    x_2    \\
    \vdots \\
    x_n
  \end{pmatrix}_{\mathcal{V}}
  \mapsto \left( A \begin{pmatrix}
      x_1    \\
      x_2    \\
      \vdots \\
      x_n
    \end{pmatrix} \right)_\mathcal{W}
\end{equation*}

\sectionline

このように、座標写像を用いることで、$V$から$W$への線形写像$f$から、$m \times n$型行列が得られる

この行列$A$を、基底$\mathcal{V}, \, \mathcal{W}$に関する$f$の\keyword{表現行列}という

\br

つまり、
\begin{shaded}
  基底$\mathcal{V}, \, \mathcal{W}$を固定して考えるときは、$f$を$A$と同一視できる
\end{shaded}
ということになり、このとき、
\begin{shaded}
  表現行列は線形写像の「成分表示」
\end{shaded}
と解釈できる

\sectionline
\section{表現行列の構成}\label{sec:construction-of-matrix-rep}
\marginnote{\refbookA p105 \\ \refbookC p96〜97}

\secref{sec:matrix-description-linear-map}で述べたように、数ベクトル空間の間の線形写像を定める行列は、各基本ベクトル$\vb*{e}_j$の$f$による像
\begin{equation*}
  f(\vb*{e}_j) = \vb*{a}_j = \begin{pmatrix}
    a_{1j} \\
    a_{2j} \\
    \vdots \\
    a_{mj}
  \end{pmatrix} \quad (1 \leq j \leq n)
\end{equation*}
を用いて、
\begin{equation*}
  \left( f(\vb*{e}_1), \ldots, f(\vb*{e}_n) \right) = \left( \vb*{a}_1, \ldots, \vb*{a}_n \right) = A
\end{equation*}
のように構成された

\br

この表現行列の構成を、部分空間$V,\,W$の基底をそれぞれ$\mathcal{V} = \{ \vb*{v}_1,\ldots, \vb*{v}_n\}, \, \mathcal{W} = \{\vb*{w}_1, \ldots, \vb*{w}_m\}$として一般化する

\br

このとき、$\vb*{a}_j$は座標写像$\Phi_{\mathcal{W}}$によって、
\begin{equation*}
  \Phi_{\mathcal{W}}(\vb*{a}_j) = \sum_{i=1}^m a_{ij} \vb*{w}_i \quad (1 \leq j \leq n)
\end{equation*}
のように$W$に写される

\br

また、$\vb*{e}_j$は座標写像$\Phi_{\mathcal{V}}$によって、
\begin{equation*}
  \Phi_{\mathcal{V}}(\vb*{e}_j) = \sum_{i=1}^n e_{ij} \vb*{v}_i \quad (1 \leq j \leq n)
\end{equation*}
のように$V$に写されるが、これは$\vb*{v}_j$そのものである

たとえば、$j=1$のときは、
\begin{equation*}
  \Phi_{\mathcal{V}}(\vb*{e}_1) = \sum_{i=1}^n e_{i1} \vb*{v}_i = \vb*{v}_1
\end{equation*}
となる

\br

よって、$\vb*{e}_j \mapsto \vb*{a}_j$という写像は、
\begin{equation*}
  \vb*{v}_j \mapsto \Phi_{\mathcal{W}}(\vb*{a}_j)
\end{equation*}
という$V$から$W$への写像$f$に対応する

(この対応は、可換図式からも明らか)

\br

記号を書き換えると、
\begin{equation*}
  f(\vb*{v}_j) = \Phi_{\mathcal{W}}(\vb*{a}_j) = \sum_{i=1}^m a_{ij} \vb*{w}_i
\end{equation*}
となり、右辺はさらに、
\begin{equation*}
  \sum_{i=1}^m a_{ij} \vb*{w}_i = (\vb*{w}_1, \ldots, \vb*{w}_m) \begin{pmatrix}
    a_{1j} \\
    \vdots \\
    a_{mj}
  \end{pmatrix}
\end{equation*}
と変形できるので、まとめると、
\begin{equation*}
  \left( f(\vb*{v}_1),\ldots, f(\vb*{v}_n) \right) = \left( \vb*{w}_1, \ldots, \vb*{w}_m \right) A
\end{equation*}
と表せる

\end{document}
