\documentclass[../../../topic_linear-algebra]{subfiles}

\usepackage{xr-hyper}
\externaldocument{../../../.tex_intermediates/topic_linear-algebra}

\begin{document}

\sectionline
\section{線形変換の表現行列}
\marginnote{\refbookA p106〜107}

$V$を$n$次元の線形空間とし、$f$を$V$の線形変換、すなわち$V$から$V$自身への線形写像とする

$V$の基底$\mathcal{V}$を選ぶとき、次の可換図式によって$n$次正方行列$A$が定められる
\begin{center}
  \begin{tikzcd}
    V \arrow[r,"f"] & V \\
    \mathbb{R}^n \arrow[u,"{\Phi_{\mathcal{V}}}"] \arrow[r,"A\cdot"] & \mathbb{R}^n \arrow[u,"{\Phi_{\mathcal{V}}}"]
  \end{tikzcd}
\end{center}

写像の定義される空間と、写す先の空間が同じなので、どちらに対しても同じ基底を用いることができる

\br

もちろん、考える問題によっては別な基底を用いても構わないが、線形変換に対しては1つの基底を用いるのが自然である

\sectionline
\section{数ベクトル空間の基底変換行列}
\marginnote{\refbookA p108〜109 \\ \refbookF p225}

$V = \mathbb{R}^n$とし、標準基底$\mathcal{E}$によって行列$A$で表現される線形変換を$f$とする

別な基底$\mathcal{V}$によって$f$を表現する行列を$B$とするとき、$B$をどうやって計算すればよいかを考える

\br

$B$を定める原理は、\secref{sec:construction-of-matrix-rep}で議論したように、
\begin{equation*}
  (f(\vb*{v}_1), \ldots, f(\vb*{v}_n)) = (\vb*{v}_1, \ldots, \vb*{v}_n) B
\end{equation*}

ここで、$\vb*{v}_i$や$f(\vb*{v}_i)$は$\mathbb{R}^n$の元なので、$(f(\vb*{v}_1), \ldots, f(\vb*{v}_n))$や$(\vb*{v}_1, \ldots, \vb*{v}_n)$は$n$次の正方行列であるとみなせる

そこで、
\begin{equation*}
  P = (\vb*{v}_1, \ldots, \vb*{v}_n)
\end{equation*}
とおくとき、次に示すように$P$は正則行列である

\begin{theorem*}{基底変換行列の正則性}
  基底の変換行列は正則行列である
\end{theorem*}

\begin{proof}
  $P$の列ベクトルは基底であるため、線形独立である

  \thmref{thm:invertible-iff-col-indep}で示したように、正則行列であることは、列ベクトルが線形独立であることと同値である $\qed$
\end{proof}

また、$B$を決める式
\begin{equation*}
  (f(\vb*{v}_1), \ldots, f(\vb*{v}_n)) = (\vb*{v}_1, \ldots, \vb*{v}_n) B
\end{equation*}
の左辺は、次のように書ける
\begin{align*}
  (f(\vb*{v}_1), \ldots, f(\vb*{v}_n)) & = (A\vb*{v}_1, \ldots, A\vb*{v}_n) \\
                                       & = A (\vb*{v}_1, \ldots, \vb*{v}_n) \\
                                       & = A P
\end{align*}

よって、$B$を決める式は、
\begin{equation*}
  AP = PB
\end{equation*}
となり、$P$は正則である(逆行列が存在する)ので、両辺に左から$P^{-1}$をかけて、
\begin{equation*}
  B = P^{-1} A P
\end{equation*}
が得られる

\br

行列$P$は、標準基底$\mathcal{E}$から基底$\mathcal{V}$への\keyword{基底変換行列}と呼ばれる

\sectionline
\section{線形空間の基底変換行列}
\marginnote{\refbookA p110〜111 \\ \refbookF p215〜219}

$V$を線形空間とし、$V$の基底$\mathcal{V} = \{\vb*{v}_1, \ldots, \vb*{v}_n\}$を別な基底$\mathcal{V}' = \{\vb*{v}'_1, \ldots, \vb*{v}'_n\}$に取り替えることを考える

\br

このとき、$\{\vb*{v}_1, \ldots, \vb*{v}_n\}$が$V$の基底であることから、$V$の元である$\vb*{v}'_1, \ldots, \vb*{v}'_n$は、$\{\vb*{v}_1, \ldots, \vb*{v}_n\}$の線形結合で表される

そこで、
\begin{equation*}
  \vb*{v}'_i = p_{1i} \vb*{v}_1 + p_{2i} \vb*{v}_2 + \cdots + p_{ni} \vb*{v}_n
\end{equation*}
すなわち、
\begin{equation*}
  (\vb*{v}'_1, \ldots, \vb*{v}'_n) = (\vb*{v}_1, \ldots, \vb*{v}_n) (p_{ij})
\end{equation*}
とおく

\br

このとき、写像$f\colon V \to V$を
\begin{equation*}
  \left\{
  \begin{array}{rcl}
    f(\vb*{v}_1) & = & \vb*{v}'_1 \\
    f(\vb*{v}_2) & = & \vb*{v}'_2 \\
    \vdots       &   & \vdots     \\
    f(\vb*{v}_n) & = & \vb*{v}'_n
  \end{array}
  \right.
\end{equation*}
を満たすものとして定義する

これはすなわち、基底$\mathcal{V}$を構成するそれぞれのベクトルを、基底$\mathcal{V}'$を構成するベクトルに順に写す線形変換であり、
\begin{equation*}
  (f(\vb*{v}_1), \ldots, f(\vb*{v}_n) ) = (\vb*{v}'_1, \ldots, \vb*{v}'_n)
\end{equation*}
を満たすものである

\br

すると、行列$P = (p_{ij})$を定める式は、
\begin{equation*}
  (f(\vb*{v}_1), \ldots, f(\vb*{v}_n)) = (\vb*{v}_1, \ldots, \vb*{v}_n) P
\end{equation*}
と書ける

よって、$P$は基底変換$\mathcal{V} \rightharpoonup \mathcal{V}'$を表す線形写像$f$の表現行列である

この意味で、$P$を基底変換$\mathcal{V} \rightharpoonup \mathcal{V}'$の\keyword{基底変換行列}と呼ぶ

\begin{theorem*}{線形空間の基底変換行列}
  $V$を線形空間とし、$\mathcal{V} = \{\vb*{v}_i\}_{i=1}^n,\, \mathcal{V}' = \{\vb*{v}'_i\}_{i=1}^n$を$V$の基底とするとき、基底変換$\mathcal{V} \rightharpoonup \mathcal{V}'$の変換行列$P$は、
  \begin{equation*}
    (\vb*{v}'_1, \ldots, \vb*{v}'_n) = (\vb*{v}_1, \ldots, \vb*{v}_n) P
  \end{equation*}
  により定まる
\end{theorem*}

この行列$P$は、座標写像を介して考えると、次の可換図式で定まるものである

\begin{equation*}
  \begin{tikzcd}
    \mathbb{R}^n \arrow[dd, "P\cdot"] \arrow[rd,"{\Phi_{\mathcal{V}'}}"] & \\
    & V \\
    \mathbb{R}^n \arrow[ru,"{\Phi_{\mathcal{V}}}"]
  \end{tikzcd}
\end{equation*}

つまり、$P$は、
\begin{equation*}
  \Phi_{\mathcal{V}}^{-1} \circ \Phi_{\mathcal{V}'} \colon \mathbb{R}^n \to \mathbb{R}^n
\end{equation*}
の標準基底に関する表現行列である

\sectionline

一方、この行列$P$はベクトルの成分表示の変換に用いることもできる

\begin{theorem}{座標ベクトルの変換則}{coordinate-change-rule}
  基底変換$\mathcal{V} \rightharpoonup \mathcal{V}'$の変換行列を$P$とし、ベクトル$\vb*{a} \in V$の$\mathcal{V},\, \mathcal{V}'$に関する座標ベクトルをそれぞれ$\vb*{x},\, \vb*{x}'$とするとき、
  \begin{equation*}
    \vb*{x} = P \vb*{x}'
  \end{equation*}
  が成り立つ
\end{theorem}

\begin{proof}
  ベクトル$\vb*{a}$の2種類の基底$\vb*{v}_1, \ldots, \vb*{v}_n$と$\vb*{v}'_1, \ldots, \vb*{v}'_n$に関する成分
  \begin{equation*}
    \begin{pmatrix}
      x_1    \\
      x_2    \\
      \vdots \\
      x_n
    \end{pmatrix},\quad
    \begin{pmatrix}
      x'_1   \\
      x'_2   \\
      \vdots \\
      x'_n
    \end{pmatrix}
  \end{equation*}
  を考えると、$\vb*{a}$の2通りの表現
  \begin{align*}
    \vb*{a} & = x_1 \vb*{v}_1 + x_2 \vb*{v}_2 + \cdots + x_n \vb*{v}_n \\
            & = (\vb*{v}_1, \ldots, \vb*{v}_n) \begin{pmatrix}
                                                 x_1    \\
                                                 x_2    \\
                                                 \vdots \\
                                                 x_n
                                               \end{pmatrix}
  \end{align*}
  \begin{align*}
    \vb*{a} & = x'_1 \vb*{v}'_1 + x'_2 \vb*{v}'_2 + \cdots + x'_n \vb*{v}'_n \\
            & = (\vb*{v}'_1, \ldots, \vb*{v}'_n) \begin{pmatrix}
                                                   x'_1   \\
                                                   x'_2   \\
                                                   \vdots \\
                                                   x'_n
                                                 \end{pmatrix}              \\
            & = (\vb*{v}_1, \ldots, \vb*{v}_n) P \begin{pmatrix}
                                                   x'_1   \\
                                                   x'_2   \\
                                                   \vdots \\
                                                   x'_n
                                                 \end{pmatrix}
  \end{align*}
  が得られる

  どちらも$(\vb*{v}_1, \ldots, \vb*{v}_n)$との積の形、すなわち$\vb*{v}_1, \ldots, \vb*{v}_n$の線形結合として表されている

  ここで、\thmref{thm:lin-indep-iff-unique-lincomb}より、基底$\vb*{v}_1, \ldots, \vb*{v}_n$の線型独立性から、その線形結合は一意的であるので、係数比較ができて、
  \begin{equation*}
    \begin{pmatrix}
      x_1    \\
      x_2    \\
      \vdots \\
      x_n
    \end{pmatrix} = P \begin{pmatrix}
      x'_1   \\
      x'_2   \\
      \vdots \\
      x'_n
    \end{pmatrix}
  \end{equation*}
  が成り立つ $\qed$
\end{proof}

\sectionline
\section{基底変換による表現行列の変化}
\marginnote{\refbookA p112〜113 \\ \refbookF p215、p223〜225}

$f\colon V \to W$を線形写像とする

\begin{equation*}
  \begin{tikzcd}
    V \arrow[r,"f"] & W
  \end{tikzcd}
\end{equation*}

$V$の基底$\mathcal{V}$と$W$の基底$\mathcal{W}$に関する$f$の表現行列を$A$とする

\begin{equation*}
  \begin{tikzcd}
    V \arrow[r,"f"] & W \\
    \mathbb{R}^n \arrow[u,"{\Phi_{\mathcal{V}}}"] \arrow[r,"A\cdot"] & \mathbb{R}^m \arrow[u,"{\Phi_{\mathcal{W}}}"]
  \end{tikzcd}
\end{equation*}

また、$V$の基底$\mathcal{V}'$と$W$の基底$\mathcal{W}'$に基底を変えるとき、$f$の表現行列を$B$とする

\begin{equation*}
  \begin{tikzcd}
    \mathbb{R}^n \arrow[d,"{\Phi_{\mathcal{V}'}}"] \arrow[r,"B\cdot"] & \mathbb{R}^m \arrow[d,"{\Phi_{\mathcal{W}'}}"] \\
    V \arrow[r,"f"] & W
  \end{tikzcd}
\end{equation*}

基底変換$\mathcal{V} \rightharpoonup \mathcal{V}'$の変換行列を$P$、$\mathcal{W} \rightharpoonup \mathcal{W}'$の変換行列を$Q$とするとき、次の可換図式で整理できる

\begin{equation*}
  \begin{tikzcd}
    \mathbb{R}^n \arrow[dd,"P\cdot"] \arrow[rd,"{\Phi_{\mathcal{V}'}}"] \arrow[rrr,"B\cdot"] &&& \mathbb{R}^m \arrow[dd,"Q\cdot"] \arrow[ld,"{\Phi_{\mathcal{W}'}}"] \\
    & V \arrow[r,"f"] & W & \\
    \mathbb{R}^n \arrow[ru,"{\Phi_{\mathcal{V}}}"] \arrow[rrr,"A\cdot"] &&& \mathbb{R}^m \arrow[lu,"{\Phi_{\mathcal{W}}}"]
  \end{tikzcd}
\end{equation*}

ここで、行列$A$によって表現される写像を$F_A$、他の行列についても同様に表すと、

\begin{equation*}
  \begin{tikzcd}
    \mathbb{R}^n \arrow[dd,"F_P"] \arrow[rd,"{\Phi_{\mathcal{V}'}}"] \arrow[rrr,"F_B"] &&& \mathbb{R}^m \arrow[dd,"F_Q"] \arrow[ld,"{\Phi_{\mathcal{W}'}}"] \\
    & V \arrow[r,"f"] & W & \\
    \mathbb{R}^n \arrow[ru,"{\Phi_{\mathcal{V}}}"] \arrow[rrr,"F_A"] &&& \mathbb{R}^m \arrow[lu,"{\Phi_{\mathcal{W}}}"]
  \end{tikzcd}
\end{equation*}

このとき、左上の$\mathbb{R}^n$から右下の$\mathbb{R}^m$への写像は、
\begin{equation*}
  F_A \circ F_P,\quad F_Q \circ F_B
\end{equation*}
という2通りの表現ができる

すなわち、
\begin{equation*}
  F_A \circ F_P = F_Q \circ F_B
\end{equation*}

合成写像は行列の積に対応するので、
\begin{equation*}
  AP = QB
\end{equation*}

ここで、$Q$は$\mathbb{R}^m \to \mathbb{R}^m$の標準基底に関する表現行列であるから、正則行列である

そこで、左から$Q^{-1}$をかけて、
\begin{equation*}
  B = Q^{-1} AP
\end{equation*}
が得られる

\begin{theorem}{基底変換に伴う表現行列の変換}{change-of-representation-matrix}
  線形写像$f\colon V \to W$の基底$\mathcal{V},\, \mathcal{W}$に関する表現行列を$A$とし、同じ線形写像$f$の別な基底$\mathcal{V}',\, \mathcal{W}'$に関する表現行列を$B$とするとき、基底変換$\mathcal{V} \rightharpoonup \mathcal{V}'$の変換行列を$P$、$\mathcal{W} \rightharpoonup \mathcal{W}'$の変換行列を$Q$とすると、
  \begin{equation*}
    B = Q^{-1} A P
  \end{equation*}
  が成り立つ
\end{theorem}

\sectionline

実用上は$V=W$である場合が特に重要で、この場合には$P=Q$とすることができるので、
\begin{equation*}
  B = P^{-1} A P
\end{equation*}
が成り立つ

\begin{theorem}{基底変換に伴う表現行列の変換(線形変換の場合)}{similarity-under-basis-change}
  線形変換$f\colon V \to V$の基底$\mathcal{V}$に関する表現行列を$A$とし、同じ線形変換$f$の別な基底$\mathcal{V}'$に関する表現行列を$B$とするとき、基底変換$\mathcal{V} \rightharpoonup \mathcal{V}'$の変換行列を$P$とすると、
  \begin{equation*}
    B = P^{-1} A P
  \end{equation*}
  が成り立つ
\end{theorem}

\end{document}
