\documentclass[../../../topic_linear-algebra]{subfiles}

\usepackage{xr-hyper}
\externaldocument{../../../.tex_intermediates/topic_linear-algebra}

\begin{document}

\sectionline
\section{基底変換による座標ベクトルの変化}
\marginnote{\refbookS p60〜61 \\ \refbookA p110〜111 \\ \refbookF p215〜219}

基底変換行列$P$は、ベクトルの成分表示の変換に用いることもできる。

次の定理は、$P$を定めた図式\eqref{eq:basis-change-diagram}において、線形空間の元の対応を考えると一目瞭然である。

\begin{theorem}{座標ベクトルの変換則}{coordinate-change-rule}
  基底変換$\mathcal{V} \rightharpoonup \mathcal{V}'$の変換行列を$P$とし、ベクトル$\vb*{v} \in V$の$\mathcal{V},\, \mathcal{V}'$に関する座標ベクトルをそれぞれ$\vb*{x},\, \vb*{x}'$とするとき、次が成り立つ。
  \begin{equation*}
    \vb*{x} = P \vb*{x}'
  \end{equation*}
\end{theorem}

\begin{figure}[h]
  \centering
  \begin{minipage}{0.4\columnwidth}
    \caption*{\bfseries 線形空間の対応}
    \begin{equation*}
      \begin{tikzcd}[every label/.append style = {font = \small}]
        K^n \arrow[dd, "P\times"'] \arrow[rd,"{\Phi_{\mathcal{V}'}}"] & \\
        & V \\
        K^n \arrow[ru,"{\Phi_{\mathcal{V}}}"']
      \end{tikzcd}
    \end{equation*}
  \end{minipage}
  \begin{minipage}{0.4\columnwidth}
    \caption*{\bfseries 線形空間の元の対応}
    \begin{equation*}
      \begin{tikzcd}[every label/.append style = {font = \small}]
        \vb*{x}' \arrow[dd,mapsto, "P\times"'] \arrow[rd,mapsto,"{\Phi_{\mathcal{V}'}}"] & \\
        & \vb*{v} \\
        \vb*{x} \arrow[ru,mapsto,"{\Phi_{\mathcal{V}}}"']
      \end{tikzcd}
    \end{equation*}
  \end{minipage}
\end{figure}

\end{document}
