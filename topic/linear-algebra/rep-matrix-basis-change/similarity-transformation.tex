\documentclass[../../../topic_linear-algebra]{subfiles}

\begin{document}

\sectionline
\section{相似な行列と相似変換}
\marginnote{\refbookJ p83〜85}

「ある種の操作を行ったら同一のものになるもの」を「互いに\keyword{相似}」と呼ぶ。

\br

たとえば、二つ以上の図形が「相似」であるとは、平行移動、回転、反転、拡大縮小などの操作を行うとそれら図形をぴったり重ねることができるという意味だった。

\br

行列に対する「相似」は、次のように定める。

\begin{definition}{行列の相似}{similar-matrices}
  正方行列$A,B$に対して、正則行列$P$が存在して次式が成り立つとき、$A$と$B$は\keyword{相似}であるという。
  \begin{equation*}
    B = P^{-1} A P
  \end{equation*}
\end{definition}

このような変換を\keyword{相似変換}という。

\br

$A$と$B$が相似であるとき、$A$と$B$は1つの線形変換$f$を異なる基底によって表現して得られた行列であるという関係にある。(\thmref{thm:similarity-under-basis-change})

\end{document}
