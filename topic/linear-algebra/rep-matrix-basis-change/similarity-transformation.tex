\documentclass[../../../topic_linear-algebra]{subfiles}

\begin{document}

\sectionline
\section{相似な行列と相似変換}
\marginnote{\refbookJ p83〜85}

\begin{mindflow}
  \placeholder{再編予定}
\end{mindflow}

実用上は$V=W$である場合が特に重要で、この場合には$P=Q$とすることができるので、
\begin{equation*}
  B = P^{-1} A P
\end{equation*}
が成り立つ

\begin{theorem}{基底変換に伴う表現行列の変換(線形変換の場合)}{similarity-under-basis-change}
  線形変換$f\colon V \to V$の基底$\mathcal{V}$に関する表現行列を$A$とし、同じ線形変換$f$の別な基底$\mathcal{V}'$に関する表現行列を$B$とするとき、基底変換$\mathcal{V} \rightharpoonup \mathcal{V}'$の変換行列を$P$とすると、
  \begin{equation*}
    B = P^{-1} A P
  \end{equation*}
  が成り立つ
\end{theorem}

「ある種の操作を行ったら同一のものになるもの」を「互いに\keyword{相似}」と呼ぶ

\br

たとえば、二つ以上の図形が「相似」であるとは、平行移動、回転、反転、拡大縮小などの操作を行うとそれら図形をぴったり重ねることができるという意味だった

\br

行列に対する「相似」は、次のように定める

\begin{definition}{行列の相似}{similar-matrices}
  正方行列$A,\,B$に対して、正則行列$P$が存在して、
  \begin{equation*}
    B = P^{-1} A P
  \end{equation*}
  が成り立つとき、$A$と$B$は\keyword{相似}であるという
\end{definition}

このような変換を\keyword{相似変換}と呼ぶ

\br

$A$と$B$が相似であるとき、$A$と$B$は1つの線形変換$f$を異なる基底によって表現して得られた行列であるという関係にある

\end{document}
