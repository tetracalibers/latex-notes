\documentclass[../../../topic_linear-algebra]{subfiles}

\usepackage{xr-hyper}
\externaldocument{../../../.tex_intermediates/topic_linear-algebra}

\begin{document}

\sectionline
\section{表現行列を定める規則}\label{sec:construction-of-matrix-rep}
\marginnote{
  \refbookA p105 \\ \refbookC p96〜97 \\
  \refweb{【表現行列】線形写像の行列表示を詳しく}{https://mathlandscape.com/map-matrix/}
}

行列表示を使えば、線形空間と線形写像についての問題を、ベクトルと行列についての問題に帰着させて解くことができる。

では、線形写像$f\colon V \to W$、$V$の基底$\mathcal{V}$、$W$の基底$\mathcal{W}$が与えられたとき、$f$の表現行列$A$は具体的にどのような規則で構成できるだろうか。

\br

\secref{sec:matrix-description-linear-map}で述べたように、数ベクトル空間の間の線形写像を定める行列は、標準基底$\vb*{e}_j$の$f$による像
\begin{equation*}
  f(\vb*{e}_j) = \vb*{a}_j = \begin{pmatrix}
    a_{1j} \\
    \vdots \\
    a_{mj}
  \end{pmatrix} \quad (1 \leq j \leq n)
\end{equation*}
を横に並べたもの、すなわち、
\begin{equation*}
  \begin{pmatrix}
    f(\vb*{e}_1) & \cdots & f(\vb*{e}_n)
  \end{pmatrix} = \begin{pmatrix}
    \vb*{a}_1 & \cdots & \vb*{a}_n
  \end{pmatrix} = A
\end{equation*}
として構成された。

\br

このような表現行列の構成を、一般の線形空間$V,W$の基底$\mathcal{V}, \mathcal{W}$に関して一般化しよう。

\br

$V, W$をそれぞれ$m,n$次元線形空間とし、それらの基底を$\mathcal{V} = \{ \vb*{v}_1, \ldots, \vb*{v}_m \}$、$\mathcal{W} = \{ \vb*{w}_1, \ldots, \vb*{w}_n \}$と定める。

\br

このとき、$\vb*{a}_j$は座標写像$\Phi_{\mathcal{W}}$によって、次のように$W$に写される。
\begin{equation*}
  \Phi_{\mathcal{W}}(\vb*{a}_j) = \sum_{i=1}^m a_{ij} \vb*{w}_i \quad (1 \leq j \leq n)
\end{equation*}

また、$\vb*{e}_j$は座標写像$\Phi_{\mathcal{V}}$によって、
\begin{equation*}
  \Phi_{\mathcal{V}}(\vb*{e}_j) = \sum_{i=1}^n e_{ij} \vb*{v}_i \quad (1 \leq j \leq n)
\end{equation*}
のように$V$に写されるが、これは$\vb*{v}_j$そのものである。

\begin{handout}[補足:$\Phi_{\mathcal{V}}(\vb*{e}_j) = \vb*{v}_j$の確認]
  たとえば、$j=1$のときは、次のようになる。
  \begin{equation*}
    \Phi_{\mathcal{V}}(\vb*{e}_1) = \sum_{i=1}^n e_{i1} \vb*{v}_i = \vb*{v}_1
  \end{equation*}
\end{handout}

\br

よって、$\vb*{e}_j \mapsto \vb*{a}_j$という写像は、
\begin{equation*}
  \vb*{v}_j \mapsto \Phi_{\mathcal{W}}(\vb*{a}_j)
\end{equation*}
という$V$から$W$への写像$f$に対応する。
\begin{equation}
  \begin{tikzcd}[every label/.append style = {font = \small}]
    \vb*{v}_j \arrow[r,mapsto,"f"] & \vb*{w}_j \\
    \vb*{e}_j \arrow[u,mapsto,"{\Phi_{\mathcal{V}}}"] \arrow[r,mapsto,"A \times"] & \vb*{a}_j \arrow[u,mapsto,"{\Phi_{\mathcal{W}}}"']
  \end{tikzcd}
\end{equation}

\br

記号を書き換えると、
\begin{equation*}
  f(\vb*{v}_j) = \Phi_{\mathcal{W}}(\vb*{a}_j) = \sum_{i=1}^m a_{ij} \vb*{w}_i
\end{equation*}
となり、これで$\vb*{v}_j$の写り先が決まったので、\thmref{thm:linear-map-determined-by-basis}より$f$を定めることができる。
\begin{align*}
  f(\vb*{v}_1) &= a_{11}\vb*{w}_1 + a_{12}\vb*{w}_2 + \cdots + a_{1n}\vb*{w}_n \\
  f(\vb*{v}_2) &= a_{21}\vb*{w}_1 + a_{22}\vb*{w}_2 + \cdots + a_{2n}\vb*{w}_n \\
  & \ldots \\
  f(\vb*{v}_m) &= a_{m1}\vb*{w}_1 + a_{m2}\vb*{w}_2 + \cdots + a_{mn}\vb*{w}_n
\end{align*}

上の$m$個の式をまとめて、次のように書くことができる。
\begin{equation*}
  \begin{pmatrix}
    f(\vb*{v}_1) & \cdots & f(\vb*{v}_m)
  \end{pmatrix} 
  = \begin{pmatrix}
    \vb*{w}_1 & \cdots & \vb*{w}_n
  \end{pmatrix} \begin{pmatrix} 
  a_{11} & a_{12} & \dots  & a_{1n} \\
  a_{21} & a_{22} & \dots  & a_{2n} \\
  \vdots & \vdots & \ddots & \vdots \\
  a_{m1} & a_{m2} & \dots  & a_{mn}
\end{pmatrix} 
\end{equation*}

\br

このときの行列$A = (a_{ij})$が、$f$の\keyword{表現行列}あるいは\keyword{行列表示}と呼ばれるものとなる。
\begin{equation*}
  \begin{pmatrix}
    f(\vb*{v}_1) & \cdots & f(\vb*{v}_m)
  \end{pmatrix} = \begin{pmatrix}
    \vb*{w}_1 & \cdots & \vb*{w}_n
  \end{pmatrix} A
\end{equation*}

\br

\begin{theorem*}{線形写像の行列表現の構成}
  $V, W$をそれぞれ$m,n$次元線形空間とし、それらの基底を$\{ \vb*{v}_1, \ldots, \vb*{v}_m \}$、$\{ \vb*{w}_1, \ldots, \vb*{w}_n \}$と定める。
  このとき、線形写像$f\colon V \to W$の表現行列$A$は、次式を満たすものとして構成される。
  \begin{equation*}
    \begin{pmatrix}
      f(\vb*{v}_1) & \cdots & f(\vb*{v}_m)
    \end{pmatrix} = \begin{pmatrix}
      \vb*{w}_1 & \cdots & \vb*{w}_n
    \end{pmatrix} A
  \end{equation*}
\end{theorem*}

\end{document}
