\documentclass[../../../topic_linear-algebra]{subfiles}

\begin{document}

\sectionline
\section{対角行列とスカラー行列}

\begin{definition}{対角行列}
  対角成分以外の成分がすべて0である正方行列を\keyword{対角行列}と呼ぶ

  $a_{ii} = c_i \quad (1 \leq i \leq n)$である対角行列を次のように表す
  \begin{equation*}
    \diag(c_1, c_2, \ldots, c_n) = \begin{pmatrix}
      c_1    & 0      & \cdots & 0      \\
      0      & c_2    & \cdots & 0      \\
      \vdots & \vdots & \ddots & \vdots \\
      0      & 0      & \cdots & c_n
    \end{pmatrix}
  \end{equation*}
\end{definition}

\br

対角行列の特別な場合として、すべての対角成分が同じ値である行列は\keyword{スカラー行列}と呼ばれる

\begin{definition}{スカラー行列}
  $c$をスカラーとするとき、$cE$の形の行列を\keyword{スカラー行列}という
  \begin{equation*}
    cE = \begin{pmatrix}
      c      & 0      & \cdots & 0      \\
      0      & c      & \cdots & 0      \\
      \vdots & \vdots & \ddots & \vdots \\
      0      & 0      & \cdots & c
    \end{pmatrix}
  \end{equation*}
\end{definition}

\sectionline
\section{対角行列とスカラー倍}

行列$A$にスカラー行列をかけることは、
\begin{equation*}
  (cE)A = A(cE) = cA
\end{equation*}
のように、スカラー$c$をかけるのと同じである

\br

発展して、対角行列の場合には次のことがいえる

\begin{theorem}{対角行列と列ベクトルのスカラー倍}
  右から対角行列をかけると、各列ベクトルがスカラー倍になる

  すなわち、$A = (\vb*{a}_1, \vb*{a}_2, \ldots, \vb*{a}_n)$とすると、
  \begin{equation*}
    A \cdot \diag(c_1, c_2, \ldots, c_n) = (c_1\vb*{a}_1, c_2\vb*{a}_2, \ldots, c_n\vb*{a}_n)
  \end{equation*}
  が成り立つ
\end{theorem}

\begin{proof}
  \todo{\refbookA p63 (問2.8)}
\end{proof}

\sectionline
\section{ブロック対角行列}
\marginnote{\refbookL p50〜51}

\todo{}

\end{document}
