\documentclass[../../../topic_linear-algebra]{subfiles}

\begin{document}

\sectionline
\section{対角行列の嬉しさ:冪乗の計算}
\marginnote{\refbookL p42}

$A$が対角行列の場合、$\vb*{y} = A\vb*{x}$は、行ごとのサブシステムとして各行を独立に計算できた
\begin{align*}
  y_1 & = a_{11} x_1    \\
      & \vdotswithin{=} \\
  y_n & = a_{nn} x_n
\end{align*}

\br

このように各行に分けて「1次元問題が$n$本あるだけ」と考えると、対角行列どうしの積や冪乗も、簡単に計算できることがわかる

\begin{equation*}
  \begin{pmatrix}
    a_1 &        &     \\
        & \ddots &     \\
        &        & a_n
  \end{pmatrix} \begin{pmatrix}
    b_1 &        &     \\
        & \ddots &     \\
        &        & b_n
  \end{pmatrix} = \begin{pmatrix}
    a_1 b_1 &        &         \\
            & \ddots &         \\
            &        & a_n b_n
  \end{pmatrix}
\end{equation*}

\br

\begin{equation*}
  \begin{pmatrix}
    a_1 &        &     \\
        & \ddots &     \\
        &        & a_n
  \end{pmatrix}^k = \begin{pmatrix}
    a_1^k &        &       \\
          & \ddots &       \\
          &        & a_n^k
  \end{pmatrix}
\end{equation*}

\end{document}
