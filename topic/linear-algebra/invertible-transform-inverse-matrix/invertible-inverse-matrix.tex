\documentclass[../../../topic_linear-algebra]{subfiles}

\begin{document}

\sectionline
\section{正則行列}
\marginnote{\refbookA p71}

\begin{definition}{正則}
  線形変換$f$は全単射であるとき、\keyword{正則}な線形変換であるという
\end{definition}

\begin{definition}{正則行列}
  正方行列$A$は、それが正則な線形変換を与えるとき、\keyword{正則行列}であるという
\end{definition}

「線形代数における鳩の巣原理」から、次のことがいえる

\begin{theorem}{正則の判定と階数}
  $n$次正方行列$A$に対して、次は同値である
  \begin{enumerate}[label=\romanlabel]
    \item $A$が正則行列
    \item $\rank(A) = n$
  \end{enumerate}
\end{theorem}

上の定理は、線形変換$f$(もしくは正方行列$A$)が\keyword{正則}かどうかについて、\keyword{階数}という1つの数値で判定できることを示している

\sectionline

\begin{theorem}{正則の判定と線型独立性}
  $n$次正方行列$A$に対して、次は同値である
  \begin{enumerate}[label=\romanlabel]
    \item $A = (\vb*{a}_1 , \vb*{a}_2 , \cdots , \vb*{a}_n)$が正則
    \item $\vb*{a}_1, \vb*{a}_2, \dots, \vb*{a}_n$が線型独立
  \end{enumerate}
\end{theorem}

\begin{proof}
  \todo{\refbookA p71 命題2.4.4}
\end{proof}

\sectionline
\section{逆行列}
\marginnote{\refbookA p71〜72}

写像$f$が全単射であれば、\keyword{逆写像}$f^{-1}$が存在する

\begin{theorem}{逆写像の線形性}
  $f$を$\mathbb{R}^n$の正則な線形変換とするとき、逆写像$f^{-1}$は線形である
\end{theorem}

\begin{proof}
  \todo{\refbookA p71 問2.16}
\end{proof}

$n$次正則行列$A$は、正則な線形変換$f\colon \mathbb{R}^n \to \mathbb{R}^n$と対応している

逆写像$f^{-1}$が存在し、線形であるから、ある$n$次正方行列$B$が対応するはずである

\br

$f \circ f^{-1} = f^{-1} \circ f = \id_{\mathbb{R}^n}$であり、線形写像の合成は行列の積に対応するから、
\begin{equation*}
  AB = BA = E
\end{equation*}
が成り立つ

\br

このような$B$を$A$の\keyword{逆行列}と呼び、$A^{-1}$と書く

\sectionline

\begin{theorem}{逆行列の一意性}
  正方行列$A$に対して、$A$の逆行列が存在するならば、それは一意的である
\end{theorem}

\begin{proof}
  \todo{\refbookA p71 問2.17}
\end{proof}

\sectionline
\section{逆行列の計算法と線形方程式}
\marginnote{\refbookA p72〜73}

正則行列$A$に対して、方程式$A\vb*{x} = \vb*{b}$のただ1つの解は次で与えられる
\begin{equation*}
  \vb*{x} = A^{-1} \vb*{b}
\end{equation*}
$A^{-1}$が計算できれば、行列のかけ算によって線型方程式の解が求められる

\sectionline

正則行列$A$の逆行列を計算するために、次の定理に注目しよう

\begin{theorem}{逆行列の計算法の原理}
  正方行列$A$に対して、$AB=E$を満たす正方行列$B$があるならば、$A$は正則であり、$B$は$A$の逆行列である
\end{theorem}

\begin{proof}
  \todo{\refbookA p72 命題2.4.6}
\end{proof}

上の定理の証明は、逆行列の計算法のヒントを含んでいる

$A$の逆行列$B$を求めるには、$n$個の線形方程式
\begin{equation*}
  A\vb*{b}_i = \vb*{e}_i \quad ( 1    \leq i \leq n )
\end{equation*}
を解けばよい

\br

$A$は階数$n$の$n$次正方行列なので、行変形で$A$から$E$に到達することができる

\br

$\vb*{b}_i$を求めるには、行変形により
\begin{equation*}
  (A \mid \vb*{e}_i) \rightarrow \cdots \rightarrow (E \mid \vb*{b}_i)
\end{equation*}
とすればよい

\br

$i$ごとに掃き出し法を何度も実行しないといけないのかと思いきや、一度にまとめられる
\begin{gather*}
  (A \mid E) = (A \mid \vb*{e}_1, \cdots , \vb*{e}_n) \rightarrow \\ \cdots \rightarrow (E \mid \vb*{b}_1 , \cdots , \vb*{b}_n) = (E \mid B)
\end{gather*}
このようにすれば、行変形は1通りで十分である

\sectionline
\section{正則行列と対角行列}
\marginnote{\refbookA p74〜75}

\begin{theorem}{上三角行列の正則性}
  対角成分がすべて0でない上三角行列は正則である
\end{theorem}

\begin{proof}
  \todo{\refbookA p74 命題2.4.9}
\end{proof}

\sectionline

\begin{theorem}{行基本変形と対角行列}
  正則行列$A$に対して、行のスカラー倍以外の行基本変形を繰り返し行って対角行列にできる
\end{theorem}

\begin{proof}
  \todo{\refbookA p75 命題2.4.12}
\end{proof}

\end{document}
