\documentclass[../../../topic_linear-algebra]{subfiles}

\begin{document}

\sectionline
\section{階数による正則判定}
\marginnote{\refbookA p71}

\note{「解の一意性」の後に移動予定}
\br

「線形代数における鳩の巣原理」から、次のことがいえる

\begin{theorem}{正則の判定と階数}\label{thm:invertible-iff-full-rank}
  $n$次正方行列$A$に対して、
  \begin{equation*}
    A \text{が正則行列} \Longleftrightarrow \rank(A) = n
  \end{equation*}
\end{theorem}

この定理は、線形変換$f$(もしくは正方行列$A$)が\keyword{正則}かどうかについて、\keyword{階数}という1つの数値で判定できることを示している

\sectionline

\begin{theorem}{列ベクトルの線型独立性による正則の判定}\label{thm:invertible-iff-col-indep}
  $n$次正方行列
  \begin{equation*}
    A = (\vb*{a}_1 , \cdots , \vb*{a}_n)
  \end{equation*}
  に対して、次が成り立つ
  \begin{equation*}
    A \text{が正則行列} \Longleftrightarrow \vb*{a}_1, \dots, \vb*{a}_n \text{が線型独立}
  \end{equation*}
\end{theorem}

\begin{proof}
  $\vb*{a}_1, \dots, \vb*{a}_n \in \mathbb{R}^n$が線型独立であることは、
  \begin{equation*}
    \rank(A) = n
  \end{equation*}
  と同値であることを\hyperref[thm:lin-indep-iff-rank-n]{以前}示した

  さらに、\hyperref[thm:invertible-iff-full-rank]{先ほど示した定理}より、$\rank(A) =n$は$A$が正則行列であることと同値である $\qed$
\end{proof}

\sectionline
\section{逆行列の計算法}
\marginnote{\refbookA p72〜73}

\note{「基本変形と基本行列」の章で扱う定理と被るので、削除予定}
\br

正則行列$A$の逆行列を計算するために、次の定理に注目しよう

\begin{theorem}{逆行列の計算法の原理}
  正方行列$A$に対して、$AB=E$を満たす正方行列$B$があるならば、$A$は正則であり、$B$は$A$の逆行列である
\end{theorem}

\begin{proof}
  \todo{\refbookA p72 命題2.4.6}
\end{proof}

上の定理の証明は、逆行列の計算法のヒントを含んでいる

$A$の逆行列$B$を求めるには、$n$個の線形方程式
\begin{equation*}
  A\vb*{b}_i = \vb*{e}_i \quad ( 1    \leq i \leq n )
\end{equation*}
を解けばよい

\br

$A$は階数$n$の$n$次正方行列なので、行変形で$A$から$E$に到達することができる

\br

$\vb*{b}_i$を求めるには、行変形により
\begin{equation*}
  (A \mid \vb*{e}_i) \rightarrow \cdots \rightarrow (E \mid \vb*{b}_i)
\end{equation*}
とすればよい

\br

$i$ごとに掃き出し法を何度も実行しないといけないのかと思いきや、一度にまとめられる
\begin{gather*}
  (A \mid E) = (A \mid \vb*{e}_1, \cdots , \vb*{e}_n) \rightarrow \\ \cdots \rightarrow (E \mid \vb*{b}_1 , \cdots , \vb*{b}_n) = (E \mid B)
\end{gather*}
このようにすれば、行変形は1通りで十分である

\end{document}
