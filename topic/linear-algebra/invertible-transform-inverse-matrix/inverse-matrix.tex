\documentclass[../../../topic_linear-algebra]{subfiles}

\begin{document}

\sectionline
\section{逆行列}
\marginnote{\refbookL p43〜44}

「写り先$\vb*{y}$から元の点$\vb*{x}$を答える」という写像に対応する行列を\keyword{逆行列}といい、$A^{-1}$と表す。

\br

この行列$A^{-1}$は、
\begin{itemize}
  \item どんな$\vb*{x}$を持ってきても、$A\vb*{x} = \vb*{y}$ならば$A^{-1}\vb*{y} = \vb*{x}$
  \item どんな$\vb*{y}$を持ってきても、$A^{-1}\vb*{y} = \vb*{x}$ならば$A\vb*{x} = \vb*{y}$
\end{itemize}
となるような行列である。

\begin{center}
  \begin{tikzcd}
    \vb*{x} \arrow[r,bend left,"A" font=\normalsize] & \vb*{y} \arrow[l,bend left,"A^{-1}" font=\normalsize]
  \end{tikzcd}
\end{center}

別の言い方をすると、
\begin{itemize}
  \item $A$して$A^{-1}$したら元に戻る
  \item $A^{-1}$して$A$したら元に戻る
\end{itemize}
となるような行列$A^{-1}$を逆行列として定義する。

\begin{definition}{逆行列}
  正方行列$A$に対して、次式を満たす行列$X$を$A$の\keyword{逆行列}といい、$A^{-1}$と表す。
  \begin{equation*}
    AX = XE = E
  \end{equation*}
\end{definition}

\end{document}
