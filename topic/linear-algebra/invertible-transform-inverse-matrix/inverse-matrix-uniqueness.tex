\documentclass[../../../topic_linear-algebra]{subfiles}

\begin{document}

\sectionline
\section{逆行列の一意性}

逆行列は、存在するとしてもただ1つしか存在しない。

\begin{theorem*}{逆行列の一意性}
  正方行列$A$に対して、$A$の逆行列が存在するならば、それは一意的である。
\end{theorem*}

\begin{proof}
  $A$の逆行列が$B_1$と$B_2$の2つあるとする。
  \begin{equation*}
    AB_1 = B_1A = E \quad \text{かつ} \quad AB_2 = B_2A = E
  \end{equation*}
  $AB_2 = E$の両辺に$B_1$をかけると、
  \begin{equation*}
    B_1 = B_1 AB_2 = (B_1 A) B_2 = E B_2 = B_2
  \end{equation*}
  よって、$B_1 = B_2$となり、逆行列は一意的である。 $\qed$
\end{proof}

\end{document}
