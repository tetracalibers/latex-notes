\documentclass[../../../topic_linear-algebra]{subfiles}

\begin{document}

\sectionline
\section{行列の演算と逆行列}

\subsection{逆行列の逆行列}
\marginnote{\refbookL p46}

「$A$の取り消し」を取り消すには、$A$すればよい。

\begin{theorem}{逆行列に対する逆行列}{inverse-of-inverse}
  正則行列$A$の逆行列$A^{-1}$は正則であり、その逆行列は$A$である。
  \begin{equation*}
    (A^{-1})^{-1} = A
  \end{equation*}
\end{theorem}

\begin{proof}
  $A$の逆行列が$A^{-1}$であることから、
  \begin{equation*}
    AA^{-1} = A^{-1}A = E
  \end{equation*}
  この式は、$A^{-1}$が正則であり、その逆行列が$A$であることを示す式でもある。 $\qed$
\end{proof}

\subsection{行列の積の逆行列}
\marginnote{\refbookL p46}

「$B$して$A$したもの」を元に戻すには、まず$A$を取り消してから$B$を取り消す必要がある。

\begin{theorem}{正則行列の積に対する逆行列}{inverse-of-product}
  正則行列$A,\,B$の積$AB$は正則であり、その逆行列は次のようになる。
  \begin{equation*}
    (AB)^{-1} = B^{-1} A^{-1}
  \end{equation*}
\end{theorem}

\begin{proof}
  \begin{align*}
    (AB)(B^{-1} A^{-1}) & = A(B B^{-1}) A^{-1} \\
                        & = A E A^{-1}         \\
                        & = E
  \end{align*}
  であり、同様に
  \begin{align*}
    (B^{-1} A^{-1})(AB) & = B^{-1} (A^{-1} A) B \\
                        & = B^{-1} E B          \\
                        & = E
  \end{align*}
  であるので、
  \begin{equation*}
    (AB)^{-1} = B^{-1} A^{-1}
  \end{equation*}
  が成り立つ。 $\qed$
\end{proof}

\subsection{転置行列の正則性}
\marginnote{\refbookA p88}

\begin{theorem}{正則行列の転置の正則性}{transpose-of-invertible}
  正則行列$A$に対して、その転置行列${}^t A$も正則である。
\end{theorem}

\begin{proof}
  $A$が正則であることから、その逆行列$A^{-1}$が存在し、
  \begin{equation*}
    A^{-1}A = E
  \end{equation*}
  両辺の転置をとると、右辺の単位行列は転置しても単位行列であり、左辺には\hyperref[thm:inverse-of-product]{正則行列の積に対する逆行列の公式}を用いて、
  \begin{equation*}
    {}^t(A^{-1}A) = {}^t A {}^t (A^{-1}) = E
  \end{equation*}
  この等式より、${}^t A$の逆行列は${}^t (A^{-1})$であることがわかる。 $\qed$
\end{proof}

\sectionline
\section{三角行列の正則性}
\marginnote{\refbookA p74}

\begin{theorem*}{上三角行列の正則性}
  対角成分がすべて0でない上三角行列は正則である。
\end{theorem*}

\begin{proof}
  \todo{\refbookA p74 命題2.4.9}
\end{proof}

\br

\begin{theorem}{正則な上三角行列の逆行列}{inverse-of-upper-triangular}
  正則な上三角行列は、その逆行列も上三角行列である。
\end{theorem}

\begin{proof}
  \todo{}
\end{proof}

\br

正則な上三角行列と関連して、次の事実が成り立つ。

\begin{theorem*}{行基本変形と対角行列}
  正則行列$A$に対して、行のスカラー倍以外の行基本変形を繰り返し行って対角行列にできる。
\end{theorem*}

\begin{proof}
  \todo{\refbookA p75 命題2.4.12}
\end{proof}

\sectionline
\section{正則行列と対角行列}
\marginnote{\refbookA p74〜75}

\todo{\refbookL p46〜47}

\begin{theorem}{ブロック対角行列の正則性}{block-diagonal-invertibility}
  次のようなブロック対角行列$M$において、対角ブロック$A,\,B$が正則であれば、$M$も正則である。
  \begin{equation*}
    M = \begin{pNiceArray}{cc|cc}[xdots={horizontal-labels,line-style = <->},first-row,last-col,margin,columns-width =1em]
      \Hdotsfor{2}^{l} & \Hdotsfor{2}^{n-l} \\
      \Block{2-2}<\large>{A} & & \Block{2-2}<\large>{O} && \Vdotsfor{2}^{l}  \\
      &&& \\
      \hline
      \Block{2-2}<\large>{O}& & \Block{2-2}<\large>{B} && \Vdotsfor{2}^{n-l} \\
      &&&
    \end{pNiceArray}
  \end{equation*}
\end{theorem}

\begin{proof}
  $A$と$B$が正則であるから、逆行列$A^{-1}$と$B^{-1}$が存在する。

  それらを用いて、次のような積を考える。
  \begin{align*}
    \begin{pmatrix}
      A & O \\
      O & B
    \end{pmatrix} \begin{pmatrix}
                    A^{-1} & O      \\
                    O      & B^{-1}
                  \end{pmatrix} & = \begin{pmatrix}
                                      A A^{-1} & O        \\
                                      O        & B B^{-1}
                                    \end{pmatrix} \\
                                  & = \begin{pmatrix}
                                        E_l & O       \\
                                        O   & E_{n-l}
                                      \end{pmatrix}    \\
                                  & = E_n
  \end{align*}
  この等式は、$M$の逆行列の存在を示している。
  \begin{equation*}
    M \begin{pmatrix}
      A^{-1} & O      \\
      O      & B^{-1}
    \end{pmatrix} = E_n
  \end{equation*}
  つまり、対角ブロックがそれぞれ正則であれば、それらの逆行列を並べることで全体の逆行列が構成できる。

  このようにして、$M$が正則であることがわかる。 $\qed$
\end{proof}

\end{document}
