\documentclass[../../../topic_linear-algebra]{subfiles}

\begin{document}

\sectionline
\section{逆写像と逆行列の対応}
\marginnote{\refbookA p71〜72}

一般に、写像$f$が全単射であれば、\keyword{逆写像}$f^{-1}$が存在する。

\begin{theorem*}{逆写像の線形性}
  $f$を$\mathbb{R}^n$の正則な線形変換とするとき、逆写像$f^{-1}$は線形である
\end{theorem*}

\begin{proof}
  $\vb*{x},\vb*{y} \in \mathbb{R}^n,\, c \in \mathbb{R}$とし、次の2つを示せばよい
  \begin{enumerate}[label=\romanlabel]
    \item $f^{-1}(\vb*{x} + \vb*{y}) = f^{-1}(\vb*{x}) + f^{-1}(\vb*{y})$
    \item $f^{-1}(c\vb*{x}) = c f^{-1}(\vb*{x})$
  \end{enumerate}

  \begin{subpattern}{(\romannum{i})}
    $f \circ f^{-1}$は恒等写像であるから、
    \begin{align*}
      \vb*{x}           & = f \circ f^{-1}(\vb*{x})           \\
      \vb*{y}           & = f \circ f^{-1}(\vb*{y})           \\
      \vb*{x} + \vb*{y} & = f \circ f^{-1}(\vb*{x} + \vb*{y})
    \end{align*}
    また、$f$は線形写像であるから、
    \begin{equation*}
      f \circ f^{-1}(\vb*{x} + \vb*{y}) = f(f^{-1}(\vb*{x}) + f^{-1}(\vb*{y}))
    \end{equation*}
    $f \circ f^{-1}(\vb*{v})$は、$f (f^{-1}(\vb*{v}))$を意味する記号なので、
    \begin{equation*}
      f(f^{-1}(\vb*{x}+\vb*{y})) = f(f^{-1}(\vb*{x}) + f^{-1}(\vb*{y}))
    \end{equation*}
    両辺を$f^{-1}$で写すと、
    \begin{equation*}
      f^{-1}(\vb*{x} + \vb*{y}) = f^{-1}(\vb*{x}) + f^{-1}(\vb*{y})
    \end{equation*}
    となり、(\romannum{i})が示された $\qed$
  \end{subpattern}

  \begin{subpattern}{(\romannum{ii})}
    $f \circ f^{-1}$は恒等写像であるから、
    \begin{align*}
      \vb*{x}  & = f \circ f^{-1}(\vb*{x}) = f(f^{-1}(\vb*{x}))   \\
      c\vb*{x} & = f \circ f^{-1}(c\vb*{x}) = f(f^{-1}(c\vb*{x}))
    \end{align*}
    $\vb*{x}= f(f^{-1}(\vb*{x}))$の両辺に$c$をかけた、次も成り立つ
    \begin{equation*}
      c\vb*{x} = c f(f^{-1}(\vb*{x}))
    \end{equation*}
    さらに、$f$は線形写像であるから、
    \begin{equation*}
      c f(f^{-1}(\vb*{x})) = f(c f^{-1}(\vb*{x}))
    \end{equation*}
    ここまでの$c\vb*{x}$の複数の表現により、次式が成り立つ
    \begin{equation*}
      f(f^{-1}(c\vb*{x})) = f(c f^{-1}(\vb*{x}))
    \end{equation*}
    両辺を$f^{-1}$で写すと、
    \begin{equation*}
      f^{-1}(c\vb*{x}) = c f^{-1}(\vb*{x})
    \end{equation*}
    となり、(\romannum{ii})が示された $\qed$
  \end{subpattern}
\end{proof}

\br

$n$次正則行列$A$は、正則な線形変換$f\colon \mathbb{R}^n \to \mathbb{R}^n$と対応している。

逆写像$f^{-1}$が存在し、線形であるから、ある$n$次正方行列$B$が対応するはずである。

\br

$f \circ f^{-1} = f^{-1} \circ f = \id_{\mathbb{R}^n}$であり、線形写像の合成は行列の積に対応するから、
\begin{equation*}
  AB = BA = E
\end{equation*}
が成り立つ。

\br

このように、逆写像の性質から、逆行列の定義式を導くこともできる。

\end{document}
