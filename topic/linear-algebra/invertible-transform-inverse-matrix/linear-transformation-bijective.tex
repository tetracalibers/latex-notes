\documentclass[../../../topic_linear-algebra]{subfiles}

\begin{document}

\sectionline
\section{線形変換の全単射性}
\marginnote{\refbookA p70}

$\mathbb{R}^n$からそれ自身への線形写像$f$を$\mathbb{R}^n$の\keyword{線形変換}と呼ぶのだった

一般の線形写像と対比して、線形変換の大きな特徴は次が成り立つことである

\begin{theorem}{線形代数における鳩の巣原理}
  $f$を$\mathbb{R}^n$の線形変換とし、$A$を$f$の表現行列とするとき、次はすべて同値である
  \begin{enumerate}[label=\romanlabel]
    \item $f$は単射
    \item $f$は全射
    \item $f$は全単射
    \item $\rank(A) = n$
  \end{enumerate}
\end{theorem}

\begin{proof}
  \todo{\refbookA p70 定理2.4.1}
\end{proof}

単射と全射は、一般には一方から他方が導かれるわけではない2つの性質だが、$\mathbb{R}^n$からそれ自身への線形写像(線形変換)の場合は同値になる

\br

上の定理は、いわば線形代数版「鳩の巣原理」である

\begin{shaded}
  有限集合$X = \{ 1, 2, \dots, n \}$からそれ自身への写像$f$に対して、単射と全射は同値である
\end{shaded}
この事実は\keyword{鳩の巣原理}と呼ばれる

\br

鳩の巣原理は、歴史的には\keyword{部屋割り論法}とも呼ばれ、
\begin{shaded}
  $n$個のものを$m$個の箱に入れるとき、$n > m$であれば、少なくとも1個の箱には1個より多いものが中にある
\end{shaded}
ことを指す

\br

ここで鳩の巣原理と呼んだのはこの命題そのものではないが、その変種と考えてよい

\end{document}
