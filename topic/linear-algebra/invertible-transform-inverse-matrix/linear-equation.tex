\documentclass[../../../topic_linear-algebra]{subfiles}

\begin{document}

\sectionline
\section{逆行列による連立一次方程式の解}

正則行列$A$に対して、方程式$A\vb*{x} = \vb*{b}$のただ1つの解は次で与えられる。
\begin{equation*}
  \vb*{x} = A^{-1} \vb*{b}
\end{equation*}

これが「ただ1つ」の解といえるのは、係数行列$A$が与えられれば、その逆行列$A^{-1}$は一意的に定まるからである。

\br

つまり、$A$が正則行列であり、その逆行列$A^{-1}$が求まれば、行列のかけ算によって連立一次方程式の解が求められる。

\end{document}
