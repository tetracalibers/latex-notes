\documentclass[../../../topic_linear-algebra]{subfiles}

\begin{document}

\sectionline
\section{線形変換と逆問題}

$\vb*{y} = A\vb*{x}$という形の式は、$\vb*{x}$と$\vb*{y}$の次元が同じならば、連立一次方程式として捉えることができた。
\begin{equation*}
  \begin{pmatrix}
    y_1    \\
    \vdots \\
    y_n
  \end{pmatrix} = \begin{pmatrix}
    a_{11} & \cdots & a_{1n} \\
    \vdots & \ddots & \vdots \\
    a_{n1} & \cdots & a_{nn}
  \end{pmatrix} \begin{pmatrix}
    x_1    \\
    \vdots \\
    x_n
  \end{pmatrix}
\end{equation*}
そして、このような形の連立方程式を解くことは、「$\vb*{y}$から$\vb*{x}$を推定する」という逆問題を解くことに相当する。

\br

一方、$\vb*{y} = A\vb*{x}$という式は、線形写像を表す式とみることもできる。

特に、$\mathbb{R}^n$からそれ自身への線形写像$f$を$\mathbb{R}^n$の\keyword{線形変換}と呼ぶのだった。

\br

言い換えると、表現行列$A$で表される線形写像$\vb*{y} = A\vb*{x}$が線形変換と呼べるのは、$\vb*{x}$と$\vb*{y}$の次元が同じ場合である。

このように、線形変換と連立一次方程式を関連づけて考えることができる。

\end{document}
