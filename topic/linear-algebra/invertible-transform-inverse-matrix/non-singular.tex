\documentclass[../../../topic_linear-algebra]{subfiles}

\begin{document}

\sectionline
\section{正則性と全単射性}

$A$の逆行列は、いつでも存在するとは限らない。

\begin{definition*}{正則(行列の言葉で)}
  正方行列$A$の逆行列が存在するとき、$A$は\keyword{正則}であるという。
\end{definition*}

\br

$A$の逆行列が存在するには、$A$が表す写像が\keyword{全単射}である、つまり$A$によって「潰れない・はみ出さない」ことが必要である。

\begin{itemize}
  \item 潰れてしまえば、元の$\vb*{x}$はわからない(単射でない場合)
  \item はみ出してしまえば、元の$\vb*{x}$は存在しない(全射でない場合)
\end{itemize}

\begin{definition*}{正則(写像の言葉で)}
  線形変換$f$が全単射であるとき、$f$は\keyword{正則}であるという。
  正方行列$A$が正則な線形変換を与えるとき、$A$は\keyword{正則行列}であるという。
\end{definition*}

\end{document}
