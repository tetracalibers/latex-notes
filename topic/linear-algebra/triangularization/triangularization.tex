\documentclass[../../../topic_linear-algebra]{subfiles}

\begin{document}

\sectionline
\section{行列の三角化}
\marginnote{\refbookF p293〜294 \\ \refbookA p195〜196 \\ \refbookC p191〜196}

対角化の次善の策として、\keyword{三角化}という方法がある

\begin{theorem}{三角化定理}
  $A$を$n$次複素正方行列とするとき、ある正則行列$P$が存在して、$P^{-1}AP$が上三角行列になる

  その対角成分は重複度を含めて$A$の固有値と一致する
\end{theorem}

\begin{proof}
  \begin{subpattern}{\bfseries 三角化できること}
    $n$に関する帰納法を用いる

    \br

    $n=1$のとき、$A$は$1\times 1$型行列なので、上三角行列である

    \br

    $n \geq 2$のとき、$\vb*{v}_1$を$A$の固有ベクトルとし、その固有値を$\alpha_1$とする

    $\vb*{v}_2,\ldots,\vb*{v}_n$を追加して、$\mathbb{C}^n$の基底に延長する

    $P_1 = (\vb*{v}_1,\ldots,\vb*{v}_n)$とおくと、
    \begin{equation*}
      P_1^{-1}AP_1 = \begin{pmatrix}
        \alpha_1 & *   \\
        \vb*{0}  & A_1
      \end{pmatrix}
    \end{equation*}
    ここで、$A_1$は$(n-1)$次正方行列である

    \br

    帰納法の仮定より、$(n-1)$次の正則行列$P_2$を選べば、$P_2^{-1}A_1P_2$は上三角行列になる

    \br

    そこで、
    \begin{equation*}
      P = P_1 \begin{pmatrix}
        1       & {}^t\vb*{0} \\
        \vb*{0} & P_2
      \end{pmatrix}
    \end{equation*}
    とおくと、\hyperref[thm:block-diagonal-invertibility]{$P_2$が正則であることから、$P$は正則}である

    $P$の逆行列は、
    \begin{equation*}
      P^{-1} = \begin{pmatrix}
        1       & {}^t\vb*{0} \\
        \vb*{0} & P_2^{-1}
      \end{pmatrix} P_1^{-1}
    \end{equation*}
    であるので、
    \begin{align*}
      P^{-1}AP & = \begin{pmatrix}
                     1       & {}^t\vb*{0} \\
                     \vb*{0} & P_2^{-1}
                   \end{pmatrix} P_1^{-1} A P_1 \begin{pmatrix}
                                                  1       & {}^t\vb*{0} \\
                                                  \vb*{0} & P_2
                                                \end{pmatrix} \\
               & = \begin{pmatrix}
                     1       & {}^t\vb*{0} \\
                     \vb*{0} & P_2^{-1}
                   \end{pmatrix} \begin{pmatrix}
                                   \alpha_1 & *   \\
                                   \vb*{0}  & A_1
                                 \end{pmatrix}\begin{pmatrix}
                                                1       & {}^t\vb*{0} \\
                                                \vb*{0} & P_2
                                              \end{pmatrix}   \\
               & = \begin{pmatrix}
                     \alpha_1 & *              \\
                     \vb*{0}  & P_2^{-1}A_1P_2
                   \end{pmatrix}
    \end{align*}
    $P_2^{-1}A_1P_2$は上三角行列であるから、$P^{-1}AP$も上三角行列となる $\qed$
  \end{subpattern}

  \begin{subpattern}{\bfseries 対角成分が固有値と一致すること}
    一般に、\hyperref[thm:det-of-triangular-matrix]{三角行列の行列式は対角成分の積}になる

    このことから、$n$次上三角行列$B = (b_{ij})$に対して、
    \begin{equation*}
      \Phi_B(x) = \det(xE - B) = (x-b_{11})\cdots (x-b_{nn})
    \end{equation*}
    が成り立つため、$B$の固有値は、特性方程式
    \begin{equation*}
      (x-b_{11})\cdots (x-b_{nn}) = 0
    \end{equation*}
    の解$b_{11},\ldots,b_{nn}$となる

    \br

    さて、$P^{-1}AP$と$A$は\hyperref[thm:char-poly-of-similar-matrices]{相似な行列であるので、その特性多項式は一致}する
    \begin{equation*}
      \Phi_A(x) = \Phi_{P^{-1}AP}(x) = \det(xE - P^{-1}AP)
    \end{equation*}
    よって、$P^{-1}AP$が上三角行列ならば、$A = (a_{ij})$とおくと、
    \begin{equation*}
      \Phi_A(x) = (x - a_{11})\cdots (x - a_{nn})
    \end{equation*}
    が成り立ち、$A$の固有値は$A$の対角成分$a_{11},\ldots,a_{nn}$となる $\qed$
  \end{subpattern}
\end{proof}

\sectionline
\section{ユニタリ行列による三角化}
\marginnote{\refbookF p294〜295 \\ \refbookC p196}

\hyperref[thm:unitary-diagonalization-of-normal]{ユニタリ行列によって対角化できる行列は正規行列}であった

したがって、正規行列以外の行列は、ユニタリ行列によって対角化することはできないが、ユニタリ行列によって三角化することはできる

\begin{theorem}{todo}
  $n$次複素正方行列$A$に対して、適当なユニタリ行列$U$により、$U^{-1}AU$を上三角行列($A$の\keyword{シューア形})にすることができる
\end{theorem}

\begin{proof}
  \todo{}
\end{proof}

\end{document}
