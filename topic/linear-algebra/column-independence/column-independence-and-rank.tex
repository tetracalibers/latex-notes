\documentclass[../../../topic_linear-algebra]{subfiles}

\begin{document}

\sectionline
\section{列ベクトルの線型独立性と階数}
\marginnote{\refbookA p40〜41}

斉次形方程式$A\vb*{x} = \vb*{o}$の非自明解の存在に対して、次の解釈もできる。

\begin{theorem}{斉次形方程式の非自明解の存在と線形従属}
  $m \times n$型行列$A$の列ベクトルを$\vb*{a}_1, \dots, \vb*{a}_n$とするとき、
  \begin{equation*}
    A\vb*{x} = \vb*{o}\text{に自明でない解がある}
    \Longleftrightarrow \vb*{a}_1, \dots, \vb*{a}_n\text{が線形従属}
  \end{equation*}
\end{theorem}

\begin{proof}
  $A\vb*{x} = \vb*{o}$は、ベクトルの等式
  \begin{equation*}
    x_1 \vb*{a}_1 + \cdots + x_n \vb*{a}_n = \vb*{o}
  \end{equation*}
  と同じものである。

  \begin{subpattern}{$\Longrightarrow$}
    もし自明でない解があるならば、$x_1, \dots, x_n$のうち少なくとも1つは0ではない。

    $x_1 \vb*{a}_1 + \cdots + x_n \vb*{a}_n = \vb*{o}$が成り立つもとで、0でない係数が存在するということは、$\vb*{a}_1, \dots, \vb*{a}_n$が線形従属であることを意味する。 $\qed$
  \end{subpattern}

  \begin{subpattern}{$\Longleftarrow$}
    対偶を示す。

    $\vb*{a}_1, \dots, \vb*{a}_n$が線形独立であれば、
    \begin{equation*}
      x_1 \vb*{a}_1 + \cdots + x_n \vb*{a}_n = \vb*{o}
    \end{equation*}
    において、すべての係数$x_1, \dots, x_n$は0でなければならない。

    よって、0以外の解(非自明解)は存在しないことになる。 $\qed$
  \end{subpattern}
\end{proof}

\sectionline

斉次形方程式に自明でない解が存在することは、$\rank(A) \neq n$、すなわち解の自由度が0ではないことと同値であった。

\br

一般に、斉次形の線型方程式$A\vb*{x} = \vb*{o}$の解の自由度は、$n$を変数の個数とするとき$n - \rank(A)$なので、次が成り立つ。

\begin{theorem}{列ベクトルの線型独立性と階数}\label{thm:lin-indep-iff-rank-n}
  $\vb*{a}_1, \dots, \vb*{a}_n \in \mathbb{R}^m$に対して、$A=(\vb*{a}_1, \dots, \vb*{a}_n)$とおくと、
  \begin{equation*}
    \vb*{a}_1, \dots, \vb*{a}_n\text{が線型独立} \Longleftrightarrow \rank(A) = n
  \end{equation*}
\end{theorem}

\end{document}
