\documentclass[../../../topic_linear-algebra]{subfiles}

\begin{document}

\sectionline
\section{列ベクトルの線型独立性と行基本変形}
\marginnote{\refbookA p42〜44}

\hyperref[thm:lin-indep-iff-rank-n]{列ベクトルの線型独立性と階数}では、列ベクトルの線型独立性を行列の階数で言い換えられることを示した。

\br

行列の階数は、行基本変形を施した結果である行階段形からわかるものである。

もしも行基本変形によって列ベクトルの線型独立性が変化するとしたら、階数との関係も変わってしまうのではないだろうか。

\br

この心配が杞憂であることは、次の定理からわかる。

\begin{theorem}{行基本変形による線型独立性の不変性}{row-operation-preserves-dependence}
  行変形は列ベクトルの線形関係を保つ。

  すなわち、行列$A = (\vb*{a}_1, \dots, \vb*{a}_n)$に行の変形を施して$B = (\vb*{b}_1, \dots, \vb*{b}_n)$が得られたとするとき、
  \begin{equation*}
    \sum_{i=1}^n c_i \vb*{a}_i = \vb*{o} \Longleftrightarrow \sum_{i=1}^n c_i \vb*{b}_i = \vb*{o}
  \end{equation*}

  特に、
  \begin{equation*}
    \{ \vb*{a}_1, \dots, \vb*{a}_n \} \text{が線型独立} \Longleftrightarrow \{ \vb*{b}_1, \dots, \vb*{b}_n \} \text{が線型独立}
  \end{equation*}
\end{theorem}

\begin{proof}
  \hyperref[thm:row-operation-by-elementary-matrices]{基本行列の積による行変形の構成}より、$P$を基本行列の積(正則行列)とすると、$B = PA$が成り立つ。

  \br

  よって、$\vb*{b}_i = P \vb*{a}_i$であり、線形関係式
  \begin{equation*}
    \sum_{i=1}^n c_i \vb*{a}_i = \vb*{0}
  \end{equation*}
  に左から$P$をかけることで、
  \begin{equation*}
    \sum_{i=1}^n c_i \vb*{b}_i = \vb*{0}
  \end{equation*}
  が得られる。

  \br

  逆に、$\displaystyle\sum_{i=1}^n c_i \vb*{b}_i = \vb*{0}$が成り立つとき、$P^{-1}$を左からかけることで、
  \begin{equation*}
    \sum_{i=1}^n c_i \vb*{a}_i = \vb*{0}
  \end{equation*}
  が得られる。

  \br

  したがって、
  \begin{equation*}
    \sum_{i=1}^n c_i \vb*{a}_i = \vb*{0} \Longleftrightarrow \sum_{i=1}^n c_i \vb*{b}_i = \vb*{0}
  \end{equation*}
  が成り立つ。 $\qed$
\end{proof}

\end{document}
