\documentclass[../../../topic_linear-algebra]{subfiles}

\begin{document}

\sectionline
\section{列ベクトルの線型独立性と行基本変形}
\marginnote{\refbookA p42〜44}

行列の階数は、行基本変形を施した結果である行階段形からわかる。

\br

このとき、行階段形に至るまでの行変形の仕方は一通りとは限らない。

では、変形の仕方によって階数が変わることはないのだろうか?

\br

その問いに答える第一歩となるのが、次の定理である。

\begin{theorem}{行基本変形による線型独立性の不変性}\label{thm:row-operation-preserves-dependence}
  行変形は列ベクトルの線形関係を保つ。

  すなわち、行列$A = (\vb*{a}_1, \dots, \vb*{a}_n)$に行の変形を施して$B = (\vb*{b}_1, \dots, \vb*{b}_n)$が得られたとするとき、
  \begin{equation*}
    \sum_{i=1}^n c_i \vb*{a}_i = \vb*{o} \Longleftrightarrow \sum_{i=1}^n c_i \vb*{b}_i = \vb*{o}
  \end{equation*}

  特に、
  \begin{equation*}
    \{ \vb*{a}_1, \dots, \vb*{a}_n \} \text{が線型独立} \Longleftrightarrow \{ \vb*{b}_1, \dots, \vb*{b}_n \} \text{が線型独立}
  \end{equation*}
\end{theorem}

\begin{proof}
  \todo{\refbookA p42 (命題 1.6.8)}
\end{proof}

\end{document}
