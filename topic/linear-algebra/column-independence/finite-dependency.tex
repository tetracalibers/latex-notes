\documentclass[../../../topic_linear-algebra]{subfiles}

\begin{document}

\sectionline
\section{非自明解の存在と有限従属性}
\marginnote{\refbookA p40〜41}

斉次形方程式$A\vb*{x} = \vb*{o}$の非自明解の存在に対して、次の解釈もできる。

\begin{theorem*}{斉次形方程式の非自明解の存在と線形従属}
  $m \times n$型行列$A$の列ベクトルを$\vb*{a}_1, \dots, \vb*{a}_n$とするとき、
  \begin{equation*}
    A\vb*{x} = \vb*{o}\text{に自明でない解がある}
    \Longleftrightarrow \vb*{a}_1, \dots, \vb*{a}_n\text{が線形従属}
  \end{equation*}
\end{theorem*}

\begin{proof}
  $A\vb*{x} = \vb*{o}$は、ベクトルの等式
  \begin{equation*}
    x_1 \vb*{a}_1 + \cdots + x_n \vb*{a}_n = \vb*{o}
  \end{equation*}
  と同じものである。

  \begin{subpattern}{$\Longrightarrow$}
    もし自明でない解があるならば、$x_1, \dots, x_n$のうち少なくとも1つは0ではない。

    $x_1 \vb*{a}_1 + \cdots + x_n \vb*{a}_n = \vb*{o}$が成り立つもとで、0でない係数が存在するということは、$\vb*{a}_1, \dots, \vb*{a}_n$が線形従属であることを意味する。 $\qed$
  \end{subpattern}

  \begin{subpattern}{$\Longleftarrow$}
    対偶を示す。

    $\vb*{a}_1, \dots, \vb*{a}_n$が線形独立であれば、
    \begin{equation*}
      x_1 \vb*{a}_1 + \cdots + x_n \vb*{a}_n = \vb*{o}
    \end{equation*}
    において、すべての係数$x_1, \dots, x_n$は0でなければならない。

    よって、0以外の解(非自明解)は存在しないことになる。 $\qed$
  \end{subpattern}
\end{proof}

\br

この命題の否定をとると、
\begin{equation*}
  A\vb*{x} = \vb*{o}\text{には自明解しか存在しない}
  \Longleftrightarrow \vb*{a}_1, \dots, \vb*{a}_n\text{が線形独立}
\end{equation*}
となる。

\br

ここで、\hyperref[thm:homogeneous-trivial-iff-full-col-rank]{斉次形方程式の非自明解の存在条件}より、斉次形方程式$A\vb*{x} = \vb*{o}$において自明解しか存在しないことは、$\rank(A) = n$、すなわち\hyperref[sec:degrees-of-freedom]{解の自由度}が0であることと同値であった。

\br

つまり、次が成り立つことがわかる。

\begin{theorem}{列ベクトルの線型独立性と階数}{lin-indep-iff-rank-n}
  $\vb*{a}_1, \dots, \vb*{a}_n \in \mathbb{R}^m$に対して、$A=(\vb*{a}_1, \dots, \vb*{a}_n)$とおくと、
  \begin{equation*}
    \vb*{a}_1, \dots, \vb*{a}_n\text{が線型独立} \Longleftrightarrow \rank(A) = n
  \end{equation*}
\end{theorem}

\subsection{有限従属性(方程式の視点)}

$\rank(A) = n$が成り立つ条件をさらに言い換えてみよう。

\br

$\vb*{a}_1, \dots, \vb*{a}_n \in \mathbb{R}^m$に対して、$A=(\vb*{a}_1, \dots, \vb*{a}_n)$とおくと、$A$は$m \times n$型行列である。

\br

\hyperref[thm:rank-bounds]{階数のとりうる値の範囲}より、
\begin{equation*}
  \rank A \leq \min(m,n)
\end{equation*}
であるから、もしも列の方が行よりも多い、つまり$n > m$であれば、$\rank A \leq m < n$となり、$\rank A = n$が成り立つことはない。

\br

よって、次の関係がいえる。
\begin{align*}
  A\vb*{x} = \vb*{o}\text{に自明でない解がある} & \Longleftrightarrow \vb*{a}_1, \dots, \vb*{a}_n\text{が線形従属} \\
                                      & \Longleftrightarrow \rank A \neq n
\end{align*}

ここで$n$は変数の個数、$m$は方程式の個数であるので、$n > m$という状況を次のようにまとめることができる。

\begin{theorem*}{斉次形方程式における有限従属性}
  斉次線型方程式$A\vb*{x} = \vb*{o}$において、変数の個数が方程式の個数よりも多いときには、非自明な解が存在する。
\end{theorem*}

\subsection{有限従属性(ベクトルの集合における視点)}

連立方程式の文脈に限定せず、より抽象的に言い換えたものが次の定理である。

\begin{theorem}{有限従属性定理}{finite-dependency}
  $\mathbb{R}^m$内の$m$個よりも多いベクトルからなる集合は線形従属である。
\end{theorem}

$n > m$の場合、$\vb*{a}_1, \dots, \vb*{a}_n \in \mathbb{R}^m$は線形従属となることを述べている。

\br

この結論は、幾何的な直観からは自然だといえる。

たとえば、平面$\mathbb{R}^2$内に3つ以上のベクトルがあれば、自動的に線形従属になる。

\br

この事実は、\keyword{次元}の概念を議論する際の基礎となる。

\end{document}
