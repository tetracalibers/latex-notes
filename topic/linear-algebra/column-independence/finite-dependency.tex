\documentclass[../../../topic_linear-algebra]{subfiles}

\begin{document}

\sectionline
\section{有限従属性定理}

次の事実は、\keyword{次元}の概念を議論する際の基礎になる。

\begin{theorem}{有限従属性定理}\label{thm:finite-dependency}
  $\mathbb{R}^m$内の$m$個よりも多いベクトルからなる集合は線形従属である。
\end{theorem}

\begin{proof}
  \todo{\refbookA p41 (系1.6.6)}
\end{proof}

\br

この結論は、幾何的な直観からは自然だといえる。

たとえば、平面$\mathbb{R}^2$内の3つ以上のベクトルがあれば、自動的に線形従属になる。

\br

また、同じことを線型方程式の文脈に言い換えると、次のようになる。

\begin{theorem}{有限従属性定理の線型方程式版}
  斉次線型方程式$A\vb*{x} = \vb*{o}$において、変数の個数が方程式の個数よりも多いときには、非自明な解が存在する。
\end{theorem}

\end{document}
