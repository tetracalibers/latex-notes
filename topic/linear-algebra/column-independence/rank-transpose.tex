\documentclass[../../../topic_linear-algebra]{subfiles}

\usepackage{xr-hyper}
\externaldocument{../../../.tex_intermediates/topic_linear-algebra}

\begin{document}

\sectionline
\section{転置による階数の不変性}
\marginnote{\refbookA p88 \\ \refbookF p78〜80}

ここまで、行列の列ベクトルと階数の関係を考察してきたが、行ベクトルと階数の関係はどうだろうか?

\br

\thmref{thm:rank-normal-form-by-regular-matrices}を用いて、次の重要な事実を証明することができる。

\begin{theorem*}{転置に関する階数の不変性}
  任意の行列$A$に対して、
  \begin{equation*}
    \rank A = \rank {}^t A
  \end{equation*}
\end{theorem*}

\begin{proof}
  $A$の階数標準形を$B$とすると、$B=PAQ$となる正則行列$P,\,Q$をとることができる。

  両辺の転置をとると、
  \begin{equation*}
    {}^t B = {}^t (PAQ) = {}^t Q {}^t A {}^t P
  \end{equation*}
  となり、ここで、\thmref{thm:transpose-of-invertible}より、${}^t P,\,{}^t Q$も正則行列である。

  よって、${}^tA$の階数標準形は${}^t B$である。

  \br

  $B$は階数標準形であり、その形から明らかに
  \begin{equation*}
    \rank B = \rank {}^t B
  \end{equation*}
  が成り立つので、変形前の行列$A$についても
  \begin{equation*}
    \rank A = \rank {}^t A
  \end{equation*}
  が成り立つ。 $\qed$
\end{proof}

\br

\thmref{thm:rank-equals-max-indep-cols}より、行列$A$の階数は$A$の線型独立な列ベクトルの最大個数であったが、この定理から次のこともいえるようになった。

\begin{theorem*}{階数と線型独立な行ベクトルの最大個数}
  行列$A$の階数$\rank A$は、$A$の行ベクトルに含まれる線型独立なベクトルの最大個数と一致する。
\end{theorem*}

\br

この事実を連立方程式の視点で解釈すると、
\begin{emphabox}
  \begin{spacebox}
    \begin{center}
      係数行列$A$の階数は、\\
      独立な(本質的に意味を持つ)方程式の最大本数
    \end{center}
  \end{spacebox}
\end{emphabox}
を表しているといえる。

\end{document}
