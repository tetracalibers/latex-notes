\documentclass[../../../topic_linear-algebra]{subfiles}

\begin{document}

\sectionline
\section{列ベクトルによる階数の再解釈}
\marginnote{\refbookA p42〜44}

\begin{theorem}{列ベクトルの線形従属性と階数}
  行列$A$の列ベクトルから$\rank A$個よりも多いベクトルを選ぶと、線形従属になる。
\end{theorem}

\begin{proof}
  \todo{\refbookA p43 (命題 1.6.12)}
\end{proof}

\br

以上によって、行列の階数に関する次の理解が得られたことになる。

\begin{theorem}{階数と線型独立な列ベクトルの最大個数}\label{thm:rank-equals-max-indep-cols}
  行列$A$の階数$\rank A$は、$A$の列ベクトルに含まれる線型独立なベクトルの最大個数と一致する。
\end{theorem}

\begin{proof}
  \todo{\refbookA p43 (定理 1.6.13)}
\end{proof}

\br

これで、「行変形を繰り返して行階段形にしたときの0でない段の数」として導入した階数という量の、より本質的な意味がわかったことになる。

\br

特に、
\begin{emphabox}
  \begin{spacebox}
    \begin{center}
      行変形によって定めた階数が行変形の仕方によらない
    \end{center}
  \end{spacebox}
\end{emphabox}
という事実がこの定理からしたがう。

\end{document}
