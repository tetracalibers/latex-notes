\documentclass[../../../topic_machine-learning]{subfiles}

\begin{document}

\sectionline
\section{正則化}

複雑なモデルは多くのパラメータを持ち、柔軟に調整することができる。

しかし、モデルを柔軟にしすぎると\hyperref[sec:overfitting]{過学習}を起こす可能性がある。

\br

そこで、パラメータに制限(条件)をつけることで過学習を防ぐ手法として、\keywordJE{正則化}{regularization}がある。

\begin{definition}{正則化}
  学習時に行う、訓練誤差の最小化に加えて汎化性能を上げるための操作
\end{definition}

過学習とは、訓練データではうまくいくが、未知のデータではうまくいかない状態のことだった。

過学習を防ぐということは、未知のデータに対してもうまく動くような「\keyword{汎化能力}を与える」ことだといえる。

\sectionline
\section{目的関数の変更による正則化}

正則化では、モデルを訓練する際に、次の2つの最適化を試みる。
\begin{itemize}
  \item モデルの性能を向上させる(訓練誤差の最小化)
  \item モデルの複雑さを減らす(汎化性能の向上)
\end{itemize}

そのために、性能の指標と複雑度の指標を数値化し、それらを組み合わせた上で最適化問題を解くという方法をとる。
\begin{itemize}
  \item 性能の指標:訓練誤差
  \item 複雑度の指標:\keywordJE{正規化項}{regularization term}
\end{itemize}

つまり、目的関数を次のように正則化項を加えた形に変更する。
\begin{equation*}
  \text{\bfseries 目的関数} = \text{\bfseries 訓練誤差} + \lambda\cdot\text{\bfseries 正則化項}
\end{equation*}

ここで、$\lambda$は正の値であり、\keyword{正則化パラメータ}と呼ばれるものである。

\sectionline
\section{正則化項}

正則化項は、モデルの複雑度を数値化したものである。

パラメータの数が多いモデルや、パラメータの値が大きいモデルは、複雑になる傾向がある。

そこで、次のような正則化項を用いることが多い。
\begin{itemize}
  \item \keyword{\en{L1}ノルム}:パラメータの絶対値の合計
  \item \keyword{\en{L2}ノルム}:パラメータの二乗の合計
\end{itemize}

絶対値や二乗を使うのは、負のパラメータをなくすためである。

そうしないと、大きな負の値によって大きな正の値が相殺され、非常に複雑なモデルでもこれらの値が小さくなってしまうリスクがある。

\br

\en{L1}ノルムを使った正則化は\keyword{\en{L1}正則化}、\en{L2}ノルムを使った正則化は\keyword{\en{L2}正則化}と呼ばれる。

\sectionline
\section{正則化パラメータ}

正則化を用いても、モデルの性能を向上させようとすると複雑さが増し、モデルを単純化しようとすると性能が低下する…という網引き状態に陥ることがある。

その場合は、正則化パラメータ$\lambda$によって、性能と複雑度の間で調整を行う。

\br

\keyword{正則化パラメータ}$\lambda$は、訓練誤差に対し、正則化項をどれだけ重視するのかを決めるパラメータである。

\begin{itemize}
  \item $\lambda$が大きければ、正則化項を重視する
  \item $\lambda$が小さければ、訓練誤差を重視する
\end{itemize}

つまり、正則化パラメータの目的は、モデルの訓練プロセスにおいて、性能と単純さのどちらを重視すべきかを決めることにある。

\sectionline
\section{ハイパーパラメータ}

正則化パラメータのように、学習中に最適化するのではなく、学習時に前もって決める必要のあるパラメータを\keywordJE{ハイパーパラメータ}{hyperparameter}という。

\end{document}
