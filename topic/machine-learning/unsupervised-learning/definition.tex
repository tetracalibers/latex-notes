\documentclass[../../../topic_machine-learning]{subfiles}

\begin{document}

\sectionline
\section{教師なし学習}
\marginnote{\refbookMA p25〜26}

\keyword{教師なし学習}は、ラベルなしデータを扱う機械学習である

ラベル(予測の目的変数または正解値)がないデータから、できるだけ多くの情報を抽出することが目標となる

\br

たとえば、ラベルが付いていない動物の画像のデータセットからは、それぞれの画像が表している動物の種類はわからないため、新しい画像がどの動物なのかを予測することはできない

しかし、2つの画像が似ているかどうかなど、他にできることがある

\br

つまり、教師なし学習アルゴリズムは、類似性に基づいてデータを分類できるが、それぞれのグループが何を表すのかはわからない

\sectionline
\section{教師なし学習によるデータの前処理}
\marginnote{\refbookMA p26}

実際には、教師なし学習はラベルが付いている場合でも利用できる

教師なし学習を使ってデータの\keyword{前処理}を行うと、教師あり学習の手法の効果を高めることができる

\sectionline
\section{教師なし学習の種類}
\marginnote{\refbookMA p26}

教師なし学習には、大きく分けて3種類の学習法がある

\begin{itemize}
  \item \keyword{クラスタリング}:データを類似性に基づいてクラスタに分類する
  \item \keyword{次元削減}:データを単純化し、より少ない特徴量でデータを正確に説明する
  \item \keyword{生成学習}:既存のデータに似ている新しいデータ点を生成する
\end{itemize}

\sectionline
\section{クラスタリング}
\marginnote{\refbookMA p26〜30}

\keyword{クラスタリング}は、データセット内の要素を類似性の高いデータ点ごとにクラスタ(グループ)に分割する

\br

特徴量が3つを超えると、その次元を可視化できなくなるため、人間がクラスタを目で確認するのは不可能になる

コンピュータを使うことで、巨大なデータセットに対してもクラスタリングを行うことができる

\sectionline
\section{次元削減}
\marginnote{\refbookMA p30〜32}

\todo{}

\sectionline
\section{行列分解と特異値分解}
\marginnote{\refbookMA p32〜34}

\todo{}

\sectionline
\section{生成学習}
\marginnote{\refbookMA p34}

\todo{}

\end{document}
