\documentclass[../../../topic_machine-learning]{subfiles}

\begin{document}

\sectionline
\section{解析解と数値解}

目的関数が設定できたら、その目的関数の値を最小化するようなパラメータを求める。

\subsection{解析的なアプローチ}

目的関数の値を小さくできるようなパラメータを求めるには、「解析的に解けるか」をまず調べる。

たとえば、二次方程式の解の公式のように、解を得られる式が存在する場合は、「解析的に解ける」という。
このように直接求められる解を\keyword{解析解}という。

\subsection{逐次的なアプローチ}

一方、解析解が使えない場合は、\keyword{数値解}を求める。

適当な初期値から始め、それを逐次的に更新していくことで、解を探索する。

\br

解析的に解ける場合であっても、計算量が多すぎる場合は、このように逐次的に解くアプローチが用いられる。

\end{document}
