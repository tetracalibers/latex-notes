\documentclass[../../../topic_machine-learning]{subfiles}

\begin{document}

\sectionline
\section{教師あり学習}

\keywordJE{教師あり学習}{supervised learning}では、入力$x$と推定したい出力$y$からなるペア$(x,y)$を訓練データとして利用し、
\begin{emphabox}
  \begin{spacebox}
    \begin{center}
      入力$x$から望ましい出力$y$を予測できるような\\
      モデル$y = f(x; \theta)$を学習すること
    \end{center}
  \end{spacebox}
\end{emphabox}
を目標とする。

\br

訓練データは、\keyword{教師ありデータ}や\keyword{学習データ}と呼ばれることもある。
\begin{equation*}
  D = \{ (x_1, y_1), \ldots, (x_n, y_n) \}
\end{equation*}

\subsection{学習と推論}

教師あり学習は、\keywordJE{学習}{training}と\keywordJE{推論}{inference}という2つのフェーズに分けられる。

学習フェーズでは、訓練データをうまく推定できるように、すなわち
\begin{equation*}
  y_i = f(x_i; \theta)
\end{equation*}
となるように、モデルのパラメータを調整していく。

\br

推論フェーズでは、学習によって得られたパラメータ$\hat{\theta}$を使ったモデル$f(x;\hat{\theta})$を使い、新しいテストデータ$\tilde{x}$の出力を
\begin{equation*}
  \tilde{y} = f(\tilde{x}; \hat{\theta})
\end{equation*}
として求める。

\subsection{学習の実現}

教師あり学習では、\keyword{訓練データ}、\keyword{モデル}、\keyword{損失関数}、\keyword{目的関数}、\keyword{最適化}をそれぞれ設計して組み合わせることで、学習を実現する。

\begin{enumerate}
  \item \keyword{訓練データ}を用意する:$(x_i, y_i)_{i=1}^n$
  \item 学習対象の\keyword{モデル}を用意する:$y = f(x;\theta)$
  \item \keyword{損失関数}を設計する:$l(y,y')$
  \item \keyword{目的関数}を導出する:$L(\theta) = \sum_i l(y_i, f(x_i;\theta)) + R(\theta)$
  \item \keyword{最適化}問題を解く(\keyword{学習}):$\theta^* = \underset{\theta}{\argmin} L(\theta)$
  \item 学習して得られたモデルを\keyword{評価}する
\end{enumerate}

\note{それぞれの章へのリンクを貼る}

\end{document}
