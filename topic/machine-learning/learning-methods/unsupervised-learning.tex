\documentclass[../../../topic_machine-learning]{subfiles}

\begin{document}

\sectionline
\section{教師なし学習}

\keywordJE{教師なし学習}{unsupervised learning}は、教師(正解)がつけられていないデータ
\begin{equation*}
  D = \{  x_1, \ldots, x_n \}
\end{equation*}
を利用した学習であり、
\begin{emphabox}
  \begin{spacebox}
    \begin{center}
      データの特徴を捉え、データの最適な表現や\\
      データ間の関係を獲得すること
    \end{center}
  \end{spacebox}
\end{emphabox}
を目標とする。

\br

たとえば、教師なし学習では分類する目標がデータとして与えられないため、画像分類は学習できない。

代わりに、画像をどのようにデータとして表現できれば、後続のタスクがうまく処理できるのかを学習する。

\subsection{教師なし学習の代表例}

\todo{\refbookA p63〜}

\end{document}
