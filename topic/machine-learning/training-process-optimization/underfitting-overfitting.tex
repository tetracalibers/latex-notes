\documentclass[../../../topic_machine-learning]{subfiles}

\begin{document}

\sectionline
\section{学習不足と過学習}
\marginnote{\refbookMA p84〜88 \\ \refbookSA p133〜134}

次数を$d$とする多項式関数による多項式回帰を考える

\br

$d = 1$の場合、これは線形回帰であり、関数のグラフは直線になる

$d > 1$の場合、関数のグラフは、最大で$d-1$回カーブする曲線になる

\subsection{学習不足(次数が小さすぎる場合)}

たとえば、データ点がまったく直線状に並んでいるとはいえないデータセットに対して、うまく適合する直線を見つけることはできない

複雑なデータに$d = 1$の線形回帰を適用しても、その結果得られた直線では、未知のデータを予測することはできない

\br

これが\keywordJE{学習不足}{underfitting}の例であり、データセットが複雑であるにもかかわらず、単純なモデルしかないという状態に陥っている

\begin{definition}{学習不足}
  訓練データのパターンをうまく認識するにはモデルの複雑さが十分ではなく、未知のデータにうまく汎化(対応)できないこと
\end{definition}

学習不足が起こるのは、「単純すぎる」モデルを訓練しようとしたときである

\subsection{過学習(次数が大きすぎる場合)}

単純なデータセットに対して、次数$d$の大きい多項式回帰を適用した場合にも問題が生じる

\br

グラフのカーブが多すぎると、どのデータ点にもうまく接しているように見えるが、データが存在しないところで余分なカーブが発生する

このような曲線はデータの本質を捉えておらず、未知のデータに対して的外れな予測値を返す可能性が高い

\br

これが\keywordJE{過学習}{overfitting}の例であり、モデルが過度に柔軟(複雑)だと、データがないところでの振る舞いが不自然になってしまう

\begin{definition}{過学習}
  モデルは訓練データのパターンを認識しているが、モデルが複雑すぎるために、未知のデータにうまく汎化できないこと
\end{definition}

過学習が起こるのは、「複雑すぎる」モデルを訓練しようとしたときである

\end{document}
