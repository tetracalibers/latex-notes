\documentclass[../../../topic_machine-learning]{subfiles}

\begin{document}

\sectionline
\section{モデルとパラメータによる学習}

コンピュータはデータを使って\keywordJE{モデル}{model}を構築するという方法で問題を解く。

\begin{definition}{モデル}
  データを表すルールの集まりであり、予測を行うために使うことができる
\end{definition}

モデルは、対象の問題で獲得したい分類器や予測器、生成器などであり、入力(具体的なデータ)から出力(予測結果)を計算する関数とみなすことができる。

\subsection{パラメータ}

モデルの挙動を調整する「つまみ」となる変数を\keywordJE{パラメータ}{parameter}という。

パラメータ$\theta$によって関数の挙動が決まることを、関数が$\theta$によって「特徴づけられた」という。

\begin{definition}{パラメトリックモデル}
  パラメータによって特徴づけられたモデル
\end{definition}

パラメータ$\theta$によって挙動が決まるモデルを$f(x; \theta)$と表記する。

「$;$」以降の変数は、この関数の入力ではないことを表している。

\subsection{学習の定義}

モデルのパラメータを調整して、最適なモデルを構築することを\keyword{学習}という。

\begin{definition}{学習}
  「データから\keyword{学習}する」とは、データからモデルの最適なパラメータを推定すること
\end{definition}

人間の学習も、脳内にある神経回路のパラメータ(シナプスの重みなど)を調整して実現されている。

\sectionline
\section{パラメータ数とモデルの表現力}

モデルは、成り立つかもしれない\keyword{仮説}を表すものであり、パラメータの値によって、異なる仮説を表現することができる。

つまり、パラメータの推定は、複数存在する仮説の中から一つを選択することとみなすことができる。

\br

たとえば、モデルのパラメータ$\theta$が$\{ -1, 0, 1 \}$のいずれかの値をとる場合は、
\begin{itemize}
  \item $\theta = -1$の場合のモデルが表す仮説
  \item $\theta = 0$の場合のモデルが表す仮説
  \item $\theta = 1$の場合のモデルが表す仮説
\end{itemize}
の中から選択しているとみなせる。

\br

さまざまな仮説の中から選ぶことができる場合、「モデルの\keyword{表現力}が高い」という。

単純には、パラメータ数が多いモデルほど仮説数が多く、表現力が高いといえる。

\end{document}
