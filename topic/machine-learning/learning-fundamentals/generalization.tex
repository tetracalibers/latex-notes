\documentclass[../../../topic_machine-learning]{subfiles}

\begin{document}

\sectionline
\section{汎化能力}

計算機は多くの情報を誤りなく大量に記憶することができる。

そのため、起きうる事象を十分網羅できるようにデータを用意できれば、わざわざモデルを作らなくても、過去の似たような値をそのまま使えばよいのではないか?と考えることもできる。

\br

学習時のデータをすべてそのまま記憶し、それを予測時に利用するアプローチを\keywordJE{丸暗記}{memorization}という。

\br

しかし、世の中の多くの問題では、すべてのケースを前もって列挙したり、それらを記憶しておくことはできない。

特に、入力値が\keyword{高次元データ}(画像、音声、言語、時系列、etc.)や\keyword{連続値}である場合、すべての事例を網羅することは不可能である。

\subsection{丸暗記できない場合は機械学習が有効}

機械学習は、有限の「訓練データ」を用いて、無限ともいえる「未知のデータ」に対してもうまく動くようなモデルを作る手法といえる。

\br

未知のデータに対してもうまく動く能力を\keywordJE{汎化能力}{generalization ability}という。

\br

機械学習では、単に訓練データでうまくいくようなモデルを見つけるだけでなく、「未知のデータでどれだけうまくいくか」を表す汎化能力をどのように獲得するかが重要な課題となる。

\end{document}
