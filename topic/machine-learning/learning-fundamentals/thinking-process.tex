\documentclass[../../../topic_machine-learning]{subfiles}

\begin{document}

\sectionline
\section{意思決定のプロセス}

経験に基づいて意思決定を行うために人間が用いるプロセスは\keyword{記憶・定式化・予測フレームワーク}と呼ばれ、次の3つのステップで構成されている。

\begin{enumerate}
  \item \textbf{記憶}:過去の同じような状況を思い出す
  \item \textbf{定式化}:全般的なルールを定式化する
  \item \textbf{予測}:このルールを使って将来起こるかもしれないことを予測する
\end{enumerate}

コンピュータに「記憶・定式化・予測」フレームワークを使わせることで、コンピュータに私たちと同じように考えさせることができる。

\begin{enumerate}
  \item \textbf{記憶}:巨大なデータテーブルを調べる
  \item \textbf{定式化}:さまざまなルールや式を調べてデータに最適な\keyword{モデル}を作成する
  \item \textbf{予測}:モデルを使って未来(未知)のデータについて予測を行う
\end{enumerate}

\end{document}
