\documentclass[../../../topic_machine-learning]{subfiles}

\begin{document}

\sectionline
\section{損失関数の設計}

\keywordJE{損失関数}{loss function}は、訓練データで与えられる正解に対し、
\begin{emphabox}
  \begin{spacebox}
    \begin{center}
      予測がどれだけ間違っているのか
    \end{center}
  \end{spacebox}
\end{emphabox}
を表す関数である。

\keywordJE{コスト関数}{cost function}や\keywordJE{誤差関数}{error function}と呼ばれることもある。

\br

損失関数は、入力$\vb*{x}$、正解の出力$y$、モデルパラメータ$\theta$を引数としてとり、0以上の値を返す。
\begin{equation*}
  l(\vb*{x}, y; \theta) \geq 0
\end{equation*}
予測が正しければ0を返し、予測が間違っていれば正の値を返すようにする。

\br

\keyword{学習}は、損失関数によって表される「現在の予測の間違っている度合い」を最小化するようなパラメータを求める\keyword{最適化}問題を解くことによって実現される。

\subsection{損失関数の設計の重要性}

損失関数は、学習の目的に応じて自由に設定することができる。

\br

どのような損失関数を使うかによって、学習結果のモデルがどのような性質を持つかが決まる。

\begin{itemize}
  \item 汎化性能
  \item ノイズに強いか弱いか
  \item 平均的な性能が優れているか、最悪の場合の性能が優れているか
\end{itemize}

たとえば、損失関数が大きく間違っているサンプルを重視するなら、大きな間違いはしないようになる。
その反面、訓練データのノイズ(間違ったラベルがある場合など)に弱くなる。

\subsection{損失関数の微分の形の重要性}

損失関数の微分の性質によって、どのような解に収束するかが変わる。

\end{document}
