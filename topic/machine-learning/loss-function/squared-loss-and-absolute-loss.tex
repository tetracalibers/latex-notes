\documentclass[../../../topic_machine-learning]{subfiles}

\begin{document}

\sectionline
\section{回帰問題:二乗損失・絶対損失}

回帰問題の場合は、\keyword{二乗損失}や\keyword{絶対損失}を損失関数として使うことが一般的である。

\br

\keyword{二乗損失}は、正解データと予測値の差の二乗をとった値である。

\begin{definition}{二乗損失}
  \begin{equation*}
    l_{SE}(\vb*{x}, y; \theta) = \|f(\vb*{x}; \theta) - y\|^2
  \end{equation*}
\end{definition}

\keyword{絶対損失}は、差の絶対値をとった値である。

\begin{definition}{絶対損失}
  \begin{equation*}
    l_{AE}(\vb*{x}, y; \theta) = \|f(\vb*{x}; \theta) - y\|
  \end{equation*}
\end{definition}

二乗や絶対値をとることで、差が負になっても損失として正にできる。

\subsection{最小値からのズレを許しやすいか}

原点付近の様子を見ると、二次関数は最小値からずれても関数の値が急激には増加しない。

絶対値関数の方が、少しでもズレが生じると、関数の値が急に増える。

\br

つまり、二乗損失より絶対損失の方が、最小値からのズレを許しづらくなる。

\subsection{誤差の大きさを重視するか}

二乗損失と絶対損失では、間違えた場合に重視する部分が異なる。

\br

2次関数は、入力が大きくなるにつれ、結果が急激に大きくなる関数である。

そのため、二乗損失は、大きく間違っている場合の誤差が大きくなる。

\br

これに対し、絶対値関数は原点以外は直線(比例)であり、入力が大きくなっても結果は一定の割合で大きくなる関数である。

\br

これらの性質から、
\begin{itemize}
  \item 二乗損失は、誤差が大きい場合を重視して学習する
  \item 絶対損失は、相対的に誤差が小さい場合を重視して学習する
\end{itemize}
ということがいえる。

\end{document}
