\documentclass[../../../topic_machine-learning]{subfiles}

\begin{document}

\sectionline
\section{分類器のマージンと損失関数}

分類器が高い\keyword{確信度}で予測したにもかかわらず間違えた場合は、損失関数は大きな正の値を返すようにする。

\br

確信度は、\keywordJE{マージン}{margin}によって表される。

\begin{definition}{マージン}
  予測結果が境界面からどれだけ離れているか
\end{definition}

境界面より離れている(マージンが大きい)分類結果は、確信を持って「これはXXXである」と予測していることを表す。

逆に、境界面に近い(マージンが小さい)分類結果は、「これはどちらかといえばXXXだが、YYYかもしれない」といったように、確信度が低いことを表している。

\br

マージンが大きいにもかかわらず予測が外れている場合は、今の予測や境界面が適切ではないことを示しており、大きな更新が必要となる。

\end{document}
