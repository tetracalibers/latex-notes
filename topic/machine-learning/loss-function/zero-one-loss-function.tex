\documentclass[../../../topic_machine-learning]{subfiles}

\begin{document}

\sectionline
\section{分類問題と0/1損失関数}

\keyword{0/1損失関数}は、分類問題で用いられる損失関数であり、モデルによる分類が間違っていたら1、正しければ0を返す関数である。

\begin{definition}{0/1損失関数}
  \begin{equation*}
    l_{0/1}(\vb*{x}, y; \theta) = \begin{cases}
      0 & \text{if } f(\vb*{x}; \theta) = y \\
      1 & \text{if } f(\vb*{x}; \theta) \neq y
    \end{cases}
  \end{equation*}
\end{definition}

この損失関数を使えば分類精度を評価できるが、微分がほとんどの位置で0になるため、\keyword{勾配法}を用いた学習では利用できない。

\end{document}
