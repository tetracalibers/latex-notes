\documentclass[../../../topic_machine-learning]{subfiles}

\begin{document}

\sectionline
\section{多項式回帰}
\marginnote{\refbookMA p76〜78 \\ \refbookSA p129〜132}

データがほぼ直線上に並んでいる場合、線形回帰はうまく機能する

しかし、データがより複雑に分布している場合、直線のモデルを当てはめるのは無理がある

\br

そこで、線形回帰の拡張として\keyword{多項式回帰}がある

\br

\todo{数学的な詳細を書く}

\br

多項式回帰モデルの訓練では、訓練プロセスの前に多項式の\keywordJE{次数}{degree}を決めなければならない

\end{document}
