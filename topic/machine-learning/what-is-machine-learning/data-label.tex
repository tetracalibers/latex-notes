\documentclass[../../../topic_machine-learning]{subfiles}

\begin{document}

\sectionline
\section{データと特徴量}
\marginnote{\refbookMA p13、p19}

データがテーブルに含まれている場合、各行はデータ点である

たとえば、動物のデータセットがある場合、各行は異なる動物を表している

\br

このテーブル内の各動物は、その動物の\keyword{特徴量}によって説明される

\begin{definition}{特徴量}
  モデルが予測を行うために使うことができるデータの特性や属性
\end{definition}

データがテーブルに含まれている場合、特徴量はテーブルの列であり、特徴量は各データを説明する

\sectionline
\section{予測とラベル}
\marginnote{\refbookMA p19〜20}

特徴量の中には、\keyword{ラベル}と呼ばれる特別なものがある

一般に、特定の特徴量を他の特徴量に基づいて予測しようとしているなら、その特徴量はラベルである

\br

機械学習モデルの目標は、
\begin{shaded}
  データに含まれている\keyword{ラベル}を推測すること
\end{shaded}
であり、モデルが行う推測を\keyword{予測}と呼ぶ

\sectionline
\section{ラベル付きデータとラベルなしデータ}
\marginnote{\refbookMA p20〜21}

データには、大きく分けて、
\begin{itemize}
  \item \keyword{ラベル付きデータ}:ラベルが付いているデータ
  \item \keyword{ラベルなしデータ}:ラベルが付いていないデータ
\end{itemize}
の2種類がある

\br

予測したいと思うような列を持たないデータセットは、ラベルなしデータである

\br

ラベル付きデータとラベルなしデータは、\keyword{教師あり学習}と\keyword{教師なし学習}という2種類の機械学習を生み出している

\end{document}
