\documentclass[../../../topic_machine-learning]{subfiles}

\begin{document}

\sectionline
\section{意思決定のプロセス}
\marginnote{\refbookMA p8〜9、p15}

経験に基づいて意思決定を行うために人間が用いるプロセスは\keyword{記憶・定式化・予測フレームワーク}と呼ばれ、次の3つのステップで構成されている

\begin{enumerate}
  \item \textbf{記憶}:過去の同じような状況を思い出す
  \item \textbf{定式化}:全般的なルールを定式化する
  \item \textbf{予測}:このルールを使って将来起こるかもしれないことを予測する
\end{enumerate}

コンピュータに「記憶・定式化・予測」フレームワークを使わせることで、コンピュータに私たちと同じように考えさせることができる

\begin{enumerate}
  \item \textbf{記憶}:巨大なデータテーブルを調べる
  \item \textbf{定式化}:さまざまなルールや式を調べてデータに最適な\keyword{モデル}を作成する
  \item \textbf{予測}:モデルを使って未来(未知)のデータについて予測を行う
\end{enumerate}

\sectionline
\section{モデルとアルゴリズム}
\marginnote{\refbookMA p9、p15〜16}

コンピュータはデータを使って\keywordJE{モデル}{model}を構築するという方法で問題を解く

\begin{definition}{モデル}
  データを表すルールの集まりであり、予測を行うために使うことができる
\end{definition}

モデルは、次のようなものと考えることができる
\begin{shaded}
  既存のデータをできる限り厳密に模倣する一連のルールを使って現実を表すもの
\end{shaded}

\br

そして、最適なモデルとは、次のようなものである
\begin{shaded}
  新しいデータに最もうまく\keyword{汎化}するもの
\end{shaded}

\br

最適なモデルを構築するためのさまざまなアルゴリズムがある

\keywordJE{アルゴリズム}{algorithm}は、モデルを構築するために使ったプロセスのこと

\begin{definition}{アルゴリズム}
  問題を解いたり計算を行ったりするために使われる手続き(一連のステップ)
\end{definition}

\end{document}
