\documentclass[../../../topic_multivariable-calculus]{subfiles}

\begin{document}

\sectionline
\section{ある変数に関する偏微分係数}
\marginnote{\refbookB p21〜23}

2変数関数$z = f(x, y)$において、$x$方向の傾きを考えてみる。

このとき、$y$は定数として固定する。
\begin{equation*}
  y = a_2
\end{equation*}

この平面$y=a_2$は、$x$軸と$z$軸に平行な平面である。

この平面で関数のグラフを切り、その切り口に現れた関数のグラフを微分することを考える。

\br

切り口として現れるグラフは、$y = a_2$と$z=f(x, y)$の交線で、
\begin{equation*}
  \begin{cases} 
    x = a_1 \\ 
    z = f(a_1, y) 
  \end{cases}
\end{equation*}
という連立方程式を解いて得られる。

\br

この2式は、代入により次のような形にまとめられ、これが切り口を表している。
\begin{equation*}
  z = f(x, a_2)
\end{equation*}

切り口となる関数$z = f(x, a_2)$の$x = a_1$での接線の傾きが、求めたい$x$方向の傾きである。

この関数は$x$の1変数関数にすぎないので、$x$に関して普通に微分すればよい。

\br

$h$を微小量とし、$x = a_1$から少しだけ移動した点を$x = a_1 + h$とすると、次のように接線の傾きが計算できる。
\begin{equation*}
  \lim_{h \to 0} \frac{f(a_1 + h, a_2) - f(a_1, a_2)}{h}
\end{equation*}
この式を、関数$f(x,y)$の$(a_1,a_2)$における$x$に関する\keyword{偏微分係数}という。

偏微分の場合は、通常の微分記号$\dfrac{d}{dx}$の代わりに、$\dfrac{\partial}{\partial x}$という記号を用いる。
\begin{equation*}
  \dfrac{\partial f}{\partial x}(a_1,a_2) = \lim_{h \to 0} \frac{f(a_1 + h, a_2) - f(a_1, a_2)}{h}
\end{equation*}

\sectionline
\section{ある変数に関する偏導関数}
\marginnote{\refbookB p24}

偏微分係数は、$(a_1,a_2)$という値を1つ決めたときに、$\dfrac{\partial f}{\partial x}(a_1,a_2)$という値が1つ決まるという式である。

2つの値を入力としているので、見方を変えればこれも2変数関数である。

\br

そこで、入力$(a_1,a_2)$を変数$(x,y)$に置き換えて、
\begin{equation*}
  \dfrac{\partial f}{\partial x}(x,y) = \lim_{h \to 0} \frac{f(x + h, y) - f(x, y)}{h}
\end{equation*}
という2変数関数を新たに考える。
これを$x$に関する\keyword{偏導関数}という。

\sectionline
\section{偏微分の記号}
\marginnote{\refbookB p24〜25}

偏導関数の記号にはさまざまな表記法があるが、どれも同じものである。

\begin{description}
  \item [$\dfrac{\partial f(x,y)}{\partial x}$]~\\ 微小量の変化の比
  \item [$f_x(x,y)$]~\\ 微分の省略形$f'(x)$の代わり(何に関する偏微分かを下に添えた)
  \item [$\partial_x f(x,y)$]~\\ 偏微分するという操作を関数に施す
  \item [$\left(\dfrac{\partial f }{\partial x}\right)_y$]~\\ 関数の変数を省略した形(止めている他の変数を下に添えた)
\end{description}

\end{document}
