\documentclass[../../../topic_mapping]{subfiles}

\begin{document}

\sectionline
\section{関数}

\begin{definition}{関数}
  写像$f\colon A \to B$に対して、集合$B$が数の集合のとき、写像$f$を\keyword{関数}と呼ぶ
\end{definition}

関数$y=f(x)$は、
\begin{itemize}
  \item 「関数」としてみれば、「$x$を入力すると$y$が出力される」
  \item 「写像」としてみれば、「$x$に対して$y$を対応させる」
\end{itemize}

\sectionline
\section{関数の単射と全射}

関数が\keyword{単射}であるとは、「同じ値を取るものがない」ということ

たとえば、\keyword{単調増加関数}と\keyword{単調減少関数}は単射

連続関数$f(x)$が単射であるのは、グラフに山や谷がないとき

\sectionline

関数が\keyword{全射}であるとは、関数$f(x)$を$\mathbb{R}$への写像と見なしたとき、$y$軸上に対応する$x$がない点がないということ

\end{document}
