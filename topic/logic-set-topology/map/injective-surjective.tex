\documentclass[../../../topic_mapping]{subfiles}

\begin{document}

\sectionline
\section{単射}
\marginnote{\refbookE p56〜59}

単射とは、
\begin{shaded}
  異なる元は異なる元に写る
\end{shaded}
という性質である

\br

$A$の異なる元が$B$の異なる元に写るとき、写像$f\colon A \to B$は\keyword{単射}であるという

\begin{definition}{単射}
  写像$f\colon A \to B$に対して、$f$が\keyword{単射}であるとは、$A$の任意の要素$a, a'$に対して
  \begin{equation*}
    f(a) = f(a') \implies a = a'
  \end{equation*}
  が成り立つことをいう
\end{definition}

この主張の対偶
\begin{equation*}
  a \ne a' \implies f(a) \ne f(a')
\end{equation*}
を考えれば、\keyword{単射}であるということは、「異なる要素が$f$によって同じ要素に対応することはない」ということにほかならない

\br

単射な写像は、
\begin{shaded}
  写像の定義域を値域にそっくり「コピーする」
\end{shaded}
と考えることができる

\begin{center}
  \begin{tikzpicture}
    \def\rA{1.5cm}       % 左の円の半径
    \def\rB{2cm}         % 右の円の半径
    \def\rCopy{1cm}      % 右の円の中の小円の半径
    \def\dotSize{2pt}    % 点のサイズ
    \def\spaceX{3}       % 中心座標のX距離

    \def\leftX{-\spaceX}
    \def\rightX{\spaceX}

    % 半径2cmの円を2つ横並びに配置
    \draw[gray,thick] (\leftX,0) circle (\rA);
    \draw[gray,thick] (\rightX,0) circle (\rB);

    % 2つの円の間に矢印
    \draw[->, thick] (-1,0) -- (0.5,0) node[midway, above] {\textbf{単射}};

    % 右の円の中にさらに小さな円を配置
    \draw[fill=Rhodamine!40, draw=gray] (\rightX,-0.25) circle (\rCopy);

    % 左の円の中に2つの点を不規則に配置
    \fill (\leftX+0.25,0.5) circle (\dotSize);
    \fill (\leftX,-0.5) circle (\dotSize);

    % 右の円の中の小さな円の中に2つの点を不規則に配置
    \fill (\rightX+0.2,0.25) circle (\dotSize);
    \fill (\rightX,-0.5) circle (\dotSize);

    % 曲がった矢印を2つの点に向けて引く
    \draw[RoyalBlue, |->, very thick, shorten >= 4pt, shorten <= 4pt]
    (\leftX+0.25,0.5) to[out=45,in=135] (\rightX+0.2,0.25);
    \draw[RoyalBlue, |->, very thick, shorten >= 4pt, shorten <= 4pt]
    (\leftX,-0.5) to[out=-45,in=-135] (\rightX,-0.5);

    % ラベル
    \node[yshift=1.5cm] at (\leftX,0.5) {\Large $A$};
    \node[yshift=2cm] at (\rightX,0.5) {\Large $B$};
    \node[yshift=1.5cm, Rhodamine] at (\rightX,-3) {\bfseries $A$のコピー};
  \end{tikzpicture}
\end{center}

\sectionline
\section{全射}
\marginnote{\refbookE p57〜59}

全射とは、
\begin{shaded}
  どんな$b$も$A$の元の像になる
\end{shaded}
という性質である

\br

$B$の任意の元が$A$のある元の像となるとき、写像$f\colon A \to B$は\keyword{全射}であるという

\begin{definition}{全射}
  写像$f\colon A \to B$に対して、$f$が\keyword{全射}であるとは、
  \begin{equation*}
    f(A) = B
  \end{equation*}
  すなわち
  \begin{equation*}
    \forall b \in B, \exists a \in A\colon f(a) = b
  \end{equation*}
  が成り立つことをいう
\end{definition}

言い換えると、$B$への写像$f$が\keyword{全射}であるとは、$B$の要素に「対応していないものがない」ということ

\br

全射な写像は、
\begin{shaded}
  定義域の元の像で値域を「埋め尽くす」
\end{shaded}
と考えることができる

\begin{center}
  \begin{tikzpicture}
    % 半径2cmの円を2つ横並びに配置(少しスペースを空ける)
    \draw[gray,thick] (-3,0) circle (1.5cm);
    \draw[gray,fill=Rhodamine!40,thick] (3,0) circle (2cm);

    % 2つの円の間に矢印
    \draw[->, thick] (-1,0) -- (0.5,0) node[midway, above] {\textbf{全射}};

    % 左の円の中に2つの点を不規則に配置
    \fill (-2.75,0.5) circle (2pt);
    \fill (-3,-0.5) circle (2pt);

    % 右の円の中の小さな円の中に2つの点を不規則に配置
    \fill (3.2,0.25) circle (2pt);
    \fill (3,-0.5) circle (2pt);

    % 曲がった矢印を2つの点に向けて引く
    \draw[RoyalBlue, |->, very thick, shorten >= 4pt, shorten <= 4pt] (-2.75,0.5) to[out=45,in=135] (3.2,0.25);
    \draw[RoyalBlue, |->, very thick, shorten >= 4pt, shorten <= 4pt] (-3,-0.5) to[out=-45,in=-135] (3,-0.5);

    \node[yshift=1.5cm] at (-3,0.5) {\Large $A$};
    \node[yshift=2cm] at (3,0.5) {\Large $B$};
  \end{tikzpicture}
\end{center}

\sectionline
\section{全単射}
\marginnote{\refbookE p57〜59}

全単射とは、
\begin{shaded}
  どんな$B$の元も、ただ1つの$A$の元の像になる
\end{shaded}
という性質である

\begin{definition}{全単射}
  集合$A$から集合$B$への写像$f$が単射かつ全射であるときは、\keyword{全単射}であるという
\end{definition}

これは、写像$f$により、集合$A$の要素と集合$B$の要素が「一対一に対応している」ことにほかならない

\sectionline
\section{同型写像}

数学では、数学的構造を保つ写像が重要であり、特に、構造を保つ全単射写像のことは\keyword{同型写像}と呼ぶ

\sectionline
\section{単射・全射と合成}
\marginnote{\refbookE p59〜61}

単射や全射の性質は、写像の合成に関して閉じている

\br

\begin{theorem}{単射な写像の合成}
  単射な写像の合成は単射である
\end{theorem}

直観的には、
\begin{shaded}
  $C$の中に$A$のコピーのコピーができる
\end{shaded}
という解釈ができる

\begin{center}
  \scalebox{0.6}{
    \begin{tikzpicture}
      \def\rA{1.5cm}
      \def\rB{2cm}
      \def\rC{2.5cm}       % 円Cの半径(Bより大きい)
      \def\rCopyA{1cm}
      \def\rCopyCopyA{1.4cm} % C内の小円の半径(Aのコピー)
      \def\rSubA{0.9cm}
      \def\dotSize{2pt}
      \def\spaceX{6}

      % 各円の中心X座標
      \def\xA{-\spaceX+0.5}
      \def\xB{0}
      \def\xC{\spaceX+1.5*0.5}

      % 円A, B, C
      \begin{scope}[local bounding box=circleA]
        \draw[gray, thick] (\xA,0) circle (\rA);
      \end{scope}
      \begin{scope}[local bounding box=circleB]
        \draw[gray, thick] (\xB,0) circle (\rB);
      \end{scope}
      \begin{scope}[local bounding box=circleC]
        \draw[gray, thick] (\xC,0) circle (\rC);
      \end{scope}

      % ラベル
      \node[yshift=\rA] at (\xA,0.5) {\Large $A$};
      \node[yshift=\rB] at (\xB,0.5) {\Large $B$};
      \node[yshift=\rC] at (\xC,0.5) {\Large $C$}; % y位置も少し上へ

      % B内の小円(Aのコピー)
      \draw[fill=Rhodamine!40, draw=gray,xshift=-0.25\rB,yshift=0.25\rB] (\xB,0) circle (\rCopyA);

      % C内の小円(Aのコピー)とその部分集合
      \draw[fill=Rhodamine!40, draw=gray,xshift=-0.25*\rCopyCopyA,yshift=0.25*\rCopyCopyA] (\xC,0) circle (\rCopyCopyA);
      \draw[fill=Apricot!50, draw=gray,xshift=-0.1*\rSubA,yshift=0.1*\rSubA] (\xC,0) circle (\rSubA);

      % 点(A内)
      \fill (\xA+0.25,0.5) circle (\dotSize);
      % 点(B内)
      \fill (\xB+0.2,0.25) circle (\dotSize);
      % 点(C内の小円)
      \fill (\xC+0.2,0.25) circle (\dotSize);

      % A → B の写像矢印
      \draw[RoyalBlue, |->, very thick, shorten >= 4pt, shorten <= 4pt]
      (\xA+0.25,0.5) to[out=45,in=135] (\xB+0.2,0.25);

      % A → B を結ぶ矢印(直線の上に「単射」ラベル)
      \draw[->, thick, shorten >= 1em, shorten <= 1em] (circleA.east) -- (circleB.west) node[midway, above] {\large\textbf{単射}};
      % B → C を結ぶ矢印(直線の上に「単射」ラベル)
      \draw[->, thick, shorten >= 1em, shorten <= 1em] (circleB.east) -- (circleC.west) node[midway, above] {\large\textbf{単射}};

      % B → C の対応矢印
      \draw[RoyalBlue, |->, very thick, shorten >= 4pt, shorten <= 4pt]
      (\xB+0.2,0.25) to[out=45,in=135] (\xC+0.2,0.25);

      % ラベル(コピー情報)
      \node[Rhodamine] at (\xB,-\rCopyA) {\large\bfseries $A$のコピー};
      \node[Rhodamine] at (\xC,-\rCopyCopyA) {\large\bfseries $A$のコピーのコピー};
    \end{tikzpicture}
  }
\end{center}

\begin{proof}
  \todo{\refbookE p59}
\end{proof}

\br

\begin{theorem}{全射な写像の合成}
  全射な写像の合成は全射である
\end{theorem}

直観的には、
\begin{shaded}
  合成すると$A$の元の像で$C$は埋め尽くされる
\end{shaded}
と解釈できる

\begin{center}
  \scalebox{0.7}{
    \begin{tikzpicture}
      \def\rA{1.5cm}
      \def\rB{1.5cm}
      \def\rC{1.5cm}
      \def\dotSize{2pt}
      \def\spaceX{5}

      % 各円の中心X座標
      \def\xA{0}
      \def\xB{\spaceX}
      \def\xC{\spaceX*2}

      % 円A, B, C
      \begin{scope}[local bounding box=circleA]
        \draw[gray, thick] (\xA,0) circle (\rA);
      \end{scope}
      \begin{scope}[local bounding box=circleB]
        \draw[fill=Rhodamine!20, draw=gray, thick] (\xB,0) circle (\rB);
      \end{scope}
      \begin{scope}[local bounding box=circleC]
        \draw[fill=Rhodamine!40, draw=gray, thick] (\xC,0) circle (\rC);
      \end{scope}

      % ラベル
      \node[yshift=\rA] at (\xA,0.5) {\Large $A$};
      \node[yshift=\rB] at (\xB,0.5) {\Large $B$};
      \node[yshift=\rC] at (\xC,0.5) {\Large $C$}; % y位置も少し上へ

      % 点(A内)
      \fill (\xA+0.25,0.5) circle (\dotSize);
      % 点(B内)
      \fill (\xB+0.2,0.25) circle (\dotSize);
      % 点(C内の小円)
      \fill (\xC+0.2,0.25) circle (\dotSize);

      % A → B の写像矢印
      \draw[RoyalBlue, |->, very thick, shorten >= 4pt, shorten <= 4pt]
      (\xA+0.25,0.5) to[out=45,in=135] (\xB+0.2,0.25);
      % B → C の対応矢印
      \draw[RoyalBlue, |->, very thick, shorten >= 4pt, shorten <= 4pt]
      (\xB+0.2,0.25) to[out=45,in=135] (\xC+0.2,0.25);

      % A → B を結ぶ矢印(直線の上に「単射」ラベル)
      \draw[->, thick, shorten >= 1em, shorten <= 1em] (circleA.east) -- (circleB.west) node[midway, above] {\textbf{全射}};
      % B → C を結ぶ矢印(直線の上に「単射」ラベル)
      \draw[->, thick, shorten >= 1em, shorten <= 1em] (circleB.east) -- (circleC.west) node[midway, above] {\textbf{全射}};

      % ラベル(コピー情報)
      \node[Rhodamine, yshift=-1.5em] at (\xB,-\rB) {\bfseries $A$の元の像で埋め尽くす};
      \node[Rhodamine, yshift=-1.5em] at (\xC,-\rC) {\bfseries $B$の元の像で埋め尽くす};
    \end{tikzpicture}
  }
\end{center}

\begin{proof}
  \todo{\refbookE p60}
\end{proof}

\end{document}
