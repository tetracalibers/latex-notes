\documentclass[../../../topic_mapping]{subfiles}

\begin{document}

\sectionline
\section{単射}
\marginnote{\refbookE p56〜}

単射とは、
\begin{shaded}
  異なる元は異なる元に写る
\end{shaded}
という性質である

\br

$A$の異なる元が$B$の異なる元に写るとき、写像$f\colon A \to B$は\keyword{単射}であるという

\begin{definition}{単射}
  写像$f\colon A \to B$に対して、$f$が\keyword{単射}であるとは、$A$の任意の要素$a, a'$に対して
  \begin{equation*}
    f(a) = f(a') \implies a = a'
  \end{equation*}
  が成り立つことをいう
\end{definition}

この主張の対偶
\begin{equation*}
  a \ne a' \implies f(a) \ne f(a')
\end{equation*}
を考えれば、\keyword{単射}であるということは、「異なる要素が$f$によって同じ要素に対応することはない」ということにほかならない

\br

単射な写像は、
\begin{shaded}
  写像の定義域を値域にそっくり「コピーする」
\end{shaded}
と考えることができる

\begin{center}
  \begin{tikzpicture}
    % 半径2cmの円を2つ横並びに配置(少しスペースを空ける)
    \draw[gray,thick] (-3,0) circle (1.5cm);
    \draw[gray,thick] (3,0) circle (2cm);

    % 2つの円の間に矢印
    \draw[->, thick] (-1,0) -- (0.5,0) node[midway, above] {\textbf{単射}};

    % 右の円の中にさらに小さな円を配置
    \draw[fill=Rhodamine!40, draw=gray] (3,-0.25) circle (1cm);

    % 左の円の中に2つの点を不規則に配置
    \fill (-2.75,0.5) circle (2pt);
    \fill (-3,-0.5) circle (2pt);

    % 右の円の中の小さな円の中に2つの点を不規則に配置
    \fill (3.2,0.25) circle (2pt);
    \fill (3,-0.5) circle (2pt);

    % 曲がった矢印を2つの点に向けて引く
    \draw[RoyalBlue, |->, very thick, shorten >= 4pt, shorten <= 4pt] (-2.75,0.5) to[out=45,in=135] (3.2,0.25);
    \draw[RoyalBlue, |->, very thick, shorten >= 4pt, shorten <= 4pt] (-3,-0.5) to[out=-45,in=-135] (3,-0.5);

    \node[yshift=1.5cm] at (-3,0.5) {\Large $A$};
    \node[yshift=2cm] at (3,0.5) {\Large $B$};

    % 左の円の中の小さな円にもラベルを付ける
    \node[yshift=1.5cm, Rhodamine] at (3,-3) {\bfseries $A$のコピー};
  \end{tikzpicture}
\end{center}

\sectionline
\section{全射}
\marginnote{\refbookE p57〜}

全射とは、
\begin{shaded}
  どんな$b$も$A$の元の像になる
\end{shaded}
という性質である

\br

$B$の任意の元が$A$のある元の像となるとき、写像$f\colon A \to B$は\keyword{全射}であるという

\begin{definition}{全射}
  写像$f\colon A \to B$に対して、$f$が\keyword{全射}であるとは、
  \begin{equation*}
    f(A) = B
  \end{equation*}
  すなわち
  \begin{equation*}
    \forall b \in B, \exists a \in A\colon f(a) = b
  \end{equation*}
  が成り立つことをいう
\end{definition}

言い換えると、$B$への写像$f$が\keyword{全射}であるとは、$B$の要素に「対応していないものがない」ということ

\br

全射な写像は、
\begin{shaded}
  定義域の元の像で値域を「埋め尽くす」
\end{shaded}
と考えることができる

\begin{center}
  \begin{tikzpicture}
    % 半径2cmの円を2つ横並びに配置(少しスペースを空ける)
    \draw[gray,thick] (-3,0) circle (1.5cm);
    \draw[gray,fill=Rhodamine!40,thick] (3,0) circle (2cm);

    % 2つの円の間に矢印
    \draw[->, thick] (-1,0) -- (0.5,0) node[midway, above] {\textbf{全射}};

    % 左の円の中に2つの点を不規則に配置
    \fill (-2.75,0.5) circle (2pt);
    \fill (-3,-0.5) circle (2pt);

    % 右の円の中の小さな円の中に2つの点を不規則に配置
    \fill (3.2,0.25) circle (2pt);
    \fill (3,-0.5) circle (2pt);

    % 曲がった矢印を2つの点に向けて引く
    \draw[RoyalBlue, |->, very thick, shorten >= 4pt, shorten <= 4pt] (-2.75,0.5) to[out=45,in=135] (3.2,0.25);
    \draw[RoyalBlue, |->, very thick, shorten >= 4pt, shorten <= 4pt] (-3,-0.5) to[out=-45,in=-135] (3,-0.5);

    \node[yshift=1.5cm] at (-3,0.5) {\Large $A$};
    \node[yshift=2cm] at (3,0.5) {\Large $B$};
  \end{tikzpicture}
\end{center}

\sectionline
\section{全単射}
\marginnote{\refbookE p57〜}

全単射とは、
\begin{shaded}
  どんな$B$の元も、ただ1つの$A$の元の像になる
\end{shaded}
という性質である

\begin{definition}{全単射}
  集合$A$から集合$B$への写像$f$が単射かつ全射であるときは、\keyword{全単射}であるという
\end{definition}

これは、写像$f$により、集合$A$の要素と集合$B$の要素が「一対一に対応している」ことにほかならない

\sectionline
\section{同型写像}

数学では、数学的構造を保つ写像が重要であり、特に、構造を保つ全単射写像のことは\keyword{同型写像}と呼ぶ

\end{document}
