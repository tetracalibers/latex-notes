\documentclass[../../../topic_mapping]{subfiles}

\begin{document}

\sectionline
\section{逆写像}

\begin{definition}{逆写像}
  写像$f\colon A \to B$が全単射であるとき、対応が一対一であるので、逆向きの対応、すなわち、$B$から$A$への対応を考えることができる

  この対応により定義される写像を$f$の\keyword{逆写像}と呼び、記号で$f^{-1}$と書く
\end{definition}

\sectionline
\section{単射と全射の双対性}

\begin{definition}{左逆写像}
  写像$f\colon A \to B$に対して、写像$g\colon B \to A$が存在して、
  \begin{equation*}
    g \circ f = I_A
  \end{equation*}
  を満たすとき、$g$は$f$の\keyword{左逆写像}であるという
\end{definition}

\begin{definition}{右逆写像}
  写像$f\colon A \to B$に対して、写像$g\colon B \to A$が存在して、
  \begin{equation*}
    f \circ g = I_B
  \end{equation*}
  を満たすとき、$g$は$f$の\keyword{右逆写像}であるという
\end{definition}

\begin{theorem}{全単射の特徴づけ}
  写像$f\colon A \to B$に対して、次の2つは同値になる
  \begin{enumerate}
    \item $f$は全単射である
    \item $f$の左逆写像であり、右逆写像でもある写像が存在する
  \end{enumerate}
\end{theorem}

\sectionline

「逆写像」という観点からみることにより、「単射」と「全射」は双対的な概念であることがわかる

\begin{theorem}{単射の特徴づけ}
  写像$f\colon A \to B$に対して、次の2つは同値になる
  \begin{enumerate}
    \item $f$は単射である
    \item $f$の左逆写像が存在する
  \end{enumerate}
\end{theorem}

\begin{theorem}{全射の特徴づけ}
  写像$f\colon A \to B$に対して、次の2つは同値になる
  \begin{enumerate}
    \item $f$は全射である
    \item $f$の右逆写像が存在する
  \end{enumerate}
\end{theorem}

\end{document}
