\documentclass[../../../topic_machine-learning]{subfiles}

\begin{document}

\sectionline
\section{線形回帰}
\marginnote{\refbookMA p40 \\ \refbookSA p112、p121〜122}

できる限り多くのデータ点の近くを通る直線を求めることで、その直線の式を使って新たなデータを大まかに予測することができる

このような手法を\keyword{線形回帰}という

\br

線形回帰は、次のような手順で行われる

\begin{enumerate}
  \item モデルとして直線や平面を仮定する
  \item 誤差を測る指標(\keyword{誤差関数})を設定する
  \item 誤差が最小になるようにモデルのパラメータを調整する
\end{enumerate}

\sectionline
\section{モデルを表す式}
\marginnote{\refbookSA p122〜123 \\ \refbookMA p42}

線形回帰では、モデルとして一次式を使う

\begin{equation*}
  f(\vb*{x}) = w_0 + \sum_{d=1}^{D} w_d x_d
\end{equation*}

この式では、$D$個の特徴量$x_1,\ldots,x_D$を持つデータを考えている

\br

モデルを表す式において、それぞれの特徴量にかける係数$w_1,\ldots,w_D$を\keywordJE{重み}{weight}と呼ぶ

\br

また、モデルを表す式では、どの特徴量にも結びつかない定数$w_0$がある

この定数を\keywordJE{バイアス}{bias}と呼ぶ

\sectionline
\section{誤差関数}
\marginnote{\refbookMA p65〜 \\ \refbookSA p123}

どんな直線が適合するといえるのか、「データに当てはまる基準」も自分で設定することになる

誤差が少ない、すなわち「当てはまりがよい」ほど小さい値をとるような関数を\keywordJE{誤差関数}{error function}と呼ぶ

\begin{definition}{誤差関数}
  モデルの性能がどれくらいかを明らかにする指標であり、性能が悪いモデルに大きな値を割り当て、性能がよいモデルに小さな値を割り当てる関数
\end{definition}

誤差関数は、\keywordJE{損失関数}{loss function}や\keywordJE{コスト関数}{cost function}、最適化問題としての側面に注目した場合は\keyword{目的関数}と呼ばれることもある

\br

誤差関数を定義する一般的な方法としては、次の2つがある

\begin{itemize}
  \item \keywordJE{絶対誤差}{absolute error}:直線からデータ点までの垂直距離を合計したもの
  \item \keywordJE{二乗誤差}{square error}:直線からデータ点までの垂直距離の二乗を合計したもの
\end{itemize}

線形回帰では、誤差関数を最小にするモデル(直線)を探すことになる

誤差関数として\keyword{二乗誤差}を用いた場合、その最小化問題は\keyword{最小二乗法}と呼ばれる

\end{document}
