\documentclass[../../../topic_machine-learning]{subfiles}

\begin{document}

\sectionline
\section{パラメータとハイパーパラメータ}
\marginnote{\refbookMA p78 \\ \refbookSA p142}

回帰モデルは、\keyword{重み}と\keyword{バイアス}によって定義され、これらはモデルの\keywordJE{パラメータ}{parameter}である

\br

しかし、モデルを訓練する「前に」調整できるつまみは他にもいろいろある

\begin{itemize}
  \item \keywordJE{学習率}{learning rate}:モデルを訓練する前に選択する非常に小さな値で、訓練時にモデルを微小変化させるのに役立つ
  \item \keyword{エポック数}:ループの繰り返し回数
  \item \keyword{次数}(多項式回帰の場合)
\end{itemize}

これらのつまみは\keywordJE{ハイパーパラメータ}{hyperparameter}と呼ばれる

\br

パラメータとハイパーパラメータは、大まかには次のように見分けられる

\begin{itemize}
  \item \keyword{ハイパーパラメータ}:訓練プロセスの「前に」設定する数量
  \item \keyword{パラメータ}:訓練プロセスの「最中に」モデルが作成または変更する数量
\end{itemize}

\end{document}
