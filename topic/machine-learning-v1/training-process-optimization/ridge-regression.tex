\documentclass[../../../topic_machine-learning]{subfiles}

\begin{document}

\sectionline
\section{リッジ回帰}
\marginnote{\refbookMA p99 \\ \refbookSA p139〜140}

正則化項として\keyword{\en{L2}ノルム}を使ってモデルを訓練する場合、そのモデルを\keywordJE{リッジ回帰}{ridge regression}と呼ぶ

「ridge」は山の尾根などを表す英単語であり、誤差関数の形状から名付けられている

\br

リッジ回帰では、「パラメータのノルムは小さいほどよい」という条件を課すことで、モデルの複雑さを抑える

\br

具体的には、ノルムの二乗(\keyword{\en{L2}ノルム})すなわち、自分自身との内積
\begin{equation*}
  \vb*{w}^\top \vb*{w}
\end{equation*}
を正則化項として加える

\br

すると、誤差関数は次のように修正される
\begin{equation*}
  J^{\text{Ridge}}(\vb*{w}) = (\vb*{y} - X\vb*{w})^\top (\vb*{y} - X\vb*{w}) + \lambda \vb*{w}^\top \vb*{w}
\end{equation*}

\end{document}
