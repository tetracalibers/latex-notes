\documentclass[../../../topic_machine-learning]{subfiles}

\begin{document}

\sectionline
\section{教師あり学習}
\marginnote{\refbookMA p21}

\keywordJE{教師あり学習}{supervised learning}は、ラベル付きデータを扱う機械学習であり、その目標は\keyword{ラベル}を予測すること

ラベルが付いていない新しいデータが渡された場合、教師あり学習モデルはそのデータ点のラベルを予測する

\br

意思決定を行うためのフレームワーク「記憶・定式化・予測」は、教師あり学習の仕組みそのもの

\begin{enumerate}
  \item データセットを記憶する
  \item 特徴と考えられるものをモデル(ルール)として定式化する
  \item 新しいデータが与えられたときに、そのデータのラベルを予測する
\end{enumerate}

\sectionline
\section{回帰モデルと分類モデル}
\marginnote{\refbookMA p22〜25}

教師あり学習モデルでは、数値と状態の2種類のデータが使われる

\begin{itemize}
  \item \keyword{数値データ}:数値を用いるあらゆる種類のデータ
  \item \keyword{カテゴリ値データ}:カテゴリ(状態)を用いるあらゆる種類のデータ
\end{itemize}

\br

そして、この2種類のデータから、次の2種類の機械学習モデルが生まれた

\begin{itemize}
  \item \keyword{回帰モデル}:数値データを予測する機械学習モデル
  \item \keyword{分類モデル}:カテゴリ値データを予測する機械学習モデル
\end{itemize}

\br

\keywordJE{回帰モデル}{regression model}は数値のラベルを予測するモデルであり、この数値を特徴量に基づいて予測する

\br

\keywordJE{分類モデル}{classification model}は状態の有限集合に含まれている状態を予測するモデルである(カテゴリ値データでは、各データ点に有限のカテゴリ集合が紐づけられる)

\end{document}
