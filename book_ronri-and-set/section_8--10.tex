\documentclass[../book_ronri-and-set]{subfiles}

\begin{document}

\sectionline
\section{集合}

\keyword{集合}とは「ものの集まり」のことであり、その「ものの集まり」に入っているか、あるいは、入っていないかが客観的に判断できるもの

\sectionline
\section{集合の要素}

集合を構成する個々の「もの」を、その集合の\keyword{要素}あるいは\keyword{元}と呼ぶ

\br

$x$が集合$A$の要素であるとき、$x$は$A$に\keyword{含まれる}、あるいは\keyword{属する}と言い、記号では$x \in A$と書く

\sectionline
\section{集合の表記法}

次のような2つの方法がある
\begin{itemize}
  \item $\{x_1, x_2, \cdots \}$(集合を書き並べる方法:\keyword{外延的記法})
  \item $\{x \mid x \text{は条件〜を満たす}\}$(要素になる条件を書く方法:\keyword{内包的記法})
\end{itemize}

集合では、このように、要素を括弧$\{\}$で囲んで記述する

\sectionline
\section{集合の「等しい」}

集合$A$と集合$B$が\keyword{等しい}とは、
\begin{shaded*}
  $A$の要素がすべて$B$の要素であり、かつ、$B$の要素がすべて$A$の要素である
\end{shaded*}
ことを言う

集合$A$と集合$B$が等しいとき、$A = B$と書く

\sectionline
\section{有限集合と無限集合}

集合に含まれる要素の個数が有限個のとき\keyword{有限集合}といい、無限個のとき\keyword{無限集合}と呼ぶ

\sectionline
\section{空集合}

「要素が1つもない集まり」も、1つの集合とみなして、\keyword{空集合}と呼び、記号$\emptyset$で表す

\end{document}
