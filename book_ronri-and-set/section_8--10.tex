\documentclass[../book_ronri-and-set]{subfiles}

\begin{document}

\sectionline
\section{集合}

\keyword{集合}とは「ものの集まり」のことであり、その「ものの集まり」に入っているか、あるいは、入っていないかが客観的に判断できるもの

\sectionline
\section{集合の要素}

集合を構成する個々の「もの」を、その集合の\keyword{要素}あるいは\keyword{元}と呼ぶ

\br

$x$が集合$A$の要素であるとき、$x$は$A$に\keyword{含まれる}、あるいは\keyword{属する}と言い、記号では$x \in A$と書く

\sectionline
\section{集合の表記法}

次のような2つの方法がある
\begin{itemize}
  \item $\{x_1, x_2, \cdots \}$(集合を書き並べる方法:\keyword{外延的記法})
  \item $\{x \mid x \text{は条件〜を満たす}\}$(要素になる条件を書く方法:\keyword{内包的記法})
\end{itemize}

集合では、このように、要素を括弧$\{\}$で囲んで記述する

\sectionline
\section{集合の「等しい」}

集合$A$と集合$B$が\keyword{等しい}とは、
\begin{shaded*}
  $A$の要素がすべて$B$の要素であり、かつ、$B$の要素がすべて$A$の要素である
\end{shaded*}
ことを言う

集合$A$と集合$B$が等しいとき、$A = B$と書く

\sectionline
\section{有限集合と無限集合}

集合に含まれる要素の個数が有限個のとき\keyword{有限集合}といい、無限個のとき\keyword{無限集合}と呼ぶ

\sectionline
\section{空集合}

「要素が1つもない集まり」も、1つの集合とみなして、\keyword{空集合}と呼び、記号$\emptyset$で表す

\sectionline
\section{部分集合}

2つの集合$A$と$B$に対して、$A$は$B$の\keyword{部分集合}である($A$は$B$に\keyword{含まれる})とは、
\begin{shaded*}
  $A$のすべての要素が$B$の要素になっている
\end{shaded*}
ことを言い、記号では$A \subset B$と書く

\br

「$A \subset B$かつ$B \subset A$である」ことは、$A = B$であることに他ならない

\sectionline
\section{共通部分}

いくつかの集合があったとき、それらの「共通の部分」、すなわち、
\begin{shaded*}
  それらの共通の要素を集めてできた集合
\end{shaded*}
のことを\keyword{共通部分}という

共通部分には$\cap$という記号が用いられる

\sectionline

たとえば、2つの集合$A,\,B$に対して、$A$と$B$のどちらにも含まれている要素の全体からなる集合を$A$と$B$の\keyword{共通部分}と呼び、記号では$A \cap B$と書く

\br

すなわち、
\begin{equation*}
  A \cap B = \{x \mid x \in A \land x \in B\}
\end{equation*}

\sectionline

有限個の集合でも同様に、集合$A_1, A_2, \cdots , A_n$に対して、すべての$A_i$に含まれている要素の全体からなる集合を、集合$A_1, A_2, \cdots , A_n$の\keyword{共通部分}と呼び、記号で
\begin{equation*}
  A_1 \cap A_2 \cap \cdots \cap A_n \quad \text{あるいは} \quad \bigcap_{i=1}^n A_i
\end{equation*}
と書く

\br

すなわち、
\begin{align*}
  A_1 \cap A_2 \cap \cdots \cap A_n
   & = \{x \mid \forall A_i , c \in A_i \}               \\
   & = \{x \mid x \in A_1 \land \cdots \land x \in A_n\}
\end{align*}

\sectionline

いくつかの集合があって、それらのどの2つも共通部分をもたないとき、それらは\keyword{互いに素}であるという

\sectionline
\section{和集合}

いくつかの集合があったとき、
\begin{shaded*}
  それらの集合をすべて集めてできた集合
\end{shaded*}
のことを\keyword{和集合}という

和集合には$\cup$という記号が用いられる

\sectionline

たとえば、2つの集合$A,\,B$に対して、$A$と$B$のどちらかに含まれている要素の全体からなる集合を$A$と$B$の\keyword{和集合}と呼び、記号では$A \cup B$と書く

\br

すなわち、
\begin{equation*}
  A \cup B = \{x \mid x \in A \lor x \in B\}
\end{equation*}

\sectionline

有限個の集合でも同様に、集合$A_1, A_2, \cdots , A_n$に対して、ある$A_i$に含まれている要素の全体からなる集合を、集合$A_1, A_2, \cdots , A_n$の\keyword{和集合}と呼び、記号で
\begin{equation*}
  A_1 \cup A_2 \cup \cdots \cup A_n \quad \text{あるいは} \quad \bigcup_{i=1}^n A_i
\end{equation*}
と書く

\br

すなわち、
\begin{align*}
  A_1 \cup A_2 \cup \cdots \cup A_n
   & = \{x \mid \exists A_i , c \in A_i \}             \\
   & = \{x \mid x \in A_1 \lor \cdots \lor x \in A_n\}
\end{align*}

\sectionline
\section{集合と論理の間の対応関係}

「集合」と「論理」は対応しているため、論理で登場した法則は集合に対しても成り立つ

\begin{oframed}
  \paragraph{冪等法則}
  \begin{align*}
    A \cap A & = A \\
    A \cup A & = A
  \end{align*}
\end{oframed}

\begin{oframed}
  \paragraph{交換法則}
  \begin{align*}
    A \cap B & = B \cap A \\
    A \cup B & = B \cup A
  \end{align*}
\end{oframed}

\begin{oframed}
  \paragraph{結合法則}
  \begin{align*}
    A \cap (B \cap C) & = (A \cap B) \cap C \\
    A \cup (B \cup C) & = (A \cup B) \cup C
  \end{align*}
\end{oframed}

\begin{oframed}
  \paragraph{分配法則}
  \begin{align*}
    A \cap (B \cup C) & = (A \cap B) \cup (A \cap C) \\
    A \cup (B \cap C) & = (A \cup B) \cap (A \cup C)
  \end{align*}
\end{oframed}

\begin{oframed}
  \paragraph{吸収法則}
  \begin{align*}
    A \cap (A \cup B) & = A \\
    A \cup (A \cap B) & = A
  \end{align*}
\end{oframed}

たとえば、交換法則の証明は次のようになる
\begin{equation}
  \begin{WithArrows}
    & \phantom{=} A \cap B \Arrow{$\cap$の定義}\\
    & =\{x \mid x \in A \land x \in B\} \Arrow{論理の交換法則} \\
    &=\{x \mid x \in B \land x \in A\} \Arrow{$\cap$の定義} \\
    & = B \cap A
  \end{WithArrows}
\end{equation}

この証明を見てみると、「集合の性質」と「論理の性質」が対応していることがわかる

\sectionline

「集合」というのは、内包的記法により
\begin{equation*}
  \text{集合} = \{x \mid x \text{は〜である}\}
\end{equation*}
という形で表現できるが、「$x$は〜である」というのは、「論理」の命題関数である

\br

すなわち、命題関数$p(x)$を用いて、
\begin{equation*}
  \text{集合} = \{x \mid p(x)\}
\end{equation*}
と書ける

\br

このとき、$\cap$と$\cup$の定義から、
\begin{align*}
  \{x \mid p(x)\} \cap \{x \mid q(x)\} & = \{x \mid p(x) \land q(x)\} \\
  \{x \mid p(x)\} \cup \{x \mid q(x)\} & = \{x \mid p(x) \lor q(x)\}
\end{align*}
となる

\br

さらに、次の2つの主張は同値である
\begin{itemize}
  \item $p(x) \equiv q(x)$
  \item $\{x \mid p(x)\} = \{x \mid q(x)\}$
\end{itemize}

\br

もっと一般に、次の2つの主張が同値であることが確かめられる
\begin{itemize}
  \item $p(x) \Rightarrow q(x)$
  \item $\{x \mid p(x)\} \subset \{x \mid q(x)\}$
\end{itemize}

\end{document}
