\documentclass[../book_ronri-and-set]{subfiles}

\begin{document}

\sectionline
\section{命題関数}

これまで、たとえば「1234567891は素数である」というような\keyword{命題}を扱ってきた

ここで、たとえば$x$が自然数全体を動くとき、「$x$は素数である」という形の主張を\keyword{命題関数}と呼ぶ

\br

命題は記号$p$で表されたのに対し、命題関数は$p(x)$と書く

命題関数$p(x)$は、$x$の値に応じて主張が変わり、真理値が変化していく

\br

命題関数$p(x)$の$x$は、\keyword{変数}と呼ばれる

命題関数$p(x)$の変数は、実数や自然数のような数以外に、直線とか地図のような数学的対象や一般的概念をとる

\sectionline
\section{すべての〜}

命題関数$p(x)$に対して、「すべての$x$について$p(x)$である」という命題を
\begin{equation*}
  \forall x p(x)
\end{equation*}
と表す

\br

「すべての〜について〇〇である」は、
\begin{itemize}
  \item 「すべての〜は〇〇である」
  \item 「任意の〜について〇〇である」
  \item 「任意の〜は〇〇である」
\end{itemize}
と表すこともある

\br

$\forall$という記号は、「all(すべての〜)」や「any(任意の〜)」の頭文字のAを逆さにしたものに由来する

\sectionline

変数$x$が$x= a_1, a_2, \cdots , a_n$という有限個の値をとるとき、「すべての$x$について$p(x)$である」というのは、
\begin{shaded*}
  $p(a_1)$であり、かつ、$p(a_2)$であり、かつ、$\cdots$、かつ、$p(a_n)$である
\end{shaded*}
ということに他ならない

言い換えると、
\begin{equation*}
  \forall x p(x) = p(a_1) \land p(a_2) \land \cdots \land p(a_n)
\end{equation*}
ということになる

\sectionline
\section{ある〜}

命題関数$p(x)$に対して、「ある$x$について$p(x)$である」という命題を
\begin{equation*}
  \exists x p(x)
\end{equation*}
と表す

\br

「ある〜について〇〇である」は、
\begin{itemize}
  \item 「ある〜は〇〇である」
  \item 「ある〜が存在して〇〇である」
  \item 「〇〇であるような〜が存在する」
\end{itemize}
と表すこともある

\br

$\exists$という記号は、「exists(存在する)」の頭文字のEを逆さにしたものに由来する

\sectionline

変数$x$が$x= a_1, a_2, \cdots , a_n$という有限個の値をとるとき、「ある$x$について$p(x)$である」というのは、
\begin{shaded*}
  $p(a_1)$であるか、あるいは、$p(a_2)$であるか、あるいは、$\cdots$、あるいは、$p(a_n)$である
\end{shaded*}
ということに他ならない

言い換えると、
\begin{equation*}
  \exists x p(x) = p(a_1) \lor p(a_2) \lor \cdots \lor p(a_n)
\end{equation*}
ということになる

\sectionline
\section{「すべての〜」と「ある〜」}

「すべての〜」と「ある〜」の2つの概念の間には\keyword{双対性}がある

\begin{align*}
  \forall x p(x) & = p(a_1) \land p(a_2) \land \cdots \land p(a_n) \\
  \exists x p(x) & = p(a_1) \lor p(a_2) \lor \cdots \lor p(a_n)
\end{align*}
という式を比較してみると、「すべての〜($\forall$)」と「ある〜($\exists$)」は、AND($\land$)とOR($\lor$)の双対性を反映していることがわかる

\sectionline
\section{$\forall$と$\exists$を含んだ式の同値変形}

\begin{oframed}
  \paragraph{$\forall$と$\exists$の性質}
  \begin{align*}
    \forall x (p(x) \land q(x)) & \equiv \forall x p(x) \land \forall x q(x) \\
    \exists x (p(x) \lor q(x))  & \equiv \exists x p(x) \lor \exists x q(x)
  \end{align*}
\end{oframed}

これらはそれぞれ、
\begin{itemize}
  \item 「すべての〜」($\forall x$)とAND($\land$)
  \item 「ある〜」($\exists x$)とOR($\lor$)
\end{itemize}
が対応していると思って眺めるとよい

\sectionline
\section{$\forall$と$\exists$の否定}

「すべての〜」($\forall$)と「ある〜」($\exists$)を含む命題の否定は、次の\keyword{ド・モルガンの法則}で与えられる

\begin{oframed}
  \paragraph{ド・モルガンの法則(述語論理)}
  \begin{align*}
    \neg \forall x p(x) & \equiv \exists x \neg p(x) \\
    \neg \exists x p(x) & \equiv \forall x \neg p(x)
  \end{align*}
\end{oframed}

$\neg \forall x p(x) \equiv \exists x \neg p(x)$より、
\begin{shaded*}
  「すべての〜について…である」の否定は、「ある〜について…でない」
\end{shaded*}

$\neg \exists x p(x) \equiv \forall x \neg p(x)$より、
\begin{shaded*}
  「ある〜について…である」の否定は、「すべての〜について…でない」
\end{shaded*}

要するに、否定をとると、「すべての〜」は「ある〜」になり、「ある〜」は「すべての〜」になる

\sectionline

述語論理のド・モルガンの法則は、命題論理のド・モルガンの法則の一般化になっている

\br

$x$が$x = a_1, a_2, \cdots , a_n$というように、有限個の値しかとらない場合、
\begin{align*}
  \forall x p(x) & = p(a_1) \land p(a_2) \land \cdots \land p(a_n) \\
  \exists x p(x) & = p(a_1) \lor p(a_2) \lor \cdots \lor p(a_n)
\end{align*}
であり、
\begin{align*}
  \forall x \neg p(x) & = \neg p(a_1) \land \neg p(a_2) \land \cdots \land \neg p(a_n) \\
  \exists x \neg p(x) & = \neg p(a_1) \lor \neg p(a_2) \lor \cdots \lor \neg p(a_n)
\end{align*}
であるので、述語論理のド・モルガンの法則は、それぞれ次のように書き換えられる
\begin{multline}
  \neg (p(a_1) \land p(a_2) \land \cdots \land p(a_n)) \\ \equiv \neg p(a_1) \lor \neg p(a_2) \lor \cdots \lor \neg p(a_n)
\end{multline}
\begin{multline}
  \neg (p(a_1) \lor p(a_2) \lor \cdots \lor p(a_n)) \\ \equiv \neg p(a_1) \land \neg p(a_2) \land \cdots \land \neg p(a_n)
\end{multline}
これらはそれぞれ、命題論理のド・モルガンの法則の一般化になっていることは一目瞭然

\end{document}
