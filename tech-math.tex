\documentclass[b5paper]{book}

\title{画像処理・AIのための数学ノート}
\author{tomixy}

% === color ===

\usepackage[dvipsnames]{xcolor}

\definecolor{myPurple}{HTML}{CDC1FF}

% === extend ===

\usepackage{xparse}%オプション(省略可能引数)をもつコマンド作成

% === math ===

\usepackage{mathtools}
% 別の場所で参照する数式以外は番号が付かないように
\mathtoolsset{showonlyrefs=true}

\DeclareMathOperator{\Arg}{Arg}

% === font ===

\usepackage[T1]{fontenc}

% normal font, math font
%\usepackage[light,math]{anttor}
\usepackage{gfsartemisia}

% monospace font
\usepackage[scaled]{beramono}

% === layout ===

\usepackage{setspace} % 行間調整用

\usepackage[margin=15mm]{geometry} % 余白
\renewcommand{\baselinestretch}{1.5} % 行間

\setlength{\parindent}{0pt} % 段落始めでの字下げをしない

\usepackage{here} % figure環境のHオプション(画像を強制的にその場に表示する)を提供

% === tikz ===

\usepackage[dvipdfmx]{graphicx}

\usepackage{pgfplots}
\pgfplotsset{compat=1.10}
\usepgfplotslibrary{fillbetween}

\usepackage{tikz} %図を描く
\usetikzlibrary{
  positioning,
  intersections,
  calc,
  arrows.meta,
  math,
  angles,
  quotes,
  bending,
  3d,
  decorations.pathmorphing,
  decorations.pathreplacing,
  patterns,
  shadows,
  shapes.geometric,
  spy
}
\tikzstyle{axis}=[->, >=Stealth]
\tikzstyle{vector}=[->,>=Stealth,very thick, line cap=round]
\tikzstyle{auxline}=[dashed, thick] % 補助線
\tikzstyle{plotline}=[blue,very thick] % グラフ
\tikzstyle{zy-plane}=[canvas is zy plane at x=0]
\tikzstyle{xz-plane}=[canvas is xz plane at y=0]
\tikzstyle{xy-plane}=[canvas is xy plane at z=0]

\usepackage{tikz-3dplot}

\usepackage{witharrows} % 数式の間に矢印で説明を入れる
\usepackage{annotate-equations}

\usepackage[outline]{contour} % 文字の縁取り

\usepackage{froufrou} % セクションを区切る装飾
\setfroufrou{dinkus}

% === tcolorbox ===

\usepackage{tcolorbox}
\tcbuselibrary{listings,breakable,xparse,skins,hooks,theorems}

\DeclareTColorBox{definition}{m O{} }%
{
  enhanced,
  colframe=white,
  colback=yellow!10!white,
  coltitle=cyan!40!black, 
  fonttitle=\bfseries,
  breakable=true,
  underlay unbroken={
    \begin{tcbclipinterior}
      \shade[inner color=cyan!80!yellow,outer color=yellow!10!white] (interior.north east) circle (2cm);
      \draw[help lines,step=5mm,yellow!80!black,shift={(interior.north west)}] (interior.south west) grid (interior.north east);
    \end{tcbclipinterior}
  },
  underlay first={
    \begin{tcbclipinterior}
      \shade[inner color=cyan!80!yellow,outer color=yellow!10!white] (interior.north east) circle (2.5cm);
      \draw[help lines,step=5mm,yellow!80!black,shift={(interior.north west)}] (interior.south west) grid (interior.north east);
    \end{tcbclipinterior}},
  underlay last={
    \begin{tcbclipinterior}
      \draw[help lines,step=5mm,yellow!80!black,shift={(interior.north west)}] (interior.south west) grid (interior.north east);
    \end{tcbclipinterior}
  },
  title={#1},
  attach title to upper=\quad,
  bottom=5mm,
  #2
}

%付箋環境~桃色バージョン~の定義
\DeclareTColorBox{theorem}{m O{} }%
{
  enhanced,
  colframe=white,
  colback=yellow!10!white,
  coltitle=magenta!40!black, 
  fonttitle=\bfseries,
  breakable,
  underlay unbroken={
    \begin{tcbclipinterior}
      \shade[inner color=magenta!80!yellow,outer color=yellow!10!white] (interior.north east) circle (2cm);
      \draw[help lines,step=5mm,yellow!80!black,shift={(interior.north west)}] (interior.south west) grid (interior.north east);
    \end{tcbclipinterior}
  }, 
  underlay first={
    \begin{tcbclipinterior}
      \shade[inner color=magenta!80!yellow,outer color=yellow!10!white] (interior.north east) circle (2.5cm);
      \draw[help lines,step=5mm,yellow!80!black,shift={(interior.north west)}] (interior.south west) grid (interior.north east);
    \end{tcbclipinterior}
  },
  underlay last={
    \begin{tcbclipinterior}
      \draw[help lines,step=5mm,yellow!80!black,shift={(interior.north west)}] (interior.south west) grid (interior.north east);
    \end{tcbclipinterior}
  },
  title={#1},
  attach title to upper=\quad, 
  bottom=5mm,
  #2
}

\definecolor{ProofColor}{HTML}{fbfbfb}
\DeclareTColorBox{proof}{O{}}{
  enhanced,
  title={Proof},
  coltitle=magenta!60!white,
  attach boxed title to top left={
    xshift=0mm, yshift*=-\tcboxedtitleheight/2,
  },
  colback={ProofColor},
  boxed title style={
    colback=ProofColor,
    boxrule=0pt,
    frame hidden,
  },
  boxrule=0pt,
  frame hidden,
  left=5mm,
  breakable,
  overlay={
    \draw[line width=5pt,magenta!40!white,opacity=0.2]
      ([shift={(3mm,-\tcboxedtitleheight/2)}]interior.north west) 
        -- ([shift={(3mm,\tcboxedtitleheight/2)}]interior.south west);
    }
}

\definecolor{mathLabelColor}{HTML}{FF87CA}
\NewDocumentCommand{\labelmath}{O{mathLabelColor!80!gray} O{red!10} m m}{
  \tcboxmath[
    enhanced,
    frame hidden,
    colback={#2},
    opacityback=0.6,
    title={#4},
    coltitle={#1},
    fonttitle=\bfseries,
    attach boxed title to top right={yshift=-0.5mm},
    boxed title style={empty, top=0mm, bottom=0mm}
  ]{#3}
}
% 余白少なめバージョン
\NewDocumentCommand{\fitLabelMath}{O{mathLabelColor!80!gray} O{red!10} O{0.6} m m}{
  \tcboxmath[
    enhanced,
    frame hidden,
    colback={#2},
    opacityback={#3},
    title={#5},
    coltitle={#1},
    fonttitle=\bfseries,
    left=1pt, right=1pt, top=1pt, bottom=1pt,
    attach boxed title to top center={yshift=-0.5mm},
    boxed title style={empty, top=0mm, bottom=0mm}
  ]{#4}
}

\definecolor{reviewBgColor}{HTML}{F9F9F9}
\definecolor{reviewFgColor}{HTML}{6886C5}
\newtcolorbox{review}{
  colback=reviewBgColor,
  colbacktitle=reviewFgColor,
  title=REVIEW,
  enhanced,
  attach boxed title to top left={yshift=-0.18cm,xshift=-0.5mm},
  boxed title style={
    tikz={rotate=4,transform shape},
    frame code={
      \draw[decorate, reviewFgColor,decoration={random steps,segment length=2mm,amplitude=0.6pt},fill=lightgray!20] (frame.south west) rectangle (frame.north east);
    }
  },
frame code={
  \draw[decorate, reviewFgColor!50,decoration={random steps,segment length=2mm,amplitude=0.6pt}] (frame.north east) rectangle (frame.south west);
},
}

% === shotcut ===

% 偶関数に背景色とラベルをつける
\NewDocumentCommand{\evenFn}{O{0.6} m}{
  \fitLabelMath[pink][pink][#1]{#2}{偶}
}
% 奇関数に背景色とラベルをつける
\NewDocumentCommand{\oddFn}{O{0.6} m}{
  \fitLabelMath[cyan][cyan!30][#1]{#2}{奇}
}

% 元の関数と導関数の区別をつける
\NewDocumentCommand{\origFn}{O{0.6} m}{
  \fitLabelMath[cyan][cyan!30][#1]{#2}{元の関数}
}
\NewDocumentCommand{\derivFn}{O{0.6} m}{
  \fitLabelMath[magenta!70][magenta!30][#1]{#2}{導関数}
}

% ---

\begin{document}

\maketitle
\tableofcontents

\chapter{基本的な関数}

\section{指数関数}

\subsection{同じ数のかけ算の指数による表記}

\begin{definition}{指数と底}
  \newline
  同じ数$a$を$n$回掛けたものを$a$の$n$乗といい,$a^n$と表す。
  \LARGE
  \begin{equation}
    a^n = \underbrace{a \times a \times \cdots \times a}_{n\text{個の}a}
  \end{equation}
  \normalsize
  このとき、$n$を指数、$a$を底という。
\end{definition}

\subsection{指数法則}

指数を「かける回数」と捉えれば、いくつかの法則が当たり前に成り立つことがわかる。

\subsubsection{「かける回数」の和}

例えば、$a$を$m$回かけてから、続けて$a$を$n$回かける式を書いてみると、$a$は$m+n$個並ぶことになる。

\begin{equation}
  \overbrace{a\times a\times a}^{a^3} \times \overbrace{a\times a}^{a^2} = \overbrace{a\times a\times a\times a\times a}^{a^5}
\end{equation}

\begin{theorem}{指数の和に関する指数法則}
  \LARGE
  \begin{equation}
    a^m \times a^n = a^{m+n}
  \end{equation}
\end{theorem}

\subsubsection{「かける回数」の差}

例えば、$a$を$m$回かけたものを、$a$を$n$回かけたもので割ると、$m-n$個の$a$の約分が発生する。

\begin{equation}
  \dfrac{\overbrace{a\times a\times a\times a\times a\times}^{a^5}}{\underbrace{a\times a}_{a^2}} = \overbrace{a\times a\times a}^{a^3}
\end{equation}

\begin{theorem}{指数の差に関する指数法則}
  \LARGE
  \begin{equation}
    \dfrac{a^m}{a^n} = a^{m-n}
  \end{equation}
\end{theorem}

\subsubsection{「かける回数」の積}

例えば、「$a$を$m$回かけたもの」を$n$回かける式を書いてみると、$a$は$m \times n$個並ぶことになる。

\begin{equation}
  (a^2)^3 = \underbrace{\overbrace{a\times a}^{a^2} \times \overbrace{a\times a}^{a^2} \times \overbrace{a\times a}^{a^2}}_{a^6}
\end{equation}

\begin{theorem}{指数の積に関する指数法則}
  \LARGE
  \begin{equation}
    (a^m)^n = a^{mn}
  \end{equation}
\end{theorem}

\subsection{指数の拡張と指数関数}

底を固定して、指数を変化させる関数を考えたい。

指数部分に入れられる数を拡張したいが、このとき、どんな数を入れても指数法則が成り立つようにしたい。

\subsubsection{$0$の指数}

指数法則$a^m \times a^n = a^{m+n}$において、$m=0$の場合を考える。

\begin{align}
  a^0 \times a^n & = a^{0+n} \\
  a^0 \times a^n & = a^n
\end{align}

この式が成り立つためには、$a^0$は$1$である必要がある。

\begin{definition}{$0$の指数}
  \newline
  どんな数も、$0$乗すると$1$になると定義する。
  \LARGE
  \begin{equation}
    a^0 = 1
  \end{equation}
\end{definition}

そもそも、指数法則$a^m \times a^n = a^{m+n}$は、「指数の足し算が底のかけ算に対応する」ということを表している。

\begin{itemize}
  \item 「何もしない」足し算は$+ 0$
  \item 「何もしない」かけ算は$\times 1$
\end{itemize}

なので、$a^0 = 1$は「何もしない」という観点で足し算とかけ算を対応づけたものといえる。

\subsubsection{負の指数}

指数法則$a^m \times a^n = a^{m+n}$において、正の数$n$を負の数$-n$に置き換えたものを考える。

\begin{equation}
  a^m \times a^{-n} = a^{m-n}
\end{equation}

さらに、指数法則$\dfrac{a^m}{a^n} = a^{m-n}$も成り立っていてほしいので、

\begin{equation}
  a^m \times a^{-n} = \dfrac{a^m}{a^n}
\end{equation}

この式は、$a^{-n}= \dfrac{1}{a^n}$とすれば、当たり前に成り立つものとなる。

\begin{definition}{負の整数の指数}
  \newline
  $n$が正の整数であるとき、$-n$乗を次のように定義する。
  \LARGE
  \begin{equation}
    a^{-n} = \dfrac{1}{a^n}
  \end{equation}
\end{definition}

\subsubsection{有理数の指数}

指数法則$a^m \times a^n = a^{m+n}$において、指数$m, n$を$\dfrac{1}{2}$に置き換えたものを考える。

\begin{equation}
  a^{\frac{1}{2}} \times a^{\frac{1}{2}} = a^{\frac{1}{2} + \frac{1}{2}} = a
\end{equation}

$a^{\frac{1}{2}} \times a^{\frac{1}{2}}$は、$(a^{\frac{1}{2}})^2$とも書けるので、

\begin{equation}
  (a^{\frac{1}{2}})^2 = a
\end{equation}

つまり、$a^{\frac{1}{2}}$は、2乗すると$a$になる数($a$の平方根)でなければならない。

\begin{equation}
  a^{\frac{1}{2}} = \sqrt{a}
\end{equation}

同様に、$a^{\frac{1}{3}} \times a^{\frac{1}{3}} \times a^{\frac{1}{3}}$を考えてみると、

\begin{equation}
  a^{\frac{1}{3}} \times a^{\frac{1}{3}} \times a^{\frac{1}{3}} = a^{\frac{1}{3} + \frac{1}{3} + \frac{1}{3}} = a
\end{equation}

$a^{\frac{1}{3}} \times a^{\frac{1}{3}} \times a^{\frac{1}{3}}$は、$(a^{\frac{1}{3}})^3$とも書けるので、

\begin{equation}
  (a^{\frac{1}{3}})^3 = a
\end{equation}

つまり、$a^{\frac{1}{3}}$は、3乗すると$a$になる数($a$の3乗根)でなければならない。

\begin{equation}
  a^{\frac{1}{3}} = \sqrt[3]{a}
\end{equation}

このようにして、$a^{\frac{1}{n}}$は、$n$乗すると$a$になる数($a$の$n$乗根)として定義すればよい。

\begin{equation}
  a^{\frac{1}{n}} = \sqrt[n]{a}
\end{equation}

さて、分子が$1$ではない場合はどうだろうか?

$(a^m)^n = a^{mn}$において、$m$を$\dfrac{m}{n}$に置き換えたものを考えると、

\begin{equation}
  (a^{\frac{m}{n}})^n = a^{\frac{m}{n} \times n} = a^m
\end{equation}

となるので、$a^{\frac{m}{n}}$は、$n$乗したら$a^m$になる数として定義すればよい。

\begin{equation}
  a^{\frac{m}{n}} = \sqrt[n]{a^m}
\end{equation}

\begin{definition}{有理数の指数}
  \newline
  $m, n$が整数で、$n$が正の整数であるとき、$\dfrac{m}{n}$乗を次のように定義する。
  \LARGE
  \begin{equation}
    a^{\frac{m}{n}} = \sqrt[n]{a^m}
  \end{equation}
\end{definition}

\subsubsection{実数への拡張}

有理数は無数にあるので、指数$x$を有理数まで許容した関数$y=a^x$のグラフを書くと、十分に繋がった線になる。

指数が無理数の場合は、まるでグラフ上の点と点の間を埋めるように、有理数の列で近似していくことで定義できる。

\vskip\baselineskip

これで、$x$を実数とし、関数$y=a^x$を定義できる。

\begin{definition}{指数関数}
  \newline
  $a$を正の実数とし、$x$を実数とするとき、次のような関数を指数関数という。
  \LARGE
  \begin{equation}
    y = a^x
  \end{equation}
\end{definition}

\subsection{指数関数の底の変換}

用途に応じて、使いやすい指数関数の底は異なる。

\begin{itemize}
  \item $e$:微分積分学、複素数、確率論など
  \item $2$:情報理論、コンピュータサイエンスなど
  \item $10$:対数表、音声、振動、音響など
\end{itemize}

よって、これらの底を互いに変換したい場面もある。

\vskip\baselineskip

指数の底を変えることは、指数の定数倍で実現できる。

例えば、底が$4$の指数関数$4^x$を、底が$2$の指数関数に変換したいとすると、

\begin{equation}
  4^x = (2^2)^x = 2^{2x}
\end{equation}

のように、指数部分を$2$倍することで、底を$4$から$2$へと変換できる。

当たり前だが、この変換は、$4 = 2^2$という関係のおかげで成り立っている。

「$4$は$2$の何乗か?」がすぐにわかるから、$4$から$2$への底の変換が簡単にできたのだ。

\vskip\baselineskip

より一般に、$a^x$と$b^X$において、$a = b^c$という関係があるとする。

つまり、$a$は$b$の$c$乗だとわかっているなら、

\begin{equation}
  a^x = (b^c)^x = b^{cx}
\end{equation}

のように、底を$a$から$b$へと変換できる。

\begin{theorem}{指数関数の底の変換}
  \newline
  指数を定数倍することは、底を変えることと同じ操作になる。\\
  $a = b^c$という関係があるなら、次の変換が成り立つ。
  \LARGE
  \begin{equation}
    a^x = b^{cx}
  \end{equation}
\end{theorem}

ここで重要なのは、指数関数の底を変換するには、「$a$は$b$の何乗か?」がわかっている必要があるということだ。

次章では、$a = b^c$となるような$c$を表す道具として、対数を導入する。

\section{対数関数}

\subsection{対数:指数部分を関数で表す}

指数関数は、「$a$を$x$乗したら$y$になる」という関係を表現するものだった。

ここで、逆に「$y$は$a$の何乗か?」という関係を表現するものとして、対数関数を定義する。

これは、$y$から$x$を導き出す関数であるから、指数関数$y=a^x$の逆関数といえる。

\begin{definition}{対数}
  \newline
  $a^y = x$を満たす$y$を、$a$を底とする$x$の対数といい、次のように表す。
  \LARGE
  \begin{equation}
    y = \log_a x
  \end{equation}
  \normalsize
  ここで、$x$は真数、$a$は底と呼ばれる。
\end{definition}

\begin{definition}{対数関数は指数関数の逆関数}
  \newline
  対数関数$y=\log_a x$は、指数関数$x = a^y$の逆関数である。
  \LARGE
  \begin{equation}
    \log_a x = y \quad \Longleftrightarrow  \quad a^y = x
  \end{equation}
\end{definition}

対数は、指数関数の指数部分を表す。

$a^y = x$の$y$に、$y=\log_a x$を代入することで、次のような式にまとめることもできる。

\begin{theorem}{指数部分は対数で書き換えられる}
  \LARGE
  \begin{equation}
    a^{\log_a x} = x
  \end{equation}
\end{theorem}

\subsection{対数の性質}

指数法則を対数に翻訳することで、対数の性質を導くことができる。

\subsubsection{真数のかけ算は$\log$の足し算}

$x_1 = a^m, x_2 = a^n$として、指数法則$a^m \times a^n = a^{m+n}$を考える。

\begin{align}
  x_1  x_2 & = a^m \times a^n \\
           & = a^{m+n}
\end{align}

対数は指数部分を表すので、$m+n = \log_a (x_1x_2)$がいえる。

また、$x_1 = a^m$より$m = \log_a x_1$、$x_2 = a^n$より$n = \log_a x_2$と表せるから、

\begin{equation}
  m + n = \log_a x_1 + \log_a x_2 = \log_a (x_1x_2)
\end{equation}

\begin{theorem}{積の対数は対数の和}
  \LARGE
  \begin{equation}
    \log_a (x_1x_2) = \log_a x_1 + \log_a x_2
  \end{equation}
\end{theorem}

\subsubsection{真数の割り算は$\log$の引き算}

$x_1 = a^m, x_2 = a^n$として、指数法則$\dfrac{a^m}{a^n} = a^{m-n}$を考える。

\begin{align}
  \dfrac{x_1}{x_2} & = \dfrac{a^m}{a^n} \\
                   & = a^{m-n}
\end{align}

対数は指数部分を表すので、$m-n = \log_a \left( \dfrac{x_1}{x_2} \right)$がいえる。

また、$x_1 = a^m$より$m = \log_a x_1$、$x_2 = a^n$より$n = \log_a x_2$と表せるから、

\begin{equation}
  m - n = \log_a x_1 - \log_a x_2 = \log_a \left( \dfrac{x_1}{x_2} \right)
\end{equation}

\begin{theorem}{商の対数は対数の差}
  \LARGE
  \begin{equation}
    \log_a \left( \dfrac{x_1}{x_2} \right) = \log_a x_1 - \log_a x_2
  \end{equation}
\end{theorem}

\subsubsection{真数の冪乗は$\log$の指数倍}

$x = a^m$として、指数法則$(a^m)^n = a^{mn}$を考える。

\begin{align}
  x^n & = (a^m)^n \\
      & = a^{mn}
\end{align}

対数は指数部分を表すので、$mn = \log_a x^n$がいえる。

また、$x = a^m$より$m = \log_a x$と表せるから、

\begin{equation}
  mn = n \log_a x \log_a x^n
\end{equation}

\begin{theorem}{冪の対数は対数の指数倍}
  \LARGE
  \begin{equation}
    \log_a x^n = n \log_a x
  \end{equation}
\end{theorem}

\subsection{常用対数と桁数}

\begin{definition}{常用対数}
  底を$10$にした対数関数を、常用対数と呼ぶ。
  \LARGE
  \begin{equation}
    \log_{10} x
  \end{equation}
\end{definition}

\chapter{微分と積分}

\section{1変数関数の微分}

微分とは、複雑な問題も「拡大して見たら簡単に見える(かもしれない)」という発想で、わずかな変化に着目して入力と出力の関係(関数)を調べる手法といえる。

\subsection{接線:拡大したら直線に近似できる}

関数$y=f(x)$について、引数の値を$x=x_0$からわずかに増加させて、$x=x_0+\Delta x$にした場合の出力の変化を考える。

\begin{center}
  \scalebox{2}{
    \begin{tikzpicture}
      \def\xmin{-1};
      \def\xmax{3};
      \def\ymin{-1};
      \def\ymax{3};
      \def\fn#1{exp(0.5*#1) - 0.5};
      \def\dfn#1{0.5*exp(0.5*#1)}; % \fnの導関数
      \def\xi{0.75};
      \def\xj{1.5};

      % よく使う点の座標
      \coordinate (O) at (0,0);
      \coordinate (A) at (\xi, {\fn{\xi}});
      \coordinate (B) at (\xj, {\fn{\xi}});
      \coordinate (C) at (\xj, {\fn{\xj}});
      \coordinate (D) at (A |- C);

      % 座標軸
      \draw[axis] (\xmin,0) -- (\xmax,0) node [right]{$x$};
      \draw[axis] (0,\ymin) -- (0,\ymax) node [above]{$y$};

      % 原点
      \node at (O) [below left]{$O$};

      % 傾きを表す三角形
      \draw[fill=myPurple, myPurple!80!gray, opacity=0.5] (A) --(B) -- (C) -- cycle;

      \begin{scope}
        \clip (\xmin,\ymin) rectangle (\xmax,\ymax);
        % 接線
        \draw[orange] plot[domain=\xmin:\xmax] (\x,{\fn{\xi} + \dfn{\xi}*(\x-\xi)});
        % グラフ
        \draw[magenta,thick] plot[domain=\xmin:\xmax] (\x,{\fn{\x}});
      \end{scope}

      % x軸上の目盛り
      \node (X2) at (\xj,0) [below, scale=0.5]{$\strut x_0 + \Delta x$};
      \node (X1) at (\xi,0) [below, scale=0.5, baseline = (X2.base)]{$\strut x_0$};

      % y軸上の目盛り
      \node at (0,{\fn{\xi}}) [left, scale=0.5]{$f(x_0)$};
      \node at (0,{\fn{\xj}}) [left, scale=0.5]{$f(x_0 + \Delta x)$};

      % x軸からの補助線
      \draw[auxline, thin, lightgray] (\xi,0) -- (A);
      \draw[auxline, thin, lightgray] (\xj,0) -- (B);

      % y軸からの補助線
      \draw[auxline, thin, lightgray] (0,{\fn{\xi}}) -- (A);
      \draw[auxline, thin, lightgray] (0,{\fn{\xj}}) --(C);

      % \Delta xを表す矢印
      \draw[<->] ($(A)-(0,0.1)$) -- ($(B)-(0,0.1)$) node [midway, below]{$\Delta x$};
    \end{tikzpicture}
  }
\end{center}

このとき、増分の幅$\Delta x$を狭くしていく($\Delta x$の値を小さくしていく)と、$x=x_0$付近において、関数$y=f(x)$のグラフは直線にほとんど重なるようになる。

\begin{center}
  \scalebox{2}{
    \begin{tikzpicture}[spy using outlines={circle, magnification=4, size=2cm, connect spies}]
      \def\xmin{-1};
      \def\xmax{3};
      \def\ymin{-1};
      \def\ymax{3};
      \def\fn#1{exp(0.5*#1) - 0.5};
      \def\dfn#1{0.5*exp(0.5*#1)}; % \fnの導関数
      \def\xi{0.75};
      \def\xj{0.95};

      % よく使う点の座標
      \coordinate (O) at (0,0);
      \coordinate (A) at (\xi, {\fn{\xi}});
      \coordinate (B) at (\xj, {\fn{\xi}});
      \coordinate (C) at (\xj, {\fn{\xj}});
      \coordinate (D) at (A |- C);

      % 座標軸
      \draw[axis] (\xmin,0) -- (\xmax,0) node [right]{$x$};
      \draw[axis] (0,\ymin) -- (0,\ymax) node [above]{$y$};

      % 原点
      \node at (O) [below left]{$O$};

      % x軸からの補助線
      \draw[dotted, thin, lightgray] (\xi,0) -- (A);
      \draw[dotted, thin, lightgray] (\xj,0) -- (B);
      % y軸からの補助線
      \draw[dotted, thin, lightgray] (0,{\fn{\xi}}) -- (A);
      \draw[dotted, thin, lightgray] (0,{\fn{\xj}}) --(C);

      % 傾きを表す三角形
      \draw[fill=myPurple, myPurple!80!gray, opacity=0.5] (A) --(B) -- (C) -- cycle;

      \begin{scope}
        \clip (\xmin,\ymin) rectangle (\xmax,\ymax);
        % 接線
        \draw[orange] plot[domain=\xmin:\xmax] (\x,{\fn{\xi} + \dfn{\xi}*(\x-\xi)});
        % グラフ
        \draw[magenta,thick] plot[domain=\xmin:\xmax] (\x,{\fn{\x}});
      \end{scope}

      % x軸上の目盛り
      \node (X2) at (\xj,0) [below right, scale=0.5]{$\strut x_0 + \Delta x$};
      \node (X1) at (\xi,0) [below, scale=0.5, baseline = (X2.base)]{$\strut x_0$};

      % y軸上の目盛り
      \node at (0,{\fn{\xi}}) [left, scale=0.5]{$f(x_0)$};
      \node at (0,{\fn{\xj}}) [left, scale=0.5]{$f(x_0 + \Delta x)$};

      \spy [Aquamarine] on ($(A)!.5!(C)$) in node [left] at (3.5,-1.25);
    \end{tikzpicture}
  }
\end{center}

このように、関数$f(x)$は、ある点$x_0$の付近では、

\begin{equation}
  f(x) \simeq a(x - x_0) +b
\end{equation}

という直線に近似することができる。

\vskip\baselineskip

ここで、$f(x_0)$の値を考えると、

\begin{align}
  f(x_0) & = a(x_0 - x_0) + b \\
         & = a\cdot 0 + b     \\
         & = b
\end{align}

であるから、実は$b=f(x_0)$である。

\vskip\baselineskip

一方、$a$はこの直線の傾きを表す。

そもそも、傾きとは、$x$が増加したとき、$y$がどれだけ急に(速く)増加するかを表す量である。

関数のグラフを見ると、急激に上下する箇所もあれば、なだらかに変化する箇所もある。

つまり、ある点でグラフにぴったりと沿う直線(接線)を見つけたとしても、その傾きは場所によって異なる。

そこで、「傾きは位置$x$の関数」とみなして、次のように表現しよう。

\begin{equation}
  a = f'(x)
\end{equation}

これで、先ほどの直線の式を完成させることができる。

\begin{theorem}{関数の各点の接線}
  \newline
  関数$f(x)$は、ある点$x_0$の付近では、
  \Large
  \begin{equation}
    f(x) \simeq f(x_0) + f'(x)(x - x_0)
  \end{equation}
  \normalsize
  という傾き$f'(x)$の直線に近似できる。
\end{theorem}

\subsection{接線の傾きとしての導関数}

傾きは位置$x$の関数$f'(x)$としたが、この関数がどのような関数なのか、結局傾きを計算する方法がわかっていない。

直線の傾きは$x$と$y$の増加率の比として定義されているから、まずはそれぞれの増加率を数式で表現しよう。

\begin{center}
  \scalebox{2}{
    \begin{tikzpicture}
      \def\xmin{-1};
      \def\xmax{3};
      \def\ymin{-1};
      \def\ymax{3};
      \def\fn#1{exp(0.5*#1) - 0.5};
      \def\dfn#1{0.5*exp(0.5*#1)}; % \fnの導関数
      \def\xi{0.75};
      \def\xj{1.5};

      % よく使う点の座標
      \coordinate (O) at (0,0);
      \coordinate (A) at (\xi, {\fn{\xi}});
      \coordinate (B) at (\xj, {\fn{\xi}});
      \coordinate (C) at (\xj, {\fn{\xj}});
      \coordinate (D) at (A |- C);

      % 座標軸
      \draw[axis] (\xmin,0) -- (\xmax,0) node [right]{$x$};
      \draw[axis] (0,\ymin) -- (0,\ymax) node [above]{$y$};

      % 原点
      \node at (O) [below left]{$O$};

      % 傾きを表す三角形
      \draw[fill=myPurple, myPurple!80!gray, opacity=0.5] (A) --(B) -- (C) -- cycle;

      \begin{scope}
        \clip (\xmin,\ymin) rectangle (\xmax,\ymax);
        % 接線
        \draw[orange] plot[domain=\xmin:\xmax] (\x,{\fn{\xi} + \dfn{\xi}*(\x-\xi)});
        % グラフ
        \draw[magenta,thick] plot[domain=\xmin:\xmax] (\x,{\fn{\x}});
      \end{scope}

      % x軸上の目盛り
      \node (X2) at (\xj,0) [below, scale=0.5]{$\strut x + \Delta x$};
      \node (X1) at (\xi,0) [below, scale=0.5, baseline = (X2.base)]{$\strut x$};

      % y軸上の目盛り
      \node at (0,{\fn{\xi}}) [left, scale=0.5]{$f(x)$};
      \node at (0,{\fn{\xj}}) [left, scale=0.5]{$f(x + \Delta x)$};

      % x軸からの補助線
      \draw[auxline, thin, lightgray] (\xi,0) -- (A);
      \draw[auxline, thin, lightgray] (\xj,0) -- (B);

      % y軸からの補助線
      \draw[auxline, thin, lightgray] (0,{\fn{\xi}}) -- (A);
      \draw[auxline, thin, lightgray] (0,{\fn{\xj}}) --(C);

      % \Delta xを表す矢印
      \draw[<->] ($(A)-(0,0.1)$) -- ($(B)-(0,0.1)$) node [midway, below]{$\Delta x$};
      % \Delta yを表す矢印
      \draw[<->] ($(A)+(-0.1,0)$) -- ($(D)+(-0.1, 0)$) node [midway, left]{$\Delta y$};
    \end{tikzpicture}
  }
\end{center}

この図から、$y$の増加率$\Delta y$は次のように表せることがわかる。

\begin{equation}
  \Delta y = f(x + \Delta x) - f(x)
\end{equation}

この両辺を$\Delta x$で割ると、$x$の増加率$\Delta x$と$y$の増加率$\Delta y$の比率が表せる。

\begin{equation}
  \frac{\Delta y}{\Delta x} = \frac{f(x + \Delta x) - f(x)}{\Delta x}
\end{equation}

図では$\Delta x$には幅があるが、この幅を限りなく$0$に近づけると、幅というより点になる。

つまり、$\Delta x \rightarrow 0$とすれば、$\dfrac{\Delta y}{\Delta x}$は任意の点$x$での接線の傾きとなる。

「任意の点$x$での傾き」も$x$の関数であり、この関数を導関数と呼ぶ。

\begin{definition}{導関数}
  \newline
  関数$f(x)$の任意の点$x$における接線の傾き(増加の速さ)を表す関数を導関数といい、次のように定義する。
  \Large
  \begin{equation}
    f'(x) = \lim_{\Delta x \to 0} \frac{f(x + \Delta x) - f(x)}{\Delta x}
  \end{equation}
\end{definition}

\subsection{微分とその関係式}

\begin{definition}{微分}
  関数$f(x)$から、その導関数$f'(x)$を求める操作を微分という。
\end{definition}

関数のグラフから離れて、微分という「計算」を考えるにあたって、先ほどの導関数の定義式よりも都合の良い表現式がある。

$x \to 0$とした後の$\Delta x$を$dx$と書くことにして、$\displaystyle\lim_{\Delta x \to 0}$を取り払ってしまおう。

\begin{equation}
  \begin{WithArrows}
    f'(x) & = \dfrac{f(x + dx) - f(x)}{dx} \Arrow{両辺$\times dx$} \\
    f'(x)dx & = f(x + dx) - f(x) \Arrow{$f(x)$を移項} \\
    f'(x)dx + f(x) & = f(x + dx)
  \end{WithArrows}
\end{equation}

\begin{theorem}{微分の関係式}
  \Large
  \begin{equation}
    f(x + dx)= f(x) + f'(x)dx
  \end{equation}
\end{theorem}

\subsection{不連続点と微分可能性}

点$x$において連続な関数であれば、幅$\Delta x$を小さくすれば、その間の変化量$\Delta y$も小さくなるはずである。

\begin{center}
  \scalebox{2}{
    \begin{tikzpicture}
      \def\xmin{-1};
      \def\xmax{3};
      \def\ymin{-1};
      \def\ymax{3};
      \def\fn#1{exp(0.5*#1) - 0.5};
      \def\dfn#1{0.5*exp(0.5*#1)}; % \fnの導関数
      \def\xi{0.75};
      \def\xj{1};

      % よく使う点の座標
      \coordinate (O) at (0,0);
      \coordinate (A) at (\xi, {\fn{\xi}});
      \coordinate (B) at (\xj, {\fn{\xi}});
      \coordinate (C) at (\xj, {\fn{\xj}});
      \coordinate (D) at (A |- C);

      % 座標軸
      \draw[axis] (\xmin,0) -- (\xmax,0) node [right]{$x$};
      \draw[axis] (0,\ymin) -- (0,\ymax) node [above]{$y$};

      % 原点
      \node at (O) [below left]{$O$};

      % 傾きを表す三角形
      \draw[fill=myPurple, myPurple!80!gray, opacity=0.5] (A) --(B) -- (C) -- cycle;

      \begin{scope}
        \clip (\xmin,\ymin) rectangle (\xmax,\ymax);
        % 接線
        \draw[orange] plot[domain=\xmin:\xmax] (\x,{\fn{\xi} + \dfn{\xi}*(\x-\xi)});
        % グラフ
        \draw[magenta,thick] plot[domain=\xmin:\xmax] (\x,{\fn{\x}});
      \end{scope}

      % x軸上の目盛り
      \node (X2) at (\xj,0.1) [below right, scale=0.5]{$\strut x + \Delta x$};
      \node (X1) at (\xi,0.1) [below, scale=0.5, baseline = (X2.base)]{$\strut x$};

      % y軸上の目盛り
      \node at (0,{\fn{\xi}}) [left, scale=0.5]{$f(x)$};
      \node at (0,{\fn{\xj}}) [left, scale=0.5]{$f(x + \Delta x)$};

      % x軸からの補助線
      \draw[auxline, thin, lightgray] (\xi,0) -- (A);
      \draw[auxline, thin, lightgray] (\xj,0) -- (B);

      % y軸からの補助線
      \draw[auxline, thin, lightgray] (0,{\fn{\xi}}) -- (A);
      \draw[auxline, thin, lightgray] (0,{\fn{\xj}}) --(C);

      % \Delta xを表す矢印
      \draw[<->] ($(A)-(0,0.1)$) -- ($(B)-(0,0.1)$) node [midway, below]{$\Delta x$};
      % \Delta yを表す矢印
      \draw[<->] ($(A)+(-0.1,0)$) -- ($(D)+(-0.1, 0)$) node [midway, left]{$\Delta y$};
    \end{tikzpicture}
  }
\end{center}

しかし、不連続な点について考える場合は、そうはいかない。

下の図を見ると、$\Delta x$の幅を小さくしても、$\Delta y$は不連続点での関数の値の差の分までしか小さくならない。

\begin{figure}[H]
  \begin{minipage}{0.5\hsize}
    \centering
    \scalebox{1.5}{\begin{tikzpicture}
        \def\xmin{-1};
        \def\xmax{3};
        \def\ymin{-1};
        \def\ymax{3};
        \def\fnA#1{0.5*sin(deg(0.5*pi*#1))+1};
        \def\fnB#1{-0.5*cos(deg(0.5*pi*#1+0.25))+2};
        \def\xi{1};
        \def\xj{1.5};

        % よく使う点の座標
        \coordinate (O) at (0,0);
        \coordinate (A) at (\xi, {\fnA{\xi}});
        \coordinate (B) at (\xj, {\fnB{\xj}});
        \coordinate (C) at (\xj, {\fnA{\xi}});
        \coordinate (D) at (A |- B);

        % 原点
        \node at (O) [below left]{$O$};

        % 座標軸
        \draw[axis] (\xmin,0) -- (\xmax,0) node[right] {$x$};
        \draw[axis] (0,\ymin) -- (0,\ymax) node[above] {$y$};

        % 関数の描画
        \draw[domain=\xmin:\xi, samples=100, magenta,thick, smooth] plot (\x, {\fnA{\x}});
        \draw[domain=\xi:\xmax, samples=100, magenta,thick, smooth] plot (\x, {\fnB{\x}});

        % x軸上の目盛り
        \node (X2) at (\xj,0.15) [below right, scale=0.75]{$\strut x + \Delta x$};
        \node (X1) at (\xi,0.15) [below left, scale=0.75, baseline = (X2.base)]{$\strut x$};

        % y軸上の目盛り
        \node at (0,{\fnA{\xi}}) [left, scale=0.75]{$f(x)$};
        \node at (0,{\fnB{\xj}}) [left, scale=0.75]{$f(x + \Delta x)$};

        % x軸からの補助線
        \draw[auxline, thin, lightgray] (\xi,0) -- (A);
        \draw[auxline, thin, lightgray] (\xj,0) -- (B);

        % y軸からの補助線
        \draw[auxline, thin, lightgray] (0,{\fnA{\xi}}) -- (A);
        \draw[auxline, thin, lightgray] (0,{\fnB{\xj}}) --(B);

        % \Delta xを表す矢印
        \draw[<->] ($(A)-(0,0.1)$) -- ($(C)-(0,0.1)$) node [midway, below]{$\Delta x$};
        % \Delta yを表す矢印
        \draw[<->] ($(A)+(-0.1,0)$) -- ($(D)+(-0.1, 0)$) node [midway, left]{$\Delta y$};
      \end{tikzpicture}
    }
  \end{minipage}%
  \begin{minipage}{0.5\hsize}
    \centering
    \scalebox{1.5}{
      \begin{tikzpicture}
        \def\xmin{-1};
        \def\xmax{3};
        \def\ymin{-1};
        \def\ymax{3};
        \def\fnA#1{0.5*sin(deg(0.5*pi*#1))+1};
        \def\fnB#1{-0.5*cos(deg(0.5*pi*#1+0.25))+2};
        \def\xi{1};
        \def\xj{1};

        % よく使う点の座標
        \coordinate (O) at (0,0);
        \coordinate (A) at (\xi, {\fnA{\xi}});
        \coordinate (B) at (\xj, {\fnB{\xj}});
        \coordinate (C) at (\xj, {\fnA{\xi}});
        \coordinate (D) at (A |- B);

        % 原点
        \node at (O) [below left]{$O$};

        % 座標軸
        \draw[axis] (\xmin,0) -- (\xmax,0) node[right] {$x$};
        \draw[axis] (0,\ymin) -- (0,\ymax) node[above] {$y$};

        % 関数の描画
        \draw[domain=\xmin:\xi, samples=100, magenta,thick, smooth] plot (\x, {\fnA{\x}});
        \draw[domain=\xi:\xmax, samples=100, magenta,thick, smooth] plot (\x, {\fnB{\x}});

        % x軸上の目盛り
        \node (X2) at (\xj,0) [below, scale=0.6]{$ (x+ \Delta x) \simeq x$};

        % y軸上の目盛り
        \node at (0,{\fnA{\xi}}) [left, scale=0.75]{$f(x)$};
        \node at (0,{\fnB{\xj}}) [left, scale=0.75]{$f(x + \Delta x)$};

        % x軸からの補助線
        \draw[auxline, thin, lightgray] (\xi,0) -- (A);
        \draw[auxline, thin, lightgray] (\xj,0) -- (B);

        % y軸からの補助線
        \draw[auxline, thin, lightgray] (0,{\fnA{\xi}}) -- (A);
        \draw[auxline, thin, lightgray] (0,{\fnB{\xj}}) --(B);

        % \Delta xを表す矢印
        \draw ($(A)-(0,0.1)$) -- ($(C)-(0,0.1)$) node [midway, below]{$\Delta x$};
        % \Delta yを表す矢印
        \draw[<->] ($(A)+(-0.1,0)$) -- ($(D)+(-0.1, 0)$) node [midway, left]{$\Delta y$};
      \end{tikzpicture}
    }
  \end{minipage}
\end{figure}

このような不連続点においては、どんなに拡大しても、関数のグラフが直線にぴったりと重なることはない。

「拡大すれば直線に近似できる」というのが微分の考え方だが、不連続点ではこの考え方を適用できないのだ。

\vskip\baselineskip

関数の不連続点においては、微分という計算を考えることがそもそもできない。

ある点での関数のグラフが直線に重なる(微分可能である)ためには、$\Delta x \to 0$としたときに$\Delta y \to 0$となる必要がある。

\subsection{導関数のさまざまな記法}

微分を考えるときは、$\Delta x \to 0$としたときに$\Delta y \to 0$となる前提のもとで議論する。

$\Delta x \to 0$とした結果を$dx$、$\Delta y \to 0$の結果を$dy$とすると、ある点$x$での接線の傾きは、次のようにも表現できる。

\begin{equation}
  \frac{dy}{dx} = \lim_{\Delta x \to 0} \frac{\Delta y}{\Delta x}
\end{equation}

この接線の傾きが$x$の関数であることを表現したいときは、次のように書くこともある。

\begin{equation}
  \dfrac{dy}{dx}(x)
\end{equation}

これも一つの導関数(位置に応じた接線の傾きを表す関数)の表記法である。

この記法は、どの変数で微分しているかがわかりやすいという利点がある。

\begin{definition}{導関数のライプニッツ記法}
  \newline
  次のような記号はいずれも、関数$y = f(x)$の導関数を表す。
  \Large
  \begin{equation}
    \frac{dy}{dx} = \dfrac{dy}{dx}(x) = \dfrac{df}{dx} = \dfrac{d}{dx}f(x)
  \end{equation}
\end{definition}

特に、$\dfrac{d}{dx}f(x)$という記法は、$\dfrac{d}{dx}$の部分を微分操作を表す演算子として捉えて、「関数$f(x)$に微分という操作を施した」ことを表現しているように見える。

\begin{definition}{微分演算子}
  \newline
  関数を微分するという操作を表現する演算子を微分演算子という。\\
  例えば、次のような記号で表される。
  \Large
  \begin{equation}
    \dfrac{d}{dx}
  \end{equation}
\end{definition}

ところで、これまで使ってきた$f'(x)$という導関数の記法にも、名前がついている。

\begin{definition}{導関数のニュートン記法}
  \newline
  次の記号は、関数$y = f(x)$の導関数を表す。
  \Large
  \begin{equation}
    f'(x)
  \end{equation}
\end{definition}

この記法は、「$f$という関数から導出された関数が$f'$である」ことを表現している。

導関数はあくまでも関数$f$から派生したものであるから、$f$という文字はそのまま、加工されたことを表すために$'$をつけたものと解釈できる。

\subsection{微分の性質}

微分の関係式を使うことで、微分に関する有用な性質を導くことができる。

\begin{review}
  微分の関係式
  \begin{equation}
    f(x + dx)= \origFn{f(x)} + \derivFn{f'(x)} dx
  \end{equation}
\end{review}

\subsubsection{関数の一次結合の微分}

$\alpha f(x) + \beta g(x)$において、$x$を$dx$だけ微小変化させてみる。
\begin{align}
  \alpha f(x+dx) + \beta g(x+dx)
   & = \alpha \left\{ f(x) + f'(x) dx \right\} + \beta \left\{ g(x) + g'(x) dx \right\}    \\
   & = \origFn{ \alpha f(x) + \beta g(x) }+ \{ \derivFn{ \alpha f'(x) + \beta g'(x)} \} dx
\end{align}

\begin{theorem}{微分の線形性}
  \Large
  \begin{equation}
    \left( \alpha f(x) + \beta g(x) \right)' = \alpha f'(x) + \beta g'(x)
  \end{equation}
\end{theorem}

\subsubsection{関数の積の微分}

$f(x)g(x)$において、$x$を$dx$だけ微小変化させてみる。

\begin{align}
  f(x+dx)g(x+dx)
   & = \left\{ f(x) + f'(x) dx \right\} \left\{ g(x) + g'(x) dx \right\}                                            \\
   & = f(x)g(x) + f'(x)g(x)dx + f(x)g'(x)dx +  f'(x)g'(x)dx^2                                                       \\
   & = f(x)g(x) + \{ f'(x)g(x) + f(x)g'(x) \}dx + \fitLabelMath[BlueGreen][BlueGreen!40]{ f'(x)g'(x)dx^2}{2次以上の微小量}
\end{align}

ここで、$dx^2$は、$dx$より速く$0$に近づくので無視できる。

荒く言ってしまえば、$dx$でさえ微小量なのだから、$dx^2$なんて存在しないも同然だと考えてよい。

このことは、次の図を見るとイメージできる。

\begin{center}
  \begin{tikzpicture}
    \def\width{6}
    \def\height{4}
    \def\dx{2.25}

    % 座標
    \coordinate (A) at (0,0);
    \coordinate (B) at (\width,0);
    \coordinate (C) at (\width,\height);
    \coordinate (D) at (0,\height);
    \coordinate (Cx) at ($(C)+(\dx,0)$);
    \coordinate (Cy) at ($(C)+(0,\dx)$);
    \coordinate (Cxy) at ($(C)+(\dx,\dx)$);
    \coordinate (Bx) at ($(B)+(\dx,0)$);
    \coordinate (Dy) at ($(D)+(0,\dx)$);

    % % 点(デバッグ用)
    % \fill (A) circle[radius=2pt] node[below left] {A};
    % \fill (B) circle[radius=2pt] node[below right] {B};
    % \fill (C) circle[radius=2pt] node[above right] {C};
    % \fill (D) circle[radius=2pt] node[above left] {D};
    % \fill (Cx) circle[radius=2pt] node[above right] {Cx};
    % \fill (Cy) circle[radius=2pt] node[above right] {Cy};
    % \fill (Cxy) circle[radius=2pt] node[above right] {Cxy};
    % \fill (Bx) circle[radius=2pt] node[below right] {Bx};
    % \fill (Dy) circle[radius=2pt] node[above left] {Dy};

    % 長方形
    \draw [fill=lightgray!20, draw=lightgray!80!gray] (A) rectangle (C) node[pos=.5] {$f(x)g(x)$};
    \draw [fill=magenta!30, draw=magenta!70!gray] (B) rectangle (Cx) node[pos=.5] {$f'(x)g(x)dx$};
    \draw [fill=magenta!30, draw=magenta!70!gray] (D) rectangle (Cy) node[pos=.5] {$f(x)g'(x)dx$};
    \draw [fill=BlueGreen!30, draw=BlueGreen!70!gray] (C) rectangle (Cxy) node[pos=.5] {$f'(x)g'(x)dx^2$};

    % 補助線
    \def\l{1} % 補助線の長さ
    \draw[auxline] (B) -- ++(0, -\l) node[below] {$f(x)$};
    \draw[auxline] (Bx) -- ++(0, -\l) node[below] {$f(x+dx)$};
    \draw[auxline] (D) -- ++(-\l, 0) node[left] {$g(x)$};
    \draw[auxline] (Dy) -- ++(-\l, 0) node[left] {$g(x+dx)$};

    % 辺の長さを表す矢印
    \def\s{1em} % 辺と矢印の隙間
    \def\h{0.1} % 矢印の矢同士の隙間
    \draw[<->, thick, gray] ($(A)+(\h,-\s)$) -- ($(B)-(\h,\s)$) node[midway,below] {$f(x)$};
    \draw[<->, thick, gray] ($(A)+(-\s,\h)$) -- ($(D)-(\s,\h)$) node[midway,left] {$g(x)$};
    \draw[<->, thick, magenta!80] ($(B)+(\h,-\s)$) -- ($(Bx)-(\h,\s)$) node[midway,below] {$f'(x)dx$};
    \draw[<->, thick, magenta!80] ($(D)+(-\s,\h)$) -- ($(Dy)-(\s,\h)$) node[midway,left] {$g'(x)dx$};
  \end{tikzpicture}
\end{center}

$dx \to 0$のとき$dy \to 0$となる場合に微分という計算を定義するのだから、$dx$を小さくしていくと、$dy$にあたる$f(x + dx) - f(x)$(これは$f'(x)dx$と等しい)も小さくなっていく。

同様にして、$g(x + dx) - g(x)$(これは$g'(x)dx$と等しい)も小さくなっていく。

\begin{review}
  微分の関係式$f(x + dx)= f(x) + f'(x) dx$より、
  \large
  \begin{equation}
    \textcolor{magenta!80}{f'(x)dx} = f(x + dx) - f(x)
  \end{equation}
\end{review}

$dx$を小さくした場合を図示すると、

\begin{center}
  \begin{tikzpicture}
    \def\width{6}
    \def\height{4}
    \def\dx{0.15}

    % 座標
    \coordinate (A) at (0,0);
    \coordinate (B) at (\width,0);
    \coordinate (C) at (\width,\height);
    \coordinate (D) at (0,\height);
    \coordinate (Cx) at ($(C)+(\dx,0)$);
    \coordinate (Cy) at ($(C)+(0,\dx)$);
    \coordinate (Cxy) at ($(C)+(\dx,\dx)$);
    \coordinate (Bx) at ($(B)+(\dx,0)$);
    \coordinate (Dy) at ($(D)+(0,\dx)$);

    % % 点(デバッグ用)
    % \fill (A) circle[radius=2pt] node[below left] {A};
    % \fill (B) circle[radius=2pt] node[below right] {B};
    % \fill (C) circle[radius=2pt] node[above right] {C};
    % \fill (D) circle[radius=2pt] node[above left] {D};
    % \fill (Cx) circle[radius=2pt] node[above right] {Cx};
    % \fill (Cy) circle[radius=2pt] node[above right] {Cy};
    % \fill (Cxy) circle[radius=2pt] node[above right] {Cxy};
    % \fill (Bx) circle[radius=2pt] node[below right] {Bx};
    % \fill (Dy) circle[radius=2pt] node[above left] {Dy};

    % 長方形
    \draw [fill=lightgray!20, draw=lightgray!80!gray] (A) rectangle (C) node[pos=.5] {$f(x)g(x)$};
    \draw [fill=magenta!30, draw=magenta!70!gray] (B) rectangle (Cx);
    \draw [fill=magenta!30, draw=magenta!70!gray] (D) rectangle (Cy);
    \draw [fill=BlueGreen!40, draw=BlueGreen!70!gray] (C) rectangle (Cxy);
  \end{tikzpicture}
\end{center}

$\fitLabelMath[BlueGreen][BlueGreen!40]{f'(x)g'(x)dx^2}{\footnotesize 2次以上の微小量}$に相当する左上の領域は、ほとんど点になってしまうことがわかる。

\vskip\baselineskip

このように、$dx^2$の項は無視してもよいものとして、先ほどの計算式は次のようになる。

\begin{align}
  f(x+dx)g(x+dx)
   & = \origFn{f(x)g(x)} + \{ \derivFn{f'(x)g(x) + f(x)g'(x)} \}dx
\end{align}

\begin{theorem}{微分のライプニッツ則}
  \Large
  \begin{equation}
    \left( f(x) g(x) \right)' = f'(x) g(x) + f(x) g'(x)
  \end{equation}
\end{theorem}

\subsection{冪関数の微分}

具体的な関数の導関数も、微分の関係式をもとに考えることができる。

まずは、簡単な例として、冪関数$y=x^n$の微分を考えてみよう。

\subsubsection{$y=x^2$の微分}

$y=f(x)=x^2$において、$x$を$dx$だけ微小変化させると、$y$は$dy$だけ変化するとする。

すると、微分の関係式は$y + dy = f(x + dx) = (x+dx)^2$となるが、これを次のように展開して考える。

\begin{equation}
  y + dy = (x + dx)(x + dx)
\end{equation}

右辺の$(x+dx)(x+dx)$からは、

\begin{itemize}
  \item $x^2$の項が1つ
  \item $xdx$の項が2つ
  \item $dx^2$の項が1つ
\end{itemize}

現れることになる。

数式で表すと、

\begin{equation}
  \eqnmarkbox[magenta]{Y1}{y} + dy = \eqnmarkbox[magenta]{Y2}{x^2} + 2xdx + dx^2
\end{equation}
\annotatetwo{below}{Y1}{Y2}{\bfseries 同じ}

ここで$y=x^2$なので、左辺の$y$と右辺の$x^2$は相殺される。

\begin{equation}
  dy = 2xdx + \fitLabelMath[BlueGreen][BlueGreen!40]{dx^2}{高次の微小量}
\end{equation}

さらに、$dx^2$の項は無視することができる。

なぜなら、$dx$を小さくすると、$dx^2$は$dx$とは比べ物にならないくらい小さくなってしまうからだ。

\begin{center}
  \begin{tikzpicture}
    \begin{scope}[local bounding box=left]
      \def\width{4}
      \def\dx{0.6}

      % 座標
      \coordinate (A) at (0,0);
      \coordinate (B) at (\width,0);
      \coordinate (C) at (\width,\width);
      \coordinate (D) at (0,\width);
      \coordinate (Cx) at ($(C)+(\dx,0)$);
      \coordinate (Cy) at ($(C)+(0,\dx)$);
      \coordinate (Cxy) at ($(C)+(\dx,\dx)$);
      \coordinate (Bx) at ($(B)+(\dx,0)$);
      \coordinate (Dy) at ($(D)+(0,\dx)$);

      % % 点(デバッグ用)
      % \fill (A) circle[radius=2pt] node[below left] {A};
      % \fill (B) circle[radius=2pt] node[below right] {B};
      % \fill (C) circle[radius=2pt] node[above right] {C};
      % \fill (D) circle[radius=2pt] node[above left] {D};
      % \fill (Cx) circle[radius=2pt] node[above right] {Cx};
      % \fill (Cy) circle[radius=2pt] node[above right] {Cy};
      % \fill (Cxy) circle[radius=2pt] node[above right] {Cxy};
      % \fill (Bx) circle[radius=2pt] node[below right] {Bx};
      % \fill (Dy) circle[radius=2pt] node[above left] {Dy};

      % 長方形
      \draw [fill=lightgray!20, draw=lightgray!80!gray] (A) rectangle (C) node[pos=.5] {$x^2$};
      \draw [fill=magenta!30, draw=magenta!70!gray] (B) rectangle (Cx) node[pos=.5, rotate=-90] {$xdx$};
      \draw [fill=magenta!30, draw=magenta!70!gray] (D) rectangle (Cy) node[pos=.5] {$xdx$};
      \draw [fill=BlueGreen!30, draw=BlueGreen!70!gray] (C) rectangle (Cxy) node[pos=.5] {$dx^2$};

      % 辺の長さを表す矢印
      \def\s{1em} % 辺と矢印の隙間
      \def\h{0.01} % 矢印の矢同士の隙間
      \draw[<->, thick, gray] ($(A)+(\h,-\s)$) -- ($(B)-(\h,\s)$) node[midway,below] {$x$};
      \draw[<->, thick, gray] ($(A)+(-\s,\h)$) -- ($(D)-(\s,\h)$) node[midway,left] {$x$};
      \draw[<->, thick, magenta!80] ($(B)+(\h,-\s)$) -- ($(Bx)-(\h,\s)$) node[midway,below] {$dx$};
      \draw[<->, thick, magenta!80] ($(D)+(-\s,\h)$) -- ($(Dy)-(\s,\h)$) node[midway,left] {$dx$};
    \end{scope}

    \begin{scope}[xshift=0.5\textwidth, local bounding box=right]
      \def\width{4}
      \def\dx{0.1}

      % 座標
      \coordinate (A) at (0,0);
      \coordinate (B) at (\width,0);
      \coordinate (C) at (\width,\width);
      \coordinate (D) at (0,\width);
      \coordinate (Cx) at ($(C)+(\dx,0)$);
      \coordinate (Cy) at ($(C)+(0,\dx)$);
      \coordinate (Cxy) at ($(C)+(\dx,\dx)$);
      \coordinate (Bx) at ($(B)+(\dx,0)$);
      \coordinate (Dy) at ($(D)+(0,\dx)$);

      % % 点(デバッグ用)
      % \fill (A) circle[radius=2pt] node[below left] {A};
      % \fill (B) circle[radius=2pt] node[below right] {B};
      % \fill (C) circle[radius=2pt] node[above right] {C};
      % \fill (D) circle[radius=2pt] node[above left] {D};
      % \fill (Cx) circle[radius=2pt] node[above right] {Cx};
      % \fill (Cy) circle[radius=2pt] node[above right] {Cy};
      % \fill (Cxy) circle[radius=2pt] node[above right] {Cxy};
      % \fill (Bx) circle[radius=2pt] node[below right] {Bx};
      % \fill (Dy) circle[radius=2pt] node[above left] {Dy};

      % 長方形
      \draw [fill=lightgray!20, draw=lightgray!80!gray] (A) rectangle (C) node[pos=.5] {$x^2$};
      \draw [fill=magenta!30, draw=magenta!70!gray] (B) rectangle (Cx);
      \draw [fill=magenta!30, draw=magenta!70!gray] (D) rectangle (Cy);
      \draw [fill=BlueGreen!30, draw=BlueGreen!70!gray] (C) rectangle (Cxy);
    \end{scope}

    \draw[->, thick] ($(left.east)+(1em, 0)$) -- ($(left-|right.west)-(1em,0)$) node[pos=.5, above] {$dx \to 0$};
  \end{tikzpicture}
\end{center}

というわけで、次のような式が得られる。

\begin{equation}
  dy = 2xdx
\end{equation}

よって、$y=x^2$の導関数は、$y'=2x$となることがわかった。

\begin{equation}
  \frac{dy}{dx} = 2x
\end{equation}

\subsubsection{$y=x^3$の微分}

同じように、$y=x^3$の微分を考えてみよう。

\begin{equation}
  y + dy = (x + dx)(x + dx)(x + dx)
\end{equation}

右辺の$(x+dx)(x+dx)(x+dx)$からは、

\begin{itemize}
  \item $x^3$の項が1つ
  \item $x^2dx$の項が3つ
  \item $dx^3$の項が1つ
\end{itemize}

現れることになる。

\begin{equation}
  \eqnmarkbox[magenta]{Y1}{y} + dy = \eqnmarkbox[magenta]{Y2}{x^3} + 3x^2dx + dx^3
\end{equation}
\annotatetwo{below}{Y1}{Y2}{\bfseries 同じ}

ここで$y=x^3$なので、左辺の$y$と右辺の$x^3$は相殺される。

\begin{equation}
  dy = 3x^2dx + \fitLabelMath[BlueGreen][BlueGreen!40]{dx^3}{高次の微小量}
\end{equation}

さらにここでは、$dx^3$の項を無視することができる。

次の図を見てみよう。

各辺$dx$の立方体は、$dx$を小さくすると、ほぼ点にしか見えないほど小さくなる。

つまり、各辺$dx$の立方体の体積$dx^3$は、考慮する必要がない。

\begin{center}
  \tdplotsetmaincoords{60}{125}
  \begin{tikzpicture}[
      tdplot_main_coords,
      grid/.style={very thin,gray},
      axis/.style={->,blue,thick},
      cube/.style={very thick,fill=lightgray!20, draw=lightgray!80!gray},
      cube_dx/.style={very thick,fill=magenta!30, draw=magenta!70!gray, opacity=0.6},
      cube_dx3/.style={very thick,fill=BlueGreen!30, draw=BlueGreen!70!gray, opacity=0.6},
      cube hidden/.style={thick, dashed, draw=lightgray!80!gray}
    ]
    \begin{scope}[local bounding box=left]
      \def\size{3}
      \def\dx{0.6}

      % 頂点の座標
      \coordinate (O) at (0,0,0);
      \coordinate (Axy) at (\size, 0, 0);
      \coordinate (Bxy) at (\size, \size, 0);
      \coordinate (Cxy) at (0, \size, 0);
      \coordinate (Ayz) at (0, 0, \size);
      \coordinate (Byz) at (0, \size, \size);
      \coordinate (Axz) at (\size, 0, \size);
      \coordinate (Axyz) at (\size, \size, \size);

      %%% 元の立方体
      % draw the front-right of the cube
      \draw[cube] (Axy) -- (Bxy) -- (Axyz) -- (Axz) -- cycle;
      % draw the front-left of the cube
      \draw[cube] (Cxy) -- (Bxy) -- (Axyz) -- (Byz) -- cycle;
      % draw the top of the cube
      \draw[cube] (Ayz) -- (Byz) -- (Axyz) -- (Axz) -- cycle;

      %%% x軸方向にdxだけ拡張
      % 底面
      \draw[cube_dx] (Axy) -- ($(Axy)+(\dx,0,0)$) -- ($(Bxy)+(\dx,0,0)$) -- (Bxy) -- cycle;
      % 後ろの面
      \draw[cube_dx] (Axz) -- (Axy) -- ($(Axy)+(\dx,0,0)$) -- ($(Axz)+(\dx,0,0)$) -- cycle;
      % 前面
      \draw[cube_dx] (Axyz) -- (Bxy) -- ($(Bxy)+(\dx,0,0)$) -- ($(Axyz)+(\dx,0,0)$) -- cycle;
      % 上面
      \draw[cube_dx] (Axz) -- ($(Axz)+(\dx,0,0)$) -- ($(Axyz)+(\dx,0,0)$) -- (Axyz) -- cycle;
      % 正方形
      \draw[cube_dx] ($(Axy)+(\dx,0,0)$) -- ($(Bxy)+(\dx,0,0)$) -- ($(Axyz)+(\dx,0,0)$) -- ($(Axz)+(\dx,0,0)$) -- cycle;

      %%% y軸方向にdxだけ拡張
      % 底面
      \draw[cube_dx] (Bxy) -- ($(Bxy)+(0,\dx,0)$) -- ($(Cxy)+(0,\dx,0)$) -- (Cxy) -- cycle;
      % 後ろの面
      \draw[cube_dx] (Cxy) -- ($(Cxy)+(0,\dx,0)$) -- ($(Byz)+(0,\dx,0)$) -- (Byz) -- cycle;
      % 前面
      \draw[cube_dx] (Axyz) -- ($(Axyz)+(0,\dx,0)$) -- ($(Bxy)+(0,\dx,0)$) -- (Bxy) -- cycle;
      % 上面
      \draw[cube_dx] (Axyz) -- ($(Axyz)+(0,\dx,0)$) -- ($(Byz)+(0,\dx,0)$) -- (Byz) -- cycle;
      % 正方形
      \draw[cube_dx] ($(Axyz)+(0,\dx,0)$) -- ($(Bxy)+(0,\dx,0)$) -- ($(Cxy)+(0,\dx,0)$) -- ($(Byz)+(0,\dx,0)$) -- cycle;

      %%% z軸方向にdxだけ拡張
      % 後ろの面
      \draw[cube_dx] (Ayz) -- ($(Ayz)+(0,0,\dx)$) -- ($(Byz)+(0,0,\dx)$) -- (Byz) -- cycle;
      % 左の面
      \draw[cube_dx] (Ayz) -- ($(Ayz)+(0,0,\dx)$) -- ($(Axz)+(0,0,\dx)$) -- (Axz) -- cycle;
      % 右の面
      \draw[cube_dx] (Byz) -- ($(Byz)+(0,0,\dx)$) -- ($(Axyz)+(0,0,\dx)$) -- (Axyz) -- cycle;
      % 前の面
      \draw[cube_dx] (Axz) -- ($(Axz)+(0,0,\dx)$) -- ($(Axyz)+(0,0,\dx)$) -- (Axyz) -- cycle;
      % 正方形
      \draw[cube_dx] ($(Ayz)+(0,0,\dx)$) -- ($(Byz)+(0,0,\dx)$) -- ($(Axyz)+(0,0,\dx)$) -- ($(Axz)+(0,0,\dx)$) -- cycle;

      %%% 高次の微小量
      \draw[cube_dx3] (Axyz) -- ($(Axyz)+(\dx,0,0)$) -- ($(Axyz)+(\dx,0,\dx)$) -- ($(Axyz)+(0,0,\dx)$) -- cycle;
      \draw[cube_dx3] (Axyz) -- ($(Axyz)+(0,\dx,0)$) -- ($(Axyz)+(0,\dx,\dx)$) -- ($(Axyz)+(0,0,\dx)$) -- cycle;
      \draw[cube_dx3] (Axyz) -- ($(Axyz)+(\dx,0,0)$) -- ($(Axyz)+(\dx,\dx,0)$) -- ($(Axyz)+(0,\dx,0)$) -- cycle; % 底面
      \draw[cube_dx3] ($(Axyz)+(\dx,0,0)$) -- ($(Axyz)+(\dx,\dx,0)$) -- ($(Axyz)+(\dx,\dx,\dx)$) -- ($(Axyz)+(\dx,0,\dx)$) -- cycle;
      \draw[cube_dx3] ($(Axyz)+(0,\dx,0)$) -- ($(Axyz)+(\dx,\dx,0)$) -- ($(Axyz)+(\dx,\dx,\dx)$) -- ($(Axyz)+(0,\dx,\dx)$) -- cycle;
      \draw[cube_dx3] ($(Axyz)+(0,0,\dx)$) -- ($(Axyz)+(\dx,0,\dx)$) -- ($(Axyz)+(\dx,\dx,\dx)$) -- ($(Axyz)+(0,\dx,\dx)$) -- cycle; % 上面

      %%% 辺の長さを表す矢印
      % 辺と矢印の隙間
      \def\s{0.4}
      % dxを表す矢印
      \draw[<->, thick, magenta!80] ($(Axz)+(\s,0,0.15)$) -- ($(Axz)+(\s,0,\dx + 0.15)$) node[midway,left] {$dx$};
      % xを表す矢印
      \draw[<->, thick, gray] ($(Axy)+(\dx + \s,0,0.1)$) -- ($(Axz)+(\dx + \s,0,0.1)$) node[midway,left] {$x$};

      % draw dashed lines to represent hidden edges
      \draw[cube hidden] (O) -- (Axy);
      \draw[cube hidden] (O) -- (Cxy);
      \draw[cube hidden] (O) -- (Ayz);

      % % 座標軸(デバッグ用)
      % \draw[axis] (0,0,0) -- (3,0,0) node[anchor=west]{$x$};
      % \draw[axis] (0,0,0) -- (0,3,0) node[anchor=west]{$y$};
      % \draw[axis] (0,0,0) -- (0,0,3) node[anchor=west]{$z$};
      % % 点(デバッグ用)
      % \fill (O) circle[radius=2pt] node[above left] {O};
      % \fill (Axy) circle[radius=2pt] node[below left] {Axy};
      % \fill (Bxy) circle[radius=2pt] node[below right] {Bxy};
      % \fill (Cxy) circle[radius=2pt] node[above right] {Cxy};
      % \fill (Ayz) circle[radius=2pt] node[above left] {Ayz};
      % \fill (Byz) circle[radius=2pt] node[above right] {Byz};
      % \fill (Axz) circle[radius=2pt] node[above left] {Axz};
      % \fill (Axyz) circle[radius=2pt] node[above right] {Axyz};
    \end{scope}

    \begin{scope}[xshift=0.5\textwidth, local bounding box=right]
      \def\size{3}
      \def\dx{0.15}

      % 頂点の座標
      \coordinate (O) at (0,0,0);
      \coordinate (Axy) at (\size, 0, 0);
      \coordinate (Bxy) at (\size, \size, 0);
      \coordinate (Cxy) at (0, \size, 0);
      \coordinate (Ayz) at (0, 0, \size);
      \coordinate (Byz) at (0, \size, \size);
      \coordinate (Axz) at (\size, 0, \size);
      \coordinate (Axyz) at (\size, \size, \size);

      %%% 元の立方体
      % draw the front-right of the cube
      \draw[cube] (Axy) -- (Bxy) -- (Axyz) -- (Axz) -- cycle;
      % draw the front-left of the cube
      \draw[cube] (Cxy) -- (Bxy) -- (Axyz) -- (Byz) -- cycle;
      % draw the top of the cube
      \draw[cube] (Ayz) -- (Byz) -- (Axyz) -- (Axz) -- cycle;

      %%% x軸方向にdxだけ拡張
      % 底面
      \draw[cube_dx] (Axy) -- ($(Axy)+(\dx,0,0)$) -- ($(Bxy)+(\dx,0,0)$) -- (Bxy) -- cycle;
      % 後ろの面
      \draw[cube_dx] (Axz) -- (Axy) -- ($(Axy)+(\dx,0,0)$) -- ($(Axz)+(\dx,0,0)$) -- cycle;
      % 前面
      \draw[cube_dx] (Axyz) -- (Bxy) -- ($(Bxy)+(\dx,0,0)$) -- ($(Axyz)+(\dx,0,0)$) -- cycle;
      % 上面
      \draw[cube_dx] (Axz) -- ($(Axz)+(\dx,0,0)$) -- ($(Axyz)+(\dx,0,0)$) -- (Axyz) -- cycle;
      % 正方形
      \draw[cube_dx] ($(Axy)+(\dx,0,0)$) -- ($(Bxy)+(\dx,0,0)$) -- ($(Axyz)+(\dx,0,0)$) -- ($(Axz)+(\dx,0,0)$) -- cycle;

      %%% y軸方向にdxだけ拡張
      % 底面
      \draw[cube_dx] (Bxy) -- ($(Bxy)+(0,\dx,0)$) -- ($(Cxy)+(0,\dx,0)$) -- (Cxy) -- cycle;
      % 後ろの面
      \draw[cube_dx] (Cxy) -- ($(Cxy)+(0,\dx,0)$) -- ($(Byz)+(0,\dx,0)$) -- (Byz) -- cycle;
      % 前面
      \draw[cube_dx] (Axyz) -- ($(Axyz)+(0,\dx,0)$) -- ($(Bxy)+(0,\dx,0)$) -- (Bxy) -- cycle;
      % 上面
      \draw[cube_dx] (Axyz) -- ($(Axyz)+(0,\dx,0)$) -- ($(Byz)+(0,\dx,0)$) -- (Byz) -- cycle;
      % 正方形
      \draw[cube_dx] ($(Axyz)+(0,\dx,0)$) -- ($(Bxy)+(0,\dx,0)$) -- ($(Cxy)+(0,\dx,0)$) -- ($(Byz)+(0,\dx,0)$) -- cycle;

      %%% z軸方向にdxだけ拡張
      % 後ろの面
      \draw[cube_dx] (Ayz) -- ($(Ayz)+(0,0,\dx)$) -- ($(Byz)+(0,0,\dx)$) -- (Byz) -- cycle;
      % 左の面
      \draw[cube_dx] (Ayz) -- ($(Ayz)+(0,0,\dx)$) -- ($(Axz)+(0,0,\dx)$) -- (Axz) -- cycle;
      % 右の面
      \draw[cube_dx] (Byz) -- ($(Byz)+(0,0,\dx)$) -- ($(Axyz)+(0,0,\dx)$) -- (Axyz) -- cycle;
      % 前の面
      \draw[cube_dx] (Axz) -- ($(Axz)+(0,0,\dx)$) -- ($(Axyz)+(0,0,\dx)$) -- (Axyz) -- cycle;
      % 正方形
      \draw[cube_dx] ($(Ayz)+(0,0,\dx)$) -- ($(Byz)+(0,0,\dx)$) -- ($(Axyz)+(0,0,\dx)$) -- ($(Axz)+(0,0,\dx)$) -- cycle;

      %%% 高次の微小量
      \draw[cube_dx3] (Axyz) -- ($(Axyz)+(\dx,0,0)$) -- ($(Axyz)+(\dx,0,\dx)$) -- ($(Axyz)+(0,0,\dx)$) -- cycle;
      \draw[cube_dx3] (Axyz) -- ($(Axyz)+(0,\dx,0)$) -- ($(Axyz)+(0,\dx,\dx)$) -- ($(Axyz)+(0,0,\dx)$) -- cycle;
      \draw[cube_dx3] (Axyz) -- ($(Axyz)+(\dx,0,0)$) -- ($(Axyz)+(\dx,\dx,0)$) -- ($(Axyz)+(0,\dx,0)$) -- cycle; % 底面
      \draw[cube_dx3] ($(Axyz)+(\dx,0,0)$) -- ($(Axyz)+(\dx,\dx,0)$) -- ($(Axyz)+(\dx,\dx,\dx)$) -- ($(Axyz)+(\dx,0,\dx)$) -- cycle;
      \draw[cube_dx3] ($(Axyz)+(0,\dx,0)$) -- ($(Axyz)+(\dx,\dx,0)$) -- ($(Axyz)+(\dx,\dx,\dx)$) -- ($(Axyz)+(0,\dx,\dx)$) -- cycle;
      \draw[cube_dx3] ($(Axyz)+(0,0,\dx)$) -- ($(Axyz)+(\dx,0,\dx)$) -- ($(Axyz)+(\dx,\dx,\dx)$) -- ($(Axyz)+(0,\dx,\dx)$) -- cycle; % 上面

      % draw dashed lines to represent hidden edges
      \draw[cube hidden] (O) -- (Axy);
      \draw[cube hidden] (O) -- (Cxy);
      \draw[cube hidden] (O) -- (Ayz);

      % % 座標軸(デバッグ用)
      % \draw[axis] (0,0,0) -- (3,0,0) node[anchor=west]{$x$};
      % \draw[axis] (0,0,0) -- (0,3,0) node[anchor=west]{$y$};
      % \draw[axis] (0,0,0) -- (0,0,3) node[anchor=west]{$z$};
      % % 点(デバッグ用)
      % \fill (O) circle[radius=2pt] node[above left] {O};
      % \fill (Axy) circle[radius=2pt] node[below left] {Axy};
      % \fill (Bxy) circle[radius=2pt] node[below right] {Bxy};
      % \fill (Cxy) circle[radius=2pt] node[above right] {Cxy};
      % \fill (Ayz) circle[radius=2pt] node[above left] {Ayz};
      % \fill (Byz) circle[radius=2pt] node[above right] {Byz};
      % \fill (Axz) circle[radius=2pt] node[above left] {Axz};
      % \fill (Axyz) circle[radius=2pt] node[above right] {Axyz};
    \end{scope}

    \draw[->, thick] ($(left.east)+(1em, 0)$) -- ($(left-|right.west)-(1em,0)$) node[pos=.5, above] {$dx \to 0$};
  \end{tikzpicture}
\end{center}

というわけで、$y=x^3$の導関数は、$y'=3x^2$となることがわかった。

\begin{equation}
  \frac{dy}{dx} = 3x^2
\end{equation}

\subsubsection{$y=x^n$の微分($n$が自然数の場合)}

$n$が自然数だとすると、$y=x^n$の微分は、$y=x^2$や$y=x^3$の場合と同じように考えられる。

\begin{equation}
  y + dy = \underbrace{(x+dx)(x+dx) \cdots (x+dx)}_{n\text{個}}
\end{equation}

右辺の$(x+dx)(x+dx) \cdots (x+dx)$を展開しようすると、次のような3種類のかけ算が発生する。

\begin{itemize}
  \item $x$どうしのかけ算
  \item $x$と$dx$のかけ算
  \item $dx$どうしのかけ算
\end{itemize}

つまり、右辺からは、

\begin{itemize}
  \item $x^n$の項が1つ
  \item $x^{n-1}dx$の項が$n$個
  \item $dx^n$の項が1つ
\end{itemize}

という項が現れることになる。

そして、$x^n$は左辺の$y$と相殺され、$dx^n$の項は高次の微小量として無視できる。

すると、残るのは次のような式になるだろう。

\begin{equation}
  dy = nx^{n-1}dx
\end{equation}

この式は、$y=\alpha x$という直線の式によく似ている。

高次の$dx$の項$dx^n$を無視し、1次の$dx$の項だけ残したのは、微分という計算が微小範囲における直線での近似であるからだ。

あくまでも微小範囲での直線の式であることを表すために、$x, y$を$dx, dy$として、$dy=\alpha dx$という形の式になっていると考えればよい。

\begin{theorem}{自然数の冪を持つ冪関数の導関数}
  \newline
  $n$が自然数のとき、$y=x^n$の導関数は次のようになる。
  \Large
  \begin{equation}
    \frac{dy}{dx} = nx^{n-1}
  \end{equation}
\end{theorem}

\subsubsection{$y=x^n$の微分($n$が整数の場合)}

指数法則を使うことで、$n$が負の整数の場合にも拡張することができる。

\vskip\baselineskip

まずは、$y=x^{-1}$の微分を考えてみよう。

指数法則より、$y=x^{-1}$は次のように変形できる。

\begin{equation}
  \begin{WithArrows}
    y  & = \dfrac{1}{x} \Arrow{両辺$\times x$} \\
    xy & = 1
  \end{WithArrows}
\end{equation}

微小変化を加えた微分の関係式を作って、次のように展開していく。

\begin{align}
  (x +dx)(y+dy)                                                                                   & = 1                           \\
  \eqnmarkbox[magenta]{Y1}{xy} + xdy + ydx + \fitLabelMath[BlueGreen][BlueGreen!40]{dydx}{高次の微小量} & = \eqnmarkbox[magenta]{Y2}{1}
\end{align}
\annotatetwo{below}{Y1}{Y2}{\bfseries 同じ}

ここで、微小量の掛け合わせである$dydx$は無視できるほど小さい。

また、$y=\dfrac{1}{x}$より、$xy=1$なので、左辺の$xy$と右辺の$1$は相殺される。

すると、残った式は、

\begin{equation}
  \begin{WithArrows}
    xdy + ydx       & = 0            \Arrow{$ydx$を移項} \\
    xdy             & = -ydx         \Arrow{両辺$\div dx$} \\
    x\dfrac{dy}{dx} & = -y           \Arrow{両辺$\div x$} \\
    \dfrac{dy}{dx}  & = -\dfrac{y}{x}
  \end{WithArrows}
\end{equation}

$y$が残ってしまっているので、$y=\dfrac{1}{x}$を代入すると、

\begin{align}
  \dfrac{dy}{dx} & = -\dfrac{1}{x^2} \\
                 & = -x^{-2}
\end{align}

これは、冪が自然数の場合の冪関数の微分$\dfrac{dy}{dx} = nx^{n-1}$において、$n = -1$を代入したものになっている。

\vskip\baselineskip

$n$が任意の負の整数の場合も、同様に考えられる。

$y=x^{-n}$を、$x^ny = 1$として、

\begin{equation}
  \begin{WithArrows}
    \underbrace{(x+dx)(x+dx) \cdots (x+dx)}_{n\text{個}}\times(y+dy)                           & = 1 \\
    ( x^n + nx^{n-1}dx + \fitLabelMath[BlueGreen][BlueGreen!40]{dx^n}{高次の微小量} ) \times (y+dy) & = 1 \Arrow{高次の微小量を無視} \\
    ( x^n + nx^{n-1}dx ) \times (y+dy)                                                        & = 1 \\
    \eqnmarkbox[magenta]{Y1}{x^ny} + x^ndy + nx^{n-1}ydx + \fitLabelMath[BlueGreen][BlueGreen!40]{nx^{n-1}dxdy}{高次の微小量}                                           & = \eqnmarkbox[magenta]{Y2}{1} \Arrow[jump=2]{相殺&無視}\\\\
    x^ndy + nx^{n-1}ydx & = 0
  \end{WithArrows}
\end{equation}
\annotatetwo{below}{Y1}{Y2}{\bfseries 同じ}

移項してさらに整理すると、

\begin{equation}
  \begin{WithArrows}
    x^ndy & = -nx^{n-1}ydx \Arrow{両辺$\div dx$} \\
    x^n\dfrac{dy}{dx} & = -nx^{n-1}y \Arrow{両辺$\times x^{-n}$} \\
    \dfrac{dy}{dx} & = -nx^{n-1}x^{-n}y \Arrow{$y=x^{-n}$} \\
    &= -nx^{n-1}x^{-n}x^{-n} \Arrow{指数法則$x^mx^n=x^{m+n}$} \\
    &= -nx^{-n-1}
  \end{WithArrows}
\end{equation}

これもやはり、冪が自然数の場合の冪関数の微分$\dfrac{dy}{dx} = nx^{n-1}$において、$n$を$-n$に置き換えたものになっている。

つまり、自然数(正の整数)だけでなく、負の整数も許容して、次のことがいえる。

\begin{theorem}{整数の冪を持つ冪関数の導関数}
  \newline
  $n$が整数のとき、$y=x^n$の導関数は次のようになる。
  \Large
  \begin{equation}
    \frac{dy}{dx} = nx^{n-1}
  \end{equation}
\end{theorem}

\subsubsection{$y=x^n$の微分($n$が実数の場合)}

$n$が有理数の場合はどうだろうか。実はこれも、指数法則によって拡張することができる。

$m$と$n$はどちらも自然数として、$y=x^{\frac{m}{n}}$の微分を考える。

\vskip\baselineskip

まず、$y=x^{\frac{m}{n}}$は、$y^n = x^m$とまったく同じ式である。

\begin{equation}
  \begin{WithArrows}
    y^n & = x^m \Arrow{両辺$\dfrac{1}{n}$乗} \\
    y & = x^{\frac{m}{n}}
  \end{WithArrows}
\end{equation}

というわけで、$y^n = x^m$を微小変化させて、展開してみよう。

\begin{equation}
  \underbrace{(y+dy)(y+dy) \cdots (y+dy)}_{n\text{個}} = \underbrace{(x+dx)(x+dx) \cdots (x+dx)}_{m\text{個}}
\end{equation}

ここで、$n$と$m$は自然数なのだから、自然数冪のときと同じように考えて、次のような式が残ることになる。

\begin{equation}
  ny^{n-1}dy = mx^{m-1}dx
\end{equation}

よって、$\dfrac{dy}{dx}$の式の$y$を含まない形を目指すと、

\begin{equation}
  \begin{WithArrows}
    \dfrac{dy}{dx} &= \dfrac{mx^{m-1}}{ny^{n-1}} \Arrow{$y=x^{\frac{m}{n}}$} \\
    &= \dfrac{mx^{m-1}}{nx^{\frac{m}{n}(n-1)}} \\
    &= \dfrac{mx^{m-1}}{nx^{m - \frac{m}{n}}} \Arrow{指数法則$x^{a+b} = x^a x^b$} \\
    &= \dfrac{mx^m x^{-1}}{nx^mx^{-\frac{m}{n}}} \Arrow{$x^m$で約分} \\
    &= \dfrac{mx^{-1}}{n x^{-\frac{m}{n}}} \\
    &= \dfrac{m}{n}\cdot\dfrac{x^{-1}}{x^{-\frac{m}{n}}} \Arrow{指数法則$\dfrac{a^m}{a^n}=a^{m-n}$} \\
    &= \dfrac{m}{n}\cdot x^{-1-\left(-\frac{m}{n}\right)} \\
    &= \dfrac{m}{n}\cdot x^{-1 + \frac{m}{n}} \\
    &= \dfrac{m}{n}x^{\frac{m}{n}-1}
  \end{WithArrows}
\end{equation}

これは、冪が自然数の場合の冪関数の微分$\dfrac{dy}{dx} = nx^{n-1}$において、$n$を$\dfrac{m}{n}$に置き換えたものになっている。

つまり、整数だけでなく、有理数に対しても同様の導関数の式が成り立つ。

\vskip\baselineskip

ここまで来ると、無理数はどうだろうか?という疑問が生まれるが、無理数への拡張は指数法則では対応できない。

無理数に対しては、極限操作によって同様の導関数の式を導くことができ、実数全体に対して同じ導関数の式が成り立つことが示される。

\begin{theorem}{冪関数の導関数}
  \newline
  $n$が実数のとき、$y=x^n$の導関数は次のようになる。
  \Large
  \begin{equation}
    \frac{dy}{dx} = nx^{n-1}
  \end{equation}
\end{theorem}

\subsection{合成関数の微分}

\begin{theorem}{合成関数の微分}
  \Large
  \begin{equation}
    \left( g(f(x)) \right)' = f'(x)g'(f(x))
  \end{equation}
\end{theorem}

\subsection{三角関数の微分}

\begin{center}
  \scalebox{2}{
    \begin{tikzpicture}[scale=3]
      \coordinate (A) at (1,0);
      \coordinate (O) at (0,0);
      \coordinate (C1) at (30:1cm);
      \coordinate (C2) at (40:1cm);
      \coordinate (H) at (C1 |- O);
      % Hのx座標とC2のy座標を持つ点
      \coordinate (H2) at (C2 -| H);

      % OとAを結ぶ線
      \draw (O) -- (A);
      % OとC1を結ぶ線
      \draw (O) -- (C1);
      % OとC2を結ぶ線
      \draw (O) -- (C2);

      % 三角形O-H-C1
      \draw[fill=myPurple, opacity=0.6] (O) -- (H) -- (C1) -- cycle;
      % 三角形C1-H-C2
      \draw[fill=myPurple, opacity=0.6] (C2) -- (H2) -- (C1) -- cycle;

      % 角A-O-C1を表す扇形
      \draw (A) -- (O) -- (C1) pic [fill=cyan!50, angle radius=9mm, "$\theta$"] {angle = A--O--C1};
      % 角C1-O-C2を表す扇形
      \draw (C1) -- (O) -- (C2) pic [fill=magenta!40, angle radius=9mm] {angle = C1--O--C2};
      % 角C1-H-C2を表す扇形
      \draw (H2) -- (C1) -- (C2) pic [fill=cyan!60, angle radius=2mm] {angle = H2--C1--C2};

      % 直角O-H-C1
      \draw (O) -- (H) -- (C1) pic [fill=lightgray, angle radius=1.25mm] {right angle = O--H--C1};
      % 直角O-C1-C2
      \draw (O) -- (C1) --(C2) pic [fill=lightgray, angle radius=0.9mm] {right angle = O--C1--C2};
      % 直角C1-H2-C2
      \draw (C1) -- (H2) -- (C2) pic [fill=lightgray, angle radius=0.75mm] {right angle = C1--H2--C2};

      % AからC1を結ぶ円弧
      \draw[cyan, thick] (A) arc (0:30:1cm) node [midway, above, right] {$\theta$};
      % C1からC2を結ぶ円弧
      \draw[magenta, thick] (C1) arc (30:40:1cm) node [midway,sloped, below] {$d\theta$};

      % C1からH2の高さを表すベクトル
      \draw[<->] ($(C1)+ (0.05, 0)$) -- ($(H2)+(0.05,0)$) node [midway, right] {$d\theta\cos \theta$};
      % C2からH2の幅を表すベクトル
      \draw[<->] ($(C2)+ (0, 0.05)$) -- ($(H2)+(0,0.05)$) node [midway, above] {$d\theta\sin \theta$};

      % OからC2の長さを表すベクトル
      \draw [<->, orange] to ($(O)!0.05!90:(C2)$) -- ($(C2)!-0.05!90:(O)$) node [midway, above, orange] { $1$ };
      % OからAの長さを表すベクトル
      \draw [<->, orange] to ($(O)-(0,0.05)$) -- ($(A)-(0,0.05)$) node [midway, below, orange] { $1$ };

      % % sin\thetaを表すベクトル
      % \draw [<->] to ($(C1)+(0.05,0)$) -- ($(H)+(0.05,0)$) node [midway, left] { $\sin \theta$ };
      % % cos\thetaを表すベクトル
      % \draw [<->] to ($(H)-(0,0.05)$) -- ($(O)-(0,0.05)$) node [midway, below] { $\cos \theta$ };
    \end{tikzpicture}
  }
\end{center}

\subsection{ネイピア数}

指数関数を定義した際に、「どんな数も$0$乗したら$1$になる」と定義した。

つまり、指数関数$y=a^x$において、$x=0$での関数の値は$1$である。

ここでさらに、$x=0$でのグラフの傾きも$1$となるような$a$を探し、その値をネイピア数と呼ぶことにする。

\begin{definition}{ネイピア数(自然対数の底)}
  \newline
  指数関数$y=a^x$において、$x=0$での接線の傾きが$1$となるような底$a$の値をネイピア数と呼び、$e$と表す。
\end{definition}

この定義では、「$x=0$では関数の値も傾きも等しく$1$になる」という、$x=0$での振る舞いにしか言及していない。

だが、実はネイピア数を底とする指数関数は、「微分しても変わらない(すべての$x$において、関数の値と傾きが一致する)」という性質を持つ。

\subsection{ネイピア数を底とする指数関数の微分}

指数関数$y=e^x$の微分は、微分の定義から次のように計算できる。

\begin{align}
  \dfrac{d}{dx}e^x & = \lim_{\Delta x \to 0} \dfrac{e^{x+\Delta x} - e^x}{\Delta x}         \\
                   & = \lim_{\Delta x \to 0} \dfrac{e^x \cdot e^{\Delta x} - e^x}{\Delta x} \\
                   & = \lim_{\Delta x \to 0} \dfrac{e^x \cdot (e^{\Delta x} - 1)}{\Delta x} \\
                   & = e^x \cdot \lim_{\Delta x \to 0} \dfrac{e^{\Delta x} - 1}{\Delta x}
\end{align}

ここで、$\displaystyle\lim_{\Delta x \to 0} \dfrac{e^{\Delta x} - 1}{\Delta x}$は$x$によらない定数であり、

\begin{align}
  \lim_{\Delta x \to 0} \dfrac{e^{\Delta x} - 1}{\Delta x} & = \lim_{\Delta x \to 0} \dfrac{e^{0 + \Delta x} - e^0}{\Delta x}
\end{align}

というように、これは$x=0$における傾き(導関数に$x=0$を代入したもの)を表している。

そもそも、ネイピア数$e$の定義は「$x=0$での$e^x$の傾きが$1$」というものだったので、

\begin{equation}
  \lim_{\Delta x \to 0} \dfrac{e^{\Delta x} - 1}{\Delta x} = 1
\end{equation}

となり、「$e^x$は微分しても変わらない」という性質が導かれる。

\begin{equation}
  \dfrac{d}{dx}e^x = e^x
\end{equation}

\begin{theorem}{ネイピア数を底とする指数関数の微分}
  \newline
  ネイピア数を底とする指数関数は、微分しても変わらない関数である。
  \LARGE
  \begin{equation}
    \dfrac{d}{dx}e^x = e^x
  \end{equation}
\end{theorem}

\section{1変数関数の積分}

積分とは、「部分を積み重ねる」演算である。

微小部分を調べる微分と、微小部分を積み重ねる積分は、互いに逆の操作になっている。

\subsection{区分求積法:面積の再定義}

長方形の面積は、なぜ「縦$\times$横」で求められるのだろうか?

そこには、長方形の横幅分の長さを持つ線分を、長方形の高さに達するまで積み重ねるという発想がある。

\vskip\baselineskip

面積の計算を「線を積み重ねる」という発想で捉えると、あらゆる形状の面積を考えることができる。

長方形では、積み重ねる線の長さは一定だが、他の形状では、積み重ねる線の長さが変化する。

積み重ねるべき線の長さを、関数で表すことができたら…

\froufrou

関数$y=f(x)$が与えられたとき、高さ$f(x)$の線分を$a$から$b$までの区間で積み重ねることで、$x$軸とグラフに挟まれた部分の面積を求めることを考える。

\begin{center}
  \begin{tikzpicture}[
      scale=0.9,
      declare function={f(\x)=((1/3)*(\x)^(3)-3*(\x)^(2)+8*\x-3;},
      lnode/.style={text height=1em}
    ]
    % 積分区間全体を塗りつぶす
    \draw[fill=cyan!30, draw=cyan!70!gray] plot[domain=1:5,samples=167,variable=\x] ({\x},{f(\x)}) -- (5,0) -| cycle;

    % 積分区間の下端のラベル
    \node [anchor=north,lnode] at (1,0) {$a$};
    % 積分区間の上端のラベル
    \node[anchor=north,lnode] at (5,0pt) {$b$};

    % x軸とy軸
    \draw [axis] (-0.5,0) -- (6,0) node (xaxis) [below] {$x$};
    \draw [axis] (0,-0.5) -- (0,5) node [left] {$y$};

    % 原点
    \node [below left] at (0,0) {$O$};

    % 関数のグラフ
    \draw[domain=.5:5.3,samples=200,variable=\x,BurntOrange,very thick] plot ({\x},{f(\x)}) node [above right] {$y=f(x)$};
  \end{tikzpicture}
\end{center}

この考え方は、面積を求めたい部分を長方形に分割し、長方形の幅を限りなく$0$に近づけるという操作で表現できる。

\begin{center}
  \begin{tikzpicture}[
      declare function={f(\x)=((1/3)*(\x)^(3)-3*(\x)^(2)+8*\x-3;},
      lnode/.style={text height=1em}
    ]
    \begin{scope}[scale=0.9, local bounding box=left]
      \def\N{5}
      \pgfmathtruncatemacro{\M}{\N/4}

      \coordinate (start) at (.8,{f(.8)});

      \ifnum\N<22
        \foreach \X [remember=\X as \LastX (initially 0)] in {1,...,\N}
          {
            % 矩形
            \draw[fill=cyan!30, draw=cyan!70!gray] (1+\LastX*4/\N,0) rectangle (1+\X*4/\N,{f(1+\LastX*4/\N)});
            % 矩形の左上の頂点
            \draw[fill=cyan, draw=cyan] (1+\LastX*4/\N,{f(1+\LastX*4/\N)}) circle (2pt) ;

            % x軸上のラベル
            \path (1+\LastX*4/\N,0pt) coordinate (x\X);
            \ifnum\X=1
              % 積分区間の下端のラベル
              \node[anchor=north east,xshift=0.75em,lnode] at (1+\LastX*4/\N,0pt) {$a=x_{\X}$};
            \else
              % 積分区間内の各点のラベル
              \pgfmathtruncatemacro{\itest}{mod(\X,\M)}
              % 4等分した点のみラベルをつける
              \ifnum\itest=0
                \pgfmathsetmacro{\dist}{4-\LastX*4/\N}
                % 5pt以上離れている場合のみラベルをつける
                \ifdim\dist cm>5pt
                  \node [anchor=north,lnode] at (1+\LastX*4/\N,0pt) {$x_{\X}$};
                \fi
              \fi
            \fi
          }

        % \Delta x の幅を示す矢印
        \draw[<->] (x2|- 0,-1)--(x3|- 0,-1) node[above,midway] {$\Delta x$};
      \else
        % 22個以上の場合は積分区間全体を塗りつぶす
        \draw[fill=cyan!30, draw=cyan!70!gray] plot[domain=1:5,samples=167,variable=\x] ({\x},{f(\x)}) -- (5,0) -| cycle;

        % 積分区間の下端のラベル
        \node [anchor=north,lnode] at (1,0) {$a$};
      \fi

      \coordinate (end) at (5.05,{f(5.05)});

      % 積分区間の上端のラベル
      \node[anchor=north,lnode] at (5,0pt) {$b$};

      % 積分区間の上端におけるグラフの高さを示す線
      \draw [draw=cyan!70!gray] (5,0)--(5,{f(5)});

      % x軸とy軸
      \draw [axis] (-0.5,0) -- (6,0) node (xaxis) [below] {$x$};
      \draw [axis] (0,-0.5) -- (0,5) node [left] {$y$};

      % 原点
      \node [below left] at (0,0) {$O$};

      % 関数のグラフ
      \draw[domain=.5:5.3,samples=200,variable=\x,BurntOrange,very thick] plot ({\x},{f(\x)});
    \end{scope}

    \begin{scope}[scale=0.9, xshift=0.6\textwidth, local bounding box=right]
      \def\N{14}
      \pgfmathtruncatemacro{\M}{\N/4}

      \coordinate (start) at (.8,{f(.8)});

      \ifnum\N<22
        \foreach \X [remember=\X as \LastX (initially 0)] in {1,...,\N}
          {
            % 矩形
            \draw[fill=cyan!30, draw=cyan!70!gray] (1+\LastX*4/\N,0) rectangle (1+\X*4/\N,{f(1+\LastX*4/\N)});
            % 矩形の左上の頂点
            \draw[fill=cyan, draw=cyan] (1+\LastX*4/\N,{f(1+\LastX*4/\N)}) circle (2pt) ;

            % x軸上のラベル
            \path (1+\LastX*4/\N,0pt) coordinate (x\X);
            \ifnum\X=1
              % 積分区間の下端のラベル
              \node[anchor=north east,xshift=0.75em,lnode] at (1+\LastX*4/\N,0pt) {$a=x_{\X}$};
            \else
              % 積分区間内の各点のラベル
              \pgfmathtruncatemacro{\itest}{mod(\X,\M)}
              % 4等分した点のみラベルをつける
              \ifnum\itest=0
                \pgfmathsetmacro{\dist}{4-\LastX*4/\N}
                % 5pt以上離れている場合のみラベルをつける
                \ifdim\dist cm>5pt
                  \node [anchor=north,lnode] at (1+\LastX*4/\N,0pt) {$x_{\X}$};
                \fi
              \fi
            \fi
          }

        % \Delta x の幅を示す矢印
        \draw[<->] (x2|- 0,-1)--(x3|- 0,-1) node[above,midway] {$\Delta x$};
      \else
        % 22個以上の場合は積分区間全体を塗りつぶす
        \draw[fill=cyan!30, draw=cyan!70!gray] plot[domain=1:5,samples=167,variable=\x] ({\x},{f(\x)}) -- (5,0) -| cycle;

        % 積分区間の下端のラベル
        \node [anchor=north,lnode] at (1,0) {$a$};
      \fi

      \coordinate (end) at (5.05,{f(5.05)});

      % 積分区間の上端のラベル
      \node[anchor=north,lnode] at (5,0pt) {$b$};

      % 積分区間の上端におけるグラフの高さを示す線
      \draw [draw=cyan!70!gray] (5,0)--(5,{f(5)});

      % x軸とy軸
      \draw [axis] (-0.5,0) -- (6,0) node (xaxis) [below] {$x$};
      \draw [axis] (0,-0.5) -- (0,5) node [left] {$y$};

      % 原点
      \node [below left] at (0,0) {$O$};

      % 関数のグラフ
      \draw[domain=.5:5.3,samples=200,variable=\x,BurntOrange,very thick] plot ({\x},{f(\x)});
    \end{scope}

    \draw[->, thick] ($(left.east)+(1em, 0)$) -- ($(left-|right.west)-(1em,0)$) node[pos=.5, above] {$\Delta x$小};
  \end{tikzpicture}
\end{center}

$a \leq x \leq b$の区間を$n$等分して、$x_1, x_2, \ldots, x_n$とする。

分割された各長方形は、幅が$\Delta x$で、高さが$f(x)$であるので、各長方形の面積は次のように表せる。

\begin{equation}
  \Delta S = f(x) \cdot \Delta x
\end{equation}

どんどん$\Delta x$を小さくしていくと、細かい長方形分割で、面積を求めたい図形を近似できる。

\begin{center}
  \begin{tikzpicture}[
      scale=1.2,
      declare function={f(\x)=((1/3)*(\x)^(3)-3*(\x)^(2)+8*\x-3;},
      lnode/.style={text height=1em}
    ]
    \def\N{40}
    \pgfmathtruncatemacro{\M}{\N/4}

    \coordinate (start) at (.8,{f(.8)});

    \foreach \X [remember=\X as \LastX (initially 0)] in {1,...,\N}
    {
    % 矩形
    \draw[fill=cyan!30, draw=cyan!70!gray] (1+\LastX*4/\N,0) rectangle (1+\X*4/\N,{f(1+\LastX*4/\N)});
    % 矩形の左上の頂点
    \draw[fill=cyan, draw=cyan] (1+\LastX*4/\N,{f(1+\LastX*4/\N)}) circle (1pt) ;

    % x軸上のラベル
    \path (1+\LastX*4/\N,0pt) coordinate (x\X);
    \ifnum\X=1
      % 積分区間の下端のラベル
      \node[anchor=north east,xshift=0.75em,lnode] at (1+\LastX*4/\N,0pt) {$a=x_{\X}$};
    \else
      % 積分区間内の各点のラベル
      \pgfmathtruncatemacro{\itest}{mod(\X,\M)}
      % 4等分した点のみラベルをつける
      \ifnum\itest=0
        \pgfmathsetmacro{\dist}{4-\LastX*4/\N}
        % 5pt以上離れている場合のみラベルをつける
        \ifdim\dist cm>5pt
          \node [anchor=north,lnode] at (1+\LastX*4/\N,0pt) {$x_{\X}$};
        \fi
      \fi
    \fi
    }

    \coordinate (end) at (5.05,{f(5.05)});

    % 積分区間の上端のラベル
    \node[anchor=north,lnode] at (5,0pt) {$b$};

    % 積分区間の上端におけるグラフの高さを示す線
    \draw [draw=cyan!70!gray] (5,0)--(5,{f(5)});

    % x軸とy軸
    \draw [axis] (-0.5,0) -- (6,0) node (xaxis) [below] {$x$};
    \draw [axis] (0,-0.5) -- (0,5) node [left] {$y$};

    % 原点
    \node [below left] at (0,0) {$O$};

    % 関数のグラフ
    \draw[domain=.5:5.3,samples=200,variable=\x,BurntOrange,very thick] plot ({\x},{f(\x)}) node [above right] {$y=f(x)$};
  \end{tikzpicture}
\end{center}

つまり、求めたい面積は、分割した長方形の面積をすべて足し合わせることで近似できる。

\begin{equation}
  S \approx \sum_{i=1}^{n} f(x_i) \cdot \Delta x
\end{equation}

$\Delta x \to 0$の果てでは、幅を持たなくなった長方形は線分とみなせるので、もはや近似ですらなくなるだろう。

\begin{equation}
  S = \lim_{\Delta x \to 0} \sum_{i=1}^{n} f(x_i) \cdot \Delta x
\end{equation}

このような考え方は、区分求積法と呼ばれる。

\subsection{定積分}

ここで、区間$a \leq x \leq b$における関数$y=f(x)$と$x$軸の間の面積$S$を求める式を、次のように表記する。

\begin{equation}
  S = \int_{a}^{b} f(x) \, dx
\end{equation}

$\sum$は離散的な和を表す記号であり、例えば$\displaystyle\sum_{i=0}^n$であれば、$i$を$1$ずつ増やして$n$に達するまで足し合わせることを意味する。

一方、ここで新たに導入した$\int$は連続的な和を表す記号であり、微小変化を繰り返しながら足し合わせることを意味する。

\vskip\baselineskip

$\sum$は間隔を取って足し合わせるのに対し、$\int$は間隔を限りなく小さくして足し合わせる。

足し合わせる間隔を限りなく小さくするという操作は、極限を取る操作に相当するので、$\sum$の極限を取ったもの$\displaystyle \lim \sum$をまとめて$\int$という記号で表記したと捉えることができる。

さらに、$\displaystyle\lim_{\Delta x \to 0}$とした果ての$\Delta x$は、微小変化を意味する$dx$と書き換えられている。

\begin{definition}{定積分}
  \newline
  $a \leq x \leq b$の区間内における関数$f(x)$のグラフと$x$軸の間の領域の符号付き面積を求める演算を定積分と定義し、次のように表記する。
  \LARGE
  \begin{equation}
    \int_{a}^{b} f(x) dx
  \end{equation}
  \normalsize
  このとき、$f(x)$を被積分関数と呼ぶ。
\end{definition}

$f(x)$の値が負になる区間では、定積分の値も負になるため、定積分は符号付き面積を表す。

\begin{center}
  \begin{tikzpicture}
    \begin{axis}[
        name=myaxis,
        axis y line = none,
        axis x line = none,
        xmin=-3, xmax=3,
        ymin=-10, ymax=10,
        declare function={
            fn(\x) = 3*\x^3 - \x^2 - 10*\x;
          }
      ]
      % 関数f(x)のプロット
      \addplot [domain=-2:2.25, samples=100, name path=f, very thick, color=BurntOrange]
      {fn{\x}};

      % x軸
      \addplot [name path=xaxis] {0};

      % xaxisとfの交点
      \path [name intersections={of=f and xaxis, by={I1,I2,I3}}];

      % x軸上のラベル
      \node [below, name=a, magenta, text height=0.75em] at ($(I1) + 1/3*(1cm,0)$) {$a_1$};
      \node [below, name=b, magenta, text height=0.75em] at ($(I2) - 1/3*(1cm,0)$) {$b_1$};
      \node [above, name=c, cyan, text height=0.75em] at ($(I2) + 2/5*(1cm,0)$) {$a_2$};
      \node [above, name=d, cyan, text height=0.75em] at ($(I3) - 2/5*(1cm,0)$) {$b_2$};

      % 各点のx座標を取り出してレジスタに保存
      \pgfplotsextra{
        \pgfplotspointgetcoordinates{(a)}
        \pgfkeysgetvalue{/data point/x}{\ax}

        \pgfplotspointgetcoordinates{(b)}
        \pgfkeysgetvalue{/data point/x}{\bx}

        \pgfplotspointgetcoordinates{(c)}
        \pgfkeysgetvalue{/data point/x}{\cx}

        \pgfplotspointgetcoordinates{(d)}
        \pgfkeysgetvalue{/data point/x}{\dx}
      }

      % [a, b]区間の定積分
      \addplot [
        opacity=0.8, postaction={pattern=north east lines}, magenta!30, pattern color=magenta!80!gray] fill between [
          of=f and xaxis, soft clip={domain=\ax:\bx},
        ];
      % [c, d]区間の定積分
      \addplot [
        opacity=0.8, postaction={pattern=north east lines}, cyan!30, pattern color=cyan!80!gray] fill between [
          of=f and xaxis, soft clip={domain=\cx:\dx},
        ];
    \end{axis}

    % x軸
    \draw [axis] (myaxis.west) -- (myaxis.east) node [right] {$x$};
  \end{tikzpicture}
\end{center}

\chapter{複素数と複素関数}

\section{複素平面}

複素数は、実部(Real Part)と虚部(Imaginary Part)という2つの数から成る。

そのため、実部を横軸に、虚部を縦軸にとった平面を考え、1つの複素数をこの平面上の1点として表すことができる。

\begin{definition}{複素平面}
  実部を横軸、虚部を縦軸にとった平面を複素平面と呼ぶ。
\end{definition}

\begin{center}
  \begin{tikzpicture}
    \def\ang{35} % 偏角
    \def\r{3} % 半径

    % x軸の描画
    \draw [axis] (-1,0) -- (4,0) node[right] {Re};
    % y軸の描画
    \draw [axis] (0,-1) -- (0,3) node[above] {Im};

    % 原点O
    \coordinate (O) at (0,0) node[below left] {O};

    % 座標軸上の点
    \coordinate[label=below right:$a$] (X) at ({\r*cos(\ang)}, 0);
    \coordinate[label=left:$b$] (Y) at (0, {\r*sin(\ang)});

    % 複素数z
    \coordinate (Z) at ({\r*cos(\ang)}, {\r*sin(\ang)});

    % 補助線の描画
    \draw[auxline] (X) -- (Z);
    \draw[auxline] (Y) -- (Z);

    % 複素数を表す数式の描画
    \node[above right] at (Z) {$a+ib$};
  \end{tikzpicture}
\end{center}

\section{複素数の絶対値}

\begin{definition}{複素数の絶対値}
  \newline
  複素平面において、原点から複素数$z$までの距離を複素数$z$の絶対値と定義する。\\
  この距離は三平方の定理から求められ、$|z|$と表す。
  \LARGE
  \begin{equation}
    |z| \coloneqq \sqrt{x^2 + y^2}
  \end{equation}
\end{definition}

\begin{center}
  \begin{tikzpicture}
    \def\ang{35} % 偏角
    \def\r{3} % 半径

    % x軸の描画
    \draw [axis] (-1,0) -- (4,0) node[right] {Re};
    % y軸の描画
    \draw [axis] (0,-1) -- (0,3) node[above] {Im};

    % 原点O
    \coordinate (O) at (0,0) node[below left] {O};

    % 座標軸上の点
    \coordinate[label=below right:$x$] (X) at ({\r*cos(\ang)}, 0);
    \coordinate[label=left:$y$] (Y) at (0, {\r*sin(\ang)});

    % 複素数z
    \coordinate (Z) at ({\r*cos(\ang)}, {\r*sin(\ang)});

    % 三角形の描画
    \draw[fill=orange, opacity=0.4] (O) -- (X) -- (Z) -- cycle;

    % 三角形の辺のラベルの描画
    \draw (O) -- (X) node[midway, below] {$x$};
    \draw (X) -- (Z) node[midway, right] {$y$};
    \draw (O) -- (Z) node[midway, above, sloped] {$\sqrt{x^2 + y^2}$};

    % ベクトルの描画
    \draw[vector] (O) -- (Z);

    % 補助線の描画
    \draw[auxline] (X) -- (Z);
    \draw[auxline] (Y) -- (Z);

    % 複素数を表す数式の描画
    \node[above right] at (Z) {$z$};
  \end{tikzpicture}
\end{center}

\section{複素数の極形式による表現}

\begin{definition}{極形式}
  \newline
  複素数$z$は、絶対値$r$と偏角$\theta$を用いて次のように表すことができる。
  \LARGE
  \begin{equation}
    z = r(\cos\theta + i\sin\theta)
  \end{equation}
\end{definition}

\begin{center}
  \begin{tikzpicture}
    \def\ang{35} % 偏角
    \def\r{3} % 半径

    % x軸の描画
    \draw [axis] (-1,0) -- (4,0) node[right] {Re};
    % y軸の描画
    \draw [axis] (0,-1) -- (0,3) node[above] {Im};

    % 原点O
    \coordinate (O) at (0,0) node[below left] {O};

    % 座標軸上の点
    \coordinate[label=below right:$x$] (X) at ({\r*cos(\ang)}, 0);
    \coordinate[label=left:$y$] (Y) at (0, {\r*sin(\ang)});

    % 複素数z
    \coordinate (Z) at ({\r*cos(\ang)}, {\r*sin(\ang)});

    % 三角形の描画
    \draw[fill=orange, opacity=0.4] (O) -- (X) -- (Z) -- cycle;

    % 三角形の辺のラベルの描画
    \draw (O) -- (X) node[midway, below] {$r\cos\theta$};
    \draw (X) -- (Z) node[midway, right] {$r\sin\theta$};

    % ベクトルの描画
    \draw[vector] (O) -- (Z);

    % 補助線の描画
    \draw[auxline] (X) -- (Z);
    \draw[auxline] (Y) -- (Z);

    % 偏角を表す円弧の描画
    \pic [draw, ->, "$\theta$", angle eccentricity=1.5] {angle = X--O--Z};

    % 半径を表すラベルの描画
    \draw (O) -- (Z) node[midway, above] {$r$};

    % 複素数を表す数式の描画
    \node[above right] at (Z) {$z=r\cos\theta + r\sin\theta$};
  \end{tikzpicture}
\end{center}

\section{偏角と主値}

$x=r\cos\theta$、$y=r\sin\theta$に、$r=\sqrt{x^2 + y^2}$を代入して整理した関係式から、偏角を改めて定義する。

\begin{definition}{偏角}
  \newline
  複素数$z$を極形式で表現すると、\\
  $\cos\theta=\dfrac{x}{\sqrt{x^2 + y^2}}$、$\sin\theta=\dfrac{y}{\sqrt{x^2 + y^2}}$という関係が成り立つ。\\\\
  この関係を満たす$\theta$を偏角と呼び、次のように表す。
  \LARGE
  \begin{equation}
    \arg z \coloneqq \theta
  \end{equation}
\end{definition}

ここで、$\theta$を整数回$2\pi$シフトさせても(何周回っても)、複素数$z$の値は変わらない。

つまり、1つの複素数に対して偏角の値は複数考えられるので、次のような主値を定義する。

\begin{definition}{偏角の主値}
  \newline
  $0\leq \theta \leq 2\pi$、もしくは$-\pi < \theta \leq \pi$の範囲にある偏角を偏角の主値と呼び、次のように表す。
  \LARGE
  \begin{equation}
    \Arg z \coloneqq \theta
  \end{equation}
\end{definition}

\section{共役複素数}

\begin{definition}{共役複素数}
  \newline
  複素数$z=x+iy$に対して,その共役複素数$\overline{z}$を次のように定義する。
  \LARGE
  \begin{equation}
    \overline{z}\coloneqq x-iy
  \end{equation}
\end{definition}

\begin{center}
  \begin{tikzpicture}
    \def\ang{35} % 偏角
    \def\r{3} % 半径

    % x軸の描画
    \draw [axis] (-1,0) -- (4,0) node[right] {Re};
    % y軸の描画
    \draw [axis] (0,-3) -- (0,3) node[above] {Im};

    % 原点O
    \coordinate (O) at (0,0) node[below left] {O};

    % 座標軸上の点
    \coordinate[label=below right:$x$] (X) at ({\r*cos(\ang)}, 0);
    \coordinate[label=left:$y$] (Y) at (0, {\r*sin(\ang)});
    \coordinate[label=left:$-y$] (-Y) at (0, {-\r*sin(\ang)});

    % 複素数z
    \coordinate (Z) at ({\r*cos(\ang)}, {\r*sin(\ang)});
    % 共役複素数z*
    \coordinate (Z*) at ({\r*cos(\ang)}, {-\r*sin(\ang)});

    % ベクトルの描画
    \draw[vector] (O) -- (Z);
    \draw[vector] (O) -- (Z*);

    % 補助線の描画
    \draw[auxline] (X) -- (Z);
    \draw[auxline] (X) -- (Z*);
    \draw[auxline] (Y) -- (Z);
    \draw[auxline] (-Y) -- (Z*);

    % 偏角を表す円弧の描画
    \pic [draw, ->, "$\theta$", angle eccentricity=1.5] {angle = X--O--Z};
    \pic [draw, <-, "$-\theta$", angle eccentricity=1.7] {angle = Z*--O--X};

    % 半径を表すラベルの描画
    \draw (O) -- (Z) node[midway, above] {$r$};
    \draw (O) -- (Z*) node[midway, below] {$r$};

    % 点の描画
    \fill (O) circle (1.2pt);
    \fill (Z) circle (1.2pt);
    \fill (Z*) circle (1.2pt);

    % 複素数を表す数式の描画
    \node[above right] at (Z) {$z=x+iy$};
    \node[below right] at (Z*) {$\overline{z}=x-iy$};
  \end{tikzpicture}
\end{center}

\begin{theorem}{共役複素数と絶対値}
  \newline
  複素数$z$とその共役複素数$\overline{z}$の積は、$z$の絶対値の二乗に等しい。
  \LARGE
  \begin{equation}
    z\overline{z} = |z|^2
  \end{equation}
\end{theorem}

\begin{proof}
  複素数$z=x+iy$とその共役複素数$\overline{z}=x-iy$の積を計算する。
  \begin{align*}
    z\overline{z} & = (x+iy)(x-iy)             \\
                  & = x^2 - ixy + ixy - i^2y^2 \\
                  & = x^2 + y^2                \\
                  & = |z|^2
  \end{align*}
\end{proof}

\section{オイラーの公式}

\begin{center}
  \def\xang{-13}
  \def\zang{45}
  \begin{tikzpicture}[x=(\xang:0.9), y=(90:0.9), z=(\zang:1.1)]
    \def\xmax{8.8}         % max x axis
    \def\ymin{-1.5}        % min y axis
    \def\ymax{1.6}         % max y axis
    \def\zmax{1.6}         % max z axis
    \def\xf{1.17*\xmax}    % x position frequency axis
    \def\A{(0.70*\ymax)}   % amplitude
    \def\T{(0.335*\xmax)}  % period
    \def\w{\zmax/11.2}     % spacing components
    \def\ang{47}           % angle
    \def\s{\ang/360*\T}    % time component
    \def\x{\A*cos(\ang)}   % real component
    \def\y{\A*sin(\ang)}   % imaginary component
    \def\N{100}            % number of samples
    \def\tick#1#2{\draw[thick] (#1) ++ (#2:0.12) --++ (#2-180:0.24)}

    % COMPLEX PLANE
    \begin{scope}[shift={(-1.6*\zmax,0,0)}]
      % 複素平面の枠
      \draw[black,fill=white,opacity=0.3,zy-plane](-1.25*\zmax,-1.25*\ymax) rectangle (1.4*\zmax,1.25*\ymax);
      % Im axis
      \draw[axis] (0,\ymin,0) -- (0,\ymax+0.02,0) node[pos=1,left=0,zy-plane] {Im};
      % Re axis
      \draw[axis] (0,0,-\zmax) -- (0,0,\zmax+0.02) node[right=1,below=0,zy-plane] {Re} coordinate (X);
      % 複素平面上の単位円
      \draw[plotline,zy-plane] (0,0) circle(0.991*\A) coordinate (O);
      % 単位円上の点P
      \fill[red,zy-plane] (\ang:{\A}) circle(0.07) coordinate(P);
      \node[blue,zy-plane,anchor=south west,scale=0.7] at (P) {$z(t)=Ae^{i\omega t}$};
      % 動径OP
      \draw[vector,thick,zy-plane] (0,0) -- (\ang:{\A-0.03}) coordinate (R);
      % 偏角を表す円弧
      \draw pic[-{>[flex'=1]},draw=blue,angle radius=14,angle eccentricity=1,"$\omega t$"{above=0,right=-0.5,yslant=0.69,scale=0.8},blue,zy-plane] {angle = X--O--R};
      % Re軸上の目盛り
      \tick{0,0,{\A}}{90};
      \tick{0,0,{-\A}}{90};
      % Im軸上の目盛り
      \tick{0,{\A},0}{\zang};
      \tick{0,{-\A},0}{\zang};
    \end{scope}

    % IMAGINARY
    \begin{scope}[shift={(0,0,1.9*\zmax)}]
      % sinを描く平面の枠
      \draw[black,fill=white,opacity=0.3,xy-plane](-0.5*\ymax,-1.2*\ymax) rectangle (1.10*\xmax,1.25*\ymax);
      % t axis
      \draw[axis] (-0.3*\ymax,0,0) -- (\xmax,0,0) node[below right,xy-plane] {$t$ [s]};
      % Im axis
      \draw[axis] (0,\ymin,0) -- (0,\ymax,0) node[left,xy-plane] {Im};
      % sin関数のグラフ
      \draw[plotline,samples=\N,smooth,variable=\t,domain=-0.05*\T:0.95*\xmax] plot(\t,{\A*sin(360/\T*\t)},0);
      % sin関数上の点I
      \fill[red,xy-plane] ({\s},{\y}) circle(0.07) coordinate(I);
      % 点Iでの波の高さを示すベクトル
      \draw[vector,thick,xy-plane] ({\s},0) --++ (0,{\y-0.03});
      \node[xy-plane,below] at ({\s},0) {$\omega t$};
      % 波の高さを表すIm軸上の目盛り
      \tick{0,{\A},0}{180};
      \tick{0,{-\A},0}{180};
      % 周期を示す目盛り
      \tick{{\T},0,0}{90} node[right=0,below,xy-plane] {\contour{white}{$T$}};
      \tick{{2*\T},0,0}{90} node[right=0,below,xy-plane] {\contour{white}{$2T$}};
      % 関数を表す数式
      \node[blue,below=0,xy-plane] at (0.4*\xmax,1.15*\ymax,0) {$y(t)=A\sin(\omega t)$};
    \end{scope}

    % REAL
    \begin{scope}[shift={(0,-1.8*\zmax,0)}]
      % cosを描く平面の枠
      \draw[black,fill=white,opacity=0.3,xz-plane] (-0.5*\ymax,-1.4*\ymax) rectangle (1.10*\xmax,1.25*\ymax);
      % t axis
      \draw[axis] (-0.3*\ymax,0,0) -- (\xmax,0,0) node[below right,xz-plane] {$t$ [s]};
      % Re axis
      \draw[axis] (0,0,-\zmax) -- (0,0,\zmax) node[left=-1,xz-plane] {Re};
      % cos関数のグラフ
      \draw[plotline,samples=\N,smooth,variable=\t,domain=-0.05*\T:0.95*\xmax] plot(\t,0,{\A*cos(360/\T*\t)});
      % cos関数上の点R
      \fill[red,xz-plane] ({\s},{\x}) circle(0.07) coordinate(R);
      % 点Rでの波の高さを示すベクトル
      \draw[vector,thick,xz-plane] ({\s},0) --++ (0,{\x-0.03});
      \node[xz-plane,below] at ({\s},0) {$\omega t$};
      % 波の高さを表すRe軸上の目盛り
      \tick{0,0,{\A}}{180};
      \tick{0,0,{-\A}}{180};
      % 周期を示す目盛り
      \tick{{\T},0,0}{\zang} node[below,xz-plane] {$T$};
      \tick{{2*\T},0,0}{\zang} node[below,xz-plane] {$2T$};
      % 関数を表す数式
      \node[blue,above=0,xz-plane] at (0.3*\xmax,-\ymax,0) {$x(t)=A\cos(\omega t)$};
    \end{scope}

    % COMPONENTS
    % 単位円上の点P、Im軸上の点I、Re軸上の点Rを結ぶ線
    \draw[red!80!black,dashed] (P) -- ({\s},{\y},{\x}) (I) -- ({\s},{\y},{\x+0.05}) ({\s},{\y-0.06},{\x}) -- (R);
    % 空間上に写したt軸
    \draw[axis,black,thick] (-0.1*\ymax,0,0) -- (\xmax,0,0) node[below right] {$t$ [s]};
    % 空間上に写したIm軸
    \draw[axis,black,thick] (0,\ymin,0) -- (0,\ymax+0.02,0) node[above] {Im};
    % 空間上に写したRe軸
    \draw[axis,black,thick] (0,0,-\zmax) -- (0,0,\zmax+0.02);
    \node[right=0.5em,below] at (0,0,\zmax+0.02) {$Re$};
    % 空間上のグラフ
    \foreach \i [evaluate={\tmin=max(-0.05*\T,(\i-0.05)*\T); \tmax=min(0.95*\xmax,(\i+1)*\T);}] in {0,...,2} {
        \draw[plotline,samples=0.4*\N,smooth,variable=\t] plot[domain=\tmin:\tmax](\t,{\A*sin(360/\T*\t)},{\A*cos(360/\T*\t)});
      }
    % 単位円上の点P、Im軸上の点I、Re軸上の点Rを結ぶ線が交わる点Z
    \fill[red] ({\s},{\y},{\x}) circle(0.07) coordinate(Z);
    % 空間上に動径OPを写したベクトル
    \draw[vector,thick] ({\s},0,0) --++ (0,{\y-0.03},{\x-0.03});
    % 周期を表すt軸上の目盛り
    \tick{{\T},0,0}{90};
    \tick{{2*\T},0,0}{90};
    % Re軸上の目盛り
    \tick{0,0,{\A}}{90};
    \tick{0,0,{-\A}}{90};
    % Im軸上の目盛り
    \tick{0,{\A},0}{\zang};
    \tick{0,{-\A},0}{\zang};
  \end{tikzpicture}
\end{center}

\chapter{フーリエ解析}

\section{波の2つの捉え方}

波は2つの捉え方ができる。

\begin{itemize}
  \item 空間的に捉える波:波の形そのもの
  \item 時間的に捉える波:波の振動
\end{itemize}

\subsection{空間的に捉える波}

波とは、一定の間隔で同じ形が繰り返されるものである。

空間的に捉える波は、まさにその波の形そのもので、波の形を位置$x$の関数として表す。

\begin{definition}{波長}
  波を構成する最小パターンの幅を波長と呼び、$\lambda$で表す。
\end{definition}

\begin{definition}{周期関数}
  次の式を満たす関数$f(x)$を、周期$\lambda$の周期関数という。
  \LARGE
  \begin{equation}
    f(x+\lambda) = f(x)
  \end{equation}
\end{definition}

\begin{definition}{波数}
  $2\pi$の長さに含まれる、波の最小パターンの数を波数という。
  \LARGE
  \begin{equation}
    k = \dfrac{2\pi}{\lambda}
  \end{equation}
\end{definition}

\subsection{時間的に捉える波}

波を時間軸から見たとき、波を構成する最小パターンは幅ではなく時間である。

その最小パターンを周期と呼ぶ。

周期は、波を時間軸から見たときの「波長」の言い換えともいえる。

\begin{definition}{周期}
  波が1回振動するのにかかる時間を周期と呼び、$T$で表す。
\end{definition}

\begin{definition}{周期関数}
  次の式を満たす関数$f(t)$を、周期$T$の周期関数という。
  \LARGE
  \begin{equation}
    f(t+T) = f(t)
  \end{equation}
\end{definition}

\begin{definition}{周波数(振動数)}
  \newline
  単位時間に含まれる、波の最小パターンの数を周波数という。
  \LARGE
  \begin{equation}
    \nu = \dfrac{1}{T}
  \end{equation}
  \normalsize
  これは、単位時間に何回振動するかを表すため、振動数とも呼ばれる。
\end{definition}

\section{角周波数と正弦波}

\begin{definition}{角周波数}
  動径が単位時間内に進む角を角周波数と呼び、$\omega$で表す。
\end{definition}

\begin{theorem}{任意の時間における動径}
  \newline
  時間が$t$だけ経過したときの動径$\theta$は、角周波数$\omega$を使って次のように表すことができる。
  \LARGE
  \begin{equation}
    \theta = \omega t
  \end{equation}
\end{theorem}

$\sin\theta$や$\cos\theta$は、$\theta=\omega t$の関係を用いると、動径$\theta$ではなく角周波数$\omega$の関数とみることができる。

\begin{definition}{正弦波}
  $\sin\omega t$や$\cos\omega t$を、角周波数$\omega$の正弦波と呼ぶ。
\end{definition}

\subsection{角周波数と振動数の関係}

円の1周は$2\pi$であり、単位時間あたりに進む円周は角周波数$\omega$である。

\footnotesize
(角周波数は「角」の大きさとして定義したが、弧度法のおかげで、「円周」の長さとしても捉えられる。)
\normalsize

ここで、単位時間あたりに進む円周$\omega$は、1周$2\pi$のうちのどれくらいだろうか?

その答えは、$\omega$を「1周あたりの量」$2\pi$で割ったものになる。

\begin{theorem}{角周波数と円周の関係}
  \newline
  角周波数$\omega$で動径が回転するとき、その動径は単位時間に
  \LARGE
  \begin{equation}
    \dfrac{\omega}{2\pi}
  \end{equation}
  \normalsize
  だけ円を回ることになる。
\end{theorem}

ここで、三角関数は円関数とも呼ばれるように、円の1周は三角関数の1振動に対応する。

振動を円周上の回転として表す三角関数のおかげで、「どれくらい回るか?」を「どれくらい振動するか?」とみることができる。

つまり、動径が単位時間に$\dfrac{\omega}{2\pi}$だけ回転するということは、単位時間に$\dfrac{\omega}{2\pi}$だけ振動するということだ。

\begin{theorem}{角周波数と振動数の関係}
  \newline
  角周波数を$\omega$とすると、振動数$\nu$は次のように表せる。
  \LARGE
  \begin{equation}
    \nu = \dfrac{\omega}{2\pi}
  \end{equation}
  \normalsize
\end{theorem}

\subsection{角周波数と周期の関係}

ここまでで、振動数$\nu$は2通りの表し方ができることがわかった。

\begin{itemize}
  \item $\nu = \dfrac{1}{T}$(周波数:単位時間に含まれる、最小波の時間幅)
  \item $\nu = \dfrac{\omega}{2\pi}$(振動数:単位時間に含まれる、振動の回数)
\end{itemize}

この2式を組み合わせて、次のような関係が得られる。

\begin{equation}
  \omega = 2\pi\nu = \dfrac{2\pi}{T}
\end{equation}

\begin{theorem}{角周波数と周期の関係}
  \newline
  角周波数を$\omega$、周期を$T$とすると、次のような関係が成り立つ。
  \LARGE
  \begin{equation}
    \omega = \dfrac{2\pi}{T}
  \end{equation}
\end{theorem}

\section{偶関数と奇関数}

$\sin$関数と$\cos$関数は、どちらも正弦波と呼ばれるが、その性質は異なる。

$\sin$は奇関数であり、$\cos$は偶関数である。

この違いが、後に議論するフーリエ級数展開においても重要な役割を果たす。

\subsection{偶関数と奇関数は異なる対称性を持つ}

\begin{definition}{偶関数}
  \newline
  グラフが$y$軸に対して対称な関数を偶関数と呼ぶ。\\
  偶関数は、任意の$x$に対して次の関係が成り立つ関数として定義される。
  \LARGE
  \begin{equation}
    f(-x) = f(x)
  \end{equation}
\end{definition}

\begin{definition}{奇関数}
  \newline
  グラフが原点に対して対称な関数を奇関数と呼ぶ。\\
  奇関数は、任意の$x$に対して次の関係が成り立つ関数として定義される。
  \LARGE
  \begin{equation}
    f(-x) = -f(x)
  \end{equation}
\end{definition}

\begin{center}
  \begin{tikzpicture}
    % 偶関数のグラフ(左側)
    \begin{scope}[xshift=-3cm]
      \clip (-2.5, -2.5) rectangle (2.5, 2.5);
      % 座標軸
      \draw[axis] (-2.5,0) -- (2.5,0) node[below] {$x$};
      \draw[axis] (0,-2) -- (0,2) node[above] {$y$};

      % 偶関数のプロット (例: y = x^2 - 1)
      \draw[thick, domain=-1.6:1.6, smooth, samples=100] plot (\x, {\x*\x - 1});

      % ラベル
      \node[below] at (0,-2) {偶関数: $f(-x) = f(x)$};
    \end{scope}

    % 奇関数のグラフ(右側)
    \begin{scope}[xshift=3cm]
      \clip (-2.5, -2.5) rectangle (2.5, 2.5);
      % 座標軸
      \draw[axis] (-2.5,0) -- (2.5,0) node[below] {$x$};
      \draw[axis] (0,-2) -- (0,2) node[above] {$y$};

      % 奇関数のプロット (例: y = x^3 - x)
      \draw[thick, domain=-1.5:1.5, smooth, samples=100] plot (\x, {\x*\x*\x - \x});

      % ラベル
      \node[below] at (0,-2) {奇関数: $f(-x) = -f(x)$};
    \end{scope}
  \end{tikzpicture}
\end{center}

1つの関数が、この両方の性質を持つことはない。

つまり、偶関数であり奇関数でもある関数は存在しない。

\subsection{積に関する性質}

\begin{theorem}{偶関数と奇関数の積}
  偶関数と奇関数の積は、奇関数となる。
\end{theorem}

\begin{proof}
  $f(x)$を奇関数、$g(x)$を偶関数とすると、
  \begin{equation}
    \begin{WithArrows}
      f(x)g(x)   & = -f(-x)g(-x) \Arrow{両辺$-1$倍して両辺入れ替え} \\
      f(-x)g(-x) & = -f(x)g(x)
    \end{WithArrows}
  \end{equation}
  となり、引数を$-1$倍すると符号が反転するため、$f(x)g(x)$は奇関数である。
\end{proof}

\begin{theorem}{奇関数どうしの積}
  奇関数と奇関数の積は、偶関数となる。
\end{theorem}

\begin{proof}
  $f(x), g(x)$を奇関数とすると、
  \begin{align}
    f(x)g(x) & = -f(-x)\cdot\{-g(-x)\} \\
             & = f(-x)g(-x)
  \end{align}
  となり、引数を$-1$倍しても符号がそのままなので、$f(x)g(x)$は偶関数である。
\end{proof}

\begin{theorem}{偶関数どうしの積}
  偶関数と偶関数の積は、偶関数となる。
\end{theorem}

\begin{proof}
  $f(x), g(x)$を偶関数とすると、
  \begin{equation}
    \begin{WithArrows}
      f(x)g(x)   & = f(-x)g(-x) \Arrow{両辺入れ替え} \\
      f(-x)g(-x) & = f(x)g(x)
    \end{WithArrows}
  \end{equation}
  となり、引数を$-1$倍しても符号がそのままなので、$f(x)g(x)$は偶関数である。
\end{proof}

\subsection{和に関する性質}

\begin{theorem}{奇関数どうしの和}
  奇関数と奇関数の和は、奇関数となる。
\end{theorem}

\begin{proof}
  $f(x), g(x)$を奇関数とすると、
  \begin{equation}
    \begin{WithArrows}
      f(x) + g(x) &= -f(-x)-g(-x) \\
      &= -\{f(-x)+g(-x)\} \Arrow{両辺$-1$倍して両辺入れ替え} \\
      f(-x) + g(x) &= -\{f(x)+g(x)\}
    \end{WithArrows}
  \end{equation}
  となり、引数を$-1$倍すると符号が反転するため、$f(x)+g(x)$は奇関数である。
\end{proof}

\begin{theorem}{偶関数どうしの和}
  偶関数と偶関数の和は、偶関数となる。
\end{theorem}

\begin{proof}
  $f(x), g(x)$を偶関数とすると、
  \begin{equation}
    f(x)+g(x) = f(-x)+g(-x)
  \end{equation}
  となり、引数を$-1$倍しても符号がそのままなので、$f(x)+g(x)$は偶関数である。
\end{proof}

\subsection{偶関数・奇関数の積分}

\begin{theorem}{偶関数の積分公式}
  \newline
  原点に関して対称な区間$-a \leq x \leq a$において、$f(x)$が偶関数なら、次の式が成り立つ。
  \LARGE
  \begin{equation}
    \int_{-a}^{a}f(x)dx = 2\int_{0}^{a}f(x)dx
  \end{equation}
\end{theorem}

\begin{center}
  \begin{tikzpicture}
    \def\xmin{-2.5};
    \def\xmax{2.5};
    \def\ymin{-1};
    \def\ymax{2};
    \def\a{1.25};
    \def\fn#1{#1*#1}

    % 座標軸
    \draw[axis] (\xmin,0) -- (\xmax,0) node[below] {$x$};
    \draw[axis] (0,\ymin) -- (0,\ymax) node[above] {$y$};

    % x軸上の点a,-a
    \node[below] at (\a,0) {$a$};
    \node[below] at (-\a,0) {$-a$};

    % 積分が面積となる領域
    \fill [pink!80, domain=-\a:\a, variable=\x] (-\a, 0) -- plot ({\x}, {\fn{\x}}) -- (\a, 0) -- cycle;

    % 原点
    \node[below left] at (0,0) {$O$};

    % 偶関数のグラフ
    \begin{scope}
      \clip (\xmin, \ymin) rectangle (\xmax, \ymax);
      % 偶関数のプロット
      \draw[thick, domain=\xmin:\xmax, smooth, samples=100] plot (\x, {\fn{\x}});
    \end{scope}
  \end{tikzpicture}
\end{center}

\begin{theorem}{奇関数の積分公式}
  \newline
  原点に関して対称な区間$-a \leq x \leq a$において、$f(x)$が奇関数なら、次の式が成り立つ。
  \LARGE
  \begin{equation}
    \int_{-a}^{a}f(x)dx = 0
  \end{equation}
\end{theorem}

\begin{center}
  \begin{tikzpicture}
    \def\xmin{-2.5};
    \def\xmax{2.5};
    \def\ymin{-2};
    \def\ymax{2};
    \def\a{1};
    \def\fn#1{#1*#1*#1 - #1};

    % 座標軸
    \draw[axis] (\xmin,0) -- (\xmax,0) node[below] {$x$};
    \draw[axis] (0,\ymin) -- (0,\ymax) node[above] {$y$};

    % 原点
    \node[below left] at (0,0) {$O$};

    % x軸上の点a,-a
    \node[right=0.5em, below] at (\a,0) {$a$};
    \node[left=1.25em, below] at (-\a,0) {$-a$};

    % 積分が正の面積となる領域
    \fill [pink!80, domain=-\a:0, variable=\x] (-\a, 0) -- plot ({\x}, {\fn{\x}}) -- (0, 0) -- cycle;
    % 積分が負の面積となる領域
    \fill [cyan!30, domain=0:\a, variable=\x] (0, 0) -- plot ({\x}, {\fn{\x}}) -- (\a, 0) -- cycle;

    % 奇関数のグラフ
    \begin{scope}
      \clip (\xmin, \ymin) rectangle (\xmax, \ymax);
      % 奇関数のプロット
      \draw[thick, domain=\xmin:\xmax, smooth, samples=100] plot (\x, {\fn{\x}});
    \end{scope}
  \end{tikzpicture}
\end{center}

\section{フーリエ級数}

\subsection{そもそも級数とは}

\begin{definition}{級数展開}
  \newline
  ある関数$f(x)$を、より基本的な関数系
  \Large
  \begin{equation}
    \{\varphi_0(x), \varphi_1(x), \varphi_2(x), \dots\}
  \end{equation}
  \normalsize
  を使って、次のような級数で表すことを級数展開という。
  \LARGE
  \begin{equation}
    f(x) = \sum_{n=0}^{\infty} c_n \varphi_n(x)
  \end{equation}
\end{definition}

級数展開は、近似や性質の分析に役立つ。

\subsubsection{代表的な級数展開:マクローリン展開}

$f(x)$が無限回微分可能なとき、$f(x)$は多項式関数$\{x^0, x^1, x^2, \dots\}$を使って級数展開できる。

\begin{equation}
  f(x) = \sum_{n=0}^{\infty} \dfrac{f^{(n)}(0)}{n!}x^n
\end{equation}

このような級数展開をマクローリン展開という。

\subsubsection{代表的な級数展開:フーリエ級数展開}

$f(x)$が特定の条件を満たすとき、$f(x)$は三角関数を使って級数展開できる。

このような級数展開をフーリエ級数展開といい、これからの議論の対象となる。

\subsection{有限区間で定義された関数のフーリエ級数展開}

\begin{theorem}{有限区間で定義された関数のフーリエ級数展開}
  \newline
  $-\dfrac{T}{2} \leq t \leq \dfrac{T}{2}$(区間幅$T$の有限区間)で定義された関数$f(t)$について、
  \Large
  \begin{equation}
    f(t) = \labelmath{a_0 + \sum_{n=1}^{\infty} \left\{ a_n\cos\left(\dfrac{2\pi nt}{T}\right) + b_n\sin\left(\dfrac{2\pi nt}{T}\right) \right\}}{\normalsize フーリエ級数展開}
  \end{equation}
  \normalsize
  が成り立つとしたら、フーリエ係数$a_0, a_n, b_n$は次のようになる。
  \Large
  \begin{align}
    a_0 & = \dfrac{1}{T} \int_{-\frac{T}{2}}^{\frac{T}{2}} f(t) dt                                     \\
    a_n & = \dfrac{2}{T} \int_{-\frac{T}{2}}^{\frac{T}{2}} f(t) \cos\left(\dfrac{2\pi nt}{T}\right) dt \\
    b_n & = \dfrac{2}{T} \int_{-\frac{T}{2}}^{\frac{T}{2}} f(t) \sin\left(\dfrac{2\pi nt}{T}\right) dt
  \end{align}
\end{theorem}

\subsection{フーリエ級数展開の周期関数への拡張}

元の関数$f(t)$には区間の制限を設けていたが、フーリエ級数を構成する三角関数は、無限区間で定義されている。

そして、三角関数は、区間幅$T$だけずらしても同じ値をとる、周期$T$の周期関数である。

つまり、特定の区間内の関数$f(t)$の形を、無限区間内で$T$ずつずらしていっても、それを表現するフーリエ級数の式は変わらない。

関数$f(t)$が、区間の制限をなくしても同じ形を繰り返すだけ(周期関数)であれば、先ほどのフーリエ級数展開がそのまま成り立つことになる。

\begin{theorem}{周期関数のフーリエ級数展開}
  \newline
  周期$T$の周期関数$f(t)$について、
  \Large
  \begin{equation}
    f(t) = \labelmath{a_0 + \sum_{n=1}^{\infty} \left\{ a_n\cos\left(\dfrac{2\pi nt}{T}\right) + b_n\sin\left(\dfrac{2\pi nt}{T}\right) \right\}}{\normalsize フーリエ級数展開}
  \end{equation}
  \normalsize
  が成り立つとしたら、フーリエ係数$a_0, a_n, b_n$は次のようになる。
  \Large
  \begin{align}
    a_0 & = \dfrac{1}{T} \int_{-\frac{T}{2}}^{\frac{T}{2}} f(t) dt                                     \\
    a_n & = \dfrac{2}{T} \int_{-\frac{T}{2}}^{\frac{T}{2}} f(t) \cos\left(\dfrac{2\pi nt}{T}\right) dt \\
    b_n & = \dfrac{2}{T} \int_{-\frac{T}{2}}^{\frac{T}{2}} f(t) \sin\left(\dfrac{2\pi nt}{T}\right) dt
  \end{align}
\end{theorem}

\subsection{不連続点におけるフーリエ級数の値}

次のような矩形波$f(t)$では、$t=\dfrac{T}{n}$が不連続な点となる。

\begin{equation}
  f(t) = \left\{
  \begin{array}{ll}
    0 & (-\pi \leq t < 0) \\
    1 & (0 \leq t < \pi)
  \end{array}
  \right.
\end{equation}

% SQUARE WAVE
\begin{center}
  \begin{tikzpicture}
    \def\xmin{-0.7*\T}   % min x axis
    \def\xmax{6.0}       % max x axis
    \def\ymin{-1.04}     % min y axis
    \def\ymax{1.3}       % max y axis
    \def\A{0.67*\ymax}   % amplitude
    \def\T{(0.35*\xmax)} % period
    \def\f#1{\A*4/pi/(#1)*sin(360/\T*#1*Mod(\t,\T))} %Mod(360*#1*\t/\T,360)
    \def\tick#1#2{\draw[thick] (#1) ++ (#2:0.12) --++ (#2-180:0.24)}

    % AXIS
    \draw[axis,thick] (0,\ymin) -- (0,\ymax) node[left] {$y$};
    \draw[axis,thick] ({\xmin},0) -- (\xmax,0) node[below,right] {$t$ [s]};

    % PLOT
    \begin{scope}
      \clip ({0.9*\xmin},-1.1*\A) rectangle (0.95*\xmax,1.1*\A);
      \foreach \i [evaluate={\x=\i*\T/2;}] in {-2,...,5}{
          \ifodd\i
            \draw[plotline,very thick,line cap=round] (\x,{-\A}) --++ ({\T/2},0);
            \draw[plotline,dashed,thin,line cap=round]
            ({\x+\T/2},{-\A}) --++ (0,2*\A);
          \else
            \draw[plotline,very thick,line cap=round] (\x,{\A}) --++ ({\T/2},0);
            \draw[plotline,dashed,thin,line cap=round]
            ({\x+\T/2},{\A}) --++ (0,-2*\A);
          \fi
        }
    \end{scope}

    % 周期を表すラベル
    \tick{{ -\T/2},0}{90} node[below,scale=0.8] {\contour{white}{$-\dfrac{T}{2}$}};
    \tick{{  \T  },0}{90} node[below,scale=0.8] {\contour{white}{$T$}};
    \tick{{  \T/2},0}{90} node[right,below,scale=0.8] {\contour{white}{$\dfrac{T}{2}$}};
    \tick{{3*\T/2},0}{90} node[right,below,scale=0.8] {\contour{white}{$3\dfrac{T}{2}$}};
    \tick{{2*\T  },0}{90} node[right,below,scale=0.8] {\contour{white}{$2T$}};
    \tick{{5*\T/2},0}{90} node[right,below,scale=0.8] {\contour{white}{$5\dfrac{T}{2}$}};

    \tick{0,{ \A}}{  0} node[left,scale=0.9] {$A$};
    \tick{0,{-\A}}{180} node[right,scale=0.9] {$-A$};

    % 原点
    \node[below left] at (0,0) {$O$};
  \end{tikzpicture}
\end{center}

この関数をフーリエ級数展開し、$k$項までの和を求めた結果が、$s_k$のような波形となる。

% SQUARE WAVE SYNTHESIS - time domain
\begin{center}
  \begin{tikzpicture}
    \def\xmin{-0.65*\T}  % max x axis
    \def\xmax{6.0}       % max x axis
    \def\ymin{-1.04}     % min y axis
    \def\ymax{1.3}       % max y axis
    \def\A{0.67*\ymax}   % amplitude
    \def\f#1{\A*4/pi/(#1)*sin(360/\T*#1*Mod(\t,\T))} %Mod(360*#1*\t/\T,360)
    \def\T{(0.465*\xmax)} % period
    \def\N{100}            % number of samples
    \def\tick#1#2{\draw[thick] (#1) ++ (#2:0.12) --++ (#2-180:0.24)}

    % 元の矩形波を点線で描画
    \begin{scope}
      \clip ({-0.54*\T},-1.1*\A) rectangle (0.97*\xmax,1.1*\A);
      \foreach \i [evaluate={\x=\i*\T/2;}] in {-2,...,4}{
          \ifodd\i
            \draw[blue!80!black!30,line cap=round] (\x,{-\A}) --++ ({\T/2},0);
            \draw[blue!80!black!30,dashed,thin,line cap=round]
            ({\x+\T/2},{-\A}) --++ (0,2*\A);
          \else
            \draw[blue!80!black!30,line cap=round] (\x,{\A}) --++ ({\T/2},0);
            \draw[blue!80!black!30,dashed,thin,line cap=round]
            ({\x+\T/2},{\A}) --++ (0,-2*\A);
          \fi
        }
    \end{scope}

    % AXIS
    \draw[axis,thick] (0,\ymin) -- (0,\ymax) node[left] {$y$};
    \draw[axis,thick] ({\xmin},0) -- (\xmax,0) node[below,right] {$t$ [s]};

    % PLOT
    \draw[plotline, thin,samples=\N,smooth,variable=\t,domain=-0.55*\T:0.94*\xmax] plot(\t,{\f{1}});
    \draw[plotline, thin,green,samples=3*\N,smooth,variable=\t,domain=-0.54*\T:0.94*\xmax] plot(\t,{\f{1}+\f{3}});
    \draw[plotline, thin,red,samples=5*\N,smooth,variable=\t,domain=-0.53*\T:0.94*\xmax] plot(\t,{\f{1}+\f{3}+\f{5}});
    \draw[plotline, thin,orange,line width=0.7,samples=7*\N,smooth,variable=\t,domain=-0.52*\T:0.94*\xmax] plot(\t,{\f{1}+\f{3}+\f{5}+\f{7}});
    \draw[plotline, thin,purple,samples=9*\N,smooth,variable=\t,domain=-0.52*\T:0.95*\xmax] plot(\t,{\f{1}+\f{3}+\f{5}+\f{7}+\f{9}});

    % NUMBERS
    \node[blue,  above,scale=0.9] at ({0.16*\T},1.20*\A) {$s_1$};
    \node[green, below,scale=0.9] at ({0.25*\T},0.88*\A) {$s_3$};
    \node[red,   above,scale=0.9] at ({0.41*\T},1.17*\A) {$s_5$};
    \node[orange,right,scale=0.9] at ({0.48*\T},0.50*\A) {$s_7$};

    % 周期を表すラベル
    \tick{{  \T},0}{90} node[below right,scale=0.8] {$T$};
    \tick{{2*\T},0}{90} node[below right,scale=0.8] {$2T$};

    % 周波数を表すラベル
    \tick{0,{ \A}}{  0} node[left,scale=0.9] {$A$};
    \tick{0,{-\A}}{180} node[right,scale=0.9] {$-A$};

    % 原点
    \node[below left] at (0,0) {$O$};
  \end{tikzpicture}
\end{center}

$k$が大きくなるほど、$s_k$は元の矩形波$f(t)$に近づいていることがわかる。

ここで、元の関数の不連続点である$t=\dfrac{T}{n}$において、$s_k$は不連続点を通過している。

例えば、$t=0$において、$t=0$より左側では$-A$に近い値、右側では$A$に近い値をとる。

\begin{itemize}
  \item $t=0$に右から近づいていくと、$s_k$は$A$に近づいていく(右極限は$A$)
  \item $t=0$に左から近づいていくと、$s_k$は$-A$に近づいていく(左極限は$-A$)
\end{itemize}

そして、$t=0$において、$s_k$は$A$と$-A$の間の値(原点)を通過している。

一般に、不連続となる$t$において、フーリエ級数展開の値は、その点での左右の極限値の平均値となる。

\begin{theorem}{不連続点におけるフーリエ級数の収束}
  \newline
  $f(t)$が$t=a$で不連続のとき、フーリエ級数の値は左極限$f(a-0)$と右極限$f(a+0)$の平均値に収束する。
  \LARGE
  \begin{equation}
    \lim_{t \to a} s_k(t) = \dfrac{f(a-0) + f(a+0)}{2}
  \end{equation}
\end{theorem}

\subsection{フーリエ級数展開の意味}

フーリエ級数展開の式は、

\begin{itemize}
  \item $1$の係数が$a_0$
  \item $\cos\left(\dfrac{2\pi nt}{T}\right)$の係数が$a_n$
  \item $\sin\left(\dfrac{2\pi nt}{T}\right)$の係数が$b_n$
\end{itemize}

となっていた。

フーリエ級数展開は、次の基本関数系を使った級数展開といえる。

\begin{equation}
  \left\{1, \cos\left(\dfrac{2\pi nt}{T}\right), \sin\left(\dfrac{2\pi nt}{T}\right)\right\}
\end{equation}

ここで、

\begin{review}
  $\sin\omega t$や$\cos\omega t$は、角周波数$\omega$の正弦波と呼ばれる
\end{review}

ことを思い出すと、フーリエ級数展開を構成する基本関数系は、角周波数$\omega_n = \dfrac{2\pi n}{T}$の正弦波である。

(1は$\cos\dfrac{2\pi nt}{T}$における、$n=0$の場合だと考えることができる。)

つまり、フーリエ級数展開は、関数$f(t)$を角周波数$\omega_n$の正弦波に分解することである。

\begin{center}
  % SYNTHESIS 3D
  \begin{tikzpicture}[x=(-20:0.9), y=(90:0.9), z=(42:1.1)]
    \def\xmax{6.5}        % max x axis
    \def\ymin{-1.2}       % min y axis
    \def\ymax{1.6}        % max y axis
    \def\zmax{5.8}        % max z axis
    \def\xf{1.17*\xmax}   % x position frequency axis
    \def\A{(0.60*\ymax)}  % amplitude
    \def\T{(0.335*\xmax)} % period
    \def\w{\zmax/11.2}    % spacing components
    \def\N{100}           % number of samples
    \def\f#1{\A*4/pi/(#1)*sin(360/\T*#1*Mod(\t,\T))} %Mod(360*#1*\t/\T,360)
    \def\tick#1#2{\draw[thick] (#1) ++ (#2:0.12) --++ (#2-180:0.24)}

    % COMPONENTS
    \foreach \i/\col [evaluate={\z=\w*\i;}] in {11/cyan,9/purple,7/orange,5/red,3/green,1/blue}{
        %\draw[black!30] ({\T},0.1,\z) --++ (0,-0.2,0);
        %\draw[black!30] ({2*\T},0.1,\z) --++ (0,-0.2,0);
        % 分解された各波の座標軸
        \draw[axis,black!30] (0,0,\z) --++ (0.93*\xmax,0,0);
        % 分解された各波
        \draw[plotline,\col,opacity=0.8,thick,samples=\i*\N,smooth,variable=\t,domain=-0.05*\T:0.87*\xmax] plot(\t,{\f{\i}},\z);
      }

    % TIME DOMAIN
    \begin{scope}[shift={(0,0,-0.17*\zmax)}]
      % 時間領域を表す平面
      \draw[black,fill=white,opacity=0.3,xy-plane] (-0.1*\xmax,-1.25*\ymax) rectangle (1.13*\xmax,1.25*\ymax);
      % 横軸
      \draw[axis,thick] (-0.05*\xmax,0,0) -- (\xmax,0,0) node[below right,xy-plane] {$t$ [s]};
      % 縦軸
      \draw[axis,thick] (0,\ymin,0) -- (0,\ymax,0) node[left,xy-plane] {$y$};
      % 時間領域での関数のグラフ
      \draw[plotline,blue!90!black,very thick,samples=9*\N,smooth,variable=\t,domain=-0.05*\T:0.9*\xmax] plot(\t,{\f{1}+\f{3}+\f{5}+\f{7}+\f{9}+\f{11}},0);
      % 周期を示すラベル
      \tick{{\T},0,0}{90} node[below,scale=0.9,xy-plane] {\contour{white}{$T$}};
      \tick{{2*\T},0,0}{90} node[below,scale=0.9,xy-plane] {\contour{white}{$2T$}};
      % 平面を説明するラベル
      \node[scale=0.8,xy-plane] at (0.4*\xmax,-\ymax,0) {時間の世界(連続的)};
    \end{scope}

    % FREQUENCY DOMAIN
    \begin{scope}[shift={(\xf,0,0)}]
      % 周波数領域を表す平面
      \draw[black,fill=white,opacity=0.3,zy-plane] (-0.13*\zmax,-1.25*\ymax) rectangle (1.26*\zmax,1.25*\ymax);
      % 縦軸
      \draw[axis,thick] (0,0.8*\ymin,0) -- (0,\ymax,0) node[above=2,left=0,zy-plane] {$b_n$};
      % 横軸
      \draw[axis,thick] (0,0,-0.05*\zmax) --++ (0,0,1.13*\zmax) node[below right=-1,zy-plane] {$f_n$ $\left[\frac{1}{\mathrm{s}}\right]$};
      % 平面を説明するラベル
      \node[scale=0.8,zy-plane] at (0,-\ymax,0.65*\zmax) {周波数の世界(離散的)};
      % 周波数領域での関数のグラフ
      \draw[blue!30,dashed,samples=3*\N,smooth,variable=\t,domain=0.074*\zmax:1.02*\zmax] plot(0,{\A*4/pi/\t*\w},\t);
      % 各周波数成分
      \foreach \i/\col [evaluate={\z=\w*\i;}] in {11/cyan,9/purple,7/orange,5/red,3/green,1/blue}{
          % 周波数成分の高さを示す補助線
          \draw[\col,dash pattern=on 2 off 2] (0,0,\z) --++ (0,{\A*4/pi/\i},0);
          % 周波数成分の値を示す点
          \fill[\col,canvas is zy plane at x=0] (\z,{\A*4/pi/\i}) circle(0.07);
          % 横軸上の目盛りとそのラベル
          \tick{0,0,\z}{90} node[below,scale=0.85,zy-plane]{$\dfrac{\i}{T}$};
          % 横軸上の各目盛りの中点
          \foreach \i [evaluate={\z=\w*\i;}] in {2,4,...,10}{
              \fill[blue!60!black,zy-plane] (\z,0) circle(0.07);
            }
        }
    \end{scope}
  \end{tikzpicture}
\end{center}

関数$f(t)$がどのような周波数成分で構成されているか?を解き明かすのがフーリエ級数展開で、フーリエ係数は時間領域から周波数領域へのマッピングの役割を果たしている。

\subsection{フーリエ級数展開のさまざまな表現式}

フーリエ級数展開の式は、文献によって異なるいくつかの形で表現される。

\subsubsection{定数項をまとめた表現}

定数項$a_0$を、$a_n$の$n=0$の場合として考えることができる。

その場合、フーリエ級数展開は次のように表される。

\begin{theorem}{フーリエ級数展開(フーリエ係数を整理した表現)}
  \newline
  周期$T$の周期関数$f(t)$について、
  \Large
  \begin{equation}
    f(t) = \labelmath{\dfrac{a_0}{2} + \sum_{n=1}^{\infty} \left\{ a_n\cos\left(\dfrac{2\pi nt}{T}\right) + b_n\sin\left(\dfrac{2\pi nt}{T}\right) \right\}}{\normalsize フーリエ級数展開}
  \end{equation}
  \normalsize
  が成り立つとしたら、フーリエ係数$a_n, b_n$は次のようになる。
  \Large
  \begin{align}
    a_n & = \dfrac{2}{T} \int_{-\frac{T}{2}}^{\frac{T}{2}} f(t) \cos\left(\dfrac{2\pi nt}{T}\right) dt \\
    b_n & = \dfrac{2}{T} \int_{-\frac{T}{2}}^{\frac{T}{2}} f(t) \sin\left(\dfrac{2\pi nt}{T}\right) dt
  \end{align}
\end{theorem}

\subsubsection{角周波数を使った表現}

角周波数$\omega_0=\dfrac{2\pi}{T}$を使って、フーリエ級数展開の式を書き換えることもできる。

\begin{theorem}{フーリエ級数展開(角周波数を使った表現)}
  \newline
  周期$T$の周期関数$f(t)$について、角周波数$\omega_0$を用いて、
  \Large
  \begin{equation}
    f(t) = \labelmath{a_0 + \sum_{n=1}^{\infty} \left( a_n\cos \omega_0 nt + b_n\sin \omega_0 nt \right)}{\normalsize フーリエ級数展開}
  \end{equation}
  \normalsize
  が成り立つとしたら、フーリエ係数$a_0, a_n, b_n$は次のようになる。
  \Large
  \begin{align}
    a_0 & = \dfrac{1}{T} \int_{-\frac{T}{2}}^{\frac{T}{2}} f(t) dt                 \\
    a_n & = \dfrac{2}{T} \int_{-\frac{T}{2}}^{\frac{T}{2}} f(t) \cos\omega_0 nt dt \\
    b_n & = \dfrac{2}{T} \int_{-\frac{T}{2}}^{\frac{T}{2}} f(t) \sin\omega_0 nt dt
  \end{align}
\end{theorem}

\subsubsection{区間を0始まりにずらした表現}

有限区間$-\dfrac{T}{2} \leq t \leq \dfrac{T}{2}$で定義された関数のフーリエ級数展開を考えてきたが、その有限区間は区間幅が$T$であればなんでもよい。

特に、$0 \leq t \leq T$で定義された関数のフーリエ級数展開を考えることも多い。

区間を変えても、周期関数への拡張は同様の議論により成り立ち、次のことがいえる。

\begin{theorem}{フーリエ級数展開(積分区間を0始まりにした表現)}
  \newline
  周期$T$の周期関数$f(t)$について、
  \Large
  \begin{equation}
    f(t) = \labelmath{a_0 + \sum_{n=1}^{\infty} \left\{ a_n\cos\left(\dfrac{2\pi nt}{T}\right) + b_n\sin\left(\dfrac{2\pi nt}{T}\right) \right\}}{\normalsize フーリエ級数展開}
  \end{equation}
  \normalsize
  が成り立つとしたら、フーリエ係数$a_0, a_n, b_n$は次のようになる。
  \Large
  \begin{align}
    a_0 & = \dfrac{1}{T} \int_{0}^{T} f(t) dt                                     \\
    a_n & = \dfrac{2}{T} \int_{0}^{T} f(t) \cos\left(\dfrac{2\pi nt}{T}\right) dt \\
    b_n & = \dfrac{2}{T} \int_{0}^{T} f(t) \sin\left(\dfrac{2\pi nt}{T}\right) dt
  \end{align}
\end{theorem}

このフーリエ係数の式は、区間$-\dfrac{T}{2} \leq t \leq \dfrac{T}{2}$の場合の式を平行移動+置換積分することで示される。

\subsection{奇関数のフーリエ級数(フーリエ正弦級数)}

$f(t)$が奇関数の場合、それを表現するフーリエ級数には、奇関数しか入らない。

奇関数と奇関数の和が奇関数になることから、そう予想できる。

偶関数$\cos$の項が消え、奇関数$\sin$の項だけが残ることを確かめるため、各フーリエ係数を計算してみよう。

\subsubsection{定数項$a_0$}

原点に対して対称な範囲での奇関数の積分は$0$になるから、

\begin{align}
  a_0 & = \dfrac{1}{T} \int_{-\frac{T}{2}}^{\frac{T}{2}} \oddFn{f(t)} dt \\
      & = 0
\end{align}

\subsubsection{$\cos$の項の係数$a_n$}

$\int$の中身を見ると、奇関数と偶関数の積は奇関数になるので、積分結果は$0$になる。

\begin{align}
  a_n & = \dfrac{2}{T} \int_{-\frac{T}{2}}^{\frac{T}{2}} \oddFn[0.4]{\oddFn{f(t)} \evenFn{\cos\left(\dfrac{2\pi nt}{T}\right)}} dt \\
      & = 0
\end{align}

\subsubsection{$\sin$の項の係数$b_n$}

$\int$の中身を見ると、奇関数と奇関数の積は偶関数になるので、

\begin{review}
  偶関数の積分公式
  \begin{equation}
    \int_{-a}^{a}f(x)dx = 2\int_{0}^{a}f(x)dx
  \end{equation}
\end{review}

を使って計算する。

\begin{align}
  b_n & = \dfrac{2}{T} \int_{-\frac{T}{2}}^{\frac{T}{2}} \evenFn[0.4]{\oddFn{f(t)} \oddFn{\sin\left(\dfrac{2\pi nt}{T}\right)}} dt \\
      & = \dfrac{2}{T} \cdot 2 \int_{0}^{\frac{T}{2}} f(t) \sin\left(\dfrac{2\pi nt}{T}\right) dt                                  \\
      & = \dfrac{4}{T} \int_{0}^{\frac{T}{2}} f(t) \sin\left(\dfrac{2\pi nt}{T}\right) dt
\end{align}

\subsubsection{まとめ:フーリエ正弦級数}

以上より、$a_0$、$a_n$は$0$になるため、奇関数のフーリエ級数は、$\sin$の項だけで表現される。

奇関数のフーリエ級数は、フーリエ正弦級数と呼ばれる。
\begin{theorem}{フーリエ正弦級数}
  \newline
  周期$T$の周期関数$f(t)$が奇関数であり、
  \Large
  \begin{equation}
    f(t) = \labelmath{\sum_{n=1}^{\infty} b_n\sin\left(\dfrac{2\pi nt}{T}\right)}{\normalsize フーリエ正弦級数}
  \end{equation}
  \normalsize
  が成り立つとしたら、フーリエ係数$b_n$は次のようになる。
  \Large
  \begin{align}
    b_n & = \dfrac{4}{T} \int_{0}^{\frac{T}{2}} f(t) \sin\left(\dfrac{2\pi nt}{T}\right) dt
  \end{align}
\end{theorem}

\subsection{偶関数のフーリエ級数(フーリエ余弦級数)}

$f(t)$が偶関数の場合、それを表現するフーリエ級数には、偶関数しか入らない。

偶関数と偶関数の和が偶関数になることから、そう予想できる。

奇関数$\sin$の項が消え、偶関数$\cos$の項だけが残ることを確かめるため、各フーリエ係数を計算してみよう。

\subsubsection{定数項$a_0$}

偶関数の積分公式を使って計算する。

\begin{align}
  a_0 & = \dfrac{1}{T} \int_{-\frac{T}{2}}^{\frac{T}{2}} \evenFn{f(t)} dt \\
      & = \dfrac{1}{T} \cdot 2 \int_{0}^{\frac{T}{2}} f(t) dt             \\
      & = \dfrac{2}{T} \int_{0}^{\frac{T}{2}} f(t) dt
\end{align}

\subsubsection{$\cos$の項の係数$a_n$}

$\int$の中身を見ると、偶関数と偶関数の積は偶関数になるので、偶関数の積分公式を使って計算する。

\begin{align}
  b_n & = \dfrac{2}{T} \int_{-\frac{T}{2}}^{\frac{T}{2}} \evenFn[0.3]{\evenFn{f(t)} \evenFn{\cos\left(\dfrac{2\pi nt}{T}\right)}} dt \\
      & = \dfrac{2}{T} \cdot 2 \int_{0}^{\frac{T}{2}} f(t) \cos\left(\dfrac{2\pi nt}{T}\right) dt                                    \\
      & = \dfrac{4}{T} \int_{0}^{\frac{T}{2}} f(t) \cos\left(\dfrac{2\pi nt}{T}\right) dt
\end{align}

\subsubsection{$\sin$の項の係数$b_n$}

$\int$の中身を見ると、偶関数と奇関数の積は奇関数になるので、積分結果は$0$になる。

\begin{align}
  b_n & = \dfrac{2}{T} \int_{-\frac{T}{2}}^{\frac{T}{2}} \oddFn[0.4]{\evenFn{f(t)} \oddFn{\sin\left(\dfrac{2\pi nt}{T}\right)}} dt \\
      & = 0
\end{align}

\subsubsection{まとめ:フーリエ余弦級数}

以上より、$b_n$は$0$になるため、偶関数のフーリエ級数は、$\cos$の項だけで表現される。

偶関数のフーリエ級数は、フーリエ余弦級数と呼ばれる。

\begin{theorem}{フーリエ余弦級数}
  \newline
  周期$T$の周期関数$f(t)$が偶関数であり、
  \Large
  \begin{equation}
    f(t) = a_0 + \labelmath{\sum_{n=1}^{\infty} a_n\cos\left(\dfrac{2\pi nt}{T}\right)}{\normalsize フーリエ余弦級数}
  \end{equation}
  \normalsize
  が成り立つとしたら、フーリエ係数$a_0, a_n$は次のようになる。
  \Large
  \begin{align}
    a_0 & = \dfrac{2}{T} \int_{0}^{\frac{T}{2}} f(t) dt                                     \\
    a_n & = \dfrac{4}{T} \int_{0}^{\frac{T}{2}} f(t) \cos\left(\dfrac{2\pi nt}{T}\right) dt
  \end{align}
\end{theorem}

\chapter{線形システム}

\section{線形性}

\begin{definition}{線形性の1つの解釈}
  システムにおいて、次のような性質を線形性という。
  \begin{enumerate}
    \item 倍の刺激があれば、倍の反応が生まれる
    \item 2つの刺激があれば、それぞれが独立して反応する
  \end{enumerate}
\end{definition}

\end{document}
