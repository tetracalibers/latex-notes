% === hyperlink ===

% 「%」は以降の内容を「改行コードも含めて」無視するコマンド
\usepackage[%
  dvipdfmx,% 欧文ではコメントアウトする
  pdfencoding=auto, psdextra,% 数学記号を含める
  setpagesize=false,%
  bookmarks=true,%
  bookmarksdepth=tocdepth,%
  bookmarksnumbered=true,%
  colorlinks=true,%
  allcolors=MidnightBlue,%
  linkcolor=MidnightBlue,%
  pdftitle={},%
  pdfsubject={},%
  pdfauthor={},%
  pdfkeywords={}%
]{hyperref}
% PDFのしおり機能の日本語文字化けを防ぐ((u)pLaTeXのときのみかく)
\usepackage{pxjahyper}
% ref: https://tex.stackexchange.com/questions/251491/math-symbol-in-section-heading
\pdfstringdefDisableCommands{\def\varepsilon{\textepsilon}}

% 「定理 1(タイトル)」の形で、全部をひとつのリンクにする
\newcommand{\thmref}[1]{%
  \hyperref[#1]{\texttt{theorem \autoref*{#1}「\nameref*{#1}」}}%
}
\newcommand{\secref}[1]{%
  \hyperref[#1]{\texttt{\nameref*{#1}{\small[第\ref*{#1}章]}}}%
}

\NewDocumentCommand\defref{o m}{
  \IfNoValueTF{#1}
    {\hyperref[#2]{\texttt{def \autoref*{#2}「\nameref*{#2}」}}}
    {\hyperref[#2]{\texttt{#1{\footnotesize(def \autoref*{#2})}}}}
}

\usepackage{xr-hyper}
