\documentclass[b5paper,12pt]{jsarticle}

\title{Topic Note: 述語論理}
\author{tomixy}

% === color ===

% ref: https://latexcolor.com/
\definecolor{hotpink}{rgb}{1.0, 0.41, 0.71}
\definecolor{carnationpink}{rgb}{1.0, 0.65, 0.79}
\definecolor{deeppink}{rgb}{1.0, 0.08, 0.58}
\definecolor{capri}{rgb}{0.0, 0.75, 1.0}
\definecolor{rosepink}{rgb}{1.0, 0.4, 0.8}
\definecolor{princetonorange}{rgb}{1.0, 0.56, 0.0}
\definecolor{lavendermagenta}{rgb}{0.93, 0.51, 0.93}
\definecolor{malachite}{rgb}{0.04, 0.85, 0.32}
\definecolor{lawngreen}{rgb}{0.49, 0.99, 0.0}
\definecolor{periwinkle}{rgb}{0.8, 0.8, 1.0}
\definecolor{lightslategray}{rgb}{0.47, 0.53, 0.6}
\definecolor{robineggblue}{rgb}{0.0, 0.8, 0.8}
\definecolor{rosebonbon}{rgb}{0.98, 0.26, 0.62}
\definecolor{airforceblue}{rgb}{0.36, 0.54, 0.66}
\definecolor{columbiablue}{rgb}{0.61, 0.87, 1.0}
\definecolor{magnolia}{rgb}{0.97, 0.96, 1.0}
\definecolor{coolgrey}{rgb}{0.55, 0.57, 0.67}

% === box ===

\usepackage{awesomebox}

% === math ===

\usepackage{physics}
\usepackage{braket}

\usepackage{amssymb} % use \blacksquare

\usepackage{amsthm} % 定理環境とQEDコマンド
\renewcommand{\qedsymbol}{\textcolor{coolgrey}{$\blacksquare$}}

\usepackage{mathtools}
% 別の場所で参照する数式以外は番号が付かないように
\mathtoolsset{showonlyrefs=true}

\usepackage{systeme} % 連立方程式を簡単に書く
\usepackage{empheq}

\newcommand{\id}{\operatorname{id}}
\newcommand{\Id}{\operatorname{Id}}
\newcommand{\diag}{\operatorname{diag}}
\newcommand{\Ker}{\operatorname{Ker}}
\newcommand{\sgn}{\operatorname{sgn}}

\newcommand{\suchthat}{\,\, s.t. \,\,}
\newcommand{\transpose}[1]{{}^t\! #1}

% === font ===

\usepackage{amsfonts} % use \mathbb

\usepackage[T1]{fontenc}
\usepackage{lxfonts}

% monospace font
\renewcommand*\ttdefault{cmvtt}

% === layout ===

\usepackage[top=20truemm,bottom=20truemm,left=20truemm,right=60truemm,marginparwidth=40truemm,marginparsep=10truemm]{geometry} % 余白
\renewcommand{\baselinestretch}{1.25} % 行間

\usepackage{leading}

\setlength{\parindent}{0pt} % 段落始めでの字下げをしない

\usepackage{enumitem}
\newcommand{\romanlabel}{\textsf{\roman*.}}
\newcommand{\romannum}[1]{\textsf{#1}}

\usepackage[noparboxrestore]{marginnote}

\usepackage{tocloft}
% chapterのnumwidthを広くする
\setlength{\cftchapnumwidth}{5em}

\usepackage{titling}
\renewcommand{\maketitlehooka}{\textsf}

% === tikz ===

\usepackage[dvipdfmx]{graphicx}

\usepackage{tikz}
\usetikzlibrary{
  fit,
  patterns,
  decorations.pathreplacing,
  cd,
  petri,
  positioning
}

\usepackage{ifthen}
\usepackage{listofitems} % for \readlist to create arrays

\usepackage{witharrows}
\usepackage{nicematrix}

% === tcolorbox ===

\usepackage{tcolorbox}
\tcbuselibrary{listings,breakable,xparse,skins,hooks,theorems}

\newcommand{\titlegap}{\quad\\[0.1cm]}

\DeclareTColorBox{definition}{m O{} }%
{
  enhanced,
  colframe=magnolia,
  colback=magnolia!20!white,
  coltitle=black,
  fonttitle=\bfseries,
  breakable,
  sharp corners,
  title={\textcolor{Cerulean!60!black}{\faGraduationCap}\hspace{0.1em} #1},
  detach title,
  before upper={\tcbtitle\quad},
  bottom=0.5cm,
  top=0.5cm,
  right=0.5cm,
  left=0.5cm,
  #2
}

\DeclareTColorBox{theorem}{m O{} }%
{
  enhanced,
  colframe=magnolia,
  colback=magnolia!20!white,
  coltitle=black,
  fonttitle=\bfseries,
  breakable,
  sharp corners,
  title={\textcolor{magenta!70!black}{\faAnchor}\hspace{0.1em} #1},
  detach title,
  before upper={\tcbtitle\quad},
  bottom=0.5cm,
  top=0.5cm,
  right=0.5cm,
  left=0.5cm,
  #2
}

% 背景がグレー
\DeclareTColorBox{shaded}{O{} }%
{
  enhanced,
  colframe=white,
  colback=gray!10,
  breakable=true,
  sharp corners,
  detach title,
  bottom=0.25cm,
  top=0.25cm,
  right=0.25cm,
  left=0.25cm,
  #1
}

\newcommand{\ProofColor}{coolgrey}
\DeclareTColorBox{proof}{O{証明}}{%
  empty,
  title={\faBroom #1},
  attach boxed title to top left,
  sharp corners,
  boxed title style={
      empty,
      size=minimal,
      toprule=2pt,
      top=4pt,
      left=1em,
      right=1em,
      top=0.25cm,
      overlay={
          \draw[\ProofColor, double,line width=1pt] ([yshift=-1pt]frame.north west)--([yshift=-1pt]frame.north east);
        }
    },
  coltitle=\ProofColor,
  fonttitle=\bfseries,
  before=\par\medskip\noindent,
  parbox=false,
  boxsep=0pt,
  left=1em,
  right=1em,
  top=0.5cm,
  bottom=0.5cm,
  breakable,
  pad at break*=0mm,
  vfill before first,
  overlay unbroken={
      \draw[\ProofColor,line width=0.5pt]
      ([yshift=-1pt]title.north east)
      --([xshift=-0.5pt,yshift=-1pt]title.north-|frame.east)
      --([xshift=-0.5pt]frame.south east)
      --(frame.south west);
    },
  overlay first={
      \draw[\ProofColor,line width=1pt]([yshift=-1pt]title.north east)--([xshift=-0.5pt,yshift=-1pt]title.north-|frame.east)--([xshift=-0.5pt]frame.south east);
    },
  overlay middle={
      \draw[\ProofColor,line width=1pt] ([xshift=-0.5pt]frame.north east)--([xshift=-0.5pt]frame.south east);
    },
  overlay last={
      \draw[\ProofColor,line width=1pt] ([xshift=-0.5pt]frame.north east)--([xshift=-0.5pt]frame.south east)--(frame.south west);
    },%
}
\NewDocumentCommand{\patterntitle}{m}{
  \tcbox[
    enhanced,
    empty,
    boxsep=0pt,
    left=0pt,right=0pt,
    bottom=2pt,
    fonttitle=\bfseries,
    borderline south={0.5pt}{0pt}{\ProofColor},
  ]{\textcolor{\ProofColor}{#1}}
}
\renewenvironment{quote}{%
  \list{}{%
    \leftmargin0.5cm   % this is the adjusting screw
    \rightmargin\leftmargin
  }
  \item\relax
}{\endlist}
\newenvironment{subpattern}[1]{
  \patterntitle{#1}
  \begin{quote}
    }{
  \end{quote}
}

% === memo ===

\usepackage{zebra-goodies} % TODOなどの注釈

% === original ===

\newcommand{\keyword}[1]{\textcolor{RubineRed}{\textbf{#1}}}
\newcommand{\en}[1]{\textcolor{RubineRed}{\small\texttt{#1}}}
\newcommand{\keywordJE}[2]{\keyword{#1}(\en{\textcolor{RubineRed!60}{#2}})}

\newcommand{\br}{\vskip0.5\baselineskip}

\usepackage[object=vectorian]{pgfornament}
\newcommand{\sectionline}{%
  \noindent
  \begin{center}
    {\color{lightgray}
      \resizebox{0.5\linewidth}{1ex}
      {{%
            {\begin{tikzpicture}
                  \node  (C) at (0,0) {};
                  \node (D) at (9,0) {};
                  \path (C) to [ornament=85] (D);
                \end{tikzpicture}}}}}%
  \end{center}%
}

\renewcommand{\labelitemii}{$\circ$}

\newcommand{\refbook}[1]{\small ref: #1}

% === toc ===

\usepackage{tocloft}
\renewcommand{\cftsecfont}{\rmfamily}
\renewcommand{\cftsecpagefont}{\rmfamily}
\setcounter{secnumdepth}{0}

\addtocontents{toc}{\protect\thispagestyle{empty}}
\pagestyle{empty}

% === hyperlink ===

\definecolor{oxfordblue}{rgb}{0.0, 0.13, 0.28}

% 「%」は以降の内容を「改行コードも含めて」無視するコマンド
\usepackage[%
  dvipdfmx,% 欧文ではコメントアウトする
  pdfencoding=auto, psdextra,% 数学記号を含める
  setpagesize=false,%
  bookmarks=true,%
  bookmarksdepth=tocdepth,%
  bookmarksnumbered=true,%
  colorlinks=true,%
  allcolors=oxfordblue,%
  linkcolor=MidnightBlue,%
  pdftitle={},%
  pdfsubject={},%
  pdfauthor={},%
  pdfkeywords={}%
]{hyperref}
% PDFのしおり機能の日本語文字化けを防ぐ((u)pLaTeXのときのみかく)
\usepackage{pxjahyper}
% ref: https://tex.stackexchange.com/questions/251491/math-symbol-in-section-heading
\pdfstringdefDisableCommands{\def\varepsilon{\textepsilon}}


% === 参考文献 ===

\newcommand{\refbookA}{\refbook{ろんりと集合}}
\newcommand{\refbookB}{\refbook{大学数学 ほんとうに必要なのは「集合」}}

% ---

\begin{document}

\maketitle
\tableofcontents

\sectionline
\section{命題関数}

\marginnote{\refbookA}

これまで、たとえば「1234567891は素数である」というような\keyword{命題}を扱ってきた

ここで、たとえば$x$が自然数全体を動くとき、「$x$は素数である」という形の主張を\keyword{命題関数}と呼ぶ

\sectionline

\marginnote{\refbookB}

文字が一つ確定すると命題が一つ定まるというシステムを関数と捉えて\keyword{命題関数}という

\begin{definition}{命題関数}
  文章中に変数を含み、その変数を定めるごとに命題になる文章を\keyword{命題関数}という
\end{definition}

\sectionline

\marginnote{\refbookA}

命題は記号$p$で表されたのに対し、命題関数は$p(x)$と書く

命題関数$p(x)$は、$x$の値に応じて主張が変わり、真理値が変化していく

\br

命題関数$p(x)$の$x$は、\keyword{変数}と呼ばれる

命題関数$p(x)$の変数は、実数や自然数のような数以外に、直線とか地図のような数学的対象や一般的概念をとる

\sectionline
\marginnote{\refbookB}

命題関数が正しいかどうかは、あらゆる変数をすべて入れていき、その都度作られる命題がすべて正しい場合と定義される

\begin{definition}{命題関数が正しいとは}
  命題関数が正しいとは、含まれる変数を定めるごとに決まる命題が全て正しいということ
\end{definition}

\sectionline
\section{すべての〜}
\marginnote{\refbookA}

命題関数$p(x)$に対して、「すべての$x$について$p(x)$である」という命題を
\begin{equation*}
  \forall x p(x)
\end{equation*}
と表す

\br

「すべての〜について〇〇である」は、
\begin{itemize}
  \item 「すべての〜は〇〇である」
  \item 「任意の〜について〇〇である」
  \item 「任意の〜は〇〇である」
\end{itemize}
と表すこともある

\sectionline
\marginnote{\refbookB}

$\forall$と自由変数が命題関数にくっついた形は\keyword{全称命題}と呼ばれる

\begin{definition}{全称命題}
  $P(x)$を命題関数とする。このとき、
  \begin{equation*}
    ^{\forall}x, \, P(x)
  \end{equation*}
  という形の命題を\keyword{全称命題}と呼び、「すべての$x$に対して$P(x)$が成り立つ」という命題を表す
\end{definition}

\sectionline
\marginnote{\refbookB}

「すべての$x$に対してhogehogeである」という文章は、次のように記号化される
\begin{equation*}
  ^{\forall}x, \, \text{hogehoge}
\end{equation*}

$,$が「対して」を表すと考えればよい

\br

「Aに対してB」という言い方をした場合、Bは「Aによって変わりうるもの」というニュアンスがある

BはAをとった後にとるもの、この前後関係を示す言葉が「対して」という言葉

\sectionline
\marginnote{\refbookA}

$\forall$という記号は、「all(すべての〜)」や「any(任意の〜)」の頭文字のAを逆さにしたものに由来する

\br
\marginnote{\refbookB}

この記号は\keyword{全称記号}と呼ばれる

記号列として捉えて$\forall x$と書く流儀と、添字として捉えて$^{\forall}x$と書く流儀がある

\sectionline

\marginnote{\refbookA}

変数$x$が$x= a_1, a_2, \cdots , a_n$という有限個の値をとるとき、「すべての$x$について$p(x)$である」というのは、
\begin{shaded}
  $p(a_1)$であり、かつ、$p(a_2)$であり、かつ、$\cdots$、かつ、$p(a_n)$である
\end{shaded}
ということに他ならない

言い換えると、
\begin{equation*}
  \forall x p(x) = p(a_1) \land p(a_2) \land \cdots \land p(a_n)
\end{equation*}
ということになる

\sectionline
\section{ある〜}

命題関数$p(x)$に対して、「ある$x$について$p(x)$である」という命題を
\begin{equation*}
  \exists x p(x)
\end{equation*}
と表す

\br

「ある〜について〇〇である」は、
\begin{itemize}
  \item 「ある〜は〇〇である」
  \item 「ある〜が存在して〇〇である」
  \item 「〇〇であるような〜が存在する」
\end{itemize}
と表すこともある

\br

$\exists$という記号は、「exists(存在する)」の頭文字のEを逆さにしたものに由来する

\sectionline

変数$x$が$x= a_1, a_2, \cdots , a_n$という有限個の値をとるとき、「ある$x$について$p(x)$である」というのは、
\begin{shaded}
  $p(a_1)$であるか、あるいは、$p(a_2)$であるか、あるいは、$\cdots$、あるいは、$p(a_n)$である
\end{shaded}
ということに他ならない

言い換えると、
\begin{equation*}
  \exists x p(x) = p(a_1) \lor p(a_2) \lor \cdots \lor p(a_n)
\end{equation*}
ということになる

\sectionline
\section{「すべての〜」と「ある〜」}

「すべての〜」と「ある〜」の2つの概念の間には\keyword{双対性}がある

\begin{align*}
  \forall x p(x) & = p(a_1) \land p(a_2) \land \cdots \land p(a_n) \\
  \exists x p(x) & = p(a_1) \lor p(a_2) \lor \cdots \lor p(a_n)
\end{align*}
という式を比較してみると、「すべての〜($\forall$)」と「ある〜($\exists$)」は、AND($\land$)とOR($\lor$)の双対性を反映していることがわかる

\sectionline
\section{$\forall$と$\exists$を含んだ式の同値変形}

\begin{theorem}{$\forall$と$\exists$の性質}
  \begin{align*}
    \forall x (p(x) \land q(x)) & \equiv \forall x p(x) \land \forall x q(x) \\
    \exists x (p(x) \lor q(x))  & \equiv \exists x p(x) \lor \exists x q(x)
  \end{align*}
\end{theorem}

これらはそれぞれ、
\begin{itemize}
  \item 「すべての〜」($\forall x$)とAND($\land$)
  \item 「ある〜」($\exists x$)とOR($\lor$)
\end{itemize}
が対応していると思って眺めるとよい

\sectionline
\section{$\forall$と$\exists$の否定}

「すべての〜」($\forall$)と「ある〜」($\exists$)を含む命題の否定は、次の\keyword{ド・モルガンの法則}で与えられる

\begin{theorem}{ド・モルガンの法則(述語論理)}
  \begin{align*}
    \neg \forall x p(x) & \equiv \exists x \neg p(x) \\
    \neg \exists x p(x) & \equiv \forall x \neg p(x)
  \end{align*}
\end{theorem}

$\neg \forall x p(x) \equiv \exists x \neg p(x)$より、
\begin{shaded}
  「すべての〜について…である」の否定は、「ある〜について…でない」
\end{shaded}

$\neg \exists x p(x) \equiv \forall x \neg p(x)$より、
\begin{shaded}
  「ある〜について…である」の否定は、「すべての〜について…でない」
\end{shaded}

要するに、否定をとると、「すべての〜」は「ある〜」になり、「ある〜」は「すべての〜」になる

\sectionline

述語論理のド・モルガンの法則は、命題論理のド・モルガンの法則の一般化になっている

\br

$x$が$x = a_1, a_2, \cdots , a_n$というように、有限個の値しかとらない場合、
\begin{align*}
  \forall x p(x) & = p(a_1) \land p(a_2) \land \cdots \land p(a_n) \\
  \exists x p(x) & = p(a_1) \lor p(a_2) \lor \cdots \lor p(a_n)
\end{align*}
であり、
\begin{align*}
  \forall x \neg p(x) & = \neg p(a_1) \land \neg p(a_2) \land \cdots \land \neg p(a_n) \\
  \exists x \neg p(x) & = \neg p(a_1) \lor \neg p(a_2) \lor \cdots \lor \neg p(a_n)
\end{align*}
であるので、述語論理のド・モルガンの法則は、それぞれ次のように書き換えられる
\begin{multline}
  \neg (p(a_1) \land p(a_2) \land \cdots \land p(a_n)) \\ \equiv \neg p(a_1) \lor \neg p(a_2) \lor \cdots \lor \neg p(a_n)
\end{multline}
\begin{multline}
  \neg (p(a_1) \lor p(a_2) \lor \cdots \lor p(a_n)) \\ \equiv \neg p(a_1) \land \neg p(a_2) \land \cdots \land \neg p(a_n)
\end{multline}
これらはそれぞれ、命題論理のド・モルガンの法則の一般化になっていることは一目瞭然

\end{document}
