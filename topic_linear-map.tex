\documentclass[b5paper,12pt]{jsarticle}

\title{Topic Note: 線形写像と行列}
\author{tomixy}

% === color ===

% ref: https://latexcolor.com/
\definecolor{hotpink}{rgb}{1.0, 0.41, 0.71}
\definecolor{carnationpink}{rgb}{1.0, 0.65, 0.79}
\definecolor{deeppink}{rgb}{1.0, 0.08, 0.58}
\definecolor{capri}{rgb}{0.0, 0.75, 1.0}
\definecolor{rosepink}{rgb}{1.0, 0.4, 0.8}
\definecolor{princetonorange}{rgb}{1.0, 0.56, 0.0}
\definecolor{lavendermagenta}{rgb}{0.93, 0.51, 0.93}
\definecolor{malachite}{rgb}{0.04, 0.85, 0.32}
\definecolor{lawngreen}{rgb}{0.49, 0.99, 0.0}
\definecolor{periwinkle}{rgb}{0.8, 0.8, 1.0}
\definecolor{lightslategray}{rgb}{0.47, 0.53, 0.6}
\definecolor{robineggblue}{rgb}{0.0, 0.8, 0.8}
\definecolor{rosebonbon}{rgb}{0.98, 0.26, 0.62}
\definecolor{airforceblue}{rgb}{0.36, 0.54, 0.66}
\definecolor{columbiablue}{rgb}{0.61, 0.87, 1.0}
\definecolor{magnolia}{rgb}{0.97, 0.96, 1.0}
\definecolor{coolgrey}{rgb}{0.55, 0.57, 0.67}

% === box ===

\usepackage{awesomebox}

% === math ===

\usepackage{physics}
\usepackage{braket}

\usepackage{amssymb} % use \blacksquare

\usepackage{amsthm} % 定理環境とQEDコマンド
\renewcommand{\qedsymbol}{\textcolor{coolgrey}{$\blacksquare$}}

\usepackage{mathtools}
% 別の場所で参照する数式以外は番号が付かないように
\mathtoolsset{showonlyrefs=true}

\usepackage{systeme} % 連立方程式を簡単に書く
\usepackage{empheq}

\newcommand{\id}{\operatorname{id}}
\newcommand{\Id}{\operatorname{Id}}
\newcommand{\diag}{\operatorname{diag}}
\newcommand{\Ker}{\operatorname{Ker}}
\newcommand{\sgn}{\operatorname{sgn}}

\newcommand{\suchthat}{\,\, s.t. \,\,}
\newcommand{\transpose}[1]{{}^t\! #1}

% === font ===

\usepackage{amsfonts} % use \mathbb

\usepackage[T1]{fontenc}
\usepackage{lxfonts}

% monospace font
\renewcommand*\ttdefault{cmvtt}

% === layout ===

\usepackage[top=20truemm,bottom=20truemm,left=20truemm,right=60truemm,marginparwidth=40truemm,marginparsep=10truemm]{geometry} % 余白
\renewcommand{\baselinestretch}{1.25} % 行間

\usepackage{leading}

\setlength{\parindent}{0pt} % 段落始めでの字下げをしない

\usepackage{enumitem}
\newcommand{\romanlabel}{\textsf{\roman*.}}
\newcommand{\romannum}[1]{\textsf{#1}}

\usepackage[noparboxrestore]{marginnote}

\usepackage{tocloft}
% chapterのnumwidthを広くする
\setlength{\cftchapnumwidth}{5em}

\usepackage{titling}
\renewcommand{\maketitlehooka}{\textsf}

% === tikz ===

\usepackage[dvipdfmx]{graphicx}

\usepackage{tikz}
\usetikzlibrary{
  fit,
  patterns,
  decorations.pathreplacing,
  cd,
  petri,
  positioning
}

\usepackage{ifthen}
\usepackage{listofitems} % for \readlist to create arrays

\usepackage{witharrows}
\usepackage{nicematrix}

% === tcolorbox ===

\usepackage{tcolorbox}
\tcbuselibrary{listings,breakable,xparse,skins,hooks,theorems}

\newcommand{\titlegap}{\quad\\[0.1cm]}

\DeclareTColorBox{definition}{m O{} }%
{
  enhanced,
  colframe=magnolia,
  colback=magnolia!20!white,
  coltitle=black,
  fonttitle=\bfseries,
  breakable,
  sharp corners,
  title={\textcolor{Cerulean!60!black}{\faGraduationCap}\hspace{0.1em} #1},
  detach title,
  before upper={\tcbtitle\quad},
  bottom=0.5cm,
  top=0.5cm,
  right=0.5cm,
  left=0.5cm,
  #2
}

\DeclareTColorBox{theorem}{m O{} }%
{
  enhanced,
  colframe=magnolia,
  colback=magnolia!20!white,
  coltitle=black,
  fonttitle=\bfseries,
  breakable,
  sharp corners,
  title={\textcolor{magenta!70!black}{\faAnchor}\hspace{0.1em} #1},
  detach title,
  before upper={\tcbtitle\quad},
  bottom=0.5cm,
  top=0.5cm,
  right=0.5cm,
  left=0.5cm,
  #2
}

% 背景がグレー
\DeclareTColorBox{shaded}{O{} }%
{
  enhanced,
  colframe=white,
  colback=gray!10,
  breakable=true,
  sharp corners,
  detach title,
  bottom=0.25cm,
  top=0.25cm,
  right=0.25cm,
  left=0.25cm,
  #1
}

\newcommand{\ProofColor}{coolgrey}
\DeclareTColorBox{proof}{O{証明}}{%
  empty,
  title={\faBroom #1},
  attach boxed title to top left,
  sharp corners,
  boxed title style={
      empty,
      size=minimal,
      toprule=2pt,
      top=4pt,
      left=1em,
      right=1em,
      top=0.25cm,
      overlay={
          \draw[\ProofColor, double,line width=1pt] ([yshift=-1pt]frame.north west)--([yshift=-1pt]frame.north east);
        }
    },
  coltitle=\ProofColor,
  fonttitle=\bfseries,
  before=\par\medskip\noindent,
  parbox=false,
  boxsep=0pt,
  left=1em,
  right=1em,
  top=0.5cm,
  bottom=0.5cm,
  breakable,
  pad at break*=0mm,
  vfill before first,
  overlay unbroken={
      \draw[\ProofColor,line width=0.5pt]
      ([yshift=-1pt]title.north east)
      --([xshift=-0.5pt,yshift=-1pt]title.north-|frame.east)
      --([xshift=-0.5pt]frame.south east)
      --(frame.south west);
    },
  overlay first={
      \draw[\ProofColor,line width=1pt]([yshift=-1pt]title.north east)--([xshift=-0.5pt,yshift=-1pt]title.north-|frame.east)--([xshift=-0.5pt]frame.south east);
    },
  overlay middle={
      \draw[\ProofColor,line width=1pt] ([xshift=-0.5pt]frame.north east)--([xshift=-0.5pt]frame.south east);
    },
  overlay last={
      \draw[\ProofColor,line width=1pt] ([xshift=-0.5pt]frame.north east)--([xshift=-0.5pt]frame.south east)--(frame.south west);
    },%
}
\NewDocumentCommand{\patterntitle}{m}{
  \tcbox[
    enhanced,
    empty,
    boxsep=0pt,
    left=0pt,right=0pt,
    bottom=2pt,
    fonttitle=\bfseries,
    borderline south={0.5pt}{0pt}{\ProofColor},
  ]{\textcolor{\ProofColor}{#1}}
}
\renewenvironment{quote}{%
  \list{}{%
    \leftmargin0.5cm   % this is the adjusting screw
    \rightmargin\leftmargin
  }
  \item\relax
}{\endlist}
\newenvironment{subpattern}[1]{
  \patterntitle{#1}
  \begin{quote}
    }{
  \end{quote}
}

% === memo ===

\usepackage{zebra-goodies} % TODOなどの注釈

% === original ===

\newcommand{\keyword}[1]{\textcolor{RubineRed}{\textbf{#1}}}
\newcommand{\en}[1]{\textcolor{RubineRed}{\small\texttt{#1}}}
\newcommand{\keywordJE}[2]{\keyword{#1}(\en{\textcolor{RubineRed!60}{#2}})}

\newcommand{\br}{\vskip0.5\baselineskip}

\usepackage[object=vectorian]{pgfornament}
\newcommand{\sectionline}{%
  \noindent
  \begin{center}
    {\color{lightgray}
      \resizebox{0.5\linewidth}{1ex}
      {{%
            {\begin{tikzpicture}
                  \node  (C) at (0,0) {};
                  \node (D) at (9,0) {};
                  \path (C) to [ornament=85] (D);
                \end{tikzpicture}}}}}%
  \end{center}%
}

\renewcommand{\labelitemii}{$\circ$}

\newcommand{\refbook}[1]{\small ref: #1}

% === toc ===

\usepackage{tocloft}
\renewcommand{\cftsecfont}{\rmfamily}
\renewcommand{\cftsecpagefont}{\rmfamily}
\setcounter{secnumdepth}{0}

\addtocontents{toc}{\protect\thispagestyle{empty}}
\pagestyle{empty}

% === hyperlink ===

\definecolor{oxfordblue}{rgb}{0.0, 0.13, 0.28}

% 「%」は以降の内容を「改行コードも含めて」無視するコマンド
\usepackage[%
  dvipdfmx,% 欧文ではコメントアウトする
  pdfencoding=auto, psdextra,% 数学記号を含める
  setpagesize=false,%
  bookmarks=true,%
  bookmarksdepth=tocdepth,%
  bookmarksnumbered=true,%
  colorlinks=true,%
  allcolors=oxfordblue,%
  linkcolor=MidnightBlue,%
  pdftitle={},%
  pdfsubject={},%
  pdfauthor={},%
  pdfkeywords={}%
]{hyperref}
% PDFのしおり機能の日本語文字化けを防ぐ((u)pLaTeXのときのみかく)
\usepackage{pxjahyper}
% ref: https://tex.stackexchange.com/questions/251491/math-symbol-in-section-heading
\pdfstringdefDisableCommands{\def\varepsilon{\textepsilon}}


% === 参考文献 ===

\newcommand{\refbookA}{\refbook{行列と行列式の基礎}}

% ---

\begin{document}

\maketitle
\tableofcontents

\sectionline
\section{行列の導入}
\marginnote{\refbookA 1.4}

長方形に並んだ数の集まりを
\begin{equation*}
  A = \begin{pmatrix}
    a_{11} & a_{12} & \dots  & a_{1n} \\
    a_{21} & a_{22} & \dots  & a_{2n} \\
    \vdots & \vdots & \ddots & \vdots \\
    a_{m1} & a_{m2} & \dots  & a_{mn}
  \end{pmatrix}
\end{equation*}
などと書き、\keyword{行列}と呼ぶ

\br

横の数字の並びを\keyword{行}、縦の数字の並びを\keyword{列}と呼ぶ

$A$は$m$個の行と$n$個の列をもつ行列である

\br

第$i$行、第$j$列にある数字を$a_{ij}$と表し、これを$(i, j)$\keyword{成分}と呼ぶ

\br

行が$m$個、列が$n$個の行列は、\keyword{$m$行$n$列の行列}、あるいは\keyword{$m \times n$型の行列}であるという

$n \times n$型の場合、行列は正方形なので$n$次\keyword{正方行列}と呼ぶ

\sectionline

$A$の成分から第$j$列だけを取り出して$\mathbb{R}^m$のベクトルとしたものが
\begin{equation*}
  \vb*{a}_j = \begin{pmatrix}
    a_{1j} \\
    a_{2j} \\
    \vdots \\
    a_{mj}
  \end{pmatrix} \quad (1 \leq i \leq n)
\end{equation*}
であり、これを$A$の$j$番目の\keyword{列ベクトル}という

\br

$A$は、これらを横に並べたものという意味で
\begin{equation*}
  A = (\vb*{a}_1, \vb*{a}_2, \dots, \vb*{a}_n)
\end{equation*}
と書くことができる

\sectionline

\begin{definition}{行列とベクトルの積}
  $m \times n$型の行列$A = (\vb*{a}_1, \vb*{a}_2, \dots, \vb*{a}_n)$と$\vb*{v} \in \mathbb{R}^n$との\keyword{積}を
  \begin{equation*}
    A\vb*{v} = v_1 \vb*{a}_1 + v_2 \vb*{a}_2 + \cdots + v_n \vb*{a}_n
  \end{equation*}
  により定める

  ここで、$v_i$は$\vb*{v}$の第$i$成分である
\end{definition}

$A\vb*{v}$を考えるとき、ほとんどの場合は、$A$が1つ与えられていてm$\vb*{v}$がいろいろ動くという意識が強い

それは、行列$A$のことを、ベクトルを与えて別なベクトルを作る
\begin{equation*}
  \text{入力ベクトル} \vb*{v} \rightarrow \text{出力ベクトル} A\vb*{v}
\end{equation*}
という装置、すなわち\keyword{写像}だとみなすことである

\sectionline

\begin{definition}{行列のスカラー倍}
  $A$を行列、$c$をスカラーとするとき、$A$のすべての成分を$c$倍して得られる行列を$cA$とする
\end{definition}

\begin{theorem}{行列とベクトルの積の性質}
  $A, \, B$を$m \times n$型行列、$\vb*{u}, \, \vb*{v} \in \mathbb{R}^n$、$c \in \mathbb{R}$とするとき、次が成り立つ
  \begin{enumerate}[label=\romanlabel]
    \item $A(\vb*{u} + \vb*{v}) = A\vb*{u} + A\vb*{v}$
    \item $A(c\vb*{v}) = cA\vb*{v}$
  \end{enumerate}
\end{theorem}

\begin{proof}
  \todo{\refbookA p24(命題1.4.3)}
\end{proof}

\sectionline
\section{線形写像の定義}
\marginnote{\refbookA 2}

\begin{definition}{線形写像と線形性}
  写像$f\colon \mathbb{R}^n \to \mathbb{R}^m$が\keyword{線形写像}であるとは、次の2つの条件が成立することである
  \begin{enumerate}[label=\romanlabel]
    \item $f(c\vb*{v}) = cf(\vb*{v})$がすべての$c \in \mathbb{R} , \, \vb*{v} \in \mathbb{R}^n$に対して成り立つ
    \item $f(\vb*{u} + \vb*{v}) = f(\vb*{u}) + f(\vb*{v})$がすべての$\vb*{u}, \vb*{v} \in \mathbb{R}^n$に対して成り立つ
  \end{enumerate}
  これらの性質を写像$f$の\keyword{線形性}という

  また、$m=n$のとき、線形写像$f\colon \mathbb{R}^n \to \mathbb{R}^n$を$\mathbb{R}^n$の\keyword{線形変換}と呼ぶ
\end{definition}

\keyword{線形変換}は空間$\mathbb{R}^n$からそれ自身への写像なので、$\mathbb{R}^n$内において「ベクトルが変化している」(あるいは$f$が空間$\mathbb{R}^n$に\keyword{作用}している)ニュアンスとみることができる

\sectionline

$f\colon \mathbb{R}^n \to \mathbb{R}^m$を線形写像とするとき、\romannum{i}より、
\begin{equation*}
  f(0 \cdot \vb*{v}) = 0 \cdot f(\vb*{v})
\end{equation*}
なので、
\begin{equation*}
  f(\vb*{0}) = \vb*{0}
\end{equation*}
が成り立つ

\begin{theorem}{零ベクトルの像}
  零ベクトルは線形写像によって零ベクトルに写される
\end{theorem}

\sectionline

$m=n=1$のときは、線形写像$f\colon \mathbb{R}^1 \to \mathbb{R}^1$は、通常の意味の関数である

このとき、\romannum{i}の性質から、
\begin{equation*}
  f(c) = f(c \cdot 1) = c \cdot f(1) \quad (c \in \mathbb{R} = \mathbb{R}^1)
\end{equation*}
が成り立つので、$a = f(1) \in \mathbb{R}$とおくと、
\begin{equation*}
  f(x) = ax
\end{equation*}
と書ける

\begin{theorem}{比例関数}
  線形写像$f\colon \mathbb{R}^1 \to \mathbb{R}^1$は、$a$を\keyword{比例定数}とする\keyword{比例関数}である
\end{theorem}

\sectionline
\section{線形写像の表現行列}

$f\colon \mathbb{R}^n \to \mathbb{R}^m$を線形写像とするとき、各基本ベクトル$\vb*{e}_j$の$f$による像を
\begin{equation*}
  f(\vb*{e}_j) = \vb*{a}_j = \begin{pmatrix}
    a_{1j} \\
    a_{2j} \\
    \vdots \\
    a_{mj}
  \end{pmatrix}
\end{equation*}
と書くとする

これらを横に並べることによって、$m$行$n$列の行列を作る
\begin{equation*}
  A = \begin{pmatrix}
    a_{11} & a_{12} & \dots  & a_{1n} \\
    a_{21} & a_{22} & \dots  & a_{2n} \\
    \vdots & \vdots & \ddots & \vdots \\
    a_{m1} & a_{m2} & \dots  & a_{mn}
  \end{pmatrix} = (\vb*{a}_1, \vb*{a}_2, \dots, \vb*{a}_n)
\end{equation*}
この行列$A$を$f$の\keyword{表現行列}という

\br

特に、$\mathbb{R}^n$の線形変換の表現行列は$n$次\keyword{正方行列}である

\sectionline

$\mathbb{R}^n$の一般のベクトル$\vb*{v}$を、基本ベクトルの線型結合として
\begin{equation*}
  \vb*{v} = \sum_{j=1}^n v_j \vb*{e}_j
\end{equation*}
と書く

このとき、$f$の線形性より、
\begin{equation*}
  f(\vb*{v}) = \sum_{j=1}^n v_j f(\vb*{e}_j) = \sum_{j=1}^n v_j \vb*{a}_j
\end{equation*}
となる

このベクトルの第$i$成分は
\begin{equation*}
  a_{i1} v_1 + a_{i2} v_2 + \cdots + a_{in} v_n
\end{equation*}
と書ける

これは$A\vb*{v}$の第$i$成分である

\br

したがって、この記法を踏まえて、次のような表記ができる

\begin{theorem}{線形写像とその表現行列の関係}
  \begin{equation*}
    f(\vb*{v}) = A\vb*{v}
  \end{equation*}
\end{theorem}

比例関数が比例定数$a$だけで決まるのと同じように、線形写像は表現行列$A$が与えられれば決まる

\sectionline

\begin{definition}{零写像と零行列}
  $f\colon \mathbb{R}^n \to \mathbb{R}^m$を、すべての$\vb*{v} \in \mathbb{R}^n$に対して$f(\vb*{v}) = \vb*{0}$と定めたものは明らかに線形写像であり、これを\keyword{零写像}と呼ぶ

  その表現行列はすべての成分が0である行列である

  この行列を\keyword{零行列}と呼び、$O$で表す
\end{definition}

$m \times n$型であることを明示するために$O_{m,n}$と書くこともある

また、$n$次正方行列の場合は、$O_n$と書く

\sectionline

\begin{definition}{恒等写像と単位行列}
  $f\colon \mathbb{R}^n \to \mathbb{R}^n$を、すべての$\vb*{v} \in \mathbb{R}^n$に対して$f(\vb*{v}) = \vb*{v}$と定めたものは明らかに線形写像である

  これを\keyword{恒等写像}と呼び、$f = \id_{\mathbb{R}^n}$と書く

  恒等写像の表現行列は、$f(\vb*{e}_j) = \vb*{e}_j \quad (1 \leq j \leq n)$より
  \begin{equation*}
    E = (\vb*{e}_1, \vb*{e}_2, \dots, \vb*{e}_n) = \begin{pmatrix}
      1      & 0      & \dots  & 0      \\
      0      & 1      & \dots  & 0      \\
      \vdots & \vdots & \ddots & \vdots \\
      0      & 0      & \dots  & 1
    \end{pmatrix}
  \end{equation*}
  であり、これを\keyword{単位行列}と呼ぶ
\end{definition}

単位行列は正方行列であり、$n$次であることを明示したいときは$E_n$と書く

\sectionline

線形写像$f$から行列$A$を作ったのとは逆に、任意の行列から線形写像を作ることができる

\begin{theorem}{行列から線形写像を作る}
  $m \times n$型行列$A$に対して、
  \begin{equation*}
    f(\vb*{v}) = A\vb*{v} \quad (\vb*{v} \in \mathbb{R}^n)
  \end{equation*}
  によって写像$f\colon \mathbb{R}^n \to \mathbb{R}^m$を定めれば、$f$は線形写像である
\end{theorem}

\begin{proof}
  行列とベクトルの積の性質より、$f$は線形写像である

  また、$f$の定義から明らかに$A$は$f$の表現行列である $\qed$
\end{proof}

\sectionline
\section{$\mathbb{R}^2$の線形変換の例}

\todo{\refbookA p51 - p56}

\sectionline
\section{行列の積}

\begin{theorem}{線形写像の合成}
  $\mathbb{R}^n$から$\mathbb{R}^m$への線形写像$g$と、$\mathbb{R}^m$から$\mathbb{R}^l$への線形写像$f$が与えられているとき、これらを合成して得られる写像
  \begin{equation*}
    f \circ g\colon \mathbb{R}^n \xrightarrow{g} \mathbb{R}^m \xrightarrow{f} \mathbb{R}^l
  \end{equation*}
  は、$\mathbb{R}^n$から$\mathbb{R}^l$への線形写像である
\end{theorem}

\begin{proof}
  \todo{\refbookA p56 (問2.2)}
\end{proof}

$f$と$g$の表現行列をそれぞれ$A = (a_{ij}), \, B = (b_{ij})$とする

$A$は$l \times m$型、$B$は$m \times n$型の行列である

\br

このとき、$f \circ g$は$l \times n$型行列で表現される

それを$C$と書くことにして、その成分を計算しよう

そのためには、基本ベクトルの写り先を見ればよい

\br

$B$を列ベクトルに分解して$B = (\vb*{b}_1, \vb*{b}_2, \dots, \vb*{b}_n)$と書くとき、
\begin{equation*}
  (f \circ g)(\vb*{e}_j) = f(g(\vb*{e}_j)) = f(\vb*{b}_j) = A \vb*{b}_j \quad (1 \leq j \leq n)
\end{equation*}
なので、
\begin{equation*}
  C = (A\vb*{b}_1, A\vb*{b}_2, \dots, A\vb*{b}_n)
\end{equation*}
となる

$C$の$(i, j)$成分は$A\vb*{b}_j$の第$i$成分なので、
\begin{equation*}
  c_{ij} = a_{i1} b_{1j} + a_{i2} b_{2j} + \cdots + a_{im} b_{mj} = \sum_{k=1}^m a_{ik} b_{kj}
\end{equation*}
により与えられる

つまり、$C$の$(i, j)$成分を計算するときは、$A$の第$i$行、$B$の第$j$列だけを見ればよい

\begin{equation*}
  \begin{pmatrix}
           &        &       &        \\
           &        &       &        \\
    a_{i1} & a_{i2} & \dots & a_{im} \\
           &        &       &        \\
           &        &       &
  \end{pmatrix}
  \begin{pmatrix}
     & b_{1j} & \\
     & b_{2j} & \\
     & \vdots & \\
     & b_{mj} &
  \end{pmatrix} = \begin{pmatrix}
          & \vdots                                  &       \\
    \dots & \displaystyle\sum_{k=1}^m a_{ik} b_{kj} & \dots \\
          & \vdots                                  &
  \end{pmatrix}
\end{equation*}

\br

このようにして得られた$l \times n$型行列$C$を$AB$と書き、$A$と$B$の\keyword{積}と呼ぶ

\end{document}
