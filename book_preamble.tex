% === color ===

\usepackage[dvipsnames]{xcolor}

% === math ===

\usepackage{physics}

\usepackage{amsthm} % 定理環境とQEDコマンド

\usepackage{mathtools}
% 別の場所で参照する数式以外は番号が付かないように
\mathtoolsset{showonlyrefs=true}

% === font ===

\usepackage[T1]{fontenc}

% normal font, math font
%\usepackage[light,math]{anttor}
\usepackage{gfsartemisia}

% monospace font
\usepackage[scaled]{beramono}

% === layout ===

\usepackage[top=20truemm,bottom=20truemm,left=20truemm,right=20truemm]{geometry} % 余白
\renewcommand{\baselinestretch}{1.25} % 行間

\setlength{\parindent}{0pt} % 段落始めでの字下げをしない

\usepackage{enumitem}

% === tikz ===

\usepackage{froufrou} % セクションを区切る装飾
\setfroufrou{dinkus}

% === original ===

\newcommand{\keyword}[1]{\textcolor{RubineRed}{\textbf{#1}}}

\newcommand{\br}{\vskip0.5\baselineskip}
\newcommand{\sectionline}{\vspace{-\baselineskip}\froufrou\vspace{-\baselineskip}}

\renewcommand{\labelitemii}{$\circ$}

% === toc ===

\usepackage{tocloft}
\renewcommand{\cftsecfont}{\rmfamily}
\renewcommand{\cftsecpagefont}{\rmfamily}
\setcounter{secnumdepth}{0}

\addtocontents{toc}{\protect\thispagestyle{empty}}
\pagestyle{empty}

% === hyperlink ===

\definecolor{oxfordblue}{rgb}{0.0, 0.13, 0.28}

% 「%」は以降の内容を「改行コードも含めて」無視するコマンド
\usepackage[%
  dvipdfmx,% 欧文ではコメントアウトする
  setpagesize=false,%
  bookmarks=true,%
  bookmarksdepth=tocdepth,%
  bookmarksnumbered=true,%
  colorlinks=true,%
  allcolors=oxfordblue,%
  pdftitle={},%
  pdfsubject={},%
  pdfauthor={},%
  pdfkeywords={}%
]{hyperref}
% PDFのしおり機能の日本語文字化けを防ぐ((u)pLaTeXのときのみかく)
\usepackage{pxjahyper}
